%%
%% copyright quintard julien
%% 
%% kaneton
%% 
%% k4-subject.tex
%% 
%% path          /root/data/research/projects/svn/kaneton/projects/k4
%% 
%% made by mycure
%%         quintard julien   [quinta_j@epita.fr]
%% 
%% started on    Mon Feb 21 16:02:45 2005   mycure
%% last update   Mon Feb 21 16:02:46 2005   mycure
%%

\documentclass[10pt,a4wide]{article}
\usepackage[english]{babel}
\usepackage{a4wide}
\usepackage{graphicx}
\usepackage{graphics}
\usepackage{fancyheadings}
\pagestyle{fancy}

\bibliographystyle{plain}

% cible
% header
\lhead{{\scriptsize kaneton project}}
\rhead{k4 subject}
\rfoot{\scriptsize EPITA System Lab}

\title{kaneton4}

\author{Julien Quintard - \small{quinta\_j@epita.fr} \\
        Jean-Pascal Billaud - \small{billau\_j@epita.fr} \\ \\
	\small{last updated by} \\
	Julien Quintard - \small{quinta\_j@epita.fr}}

\date{\today}

\begin{document}
\maketitle

\tableofcontents

\section{Informations}

\begin{tabular}{p{7cm}l}

Date de rendu: & Lundi 4 Avril 2005 \`a 23h42 \\
Dur\'ee du projet: & 3 semaines \\
Nom du fichier de rendu: & k4.tar.gz \\
Responsable du projet: & Julien Quintard - \small{quinta\_j@epita.fr} \\
                       & Jean-Pascal Billaud - \small{billau\_j@epita.fr} \\
Newsgroups d\'edi\'es: & epita.kaneton, epita.adm.sr \\
Langages: & asm, C \\
Architectures: & Intel 32-bit \\
Nombre de personnes par groupes: & 3 \`a 5

\end{tabular}

\section{Introduction}

\paragraph{}

\end{document}

XXX processus thread_create thread_destroy possibilite d'avoir un espace d'adressage
XXX sans aucun fil d'execution.  il faut faire un load pour charger un binaire une fois
XXX l'espace d'adressage cree. une fois le binaire loade la on peut faire un thread_create

