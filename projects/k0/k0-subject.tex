%%
%% copyright quintard julien
%% 
%% kaneton
%% 
%% k0-subject.tex
%% 
%% path          /root/data/research/projects/svn/kaneton/projects/k0
%% 
%% made by mycure
%%         quintard julien   [quinta_j@epita.fr]
%% 
%% started on    Mon Feb 21 16:02:30 2005   mycure
%% last update   Fri Mar 18 13:18:52 2005   mycure
%%

\documentclass[10pt,a4wide]{article}
\usepackage[english]{babel}
\usepackage{a4wide}
\usepackage{graphicx}
\usepackage{graphics}
\usepackage{fancyheadings}
\pagestyle{fancy}

\bibliographystyle{plain}

\lhead{\scriptsize{kaneton project}}
\rhead{k0 subject}
\rfoot{\scriptsize{EPITA System Lab}}

\title{kaneton0}

\author{Julien Quintard - \small{quinta\_j@epita.fr} \\
        Jean-Pascal Billaud - \small{billau\_j@epita.fr} \\ \\
	\small{last updated by} \\
	Julien Quintard - \small{quinta\_j@epita.fr}}

\date{\today}

\begin{document}
\maketitle

\section{Informations}

\paragraph{}

\begin{tabular}{p{7cm}l}

Date de rendu: & Lundi 24 Janvier 2005 \`a 23h42 \\
Dur\'ee du projet: & 1 semaine \\
Nom du fichier de rendu: & k0.tar.gz \\
Responsable du projet: & Julien Quintard - \small{quinta\_j@epita.fr} \\
                       & Jean-Pascal Billaud - \small{billau\_j@epita.fr} \\
Newsgroups d\'edi\'es: & epita.kaneton, epita.adm.sr \\
Langages: & asm \\
Architectures: & Intel 32-bit \\
Nombre de personnes par groupe: & 3 \`a 5

\end{tabular}

\section{Introduction}

\paragraph{}

Le but du projet est de r\'ealiser un bootstrap enti\`erement \'ecrit en
assembleur.

\paragraph{}

Ce projet est destin\'e \`a faire manipuler aux \'etudiants du code
lorsque le processeur \'evolue en mode r\'eel.

\paragraph{}

Un bootstrap est un petit binaire qui s'ex\'ecute avant le kernel. Son but
est de charger en m\'emoire un certain nombre de fichiers dont le kernel
a besoin pour fonctionner.

\paragraph{}

Le bootstrap est stock\'e sur un p\'eriph\'erique comme par exemple la
disquette ou le disque dur. C'est le BIOS qui charge ce code en m\'emoire
pour finir par l'ex\'ecuter.

\paragraph{}

Dans notre cas l'\'etudiant devra d\'evelopper un petit code assembleur
installant un nouveau mode d'adressage. Une fois le nouveau mode install\'e
le programme devra charger en m\'emoire un binaire puis l'ex\'ecuter.

\paragraph{}

La particularit\'e du sujet r\'eside dans la difficult\'e de faire fonctionner
du code 16-bit avec du code 32-bit.

\section{Partie obligatoire}

\paragraph{}

Le projet consiste \`a r\'ealiser un bootstrap suivant certaines \'etapes:

\begin{enumerate}

\item Afficher un message d'invite, en utilisant les interruptions BIOS,
      indiquant l'\'evolution du processeur en mode 16-bit: ``16-bit mode''.

\item Lire le second secteur de la disquette et le stocker en ram de sorte
      que ce soit coh\'erent avec le ``ld script''.

\item R\'ecup\'erer l'``entry point'' dans le header ELF et le stocker
      dans un registre.

\item Installer le nouveau mode d'adressage nomm\'e ``Protected Mode''.

\item ``Jumper'' sur votre programme ELF.

\item Afficher un message indiquant le succ\`es de la mise en place de ce
      nouveau mode d'adressage: ``32-bit mode''.

\end{enumerate}

Votre programme sera test\'e avec des binaires ELF \'ecrits par nos soins.
Votre code doit donc fonctionner avec n'importe quel binaire ELF.

\section{Partie bonus}

\paragraph{}

Une fois la partie obligatoire termin\'ee et fonctionnant parfaitement, les
\'etudiants sont invit\'es \`a s'attaquer \`a la partie bonus:

\begin{enumerate}
\item Trouver une m\'ethode pour emp\^echer les ex\'ecutions de code dans
      la stack en mode 32-bit, et d\'emontrer l'efficacit\'e de celle-ci.
\end{enumerate}

\section{Bibliographie}

\paragraph{}

Voici la bibliographie de k0.

\subsection{Langage assembleur}

\paragraph{}

\begin{itemize}
\item IA-32 Intel Architecture Software Developer's Manual Volume 2: 
      Instruction Set Reference.
\end{itemize}

\subsection{Mode r\'eel}

\paragraph{}

\begin{itemize}
\item IA-32 Intel Architecture Software Developer's Manual Volume 1:
      Basic Architecture - Chapter 3 Basic Execution Environment.
\end{itemize}

\subsection{Mode prot\'eg\'e}

\paragraph{}

\begin{itemize}
\item IA-32 Intel Architecture Software Developer's Manual Volume 1:
      Basic Architecture - Chapter 3 Basic Execution Environment.
\item IA-32 Intel Architecture Software Developer's Manual Volume 3:
      System Programming Guide - Chapter 3 System Architecture Overview.
\item IA-32 Intel Architecture Software Developer's Manual Volume 3:
      System Programming Guide - Chapter 3 Protected-Mode Memory Management.
\end{itemize}

\end{document}
