%%
%% copyright quintard julien
%% 
%% kaneton
%% 
%% k6-subject.tex
%% 
%% path          /root/data/research/projects/svn/kaneton/projects/k6
%% 
%% made by mycure
%%         quintard julien   [quinta_j@epita.fr]
%% 
%% started on    Tue Mar 29 04:09:14 2005   mycure
%% last update   Sat Apr  2 01:55:00 2005   mycure
%%

\documentclass[10pt,a4wide]{article}
\usepackage[english]{babel}
\usepackage{a4wide}
\usepackage{graphicx}
\usepackage{graphics}
\usepackage{fancyheadings}
\pagestyle{fancy}

\bibliographystyle{plain}

\lhead{\scriptsize{kaneton project}}
\rhead{k6 subject}
\rfoot{\scriptsize{EPITA System Lab}}

\title{kaneton-6}

\author{Julien Quintard - \small{quinta\_j@epita.fr} \\
        Jean-Pascal Billaud - \small{billau\_j@epita.fr} \\ \\
	\small{last updated by} \\
	Julien Quintard - \small{quinta\_j@epita.fr}}

\date{\today}

\begin{document}
\maketitle

\section{Informations}

\begin{tabular}{p{7cm}l}

Date de rendu: & Lundi XXX 2005 \`a 23h42 \\
Dur\'ee du projet: & 2 semaines \\
Nom du fichier de rendu: & k6.tar.gz \\
Responsable du projet: & Julien Quintard - \small{quinta\_j@epita.fr} \\
                       & Jean-Pascal Billaud - \small{billau\_j@epita.fr} \\
Newsgroups d\'edi\'es: & epita.kaneton, epita.adm.sr \\
Langages: & asm, C \\
Architectures: & Intel 32-bit \\
Nombre de personnes par groupes: & 3 \`a 5

\end{tabular}

\section{Introduction}

\paragraph{}

Votre syst\`eme est d\'esormais complet. Tout ce qu'il peut manquer
ce sont des drivers, des services, bref des fonctionnalit\'es qu'un
syst\`eme d'exploitation digne de ce nom se doit de fournir.

\paragraph{}

Le but de ce projet va \^etre de d\'evelopper toute la couche bas niveau
n\'ecessaire pour pouvoir ensuite mettre en place un syst\`eme pour sauvegarder
des donn\'ees.

\section{Travail Demand\'e}

\paragraph{}

Le projet consiste donc \`a d\'evelopper un driver IDE indispensable pour
pouvoir stocker des donn\'ees sur un disque d\^ur. De plus un gestionnaire
de partitions vous sera demand\'e.

\paragraph{}

Ce projet ne demande quasiment aucune mod\'elisation et nous vous demandons
de ne pas trop en faire car le sujet suivant pourrait vous amener \`a modifier
beaucoup de choses si jamais votre conception \'etait trop avanc\'ee pour
ce projet. En d'autres termes, concentrez vous sur le pilotage du disque
d\^ur.

\paragraph{}

Nous vous demandons de fournir deux services: un service \textbf{ide} capable
de d\'etecter, de lire, d'\'ecrire etc.. les disque d\^urs pr\'esents
sur la machine et un service \textbf{part} identifiant les partitions
pr\'esentes sur chaque disque d\^ur d\'etecter par le service \textbf{ide}.

\paragraph{}

Ces deux services seront donc forc\'es de communiquer ensemble pour
s'\'echanger des informations, utilisant de fait les fonctionnalit\'es
mis en oeuvre dans les tranches pr\'ec\'edantes.

\paragraph{}

Lors du d\'emarrage de votre kernel il vous sera demand\'e d'afficher
la liste des disques d\^urs d\'etecter accompagn\'e de toutes les
informations sur ces disques: capacit\'e, modes de transfert support\'es,
modes DMA support\'es etc.. bref tout ce que vous pouvez d\'etecter.

\paragraph{}

Il vous est \'egalement demand\'e d'afficher la liste des partitions
d\'etecter avec pour chacune d'elle: la taille, le type (Win95, ext2,
reiserfs etc..) et si celle-ci est une partition \'etendue.

\section{Interfaces}

\paragraph{}

Concernant les interfaces pour la communication entre vos services, nous
vous laissons libre.

\paragraph{}

Rappellons tout de m\^eme que le travail de mod\'elisation prendra forme
dans le sujet suivant. De ce fait, ne perdez pas votre temps \`a concevoir
quelque chose de compliquer pour ce projet.

\section{Bibliographie}

\paragraph{}

Voici la bibliographie li\'ee \`a ce projet.

\subsection{IDE}

\paragraph{}

\begin{itemize}
\item ATA/ATAPI Specifications
\item CHS Translation
\end{itemize}

\subsection{Part}

\paragraph{}

\begin{itemize}
\item Partition Specifications
\end{itemize}

\end{document}
