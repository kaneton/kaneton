%%
%% copyright quintard julien
%% 
%% kaneton
%% 
%% k6-subject.tex
%% 
%% path          /root/data/research/projects/svn/kaneton/projects/k6
%% 
%% made by mycure
%%         quintard julien   [quinta_j@epita.fr]
%% 
%% started on    Tue Mar 29 04:09:14 2005   mycure
%% last update   Tue Mar 29 04:14:52 2005   mycure
%%

\documentclass[10pt,a4wide]{article}
\usepackage[english]{babel}
\usepackage{a4wide}
\usepackage{graphicx}
\usepackage{graphics}
\usepackage{fancyheadings}
\pagestyle{fancy}

\bibliographystyle{plain}

\lhead{\scriptsize{kaneton project}}
\rhead{k6 subject}
\rfoot{\scriptsize{EPITA System Lab}}

\title{kaneton-6}

\author{Julien Quintard - \small{quinta\_j@epita.fr} \\
        Jean-Pascal Billaud - \small{billau\_j@epita.fr} \\ \\
	\small{last updated by} \\
	Julien Quintard - \small{quinta\_j@epita.fr}}

\date{\today}

\begin{document}
\maketitle

\section{Informations}

\begin{tabular}{p{7cm}l}

Date de rendu: & Lundi XXX 2005 \`a 23h42 \\
Dur\'ee du projet: & 1 semaine \\
Nom du fichier de rendu: & k6.tar.gz \\
Responsable du projet: & Julien Quintard - \small{quinta\_j@epita.fr} \\
                       & Jean-Pascal Billaud - \small{billau\_j@epita.fr} \\
Newsgroups d\'edi\'es: & epita.kaneton, epita.adm.sr \\
Langages: & asm, C \\
Architectures: & Intel 32-bit \\
Nombre de personnes par groupes: & 3 \`a 5

\end{tabular}

\section{Introduction}

\paragraph{}

XXX

\section{Travail Demand\'e}

\paragraph{}

XXX

\section{Interfaces}

\paragraph{}

Voici les interfaces \`a suivre pour ce projet.

\subsection{IDE}

\paragraph{}

XXX

\subsection{Part}

\paragraph{}

XXX

--XXX
\paragraph{}

\hspace{1.5cm}int \textbf{set\_insert\_head}(t\_setid \textbf{setid},
                                             void* \textbf{object});

\paragraph{}

Cette fonction ins\`ere l'objet \textbf{object} au d\'ebut de l'ensemble
\textbf{setid}.
--XXX

\section{Types}

\paragraph{}

Voici une explication plus compl\`ete des nouveaux types que
vous devrez utilis\'es.

\subsection{t\_XXX}

\paragraph{}

XXX

\begin{verbatim}
#define ID_UNUSED               ((t_id)-1)

typedef t_uint32        t_id;
\end{verbatim}

\section{Bibliographie}

\paragraph{}

Voici la bibliographie li\'ee \`a ce projet.

\subsection{IDE}

\paragraph{}

\begin{itemize}
\item XXX
\end{itemize}

\subsection{Part}

\paragraph{}

\begin{itemize}
\item XXX
\end{itemize}

\end{document}
