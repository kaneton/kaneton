%%
%% copyright quintard julien
%% 
%% kaneton
%% 
%% languages.tex
%% 
%% path          /root/data/research/projects/svn/kaneton/notes/languages
%% 
%% made by mycure
%%         quintard julien   [quinta_j@epita.fr]
%% 
%% started on    Mon Feb 21 16:03:53 2005   mycure
%% last update   Mon Feb 21 16:03:53 2005   mycure
%%

\documentclass[10pt,a4wide]{article}
\usepackage[english]{babel}
\usepackage{a4wide}
\usepackage{graphicx}
\usepackage{graphics}
\usepackage{fancyheadings}
\pagestyle{fancy}

\bibliographystyle{plain}

\lhead{{\scriptsize kaneton project}}
\rhead{languages notes}
\rfoot{\scriptsize EPITA System Lab}

\title{languages}

\author{Julien Quintard - \small{quinta\_j@epita.fr} \\
        Jean-Pascal Billaud - \small{billau\_j@epita.fr} \\ \\
	\small{last updated by} \\
	Julien Quintard - \small{quinta\_j@epita.fr}}

\date{\today}

\begin{document}
\maketitle

\section{Notes}

\begin{enumerate}

\item pourquoi utiliser l'assembleur ?
\item compilateur nasm
\item les macros [ORG 0X0] et [BITS 16-32]
\item registres g\'en\'eraux
\item registres de contr\^oles
\item registres eflags
\item les intructions de stockages mov ...
\item les instructions d'op\'erations sur les bits shl, shr, and, or ...
\item les instructions de saut jmp, call ...
\item les instructions de comaraisons et de branchement conditionnelle cmp, test ...
\item d'autres instructions facilitant bien la vie lodsb
\item les interruptions bios int 13h, int 10h ...
\item fonctionnement de la stack

\end{enumerate}

\end{document}

