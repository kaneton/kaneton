%%
%% copyright quintard julien
%% 
%% kaneton
%% 
%% k2.tex
%% 
%% path          /root/data/research/projects/svn/kaneton/notes/k2
%% 
%% made by mycure
%%         quintard julien   [quinta_j@epita.fr]
%% 
%% started on    Mon Feb 21 16:03:31 2005   mycure
%% last update   Sun Mar 20 23:48:38 2005   mycure
%%

\documentclass[10pt,a4wide]{article}
\usepackage[english]{babel}
\usepackage{a4wide}
\usepackage{graphicx}
\usepackage{graphics}
\usepackage{fancyheadings}
\pagestyle{fancy}

\bibliographystyle{plain}

\lhead{{\scriptsize kaneton project}}
\rhead{k2 notes}
\rfoot{\scriptsize EPITA System Lab}

\title{k2}

\author{Julien Quintard - \small{quinta\_j@epita.fr} \\
        Jean-Pascal Billaud - \small{billau\_j@epita.fr} \\ \\
	\small{last updated by} \\
	Julien Quintard - \small{quinta\_j@epita.fr}}

\date{\today}

\begin{document}
\maketitle

\section{Notes}

\begin{enumerate}

\item explication g\'en\'erale du projet

\item explication des interfaces: segment, asid, pm, int, keyb, pic ...

\item explication des differents algorithmes: bitmap, bitmap avanc\'e,
      areas, (liste, arbres) etc.. pour la gestion de la m\'emoire. expliquer
      par \'etapes: bitmap, bitmap \'evolu\'e, areas en liste (ou autre
      structure de donn\'ees), areas evolu\'ees pour permettre le swapping
      et le sharing. bien expliquer que pour le sharing il faudrait y
      inclure un syst\`eme de protection, d'authentification.

\item explication de la possibilit\'e de faire du swapping mais qu'il
      faut rajouter une traduction physical -> virtual

\item explication de la structure t\_as et des champs pas et vas

\item explication de la gestion de t\_pm pm; avec une page d\'ecoup\'ee
      et toutes les areas mises dans la freelist

\item expliquer aussi: soit on un a kasid et un kas, dans ce cas t\_pm pm,
      et kas doivent etre g\'er\'es via des pages directement d\'ecoup\'ees,
      soit on n'utilise pas de kas car seul le kasid compte. lorsque celui-ci
      est utilis\'e on ne met \`a jour aucun espace d'adressage.

\item faire un exemple avec les allocations: t\_pm pm, as0, as1, pm\_rsv(),
      vm\_rsv() et l'evolution de t\_pm pm et des deux as.

\item explication de l'IDT

\item explication du driver clavier et du driver console

\item explication du PIC

\end{enumerate}

\end{document}
