%%
%% copyright quintard julien
%% 
%% kaneton
%% 
%% k4.tex
%% 
%% path          /root/data/research/projects/svn/kaneton/notes/k4
%% 
%% made by mycure
%%         quintard julien   [quinta_j@epita.fr]
%% 
%% started on    Mon Mar 21 13:16:44 2005   mycure
%% last update   Mon Mar 21 13:18:10 2005   mycure
%%

\documentclass[10pt,a4wide]{article}
\usepackage[english]{babel}
\usepackage{a4wide}
\usepackage{graphicx}
\usepackage{graphics}
\usepackage{fancyheadings}
\pagestyle{fancy}

\bibliographystyle{plain}

\lhead{{\scriptsize kaneton project}}
\rhead{k4 notes}
\rfoot{\scriptsize EPITA System Lab}

\title{k4}

\author{Julien Quintard - \small{quinta\_j@epita.fr} \\
        Jean-Pascal Billaud - \small{billau\_j@epita.fr} \\ \\
	\small{last updated by} \\
	Julien Quintard - \small{quinta\_j@epita.fr}}

\date{\today}

\begin{document}
\maketitle

\section{Notes}

\begin{enumerate}

\item explication g\'en\'erale du projet

\item explication des interfaces

\item scheduler:

\begin{verbatim}

(*) le kernel doit garder en memoire:
	. a) le nombre de processes en cours
	. b) la somme des priorites des processes
	. c) le quantum actuel

	-> quantum total = a * c
	-> quantum du process = priority * quantum total / b

(*) le process doit garder en memoire:
        . d) le nombre de threads
	. e) la somme des priorites des threads

	-> quantum du thread = priority * quantum du process / e

(*) si le quantum d'un process descend en dessous du seuil de la classe
    inferieure c'est que ca commence a devenir tendu et la par exemple
    on pourrait envisager sur un systeme distribue a migrer des processus
    et a lancer les nouveaux processus sur une autre machine.

\end{verbatim}

exemple:

\begin{verbatim}

5 tasks avec une somme de priorites de 276 avec 20ms par defaut par processus
soit 20*5=100ms pour tout le monde:
        1) B 15
        2) R 100
        3) T 33
        4) T 58
        5) I 70

IL FAUT DONC TJS GARDER LA SOMME DES PRIORITES DES PROCESSUS

1) 15*100/276   =       5.4             /* a calculer en ticks apres */
2) 100*100/276  =       36.2
3) 33*100/276   =       11.9
4) 58*100/276   =       21.0
5) 70*100/276   =       25.3

                        99.8

\end{verbatim}

\end{enumerate}

\end{document}
