%%
%% copyright quintard julien
%% 
%% kaneton
%% 
%% k7-subject.tex
%% 
%% path          /root/data/research/projects/svn/kaneton/projects/k7
%% 
%% made by mycure
%%         quintard julien   [quinta_j@epita.fr]
%% 
%% started on    Sat Apr  2 01:16:52 2005   mycure
%% last update   Sat Apr  2 01:48:44 2005   mycure
%%

\documentclass[10pt,a4wide]{article}
\usepackage[english]{babel}
\usepackage{a4wide}
\usepackage{graphicx}
\usepackage{graphics}
\usepackage{fancyheadings}
\pagestyle{fancy}

\bibliographystyle{plain}

\lhead{\scriptsize{kaneton project}}
\rhead{k7 subject}
\rfoot{\scriptsize{EPITA System Lab}}

\title{kaneton-7}

\author{Julien Quintard - \small{quinta\_j@epita.fr} \\
        Jean-Pascal Billaud - \small{billau\_j@epita.fr} \\ \\
	\small{last updated by} \\
	Julien Quintard - \small{quinta\_j@epita.fr}}

\date{\today}

\begin{document}
\maketitle

\section{Informations}

\begin{tabular}{p{7cm}l}

Date de rendu: & Lundi XXX 2005 \`a 23h42 \\
Dur\'ee du projet: & 1 semaine \\
Nom du fichier de rendu: & k7.tar.gz \\
Responsable du projet: & Julien Quintard - \small{quinta\_j@epita.fr} \\
                       & Jean-Pascal Billaud - \small{billau\_j@epita.fr} \\
Newsgroups d\'edi\'es: & epita.kaneton, epita.adm.sr \\
Langages: & asm, C \\
Architectures: & Intel 32-bit \\
Nombre de personnes par groupes: & 3 \`a 5

\end{tabular}

\section{Introduction}

\paragraph{}

\textbf{kaneton} dispose d\'esormais de deux nouveaux services proposant
des fonctionnalit\'es pour stocker des donn\'es sur des disques durs.

\paragraph{}

Le but de ce projet va \^etre de mod\'eliser un syst\`eme de gestion
des p\'eriph\'eriques g\'en\'eriques.

\paragraph{}

Ce gestionnaire de p\'eriph\'erique nomm\'e \textbf{dev} devra pouvoir
recenser tous les p\'eriph\'eriques pr\'esents sur le syst\`eme.

\paragraph{}

Gr\^ace \`a cette centralisation des ressources, il sera possible d'appliquer
des algorithmes pour pr\'evenir, \'eviter les \textit{deadlocks} mais il sera
\'egalement possible de convenir d'une norme en ce qui concerne les
drivers et d'identifier les p\'eriph\'eriques via un identificateur unique
choisi par le gestionnaire de p\'eriph\'eriques.

\section{Travail Demand\'e}

\paragraph{}

Votre travail consiste \`a mod\'eliser un gestionnaire de p\'eriph\'eriques
et \`a \'ecrire une sp\'ecification d\'ecrivant vos choix par rapports
aux probl\`emes rencontr\'es tout cela pour concevoir le syst\`eme le
plus g\'en\'erique mais \'egalement le plus performant possible.

\paragraph{}

Votre note sera fix\'ee en fonction de l'exactitude de votre sp\'ecification
par rapport \`a votre impl\'ementation, en fonction de vos id\'ees, de vos
choix mais \'egalement en fonction de votre impl\'ementation.

\paragraph{}

Cette sp\'ecification devrait permettre \`a tout d\'eveloppeur de
driver de ne pas perdre de temps \`a comprendre le fonctionnement de
votre syst\`eme en lisant les sources.

\section{Interfaces}

\paragraph{}

Il n'y a forc\'ement aucune interface de pr\'esent\'ee ici puisque c'est
le but du projet de vous les faire concevoir.

\section{Bibliographie}

\paragraph{}

La bibliographie pour ce projet est inexistante.

\end{document}
