%
% ---------- header -----------------------------------------------------------
%
% project       kaneton
%
% license       kaneton
%
% file          /home/mycure/kaneton/view/internship/check/check.tex
%
% created       julien quintard   [wed may 16 18:06:23 2007]
% updated       julien quintard   [wed may 16 19:45:22 2007]
%

%
% path
%

\def\path{../..}

%
% template
%

%%
%% licence       kaneton licence
%%
%% project       kaneton
%%
%% file          /home/mycure/kaneton/view/templates/internship.tex
%%
%% created       julien quintard   [mon feb 27 20:31:01 2006]
%% updated       julien quintard   [thu mar  2 00:24:57 2006]
%%

%
% class
%

\documentclass[10pt,a4wide]{article}

%
% packages
%

\usepackage[english]{babel}
\usepackage[T1]{fontenc}
\usepackage{a4wide}
\usepackage{fancyheadings}
\usepackage{multicol}
\usepackage{indentfirst}
\usepackage{graphicx}
\usepackage{color}
\usepackage{xcolor}
\usepackage{verbatim}
\usepackage{aeguill}

\pagestyle{fancy}

\setlength{\footrulewidth}{0.3pt}
\setlength{\parindent}{0.3cm}
\setlength{\parskip}{2ex plus 0.5ex minus 0.2ex}

%
% verbatim font
%

\definecolor{verbatimcolor}{rgb}{0,0.4,0}

\makeatletter
\renewcommand{\verbatim@font}
  {\ttfamily\footnotesize\color{verbatimcolor}\selectfont}
\makeatother

%
% header
%

\rhead{}
\rfoot{\scriptsize{The kaneton microkernel project}}

\date{\scriptsize{\today}}


%
% header
%

\lhead{\scriptsize{The kaneton microkernel Test Framework}}

%
% title
%

\title{kaneton Test Framework}

%
% authors
%

\author{\small{Solal Jacob}}

%
% document
%

\begin{document}

%
% title
%

\maketitle

%
% ---------- abstract ---------------------------------------------------------
%

\begin{abstract}

\indent The test framework ains at mesuring the reliability of the kaneton
microkernel. It is organized around two parts. On the one hand, tests are
written and executed within kaneton, on the other hand, a Python script runs on
another independant system. This document concerns users who want to build a
new tests set and also contributors who would like to upgrade the test
environment itself.

\end{abstract}

%
% --------- text --------------------------------------------------------------
%

\section{Framework architecture}

\indent As the test environment is supposed to check kaneton's reliability, it
cannot assume that kaneton is safe and therefore it cannot be installed on the
tested system. Thus, the test environment has been designed to check kaneton
from another independant system. Both of kaneton and the independant system can
communicate via the serial port.\\
\\
\indent Once the tests are written (kaneton-side), the Makefile needs to be
modified to turn the kernel into debug mode. This mode enables the tests,
executes them and sends their results through the serial port to the script.
When debug mode is enabled, kaneton first initializes the serial driver (default
settings are COM1 at 56kb/s) and then waits for commands from the serial
connection.\\
\\
The test script is written in Python and depends on a C/Python module whose
role is to manage serial communication. It is based on a dedicated protocol
also used in kaneton. All was done to simplify the script development.\\
\\
\indent Strong conventions have been established to write test scripts and to
manage the kaneton Makefile. They must be known and respected to add new tests.
These conventions are deeply detailed in the next paragraphs.


\section{How to add and to launch tests}

\subsection{How to add tests}
As already said, test writting must obey certain rules. Conventions on
directory and file naming were chosen as follow:
\begin{enumerate}
\item A test dedicated to a specific part of kaneton must be created in the same
directory tree than the piece of kaneton it is supposed to check. This new
directory tree must be placed in kaneton/check/.
\item A test is a directory named 01, 02, \ldots, N and containing the following
files:
\begin{itemize}
\item .c: the test itself which is part of kaneton
\item .res: the expected output for the corresponding .c
\item parse\_res.py: a python function which must be modify to parse the
result.
\end{itemize}
\item A file named list and containing all the names of the tests separated by a
single \textbackslash n must be placed in every level of the check directory
tree.
\\
\end{enumerate}
{\bf Example:}\\
Assuming we want to write two tests checking the printf function
which is located in kaneton/klibc/libstring/, we need to create the following
 directory tree:

\begin{verbatim}
    kaneton/check/
                 libs/
                     klibc/
                          listring/
                                  printf/
                                        01/
                                           01.c
                                           01.res
                                           pares\_res.py
                                        02/
                                           02.c
                                           02.res
                                           pares\_res.py
                                        list		 => 01\n02\n
                                  list			 => printf\n
                          list				 => libstring\n
                     list				 => klibc\n
                 list					 => libs\n

\end{verbatim}


\subsection{How to launch tests}
\begin{itemize}
\item Install the Python module by running ./domodules in kaneton/tools/python/
\item Select the tests to enable by adding their name in the appropriate `list`
text file (see Directory Convention for further details).
\item Configure kaneton to run in debug mode:
\begin{description}
\item env/users/user.conf: delete the line: {\tt override \_CHECK\_}
\item env/users/conf.h: add the line:  {\tt \#define SERIAL}
\end{description}
\item Build the kernel
\item During kernel execution, run ./check.py to get the tests results
\end{itemize}


\section{Serial driver and the test framework on kaneton}

The serial driver can be found in kaneton/core/kaneton/debug/serial.c.\\
Initialization is performed by passing the desired com\_port and baud\_rate to
the serial\_init function. Basic values as SERIAL\_8N1 and SERIAL\_FIFO\_8 are
defined for recurrent usages; you will find them in
kaneton/core/include/kaneton/serial.h.\\
Once the driver is setup, the functions serial\_read and serial\_write permit
to transfert data trough the serial port (see serial.h for further
information). Keep in mind that this serial driver was written in pole mode. So
it requires a 1GHz processor to achieve high-speed (56kbp/s) communication
without trouble.

The test framework routines can be found in
kaneton/core/kaneton/debug/debug.c.\\
A call to the debug\_init function performs all initializations needed by the
test framework. It calls serial\_init, printf\_init, allocates sufficient memory
and waits for new messages in a never ending loop. printf\_init permits to
redirect printf output towards the serial port using serial\_put function as a
parameter. debug\_recv reads in input until it receives "command". Then it
waits for the address of the command to execute, executes it and uses
serial\_put to send the command result through the serial port.\\


\section{Python script and C/Python modules}

check.py is the script which analyzes the tests results. It  can be found in
kaneton/check/. The C/Python kserialmodule in kaneton/tools/python/ must be
installed prior to launch ./check.py. Just run ./domodules to do so. This
module permits Python applications to use serial\_init, serial\_recv and
serial\_send functions, and thus to communicate with kaneton. The line\\
\\
\indent{\tt from kserial import *}\\
\\
in the main function of check.py shows how to use the kserial module.\\
\\
\indent The script check.py first initializes the serial communication by a
call to serial\_init("/dev/ttyS0"), what implies that we run check.py on Linux
on COM1. Then the script uses the function ListTest to recursively find the tests
it will have to execute. All the tests found are added to a list. This list is
parsed and test functions are sequentially sent to kaneton for execution. Their
result is compared to their corresponding .res by their associated
parse\_res.py script. Every failure or success is stored in order to compute
and display totals.

\end{document}
