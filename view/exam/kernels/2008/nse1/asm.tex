\section*{Exercice 1 : Assembleur (4 points)}

Le code assembleur suivant est ex�cut� sur un microprocesseur x86.
Les instructions sont au format AT\&T. C'est � dire : l'op�rande source
est suivie de l'op�rande destination.

Indiquez, pour chaque instruction, ce qu'elle fait. Puis expliquez de mani�re
globale ce que fait le code.

Rappel: La syntaxe de GAS permet de d�clarer des labels locaux sous forme de
num�ro. Pour faire r�f�rence � ces labels, la notation est compos�e du num�ro
en question, suivi de la lettre \verb+b+ si l'on fait r�f�rence en arri�re,
ou de la lettre \verb+f+ si l'on fait r�f�rence en avant.

\begin{enumerate}
\item
\begin{verbatim}
        lgdt gdtr
        mov %cr0, %eax
        or $1, %eax
        mov %eax, %cr0
        ljmp $0x8,$1f
1:
        mov $0x10, %ax
        mov %ax, %ds
        mov %ax, %es
        mov %ax, %fs
        mov %ax, %gs
        mov %ax, %ss
\end{verbatim}



\item
\begin{verbatim}
        mov $0xb8000, %edi
        mov $160, %eax
        mul %ebx
        add %eax, %edi
        shl $1, %ecx
        add %ecx, %edi
1:
        movb (%esi), %al
        add $1, %esi
        test %al, %al
        jz 2f
        movb %al, (%edi)
        add $2, %edi
        jmp 1b
2:
        ret
\end{verbatim}
\end{enumerate}

