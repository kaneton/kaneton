\section*{Exercice 4 : Pagination (6 points)}

Nous souhaitons developper un systeme de gestion de la m\'emoire virtuelle
pour ChicheOS sur l'architecture Intel. Trois mots d\'ecrivent la philosophie
de ChicheOS :
``maintenir, r\'eparer et pr\'evoir''. C'est pour cela que nous avons choisi
dans notre design de ne pas utiliser la technique du mirroring, contrairement
\`a kaneton.

Nous avons besoin dans ChicheOS d'un syst\`eme nous permettant de pouvoir
mapper nimporte quelle page de la m\'emoire physique dans l'espace d'adressage
du noyau, et optimiser l'utilisation du TLB (cache de translation d'adresse).

Pour cela, nous avons d\'ecid\'e d'utiliser une table de page accessible en
m\'emoire virtuelle dans l'espace d'adressage du noyau de mani\`ere similaire
\`a un cache. Ce cache cr\'eerai donc des translations vers des
adresses physiques et serai manipulable directement par le noyau.

Le but
final serai de pouvoir appliquer un algorithme de remplacement d'entr\'e
sur cette Page Table afin de mettre a jour les translations de mani\`ere
\`a ne pas supprimer les plus utilis\'ees, ce qui optimiserai par la m\`eme
occasion l'utilisation du TLB (qui est mis a jour, je le rappelle, en
fonction du contenu des Page Tables).

\begin{itemize}
  \item La Page Table ``cache'' se situe en m\'emoire physique \`a l'adresse
  physique \verb+0x100b000+, et est accessible dans l'espace d'addressage du
  kernel \`a l'adresse \verb+0xc0000000+. L'entr\'e num\'ero \verb+0x380+ du
  Page Directory du noyau contient l'adresse physique de cette Page Table.
  \item La fonction \verb+int rand_1024()+ renvoie un nombre al\'eatoire entre
  0 et 1023 inclus.
  \item La fonction \verb+void pt_entry(int entry, vaddr_t pt, paddr_t page);+ ajoute une
  entr\'ee dans une Page Table \`a l'index \verb+entry+. Cette fonction ajuste
  les param\`etres de l'entr\'ee de mani\`ere \`a cr\'eer une translation
  d'adresse privil\'egi\'ee valide pour le noyau.
  \item Les fonctions \verb+char get_entry_data(int entry, vaddr_t pt)+ et
  \verb+void set_entry_data(int entry, vaddr_t pt, char data)+ lisent et
  \'ecrivent les 3 bits du champs AVL d'une entr\'ee de Page Table. Il est
  indiqu\'e dans le manuel Intel que ces 3 bits sont ``Available for
  programmer use''.
\end{itemize}

\subsection*{Questions}

\begin{enumerate}
  \item \`A quelle quantit\'e au total de m\'emoire virtuelle notre cache nous
  donne t'il acces ?
  \item \`A quelles adresses (en m\'emoire virtuelle) peut-t'on acceder au mappings
  que l'on cr\'ee avec notre Page Table ?
  \item \'Ecrivez la fonction \verb+vaddr_t cache_get_mapping(paddr_t phys)+.
  Cette fonction retourne l'adresse virtuelle d'un mapping vers \verb+phys+
  si la translation existe d\'ej\`a, ou cr/'ee le mapping dans le cache le
  cas \'ech\'eant. L'utilisation d'un algorithme type LRU est un plus.
  \item Expliquez comment on peut fabriquer un espace d'adressage utilisateur
  \`a partir de z\'ero en utilisant la fonction \verb+cache_get_mapping()+ et
  une fonction \verb+paddr_t get_physical_page()+ qui alloue des pages de
  m\'emoire physique.
\end{enumerate}

