\section*{Exercice 4 : Pagination (6 points)}

Nous souhaitons d�velopper un syst�me de gestion de la m�moire
virtuelle pour ChicheOS sur l'architecture Intel. Trois mots d�crivent
la philosophie de ChicheOS : ``maintenir, r�parer et pr�voir''. En
cons�quence, ChicheOS �vite avec prudence la p�rilleuse technique du
\textit{mirroring}.

Le noyau ChicheOS doit pouvoir acc�der \`a n'importe quelle adresse
physique par le biais de traductions d'adresses temporaires, tout en
optimisant l'utilisation du TLB \footnote{\emph{Rappel:} Translation
  Lookaside Buffer : cache de traduction d'adresse}.

Afin de permettre la mise en place de traductions temporaires,
ChicheOS rend accessible une table de page particuli�re depuis
l'espace d'adressage du noyau. Cette table de page particuli�re est
nomm�e CPT \footnote{Chiche Page Table}. \textup{La CPT est utilis�e
  comme un \textbf{cache} de traduction d'adresses}.


\begin{itemize}
\item La CPT se situe en m�moire physique \`a l'adresse physique
  \verb+0x100b000+, et est accessible dans l'espace d'adressage du
  noyau \`a l'adresse \verb+0xC0000000+. L'entr�e num�ro \verb+0x380+
  du Page Directory du noyau contient l'adresse physique de cette Page
  Table.
\item La fonction \verb+int rand_1024()+ renvoie un nombre al�atoire
  entre 0 et 1023 inclus.
\item La fonction
  \verb+void pt_entry(int entry, vaddr_t pt, paddr_t page);+ ajoute
  une entr�e dans une Page Table \`a l'index \verb+entry+. Cette
  fonction ajuste les param�tres de l'entr�e de mani�re \`a cr�er une
  translation d'adresse privil�gi�e valide pour le noyau.
\item Les fonctions \verb+char get_entry_data(int entry, vaddr_t pt)+
  et \verb+void set_entry_data(int entry, vaddr_t pt, char data)+
  lisent et �crivent les 3 bits du champs AVL d'une entr�e de Page
  Table. Les sp�cifications Intel documentent ces 3 bits comme
  ``Available for programmer use''.
\end{itemize}

\subsection*{Questions}

\begin{enumerate}
\item Quelle quantit� au total de m�moire virtuelle est rendue
  accessible par notre cache ?

\item \`A quelles adresses (en m�moire virtuelle) est-il possible
  d'acc�der aux traductions contenues par la Page Table ?

\item Si la CPT est pleine\footnote{Toutes les traductions possibles
    sont fa\^ites}, l'ajout d'une nouvelle entr�e suppose la
  suppression d'une vieille entr�e. Pour d�terminer l'entr�e du cache
  a retirer, �crivez une fonction impl�mentant un des algorithmes de
  remplacement d'entr�es vu en cours (LRU, NFU ...). \footnote{Si vous
    s�chez, vous pouvez utiliser la fonction rand\_1024()}

\item �crivez la fonction
  \verb+vaddr_t cache_get_mapping(paddr_t phys)+. Cette fonction
  retourne l'adresse virtuelle d'une traduction vers \verb+phys+ si la
  traduction existe d�j�, ou cr�e la traduction dans le cache le cas
  �ch�ant.

\item Expliquez comment on peut fabriquer un espace d'adressage
  utilisateur \`a partir de z�ro en utilisant la fonction
  \verb+cache_get_mapping()+ et une fonction
  \verb+paddr_t get_physical_page()+ qui alloue des pages de m�moire
  physique.
\end{enumerate}

