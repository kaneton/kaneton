\section*{Exercice 3 : Interruptions (6 points)}

Le but de cet exercice est de vous faire �crire un driver pour port
s�rie sur ChicheOS.  L'architecture de la machine est de type x86, et
le contr�leur s�rie est un Vatican V6942.

\subsection*{Vatican V6942}

Le Vatican V6942 est un UART (Universal Asynchronous Receiver Transmitter).
Le Vatican V6942 est reli\'e au PIC maitre de mani\`ere \`a pouvoir
d�clencher une IRQ7.
Le Vatican V6942 contient 3 registres 8 bits accessibles sur le bus de
controle de notre processeur :

\begin{itemize}
\item CR (Control Register), accessible \`a l'adresse 0x340 sur le bus
  de contr�le.

  Il contient 3 bits importants:
  \begin{itemize}
  \item TR (Transmitter Ready) bit 0 : Le registre THR est pr�t a recevoir un nouvel octet \`a envoyer.
  \item DR (Data Ready) bit 1 : Le registre RBR contient un octet re\c{u} du port s�rie.
  \item IE (Interrupt Enable) bit 2 : Active le d�clenchement des interruptions.
  \end{itemize}

\item THR (Transmit Holding Register), accessible \`a l'adresse 0x341
  sur le bus de contr�le :

  \begin{itemize}
  \item Les octets �crits dans ce registre sont transmis sur le port s�rie
        apres un certain d�lai.
  \item Le bit TR du registre CR d�termine si le dernier octet �crit \`a
	bien �t� transmis. (Celui-ci doit etre positionn� \`a 1). Il d�termine
	donc �galement si il est possible de placer un nouvel octet \`a
	transmettre.
  \item Le composant l\`eve une interruption si le bit IE de CR est \`a
	1, et que l'octet �crit dans THR viens d'etre transmis sur le port
	s�rie.
  \end{itemize}
\end{itemize}

\subsection*{API ChicheOS}

\begin{itemize}
  \item \verb+void outb(short port, char data)+ : �crit un octet sur le bus de contr�le.
  \item \verb+char inb(short port)+ : Lis un octet depuis le bus de contr�le.
  \item \verb+SAVE_CONTEXT()+ : Sauvegarde sur la pile le contexte �tendu \`a la suite d'une interruption.
  \item \verb+RESTORE_CONTEXT()+ : Charge depuis la pile le contexte �tendu.
  \item \verb+void pic_acknlowledge(void)+ : Indique au PIC que l'on \`a pris en charge l'interruption.
  \item \verb+void pic_unmask(int irq)+ : Masque l'IRQ fourni en param�tre.
  \item \verb+void idt_set_irq_handler(int irq, void (*handler)(void))+ : Installe un gestionnaire d'interruption dans l'IDT pour l'IRQ fourni en param�tre.
\end{itemize}

\subsection*{Questions}
\begin{enumerate}
\item �crivez la fonction \verb+void serial_send(char *buff, int size)+
  pour ChicheOS, sans utiliser les interruptions.
\item �crivez la fonctions \verb+void serial_send(char *buff, int size)+
  pour ChicheOS, en tirant profit des interruptions. De plus,
  vous fournirez une fonction \verb+void serial_init()+
  qui initialisera vos structures de donn�es et activera les interruptions.

  Attention : Pour cela, il est possible que vous deviez �crire des fonctions que nous
  ne vous demandons pas explicitement.

\item Qu'apporte l'utilisation des interruptions dans notre cas ?

\item Sur wikipedia, nous pouvons trouver :

\textit{The 8259 generates spurious interrupts in response to a number of conditions.}

\textit{The first is an IRQ line being deasserted before it is acknowledged. This may occur due to noise on the IRQ lines. In edge triggered mode, the noise must maintain the line in the low state for 100nS. When the noise diminishes, a pull-up resistor returns the IRQ line to high, thus generating a false interrupt. In level triggered mode, the noise may cause a high signal level on the systems INTR line. If the system sends an acknowledgment request, the 8259 has nothing to resolve and thus sends an IRQ7 in response. This first case will generate spurious IRQ7's.}


\begin{enumerate}
\item Expliquez avec vos mots le ph�nom�ne des ``spurious interrupts''.
\item Quelles pr�cautions le noyau doit-il prendre vis a vis des ``spurious interrupts'' ?
\item Votre code est-il concern� par le probl�me ?
\end{enumerate}

\end{enumerate}

