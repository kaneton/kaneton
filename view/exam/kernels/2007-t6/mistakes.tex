\section*{Exercice 2 : Gestionnaire d'interruption (3 points)}

{\bf C\^ot\'e noyau}

Le code suivant est inspir\'e du handler d'exceptions par d\'efaut de Linux 1.0 IA32, mais plusieurs erreurs ont \'et\'e introduites.

\begin{enumerate}
\item Corrigez le handler pour que Linux 1.0 ne plante plus lors du d\'eclenchement d'une exception. Vous ne vous soucierez pas des probl\`emes li\'es \`a la s\'ecurit\'e.
\begin{itemize}
\item Aucun autre registre que ceux qui apparaissent dans ce code ne sera utilis\'e pendant le traitement du handler.
\item Ce handler ne traite que les exceptions qui fournissent un code d'erreur.
\item Le code assembleur est \'ecrit en syntax gas (source,destination)
\end{itemize}

\item Quelle autre pr\'ecaution faut-il g\'en\'eralement prendre pour rendre un handler plus s\^ur ?

\item Dans les handlers d'exceptions sous Linux IA32, faut-il sauvegarder, modifier et restaurer le registre {\em CR3} ? Pourquoi ?
\end{enumerate}



\begin{verbatim}
/* This is the default interrupt ``handler'' :-) */

int_msg:
        .asciz ``Unknown interrupt\n''

.align 2
ignore_int:
        popl %eax      ; depile le code d'erreur

        cld            ; ne pas prendre en compte cette instruction
        pushl %eax
        pushl %ecx
        pushl %edx
        push %ds
        push %es
        push %fs

        movl $(KERNEL_CS),%eax
        mov %ax,%ds
        mov %ax,%es
        mov %ax,%fs

        pushl $int_msg
        call _printk

        pop %fs
        pop %es
        pop %ds
        popl %edx
        popl %ecx
        popl %eax
        ret
\end{verbatim}




