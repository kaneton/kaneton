%
% ---------- header -----------------------------------------------------------
%
% project       kaneton
%
% license       kaneton
%
% created       julio guerra   [sat mar 15 09:44:00 2014]
%
%
% ---------- setup ------------------------------------------------------------
%

%
% path
%

\def\path{../../..}

%
% template
%

%%
%% licence       kaneton licence
%%
%% project       kaneton
%%
%% file          /home/mycure/kaneton/view/templates/exam.tex
%%
%% created       julien quintard   [fri dec  2 22:20:57 2005]
%% updated       julien quintard   [mon feb 20 23:01:50 2006]
%%

%
% compile mode
%

%
% ---------- header -----------------------------------------------------------
%
% project       kaneton
%
% license       kaneton
%
% file          /home/mycure/kane...w/talk/presentations/kaneton/kaneton.tex
%
% created       julien quintard   [mon may 14 21:02:29 2007]
% updated       julien quintard   [sun feb  6 20:50:37 2011]
%

%
% ---------- setup ------------------------------------------------------------
%

%
% path
%

\def\path{../../..}

%
% template
%

%
% ---------- header -----------------------------------------------------------
%
% project       kaneton
%
% license       kaneton
%
% file          /home/mycure/kaneton/view/template/talk.tex
%
% created       julien quintard   [wed may 16 18:17:37 2007]
% updated       julien quintard   [fri may 23 19:28:10 2008]
%

%
% ---------- class ------------------------------------------------------------
%

\documentclass[8pt]{beamer}

%
% ---------- common -----------------------------------------------------------
%

\input{\path/package/opk/presentation.tex}


%
% title
%

\title{kaneton}

%
% document
%

\begin{document}

%
% title frame
%

\begin{frame}
  \titlepage
\end{frame}

%
% outline frame
%

\begin{frame}
  \frametitle{Outline}

  \tableofcontents
\end{frame}

%
% ---------- text -------------------------------------------------------------
%


%
% overview
%

\section{Overview}

% 1)

\begin{frame}
  \frametitle{Introduction}

  \term{kaneton} is an educational project intended for students to undertake
  in order to learn about operating system internals.
\end{frame}

% 2)

\begin{frame}
  \frametitle{History}

  \begin{itemize}
    \item[2004]
      \name{Julien Quintard} and \name{Jean-Pascal Billaud} decide
      to introduce an optional low-level programming course to first-year
      enginnering students, now known as \name{kastor};
    \item[2004]
      The course having been well received, \name{SRS - Syst\`emes, R\'eseaux,
      S\'ecurit\'e} students ask them to give such an introductory course the
      same year.
    \item[2005]
      The authorization is given to them to teach a kernel development course
      to \name{SRS} students, from January to October. They therefore decide
      to provide students with the design of a microkernel and let the students
      develop it from scratch, their way. \name{kaneton} is born.
    \item[2006]
      After \name{Jean-Pascal Billaud} fled to \name{VMWare}, \name{Julien
      Quintard} started developing a reference implementation and gave
      students this year a skeleton they had to complete. In addition,
      \name{Cedric Aubouy} and \name{Renaud Lienhart} joined the teaching
      team this year.

      \-

      Besides, the \name{LSE - Laboratoire Syst\`eme EPITA} joined the project
      by putting two students on the development of the \name{kaneton}
      research implementation. \name{Matthieu Bucchianeri} and \name{Renaud
      Voltz} thus joined the project.
  \end{itemize}
\end{frame}


% 3)

\begin{frame}
  \frametitle{History}

  \begin{itemize}
    \item[2007]
      This year, \name{Matthieu Bucchianeri} and \name{Renaud Voltz} took
      over the project for a year by lecturing the course and managing the
      project.

      \-

      \name{Julian Pidancet} and \name{Pierre Duteil} joined the project
      as part of the \name{LSE} but \name{Pierre Duteil} had to leave the
      project. Therefore, \name{Elie Bleton}, who was working at the
      \name{LRDE} before, joined the project.
    \item[2008]
      \name{Julian Pidancet} and \name{Elie Bleton} took over this year
      while \name{Laurent Lec} and \name{Nicolas Grandemange} joined as part
      of the \name{LSE}.

      \-

      At the end of this year, after problems with some students as well as
      conflicts with the \name{LSE}, \name{kaneton} maintainers decided not
      to work with the laboratory anymore.
    \item[2009]
      \name{EPITA} alumni were contacted and joined the educational project
      including \name{Francois Goudal}, \name{Benoit Marcot},
      \name{Enguerrand Raymond}, \name{Jean Guyader} but also
      \name{Fabien Le-Mentec}, an \name{EPITECH} alumnus.
  \end{itemize}
\end{frame}

% 4)

\begin{frame}
  \frametitle{Model}

  The project consists for students to fill in some missing parts of the
  kernel.

  \-

  However, note that, unlike \name{Tiger}, the missing parts will never
  be two lines long.

  \-

  Indeed, in \name{kaneton}, students are asked to implement a feature, say,
  providing memory management. Thus, students are free, in a certain way,
  to implement such a feature as they wish.

  \-

  Since the testing usually consists in verifying that the kernel is able
  to provide the functionality, students should be, most of the time, able
  to implement whatever algorithms \etc{} they wish.
\end{frame}

% 5)

\begin{frame}
  \frametitle{People}

  Let's present the people working on the educational project from where they
  studied to what they are now doing:

  \begin{itemize}
    \item
      \name{Francois Goudal};
    \item
      \name{Benoit Marcot};
    \item
      \name{Jean Guyader};
    \item
      \name{Baptiste Afsa}
    \item
      \name{Louis Vatier}; and
    \item
      \name{Julien Quintard}.
  \end{itemize}
\end{frame}

% 6)

\begin{frame}
  \frametitle{Project}

  \name{kaneton} is an important assignment of the \name{SRS}/\name{GISTR}
  curriculum and, as such, must be taken seriously.

  \-

  Especially, in the last years, \name{EPITA} decided to reduce the duration
  of the project to \term{three} months.

  \-

  As such, the other assignments imposed by the specializations in this
  period have been reduced so that students can focus on \name{kaneton}.
\end{frame}

%
% design
%

\section{Design}

% 1)

\begin{frame}
  \frametitle{Overview}

  The kaneton kernel is very different from the kernels you might be
  familiar with, especially the well-known \name{Windows}, \name{Linux},
  \name{BSD} and so forth.
\end{frame}

% 2)

\begin{frame}
  \frametitle{Microkernel}

  First, kaneton is a microkernel, making it modular from the design
  perspective as well as providing properties such as security.
\end{frame}

% 3)

\begin{frame}
  \frametitle{Distributed Computing}

  kaneton has been designed from the ground up for providing the operarting
  system advanced distributed computing features.
\end{frame}

% 4)

\begin{frame}
  \frametitle{Portability}

  kaneton has been designed with portability in mind, especially through
  a specific portability system that perfectly fits the kernel design.
\end{frame}

% 5)

\begin{frame}
  \frametitle{Organisation}

  Besides being a microkernel, kaneton is well organised in the inside,
  splitting functionalites into objects and managers.
\end{frame}

%
% stages
%

\section{Stages}

% 1)

\begin{frame}
  \frametitle{k0}

  The first project, named \term{k0}, consists for students to learn
  about low-level programming.

  \-

  This project comes with a lecture regarding the boot system as well
  as a practical session.

  \-

  \name{Francois Goudal} will be in charge of this stage which will last for
  a week.
\end{frame}

% 2)

\begin{frame}
  \frametitle{k1}

  \term{k1} consists for students to provide the kaneton microkernel the
  capacity to handle events.

  \-

  During this stage, a lecture on general kernel principles and a lecture on
  interrupts will be taught.

  \-

  \name{Julien Quintard} will be in charge of this stage which will last
  for a single week.

  \-

  Note that, starting with \name{k1}, the student snapshot will be used which
  provide students a development environment, making kernel development easier.
\end{frame}

% 3)

\begin{frame}
  \frametitle{k2}

  \term{k2} consists for students to provide the kaneton microkernel a
  memory management unit so that applications as well as the kernel itself
  can reserve, share \etc{} memory.

  \-

  During this stage, a lecture on portability as well as lectures on
  memory management will be taught.

  \-

  \name{Francois Goudal} will be in charge of this stage which will last
  for three weeks.
\end{frame}

% 4)

\begin{frame}
  \frametitle{k3}

  In \term{k3}, students will have to provide kaneton with execution contexts
  such that the kernel can execute multiple threads at the \textit{same} time.

  \-

  Lectures, during this stage, will discuss topics such as interrupts,
  concurrency, multi-processing, scheduling \etc{}

  \-

  \name{Benoit Marcot} will be in charge of this stage which will last for
  three weeks.
\end{frame}

% 5)

\begin{frame}
  \frametitle{Evaluation}

  For every stage, students will have the possibility to test their
  implementation by running, a limited number of times, the test suite used
  for evaluating their work.

  \-

  Besides, at the end of each stage, after submission, the kaneton test system
  will run the test suite and issue a mark according to the test results.

  \-

  Additionally, an exam will take place at the end of the semester to make
  sure that the notions tackled throughout the course are well understood
  by every student.
\end{frame}

%
% tools
%

\section{Tools}

% 1)

\begin{frame}
  \frametitle{Overview}

  The kaneton educational project relies on tools, sometimes developed
  internally.
\end{frame}

% 2)

\begin{frame}
  \frametitle{Web Site}

  The web site contains the documentation including design papers,
  the assignments \etc{} but also hosts the wiki which should be
  the starting point for every student seeking information.

  \-

  Noteworthy is that the wiki contains courses regarding the
  inline assembly, linking, pre-processing and so on. Students
  are invited to read them all as they will come handy when
  developing the kaneton stages.

  \-

  \name{Julien Quintard} should be contacted for requests regarding the
  web site and wiki.
\end{frame}

% 3)

\begin{frame}
  \frametitle{Snapshot}

  The student snapshot has been automatically generated from the current
  kaneton implementation.

  \-

  \name{Francois Goudal} is in charge of this process, hence should be
  contacted if you believe there is a mistake.
\end{frame}

% 4)

\begin{frame}
  \frametitle{Cheat}

  Every student's kaneton implementation will be tested to make sure that
  students did not cheat by relying on implementations by previous or
  current students.

  \-

  \name{Julien Quintard} is in charge of this tool.
\end{frame}

% 5)

\begin{frame}
  \frametitle{Test}

  Students' implementation will be tested in a real environment by applying
  a complete test suite; hence, validating the implementation's behaviour.

  \-

  \name{Jean Guyader} is responsible of this tool and should be contacted
  if necessary.
\end{frame}

%
% information
%

\section{Information}

% 1)

\begin{frame}
  \frametitle{Support}

  \begin{enumerate}
    \item
      \term{Website}

      \-

      You will find on \location{http://kaneton.opaak.org} documents regarding
      the project from the design to the implementation;
    \item
      \term{Wiki}

      \-

      The wiki \location{http://wiki.opaak.org} is the best way to get
      technical information as well as to help other students by adding
      and/or improving pages' contents;
    \item
      \term{Mailing-List}

      \-

      The kaneton educational students mailing-list
      \location{students@kaneton.opaak.org} will be used by teachers as
      an official means for communicating with students.

      \-

      Therefore, every student should subscribe to this mailing-list by sending
      an email to \location{students+subscribe@kaneton.opaak.org}.

      \-

      It is not allowed to post code on the mailing list, or give pointers to
      code in the snapshot that would provide obvious solution to somebody's
      question.
  \end{enumerate}
\end{frame}

% 2)

\begin{frame}
  \frametitle{Groups}

  Except for \name{k0} which is an individual project, the other projects
  from \name{k1} to \name{k3} are done in groups of \term{two} students.

  \-

  Every group is expected to send an email to
  \location{admin@opaak.org}.

  \-

  Note that we will use students' \name{EPITA} email addresses. As such,
  make sure that you check this email box.
\end{frame}

% 3)

\begin{frame}
  \frametitle{Reliance}

  As for \name{Tiger}, every stage depends on the previous one, except
  for \name{k0}.

  \-

  As such, test suites from the previous stages will also be used for both
  testing and marking.

  \-

  Students should therefore make sure to use their test permissions for making
  sure to fix the bugs of previous stages so that such bugs do not impact
  on the current stage results, hence mark.
\end{frame}

% 4)

\begin{frame}
  \frametitle{Machine}

  This year, the machine used by the kaneton educational project will consists
  of the \term{IBM-PC} platform coupled with the \term{IA-32} microprocessor
  architecture \ie{} the most common hardware system on the market.

  \-

  Although it is always best to test your implementation on a real machine,
  it takes time to reboot a real computer. You should therefore use an
  emulator such a \name{QEMU} or \name{Bochs} as they will enable you to
  test your kernel very quickly but they will also let you develop on
  a non-\name{IBM-PC}/\name{IA-32} machine such as a \name{Mac} for example.
\end{frame}

%
% conclusion
%

\section{Conclusion}

% 1)

\begin{frame}
  \frametitle{Concepts}

  Throughout the project, you will learn so many things from terminology,
  to how a computer boots, how the kernel controls the hardware and how it
  provides abstractions as basic as execution contexts.

  \-

  At the end of the project, you will definitely know that nothing is magic
  but purely logic and often actually very simple.
\end{frame}

% 2)

\begin{frame}
  \frametitle{Implementation}

  Although, starting the project by learning how to make a computer execute
  your code, you will end up, after three months, with a running kernel
  and operating system capable of executing programs, the whole on real
  hardware like the machine you have at home.
\end{frame}

% 3)

\begin{frame}
  \frametitle{Changes}

  Over the years, the project has greatly evolved, from a no-implementation
  project, to a reference-based project.

  \-

  However, being a project developed by volunteers willing to dedicate some
  time so that other students can learn, many things are missing and/or
  can be improved including the lectures but also the project implementation.

  \-

  In conclusion, keep in mind that the project exists only because of people
  willing to transfer their knowledge and please respect their effort.
\end{frame}

% 4)

\begin{frame}
  \frametitle{Fun}

  But most of all, kaneton should be about learning through fun!
\end{frame}

% 5)

\begin{frame}
  \frametitle{Reminder}

  Remember to perform the following tasks:

  \begin{itemize}
    \item
      Send an email to \location{admin@opaak.org} regarding the composition
      of your group, before \textbf{Wednesday 16th 2pm} or you will be put
      in a group by force;
    \item
      Subscribe to the students mailing-list
      \location{students@kaneton.opaak.org} by sending an email to
      \location{students+subscribe@kaneton.opaak.org};
    \item
      Watch closely the \name{Wiki} at \location{http://wiki.opaak.org} by
      subscribing the \name{RSS} feed for example;
    \item
      We advise SRS/GISTR lab roots to set up a \name{Xen}-based environment
      as testing on emulators only will become difficult over time;
    \item
      Students must have a ``rack'' containing a \name{POSIX}-compilant
      operating system for the \name{k0} practical session.
  \end{itemize}
\end{frame}

\end{document}


%
% class
%

\documentclass[10pt,a4wide]{article}

%
% packages
%

\usepackage[english]{babel}
\usepackage[T1]{fontenc}
\usepackage{a4wide}
\usepackage{graphicx}
\usepackage{fancyheadings}
\usepackage{multicol}
\usepackage{indentfirst}
\usepackage{color}
\usepackage{ifthen}
\usepackage{comment}
\usepackage{verbatim}
\usepackage{aeguill}

\pagestyle{fancy}

\setlength{\footrulewidth}{0.3pt}
\setlength{\parindent}{0.3cm}
\setlength{\parskip}{2ex plus 0.5ex minus 0.2ex}

%
% correction environment
%

\newenvironment{correction}%
   {
     \ifthenelse
	 {
	   \equal{\kaneton-latex}{subject}
	 }
	 {
	   \comment
	 }
	 {
	   \textbf{\color{red}{ ----- correction}}
	 }
   }%
   {
     \ifthenelse
	 {
	   \equal{\kaneton-latex}{subject}
	 }
	 {
	   \endcomment
	 }
	 {
	   \textbf{\color{red}{ ----- /correction}}
	 }
   }

%
% header
%

\rfoot{\scriptsize{Exam}}

\date{\scriptsize{\today}}


%
% document
%
\usepackage[utf8]{inputenc}
\usepackage{enumitem}
\usepackage{geometry}
\newgeometry{margin=2cm}

\begin{document}

%
% title
%

\title{Noyaux et Systèmes d'Exploitation}

%
% header
%

\lhead{\scriptsize{EPITA\_ING2 - 2014\_S4 - NSE}}
\rhead{}

%
% title
%

\maketitle

%
% indentation
%

\indentation{}

%
% --------- information -------------------------------------------------------
%

\begin{center}

\textbf{Durée: 3 heures}\\
\textbf{Documents et calculatrice interdits}\\
\scriptsize{Les résultats des calculs sous forme non-développée sont admis.}\\
\scriptsize{Les questions bonus donnent des points bonus sommés à la note finale du partiel.}\\
\scriptsize{Une copie bien présentée sera toujours mieux notée qu'une autre.}

\end{center}

%
% --------- text --------------------------------------------------------------
%

%
% architecture
%

\section{{Architecture}
         {\hfill{} \normalfont{\scriptsize{N points}}}}


\subsection{Linux is obsolete}

Face à l'euphorie de la communauté devant linux, A. Tanenbaum écrit en 1992 que "linux est obsolète". L. Torvalds intervient alors et marque le début d'un long débat à propos de l'architecture des OS: monolithique pour linux (Torvalds), micro-noyau pour minix (Tanenbaum). Tanembaum affirme que les "micro-noyaux ont gagné" face aux vieux noyaux monolithiques, or linux réintroduit une architecture monolithique.

\begin{enumerate}
\item Quels sont les arguments de A. Tanembaum en faveur des architectures micro-noyau?
\item Quels sont les arguments de L. Torvalds en faveur des architectures monolithiques?
\end{enumerate}

%
% boot
%

\section{{Boot}
         {\hfill{} \normalfont{\scriptsize{N points}}}}

\subsection{Firmware}

\begin{enumerate}
\item Quel est le rôle d'un firmware?
\item Pourquoi les firmwares sont-ils en ROM (Read-Only Memory)?
\item Quelle est généralement la premère tâche réalisée par un firmware en ROM?
\item Quelle est l'interface firmware/bootloader sur les ordinateurs IBM-PC?\\
      Le BIOS fait office de firmware dans ce cas.
\end{enumerate}

\subsection{Bootloader}
\begin{enumerate}
\item Quel est le rôle d'un bootloader?
\item Pourquoi les bootloaders historiques des ordinateurs IBM-PC se décomposent-ils en plusieurs binaires?
\item Proposer une interface bootloader/firmware sur laquelle pourra se reposer votre OS.
\end{enumerate}

\subsection*{Bonus:}
Soit un système pouvant démarrer sous différents modes, maintenance et opérationnel, en fonction de la position d'un interrupteur. Proposer les interfaces firmware/bootloader et bootloader/OS permettant de réaliser ce sytème.\\
Chaque mode à son propre binaire en ROM. Le mode maintenance permet de télécharger un nouveau binaire pour le mode opérationnel.

% Interruptions %
\section{{Interruptions}
         {\hfill{} \normalfont{\scriptsize{N points}}}}

\begin{enumerate}
% Q
\item Donner une solution apportée par les interruptions (1 pt).

% Q
\item La fonction de traitement des interruptions d'un ordinateur requiert 2 ms (temps de commutation des processus inclus) par top d'horloge. La fréquence de l'horloge est de 60 Hz. Quel est le pourcentage de temps que le processeur consacre à l'horloge? Commenter.

\end{enumerate}


\subsection*{Bonus:}
Comment sont gérées les interruptions dans un contexte multi-coeur?

% Memoire %
\section{{Mémoire}
         {\hfill{} \normalfont{\scriptsize{M pts}}}}

\begin{enumerate}

% Q
\item Donner deux différences entre la ségmentation et la pagination.

% Q
\item Quel est l'inconvénient de la technique de mirroring? (1 pt)

% Q
\item Proposer une alternative au mirroring sans en perdre l'avantage majeur. (4 pt)

% Q
\item L'un des avantages des processus légers (threads) est qu'ils évitent de vider la TLB lors d'un changement de contexte entre des processus légers d'un même processus.
\begin{enumerate}
  \item Comment est indexée une telle TLB (se vidant lors d'un changement de contexte de processus) ? (1 pt)
  \item Proposez une modification simple à cette indexation afin d'éviter la vidange de la TLB en cas de changement de contexte entre processus. (1 pt)
\end{enumerate}

% R
\begin{correction}
\begin{enumerate}[label=(\alph*)]
\item Avec l'adresse virtuelle sans l'offset.
\item Ajout d'un identifiant d'espace d'adressage.
\end{enumerate}
\end{correction}

% Q
\item Sur une architecture ayant des adresses virtuelles de 32 bits, \textbf{dessiner} les mappings (association pages virtuelles/pages physique) permettant:
\begin{enumerate}[label=(\alph*)]
\item au kernel de s'exécuter dans un mode d'adressage réel 16 bits. (1 pt)
\item à deux processus de partager un buffer. (1 pt)
\item d'utiliser la technique de "copy-on-write" après une fork. (1 pt)
\end{enumerate}

\subsection*{Bonus:}
Comment permettre le partage de code (bibliothèque partagée) grâce à la ségmentation?

% fin de la partie Mémoire
\end{enumerate}


% Multi-tache / Ordonnancement %
\section{{Multi-tâche \& Ordonnancement}
         {\hfill{} \normalfont{\scriptsize{N points}}}}

\begin{enumerate}

% Q
\item Expliquez pourquoi l'implémentation du changement de contexte de
      kaneton induit une perte de performance certaine. Comment remédier à
      ce problème? Vous pourrez exposer des exemples d'implémentation dans
      d'autres noyaux.


% Q
\item Comment sont catégorisées les tâches que l'on retrouve dans les systèmes généraux? Quelles sont leur caractéristique? (2 points)

% R
\begin{correction}
\begin{itemize}
  \item Interactif: beaucoup d'E/S => exécution courte (réveil, traitement, nouvelle E/S) => rend la main vite.  1 point
  \item Traitement | Batch | Traitement par lot: utilisation du CPU intensive (e.g. calcul mathématique, etc.). 1 point
\end{itemize}
\end{correction}


% Q
\item Quels sont leur objectif? (2 points)

% R
\begin{correction}
\begin{itemize}
  \item Intéractif: temps de réponse le plus faible possible. (1 point)
  \item Batch: temps d'exécution (jusqu'à complétude) le plus faible possible - exécution la plus rapide que possible - exécution la moins interrompue que possible. (1 point)
\end{itemize}
\end{correction}


% Q
\item Proposez une heuristique simple permettant de déterminer à quelle catégorie appartient un processus. (2 points)

% R
\begin{correction}
Plusieurs réponses possibles:
\begin{itemize}
  \item Mémorisation des temps d'exécution => une tâche avec un temps d'exécution faible est interactive, une tâche avec un temps d'exécution grand est “batch”.
  \item Mémorisation des nombre de blocage avant la fin du quantum => une tâche avec beaucoup de blocage est interactive, une tâche avec peu de blocage est batch.
  \item n'importe quoi de correct :)
\end{itemize}
\end{correction}


% Q
\item Les ordonnanceurs circulaires mémorisent la liste des processus prêts et chaque processus n'apparaît qu'une seule fois dans cette liste. Que se passerait-il si un processus y apparaissait deux fois (1 point)? Pourquoi permettrait-on l'insertion multiple d'un processus dans cette liste (1 point)?

% R
\begin{correction}
Il serait élu deux fois plutôt qu'une à chaque tour complet de la liste des processus. Intéressant pour favoriser des processus pour implémenter un système primitif à priorité.
\end{correction}


% Q
\item La plupart des ordonnanceurs circulaires utilisent un quantum fixe. Donnez un argument en faveur d'un petit quantum et un autre en faveur d'un grand quantum, ainsi que la catégorie de processus favorisé. (2 points)

% R
\begin{correction}
\begin{itemize}
  \item Petit quantum => temps de réponse élevé. Favorise les processus interactifs.  (1 point)
  \item Grand quantum => temps CPU utile élevé. Favorise les processus batch. (1 point)
\end{itemize}
\end{correction}


% Q
\item Cinq travaux, A à E, arrivent pratiquement en même temps dans un centre de calcul. Leur temps d'exécution respectif est estimé à 10, 6, 2, 4, 8 minutes. Leurs priorités (déterminées de manière externe) sont 3, 5, 2, 1 et 4, la valeur 5 correspondant à la priorité la plus élevée. Déterminez le temps moyen d'attente pour chacun des algorithmes d'ordonnancement suivants. Ne tenez pas compte du temps perdu lors de la commutation des processus.
  \begin{enumerate}
  \item Tourniquet (1 point)
  \item Ordonnancement avec priorité (1 point)
  \item Premier arrivé, premier servi (ordre d'arrivée: ordre alphabétique). (1 point)
  \item Plus court d'abord. (1 point)
  \end{enumerate}
Dans le cas (a) on fait l'hypothèse que le temps processeur est équitablement réparti entre les différents travaux. Dans les cas (b) à (d), on suppose que chaque travail est exécuté jusqu'à ce qu'il se termine. Les travaux n'effectuent pas d'E/S. Détaillez vos calculs (dessins/tableaux \& moyenne sous forme de quotient admis).

% R
\begin{correction}\\
P: Processus courant\\
T: Tour/Temps d’exécution depuis un 0 absolu.\\
E: temps d'exécution restant\\

\begin{enumerate}[label=(\alph*)]

\item Round Robin: la remarque sous-entend la préemption\\
      Timeline de l’execution: * = exécution terminée

      \begin{tabular}{|c||c|c|c|c|c|c|c|c|c|c|c|c|c|c|c|c|c|c|}
      \hline
      P & A & B & C & D & E & A & B & C{*} & D & E & A & B & D & E & A & B & D{*} & E\tabularnewline
      \hline
      T & 1 & 2 & 3 & 4 & 5 & 6 & 7 & 8 & 9 & 10 & 11 & 12 & 13 & 14 & 15 & 16 & 17 & 18\tabularnewline
      \hline
      E & 9 & 5 & 1 & 3 & 7 & 8 & 4 & 0 & 2 & 6 & 7 & 3 & 1 & 5 & 6 & 2 & 0 & 4\tabularnewline
      \hline
      \end{tabular}

      \begin{tabular}{|c||c|c|c|c|c|c|c|c|c|c|c|c|}
      \hline
      P & A & B & E & A & B{*} & E & A & E & A & E{*} & A & A{*}\tabularnewline
      \hline
      T & 19 & 20 & 21 & 22 & 23 & 24 & 25 & 26 & 27 & 28 & 29 & 30\tabularnewline
      \hline
      E & 5 & 1 & 3 & 4 & 0 & 2 & 3 & 1 & 2 & 0 & 1 & 0\tabularnewline
      \hline
      \end{tabular}

      Temps d’attente:\\
      C = 8 min\\
      D = 17 min\\
      B = 23 min\\
      E = 28 min\\
      A = 30 min\\
      Moyenne = (A + B + C + D + E) / 5 = 21,2 min

\item Priorité:\\
      \begin{tabular}{|c||c|c|c|c|c|}
      \hline
      P & B & E & A & C & D\tabularnewline
      \hline
      T & 6 & 6+8=14 & 14+10=24 & 24+2=26 & 26+4=30\tabularnewline
      \hline
      \end{tabular}\\
      Moyenne = (6 + 14 + 24 + 26 + 30) / 5 = 20 minutes

\item Premier arrivé, premier servi:\\
      \begin{tabular}{|c||c|c|c|c|c|}
      \hline
      P & A & B & C & D & E\tabularnewline
      \hline
      T & 10 & 10+6=16 & 16+2=18 & 18+4=22 & 22+8=30\tabularnewline
      \hline
      \end{tabular}\\
      Moyenne = (10 + 16 + 18 + 19 + 23) / 5 = 19,2 minutes

\item Plus court d’abord:\\
      \begin{tabular}{|c||c|c|c|c|c|}
      \hline
      P & C & D & B & E & A\tabularnewline
      \hline
      T & 2 & 2+4=6 & 6+6=12 & 12+8=20 & 20+10=30\tabularnewline
      \hline
      \end{tabular}\\
      Moyenne = (2 + 6 + 12 + 20 + 30)/5 = 14 minutes
\end{enumerate}
\end{correction}

% fin de multi-tache & ordonnancement
\end{enumerate}


% Concurrence %
\section{{Concurrence}
         {\hfill{} \normalfont{\scriptsize{X points}}}}

\begin{enumerate}

% Q
\item Supposez que vous ayez à concevoir l'architecture d'un ordinateur évolué où la commutation entre les processus serait faite par le matériel, et non par logiciel via le mécanisme d'interruption. Quelles seraient les informations nécessaires au processeur ? Décrivez le fonctionnement de la commutation effectuée par le matériel (2 points).

% R
\begin{correction}
\begin{itemize}
  \item Pour faire un contexte switch, le CPU a besoin des infomartions suivantes du processus à exécuter:
  \begin{itemize}
    \item Registres Généraux (1 point)
    \item Pointeurs sur pile et sur code (program counter, next instruction pointer...), stack pointer (1 point)
    \item Control Registers (1 point)
  \end{itemize}
  Fonctionnement du context switch HW moins souple qu'un SW: il ne sait pas ce qu'il execute, il sauvegarde donc tous les registres du processeurs, il doit aussi chercher les nouvelles données dans des structures bien définies.
\end{itemize}
\end{correction}


% Q
\item Quelle est la différence entre un thread kernel et un thread user-land ?

% fin de Concurrence
\end{enumerate}



% Virtualisation et IPC %
\section{{Virtualisation et IPC}
         {\hfill{} \normalfont{\scriptsize{X points}}}}

\begin{enumerate}

% Q
\item Après avoir défini ce qu'est une IPC, donner un exemple pertinent d'utilisation.

% Q
\item Expliquer les différentes méthodes de virtualisation.

% R
\begin{correction}
\begin{itemize}
\item PV : Paravirtualiser. Le système d'exploitation est au courant qu'il s'exécute dans une machine virtuelle et est modifé en conséquence pour coopérer avec l'hyperviseur.
\item HVM : Hardware Virtualized Machine. Le système d'exploitation peut s'éxécuter sans modification à l'intérieur d'une machine virtuelle.
\end{itemize}
\end{correction}


% Q
\item Expliquer les termes suivant: VT-x, VT-d, E820, EPT, NPT

\end{enumerate}

\end{document}
