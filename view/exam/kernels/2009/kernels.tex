%
% ---------- header -----------------------------------------------------------
%
% project       kaneton
%
% license       kaneton
%
% file          /home/mycure/kaneton/view/exam/kernels/2009/kernels.tex
%
% created       julien quintard   [mon may 14 21:56:41 2007]
% updated       julien quintard   [mon jun  8 22:38:59 2009]
%

%
% ---------- setup ------------------------------------------------------------
%

%
% path
%

\def\path{../../..}

%
% template
%

%%
%% licence       kaneton licence
%%
%% project       kaneton
%%
%% file          /home/mycure/kaneton/view/templates/exam.tex
%%
%% created       julien quintard   [fri dec  2 22:20:57 2005]
%% updated       julien quintard   [mon feb 20 23:01:50 2006]
%%

%
% compile mode
%

%
% ---------- header -----------------------------------------------------------
%
% project       kaneton
%
% license       kaneton
%
% file          /home/mycure/kane...w/talk/presentations/kaneton/kaneton.tex
%
% created       julien quintard   [mon may 14 21:02:29 2007]
% updated       julien quintard   [sun feb  6 20:50:37 2011]
%

%
% ---------- setup ------------------------------------------------------------
%

%
% path
%

\def\path{../../..}

%
% template
%

%
% ---------- header -----------------------------------------------------------
%
% project       kaneton
%
% license       kaneton
%
% file          /home/mycure/kaneton/view/template/talk.tex
%
% created       julien quintard   [wed may 16 18:17:37 2007]
% updated       julien quintard   [fri may 23 19:28:10 2008]
%

%
% ---------- class ------------------------------------------------------------
%

\documentclass[8pt]{beamer}

%
% ---------- common -----------------------------------------------------------
%

\input{\path/package/opk/presentation.tex}


%
% title
%

\title{kaneton}

%
% document
%

\begin{document}

%
% title frame
%

\begin{frame}
  \titlepage
\end{frame}

%
% outline frame
%

\begin{frame}
  \frametitle{Outline}

  \tableofcontents
\end{frame}

%
% ---------- text -------------------------------------------------------------
%


%
% overview
%

\section{Overview}

% 1)

\begin{frame}
  \frametitle{Introduction}

  \term{kaneton} is an educational project intended for students to undertake
  in order to learn about operating system internals.
\end{frame}

% 2)

\begin{frame}
  \frametitle{History}

  \begin{itemize}
    \item[2004]
      \name{Julien Quintard} and \name{Jean-Pascal Billaud} decide
      to introduce an optional low-level programming course to first-year
      enginnering students, now known as \name{kastor};
    \item[2004]
      The course having been well received, \name{SRS - Syst\`emes, R\'eseaux,
      S\'ecurit\'e} students ask them to give such an introductory course the
      same year.
    \item[2005]
      The authorization is given to them to teach a kernel development course
      to \name{SRS} students, from January to October. They therefore decide
      to provide students with the design of a microkernel and let the students
      develop it from scratch, their way. \name{kaneton} is born.
    \item[2006]
      After \name{Jean-Pascal Billaud} fled to \name{VMWare}, \name{Julien
      Quintard} started developing a reference implementation and gave
      students this year a skeleton they had to complete. In addition,
      \name{Cedric Aubouy} and \name{Renaud Lienhart} joined the teaching
      team this year.

      \-

      Besides, the \name{LSE - Laboratoire Syst\`eme EPITA} joined the project
      by putting two students on the development of the \name{kaneton}
      research implementation. \name{Matthieu Bucchianeri} and \name{Renaud
      Voltz} thus joined the project.
  \end{itemize}
\end{frame}


% 3)

\begin{frame}
  \frametitle{History}

  \begin{itemize}
    \item[2007]
      This year, \name{Matthieu Bucchianeri} and \name{Renaud Voltz} took
      over the project for a year by lecturing the course and managing the
      project.

      \-

      \name{Julian Pidancet} and \name{Pierre Duteil} joined the project
      as part of the \name{LSE} but \name{Pierre Duteil} had to leave the
      project. Therefore, \name{Elie Bleton}, who was working at the
      \name{LRDE} before, joined the project.
    \item[2008]
      \name{Julian Pidancet} and \name{Elie Bleton} took over this year
      while \name{Laurent Lec} and \name{Nicolas Grandemange} joined as part
      of the \name{LSE}.

      \-

      At the end of this year, after problems with some students as well as
      conflicts with the \name{LSE}, \name{kaneton} maintainers decided not
      to work with the laboratory anymore.
    \item[2009]
      \name{EPITA} alumni were contacted and joined the educational project
      including \name{Francois Goudal}, \name{Benoit Marcot},
      \name{Enguerrand Raymond}, \name{Jean Guyader} but also
      \name{Fabien Le-Mentec}, an \name{EPITECH} alumnus.
  \end{itemize}
\end{frame}

% 4)

\begin{frame}
  \frametitle{Model}

  The project consists for students to fill in some missing parts of the
  kernel.

  \-

  However, note that, unlike \name{Tiger}, the missing parts will never
  be two lines long.

  \-

  Indeed, in \name{kaneton}, students are asked to implement a feature, say,
  providing memory management. Thus, students are free, in a certain way,
  to implement such a feature as they wish.

  \-

  Since the testing usually consists in verifying that the kernel is able
  to provide the functionality, students should be, most of the time, able
  to implement whatever algorithms \etc{} they wish.
\end{frame}

% 5)

\begin{frame}
  \frametitle{People}

  Let's present the people working on the educational project from where they
  studied to what they are now doing:

  \begin{itemize}
    \item
      \name{Francois Goudal};
    \item
      \name{Benoit Marcot};
    \item
      \name{Jean Guyader};
    \item
      \name{Baptiste Afsa}
    \item
      \name{Louis Vatier}; and
    \item
      \name{Julien Quintard}.
  \end{itemize}
\end{frame}

% 6)

\begin{frame}
  \frametitle{Project}

  \name{kaneton} is an important assignment of the \name{SRS}/\name{GISTR}
  curriculum and, as such, must be taken seriously.

  \-

  Especially, in the last years, \name{EPITA} decided to reduce the duration
  of the project to \term{three} months.

  \-

  As such, the other assignments imposed by the specializations in this
  period have been reduced so that students can focus on \name{kaneton}.
\end{frame}

%
% design
%

\section{Design}

% 1)

\begin{frame}
  \frametitle{Overview}

  The kaneton kernel is very different from the kernels you might be
  familiar with, especially the well-known \name{Windows}, \name{Linux},
  \name{BSD} and so forth.
\end{frame}

% 2)

\begin{frame}
  \frametitle{Microkernel}

  First, kaneton is a microkernel, making it modular from the design
  perspective as well as providing properties such as security.
\end{frame}

% 3)

\begin{frame}
  \frametitle{Distributed Computing}

  kaneton has been designed from the ground up for providing the operarting
  system advanced distributed computing features.
\end{frame}

% 4)

\begin{frame}
  \frametitle{Portability}

  kaneton has been designed with portability in mind, especially through
  a specific portability system that perfectly fits the kernel design.
\end{frame}

% 5)

\begin{frame}
  \frametitle{Organisation}

  Besides being a microkernel, kaneton is well organised in the inside,
  splitting functionalites into objects and managers.
\end{frame}

%
% stages
%

\section{Stages}

% 1)

\begin{frame}
  \frametitle{k0}

  The first project, named \term{k0}, consists for students to learn
  about low-level programming.

  \-

  This project comes with a lecture regarding the boot system as well
  as a practical session.

  \-

  \name{Francois Goudal} will be in charge of this stage which will last for
  a week.
\end{frame}

% 2)

\begin{frame}
  \frametitle{k1}

  \term{k1} consists for students to provide the kaneton microkernel the
  capacity to handle events.

  \-

  During this stage, a lecture on general kernel principles and a lecture on
  interrupts will be taught.

  \-

  \name{Julien Quintard} will be in charge of this stage which will last
  for a single week.

  \-

  Note that, starting with \name{k1}, the student snapshot will be used which
  provide students a development environment, making kernel development easier.
\end{frame}

% 3)

\begin{frame}
  \frametitle{k2}

  \term{k2} consists for students to provide the kaneton microkernel a
  memory management unit so that applications as well as the kernel itself
  can reserve, share \etc{} memory.

  \-

  During this stage, a lecture on portability as well as lectures on
  memory management will be taught.

  \-

  \name{Francois Goudal} will be in charge of this stage which will last
  for three weeks.
\end{frame}

% 4)

\begin{frame}
  \frametitle{k3}

  In \term{k3}, students will have to provide kaneton with execution contexts
  such that the kernel can execute multiple threads at the \textit{same} time.

  \-

  Lectures, during this stage, will discuss topics such as interrupts,
  concurrency, multi-processing, scheduling \etc{}

  \-

  \name{Benoit Marcot} will be in charge of this stage which will last for
  three weeks.
\end{frame}

% 5)

\begin{frame}
  \frametitle{Evaluation}

  For every stage, students will have the possibility to test their
  implementation by running, a limited number of times, the test suite used
  for evaluating their work.

  \-

  Besides, at the end of each stage, after submission, the kaneton test system
  will run the test suite and issue a mark according to the test results.

  \-

  Additionally, an exam will take place at the end of the semester to make
  sure that the notions tackled throughout the course are well understood
  by every student.
\end{frame}

%
% tools
%

\section{Tools}

% 1)

\begin{frame}
  \frametitle{Overview}

  The kaneton educational project relies on tools, sometimes developed
  internally.
\end{frame}

% 2)

\begin{frame}
  \frametitle{Web Site}

  The web site contains the documentation including design papers,
  the assignments \etc{} but also hosts the wiki which should be
  the starting point for every student seeking information.

  \-

  Noteworthy is that the wiki contains courses regarding the
  inline assembly, linking, pre-processing and so on. Students
  are invited to read them all as they will come handy when
  developing the kaneton stages.

  \-

  \name{Julien Quintard} should be contacted for requests regarding the
  web site and wiki.
\end{frame}

% 3)

\begin{frame}
  \frametitle{Snapshot}

  The student snapshot has been automatically generated from the current
  kaneton implementation.

  \-

  \name{Francois Goudal} is in charge of this process, hence should be
  contacted if you believe there is a mistake.
\end{frame}

% 4)

\begin{frame}
  \frametitle{Cheat}

  Every student's kaneton implementation will be tested to make sure that
  students did not cheat by relying on implementations by previous or
  current students.

  \-

  \name{Julien Quintard} is in charge of this tool.
\end{frame}

% 5)

\begin{frame}
  \frametitle{Test}

  Students' implementation will be tested in a real environment by applying
  a complete test suite; hence, validating the implementation's behaviour.

  \-

  \name{Jean Guyader} is responsible of this tool and should be contacted
  if necessary.
\end{frame}

%
% information
%

\section{Information}

% 1)

\begin{frame}
  \frametitle{Support}

  \begin{enumerate}
    \item
      \term{Website}

      \-

      You will find on \location{http://kaneton.opaak.org} documents regarding
      the project from the design to the implementation;
    \item
      \term{Wiki}

      \-

      The wiki \location{http://wiki.opaak.org} is the best way to get
      technical information as well as to help other students by adding
      and/or improving pages' contents;
    \item
      \term{Mailing-List}

      \-

      The kaneton educational students mailing-list
      \location{students@kaneton.opaak.org} will be used by teachers as
      an official means for communicating with students.

      \-

      Therefore, every student should subscribe to this mailing-list by sending
      an email to \location{students+subscribe@kaneton.opaak.org}.

      \-

      It is not allowed to post code on the mailing list, or give pointers to
      code in the snapshot that would provide obvious solution to somebody's
      question.
  \end{enumerate}
\end{frame}

% 2)

\begin{frame}
  \frametitle{Groups}

  Except for \name{k0} which is an individual project, the other projects
  from \name{k1} to \name{k3} are done in groups of \term{two} students.

  \-

  Every group is expected to send an email to
  \location{admin@opaak.org}.

  \-

  Note that we will use students' \name{EPITA} email addresses. As such,
  make sure that you check this email box.
\end{frame}

% 3)

\begin{frame}
  \frametitle{Reliance}

  As for \name{Tiger}, every stage depends on the previous one, except
  for \name{k0}.

  \-

  As such, test suites from the previous stages will also be used for both
  testing and marking.

  \-

  Students should therefore make sure to use their test permissions for making
  sure to fix the bugs of previous stages so that such bugs do not impact
  on the current stage results, hence mark.
\end{frame}

% 4)

\begin{frame}
  \frametitle{Machine}

  This year, the machine used by the kaneton educational project will consists
  of the \term{IBM-PC} platform coupled with the \term{IA-32} microprocessor
  architecture \ie{} the most common hardware system on the market.

  \-

  Although it is always best to test your implementation on a real machine,
  it takes time to reboot a real computer. You should therefore use an
  emulator such a \name{QEMU} or \name{Bochs} as they will enable you to
  test your kernel very quickly but they will also let you develop on
  a non-\name{IBM-PC}/\name{IA-32} machine such as a \name{Mac} for example.
\end{frame}

%
% conclusion
%

\section{Conclusion}

% 1)

\begin{frame}
  \frametitle{Concepts}

  Throughout the project, you will learn so many things from terminology,
  to how a computer boots, how the kernel controls the hardware and how it
  provides abstractions as basic as execution contexts.

  \-

  At the end of the project, you will definitely know that nothing is magic
  but purely logic and often actually very simple.
\end{frame}

% 2)

\begin{frame}
  \frametitle{Implementation}

  Although, starting the project by learning how to make a computer execute
  your code, you will end up, after three months, with a running kernel
  and operating system capable of executing programs, the whole on real
  hardware like the machine you have at home.
\end{frame}

% 3)

\begin{frame}
  \frametitle{Changes}

  Over the years, the project has greatly evolved, from a no-implementation
  project, to a reference-based project.

  \-

  However, being a project developed by volunteers willing to dedicate some
  time so that other students can learn, many things are missing and/or
  can be improved including the lectures but also the project implementation.

  \-

  In conclusion, keep in mind that the project exists only because of people
  willing to transfer their knowledge and please respect their effort.
\end{frame}

% 4)

\begin{frame}
  \frametitle{Fun}

  But most of all, kaneton should be about learning through fun!
\end{frame}

% 5)

\begin{frame}
  \frametitle{Reminder}

  Remember to perform the following tasks:

  \begin{itemize}
    \item
      Send an email to \location{admin@opaak.org} regarding the composition
      of your group, before \textbf{Wednesday 16th 2pm} or you will be put
      in a group by force;
    \item
      Subscribe to the students mailing-list
      \location{students@kaneton.opaak.org} by sending an email to
      \location{students+subscribe@kaneton.opaak.org};
    \item
      Watch closely the \name{Wiki} at \location{http://wiki.opaak.org} by
      subscribing the \name{RSS} feed for example;
    \item
      We advise SRS/GISTR lab roots to set up a \name{Xen}-based environment
      as testing on emulators only will become difficult over time;
    \item
      Students must have a ``rack'' containing a \name{POSIX}-compilant
      operating system for the \name{k0} practical session.
  \end{itemize}
\end{frame}

\end{document}


%
% class
%

\documentclass[10pt,a4wide]{article}

%
% packages
%

\usepackage[english]{babel}
\usepackage[T1]{fontenc}
\usepackage{a4wide}
\usepackage{graphicx}
\usepackage{fancyheadings}
\usepackage{multicol}
\usepackage{indentfirst}
\usepackage{color}
\usepackage{ifthen}
\usepackage{comment}
\usepackage{verbatim}
\usepackage{aeguill}

\pagestyle{fancy}

\setlength{\footrulewidth}{0.3pt}
\setlength{\parindent}{0.3cm}
\setlength{\parskip}{2ex plus 0.5ex minus 0.2ex}

%
% correction environment
%

\newenvironment{correction}%
   {
     \ifthenelse
	 {
	   \equal{\kaneton-latex}{subject}
	 }
	 {
	   \comment
	 }
	 {
	   \textbf{\color{red}{ ----- correction}}
	 }
   }%
   {
     \ifthenelse
	 {
	   \equal{\kaneton-latex}{subject}
	 }
	 {
	   \endcomment
	 }
	 {
	   \textbf{\color{red}{ ----- /correction}}
	 }
   }

%
% header
%

\rfoot{\scriptsize{Exam}}

\date{\scriptsize{\today}}


%
% title
%

\title{Noyaux et Syst\`emes d'Exploitation}

%
% header
%

\lhead{\scriptsize{EPITA\_ING2 - 2010\_S4 - NSE - Julien Quintard}}
\rhead{}

%
% document
%

\begin{document}

%
% title
%

\maketitle

%
% identation
%

\indentation{}

%
% --------- information -------------------------------------------------------
%

\begin{center}

\textbf{Documents Interdits}

\textbf{Dur\'ee 3 heures}

\scriptsize{Une copie bien pr\'esent\'ee sera toujours mieux not\'ee
            qu'une autre.}

\end{center}

%
% --------- text --------------------------------------------------------------
%

%
% boot
%

\section{Boot
         {\hfill{} \normalfont{\scriptsize{1 point}}}}

D\'ecrivez rapidement les \'etapes principales effectu\'ees par un
\name{bootloader}.

\begin{correction}

Un bootloader commence par localiser l'image du noyau du syst\`eme
d'exploitation qu'il doit d\`emarrer. Cette image peut \^etre stock\'ee sous
diff\'erentes formes (fichier dans un filesystem, donn\'ees brutes sur un
p\'eriph\'erique de stockage, image t\'el\'echargable par le r\'eseau, \ldots.
Une fois l'image localis\'ee, il doit la charger en m\'emoire. Le format de
l'image peut \^etre tr\`es vari\'e, il peut s'agir d'un binaire ELF, d'un
ex\'ecutable PE, d'un dump brut de la m\'emoire, \ldots.
Une fois le noyau charg\'e en m\'emoire, le bootloader transf\`ere
l'ex\'ecution sur le point d'entr\'ee du noyau.

\end{correction}

%
% memoire virtuelle
%

\section{M\'emoire virtuelle
         \normalfont{{\hfill{} \scriptsize{2 point}}}}

D\'ecrivez succintement en quoi consiste la notion de m\'emoire virtuelle, et
quels sont les avantages de ce m\'ecanisme.

\begin{correction}

La m\'emoire virtuelle consiste \`a s\'eparer les adresses manipul\'ees par les
programmes des adresses physiques. Cela permet entre autres :
\begin{itemize}
\item D'ex\'ecuter plusieurs t\^aches sur une m\^eme machine sans que les
t\^aches ne doivent \^etre compil\'ees (link\'ees) sp\'ecifiquement, en
fonction des autres t\^aches.
\item De prot\'eger les t\^aches s'ex\'ecutant au sein de la machine, chaque
t\^ache ayant son propre espace d'adressage, et ne pouvant donc en aucun cas
acc\'eder aux emplacements m\'emoire des autres t\^aches.
\item D'\'etendre virtuellement la m\'emoire physique gr\^ace aux
p\'eriph\'eriques de stockage, gr\^ace au m\'ecanisme de swap.
\end{itemize}

\end{correction}

%
% segmentation et pagination
%

\section{Segmentation et Pagination
         \normalfont{{\hfill{} \scriptsize{3 points}}}}

Pour chacun des deux principaux m\'ecanismes de m\'emoire virtuelle,
d\'ecrivez bri\`evement son fonctionnement g\'en\'eral.

Vous ferez attention \`a ne pas d\'ecrire des choses propres \`a
l'architecture \name{ia32}, seul le concept g\'en\'eral est
demand\'e.

\begin{correction}

La segmentation consiste \`a diviser la m\'emoire physique en segments. Un
segment est d\'efini par une adresse de base et une taille. Le syst\`eme
d'exploitation maintient une liste de segments. Lorsque le processeur manipule
une adresse virtuelle, il va chercher dans un registre sp\'ecial, le segment
selector, quel segment utiliser. Il va ensuite additionner l'adresse virtuelle
avec l'adresse de base du segment, et v\'erifier que l'adresse virtuelle est
inf\'erieure \`a la taille du segment. Le r\'esultat de l'addition d\'etermine
l'adresse physique.\\

La pagination consiste \`a diviser la m\'emoire physique en un grand nombre de
zones de taille identique, appel\'ees pages. Une adresse virtuelle sera
d\'ecoup\'ee en 2 parties, la premi\`ere servant \`a identifier la page, et la
deuxi\`eme servant d'offset au sein de la page. Le syst\`eme d'exploitation
maintient une structure qui fait la correspondance entre une adresse virtuelle
de base d'une page, et son adresse en m\'emoire physique. Lors d'un acc\`es
m\'emoire, la MMU interroge son cache, par l'interm\'ediaire des TLB. Si la
correspondance n'est pas trouv\'ee au sein des TLB, la MMU g\'en\`ere une
exception pour que le syst\`eme d'exploitation puisse r\'esoudre la
correspondance et remplir le cache correctement, afin que la MMU puisse,
apr\`es que l'OS ait r\'esum\'e, r\'ealiser l'acc\`es en m\'emoire physique.

\end{correction}

%
% peripheriques
%

\subsection{P\'eriph\'eriques
            \normalfont{{\hfill{} \scriptsize{4 points}}}}

La platforme---processeur et chipset---met \`a disposition plusieurs
m\'ecanismes pour communiquer avec les p\'eriph\'eriques.

D\'ecrivez ces m\'ecanismes puis donnez des exemples d'utilisation de
chacun d\'entre eux.

\begin{correction}

XXX

\end{correction}

%
% virtualisation
%

\section{Virtualisation
         \normalfont{{\hfill{} \scriptsize{2 points}}}}

Quelle est la diff\'erence entre un \name{hyperviseur}
type 1 et type 2 ?

\begin{correction}

XXX

\end{correction}

% 
% securite
%

\section{S\'ecurit\'e
         \normalfont{{\hfill{} \scriptsize{4 points}}}}

Un service d\'esire fournir \`a certains clients la possibilit\'e
d'effectuer des op\'erations via l'utilisation de \name{capabilities}.

\'Etant donn\'e que le service ne g\`ere aucun objet mais propose
aux clients d'effectuer un maximum de $13$ op\'erations diff\'erentes,
veuillez d\'efinir le format des \name{capabilities} \`a utiliser en
d\'ecrivant la taille et l'utilit\'e de chaque champ.

Enfin, vous d\'etaillerez le processus de v\'erification effectu\'e
par le service en question.

\begin{correction}

Consid\'erant un syst\`eme d'exploitation non r\'eparti, un champ sera
utilis\'e pour identifi\'e le service en question, un \texttt{i\_task}
sur $64$ bits par exemple. Un champ de $13$ bits sera utilis\'e pour
d\'ecrire les $13$ op\'erations autoris\'ees. Enfin, un champ \name{Check}
de $64$ bits, par exemple, sera utilis\'e pour s\'ecuriser la
\name{capability}.

Le processus de v\'erification consiste \`a r\'ecup\'erer le \name{Check}
d'origine \`a partir d'une structure interne au service, \`a appliquer un
\name{XOR} entre le \name{Check} d'origine et le champ \name{Operations}
de la \name{capability} re\c{c}ue pour enfin appliquer une fonction
\`a sens unique. Il suffit finalement de comparer le r\'esultat avec le
champ \name{Check} contenu dans la \name{capability}. Si ils sont
\'egaux, alors la \name{capability} est valide.

\end{correction}

%
% windows nt
%

\section{Windows NT
         \normalfont{{\hfill{} \scriptsize{4 points}}}}

Dans un texte structur\'e, d\'ecrivez les principes fondamentaux autour
desquels le noyau \name{NT} a \'et\'e con\c{c}u. En particulier, vous
\'enoncerez de mani\`ere claire les probl\'ematiques cibl\'ees par ce type de
noyau, puis vous mettrez l'accent sur les r\'eponses apport\'ees par \name{NT}
et son design.

Lorsque cela s'av\`ere pertinent, vous incluerez des d\'etails d'implantation.
\`A titre d'exemple, des comparaisons pourront \^etre faites avec d'autres
noyaux existants, tels que \name{Linux}.

\begin{correction}

Le noyau \name{NT} est con\c{c}u et d\'{e}velopp\'{e} au d\'{e}but des ann\'{e}es
1990 par une \'{e}quipe dirig\'{e}e par \name{Dave Cutler} (\name{DEC, VMS}). C'est
un noyau monolithique qualifi\'{e} par certains d'hybride du fait de la communication
par messages entre les drivers. Tous les drivers partagent le m\^{e}me espace d'adressage,
ce qui le diff\'{e}rencie donc d'un micronoyau. Le choix d'une communication par message
r\'{e}pond ainsi \`{a} une probl\'{e}matique de modularit\'{e}, et non d'isolation.

Depuis sa cr\'{e}ation, les aspects fondamentaux du noyau n'ont pas chang\'{e}
malgr\'{e} l'\'{e}volution des architectures: passage au multiprocesseur, multiplication
des resources \`{a} g\'{e}rer (plus de devices, plus de fichiers ouverts)... Ceci est
en quelque sorte la preuve de la flexibilit\'{e} du design, qui s'articule autour des
points suivants:

\begin{itemize}
  \item
    Mod\`{e}le d'IO asynchrones: le noyau g\`{e}re toutes les IOs de mani\`{e}re
    asynchrone. De ce fait, l'appel \`{a} une fonction synchrone (ie. \name{WriteFile})
    est implement\'{e} par dessus ce mod\`{e}le. L'int\'{e}r\^{e}t principal d'un
    mod\`{e}le asynchrone est de permettre l'ex\'{e}cution de code li\'{e} au cpu
    (\name{cpu bound}) en parall\`{e}le de l'attente de la completion d'une IO
    (\name{io bound}). Par exemple, une solution de chiffrement peut maintenir une liste
    de donn\'{e}es \`{a} chiffrer dans l'attente de la compl\'{e}tion d'une IO. On a de
    cette mani\`{e}re une progression globale du syst\`{e}me.
  \item
    Mod\`{e}le de thread 1:1: Chaque thread utilisateur est repr\'{e}sent\'{e} par
    un thread noyau. Puisque la programmation multithread\'{e}e est un mod\`{e}le
    largement adopt\'{e} par les programmeurs, un mapping 1:1 permet de gagner en
    performance en \'{e}vitant \`{a} un processus d'\^{e}tre bloqu\'{e} par un
    thread dans l'attente de la compl\'{e}tion d'une IO.
  \item
    Mod\`{e}le orient\'{e} objet: Toute resource est manipul\'{e}e par le noyau
    via une interface commune, celle d'objet. Cette interface comprend entre autre
    les proc\'{e}dures d'ouverture, de fermeture et de securit\'{e}. Chaque type de
    resource a un objet correspondant: \name{FILE\_OBJECT}, \name{DEVICE\_OBJECT},
    \name{DRIVER\_OBJECT} ... Un programme utilisateur manipule tous les objets
    par \name{HANDLE}.
  \item
    Le gestionnaire d'objet (\name{Object Manager}) centralise les aspects fondamentaux
    li\'{e}s \`{a} la gestion des resources, \`{a} savoir:
    \begin{itemize}
      \item
      Le nommage: la r\'{e}cup\'{e}ration d'une resource \`{a} partir d'un nom. Ce
      nom est structur\'{e} de la m\^{e}me fa\c{c}on qu'un chemin dans un syst\`{e}me
      de fichier. Il y a en g\'{e}n\'{e}ral un r\'{e}pertoire par type de resource,
      et les liens symboliques sont support\'{e}s. C'est de cette facon qu'une lettre
      (ie. \name{C:}) peut \^{e}tre li\'{e}e \`{a} un volume (ie. \name{/Device/HarddiskVolume0}).
      \item
      La recherche: la r\'{e}cup\'{e}ration de l'instance d'une resource \`{a} partir
      d'un \name{HANDLE}.
      \item
      la s\'{e}curit\'{e}: lorsqu'un processus utilisateur effectue une operation sur
      un objet (ie. ReadFile), le noyau doit valider cette operation. Pour ce faire,
      la proc\'{e}dure de s\'{e}curit\'{e} associ\'{e}e au type d'objet sera invoqu\'{e}e,
      et cela quel que soit le type d'objet. La fonction document\'{e}e est
      \name{ObReferenceObjectByHandle}. Cela permet une gestion du contr\^{o}le d'acc\`{e}s
      centralis\'{e}e et coh\'{e}rente, ce qui n'est pas le cas dans \name{Linux}.n
    \end{itemize}
  \item
    Mod\`{e}le de driver en stack: Pour permettre l'interaction entre les differents
    periph\'{e}riques, le noyau organise les drivers sous la forme de stack selon leur
    relation. \`{A} cette fin, le gestionnaire d'IO expose deux primitives:
    \begin{itemize}
      \item
      Une structure d\'{e}crivant les messages: \name{IRP} (Interrupt Request Packet),
      \item
      Une fonction de message passing: \name{IoCallDriver}.
    \end{itemize}
    De cette mani\`{e}re, des interactions dynamiques complexes peuvent etre modelis\'{e}es:
    solutions de stockage, chiffrement de disque, piles r\'{e}seau, syst\`{e}mes de fichier
    virtuels ...
    Ce mod\`{e}le permet \`{a} des drivers propri\'{e}taires de cohabiter, pour autant que
    chacun respecte l'ensemble des conventions impos\'{e}es, telles que le routage des
    messages aux drivers sous jaccents ansi que leur compl\'{e}tion.
\end{itemize}

Aussi, il est important de noter la qualit\'{e} du kit de d\'{e}vloppement de driver pour
\name{NT} (ie. \name{Windows Driver Kit}), ainsi que les outils associ\'{e}s: injection de
faute, analyse dynamique d'erreurs communes lors de la conception d'un driver...

\end{correction}

\end{document}
