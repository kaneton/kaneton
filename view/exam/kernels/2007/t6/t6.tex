%
% ---------- header -----------------------------------------------------------
%
% project       kaneton
%
% license       kaneton
%
% file          /home/mycure/kaneton/view/exam/kernels/2007/t6/t6.tex
%
% created       julien quintard   [mon may 14 22:02:59 2007]
% updated       julien quintard   [thu may 22 19:38:00 2008]
%

%
% ---------- setup ------------------------------------------------------------
%

%
% path
%

\def\path{../../../..}

%
% template
%

%%
%% licence       kaneton licence
%%
%% project       kaneton
%%
%% file          /home/buckman/kaneton/view/templates/exam.tex
%%
%% created       julien quintard   [fri dec  2 22:20:57 2005]
%% updated       matthieu bucchianeri   [sat feb 10 13:47:41 2007]
%%

%
% kaneton latex
%

%%
%% licence       kaneton licence
%%
%% project       kaneton
%%
%% file          /home/mycure/kaneton/view/papers/kaneton/kaneton.tex
%%
%% created       julien quintard   [thu dec  8 00:26:00 2005]
%% updated       julien quintard   [thu mar  2 13:59:43 2006]
%%

%
% template
%

%%
%% licence       kaneton licence
%%
%% project       kaneton
%%
%% file          /home/mycure/kaneton/view/templates/book.tex
%%
%% created       julien quintard   [wed mar  1 23:45:22 2006]
%% updated       julien quintard   [thu may  4 12:36:54 2006]
%%

%
% class
%

\documentclass[10pt,a4wide]{book}

%
% packages
%

\usepackage[english]{babel}
\usepackage[T1]{fontenc}
\usepackage{a4wide}
\usepackage{fancyheadings}
\usepackage{multicol}
\usepackage{indentfirst}
\usepackage{graphicx}
\usepackage{color}
\usepackage{xcolor}
\usepackage{verbatim}

\usepackage{aeguill}

\usepackage[Lenny]{../../../tools/latex/fncychap}

\pagestyle{fancy}

\setlength{\footrulewidth}{0.3pt}
\setlength{\parindent}{0.3cm}
\setlength{\parskip}{2ex plus 0.5ex minus 0.2ex}

%
% logos
%

\newcommand{\logos}
  {
    \begin{center}
      \includegraphics[scale=0.8]{../../logos/kaneton.pdf}
    \end{center}
  }

%
% colors
%

\definecolor{functioncolor}{rgb}{0.40,0.00,0.00}
\definecolor{commandcolor}{rgb}{0.00,0.00,0.40}
\definecolor{verbatimcolor}{rgb}{0.00,0.40,0.00}
\definecolor{noticecolor}{rgb}{0.87,0.84,0.02}

%
% function
%

\newcommand\function[3]{
  \begin{tabular}{p{0.2cm}p{13.8cm}}
  & {\color{functioncolor}\textbf{#1}}#2
  \end{tabular}

  \begin{tabular}{p{1cm}p{13cm}}
  & #3
  \end{tabular}}

%
% align
%

\newcommand\align[1]{
  \\ & \hspace{#1}}

%
% argument
%

\newcommand\argument[1]{\textit{#1}}

%
% command
%

\newcommand\command[2]{
  \begin{tabular}{p{0.2cm}p{13.8cm}}
  & {\color{commandcolor}\textbf{#1}}
  \end{tabular}

  \begin{tabular}{p{1cm}p{13cm}}
  & #2
  \end{tabular}}

%
% notice
%

\newcommand\notice[1]{
  {\color{noticecolor}\textbf{Notice}}

  \begin{tabular}{p{0.2cm}p{13.8cm}}
  & #1
  \end{tabular}}

%
% example
%

\newcommand\example[1]{
  \textit{Example:}

  \begin{tabular}{p{0.2cm}p{13.8cm}}
  & \textit{#1}
  \end{tabular}}

%
% warning XXX
%

%
% verbatim stuff
%

\makeatletter

\renewcommand{\verbatim@font}
  {\ttfamily\footnotesize\color{verbatimcolor}\selectfont}

\def\verbatim@processline{\hskip15ex\the\verbatim@line\par}

\makeatother

%
% header
%

\rhead{}
\rfoot{\scriptsize{The kaneton microkernel project}}

\date{\scriptsize{\today}}


%
% header
%

\lhead{\scriptsize{The kaneton microkernel project reference}}
\rhead{}

%
% title
%

\title{The kaneton microkernel reference
       \logos}

%
% authors
%

\author{\small{Julien Quintard},
        \small{Matthieu Bucchianeri},
        \small{Renaud Voltz}}

%
% document
%

\begin{document}

%
% title
%

\maketitle

%
% --------- text --------------------------------------------------------------
%

%
% authors
%

This document describes the kaneton microkernel reference project.

This document should be used by every student willing implement the
kaneton microkernel.

All the kaneton documents are available on the official website:
\textbf{http://www.kaneton.org}.

This document is under the \textbf{kaneton license}.

This document will reference the kaneton people by the word \textit{we}.

\newpage

The kaneton project was introduced at EPITA by two computer science
students:

\begin{itemize}
  \item
    Julien Quintard \footnote{quinta\_j@epita.fr}
  \item
    Jean-Pascal Billaud \footnote{billau\_j@epita.fr}
\end{itemize}

This document especially describes the kaneton reference microkernel
developed by:

\begin{itemize}
  \item
    Julien Quintard
  \item
    Matthieu Bucchianeri \footnote{bucchi\_m@epita.fr}
  \item
    Renaud Voltz \footnote{voltz\_r@epita.fr}
\end{itemize}

Nevertheless, many people contributed to this project and we thank them.

%
% toc
%

\tableofcontents

%
% chapters
%

XXX revoir ordre pour mieux introduire kaneton: design.pdf

%
% ---------- header -----------------------------------------------------------
%
% project       kaneton
%
% license       kaneton
%
% file          /home/mycure/kaneton/view/book/kaneton/goals.tex
%
% created       julien quintard   [fri jun  1 13:58:12 2007]
% updated       julien quintard   [mon may 19 23:09:48 2008]
%

%
% ---------- goals ------------------------------------------------------------
%

\chapter{Goals}
\label{chapter:goals}

In this chapter we will briefly introduce the kaneton microkernel
through the kaneton microkernel goals.

\newpage

%
% ---------- text -------------------------------------------------------------
%

The project was primarily designed by two students in computer science,
\name{Julien Quintard} and \name{Jean-Pascal Billaud}.

These two students previously actively contributed to the development
of a nanokernel-based operating system project in a French research laboratory.
This system was not powerful enough, especially from the design point of view.

Therefore, the two students started the design of a new microkernel
by their own, called \term{kaneton}, for educational purposes.

The design was based on five fundamental guidelines.

\begin{enumerate}
  \item
    \textbf{Educational}

    \-

    The kaneton project is built to become an educational project. The design
    as well as the implementation must therefore be as understandable as
    possible so that everyone interested in kernel internals can go through the
    documents and source code and actually understand how it works.

    \-

    This \textit{understandable} property can be achieved through a very clear
    and coherent design. Moreover, the implementation should be written using
    modern tools and techniques to make the code as generic as possible and
    easily readable.
  \item
    \textbf{Portability}

    \-

    The microkernel was particularly designed to be portable. The designers
    tried to develop a portability system powerful enough to port kaneton on
    any, existing or not, architectures.
  \item
    \textbf{Maintanability}

    \-

    Although microkernel-based operating systems rely on a modular design,
    kaneton designers also wanted the microkernel itself to be modular and
    maintainable.
  \item
    \textbf{Distributed Computing}

    \-

    The kaneton microkernel must be designed to fit distributed operating
    systems requirements. Indeed, the kaneton microkernel was developed in
    order to design and implement a distributed operating system named
    \term{kayou}.

    \-

    This point led to many specific choices in the kaneton microkernel design.
  \item
    \textbf{Demystification}

    \-

    kaneton people wanted to break some well-known kind of computer
    science rules. Indeed, for instance, many computer scientists consider
    the source code as the actual documentation. Also, for many low-level
    programmers, the kernel boot source code and more generally the
    kernel source code itself cannot be understandable, clear and coherent as
    it is related to low-level programming: microprocessor, devices \etc{}

    kaneton people paid particular attention to the microkernel source code to
    be easily understandable, maintainable and extendable. Moreover, kaneton
    people tried to write documentation for every part of the project.
\end{enumerate}

Notice that building an educational microkernel project is nothing innovative.
Indeed few other projects already exist; the most popular being \name{MINIX}
from \name{Vrije Universiteit}, \name{NachOS} from \name{Berkeley University}
or \name{PintOS} from \name{Stanford University}.

kaneton people tried to design and implement a modern microkernel since, the
original \name{MINIX} microkernel for example, do not use modern development
tools. Moreover, the kaneton source code is heavily commented and use modern
languages techniques while trying to stay easily understandable.

The educational characteristic of kaneton does not constraint it from being
optimised afterwards. kaneton people believe that implementing optimised
algorithms in the first place does not lead to maintainable implementations.

Finally, note that the kaneton project is actually composed of two projects:
the \name{kaneton microkernel \term{educational} project} which provides
everything necessary to students willing to learn about kernels internals;
and the \name{kaneton microkernel \term{research} project} which focuses on
designing and implementing a powerful, reliable, flexible microkernel.
Obviously these two projects are highly related as the kaneton educational
project relies on the implementation of the kaneton research project.

%%
%% licence       kaneton licence
%%
%% project       kaneton
%%
%% file          /home/mycure/kaneton/view/papers/kaneton/overview.tex
%%
%% created       matthieu bucchianeri   [mon jan 30 17:09:45 2006]
%% updated       julien quintard   [thu mar  2 13:12:22 2006]
%%

%
% overview
%

\chapter{Overview}

XXX ce chapitre va vous aider a reconnaitre les fonctionnalites principale
XXX d'un kernel dans kaneton.

The kaneton microkernel is only the core of an operating system.
Main tasks like hardware drivers or user services are implemented as
\textbf{servers}. So the microkernel only has a few functionalities to
provide:

\begin{itemize}
  \item
    Memory management.
  \item
    Process management.
  \item
    Communication.
  \item
    Events.
\end{itemize}

In this chapter we will describe briefly these tasks and all the
associated managers.

%
% memory management
%

\section{Memory Management}

Handling the memory -- from virtual address space to physical
addressing -- is done by three major managers, the \textbf{as},
\textbf{segment} and \textbf{region} managers.

%
% as
%

\subsection{as}

The address space manager just manages the different address spaces
used by the kaneton tasks.

In kaneton, we call an \textbf{as - address space} a list of memory
locations referenced by a task. Each task has its own address space.

%
% segment
%

\subsection{segment}

The segment manager just manages the segments reserved by
the different kaneton entities including the kernel, the drivers etc..

In kaneton terms a \textbf{segment} is a contiguous area of reserved
physical memory.

%
% region
%

\subsection{region}

The region manager keeps track of regions used to map segments for
each address space reserved on the system.

In kaneton, a \textbf{region} is contiguous area of virtual memory
mapping a segment's part.

%
% process management
%

\section{Process Management}

XXX

%
% communication
%

\section{Communication}

XXX

%
% events
%

\section{Events}

XXX

%%
%% copyright quintard julien
%% 
%% kaneton
%% 
%% development-environment.tex
%% 
%% path          /home/mycure/kaneton
%% 
%% made by mycure
%%         quintard julien   [quinta_j@epita.fr]
%% 
%% started on    Tue Jul  5 12:23:08 2005   mycure
%% last update   Sun Oct 23 02:55:45 2005   mycure
%%

%
% class
%

\documentclass[8pt]{beamer}

%
% packages
%

\usepackage{pgf,pgfarrows,pgfnodes,pgfautomata,pgfheaps,pgfshade}
\usepackage{colortbl}
\usepackage{times}
\usepackage{amsmath,amssymb}
\usepackage{graphics}
\usepackage{graphicx}
\usepackage{color}
\usepackage{xcolor}
\usepackage[english]{babel}
\usepackage{enumerate}
\usepackage[latin1]{inputenc}

%
% style
%

\usepackage{beamerthemesplit}
\setbeamercovered{dynamic}

%
% verbatim font
%

\definecolor{verbatimcolor}{rgb}{0,0.4,0}

\makeatletter
\renewcommand{\verbatim@font}
  {\ttfamily\footnotesize\color{verbatimcolor}\selectfont}
\makeatother

%
% new line
%

\newcommand{\nl}[0]{\vspace{0.4cm}}

%
% title
%

\title{Development Environment}

%
% authors
%

\author
{
  Julien~Quintard\inst{1} \\
  {\tiny julien.quintard@gmail.com}
}

\institute
{
  \inst{1} kaneton distributed operating system project
}

%
% date
%

\date{\today}

%
% logos
%

\pgfdeclareimage[interpolate=true,width=34pt,height=18pt]
                {epita}{../../logos/epita}
\pgfdeclareimage[interpolate=true,width=49pt,height=18pt]
                {upmc}{../../logos/upmc}
\pgfdeclareimage[interpolate=true,width=25pt,height=18pt]
                {lse}{../../logos/lse}

%
% table of contents at the beginning of each section
%

\AtBeginSection[]
{
  \begin{frame}<beamer>
   \frametitle{Outline}
    \tableofcontents[current]
  \end{frame}
}

%
% table of contents at the beginning of each subsection
%

\AtBeginSubsection[]
{
  \begin{frame}<beamer>
   \frametitle{Outline}
    \tableofcontents[current,currentsubsection]
  \end{frame}
}

%
% document
%

\begin{document}

%
% title frame
%

\begin{frame}
  \titlepage

  \begin{center}
    \pgfuseimage{epita} \hspace{0.1cm} \pgfuseimage{upmc} \hspace{0.1cm}
    \pgfuseimage{lse} \hspace{0.1cm}
  \end{center}
\end{frame}

%
% outline frame
%

\begin{frame}
  \frametitle{Outline}
  \tableofcontents
\end{frame}

%
% overview
%

\section{Overview}

% 1)

\begin{frame}
  \frametitle{Introduction}

  From the previous years, a development environment was introduced.

  \nl

  The questions are:

  \begin{enumerate}[<+->]
    \item
      Why?
    \item
      What are the advantages and disadvantages of such a
      development environment?
    \item
      How did the other promotions do?
  \end{enumerate}
\end{frame}

% 2)

\begin{frame}
  \frametitle{Explanations}

  Over the years, the kaneton project evolved, starting with a very
  simple introduction to low-level programming, to microkernel
  development and finally to a distributed operating system project.

  \nl

  Going always further implies many modifications in the project
  including:

  \begin{itemize}[<+->]
    \item
      The courses given which now go from the Intel processor to
      the distributed operating system concepts
    \item
      The assignments which always evolve to study advanced topics
    \item
      The context because we now have to provide parts of the microkernel
      to avoid students a development from scratch
    \item
      .. and so the requirements
  \end{itemize}
\end{frame}

% 3)

\begin{frame}
  \frametitle{The Courses}

  The kaneton project now comes with four courses:

  \begin{enumerate}
    \item
      The design of the kaneton distributed operating system including
      the microkernel
    \item
      The Intel processor
    \item
      The kernel concepts
    \item
      The distributed operating system concepts
  \end{enumerate}
\end{frame}

% 4)

\begin{frame}
  \frametitle{The Assignments}

  During the year 2005, the students develop a poor microkernel
  from scratch with few functionalities, a driver and finally a baby
  file system.

  \nl

  We cannot ask the students of the year 2006 to develop the same project
  but to go further to study advanced topics like distributed algorithms.

  \nl

  So, we cannot ask the students to develop every parts of the microkernel
  because this takes much time and implies to not study advanced
  topics.
\end{frame}

% 5)

\begin{frame}
  \frametitle{The Context}

  Providing students parts of the microkernel is not enough.

  \nl

  Indeed, we decided to provide a complete development environment
  including:

  \begin{itemize}
    \item
      Makefiles
    \item
      Shell scripts
    \item
      Papers
    \item
      Tools
    \item
      .. everything you need to start microkernel development
  \end{itemize}
\end{frame}

% 6)

\begin{frame}
  \frametitle{Why?}

  The remaining question is:

  \nl

  \textbf{Why providing such a development environment and not letting us
    develop one ourself?}

  \nl

  The answers simply are:

  \begin{itemize}
    \item
      Developing such a development environment takes much time and
      need experience
    \item
      This development environment include very powerful features:
      multiusers cooperation, different operating systems etc..
    \item
      Finally, students will not be able to create such a complicated
      development tree so it is provided to not waste time.
  \end{itemize}
\end{frame}

% 7)

\begin{frame}
  \frametitle{The Requirements}

  The students starting the kaneton project should think that they
  will learn many many things during the year.

  \nl

  This year, we are trying to lead students to a distributed operating
  system.

  \nl

  This implies more concepts, algorithms and techniques to learn.

  \nl

  To do this we introduced more courses but the students will have
  to work hard to be able to success.
\end{frame}

% 8)

\begin{frame}[containsverbatim]
  \frametitle{Tree}

  \begin{center}

  \begin{verbatim}
    /
      conf/
      core/
      doc/
      drivers/
      env/
      export/
      libs/
      papers/
      programs/
      services/
      tools/
  \end{verbatim}

  \end{center}
\end{frame}

%
% conf
%

\section{conf}

% 1)

\begin{frame}
  \frametitle{Overview}

  The \textbf{conf} directory contains user variables used to parameterise:

  \begin{itemize}
    \item
      the development environment: makefiles, scripts etc..
    \item
      the kernel
  \end{itemize}

  \nl

  This configuration system is very interesting coupled with versionning
  system.

  \nl

  Indeed, you can develop using special compilation flags, specific kernel
  configuration without conflicts with other developers.
\end{frame}

% 2)

\begin{frame}[containsverbatim]
  \frametitle{Tree}

  \begin{verbatim}
    conf/
      mycure/
        conf.c
        conf.h
        kaneton.conf
        modules.conf
        mycure.conf
      pwipwi/
      chiche/
  \end{verbatim}

  This configuration system uses the shell variable \$USER to find
  the main configuration file: \textbf{conf/\$USER/\$USER.conf}.
\end{frame}

% 3)

\begin{frame}
  \frametitle{conf.c}

  This file is not used yet.
\end{frame}

% 4)

\begin{frame}
  \frametitle{conf.h}

  This file contains macros to configure the kernel:

  \begin{itemize}
    \item
      \textbf{CONF\_TITLE}
    \item
      \textbf{CONF\_VERSION}
    \item
      \textbf{CONF\_DEBUG}
    \item
      etc..
  \end{itemize}

  \nl

  This file is included by the kernel code.
\end{frame}

% 5)

\begin{frame}
  \frametitle{kaneton.conf}

  This configuration file is used to pass arguments at the runtime to the
  servers.

  \nl

  This file is also used to configure kernel and servers input variables.
\end{frame}

% 6)

\begin{frame}
  \frametitle{modules.conf}

  This file contains the list of the modules to be loaded by the
  multi-bootloader.

  \nl

  These modules will be passed to the kernel at the boot time.

  \nl

  Be careful, a module here is not a module in the Linux or BSD terms.

  \nl

  A module is simply a file to load.
\end{frame}

% 7)

\begin{frame}
  \frametitle{\$USER.conf}

  Finally the main configuration file contains the configuration
  variables for the development environment.

  \nl

  This file uses the syntax of the make files.

  \nl

  Every variable defined in this file will be used by the makefiles
  and the scripts.
\end{frame}

%
% env
%

\section{env}

% 1)

\begin{frame}
  \frametitle{Overview}

  The \textbf{env} directory contains the different development environments.

  \nl

  This directory is the heart of the kaneton development system.

  \nl

  Indeed, a user can develop the kaneton project on a Mac machine using
  cross compilation for Intel processors ('cause PowerPC processor)
  while another one is using a FreeBSD operating system on an Intel processor.

  \nl

  So, the development environment has to deal with these different operating
  systems and architectures just for the development.
\end{frame}

% 2)

\begin{frame}
  \frametitle{Our System}

  To do this, we decided to introduce an environment system.

  \nl

  Every time a user gets the kaneton development tarball, he first has to
  create his development environment given a couple operating system and
  architecture which leads to an environment.

  \nl

  Once the environment is installed, the user can develop, compile the kernel
  etc.. without problems because everything (makefiles, scripts etc..) use
  the binaries, variables etc.. for his environment.

  \nl

  The environment is specified in the user configuration file.
\end{frame}

% 3)

\begin{frame}[containsverbatim]
  \frametitle{Tree}

  \begin{verbatim}
    env/
      clean.sh
      init.sh
      unix/
        clean.sh
        init.sh
        kaneton.mk
      macos-powerpc.ia32/
  \end{verbatim}

  \nl

  Here the \textbf{unix} is considered as the generic unix
  environment but everyone can add a specific linux, FreeBSD, Solaris etc..
  environment.
\end{frame}

% 4)

\begin{frame}
  \frametitle{init.sh}

  The \textbf{init.sh} shell script is used to install the development
  environment.

  \nl

  This script first gets the configuration variables from the user
  configuration file, then calls the specific \textbf{init.sh} script
  of the given environment.

  \nl

  Finally the script installs some links and initialises the makefile
  dependencies.

  \nl

  The \textit{[environment]}/init.sh shell script is used to install
  specific stuff.
\end{frame}

% 5)

\begin{frame}
  \frametitle{clean.sh}

  The \textbf{clean.sh} shell script just cleans the environment.

  \nl

  This shell script also call the environment specific clean.sh script.
\end{frame}

% 6)

\begin{frame}
  \frametitle{kaneton.mk}

  The \textbf{kaneton.mk} makefile dependency is the heart of the
  kaneton compilation system.

  \nl

  Indeed, every makefile is composed of calls to special routines
  which are implemented by the makefile dependency depending on the
  environment: operating system plus architecture source and destination.

  \nl

  Moreover the \textbf{kaneton.mk} makefile dependency includes the
  user configuration file so each makefile of the system is able to
  use user defined variables.

  \nl

  The kaneton compilation system uses a very special gmake feature:
  the makefile \textbf{call} function.
\end{frame}

% 7)

\begin{frame}[containsverbatim]
  \frametitle{Use}

  \begin{verbatim}
    $ make init
    [+] installing environment

    [+] your current configuration:
    [+]   environment:              unix
    [+]   architecture:             ia32
    [+]   multi-bootloader:         grub

    [...]

    $ make clean
    [+] cleaning environment

    [...]

    $ 
  \end{verbatim}
\end{frame}

%
% tools
%

\section{tools}

% 1)

\begin{frame}
  \frametitle{Overview}

  The \textbf{tools} directory contains programs, scripts, special
  files used by the kaneton project.

  \nl

  For example a script to initialise and install modules on a grub
  bootloader boot device is provided in the subdirectory
  \textit{scripts/multi-bootloaders/grub/}.

  \nl

  The \textbf{tools} directory also contains the ld scripts used
  to correctly compile the bootstrap, the bootloader, the kernel, the
  drivers, the services and the programs.
\end{frame}

% 2)

\begin{frame}[containsverbatim]
  \frametitle{Tree}

  \begin{verbatim}
    tools/
      scripts/
        ld/
          arch/
            ia32/
              bootstrap.lds
              bootloader.lds
              kaneton.lds
              driver.lds
              service.lds
              user.lds
        multi-bootloaders/
          grub/
          lilo/
        prototypes/
          mkp.py
  \end{verbatim}
\end{frame}

% 3)

\begin{frame}[containsverbatim]
  \frametitle{Use}

  \begin{verbatim}
    $ make build
    [+] initialising boot system

    [+] boot system initialised successfully
    $ make install
    [+] initialising boot system

    [+] /tmp/menu.lst
    [+] core/bootloader/bootloader
    [+] core/kaneton/kaneton
    [+] conf/mycure/kaneton.conf
    [+] drivers/cons/cons
    [+] services/dsh/dsh

    [+] boot system initialised successfully
    $ 
  \end{verbatim}
\end{frame}

% 4)

\begin{frame}[containsverbatim]
  \frametitle{Prototypes}

  The compilation system permits to generate the prototypes in a very easy
  and elegant way.

  \begin{verbatim}
    $ make proto
    [PROTOTYPES]            libdata.h
    [PROTOTYPES]            libstring.h
    [PROTOTYPES]            libsys.h
    [PROTOTYPES]            bootloader.h
    [PROTOTYPES]            ia32.h
    [PROTOTYPES]            kaneton.h
    [PROTOTYPES]            as.h
    [PROTOTYPES]            conf.h
    [PROTOTYPES]            serial.h

    [...]

    $ 
  \end{verbatim}
\end{frame}

% 5)

\begin{frame}[containsverbatim]
  \frametitle{Explanations}

  This system is based on tags in the header files which specify
  from which files to extract prototypes.

  \nl

  The tags are of the form:

  \begin{verbatim}
    /*
     * ---------- prototypes -------------------------------------------------
     *
     *      ../../kaneton/set/set.c
     *      ../../kaneton/set/set_array.c
     *      ../../kaneton/set/set_ll.c
     *      ../../kaneton/set/set_bpt.c
     */
  \end{verbatim}
\end{frame}

% 5)

\begin{frame}[containsverbatim]
  \frametitle{Dependencies}

  The compilation system uses full dependencies between files.

  \nl

  To regenerate the dependencies, for example when adding a
  \textit{\#include} c-preprocessor directive in a source file:

  \begin{verbatim}
    $ make dep
    [REMOVE]                .makefile.mk
    [DEPENDENCIES]          dump.c
    [DEPENDENCIES]          alloc.c
    [DEPENDENCIES]          sum2.c

    [...]

    $ 
  \end{verbatim}
\end{frame}

%
% libs
%

\section{libs}

% 1)

\begin{frame}
  \frametitle{Overview}

  The \textbf{libs} directory contains the libraries used by the kaneton
  project like:

  \begin{itemize}
    \item
      libc
    \item
      crt
    \item
      libposix
    \item
      etc..
  \end{itemize}
\end{frame}

%
% core
%

\section{core}

% 1)

\begin{frame}
  \frametitle{Overview}

  The \textbf{core} directory contains the source code for the microkernel
  including the bootstrap, the bootloader and the kernel itsef.

  \nl

  Each part contains an \textbf{arch} directory used for architecture
  dependent soure code.
\end{frame}

% 2)

\begin{frame}[containsverbatim]
  \frametitle{Tree}

  \begin{verbatim}
    core/
      bootstrap/
        arch/
          ia32/ <---;
          machdep --+
      bootloader/
        arch/
      kaneton/
        arch/
        as/
        conf/
        debug/
        id/
        segment/
        set/
        stats/
  \end{verbatim}
\end{frame}

%
% drivers
%

\section{drivers}

% 1)

\begin{frame}
  \frametitle{Overview}

  The \textbf{drivers} directory contains the drivers of the kaneton
  microkernel.

  \nl

  A driver, in the kaneton terms, is a microkernel server which is allowed
  to communicate with hardware devices.
\end{frame}

% 2)

\begin{frame}[containsverbatim]
  \frametitle{Tree}

  \begin{verbatim}
    drivers/
      cons/
        Makefile
        cons.c
      dma/
      kbd/
      ide/
  \end{verbatim}
\end{frame}

%
% services
%

\section{services}

% 1)

\begin{frame}
  \frametitle{Overview}

  The \textbf{services} directory contains the services of the kaneton
  microkernel.

  \nl

  A service, in the kaneton terms, in simply a server which does not
  communicate with the hardware.
\end{frame}

% 2)

\begin{frame}[containsverbatim]
  \frametitle{Tree}

  \begin{verbatim}
    services/
      dsh/
      mod/
        Makefile
        mod.c
        modfs.c
  \end{verbatim}
\end{frame}

%
% programs
%

\section{programs}

% 1)

\begin{frame}
  \frametitle{Overview}

  The \textbf{programs} directory contains the sources of common
  programs.

  \nl

  A program in the kaneton terms is just a non-privilegied
  process.
\end{frame}

% 2)

\begin{frame}[containsverbatim]
  \frametitle{Tree}

  \begin{verbatim}
    programs/
      ls/
      wc/
      cat/
      mount/
      umount/
      gcc/
      emacs/
  \end{verbatim}
\end{frame}

%
% export
%

\section{export}

% 1)

\begin{frame}
  \frametitle{Overview}

  The \textbf{export} directory is used to create kaneton distribution.

  \nl

  This feature is especially used by the maintainers of the kaneton
  project which create very special kaneton distribution for
  the students.
\end{frame}

% 2)

\begin{frame}[containsverbatim]
  \frametitle{Use}

  The only way to export kaneton is to do like this:

  \begin{verbatim}
    $ make export
    [!] usage: exporter.sh [stage]

    available stages: k0 k1 k2 k3 k4 k5 k6 k7 k8 k9 kaneton dist
    $ make export-k3
  \end{verbatim}

  \begin{itemize}
    \item
      \textbf{k[0-9]}: create a special kaneton version for the k[0-9]
      subproject
    \item
      \textbf{kaneton}: create an entire kaneton version for the lastest
      subproject
    \item
      \textbf{dist}: create an entire backup of the kaneton development
      project
  \end{itemize}
\end{frame}

%
% papers
%

\section{papers}

% 1)

\begin{frame}
  \frametitle{Overview}

  The \textbf{papers} directory contains the papers and lectures
  in relation with the kaneton project.

  \nl

  We prefered set the papers directly into the tarball so every student
  can easily read them.
\end{frame}

% 2)

\begin{frame}[containsverbatim]
  \frametitle{Tree}

  \begin{verbatim}
    papers/
      assignments/
      design/
      kaneton/
      seminar/
      lectures/
        kernels/
        inline-assembly/
        c-preprocessor/
        distributed-operating-systems/
        arch-ia32/
  \end{verbatim}
\end{frame}

% 3)

\begin{frame}[containsverbatim]
  \frametitle{Use}

  \begin{verbatim}
    $ make view
    [+] papers:

    [+]   assignments
    [+]   design
    [+]   arch-ia32
    [+]   c-preprocessor
    [+]   distributed-operating-systems
    [+]   inline-assembly
    [+]   kernels
    [+]   development-environment

    [!] usage: viewer.sh [paper]
    $ make view-design
  \end{verbatim}
\end{frame}

%
% doc
%

\section{doc}

% 1)

\begin{frame}
  \frametitle{Overview}

  The \textbf{doc} directory contains every document useful for
  the development of the kaneton project.

  \nl

  This directory will theorically contain documents on the different
  architectures, documents on some hardware devices like ide, usb etc..
\end{frame}

\end{document}

%
% ---------- header -----------------------------------------------------------
%
% project       kaneton
%
% license       kaneton
%
% file          /home/mycure/kaneton/view/book/development/source-tree.tex
%
% created       julien quintard   [thu may 17 22:41:36 2007]
% updated       julien quintard   [thu may 31 08:34:23 2007]
%

%
% ---------- source tree ------------------------------------------------------
%

\chapter{Source Tree}
\label{chapter:source tree}

In this chapter we will briefly describe the kaneton microkernel project
source tree.

\newpage

%
% ---------- text -------------------------------------------------------------
%

The kaneton microkernel reference source tree looks like the following
listing:

\begin{verbatim}
cheat/
configure/
environment/
export/
history/
kaneton/
library/
license/
test/
tool/
transcript/
view/
\end{verbatim}

%
% cheat/
%

\subsection*{cheat/}

Since the kaneton microkernel is implemented by students, the kaneton
people need to check whether students are cheating by re-using parts of
previous years projects or other kernel source codes available on the
\textit{Internet}.

To avoid cheating, kaneton people developed a software checking for
commonalities between different source codes.

This directory contains scripts that performs these verifications. However,
the students work over the years are not stored in this directory but in
the \textit{history/} directory instead.

%
% configure/
%

\subsection*{configure/}

This directory contains everything necessary for configuring its own
kaneton microkernel development environment through the compiling process
to the boot system.

Any new contributor should first look at this directory. However, note that
this directory mainly contains tools targeting final-users rather than
kaneton contributors. Indeed, for instance, the \textit{configure} utility
aims at providing a user-friendly way for configuration but does not take
advantage of the power of the kaneton development environment.

Contributors should then learn about how the development environment works
while final-users should use the \textit{configure} tool.

%
% environment/
%

\subsection*{environment/}

This directory contains everything necessary to the kaneton development
environment.

The kaneton development environment allows different developers to
interact on the development of the same microkernel in a pretty easy way.

The development environment aims at providing developers to possibility to
work in a collaborative manner without interfering with each other. These
developers are likely to run different operating systems on different
microprocessors. In addition, the kaneton microkernel can be targeted for
different microprocessor architectures. The development environment was
introduced to cope with these combinations by providing profiles, each
profile describing the behaviour of a component: underlying operating system,
target architecture, user-specific stuff etc.

As a result, each developer can use a different operating system and
microprocessor architecture with its own specific compiling flags, kaneton
parameters etc. without modifying another developer's configuration.

The development environment is detailed in \textit{Section
\ref{section:environment}}.

%
% export/
%

\subsection*{export/}

The \textit{export/} directory contains scripts used to generate a kaneton
tarball in order to be distributed to the students at the beginning of the
kaneton educational project.

Indeed, these scripts rearrange the kaneton hierarchy hidding some important
directories the students do not need to be aware of. Moreover some source
code parts are removed since the students have to rewrite these pieces
of code as assignments.

These scripts are also used for making backups and distribution tarbalss of
the kaneton microkernel.

%
% history/
%

\subsection*{history/}

The \textit{history/} directory contains the students work over the years
in the universities and schools the kaneton project was used as an operating
system course's implementation material.

The tools of the \textit{cheat/} directory use these students works for
performing cheating verifications.

%
% kaneton/
%

\subsection*{kaneton/}

This directory is the most important of the project since it contains
the whole microkernel source code.

The directory is composed of three important subdirectories: \textit{core/},
\textit{platform/} and \textit{architecture/}. These subdirectories are
described next.

% core/

\subsubsection*{core/}

This directory contains the kaneton core source code.

The directory is divided as shown below:

\begin{verbatim}
as/
region/
sched/
segment/
set/
task/
thread/
[...]
\end{verbatim}

Each directory represents a kaneton core manager. For more information on
the kaneton core, please refer to the appropriate document:
\textit{The kaneton microkernel :: core}

% platform/

\subsubsection*{platform/}

This directory contains everything in relation with what the kaneton
microkernel project calls a \textit{platform}. The platform represents the
board supporting the devices: microprocessor, memory, peripherals etc.

This directory obviously contains subdirectories for each platform
supported by the kaneton microkernel.

% architecture/

\subsubsection*{architecture/}

The \textit{architecture/} directory contains the source-code related to
the microprocessor architectures supported by the kaneton microkernel.

This directory is composed of subdirectories, each one representing a
supported architecture: \textit{ia32}, \textit{mips64} etc. Note that each
architecture can be specialised. For instance, the \textit{ia32/optimised}
architecture represents an optimised implementation of the \textit{Intel IA-32}
microprocessor architecture.

%
% library/
%

\subsection*{library/}

This directory contains the libraries used by the kaneton microkernel itself,
the kaneton microkernel servers or maybe both. This directory especially
contains the standard \textit{kaneton C library}.

%
% license/
%

\subsection*{license/}

This directory contains the licenses used for any program or document
in relation with the kaneton microkernel project. Indeed, the kaneton
microkernel is under the \textit{kaneton license} which is described in
depth in the documents contained in this directory. Note that these licenses
are also available in \textit{Chapter \ref{chapter:licenses}}.

Each student has to read and agree with the kaneton license before
implementing or even using the kaneton microkernel project..

Indeed, every user of the kaneton-related stuff is considered as having
implicitly accepted the kaneton license.

%
% test/
%

\subsection*{test/}

Since the kaneton microkernel is used as a material for operating system
courses, the kaneton microkernel reference, which is the basis of students
work, must be extremely reliable.

The kaneton project therefore contains a set of tools in order to validate
the kaneton reference implementation behaviour. These tools are also used
for evaluating the correctness of the students implementation.

The \textit{test/} directory contains the set of kaneton scripts and tests
for validating a kaneton microkernel implementation.

%
% tool/
%

\subsection*{tool/}

This directory contains additional scripts and configuration files used by
the kaneton development environment or the kaneton developers.

As examples, this directory contains scripts for generating prototypes,
building a boot device etc.

%
% transcript/
%

\subsection*{transcript/}

This directory contains real-time recorded sessions. These sessions can be
replayed in order to present a feature of the development environment or
of the kaneton microkernel.

%
% view/
%

\subsection*{view/}

This directory contains all the kaneton documents including kaneton
administrative documents, examinations, lectures materials, kaneton papers
and books etc.

Additionally, scripts are provided in order to very easily build and
display these documents.
%%
%% licence       kaneton licence
%%
%% project       kaneton
%%
%% file          /home/mycure/kaneton/view/papers/kaneton/coding-style.tex
%%
%% created       matthieu bucchianeri   [mon jan 30 17:32:57 2006]
%% updated       julien quintard   [thu mar  2 13:57:17 2006]
%%

%
% coding style
%

\chapter{Coding style}

The kaneton project developers try to follow a coding style. This
coding style was introduced to normalize the source code, leading to a
more readable source code.

Nevertheless, you can adapt this coding style to your own but try to
follow the rules.

%
% case
%

\section{Case}

The whole kaneton source code is written using lower case letters.

Moreover, every text including comments etc.. must be written using
lower case letters

%
% headers
%

\section{Headers}

Each file must start with an header formatted as shown below:

\begin{verbatim}
/*
 * licence       kaneton licence
 *
 * project       kaneton
 *
 * file          /home/mycure/kaneton/core/kaneton/as/as.c
 *
 * created       julien quintard   [fri feb 11 02:23:41 2005]
 * updated       matthieu bucchianeri   [mon jan 30 20:30:57 2006]
 */
\end{verbatim}

An emacs configuration file for automatically generating and updating
this header can be found in \textit{tools/emacs}.

Additionally, you need to set two environment variables to generate
a correct kaneton header:

\begin{itemize}
  \item
    \textbf{EC\_LICENCE} must be set to ``kaneton licence''.
  \item
    \textbf{EC\_DEVELOPER} must be set to your first name and last name.
\end{itemize}

Please, do not use nicknames in headers.

%
% naming convention
%

\section{Naming Convenions}

To keep the code as clear as possible, there are several conventions on
types, functions and variables naming.

%
% variables
%

\subsection{Variables}

Here are a few rules you are encouraged to follow:

\begin{itemize}
  \item
    \textbf{sz} suffix for variables representing a size.

    \begin{verbatim}
      #define PAGESZ          4096

      int                     modsz;
    \end{verbatim}
  \item
    \textbf{n} prefix for variables representing a number of objects.

    \begin{verbatim}
      int                     nclusters;
    \end{verbatim}
  \item
    etc..
\end{itemize}

Moreover, the types are used as pre-names:

\begin{verbatim}
t_vaddr                 video_vaddr;
\end{verbatim}

This example is not correct, instead prefer:

\begin{verbatim}
t_vaddr                 video;
\end{verbatim}

%
% functions
%

\subsection{Functions}

Function names must be prefixed by the file name, context name they are
implemented in.

For example, a function part of the address space manager must be prefixed
by \textit{as\_}.

These names must be chosen carefully: they must explicitely define
what the function does without being too long.

%
% types
%

\subsection{Types}

As variables and functions, type names must be expressed in english
with lower case letters.

Here are the prefixes you must use when writing your own types:

\begin{itemize}
  \item
    \textbf{m\_} for managers main structures.
  \item
    \textbf{o\_} for kaneton objects.
  \item
    \textbf{i\_} for interfaces.
  \item
    \textbf{d\_} for architecture-dependent structures.
  \item
    \textbf{s\_} for general purpose structures.
  \item
    \textbf{t\_} for basic and general purpose typedefs.
  \item
    \textbf{c\_} for kaneton capabilities.
  \item
    \textbf{u\_} for kaneton unique identifiers.
\end{itemize}

Notice that \textbf{d\_} can be combined with other prefixes, for
example \textbf{do\_} for a dependent object.

%
% includes
%

\section{Includes}

To keep the code clear and compact, developers only need to include a
minimal number of header files:

\begin{itemize}
  \item
    \textbf{kaneton.h} for the microkernel declarations.
  \item
    \textbf{klibc.h} for the kaneton specific C library.
\end{itemize}

These files are located in the include path, so do not use relative include
path.

\begin{verbatim}
#include <libc.h>
#include <kaneton.h>

int             main(int                argc,
                     char**             argv)
{
  [...]

  return (0);
}
\end{verbatim}

All include files must be protected against multiple inclusions. The
guard macro to use must be named using the directory name, one underscore,
the file name, one underscore and a capital ``H''.

For example, the file \textit{core/include/kaneton/segment.h} will be
guarded as follow:

\begin{verbatim}
#ifndef KANETON_SEGMENT_H
#define KANETON_SEGMENT_H	1

[...]

#endif
\end{verbatim}

In addition, for architecture-dependent files, the guard macro must begin
with the architecture name; for example for the Intel architecture:
\textit{IA32\_KANETON\_SEGMENT\_H}.

%
% types
%

\section{Types}

You may use as soon as possible standard types: \textbf{t\_uint8},
\textbf{t\_sint32}, \textbf{t\_uint64} etc..

This nomenclature is more understandable than
\textbf{unsigned long long int}.

%
% return values
%

\section{Return Values}

Every function must report whether it successed or failed.

In kaneton, functions' return type must be \textbf{t\_error}.

A function will return \textbf{ERROR\_NONE} on success and anything
else on error, for example \textbf{ERROR\_UNKNOWN} to indicate a non-specific
error.

%
% indentation
%

\section{Indentation}

There are several indentation rules in kaneton.

\begin{enumerate}
  \item
    Field names of structures and unions must be aligned with the
    structure or union name.

    \begin{verbatim}
      struct       s_set
      {
        u_set      id;
        t_setsz    size;
        t_type     type;
      };
    \end{verbatim}

    or

    \begin{verbatim}
      typedef struct
      {
        o_id       id;
        u_stats    stats;
        u_set      container;
      }            m_as;
    \end{verbatim}
  \item
    Macros and variables must be aligned as shown below:

    \begin{verbatim}
      #define TASK_PRIOR_CORE     230
      #define TASK_HPRIOR_CORE    250
      #define TASK_LPRIOR_CORE    210

      m_task*                     task;
      u_task                      ktask = ID_UNUSED;
    \end{verbatim}

    This rule also applies for variables declarations in functions.
  \item
    Function prototypes and bodies should look like this:

    \begin{verbatim}
      t_error             stats_function(u_stats          id,
                                         char*            function,
                                         t_stats_func**   f)
      {
        t_sint64          slot = -1;
        t_sint64          i;

        [...]
      }
    \end{verbatim}

    Notice that argument names are aligned between each other,
    and variable names are aligned with function name and between
    each other.

    Try to respect this alignment between functions in a single file:
    function names may be all aligned and argument names also.
\end{enumerate}

%
% comments
%

\section{Comments}

As kaneton is intended to be a pedagogical project with clear and
understandable source code; no need to say that comments take a very
important part of this objective.

Every file must begin with a comment describing what is done in this
code via the \textit{information} section.

Moreover, every function must be preceded by a comment defining its
behavior.

For complex functions and yo prevent direct comments in the source code,
we used \textbf{steps}:

\begin{itemize}
  \item
    Each critical code section in a function is preceded by a step
    number.
  \item
    The function header comment contains steps descriptions.
\end{itemize}

An example is present below:

\begin{verbatim}
/*
 * this function shows the usage of comments and steps.
 *
 * steps:
 *
 * 1) compute the index.
 * 2) make the operation.
 * 3) check the result.
 */

t_error         test_foobar(int      a,
                            int      b,
                            int*     c)
{
  int           index;

  /*
   * 1)
   */

  index = text_make_index(a, b);

  /*
   * 2)
   */

  index = index * a + b;

  /*
   * 3)
   */

  if (index < 0)
    return (ERROR_UNKNOWN);

  *c = index;

  return (ERROR_NONE);
}
\end{verbatim}

%
% sections
%

\section{Sections}

kaneton files are divided in multiple sections.

Section are delimited as shown below:

\begin{verbatim}
/*
 * ---------- includes ------------------------------------------------
 */
\end{verbatim}

Possible sections in a file are:

\begin{itemize}
  \item
    \textbf{header files}: information, dependencies, defines, types,
    prototypes, macros, etc..
  \item
    \textbf{source files}: information, extern, globals, includes,
    functions, etc..
  \item
    \textbf{make files}: dependencies, directives, variables, rules, etc..
\end{itemize}

Moreover, every important file, for example the main file of each
kaneton manager, have to contain a section \textit{information} describing
the whole manager.

In addition, a section named \textit{assignments} is generally necessary
for manager will be filled in by the students. This section briefly describes
the work to be done by the students.

%
% macros
%

\section{Macros}

The kaneton microkernel uses few fundamental macros lited below:

\begin{itemize}
  \item
    \textbf{\_\_\_bootloader} indicates that this source code belongs to
    the bootloader.
  \item
    \textbf{\_\_\_kernel} indicates that this source code belongs to the
    microkernel.
  \item
    \textbf{\_\_\_kaneton} indicates that this kernel is the kaneton
    microkernel.
  \item
    \textbf{\_\_\_wordsz} indicates the word size: 16-bit, 32-bit,
    64-bit etc..
  \item
    \textbf{\_\_\_endian} indicates the endianness.
\end{itemize}

%%
%% licence       kaneton licence
%%
%% project       kaneton
%%
%% file          /home/mycure/kaneton/view/papers/kaneton/core.tex
%%
%% created       matthieu bucchianeri   [mon jan 30 17:33:29 2006]
%% updated       julien quintard   [fri mar 10 01:50:49 2006]
%%

%
% core
%

\chapter{Core}

\newpage

%
% text
%

%
% bootstrap
%

\section{Bootstrap}

XXX

%
% bootloader
%

\section{Bootloader}

XXX

%
% kaneton
%

\section{kaneton}

XXX

%
% id
%

\section{id}

The following rules describes a typical use of id objects in kaneton:

\begin{itemize}
  \item
    The \textbf{init} function of a manager calls \textbf{id\_build}
    to generate a \textbf{o\_id} object.
  \item
    Functions generating new objects will use \textbf{id\_reserve} to
    reserve new identifiers for the created objects.
  \item
    Functions removing objets will call \textbf{id\_destroy} to release
    identifiers of destroyed objets.
  \item
    The \textbf{clean} function of a manager will release the identifier
    generator with \textbf{id\_release}.
\end{itemize}

%
% ---------- header -----------------------------------------------------------
%
% project       kaneton
%
% license       kaneton
%
% file          /home/mycure/kaneton/view/book/development/tools.tex
%
% created       julien quintard   [sun may 20 14:48:11 2007]
% updated       julien quintard   [thu may 31 06:46:06 2007]
%

%
% ---------- tools ------------------------------------------------------------
%

\chapter{Tools}
\label{chapter:tools}

This chapter describes every tool kaneton contributors use on a daily-basis.

\newpage

%
% ---------- text -------------------------------------------------------------
%

%
% internal
%

\section{Internal}

The kaneton project contains several tools which makes the developer's life
easier. This section describes these tools in order for the contributor to
use it but also to improve them.

% environment

\input{environment.tex}

% configure

\input{configure.tex}

% view

\input{view.tex}

% export

\input{export.tex}

% transcript

\input{transcript.tex}

% cheat

\input{cheat.tex}

% test

\input{test.tex}

% control panel

\input{control-panel.tex}

%
% external
%

\section{External}

The kaneton contributors use several other tools for the communication, the
development etc.. These tools are described in the following sections.

% mailing-list

\input{mailing-list.tex}

% repository

\input{repository.tex}

% wiki

\input{wiki.tex}

% project management

\input{project-management.tex}

%
% ---------- header -----------------------------------------------------------
%
% project       kaneton
%
% license       kaneton
%
% file          /home/mycure/kaneton/view/internship/check/check.tex
%
% created       julien quintard   [wed may 16 18:06:23 2007]
% updated       julien quintard   [thu may 22 16:30:20 2008]
%

%
% ---------- setup ------------------------------------------------------------
%

%
% path
%

\def\path{../..}

%
% template
%

\input{\path/template/internship.tex}

%
% header
%

\lhead{\scriptsize{The kaneton microkernel :: check}}

%
% title
%

\title{The kaneton microkernel :: check}

%
% authors
%

\author{\small{Solal Jacob}}

%
% document
%

\begin{document}

%
% title
%

\maketitle

%
% ---------- abstract ---------------------------------------------------------
%

\begin{abstract}

\indent The test framework ains at mesuring the reliability of the kaneton
microkernel. It is organized around two parts. On the one hand, tests are
written and executed within kaneton, on the other hand, a Python script runs on
another independant system. This document concerns users who want to build a
new tests set and also contributors who would like to upgrade the test
environment itself.

\end{abstract}

%
% --------- text --------------------------------------------------------------
%

\section{Framework architecture}

\indent As the test environment is supposed to check kaneton's reliability, it
cannot assume that kaneton is safe and therefore it cannot be installed on the
tested system. Thus, the test environment has been designed to check kaneton
from another independant system. Both of kaneton and the independant system can
communicate via the serial port.\\
\\
\indent Once the tests are written (kaneton-side), the Makefile needs to be
modified to turn the kernel into debug mode. This mode enables the tests,
executes them and sends their results through the serial port to the script.
When debug mode is enabled, kaneton first initializes the serial driver (default
settings are COM1 at 56kb/s) and then waits for commands from the serial
connection.\\
\\
The test script is written in Python and depends on a C/Python module whose
role is to manage serial communication. It is based on a dedicated protocol
also used in kaneton. All was done to simplify the script development.\\
\\
\indent Strong conventions have been established to write test scripts and to
manage the kaneton Makefile. They must be known and respected to add new tests.
These conventions are deeply detailed in the next paragraphs.


\section{How to add and to launch tests}

\subsection{How to add tests}
As already said, test writting must obey certain rules. Conventions on
directory and file naming were chosen as follow:
\begin{enumerate}
\item A test dedicated to a specific part of kaneton must be created in the same
directory tree than the piece of kaneton it is supposed to check. This new
directory tree must be placed in kaneton/check/.
\item A test is a directory named 01, 02, \ldots, N and containing the following
files:
\begin{itemize}
\item .c: the test itself which is part of kaneton
\item .res: the expected output for the corresponding .c
\item parse\_res.py: a python function which must be modify to parse the
result.
\end{itemize}
\item A file named list and containing all the names of the tests separated by a
single \textbackslash n must be placed in every level of the check directory
tree.
\\
\end{enumerate}
{\bf Example:}\\
Assuming we want to write two tests checking the printf function
which is located in kaneton/klibc/libstring/, we need to create the following
 directory tree:

\begin{verbatim}
    kaneton/check/
                 libs/
                     klibc/
                          listring/
                                  printf/
                                        01/
                                           01.c
                                           01.res
                                           pares\_res.py
                                        02/
                                           02.c
                                           02.res
                                           pares\_res.py
                                        list		 => 01\n02\n
                                  list			 => printf\n
                          list				 => libstring\n
                     list				 => klibc\n
                 list					 => libs\n

\end{verbatim}


\subsection{How to launch tests}
\begin{itemize}
\item Install the Python module by running ./domodules in kaneton/tools/python/
\item Select the tests to enable by adding their name in the appropriate `list`
text file (see Directory Convention for further details).
\item Configure kaneton to run in debug mode:
\begin{description}
\item environment/users/user.conf: delete the line: {\tt override \_CHECK\_}
\item environment/users/conf.h: add the line:  {\tt \#define SERIAL}
\end{description}
\item Build the kernel
\item During kernel execution, run ./check.py to get the tests results
\end{itemize}


\section{Serial driver and the test framework on kaneton}

The serial driver can be found in kaneton/core/kaneton/debug/serial.c.\\
Initialization is performed by passing the desired com\_port and baud\_rate to
the serial\_init function. Basic values as SERIAL\_8N1 and SERIAL\_FIFO\_8 are
defined for recurrent usages; you will find them in
kaneton/core/include/kaneton/serial.h.\\
Once the driver is setup, the functions serial\_read and serial\_write permit
to transfert data trough the serial port (see serial.h for further
information). Keep in mind that this serial driver was written in pole mode. So
it requires a 1GHz processor to achieve high-speed (56kbp/s) communication
without trouble.

The test framework routines can be found in
kaneton/core/kaneton/debug/debug.c.\\
A call to the debug\_init function performs all initializations needed by the
test framework. It calls serial\_init, printf\_init, allocates sufficient memory
and waits for new messages in a never ending loop. printf\_init permits to
redirect printf output towards the serial port using serial\_put function as a
parameter. debug\_recv reads in input until it receives "command". Then it
waits for the address of the command to execute, executes it and uses
serial\_put to send the command result through the serial port.\\


\section{Python script and C/Python modules}

check.py is the script which analyzes the tests results. It  can be found in
kaneton/check/. The C/Python kserialmodule in kaneton/tools/python/ must be
installed prior to launch ./check.py. Just run ./domodules to do so. This
module permits Python applications to use serial\_init, serial\_recv and
serial\_send functions, and thus to communicate with kaneton. The line\\
\\
\indent{\tt from kserial import *}\\
\\
in the main function of check.py shows how to use the kserial module.\\
\\
\indent The script check.py first initializes the serial communication by a
call to serial\_init("/dev/ttyS0"), what implies that we run check.py on Linux
on COM1. Then the script uses the function ListTest to recursively find the tests
it will have to execute. All the tests found are added to a list. This list is
parsed and test functions are sequentially sent to kaneton for execution. Their
result is compared to their corresponding .res by their associated
parse\_res.py script. Every failure or success is stored in order to compute
and display totals.

\end{document}

%%
%% licence       kaneton licence
%%
%% project       kaneton
%%
%% file          /home/mycure/kaneton/view/papers/kaneton/glossary.tex
%%
%% created       matthieu bucchianeri   [mon jan 30 17:34:32 2006]
%% updated       julien quintard   [thu mar  2 13:07:22 2006]
%%

%
% glossary
%

\chapter{Glossary}

\subsubsection{as}

An address space is an entity representing addressable memory,
physical and virtual, associated to a task. In kaneton, an address space
is composed of a set of segments and a set of regions.


%
% epilogue
%

XXX ?

\end{document}


%
% class
%

\documentclass[10pt,a4wide]{article}

%
% packages
%

\usepackage[english]{babel}
\usepackage[T1]{fontenc}
\usepackage{a4wide}
\usepackage{graphicx}
\usepackage{fancyheadings}
\usepackage{multicol}
\usepackage{indentfirst}
\usepackage{color}
\usepackage{ifthen}
\usepackage{comment}
\usepackage{verbatim}
\usepackage{aeguill}

\pagestyle{fancy}

\setlength{\footrulewidth}{0.3pt}
\setlength{\parindent}{0.3cm}
\setlength{\parskip}{2ex plus 0.5ex minus 0.2ex}

%
% correction environment
%

\newenvironment{correction}%
   {
     \ifthenelse
	 {
	   \equal{\kaneton-latex}{subject}
	 }
	 {
	   \comment
	 }
	 {
	   \textbf{\color{red}{ ----- correction}}
	 }
   }%
   {
     \ifthenelse
	 {
	   \equal{\kaneton-latex}{subject}
	 }
	 {
	   \endcomment
	 }
	 {
	   \textbf{\color{red}{ ----- /correction}}
	 }
   }

%
% colors
%

\definecolor{functioncolor}{rgb}{0.40,0.00,0.00}

%
% function
%

\newcommand\function[3]{
  \begin{tabular}{p{0.2cm}p{13.8cm}}
  & {\color{functioncolor}\textbf{#1}}#2
  \end{tabular}

  \begin{tabular}{p{1cm}p{13cm}}
  & #3
  \end{tabular}}

%
% header
%

\rhead{}
\rfoot{\scriptsize{Exam}}

\date{\scriptsize{\today}}


%
% header
%

\lhead{\scriptsize{Noyaux et Syst\`emes d'exploitation}}
\rhead{\scriptsize{2007}}

%
% title
%

\title{{\huge {\bf Noyaux et Syst�mes d'exploitation}}}

%
% authors
%

\author{{Matthieu Bucchianeri \& Renaud Voltz}}

%
% document
%

\begin{document}

%
% title
%

\maketitle

%
% identation
%

\indentation{}

%
% --------- information -------------------------------------------------------
%

\begin{center}
\vspace{1cm}

{\huge {\bf Partiel 2}}

\vspace{2cm}

{\large
{\bf Ordinateurs, PDA et t�l�phones non autoris�s}

{\bf Documents autoris�s}

\vspace{1cm}

{\bf Dur�e : 2 heures}
}
\end{center}

\vspace{2cm}

{\large {\bf Important :}}\\
\begin{itemize}
\item Le bar�me est donn� � titre indicatif, en cas de modifications,
      aucune contestation ne sera possible.

\item Un point de malus sur la note globale sanctionnera les copies peu
      soign\'ees ou mal orthographi\'ees.

\item Vos r\'eponses seront concises : {\bf courtes, claires et pr\'ecises}.

\item Lorsqu'il vous est demand\'e d'\'ecrire du code (exercices 4 et 5),
     vous pourrez n\'egliger la gestion des erreurs uniquement lorsqu'elle
     n'apporte rien au probl\`eme.
\end{itemize}


%
% --------- text --------------------------------------------------------------
%

\vspace{3cm}

\newpage
\part*{Exercices}

%
% ---------- header -----------------------------------------------------------
%
% project       kaneton
%
% license       kaneton
%
% file          /home/mycure/kane...ecture/kernels/interrupts/interrupts.tex
%
% created       julien quintard   [wed may 16 19:17:31 2007]
% updated       julien quintard   [wed may 16 19:17:32 2007]
%

\pgfdeclareimage[interpolate=true,width=180pt]
                {stack}
                {figures/stack}

\pgfdeclareimage[interpolate=true,width=140pt]
                {rings}
                {figures/rings}

\pgfdeclareimage[interpolate=true,width=190pt]
                {isr_kernel}
                {figures/isr_kernel}

\pgfdeclareimage[interpolate=true,width=190pt]
                {isr_user}
                {figures/isr_user}

\pgfdeclareimage[interpolate=true,width=300pt]
                {event}
                {figures/event}

\pgfdeclareimage[interpolate=true,width=230pt]
                {ivt}
                {figures/ivt}

\pgfdeclareimage[interpolate=true,width=270pt]
                {pic}
                {figures/pic}

\pgfdeclareimage[interpolate=true,width=220pt]
                {m68kivt}
                {figures/m68kivt}

\pgfdeclareimage[interpolate=true,width=310pt]
                {nested}
                {figures/nested}

\section{The execution stack}

%
%
%

\begin{frame}
  \frametitle{The execution stack}

  The stack is a LIFO structure, mainly used to:
  \begin{itemize}
    \item save the instruction to return to after a subroutine call
    \item pass parameters to a subroutine
    \item allocate space for subroutines local variables
  \end{itemize}

  \-

  On more specific occasions, the stack can also be used to backup an
  environment state.

\end{frame}

%
%
%

\begin{frame}
\frametitle{The stack's structure}

  The stack is composed of frames: everytime a subroutine is called, a new
  frame is pushed. Also the frame for the executing subroutine is always on
  the top of the stack. When the subroutine returns, this frame is poped.

  \-

  Thus, the current stack state is described by the contents of 2 registers:
  \begin{enumerate}
    \item the {\em stack pointer} contains the address of the top of the stack
    \item the {\em base pointer} contains the base address of the current stack
      frame
  \end{enumerate}

  \begin{center}
    \pgfuseimage{stack}
  \end{center}

\end{frame}

%
%
%

\begin{frame}
  \frametitle{Execution stack \& multithreading}

  Tasks cannot share their execution stack:

  \begin{itemize}
    \item {\bf For safety reasons}\\
      The execution stack contains private data for the running task.
      Corrupting those data results in corrupting the whole task's behavior.

      \-

    \item {\bf For practical reasons}\\
      Let us consider the following scenario:

      \begin{enumerate}
        \item {\em task1} pushes {\em data1}
	\item scheduler transfers execution from {\em task1} to {\em task2}
	\item {\em task2} pushes {\em data2}
	\item scheduler transfers execution from {\em task2} to {\em task1}
	\item {\em task1} pops {\em data2}
      \end{enumerate}

      \-

      At step 5, {\em task1} expects to pop data1 it pushed at step 1. But if
      {\em task1} and {\em task2} both share the same execution stack,
      {\em task1} will actually pop {\em data2} that {\em task2} pushed at step
      3. This results in a double issue:

      \begin{itemize}
	\item {\em task1} will continue its execution with a corrupted value.
	\item {\em task1} should not be able to access {\em task2}'s private data.
      \end{itemize}

  \end{itemize}

\end{frame}

\section{Privilege levels}

%
%
%

\begin{frame}
  \frametitle{Privilege rings}

  \begin{center}
    \pgfuseimage{rings}
  \end{center}

  \-

  {\bf Fundamental rule:}\\
  An application cannot increase its own privileges.

\end{frame}

%
%
%

\begin{frame}
  \frametitle{Privilege levels}

  \begin{enumerate}
  \item {\bf Kernelland}\\
    This privilege level is often called {\em supervisor mode} by microprocessor
    fabricants. In modern operating systems, this ring is dedicated to the kernel.

    \-

  \item {\bf Userland}\\
    Also called {\em user mode}, this level is opposed to Kernelland since it
    imposes privilege restrictions. Userland is dedicated to user programs
    which should not modify the system or other programs behavior. It is
    characterized by the following restrictions:

    \begin{itemize}
      \item restricted instruction set
      \item restricted memory access
      \item restricted I/O access
    \end{itemize}
  \end{enumerate}

  \-

  Other intermediate levels can be attributed to special applications such as
  drivers or trusted operating system services.

  \-

  In an operating system which uses more than 2 privilege levels, Userland is
  considered as the less privileged one.

\end{frame}

\section{Events}

%
%
%

\begin{frame}
  \frametitle{Overview}

  Interrupts were introduced to replace polling which is based on active
  loops. Active loops are very performant when they handle lots of events, but
  waste the CPU time when they are waiting for a condition that does not
  happen.

  \-

  Interrupts make it possible to make a process sleep until an event wakes it
  up. Thus microprocessor cycles are saved to serve other processes. On certain
  embedded architectures, interrupts are also a easy way to save energy.

  \-

  Interrupts are a critical part of code in the kernels implementation:

  \begin{itemize}
    \item The code is non-portable accros compilers and platforms
    \item the code is difficult to debug
    \item kernel performances highly depends on interrupt handling implementation
  \end{itemize}

\end{frame}


%
%
%

\begin{frame}
  \frametitle{Definition of an event}

  {\bf Definition:}\\
  Events indicate to the processor that something happened in the system
  and requires kernel attention. They typically result in a forced transfer of
  execution from the currently running task to a special kernel routine  called
  an {\em interrupt service routine} (ISR) or an {\em interrupt handler}.\\

  \-

  Events are categorized in 3 classes:

  \begin{itemize}
    \item {\bf Exceptions}\\
      Raised by the processor itself when detecting an internal error.\\
      Example: {\em divide by zero}, {\em page fault}, {\em invalide opcode}\ldots
    \item {\bf Software interrupts (or trap)}\\
      Caused by the running program when executing a special ASM instruction.
    \item {\bf Hardware interrupts (or IRQ)}\\
      Provoked by external hardware devices.
  \end{itemize}

  \-

  From the microprocessor point of view, exceptions and software interrupts are
  seen as internal interrupts, whereas IRQ are treated as external interrupts.

\end{frame}

%
%
%

\begin{frame}
  \frametitle{The Interrupt Service Routine (ISR)}

  The {\em Interrupt Service Routine} (ISR) is a kernel procedure which is
  called to handle an event. The kernel must provide a specific ISR for each
  event which may occur.

   \-

   Each ISR is characterized by:

  \begin{enumerate}
    \item the interrupt vector to which it is associated
    \item the privilege level which is required to invoke this ISR.
  \end{enumerate}

  \-


  ISR are kernel routines (running with kernel's privileges) that can be
  executed on Userland programs needs, when they raise an interrupt.

  \-

  Some ISR are famous:

  \begin{itemize}
    \item Exceptions ISR: page fault handler
    \item Software Interrupts ISR: system calls
    \item IRQ ISR: drivers I/O routines
  \end{itemize}

\end{frame}

%
%
%

\begin{frame}
  \frametitle{Events \& privilege level switching}

  \begin{enumerate}
    \item ISR interrupts a Kernelland routine:
      \begin{center}
        \pgfuseimage{isr_kernel}
      \end{center}

    \item ISR interrupts an Userland routine:
      \begin{center}
        \pgfuseimage{isr_user}
      \end{center}
  \end{enumerate}

\end{frame}

%
%
%

\begin{frame}
  \frametitle{Events \& privilege level switching}

  An event interrupts the currently running program and invokes the ISR
  execution. As an ISR is always executed in Kernelland, two cases may happen:

  \begin{enumerate}
    \item {\bf ISR interrupts a Kernelland routine}
    The processor is already in {\em superisor mode} when the execution is
    transfered to the ISR.

    \item {\bf ISR interrupts an Userland routine}
    The processor is operating in {\em user mode} when it detects the event.
    In such a situation, the microprocessor behaves like this:

    \begin{itemize}
      \item switch from {\em user mode} to {\em supervisor mode}
      \item execute the ISR in {\em supervisor mode}
      \item switch from {\em supervisor mode} to {\em user mode}
    \end{itemize}

  \end{enumerate}

  \-

  {\bf Questions for both cases:}

  \begin{itemize}
    \item How does the processor find the ISR for a given interrupt ?
    \item How does the processor resume the interrupted routine ?
  \end{itemize}

\end{frame}

%
%
%

\begin{frame}
  \frametitle{The interrupt vector}

  Every event in the system is given its own identifier called an
  {\bf interrupt vector}. The {\em interrupt vector} specifies the source of
  the interrupt, and therefore, it can be used to determine which ISR should
  be executed.

  \-

  In the case of an exception or a software interrupt, the microprocessor
  deduces the {\em interrupt vector} internally.

  \-

  Concerning the IRQ,
  Some embedded microprocessors provide a set of interrupt lines, to which
  hardware devices are directly wired. Such microprocessors associate an
  {\em interrupt vector} for each IRQ line.

  \-

  Other microprocessors do not provide any unit to handle multiple IRQ lines.
  Those microprocessors communicate with an external chip called an
  {\bf Interrupt Controller}.


\end{frame}

%
%
%

\begin{frame}
  \frametitle{The interrupt controller}

  The Interrupt Controller multiplexes the hardware devices interrupts requests
  towards the single microprocessor IRQ line. The {\em interrupt vector} is
  transfered via the data bus:

  \-

  \begin{center}
    \pgfuseimage{pic}
  \end{center}

\end{frame}

%
%
%

\begin{frame}
  \frametitle{Finding the ISR}

  An {\em interrupt vector} is loaded in a microprocessor internal register
  everytime an event is raised. Depending on wether this register is visible
  by the software, the kernel handles events 2 different manners:

  \-

  \begin{enumerate}
    \item If the {\em interrupt vector register} is not visible\\
      The kernel has no way to determine the source of the event. Also, the
      kernel is unable to chose the corresponding ISR. Therefore, the ISR
      dispatch must be assumed by the microprocessor itself.

      \-

    \item If the {\em interrupt vector register} is visible\\
      The microprocessor loads its PC with the base address of a generic ISR.
      Then, the kernel can load the {\em interrupt vector} from the
      microprocessor register, and easily select the ISR to handle the event.
      The dispatch is made by the kernel.
  \end{enumerate}

\end{frame}


%
%
%

\begin{frame}
  \frametitle{Interrupt Vector Table}

  On architectures such as IA-32, event identification is done internally. That
  means that the microprocessor does not provide a way to tell the kernel which
  {\em interrupt vector} has been detected. Instead, the microprocessor
  determines itself which ISR must be executed:

  \-

  \begin{enumerate}
    \item the microprocessor gets the {\em interrupt vector}
    \item the microprocessor determines the corresponding ISR
    \item the microprocessor loads the ISR's first instruction into the
      {\em Programm Counter}
    \item the ISR handles the event
  \end{enumerate}

  \-

  That implies that the microprocessor can find the addresse of each ISR. To do
  so, the microprocessor consults an {\bf interrupt vector table}, whose base
  address must be stored in a dedicated register.

\end{frame}

%
%
%

\begin{frame}
  \frametitle{Interrupt Vector Table}

  \begin{center}
    \pgfuseimage{ivt}
  \end{center}

\end{frame}

%
%
%

\begin{frame}
  \frametitle{the Cause register}

  On other architectures, like MIPS, the microprocessor delegates the event
  handling to the kernel:

  \-

  \begin{itemize}
    \item the microprocessor stores the {\em interrupt vector} in the
    {\em Cause} register
    \item the microprocessor loads the {\em Program Counter} with the
      address of the generic handler's first instruction
    \item the generic handler retrieves the {\em interrupt vector} from
      the {\em Cause} register
    \item the generic handler execute the appropriated ISR
  \end{itemize}

  \-

  This implies that the {\em Cause} register is visible from the kernel.
\end{frame}


%
%
%

\begin{frame}
  \frametitle{}

  SCHEMA SANS IVT (MIPS)

\end{frame}

%
%
%

\begin{frame}
  \frametitle{Returning from an interrupt}

  resuming the interrupted program (or not)


\end{frame}

%
%
%

\begin{frame}
  \frametitle{The execution context}

  The {\bf execution context} is the state of the program execution at a given
  time.

  \-

  Concretly, an image of the {\em execution context} can be obtained by saving
  the state of:

  \begin{itemize}
    \item the stack
    \item the memory mapping
    \item microprocessor registers
  \end{itemize}

  \-

  Loading an {\em execution context} from an image, back to the microprocessor
  registers, makes the program resume from where the context image was first
  saved.

  \-

  Thus it becomes possible to interrupt a program, backup its {\em execution
  context}, execute an ISR and resume the interrupted program in a transparent
  manner by simply restoring its last {\em execution context}.

\end{frame}

%
%
%

\begin{frame}
  \frametitle{Backing up the context}

  {\bf Q:} Where is saved the context ?\\
  {\bf A:} The easiest way to backup the context is to push it on the stack.

  \-

  {\bf Q;} Why not in memory ?\\
  {\bf A:} Allocation times would be too long. Allocation algorithms may
  corrupt the context before it was actually saved !

  \-

  {\bf Q:} Why on the stack ?\\
  {\bf A:} In the case of nested interrupts, a LIFO structure is naturally
  appropriate.

  \-

  {\bf Q:} Whose stack ?\\
  {\bf A:} The {\em execution context} should be saved and restored by whoever is
  interrupting the program. So on the ISR's stack.

  \-

  {\bf Q:} When should the context be saved ?\\
  {\bf A:} As soon as the program is interrupted to ensure that no other
  operation corrupts the {\em execution context} before it is saved. In the same
  way, no operation should be executed between the context restoration and the
  return from interrupt.

\end{frame}

%
%
%

\begin{frame}
  \frametitle{Event mechanism summary}

  \pgfuseimage{event}

\end{frame}

%
%
%

\begin{frame}
  \frametitle{Event mechanism summary}

  The previous figure illustrates what happens when a program is interrupted:

  \-

  \begin{enumerate}
    \item The microprocessor performs a privilege switch from User to Supervisor
      mode:
      \begin{itemize}
        \item Current stack switches from the program (user) stack to the
	  interrupt (kernel) stack.
        \item The {\em Program Counter} and the {\em Program Status Word} are
	 saved on the kernel stack.
      \end{itemize}
    \item The event handler saves the task's execution context on the kernel
      stack.
    \item The event handler execute the appropriate ISR.
    \item The event handler restores the task's execution context from the
      kernel stack.
    \item The event handler provokes a {\em Return From Interrupt}. Therefore,
      the microprocessor switches back to User mode:
    \begin{itemize}
      \item The last {\em Program Status Word} and the {\em Program Counter}
        of the interrupted program are restored with the values saved on the
	stack at step 1.
      \item Current stack switches back to the interrupted program's one.
    \end{itemize}
  \end{enumerate}

  \-

  Steps 1 and 5 are automatically performed by the microprocessor. Some
  microprocessors also handle steps 2 and 4.

\end{frame}

%
%
%

\begin{frame}
  \frametitle{Exercise}

  Motorola M68000 specifications:

  \begin{itemize}
    \item Operating modes
      \begin{itemize}
        \item User
	\item Supervisor
      \end{itemize}

    \item Registers:
      \begin{itemize}
        \item {\em D0}-{\em D7}: data registers
        \item {\em A0}-{\em A7}: address registers (A7 is the stack pointer)
        \item {\em PC}: program counter
        \item {\em CCR}
      \end{itemize}

    \item ISR dispatch is organized through an interrupt vector located at
      address 0x0. This 1024-bytes table looks like the one below:

      \-
      \begin{center}
      \pgfuseimage{m68kivt}
      \end{center}
  \end{itemize}

\end{frame}

%
%
%

\begin{frame}
  \frametitle{Exercise}

  \-

  When interrupted, the M68000 automatically saves the {\em CCR} and the
  {\em PC} onto the Supervisor stack {\em A7'}. The return from interrupt is
  assumed by the {\tt RTI} instruction.

  \-

  Write the code which could permit to 

\end{frame}

%
%
%

\begin{frame}[containsverbatim]
  \frametitle{Solution}

  \begin{verbatim}
event_init:
       moveq.l  #1, d0                  ; arg1 = 1 (interrupt id)
       lea      handler_1, a0           ; arg2 = handler_1 (interrupt handler)
       jsr      init_handler            ; call init_handler() subroutine
       rts

isr_1:
       ...                              ; ISR treatment
       rts

init_isr:
       movem.l  d0/a1, -(a7)            ; save a1 on the stack
       lsl.l    #2, d0                  ; a1 <- a1 * 4
       move.l   d0, a1                  ; 
       move.l   a0, 8(a1)               ; fill the interrupt vector entry with the handler address
       movem.l  (a7)+,d0/a1             ; restore a1 from the stack
       rts

handler_1:
       movem.l  d0-d7/aO-a6, -(a7)      ; save the execution context on the stack
       jsr      isr_1                   ; call the Interrupt Service Routine
       movem.l  (a7)+, dO-d7/a0-a6      ; restore the execution context from the stack
       rti                              ; return from interrupt
  \end{verbatim}

\end{frame}

%
%
%

\begin{frame}
  \frametitle{Stack size}

  \begin{quote}
    With experience, one learns the standard, scientific way to compute the
    proper size for a stack: Pick a size at random and hope.

    \begin{flushright}
     Jack Ganssle
    \end{flushright}
  \end{quote}


  \-

  \begin{itemize}

    \item {\bf Testing}\\
      Required stack size can easily be approximated by testing the stack needs
      in real conditions. Testing often underestimates the real needs.

    \-

    \item {\bf Analyzis}\\
      On the opposite, analyzis approach is much more complex and often
      overestimates the real needs.
  \end{itemize}

\end{frame}

%
%
%

\begin{frame}
  \frametitle{Latency}

  Interrupt latency is the amount of time elapsed:

  \begin{itemize}
  \item From the moment the microprocessor detects an interrupt, until the
    execution of the ISR's first instruction.
  \item From the moment the ISR has finished executing, until the interrupted
    porgram is finally resumed.
  \end{itemize}

  \-

  During these times, neither the microprocessor nor the kernel do actually
  handle the event. They just install a safe environment to run the ISR and to
  resume the interrupted program. In other words, latency is the amount of
  {\bf lost time} when an interrupt is handled.

  \-

  Part of the latency is due to inevitable microprocessor internal operations
  (such as privilege/stack switches, privilege checking\ldots). But the rest of
  the latency is caused by the event handler and must be minimalized by the
  programmer.

\end{frame}

%
%
%

\begin{frame}
  \frametitle{Overload}

  Interrupt handling overloads the system when:

  \begin{enumerate}
    \item latency is too strong / interrupt handling is too slow
    \item too many interrupts occur
  \end{enumerate}

  \-

  In both cases, a system overload leads to serious issues:

  \begin{itemize}
    \item missing interrupts
    \item starving user programs
    \item making interrupt handling durations unpredictable for real-time systems
  \end{itemize}

  \-

  reducing overload:
  \begin{itemize}
    \item optimizing interrupt handlers and ISR
    \item limitating the events rate (using DMA or buffering)
    \item using polling
  \end{itemize}

\end{frame}

%
%
%

\begin{frame}
  \frametitle{Nested events}

  Some processors support nested interrupts. An ISR that interrupts a program
  could itself be interrupted by the ISR for a higher-priority interrupt:
  \begin{center}
    \pgfuseimage{nested}
  \end{center}

  \-

  {\bf Rule:} an ISR cannot be interrupted by an event for a lower or equal
  pritority level.

\end{frame}

%
%
%

\begin{frame}
  \frametitle{Nested events \& real-time}

  Real-time systems commonly implement nested events.

  \-

  {\bf Advantages}

  \begin{itemize}
    \item Make the system more reactive, especially for high-priority
      interrupts.
    \item Allow to use events in the ISR.
  \end{itemize}

  \-

  {\bf Disadvantages}

  \begin{itemize}
    \item Consumes larger amounts of stack.
    \item Considering an IRQ which needs to be handled in a time T, what
      would happen if the ISR for this interrupt were itself interrupted by an
      event of a higher priority ?
  \end{itemize}

\end{frame}

\section*{Exercice 2 : Gestionnaire d'interruption (3 points)}

{\bf C\^ot\'e noyau}

Un SIGL a essay\'e d'optimiser le handler d'exceptions par d\'efaut dans la version 1.0 de Linux. Trouvez ses erreurs et proposez une correction.

{\bf Note:} Linux 1.0 est sortie en Mars 1994. \`A l'\'epoque Linux n'avait pas \'et\'e port\'e sur d'autres arcitectures que IA32.

\begin{verbatim}
/* This is the default interrupt ``handler'' :-) */
int_msg:
        .asciz ``Unknown interrupt\n''

.align 2
ignore_int:

        //code with sides effects before context saving


        cld                         ; ne pas prendre en compte
        pushl %eax
        pushl %ecx
        pushl %edx
        push %ds
        push %es
        push %fs

        movl $(KERNEL_CS),%eax
        mov %ax,%ds
        mov %ax,%es
        mov %ax,%fs

        pushl $int_msg
        call _printk

        pop %fs
        pop %es
        pop %ds
        popl %edx
        popl %ecx
        popl %eax

        ret

//pop microprocessor info
\end{verbatim}

\begin{itemize}
\item Vous consid\'erez qu'aucun autre registre que ceux qui apparaissent dans ce code ne sera utilis\'e pendant le traitement du handler.
\end{itemize}

\begin{correction}
\begin{verbatim}
/* This is the default interrupt ``handler'' :-) */
int_msg:
        .asciz ``Unknown interrupt\n''

.align 2
ignore_int:
        cld
        pushl %eax
        pushl %ecx
        pushl %edx
        push %ds
        push %es
        push %fs

        movl $(KERNEL_DS),%eax
        mov %ax,%ds
        mov %ax,%es
        mov %ax,%fs

        pushl $int_msg
        call _printk
        popl %eax

        pop %fs
        pop %es
        pop %ds
        popl %edx
        popl %ecx
        popl %eax

        iret
\end{verbatim}
\end{correction}





%
% ---------- header -----------------------------------------------------------
%
% project       kaneton
%
% license       kaneton
%
% file          /home/mycure/kane...ecture/kernels/scheduling/scheduling.tex
%
% created       julien quintard   [fri oct 24 17:31:58 2008]
% updated       julien quintard   [wed apr 22 11:14:18 2009]
%

%
% ---------- setup ------------------------------------------------------------
%

%
% path
%

\def\path{../../..}

%
% template
%

%%
%% copyright     (c) julien quintard
%%
%% project       kaneton
%%
%% file          /home/mycure/kaneton/view/templates/lecture.tex
%%
%% created       julien quintard   [sat nov 19 17:13:03 2005]
%% updated       julien quintard   [fri dec  2 22:36:34 2005]
%%

%
% class
%

\documentclass[8pt]{beamer}

%
% packages
%

\usepackage{pgf,pgfarrows,pgfnodes,pgfautomata,pgfheaps,pgfshade}
\usepackage{colortbl}
\usepackage{times}
\usepackage{amsmath,amssymb}
\usepackage{graphics}
\usepackage{graphicx}
\usepackage{color}
\usepackage{xcolor}
\usepackage[english]{babel}
\usepackage{enumerate}
\usepackage[latin1]{inputenc}

%
% style
%

\usepackage{beamerthemesplit}
\setbeamercovered{dynamic}

%
% verbatim font
%

\definecolor{verbatimcolor}{rgb}{0,0.4,0}

\makeatletter
\renewcommand{\verbatim@font}
  {\ttfamily\footnotesize\color{verbatimcolor}\selectfont}
\makeatother

%
% new line
%

\newcommand{\nl}[0]{\vspace{0.4cm}}

%
% date
%

\date{\today}

%
% logos
%

\pgfdeclareimage[interpolate=true,width=34pt,height=18pt]
                {epita}{../../logos/epita}
\pgfdeclareimage[interpolate=true,width=49pt,height=18pt]
                {upmc}{../../logos/upmc}
\pgfdeclareimage[interpolate=true,width=25pt,height=18pt]
                {lse}{../../logos/lse}

\newcommand{\logos}
  {
    \pgfuseimage{epita}
  }

%
% institute
%

\institute
{
  \inst{1} kaneton microkernel project
}

%
% table of contents at the beginning of each section
%

\AtBeginSection[]
{
  \begin{frame}<beamer>
   \frametitle{Outline}
    \tableofcontents[current]
  \end{frame}
}

%
% table of contents at the beginning of each subsection
%

\AtBeginSubsection[]
{
  \begin{frame}<beamer>
   \frametitle{Outline}
    \tableofcontents[current,currentsubsection]
  \end{frame}
}


%
% title
%

\title{Scheduling}

%
% document
%

\begin{document}

%
% title frame
%

\begin{frame}
  \titlepage
\end{frame}

%
% outline frame
%

\begin{frame}
  \frametitle{Outline}

  \tableofcontents
\end{frame}

%
% figures
%



%
% ---------- text -------------------------------------------------------------
%

%
% introduction
%

\section{Introduction}

% 1)

\begin{frame}
  \frametitle{Overview}

  This course targets \textbf{multitasking, multiprocessing or multithreading} in an operating system.

  \-

Former operating systems like MS-DOS used to run only one program at a time, then next generation of operating systems introduced \textbf{cooperative multitasking}. Finally, although software concepts were known for a long time, and the hardware allowed it, the true multitasking arrived with \textbf{preemptive} operating systems.    
 
\end{frame}

% 2)

\begin{frame}
  \frametitle{Assumptions}

 Previous lectures should be read and understood : memory and interrupts are the cornerstone of any kernel. These lectures will be intensively refered to.

\-

The bibliography contains many good papers and books, read it!

\end{frame}



%
% overview
%

\section{Overview}

\section{Threads and processes}

% 1)

\begin{frame}
  \frametitle{Threads :: Definition}

Threads are the entities scheduled for execution on the CPU. A thread is a single sequence stream within a process. Because threads have some properties of processes, they are sometimes called lightweight processes.

\-

In many respect, threads are a popular way to improve application through parallelism. The CPU switches rapidly back and forth among threads, giving illusaion that the threads are running in parallel.

\-

Each thread has its own stack. Since thread will generally call different procedures and thus a different execution history. This is why thread needs its own stack.

\-

A thread has, or consists of, a program counter (PC), a register set and a stack space. Threads of a process share together their code and data section, and operating system ressources like file descriptors and signals.

\end{frame}

% 2)

\begin{frame}
  \frametitle{Threads :: Processes versus threads}

Similarities:

\begin{itemize}
\item
Like processes, thread share CPU and only one thread is active (running) at the time.
\item
Like processes, threads within a process execute sequentially.
\item
Like processes, threads can create children.
\item
Like processes, if one thread is blocked, another thread can run.
\end{itemize}

\-

Differencies:

\begin{itemize}
\item
Unlike processes, threads are not independent one from antother.
\item
Unlike processes, all threads can access every address in the task.
\item
Unlike processes, thread are designed to assist one another. It is important to notice that processes might or might not assist one another because processes may have been created by different user. Process are to compete to get access to the CPU, not to cooperate.
\end{itemize}

\end{frame}



%
% thread implementation
%

\section{Thread implementation}

\subsection{In user-space}

%1)

\begin{frame}
  \frametitle{Presentation}

Each process owns a private \textbf{thread table} to keep track of the threads in that process. When a thread does something that may cause it to become blocked locally, for example waiting for another thread in its process to complete some work, it call a run-time system procedure. This procedure checks if that thread should be put in a blocked state, and if so starts to run another ready thread. Thread's context are stored locally, in the \textbf{thread table}.

\end{frame}

%2)

\begin{frame}
  \frametitle{Advantages}

Some architectures offer ways (understand \textit{instructions}) to store and load general-purpose registers, stack pointer and program counter. Therefore, the thread context-switch may be extremly fast with a thread implementation in user-space. Another benefit from this kind of implementation is the availability of threads on systems which do not initially support threads (no kernel support of threads).

\end{frame}

%3)

\begin{frame}
  \frametitle{Advantages}

\begin{itemize}
\item
Available on operating systems which does not offer a native support of threads (threads facilities in user-space usually come as a library linked with the threaded application). User-level threads do not require modification of the operating system.
\item
A thread must explicitely call the thread scheduler when it finished its work (with a call to \code{thread\_yield()} fo example). Therefore, thread in user-space implementation is often called \textbf{cooperative} multi-threading.
\item
Threads in user-space are lightweight, saving and restoring context are just local procedures. Much more efficient than issuing system calls to the kernel (no context save/restore, no cache flush, ...) ! Scheduling is very fast.
\item
Another advantage is that users can define thei own scheduliung algorithm for a precise task.
\end{itemize}

\end{frame}

%4)

\begin{frame}
  \frametitle{Drawbacks}

\begin{itemize}
\item
Despite their overall better performance, threads in user-space suffer from a major problem : how blocking system calls are implemented ? If a thread blocks in the kernel, how the kernel can notice the user-space scheduler to reschedule the task ? (solutions : rewrite syscalls as non-blocking (forget about it), \textbf{door upcall}, \etc{}, \textbf{wrapper}, ...).
\item
On a page-fault, the thread will block until the disk I/O for a remote page is completed ! Even though another thread could be executed during the I/O access.
\item
Roud-robin fashion scheduling is almost impossible since there is no clock interrupts inside a process.
\end{itemize}

\end{frame}

%5)

\begin{frame}
  \frametitle{Final words about user-space threads}

Despite better performance, user-space threads lack \textbf{preemption} facilities. Threads are precisely usefull in application which issue numerous possibly blocking syscalls such as servers \etc{}

\end{frame}


\subsection{In kernel space}

%6)

\begin{frame}
  \frametitle{Implementing thread in Kernel (kernel-thread)}

When a thread wants to create a new thread, or destroy an existing one, it makes a syscall to the thread manager of the kernel. It is the kernel which manage threads's context saving and restoring into kernel's structures.
\end{frame}

%7)

\begin{frame}
  \frametitle{Advantages}

\begin{itemize}
\item
All call that might block are implemented as system call, therefore if the kernel will block into a system call (semaphore \textit{P} operation for instance), it can schedule another thread to run while the blocked thread wait for the completion of his syscall. This is one aspect of \textbf{preemption}.
\item
Because the kernel knows how many threads a process own, it can give more time to an heavy process.
\item
SMP support.
\end{itemize}

\end{frame}

%8)

\begin{frame}
  \frametitle{Drawbacks}

\begin{itemize}
\item
Issuing syscalls is much more heavier than calling run-time procedures. Therefore creating, destroying threads, and any other possibly blocking syscalls are implemented at greater cost, much more overhrad will be incured.
\item
The kernel must support this thread implementation.
\end{itemize}

\end{frame}


\subsection{Hybrid implementation}

\begin{frame}
  \frametitle{Scheduler activations}

The goal of \textbf{scheduler activation} is to combine the advantages of user threads (good performance) with the advantages of kernel threads (not having to use a lot of tricks to make thing work).

\-

The kernel gives a certain number of virtual processors to processes, and the user-space runtime allocate these processors to its user-space threads. Obviously, virtual processors are replaced by real processors on multiprocessors systems.

\-

Another feature of scheduler activation is the \textbf{door upcall}, whenever a thread make a blocking syscall, the kernel has some ways to notice the user-space that this threads is now blocked. This is roughly analogous to a signal in UNIX.

\end{frame}


%
% kernel implementation
%

\section{Kernel implementation}

% 1)

\begin{frame}
  \frametitle{Kernel managers}

In order to do a kernel implementation of multi-processing (also known as multi-tasking in old documents about kernel), and one step further, of multi-threading, the kernel will be strongly overhauled. Many services and modifications have to be undertaken.

\-

Some of these modifications and additions to the core code are :

\begin{itemize}
\item
The dispatcher : the dispatcher is in charge of management of queues of runnable threads, and of thread preemption.

\item
The scheduler : the scheduler is in charge of telling what is the next thread who should get the processor. Strong theory about tasks scheduling, and hard to chose tradeoffs are part of the scheduler implementation.

\end{itemize}

\end{frame}


% 2)

\begin{frame}
\frametitle{The context-switch}

The context-switch is the process of giving the processor to another thread than the one currently running on. Although it is not obvious at a first glance, few things define a task, it is called a \textbf{context}. On most of processors, the hardware requirement to define a task would be :

\begin{itemize}
\item
a program counter
\item
a stack pointer
\item
general purpose registers
\end{itemize}

\-

And for advanced processors :

\-

\begin{itemize}
\item
a page-directory base address
\item
an interrupt stack pointer
\item
\etc{}
\end{itemize}

\end{frame}

% 3)

\begin{frame}
\frametitle{The context-switch}

Therefore, to give the processor to another task than the one currently running on, it is only required to restore the thread context : the program counter where the thread which is the last instruction that the thread executed before getting preempted, a stack pointer because each thread needs its own stack (try to figure out why), and a set of general purpose registers containing data for processing, arguments being passed to a function before a call, \etc{}

\-

Context switches on most architextures are a relatively expensive operation and as such they are avoided as much as possible. Quite a bit of actual work can be done during the time it takes to perform a context-switch.

\end{frame}

% 3)

\begin{frame}
\frametitle{Source of calls to the context-switch routine}

The context-switch routine initiates the context-switching of a thread off a processor, figures out which thread should run next, and context-switches the selected thread ont a processor for execution. It is called from many places within the operating-system.

\-

Some of situations or function calls that will induce a call to the context-switch routine on Solaris 2.6 are :

\begin{itemize}
\item
At the end of an interrupt thread
\item
After the creation of a kernel thread
\item
After a sleep or a wakeup of a thread
\item
When a thread migrate from one processor to another one
\item
When a processor state is changed to \textit{pause}
\item
During a mutex lock acquisition
\item
During a semaphore \textit{p} operation
\end{itemize}

\-

Many other operations could result in a call to the context-switch routine, for instance when the kernel offers message-passing facility, or during blocking I/O request.

\end{frame}

%4)

\begin{frame}
\frametitle{Priority-inversion}


Priority-inversion is evil.

\end{frame}

%5)

\begin{frame}
\frametitle{The solution to priority-inversion}

Solutions address situations where a thread is holding a critical ressources, such as a mutex lock, and a few extra ticks of execution time will allow the thread to complete its task and free the lock. Otherwise, if the thread is taken off the processor before releasing the ressource, other threads that need the same ressource will begin to block as long as the owner thread get rescheduled on a processor.

\end{frame}

%6)

\begin{frame}
\frametitle{The solution to priority-inversion}

\textbf{Priority inheritance} : The most common solution in microkernel architecture is the priority inheritance. When a thread get preempted whereas holding a critical ressource, if a higher priority thread needs that ressource, the owner thread will be assigned with the waiting thread priority as long as owning the critical ressource.

\-

\textbf{Quantum stretching} : Operating-systems such as Solaris offer a way to give a kernel thread some extra timeslices by stretching a thread's time quantum for a short time : this is known as \textit{preemption control}. What this does is effectively give the kthread a few extra clock ticks of execution time on a processor, beyond its time quantum, before it is switched off. This feature addresses 

\-

\textbf{Policies} : A simple solution is to establish policies : a thread should never ever wait for a ressource that a lower priority thread could acquire. In other word, differents priority threads and a ressource don't mix.

\end{frame}

\section{Processes \& threads in various kernels}

\subsection{Processes \& threads in Linux 2.6.x}

% 1)

\begin{frame}
\frametitle{Priority-inversion}

Linux, like many other operating systems, regards threads as simply processes that might share certain ressources. Instead of being something different than a thread or a group of threads, a process in Linux is simply a group threads that share something called a \textit{thread group ID (TGID)} and whatever ressource are necessary. 

\-

Some clarification must be done about Linux's treatment of processes ans threads with the terms themselves. The term \textit{task} is used in Linux to mean a thread, therefore it does not match the POSIX sense, which considers a task as a process. In the Linux task structure \code{task\_struct} (one of which exists for each thread), the TGID that is a process's POSIX PID is stored as \code{[task\_struct]->tgid}. Linux assigns a unique \textbf{PID} to each thread in \code{[task\_struct]->pid}, but the POSIX PID that most people think of is the task's TGID.

This approach makes spawning threads on Linux much faster than other operating systems like BSB or Windows.

\end{frame}

\subsection{Processes \& threads in Solaris 2.6 and after}

% 1)

\begin{frame}
\frametitle{The gap between multiprocessing and multiprocessor}

Where as most of operating systems saw threading as a way to emulate a multiple processors behavior on a single computer, Sun Microsystems directly saw threads as a process to enable parallel execution on a multiprocessor system. Over a decade, Solaris thread's implementation changed. 

\begin{itemize}
\item
System size growth (in terms of processors and memory).
\item
Multithreaded technique evolved and matured.
\item
API became standardized.
\item
Threads became ubiquitous in modern operating systems.
\item
Threaded application becam the norm, not the exception.
\end{itemize}

\end{frame}

% 2)

\begin{frame}
\frametitle{The Solaris's two-tiered model}

The very fist versions of Solaris was shipped with a MxN thread model. In this model:

\begin{itemize}
\item
Threads exist in two domains: kernel and user domain.
\item
User threads are not visible to the kernel, instead the kernel see \textbf{LightWeight Processes} (or \textit{LWPs}) which contains one or more user threads.
\item
Kernel threads (in Solaris, \textit{LWPs}) are the only schedulable entity the kernel knows.
\end{itemize}

\-

This model is often regarded as a \textbf{two-tiered model}.

\-

(schema)

\end{frame}

% 3)

\begin{frame}
\frametitle{Two-tiered advantages}

The first advantage of a two-tiered architecture is that application thread synchronization can be done either via the kernel (using system calls), or at the user level (leveraging atomic test-and-set facilities in the multiprocessor hardware)[note: see \textbf{lwarx/strx} instruction on PowerPC]. A system call may take many hundreds or even thousands of instructions, but a simple compare-and-swap operation takes just one (although this may involve a considerable number of clock cycles to complete).

\-

The other advantage of a two-tiered implementation is that the cost of context switching in a user-level scheduler is considerably less expensive than with the kernel's scheduler. In the early days, it was expected that user-level scheduling would be orders of magnitude faster than kernel scheduling.

\-

The first advantage has proven to be significant, but the second advantage has not. Building a user-level scheduler that works well in tandem with the kernel's scheduler is a significant challenge. However, the kernel's ability to efficiently schedule many threads has improved.

\end{frame}

% 4)

\begin{frame}
\frametitle{Recent innovations}

Solaris 2.5 was a watersched in Solaris thread's implementation. Solaris 2.5.1 provided support for \textit{UltraSPARC} multiprocessor, and therefore saw big changes.

\end{frame}

% 5)

\begin{frame}
\frametitle{Preemption control}

Solaris 2.6 software (August, 1997) addressed a number of issues relating to the visibility of userlevel threads to the kernel. One issue was that the kernel had no knowledge of user-level thread synchronization. An application thread can acquire a mutex using a simple, atomic test-and-set instruction, without troubling the kernel. In well-designed, scalable threaded applications, mutexes are held only for very short durations. It is not a good idea to preempt a thread that  is holding mutex, since other threads may end up having to wait for the mutex holder to run again.

\-

Solaris 2.6 software introduced the concept of the 'don't preempt me' flag. When acquiring a lock (or any other critical operation), the user thread set the flag in a shared structure of the corresponding LWP. When rescheduling, the kernel see the flag and decide to postpone the thread preemption for a short period of time.

\-

Obviously some mechanisms are set to avoid any abuses of this preivilege by the application.

\end{frame}

% 6)

\begin{frame}
\frametitle{Scheduler Activation}

Prior to Solaris 2.6 software, the kernel used a special signal, SIGWAITING, to inform the threads library that all LWPs were blocked in the kernel. This gave the library the opportunity to create another LWP so it could to continue to run other, nonblocking threads. In Solaris 2.6 software, this mechanism was augmented by the preferential use of a \textbf{door upcall}. Essentially, this involves the kernel being able to call into the user-level thread scheduler to adjust the number of LWPs in the process pool of LWPs. This door mechanism is more efficient than a signal, but if necessary, Solaris 2.6 software falls back to using the SIGWAITING mechanism.

\end{frame}

% 7)

\begin{frame}
\frametitle{Breaking the 32-Bit Barrier}

Of course, the most significant innovation in Solaris 7 software was the introduction of 64-bit address spaces. Until then, 32-bit addressing limited each process to just four gigabytes (GB) of virtual memory. For threaded applications the issue was more acute, since thread stacks had to share the four-gigabyte process address space with program text and data.

\-

By default, each thread stack has an adjacent \textit{red zone}, an unmapped region of at least one page that will trap stack overflows. Although red zones do not consume physical memory, they do reduce the amount of address space available for other purposes. Each kernel thread also has a stack and a red zone. The kernel thread stack size is configurable. In 32-bit kernels on UltraSPARC systems, it defaults to eight kilobytes (KB).  All kernel thread stacks are allocated in the \textit{kernel pageable segment} known as \textit{segkp}, but this is fixed at 512 MB for all 32-bit kernels and the 64-bit kernel of Solaris 7 software.

\end{frame}

% 7bis)

\begin{frame}
\frametitle{Breaking the 32-Bit Barrier}

So, with 16-KB stacks (each with an eight-KB red zone) there is a limit of approximately 21,000 stacks (a maximum of 21,000 LWPs system-wide), consuming about 340 MB of physical memory. From Solaris 8 software onwards, the 64-bit kernel allows segkp to be sized from 512 MB to 24 GB, with the default being two gigabtye (sufficient for more than 87,000 kernel thread stacks/LWPs).

\end{frame}

% 8)

\begin{frame}
\frametitle{What's wrong with the MxN implementation}

For some reasons, the MxN implementation has been retired since Solaris 2.8. We will discuss the two main reasons which lead to this removal.

\end{frame}

% 9)

\begin{frame}
\frametitle{Threads and Signals Don't Mix}

The implementation of synchronous signals in a multithreaded environment is not difficult. The signal can simply be delivered straight to the LWP on which the thread that triggered the signal is running. However, asynchronous signals are a different matter in an MxN implementation, since the only application thread that has the signal unmasked may not be currently running on an LWP.

\-

The old MxN implementation attempted to solve the asynchronous signal issue by introducing a dedicated LWP, Asynchronous Signal LWP (ASLWP), into the process model. Whenever the kernel wants to deliver an asynchronous signal to a process and ASLWP is present, the kernel will notify ASLWP instead of attempting to deliver the signal directly to an LWP. Once ASLWP sees a suitable application thread running on an LWP, it asks the kernel to redirect the pending signal to that LWP. Not only is this very complex to implement correctly, but it also means that ASLWP can become a bottleneck where there is a high volume of signals.

\end{frame}

% 10)

\begin{frame}
\frametitle{Bringing fresh stuffs with Solaris 9}

Solaris 9 brought new concepts about thread's implementation. Thereafter shows some of the most noticeable innovations \etc{}

\end{frame}

% 11)

\begin{frame}
\frametitle{Bringing fresh stuffs with Solaris 9}

\textbf{MxN Implementation Retired} : The old MxN implementation has been gracefully retired and replaced with an enhanced version of the 1:1 implementation.

\-

\textbf{Adaptive Mutexes Revisited} : A thread waiting for a process private mutex will spin only if the lock holder is currently running on a CPU. The process shared case remains unchanged. If there is more than one CPU, the thread will always spin before blocking. If there is only one processor on the system, there is no need to spin because there is a little chance that anything release the lock (aside from \textit{SEU}), therefore the thread shall be immediately preempted.

\-

\textbf{Adaptive Mutex Unlock} : When a thread is about to release the mutex it is holding, it looks to see if there are any waiters spinning. If so, it releases the lock and spins itself for a short while to see if the lock is acquired by another thread. This boosts performance because the thread releasing the lock no longer needs to enter the kernel to wake up a waiter.

\end{frame}

\subsection{Processes \& threads in QNX}

% 1)

\begin{frame}
\frametitle{Thread data structure}

QNX supports POSIX threads model. Like many kernels, a thread is defined by a private structure (each thread own one) which contains:

\-

\begin{itemize}
\item
\textit{tid} : identifying a thread by a unique \textit{thread ID}
\item
name : each thread can have a name
\item
register set : the thread \textbf{context}
\item
stack : stored within the address space of its process
\item
signal mask : each thread has its own signal mask
\item
cancellation handlers : callback function that are executed when the thread terminates
\end{itemize}

\end{frame}

% 2)

\begin{frame}
\frametitle{Thred Local Storage}

Beside these features, QNX adds a more singular feature to its thread implementation. Remember that each process is MMU-protected from each other, and each process may contain one or more threads that share the process's address space. The main difference between QNX threads and much of other operating systems threads implementation is that threads still have some \textbf{private} data are called \textit{thread local storage} or \textit{TLS}. The TLS is used to store \textbf{per-thread} information, and provides a mechanism for associating a process global integer key with a unique per-thread data-value.

\end{frame}

% 3)

\begin{frame}
\frametitle{Thread life cycle}

QNX offers significantly more thread's state than any other common kernels. QNX's thread may be in one of the following states : (schema).

\end{frame}



\section{Kernel tuning}

% 1)

\begin{frame}
\frametitle{Everything as a thread}

In microkernel, very few code make part of the essential monolitic code of the kernel, everything in a microkernel must be considered as a thread.

\-

Therefore, most of kernel services are regarded as a thread : for example the code for waiting for a thread is ported to a \textbf{kernel-thread}. Kernel-thread are like user-defined threads, excepts there are created directly by the kernel. Finally very few code remains monolitic.

\-

The drawback of this implementation is greater cost of calling the system. Refer to the \name{mach} kernel for performance benchmarks.

\end{frame}

% 2)

\begin{frame}
\frametitle{Secure the interrupt handlers}

One most of systems (including Solaris), interrupts handler are ported threads. It brings two big advantages :

\begin{itemize}
\item
Security: interrupts handlers no more run with superuser privilege.
\item
Priority: even though the low-level handler can preempt the system, the code for the user-defined interrupt handler can be assigned a priority and run as a thread. Therefore an implicit priority can be assigned to any interrupt-handlers, just with assigning a certain priority to the interrupt handler thread.
\end{itemize}

\end{frame}



%
% multicore
%

\section{Multicore}

% 1)

\begin{frame}
\frametitle{Asymetrical MultiProcessor}

Each processor gets :
\begin{itemize}
\item
One space memory
\item
One image of the kernel loaded into his private memory
\end{itemize}

\-

Communication between cores must be explicit, with the use of a communication protocol. Processors must exchange gloabl data such as global variables, and global events such as process preemption, ressource freeing, ...

\-

The good thing about AMP is it require a little modification of the kernel core.

\end{frame}

% 2)

\begin{frame}
\frametitle{Drawbacks of AMP}

In this context, migrating task from one core to another became extremly painful : a complex protocol has to be setup to migrate the task context frome one kernel to another.

\-

Moreover, because each processor needs one kernel image, the memory is quickly wasted with additional processor. Because of bus contention, it is almost impossible to have more than two processors in this architecture.

\end{frame}

% 3)

\begin{frame}
\frametitle{Symetrical MultiProcessing}

Each processor sees the same memory, there is only one kernel image in memory. This image is the same for every processors. Nonetheless, each processors cannot be in the same kernel service (or running the same threads) at the same time.

\-

Communication between cores became implicit. Every processors shares the same memory, therefore access to global variables are seen locals. No need to use some heavy message passing protocol, but take care of concurrent access !

\-

Easy task migration (and possibly load balancing) : tasks are dispatched on the first available processor.

\-

Working with more than two processor is transparent, and an SMP system will work even if only one processor is available, which may make think of reduncy.

\end{frame}

% 4)

\begin{frame}
\frametitle{Kernel modification}

Some complex modifications of code have to be undertaken :

\begin{itemize}
\item
Cache coherency
\item
Scheduling and cache contention
\item
I/O routing
\item
Task affinity
\item
Atomic access
\end{itemize}

\end{frame}

% 5) 

\begin{frame}
\frametitle{Enforcing atomic access}

Atomic access can be achieved with many different means:

\-

The first mean is to assert an exclusive access on the data bus : for example with the LOCK instruction on Intel. This instruction prevent any other processor to access the memory during a lap of time.

\-

Another mean is to use atomic instructions provided by the architecture (like \textit{lwarx/strcx} on PowerPC).

\end{frame}

% 6)

\begin{frame}
\frametitle{Asserting cache coherency}

\textbf{Race condition} happens when two or more processors have a diffrent image of the main portion of memory, stored into their L1 cache.

\-

To assert the cache coherency, functions should be protected with spinlock or reentrant. Modern architectures like Intel offer an hardware solution : the snooper. The snooper spy data bus accesses made by processors and is in charge of assuring the cache coherency.

\end{frame}

% 7)

\begin{frame}
\frametitle{InterProcessor Interrupt}

Some actions of one processors must be notified to other processors immediately.

\begin{itemize}
\item
Task suspension
\item
Task wakeup
\item
Semaphore \textit{verhog}
\item
Message send
\item
\etc{}
\end{itemize}

\end{frame}


%
% conclusion
%

\section{Conclusion}

% 1)

\begin{frame}
  \frametitle{Conclusion}

  In this course, basic concepts of multitasking and associated scheduling algorithms has been shown.

 

\end{frame}



%
% bibliography
%

\section{Bibliography}

\begin{thebibliography}{3}
  \bibitem{Hp-UX}
HP-UX Linker and Libraries User's Guide, HP 9000 Computers, HP

  \bibitem{Sun}
Linkers and Libraries Guide, Sun Microsystems

\end{thebibliography}


\end{document}

\section*{Exercice 4 : Sleep/Wakeup (X points)}


\section*{Exercice 5 : Unit� � virgule flottante (5.5 points)}

{\bf x87 FPU :}

L'unit\'e \`a virgule flottante (FPU) du Pentium, la x87 FPU, est un
composant int\'egr\'e au microprocesseur et offrant le support des nombres
flottants IEEE 754.

La FPU contient les registres suivants :

\begin{itemize}
\item
  Status Register : contient les flags (Z, N\ldots)
\item
  Control Register : contient le masque d'exception
\item
  Tag Word : indique l'�tat des registres R0-R7
\item
  Instruction Pointer : pointeur vers la derni�re instruction
  flottante ex�cut�e
\item
  Instruction Opcode : opcode de la derni�re instruction flottante
  ex�cut�e
\item
  Data Pointer : pointeur vers la derni�re op�rande charg�e en
  m�moire
\item
  R0-R7 : huit registres de calcul (aussi appel�s ST(0)-ST(7))
\end{itemize}

Les instructions suivantes (sous forme de macros) manipulent
le contexte de la FPU :

\begin{itemize}
\item
  \verb|FSAVE(ptr)| : sauvegarde le contexte FPU � l'adresse
  \emph{ptr}
\item
  \verb|FRSTOR(ptr)| : restaure le contexte FPU depuis l'adresse
  \emph{ptr}
\end{itemize}

Lorsque le flag TS du registre CR0 est � 1 et qu'une instruction
flottante est ex�cut�e, une exception 7 (Device Not Available) est
lev�e. Le flag TS est mis � 0 par la macro CLTS() ou mis � 1 par
STS().

Dans l'exercice qui suit, vous utiliserez ce m�canisme
d'exception pour effectuer les changements de contexte uniquement au
moment opportun.

\begin{description}
\item {\bf Impl�mentation du changement de contexte pour la FPU}

\begin{enumerate}
\item
  Quatre threads tournent sur notre syst�me : T1, T2, T3 et T4 sont
  ex�cut�s l'un apr\`es l'autre (Round-Robin).

  Les threads T1, T2 et T4 effectuent des calculs flottants.

  \`A partir du sch�ma ci-dessous, identifiez � quels instants vous
  allez devoir changer le contexte FPU.

  \begin{center}
    \includegraphics[width=\linewidth]{figures/cs-fpu}
  \end{center}

  D\'eduisez-en la valeur du flag TS au fil de l'ex�cution de chacun des
  threads.

\item
  La structure permettant de stocker le contexte FPU vous est
  donn�e. Elle se nomme \emph{t\_x87\_context}. Elle contient tous les
  registres cit�s plus haut. Ne vous int\'eressez pas au d\'etail de son contenu,
  mais sachez que la structure est compatible avec \verb|FSAVE| et \verb|FRSTOR|.

  Dans quel objet kaneton allez-vous stocker le contexte FPU d'un
  thread ? Dans quelle fonction de l'interface kaneton allez-vous
  initializer ce contexte ?

  \textbf{Dans la suite de l'excercice, consid\'erez que les
  structures de contexte FPU sont correctement remplies.}
\item
  \'Ecrivez le code du gestionnaire d'interruption Device Not
  Available. C'est bien evidemment dans ce dernier que vous devez
  �changer les contextes FPU.

  Indiquez � quel endroit du code (aussi bien celui que vous venez
  d'�crire que celui des autres fonctions du scheduler) vous devez
  mettre � jour TS.

\end{enumerate}
\end{description}


\newpage
\part*{Annexe}

\section{kaneton as manager}

{\large {\bf as object}}

\begin{verbatim}
       typedef struct
         {
           i_as         asid;

           i_task       tskid;

           i_set        segments;
           i_set        regions;

           machine_data(o_as);
       }                o_as;
\end{verbatim}

\subsection*{as interface}

\function{}{as\_vaddr}{(i\_as \argument{id},
                      t\_paddr \argument{physical},
                      t\_vaddr \argument{virtual})}
	 {
	   This function translates a physical address into its virtual
	   address.
	 }

\function{}{as\_paddr}{(i\_as \argument{id},
                      t\_vaddr \argument{virtual},
                      t\_paddr \argument{physical})}
	 {
	   This function translates a virtual address into its physical
	   address.
	 }

\function{}{as\_clone}{(i\_task \argument{task},
                      i\_as \argument{old},
                      i\_as* \argument{new})}
	 {
	   This function clones an address space taking care of cloning
	   all the necessary: segments, regions etc\ldots
	 }

\function{}{as\_reserve}{(i\_task \argument{task},
                        i\_as* \argument{id})}
	 {
	   This function reserves an address space object for the
	   task \argument{task} object.

	   The reserved address space object's identifier is returned
	   in \argument{id}.
	 }

\function{}{as\_release}{(i\_as \argument{id})}
	 {
	   This function just releases the address space \argument{id}.
	 }

\function{}{as\_get}{(i\_as \argument{id},
                    o\_as** \argument{o})}
	 {
	   This function should only be used by the as manager, the segment
	   manager and the region manager. It just returns the address space
	   object corresponding to the address space identifier \argument{id}.
	 }



\section{kaneton segment manager}

\subsection*{segment object}

\begin{verbatim}
      typedef struct
        {
          i_segment    segid;

          i_as         asid;

          t_paddr      address;
          t_psize      size;

          t_perms      perms;

          machine_data(o_segment);
      }                o_segment;
\end{verbatim}

\subsection*{segment interface}

\function{}{segment\_clone}{(i\_as \argument{as},
                           i\_segment \argument{old},
                           i\_segment* \argument{new})}
	 {
	   This function clones a segment which will then belongs to
	   the address space object \argument{as}. Cloning a segment
	   consists in reserving a new segment with the
	   exact same properties and copying its content.
	 }

\function{}{segment\_read}{(i\_segment \argument{id},
                          t\_paddr \argument{offset},
                          void* \argument{buffer},
                          t\_psize \argument{size})}
	 {
	   This function reads \argument{size} bytes at offset
	   \argument{offset} from the segment \argument{id}.
	 }

\function{}{segment\_write}{(i\_segment \argument{id},
                           t\_paddr \argument{offset},
                           const void* \argument{buffer},
                           t\_psize \argument{size})}
	 {
	   This function writes the data of \argument{buffer} into the
	   segment \argument{id}.
	 }

\function{}{segment\_copy}{(i\_segment \argument{dst},
                          t\_paddr \argument{offd},
                          i\_segment \argument{src},
                          t\_paddr \argument{offs},
                          t\_psize \argument{size})}
	 {
	   This function copies data from segment \argument{src} to
	   segment \argument{dst}.
	 }

\function{}{segment\_reserve}{(i\_as \argument{as},
                             t\_psize \argument{size},
                             t\_perms \argument{perms},
                             i\_segment* \argument{id})}
	 {
	   This function reserves a segment with specified properties.
	 }

\function{}{segment\_release}{(i\_segment \argument{id})}
	 {
	   This function releases the segment \argument{id}.
	 }

\function{}{segment\_perms}{(i\_segment \argument{id},
                           t\_perms \argument{perms})}
	 {
	   This function changes the permissions of the segment \argument{id}.
	 }

\section{kaneton region manager}

\subsection*{region object}

\begin{verbatim}
      typedef struct
        {
          i_region     regid;

          i_segment    segid;

          t_vaddr      address;
          t_paddr      offset;
          t_vsize      size;
          t_opts       opts;

          machine_data(o_region);
      }                o_region;
\end{verbatim}

\subsection*{region interface}

\function{}{region\_reserve}{(i\_as \argument{as},
                            i\_segment \argument{segment},
                            t\_paddr \argument{offset},
                            t\_opts \argument{opts},
                            t\_vaddr \argument{address},
                            t\_vsize \argument{size},
                            i\_region* \argument{id})}
	 {
	   This function reserves a region with specified properties.
	 }

\function{}{region\_release}{(i\_as \argument{as},
                            i\_region \argument{id})}
	 {
	   This function releases the region \argument{id} that belongs
	   to the address space object \argument{as}.
	 }


\newpage

\section{kaneton libmips}

\subsection*{Interface}

\function{t\_vaddr}{mips\_bad\_vaddr}{}
{
  Get the virtual address that caused the fault.
}

\function{}{mips\_set\_tlb}{(t\_uint32 \argument{entry},
			   t\_uint32 \argument{id},
			   t\_vaddr \argument{vaddr},
			   t\_paddr \argument{paddr},
			   t\_perms  \argument{perms})}
{
  This function fills the entry of index \argument{entry} of the TLB
  with a translation of virtual address \argument{vaddr} in address
  space \argument{id} to the physical address \argument{paddr}.

  Permissions \argument{perms} are associated with the translation.
}

\function{}{mips\_get\_tlb}{(t\_uint32 \argument{entry},
			   t\_vaddr* \argument{vaddr},
			   t\_uint32* \argument{id})}
{
  This function get the virtual address \argument{vaddr} and the
  address space identifier \argument{id} translated by the entry
  \argument{entry} of the TLB.
}

\function{}{mips\_del\_tlb}{(t\_uint32 \argument{entry})}
{
  Mark the \argument{entry} of the TLB as invalid (remove it).
}

\function{t\_uint32}{mips\_number\_entry}{}
{
  Get the total number of entries in the TLB.
}

\function{t\_uint32}{mips\_random\_entry}{}
{
  Get the index of a random entry in the TLB that can be replaced.
}

\end{document}

