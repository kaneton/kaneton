\section{kaneton as manager}

{\large {\bf as object}}

\begin{verbatim}
       typedef struct
         {
           i_as         asid;

           i_task       tskid;

           i_set        segments;
           i_set        regions;

           machdep_data(o_as);
       }                o_as;
\end{verbatim}

\subsection*{as interface}

\function{as\_vaddr}{(i\_as \argument{id},
                      t\_paddr \argument{physical},
                      t\_vaddr \argument{virtual})}
	 {
	   This function translates a physical address into its virtual
	   address.
	 }

\function{as\_paddr}{(i\_as \argument{id},
                      t\_vaddr \argument{virtual},
                      t\_paddr \argument{physical})}
	 {
	   This function translates a virtual address into its physical
	   address.
	 }

\function{as\_clone}{(i\_task \argument{task},
                      i\_as \argument{old},
                      i\_as* \argument{new})}
	 {
	   This function clones an address space taking care of cloning
	   all the necessary: segments, regions etc\ldots
	 }

\function{as\_reserve}{(i\_task \argument{task},
                        i\_as* \argument{id})}
	 {
	   This function reserves an address space object for the
	   task \argument{task} object.

	   The reserved address space object's identifier is returned
	   in \argument{id}.
	 }

\function{as\_release}{(i\_as \argument{id})}
	 {
	   This function just releases the address space \argument{id}.
	 }

\function{as\_get}{(i\_as \argument{id},
                    o\_as** \argument{o})}
	 {
	   This function should only be used by the as manager, the segment
	   manager and the region manager. It just returns the address space
	   object corresponding to the address space identifier \argument{id}.
	 }



\section{kaneton segment manager}

\subsection*{segment object}

\begin{verbatim}
      typedef struct
        {
          i_segment    segid;

          i_as         asid;

          t_paddr      address;
          t_psize      size;

          t_perms      perms;

          machdep_data(o_segment);
      }                o_segment;
\end{verbatim}

\subsection*{segment interface}

\function{segment\_clone}{(i\_as \argument{as},
                           i\_segment \argument{old},
                           i\_segment* \argument{new})}
	 {
	   This function clones a segment which will then belongs to
	   the address space object \argument{as}. Cloning a segment
	   consists in reserving a new segment with the
	   exact same properties and copying its content.
	 }

\function{segment\_read}{(i\_segment \argument{id},
                          t\_paddr \argument{offset},
                          void* \argument{buffer},
                          t\_psize \argument{size})}
	 {
	   This function reads \argument{size} bytes at offset
	   \argument{offset} from the segment \argument{id}.
	 }

\function{segment\_write}{(i\_segment \argument{id},
                           t\_paddr \argument{offset},
                           const void* \argument{buffer},
                           t\_psize \argument{size})}
	 {
	   This function writes the data of \argument{buffer} into the
	   segment \argument{id}.
	 }

\function{segment\_copy}{(i\_segment \argument{dst},
                          t\_paddr \argument{offd},
                          i\_segment \argument{src},
                          t\_paddr \argument{offs},
                          t\_psize \argument{size})}
	 {
	   This function copies data from segment \argument{src} to
	   segment \argument{dst}.
	 }

\function{segment\_reserve}{(i\_as \argument{as},
                             t\_psize \argument{size},
                             t\_perms \argument{perms},
                             i\_segment* \argument{id})}
	 {
	   This function reserves a segment with specified properties.
	 }

\function{segment\_release}{(i\_segment \argument{id})}
	 {
	   This function releases the segment \argument{id}.
	 }

\function{segment\_perms}{(i\_segment \argument{id},
                           t\_perms \argument{perms})}
	 {
	   This function changes the permissions of the segment \argument{id}.
	 }

\section{kaneton region manager}

\subsection*{region object}

\begin{verbatim}
      typedef struct
        {
          i_region     regid;

          i_segment    segid;

          t_vaddr      address;
          t_paddr      offset;
          t_vsize      size;
          t_opts       opts;

          machdep_data(o_region);
      }                o_region;
\end{verbatim}

\subsection*{region interface}

\function{region\_reserve}{(i\_as \argument{as},
                            i\_segment \argument{segment},
                            t\_paddr \argument{offset},
                            t\_opts \argument{opts},
                            t\_vaddr \argument{address},
                            t\_vsize \argument{size},
                            i\_region* \argument{id})}
	 {
	   This function reserves a region with specified properties.
	 }

\function{region\_release}{(i\_as \argument{as},
                            i\_region \argument{id})}
	 {
	   This function releases the region \argument{id} that belongs
	   to the address space object \argument{as}.
	 }


\newpage

\section{kaneton libmips}

\subsection*{Interface}

\functionn{t\_vaddr}{mips\_bad\_vaddr}{()}
{
  Get the virtual address that caused the fault.
}

\function{mips\_set\_tlb}{(t\_uint32 \argument{entry},
			   t\_uint32 \argument{id},
			   t\_vaddr \argument{vaddr},
			   t\_paddr \argument{paddr},
			   t\_perms  \argument{perms})}
{
  This function fills the entry of index \argument{entry} of the TLB
  with a translation of virtual address \argument{vaddr} in address
  space \argument{id} to the physical address \argument{paddr}.

  Permissions \argument{perms} are associated with the translation.
}

\function{mips\_get\_tlb}{(t\_uint32 \argument{entry},
			   t\_vaddr* \argument{vaddr},
			   t\_uint32* \argument{id})}
{
  This function get the virtual address \argument{vaddr} and the
  address space identifier \argument{id} translated by the entry
  \argument{entry} of the TLB.
}

\function{mips\_del\_tlb}{(t\_uint32 \argument{entry})}
{
  Mark the \argument{entry} of the TLB as invalid (remove it).
}

\functionn{t\_uint32}{mips\_number\_entry}{()}
{
  Get the total number of entries in the TLB.
}

\functionn{t\_uint32}{mips\_random\_entry}{()}
{
  Get the index of a random entry in the TLB that can be replaced.
}

\end{document}
