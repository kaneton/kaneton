\section*{Exercice 3 : Pagination (7 points)}

{\bf Rappel :}

Contrairement aux microprocesseurs IA-32 qui fournissent un
algorithme de remplacement automatique des entr�es TLB bas� sur un
arbre (le page-directory et les page-tables), le microprocesseur MIPS
R3000 n'offre aucune aide au programmeur en cas de TLB-miss.

Lorsque la MMU traduit une adresse et qu'aucune entr�e dans les TLB
 n'est satisfaisante, une exception est d\'eclench\'ee. C'est alors
au noyau de mettre \`a jour les TLB, ou de d�clarer une erreur fatale.

La taille des pages sur le MIPS R3000 est fix�e � 4ko.

\begin{description}
\item {\bf A - Question de cours (1.5 points)}
\begin{enumerate}
\item D\'ecrivez en quelques lignes le principe g\'en\'eral de la
traduction d'adresses en m\'emoire pagin\'ee.
 Vous d\'etaillerez notamment le fonctionnement interne de la MMU.

\begin{correction}

La pagination consiste � diviser la m�moire physique en zones de
taille �gale nomm�es pages. Sur certaines architectures, il peut y
avoir plusieurs tailles de page.

Chaque processus dispose alors d'un ensemble de pages qu'il peut
adresser, on l'appelle l'espace d'adressage (ou espace d'adressage
virtuel).

L'unit� de gestion de m�moire (MMU) du microprocesseur traduit les
adresses virtuelles en adresses physiques, en suivant certaines r�gles
d�crites par le microprocesseur lui m�me ou par le noyau.

On appelle TLB un cache de traduction d'adresse. Ce cache d�crit la
correspondance d'une page dans la m�moire virtuelle d'un espace
d'adressage avec une page en m�moire physique. Ce cache est interrog�
� chaque traduction par la MMU.

\end{correction}

\end{enumerate}
\end{description}

\begin{description}
\item {\bf B - Impl\'ementation de la pagination sur MIPS (5.5 points)}\\
\\
Dans cet exercice, vous allez impl\'ementer le manager \emph{region} de
kaneton pour un microprocesseur de la famille MIPS.

\begin{enumerate}
\item Les TLB contiennent des entr�es provenant de tous les espaces
d'adressages confondus. Pour indiquer � quel espace d'adressage
une entr�e TLB appartient, chaque entr\'ee contient un identifiant
cod\'e sur 6 bits.\\
\\
Allez-vous devoir faire des modifications dans la partie d�pendante de
\emph{as} ?\\
Si oui, lesquelles ?\\

\begin{correction}

Nous allons devoir attribuer des identifiants sur 6 bits pour chaque
espace d'adressage. Il faut donc ajouter un champ \emph{id} dans la
partie d�pendante de l'objet \emph{o\_as}.

Cette valeur sera attribu�e par le code d�pendant de \emph{as\_reserve}.

\end{correction}

\item En quoi le fait d'encoder le champ identifiant sur 6 bits peut-il
poser probl�me ?\\
\textbf{Pour la suite de l'exercice, vous ne devez pas consid\'erer le
        probl�me identifi� ci-dessus.}\\

\begin{correction}

6 bits permettent 64 valeurs, ce qui implique qu'il ne peut y avoir
plus de 64 espaces d'adressages (donc 64 t�ches) simultan�ment. Au
dela, il faut alors user d'astuces pour g�rer le TLB.

\end{correction}

\item \'Ecrivez le code du gestionnaire de TLB-miss.\\
Vous vous appuyerez sur les fonctions de la biblioth\`eque MIPS et de
l'interface kaneton (voir annexe).\\
\\
Voici Le prototype du gestionnaire de TLB-miss :

\begin{description}
\item {\bf tlb\_miss}({\em i\_as asid})\\
  \emph{asid} est l'identifiant de l'espace d'adressage dans lequel
  s'est produite l'erreur.\\
\end{description}

\begin{correction}

\begin{verbatim}
void          tlb_miss(i_as asid)
{
  o_as*       as;
  t_iterator  it;
  t_vaddr     badvaddr = mips_bad_vaddr();

  as_get(asid, &as);

  /* browse all the regions of the address space */
  set_foreach(as->regions, it)
  {
    o_region*    region;

    set_object(as->regions, it, &region);

    /* check if this region matches */
    if (region->address <= badvaddr &&
        region->address + region->size > badvaddr)
    {
      o_segment*  segment;
      t_vaddr     vaddr;
      t_paddr     paddr;

      /* get the segment mapped by the region */
      segment_get(region->segid, &segment);

      /* compute the virtual address and the physical address */
      /* to put in the TLB */
      vaddr = badvaddr & (~PAGESZ);
      paddr = (badvaddr - vaddr) + region->offset + segment->address;

      /* add the corresponding TLB entry */
      mips_set_tlb(mips_random_entry(),
                   as->machine.id, vaddr,
                   paddr, segment->perms);

      /* resume execution */
      return;
    }
  }

  /* region does not exist SIGSEGV */
}
\end{verbatim}

\end{correction}

\item \'Ecrivez le code d�pendant de la fonction {\em region\_release}. Vous
pouvez utiliser les fonctions de l'interface kaneton et celles de la
biblioth�que MIPS.\\

\begin{correction}

\begin{verbatim}
t_status       mipsr3000_region_release(i_as asid, i_region regid)
{
  o_as*       as;
  o_region*   region;
  t_uint32    i;

  as_get(asid, &as);
  region_get(asid, regid, &region);

  /* loop through the TLB */
  for (i = 0; i < mips_number_entry(); i++)
  {
    t_vaddr     vaddr;
    t_uint32    id;

    /* get the i-th entry */
    if (mips_get_tlb(i, &vaddr, &id) == STATUS_OK)
    {
      /* check if this entry correspond to a page in the */
      /* region to be removed */
      if (as->machine.id == id &&
          reg->address <= vaddr &&
          reg->address + region->size > vaddr)
      {
        /* delete the entry */
        mips_del_tlb(i);
      }
    }
  }
}
\end{verbatim}

\end{correction}

\end{enumerate}


\item {\bf C - Optimisations (bonus)}

\begin{enumerate}
\item Indiquez pourquoi votre code n'est pas suffisament performant.\\
  Proposez une (ou plusieurs) solution(s) pour gagner en
performance.

\begin{correction}

Un parcours de liste est une longue op�ration. Il faudrait utiliser
une autre structure de donn�es pour stocker et chercher les regions
plus rapidement en fonction de l'adresse virtuelle qui a caus�
l'exception.

\end{correction}

\end{enumerate}

\end{description}
