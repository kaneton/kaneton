%%
%% licence       kaneton licence
%%
%% project       kaneton
%%
%% file          /home/mycure/kaneton/view/papers/assignments/assignments.tex
%%
%% created       julien quintard   [wed dec  7 16:53:52 2005]
%% updated       julien quintard   [sun feb 26 18:13:50 2006]
%%

%
% template
%

%%
%% copyright     (c) julien quintard
%%
%% project       kaneton
%%
%% file          /home/mycure/kaneton/view/templates/paper.tex
%%
%% created       julien quintard   [sat nov 19 18:11:23 2005]
%% updated       julien quintard   [tue dec 13 01:15:46 2005]
%%

%
% class
%

\documentclass[10pt,a4wide]{article}

%
% packages
%

\usepackage[english]{babel}
\usepackage{a4wide}
\usepackage{fancyheadings}
\usepackage{multicol}
\usepackage{indentfirst}
\usepackage{graphicx}
\usepackage{color}
\usepackage{xcolor}
\usepackage{verbatim}

\pagestyle{fancy}

\setlength{\footrulewidth}{0.3pt}
\setlength{\parindent}{0.3cm}
\setlength{\parskip}{2ex plus 0.5ex minus 0.2ex}

%
% verbatim font
%

\definecolor{verbatimcolor}{rgb}{0,0.4,0}

\makeatletter
\renewcommand{\verbatim@font}
  {\ttfamily\footnotesize\color{verbatimcolor}\selectfont}
\makeatother

%
% header
%

\rfoot{\scriptsize{The kaneton microkernel project}}

\date{\scriptsize{\today}}


%
% header
%

\lhead{\scriptsize{The kaneton microkernel project assignments}}

%
% title
%

\title{The kaneton microkernel project assignments}

%
% authors
%

\author{\small{Julien Quintard},
        \small{Matthieu Bucchianeri} and
        \small{Renaud Voltz}}

%
% prototype
%

\newcommand\prototype[1]{\hspace{1.5cm}#1}

%
% document
%

\begin{document}

%
% title
%

\maketitle

\begin{center}
  \includegraphics[scale=0.8]{../../logos/kaneton.jpg}
\end{center}

\newpage

\tableofcontents

\newpage

%
% --------- text --------------------------------------------------------------
%

%
% the project
%

\section{Overview}

To get more information about the kaneton project and the kaneton
microkernel reference, please take a look at the kaneton reference
document.

All the kaneton documents are available on the official website:
\textbf{http://www.kaneton.org}.

The present \textit{Assignments} document must be used by students
willing implement the kaneton microkernel design.

%%
%% licence       kaneton licence
%%
%% project       kaneton
%%
%% file          /home/buckman/kaneton/view/papers/assignments/k0.tex
%%
%% created       matthieu bucchianeri   [tue feb  7 11:49:23 2006]
%% updated       matthieu bucchianeri   [tue feb  7 11:49:25 2006]
%%

%
% k0
%

\section{k0}

\subsection{Basic rules}

The \textbf{k0} project consists in the development of the bootstrap.

This project is very specific and will not be re-used in the future
projects.

The only  goal is to develop  a bootstrap in assembly,  loading an ELF
binary object from  floppy drive into main memory  and finally jumping
on the binary entry point.

%
% ia32
%

\subsection{IA-32 implementation}

The student will have to install and activate the protected mode before
lauching the binary object.

Moreover, the project's diffculty resides in the development of an
application first evolving in real mode, then in protected mode, all
in a single assembly file.


%%
%% licence       kaneton licence
%%
%% project       kaneton
%%
%% file          /home/mycure/kaneton/view/papers/assignments/k1.tex
%%
%% created       matthieu bucchianeri   [tue feb  7 11:49:38 2006]
%% updated       julien quintard   [sat may  6 12:15:50 2006]
%%

%
% k1
%

\chapter{k1}

The \textbf{k1} project consists in the development of the bootloader.

The bootloader, as the bootstrap, generally installs a better execution
environment for the future core execution.

Moreover, the kaneton bootloader must provide information on the current
microprocessor's architecture including the location of the pre-reserved
physical memory areas etc..

\newpage

%
% informations
%

\section{Informations}

\begin{tabular}{p{7cm}l}
Duration: & One week \\
File name: & \textit{[group]}-k1.tar.gz \\
In charge: & Julien Quintard \\
Newgroup: & kaneton-students@googlegroups.com \\
Languages: & Assembly and C \\
Students per group: & Three \\
\end{tabular}

%
% assignments
%

\section{Assignments}

In this project, the students have to develop the entire bootloader.
This means that no interface will be provided.

The only requirement is your bootloader to be compliant with the
initialization structure passed to the kaneton microkernel.

If this structure is not correctly built, then the kernel would not
be able to run.

Below are listed the bootloader's steps:

\begin{enumerate}
  \item
    The bootloader generally installs a more evolved memory addressing model,
    depending on the microprocessor's architecture.
  \item
    The bootloader relocates the stuff, if necessary, depending on the
    microprocessor's architecture. Then, the bootloader builds the
    initialization structure.

    The stuff includes:

    \begin{itemize}
      \item
	The kernel code.
      \item
	The modules.
      \item
	The pre-reserved core segments.
      \item
	The pre-reserved core regions.
      \item
	The kernel stack.
      \item
	The alloc survival area.
    \end{itemize}
  \item
    Then, the booloader prepares the kernel stack and calls the core
    providing it the complete initialization structure.
\end{enumerate}

The relocation is not really necessary but we wanted the students
to understand low-level programming and more especially programming
in a very strict environment with no fine-grain allocator provided.

The main goal of this project is to lead the students to understand the
project source organization. Indeed, a complete development environment
is provided; therefore, the students should learn to use system defines,
internal facilities, kaneton tools etc.. to develop more quickly.

Displaying the initialization structure's information is the
minimum requirement. Having an enhanced display of pre-reserved core
segments, regions and modules will be very appreciated.

The code provided must be located in the directory
\textit{kaneton/bootloader/arch/[architecture]/}.

%
% tips
%

\section{Tips}

Try to think about a very simple memory allocator delivering single
pages or contiguous areas. This allocator should help students to
allocate dynamic memory.

Let's assume that the kernel is the first given module.

%
% additional work
%

\section{Additional Work}

Moreover, students should write a tiny console driver. Try to make this
driver code generic so you can reuse it in the next steps.

%
% ia32
%

\section{Intel Architecture 32-bit}

The bootloader is highly specific to the microprocessor's architecture
since its main roles are to:

\begin{itemize}
  \item
    Install new memory addressing models.
  \item
    Build the pre-reserved core segments area.
  \item
    Build the pre-reserved core regions area.
\end{itemize}

The other tasks are independent of the architecture.

We will so detail the three previous points.

%
% memory addressing models
%

\subsection{Memory Addressing Models}

Since the kaneton microkernel can be implemented for multiple
sub-architectures, different Intel memory addressing models can be used.

We will assume in this document that the students implementing the
Intel architecture are using the \textit{ia32-virtual} sub-architecture,
meaning that they are implementing the kaneton microkernel with true
virtual memory.

Students willing implement kaneton with other specific Intel architecture
facilities should be able to adapt the assignments to their case.

So, in the case of the \textit{ia32-virtual} architecture, the bootloader
has to install two memory addressing models:

\begin{enumerate}
  \item
    First, the protected mode even if the bootstrap maybe installed it
    before.

    Indeed, the bootloader needs to rebuild the protected mode to be sure
    it is installed and to be able to maintain it.
  \item
    Then, the paging mode to enable true virtual memory.
\end{enumerate}

We advise the student to install the protected mode, then to relocate the
stuff above the 16Mb and build the initialization structure and finally to install the
virtual memory before calling the kernel.

In this way, you will handle the hardest architecture problem, the virtual
memory, at the end, before performing the very last step, calling the kernel.

%
% core segments
%

\subsection{Core Segments}

The core segments area passed to the kernel by the bootloader via
the initialization structure is used to specify the microkernel the
physical memory areas which have to be marked as pre-reserved by the core.

These memory areas are memory areas to keep from being erased like the
kernel code, the kernel stack, the initialization structure, the modules,
the core segments area, the core regions area, the survival area but also
architecture-specific areas like the ISA memory area etc..

%
% core regions
%

\subsection{Core Regions}

The core regions area must contain everything necessary to describe
the future core virtual memory state.

In other words, the core regions represent the future virtual memory
state of the core. By \textit{future} we mean once the core's region
manager is fully initialized. Indeed, the core's region manager will
use the initialization structure and more precisly the core regions to
properly rebuild the virtual memory.

So, on the Intel sub-architectures, these core regions include the mapped
video memory, the kernel code, the kernel stack, the initialization
structure, the survival area and any additional Intel specific elements
like the Global Descriptor Table etc..

Notice that it might be smart not to include the first memory page in
the core regions so null pointers dereferencing will cause page faults.

%
% layout
%

\subsection{Layout}

Below is a sample physical memory layout for the Intel sub-architectures:

\begin{center}
  \includegraphics[scale=0.7]{figures/k1-memory-layout.pdf}
\end{center}


%%
%% licence       kaneton licence
%%
%% project       kaneton
%%
%% file          /home/buckman/kaneton/view/books/assignments-k1/k1.tex
%%
%% created       matthieu bucchianeri   [tue feb  7 11:49:56 2006]
%% updated       matthieu bucchianeri   [sun feb 11 22:54:06 2007]
%%

%
% k2
%

\chapter{K2: event management}

%
% informations
%

\begin{tabular}{p{7cm}l}
Duration: & 2 weeks \\
Directory name: & kaneton/ \\
In charge: & Matthieu Bucchianeri \& Renaud Voltz\\
Mailing-list: & kaneton-students@googlegroups.com \\
Languages: & C \\
Students per group: & 2 (same groups as for K1) \\
\end{tabular}

\section{Abstract}

K2 project consists in developing XXX

\begin{enumerate}
  \item
    {\bf The event manager}\\
    XXX
  \item
    {\bf The time manager}\\
    XXX
\end{enumerate}

XXX

%
% event manager
%

\newpage

\section{\textbf{event} manager}

\begin{itemize}
  \item {\bf Overview}\\

    The event manager in kaneton is used to hook exceptions, IRQ and
    software interrupts and to redirect them to the correct ISR
    (\emph{Interrupt Service Routine}).

    There are two possibilities when redirecting events:
    \begin{itemize}
      \item
	Callbacks. Callbacks are function that can be registered for a
	given event and called when the event occurs.
      \item
	Messages. IPC can be thrown on events. \textbf{At this point
	of the project, these kind of hooks must not be implemented}.
    \end{itemize}

  \item {\bf Assignments}\\

    In K2 you will have to develop the low-level part of the event
    manager for Intel IA-32 architecture.

    Low-level includes that you will write some code in the
    machine-dependent part of the manager. But for the first time, you
    will also have to write significant part of the IA-32 library
    (\emph{libia32}).

    For example, the functions used to fill the interrupt vector table
    (called IDT on IA-32) or to initialize the PIC will be placed in
    libia32.

    When receinving events, your kernel \textbf{must} save and restore
    the execution context, for the moment only the processor's
    registers need to be saved (address space switch is for K3).

  \item {\bf Interface}\\
\function{event\_show}{(i\_event \argument{id})}
	 {
	   This function displays information on an event object.
	 }

\function{event\_dump}{(void)}
	 {
	   This function displays information on all the event objects.
	 }

\function{event\_reserve}{(i\_event \argument{id},
                           e\_event\_type \argument{type},
                           u\_event\_handler \argument{handler})}
	 {
	   This function installs an event handler.

	   \argument{type} can be :
	   \begin{itemize}
	     \item
	       \textbf{EVENT\_FUNCTION}: XXX
	     \item
	       \textbf{EVENT\_MESSAGE}: XXX
	   \end{itemize}

	   The union \argument{handler} contains XXX
	 }

\function{event\_release}{(i\_event \argument{id})}
	 {
	   This function releases an event handler.
	 }

\function{event\_get}{(i\_event \argument{id},
                       o\_event** \argument{o})}
	 {
	   This function returns in \argument{o} the event object
	   corresponding to \argument{id}.
	 }

\function{event\_init}{(void)}
	 {
	   This function initializes the event manager.
	 }

\function{event\_clean}{(void)}
	 {
	   This function cleans the event manager.
	 }
  \item {\bf {Files}}\\

    \begin{tabular}{| l | l |}
      \hline
      machine-independent & {\em kaneton/core/event/event.c}\\
      &  {\em kaneton/include/core/event.h}\\\hline
      machine-dependent & {\em kaneton/core/arch/ibm-pc.ia32-virtual/event.c}\\
      & {\em kaneton/include/arch/ibm-pc.ia32-virtual/core/event.h}\\\hline
      libarch & {\em libs/libia32/pmode/idt.c}\\
      & {\em libs/libia32/include/pmode/idt.h}\\
      & {\em libs/libia32/interrupt/*.c}\\
      & {\em libs/libia32/include/interrupt/*.h}\\\hline
    \end{tabular}
\end{itemize}


%
% tine manager
%

\newpage

\section{\textbf{time} manager}
\begin{itemize}
  \item {\bf Overview}\\

    The timer manager is helpful to create one-shot or repeated
    periodical actions. Like in the event manager, the action
    performed can be either a callback or a message (\textbf{only
    callbacks must be supported for K2}).

  \item {\bf Assignments}\\

    Your work is to develop the whole timer manager.

    The component required to measure time is called the PIT on
    IA-32. Code used to program the PIT must be placed in the libia32.

  \item {\bf Interface}\\
\function{timer\_show}{(i\_timer \argument{id})}
	 {
	   This function displays information on a timer object.
	 }

\function{timer\_dump}{(void)}
	 {
	   This function displays information on all the timers.
	 }

\function{timer\_reserve}{(t\_type \argument{type},
                           u\_timer\_handler \argument{handler},
                           t\_uint32 \argument{delay},
                           t\_uint32 \argument{repeat},
                           i\_timer* \argument{id})}
	 {
	   This function reserves a timer which will expire in
	   \argument{delay} microseconds.

	   The arguments \argument{type} and \argument{handler} works
	   the same way as for \emph{event\_reserve}.

	   The \argument{repeat} argument specifies if the timer
	   must be re-inserted once expired.
	 }

\function{timer\_release}{(i\_timer \argument{id})}
	 {
	   This function releases a timer object.
	 }

\function{timer\_delay}{(i\_timer \argument{id},
                         t\_uint32 \argument{delay})}
	 {
	   This function updates the delay.
	 }

\function{timer\_repeat}{(i\_timer \argument{id},
                          t\_uint32 \argument{repeat})}
	 {
	   This function updates the repeat property.
	 }

\function{timer\_modify}{(i\_timer \argument{id},
                          t\_uint32 \argument{delay},
                          t\_uint32 \argument{repeat})}
	 {
	   This function combines the effect of the two previous one.
	 }

\function{timer\_get}{(i\_timer \argument{id},
                       o\_timer** \argument{o})}
	 {
	   This function returns in \argument{o} the timer object
	   corresponding to \argument{id}.
	 }

\function{timer\_init}{(void)}
	 {
	   This function initializes the timer manager.
	 }

\function{timer\_clean}{(void)}
	 {
	   This function cleans the timer manager.
	 }
  \item {\bf Files}\\

    \begin{tabular}{| l | l |}
      \hline
      machine-independent & {\em kaneton/core/time/timer.c}\\
      &  {\em kaneton/include/core/timer.h}\\\hline
      machine-dependent & {\em kaneton/core/arch/ibm-pc.ia32-virtual/timer.c}\\
      & {\em kaneton/include/arch/ibm-pc.ia32-virtual/core/timer.h}\\\hline
      libarch & {\em libs/libia32/time/timer.c}\\
      &  {\em libs/libia32/include/time/pit.h}\\\hline
    \end{tabular}

\end{itemize}

%
% advanced topics
%

\newpage

\section{Bonuses}

kaneton microkernel is first of all a pedagogical project which do not
aims at being optimized. That is why, when nothing is specified, you
always will implement the simplest algorithms.\\
\\
Nevertheless, we will always encourage students who want to write
additional bonuses, as far as they respect the following rules:

\begin{enumerate}
  \item Bonuses will be evaluated only if a basic implementation is
  actually working.
  \item Bonuses must be either picked from the following list, or
  accepted by the kaneton team.\\
\end{enumerate}

Bonuses ideas:
\begin{itemize}
\item More accurate timer using the APIC
\item Managing priority among interrupts
\item Managing nested interrupts
\end{itemize}


%%
%% licence       kaneton licence
%%
%% project       kaneton
%%
%% file          /home/buckman/kaneton/view/books/assignments-k2/k2.tex
%%
%% created       matthieu bucchianeri   [tue feb  7 11:49:56 2006]
%% updated       matthieu bucchianeri   [wed mar 14 23:46:36 2007]
%%

%
% k2
%

\chapter{K3: tasks \& scheduling}

%
% informations
%

\begin{tabular}{p{7cm}l}
Duration: & 2 weeks \\
Directory name: & kaneton/ \\
In charge: & Matthieu Bucchianeri \& Renaud Voltz\\
Mailing-list: & kaneton-students@googlegroups.com \\
Languages: & C \\
Students per group: & 2 (same groups as for previous stages) \\
\end{tabular}

\section{Abstract}

K2 project consists in developing XXX
The concerned managers are:

\begin{enumerate}
  \item
    {\bf The task manager}\\
    XXX
  \item
    {\bf The thread manager}\\
    XXX
  \item
    {\bf The sched manager}\\
    XXX
\end{enumerate}

XXX

\textbf{Important}: do no forget the FIXME in the machine-dependent
part of the as manager, near the end of the file.

Remember that you can add code even if there is no FIXME.


%
% task manager
%

\newpage

\section{\textbf{task} manager}

\begin{itemize}
  \item {\bf Overview}\\

    The \textbf{task manager} provides a complete interface for the
    task objects manipulation.

    A \textbf{task object} \textit{o\_task} describes a complete
    execution entity. Nevertheless, a task object is never scheduled
    since a task is not an active entity (threads are).

    A task is composed of an address space \textit{o\_as} and threads
    \textit{o\_thread}.

  \item {\bf Assignments}\\

    No assignments.

  \item {\bf Interface}\\

\function{task\_show}{(i\_task \argument{id})}
	 {
	   This function displays information on the task \argument{id}.
	 }

\function{task\_dump}{(void)}
	 {
	   This function displays information on every task.
	 }

\function{task\_clone}{(i\_task \argument{old},
                        i\_task* \argument{new})}
	 {
	   This function clones a task.

	   This function must take care of cloning everything necessary
	   including the address space and the threads.
	 }

\function{task\_reserve}{(t\_class \argument{class},
                          t\_behav \argument{behav},
                          t\_prior \argument{prior},
                          i\_task* \argument{id})}
	 {
	   This function reserves a task object with the given
	   properties: \argument{class}, \argument{behav} and
	   \argument{prior}.

	   Note that once reserved, the task is marked as stopped.

	   XXX
	 }

\function{task\_release}{(i\_task \argument{id})}
	 {
	   This function releases the task object \argument{id}.
	 }

\function{task\_priority}{(i\_task \argument{id},
                           t\_prior \argument{prior})}
	 {
	   This function updates the task's priority to \argument{prior}.
	 }

\function{task\_state}{(i\_task \argument{id},
                        t\_state \argument{sched})}
	 {
	   XXX
	 }

\function{task\_wait}{(i\_task \argument{id},
                       t\_opts \argument{opts},
                       t\_wait* \argument{wait})}
	 {
	   This function waits for state change in one or more tasks
	   depending on the options \argument{opts}.

	   \notice{This feature is not yet implemented.}
	 }

\function{task\_get}{(i\_task \argument{id},
                      o\_task** \argument{o})}
	 {
	   This function returns in \argument{o} the task object corresponding
	   to \argument{id}.
	 }

\function{task\_init}{(void)}
	 {
	   This function initializes the task manager.
	 }

\function{task\_clean}{(void)}
	 {
	   This function cleans the task manager.
	 }

  \item {\bf {Files}}\\

    \begin{tabular}{| l | l |}
      \hline
      machine-independent & {\em kaneton/core/task/task.c}\\
      &  {\em kaneton/include/core/task.h}\\\hline
      machine-dependent & {\em kaneton/core/arch/ibm-pc.ia32-virtual/task.c}\\
      & {\em kaneton/include/arch/ibm-pc.ia32-virtual/core/task.h}\\\hline
      libarch & {\em libs/libia32/task/*.c}\\
      & {\em libs/libia32/include/task/*.h}\\\hline
    \end{tabular}
\end{itemize}


%
% thread manager
%

\newpage

\section{\textbf{thread} manager}
\begin{itemize}
  \item {\bf Overview}\\

    The \textbf{thread manager} manages the real active execution
    context called threads.

    A \textbf{thread object} is an active entity which describes the
    current state of an execution context including the
    \textit{program counter}, the \textit{stack pointer} etc..

    Indeed, a thread is composed of a current \textit{program counter}
    which keeps the next instruction address to execute and the
    \textit{stack pointer} which keeps the stack address. With these
    two characteristics, a thread can be described.

    Needless to say, some additional architecture-dependent properties
    are required to fully describe a thread context.

  \item {\bf Assignments}\\

    XXX

  \item {\bf Interface}\\

\function{thread\_show}{(i\_thread \argument{id})}
	 {
	   This function displays information on the thread \argument{id}.
	 }

\function{thread\_dump}{(void)}
	 {
	   This function displays information on all the threads.
	 }

\function{thread\_give}{(i\_task \argument{task},
                         i\_thread \argument{id})}
	 {
	   This function gives the thread object \argument{id} to the
	   task \argument{task}.
	 }

\function{thread\_clone}{(i\_task \argument{task},
                          i\_thread \argument{old},
                          i\_thread* \argument{new})}
	 {
	   This function clones the task \argument{old} into a new one
	   \argument{new}.

	   This new thread, having the exact same properties as \argument{old},
	   will be held by the task \argument{task}.
	 }

\function{thread\_reserve}{(i\_task \argument{task},
                            t\_prior \argument{prior},
                            i\_thread* \argument{id})}
	 {
	   This function reserves a thread for the task \argument{task}
	   given the default thread priority \argument{prior}.

	   Note that once reserved, the thread is marked as stopped.
	 }

\function{thread\_release}{(i\_thread \argument{id})}
	 {
	   This function releases the thread object \argument{id}.
	 }

\function{thread\_priority}{(i\_thread \argument{id},
                             t\_prior \argument{prior})}
	 {
	   This function updates the current thread priority
	   to \argument{prior}.
	 }

\function{thread\_state}{(i\_thread \argument{id},
                          t\_state \argument{sched})}
	 {
	   XXX
	 }

\function{thread\_wait}{(i\_thread \argument{id},
                         t\_opts \argument{opts},
                         t\_wait* \argument{wait})}
	 {
	   This function acts like the function \textbf{task\_wait}().

	   This function waits for the thread's state to change depending on
	   the options \argument{opts}.

	   \notice{This feature is not yet implemented.}
	 }

\function{thread\_get}{(i\_thread \argument{id},
                        o\_thread** \argument{o})}
	 {
	   This function returns in \argument{o} the thread object
	   corresponding to \argument{id}.
	 }

\function{thread\_flush}{(i\_task \argument{task})}
	 {
	   This function removes every thread that belongs to the
	   task object \argument{task}.
	 }

\function{thread\_stack}{(i\_thread \argument{id},
                          t\_stack \argument{size})}
	 {
	   This function allocates a stack of \argument{size} bytes
	   for the thread \argument{id}.

	   XXX
	 }

\function{thread\_load}{(i\_thread \argument{id},
                         t\_thread\_context \argument{context})}
	 {
	   This function loads a new execution context in the thread
	   object \argument{id}.

	   A thread execution context \textit{t\_thread\_context}
	   only contains the \textit{program counter} and the
	   \textit{stack pointer}.

	   XXX
	 }

\function{thread\_store}{(i\_thread \argument{id},
                          t\_thread\_context* \argument{context})}
	 {
	   This function stores in \argument{context} the current
	   thread execution context of the thread object \argument{id}.

	   XXX
	 }

\function{thread\_init}{(void)}
	 {
	   This function initializes the thread manager.
	 }

\function{thread\_clean}{(void)}
	 {
	   This function cleans the thread manager.
	 }

  \item {\bf Files}\\

    \begin{tabular}{| l | l |}
      \hline
      machine-independent & {\em kaneton/core/thread/thread.c}\\
      &  {\em kaneton/include/core/thread.h}\\\hline
      machine-dependent & {\em kaneton/core/arch/ibm-pc.ia32-virtual/thread.c}\\
      & {\em kaneton/include/arch/ibm-pc.ia32-virtual/core/thread.h}\\\hline
      libarch & {\em libs/libia32/task/*.c}\\
      &  {\em libs/libia32/include/task/*.h}\\\hline
    \end{tabular}

\end{itemize}


%
% sched manager
%

\newpage

\section{\textbf{sched} manager}
\begin{itemize}
  \item {\bf Overview}\\

    XXX

  \item {\bf Assignments}\\

    XXX

  \item {\bf Interface}\\

\function{sched\_dump}{(void)}
	 {
	   This function displays the scheduler state.
	 }

\function{sched\_quantum}{(t\_quantum \argument{quantum})}
	 {
	   This function sets the scheduler quantum to \argument{quantum}.

	   XXX
	 }

\function{sched\_yield}{(i\_cpu \argument{cpuid})}
	 {
	   This function permits the current task to relinquish
	   the processor voluntarily.

	   Don't care about the argument \argument{cpuid}.
	 }

\function{sched\_add}{(i\_thread \argument{thread})}
	 {
	   This function adds a runnable thread to the scheduler.
	 }

\function{sched\_remove}{(i\_thread \argument{thread})}
	 {
	   This function remove a thread from the scheduler.
	 }

\function{sched\_update}{(i\_thread \argument{thread})}
	 {
	   This function asks the scheduler to update the thread
	   \argument{thread} in its internal data structures since
	   for example the thread's priority just changed.
	 }

\function{sched\_current}{(i\_thread* \argument{thread})}
	 {
	   This function returns in \argument{thread} the identifier
	   of the thread currently executed.
	 }

\function{sched\_switch}{(void)}
	 {
	   This function just schedules a new elected thread.

	   \textbf{Note}: the machine-dependent code of this function
	   takes an additional parameter.

	   \function{ia32\_sched\_switch}{(i\_thread \argument{elected})}
		    {
		      The argument \argument{elected} is the new
		      thread to run.
		    }
	 }

\function{sched\_init}{(void)}
	 {
	   This function initializes the scheduler.
	 }

\function{sched\_clean}{(void)}
	 {
	   This function cleans the scheduler.
	 }

  \item {\bf Files}\\

    \begin{tabular}{| l | l |}
      \hline
      machine-independent & {\em kaneton/core/sched/sched.c}\\
      &  {\em kaneton/include/core/sched.h}\\\hline
      machine-dependent & {\em kaneton/core/arch/ibm-pc.ia32-virtual/sched.c}\\
      & {\em kaneton/include/arch/ibm-pc.ia32-virtual/core/sched.h}\\\hline
      libarch & {\em libs/libia32/task/*.c}\\
      &  {\em libs/libia32/include/task/*.h}\\\hline
    \end{tabular}

\end{itemize}

%
% advanced topics
%

\newpage

\section{Bonuses}

kaneton microkernel is first of all a pedagogical project which do not
aims at being optimized. That is why, when nothing is specified, you
always will implement the simplest algorithms.\\
\\
Nevertheless, we will always encourage students who want to write
additional bonuses, as far as they respect the following rules:

\begin{enumerate}
  \item Bonuses will be evaluated only if a basic implementation is
  actually working.
  \item Bonuses must be either picked from the following list, or
  accepted by the kaneton team.\\
\end{enumerate}

Bonuses ideas:
\begin{itemize}
\item Implement a better scheduler. Feel free to implement any other
algorithm. You can try the multi-level feedback queue explained during
the lesson
\end{itemize}



%%
%% licence       kaneton licence
%%
%% project       kaneton
%%
%% file          view/books/assignments/2007/k4/k4.tex
%%
%% created	 pidancet julian	[tue may29 16:34:56 2006]
%% updated	 pidancet julian	[tue may29 16:34:56 2006]
%%

%
% k2
%

\chapter{K4: IPC Management}

%
% informations
%

\begin{tabular}{p{7cm}l}
Duration: & 2 weeks \\
Directory name: & kaneton/ \\
In charge: Julian Pidancet \& Matthieu Bucchianeri \& Renaud Voltz\\
Mailing-list: & kaneton-students@googlegroups.com \\
Languages: & C, assembly \\
Students per group: & 2 (same groups as for K3) \\
\end{tabular}

\section{Abstract}

K4 project consists in developing the last part of the microkernel : the
IPC (Inter-process Communication). As you know, in a microkernel, contrary
to a monolithic kernel, every kernel services are running independently
from the core, in separate userspace privileged tasks. Though, in order
to keep coherency, tasks need to interact : sending, receiving messages,
and also synchronizing each others. The IPC model chosen for Kaneton is
``message passing''

The concerned parts are:

\begin{enumerate}
  \item
    {\bf The message manager, machine-independant part}\\
    Inter-process messaging primitives. Send and receive functions,
    synchronous and asynchronous.
  \item
    {\bf The message manager, IA-32 part}\\
    Handling soft-interrupt (syscall to the messaging primitives).
\end{enumerate}

\newpage

\section{message manager, \textbf{machine-independant part}}

\begin{itemize}
  \item {\bf Assignments}\\

  In the machine-independant part of the message manager, you will have to
  implement the messaging primitives for sending and receiving messages.
  These primitives will have to be accessible by syscall, to permit
  running tasks to send and receive messages from userspace.

  \item {\bf Interface}\\

\function{message\_async\_send}{(i\_task \argument{sender},
				 i\_node \argument{dest},
				 t\_tag \argument{tag},
				 t\_vaddr \argument{data},
				 t\_size \argument{size})}
	 {

	 }


  \item {\bf {Files}}\\

    \begin{tabular}{| l | l |}
      \hline
      machine-independent & {\em kaneton/core/message/message.c}\\
      &  {\em kaneton/include/core/message.h}\\\hline
    \end{tabular}
\end{itemize}

\newpage

\section{message manager \textbf{IA-32 part}}
\begin{itemize}
  \item {\bf Assignments}\\

    The goal of this part is to permit software to send and receive
    data from userspace.\\

    Your work is splitted in two parts. First part is to write syscall
    handlers for the 4 given primitives. Second part is to write interface
    sending and receiving functions that will provide task-side syscalls.

  \item {\bf {Files}}\\

    \begin{tabular}{| l | l |}
      \hline
      machine-dependent & {\em kaneton/core/arch/ibm-pc.ia32-virtual/message.c}\\
      & {\em kaneton/include/arch/ibm-pc.ia32-virtual/core/message.h}\\\hline
    \end{tabular}

\end{itemize}

%
% advanced topics
%

\newpage

\section{Appendix}

\textbf{Example of event\_reserve}

\begin{verbatim}
void          ia32_pf_handler(t_id         id,
                              t_uint32     error_code)
{
  t_uint32    addr;

  SCR2(addr);
  printf("#PF @ %p\n", addr);

  while (1)
    ;
}

...

event_reserve(14, EVENT_FUNCTION, EVENT_HANDLER(ia32_pf_handler));
\end{verbatim}

\textbf{Example of timer\_reserve}

\begin{verbatim}
void          sched_switch(void)
{
  // FIXME: chiche just stole this code
}

...

timer_reserve(EVENT_FUNCTION, TIMER_HANDLER(sched_switch),
              sched->quantum, TIMER_REPEAT_ENABLE,
              &sched->machdep.timer);
\end{verbatim}


\end{document}
