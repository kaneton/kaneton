%%
%% licence       kaneton licence
%%
%% project       kaneton
%%
%% file          /home/mycure/kaneton/view/papers/assignments/assignments.tex
%%
%% created       julien quintard   [wed dec  7 16:53:52 2005]
%% updated       julien quintard   [wed dec  7 17:20:14 2005]
%%

%
% template
%

%%
%% copyright     (c) julien quintard
%%
%% project       kaneton
%%
%% file          /home/mycure/kaneton/view/templates/paper.tex
%%
%% created       julien quintard   [sat nov 19 18:11:23 2005]
%% updated       julien quintard   [tue dec 13 01:15:46 2005]
%%

%
% class
%

\documentclass[10pt,a4wide]{article}

%
% packages
%

\usepackage[english]{babel}
\usepackage{a4wide}
\usepackage{fancyheadings}
\usepackage{multicol}
\usepackage{indentfirst}
\usepackage{graphicx}
\usepackage{color}
\usepackage{xcolor}
\usepackage{verbatim}

\pagestyle{fancy}

\setlength{\footrulewidth}{0.3pt}
\setlength{\parindent}{0.3cm}
\setlength{\parskip}{2ex plus 0.5ex minus 0.2ex}

%
% verbatim font
%

\definecolor{verbatimcolor}{rgb}{0,0.4,0}

\makeatletter
\renewcommand{\verbatim@font}
  {\ttfamily\footnotesize\color{verbatimcolor}\selectfont}
\makeatother

%
% header
%

\rfoot{\scriptsize{The kaneton microkernel project}}

\date{\scriptsize{\today}}


%
% header
%

\lhead{\scriptsize{The kaneton microkernel project assignments}}

%
% title
%

\title{The kaneton microkernel project assignments}

%
% authors
%

\author{\small{Julien Quintard}}

%
% document
%

\begin{document}

%
% title
%

\maketitle

%
% --------- text --------------------------------------------------------------
%

%
% the project
%

\section{The Project}

%
% k0
%

\section{k0}

%
% k1
%

\section{k1}

The \textbf{k0} project consists in the development of the bootloader.

This bootloader just relocates the stuff needed by the futur kernel
execution.

The relocation is not really necessary but we wanted the students
to understand low-level programming and more especially programming
in a very strict environment with no fine-grained allocator provided.

So in this project, the student has to write the entire code of the
bootloader. The only requirement is to be compliant with the structure
passed to the kernel.

This structure called \textbf{t\_init} is defined in the
file: \textit{core/include/kaneton/init.h}.

The code provided must be located in the directory
\textit{core/bootloader/arch/[architecture]/}.

%
% k2
%

%
% k3
%

%
% k4
%

%
% k5
%

%
% kn
%



\end{document}
