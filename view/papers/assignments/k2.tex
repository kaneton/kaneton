%%
%% licence       kaneton licence
%%
%% project       kaneton
%%
%% file          /home/mycure/kaneton/view/papers/assignments/k2.tex
%%
%% created       matthieu bucchianeri   [tue feb  7 11:49:56 2006]
%% updated       julien quintard   [mon apr  3 18:47:09 2006]
%%

%
% k2
%

\section{k2}

%
% informations
%

\subsection{Informations}

\begin{tabular}{p{7cm}l}
Deadline: & XXX, 23h42 \\
Duration: & Two weeks \\
File name: & k2.tar.gz \\
In charge of: & Julien Quintard - \small{quinta\_j@epita.fr} \\
Newsgroups: & epita.cours.kaneton \\
Languages: & Assembly and C \\
Architectures: & Intel Architecture 32-bit \\
Students per group: & Three \\
\end{tabular}

%
% overview
%

\subsection{Overview}

The \textbf{k2} project consists in the development of parts of
the kaneton core including:

\begin{itemize}
  \item
    The id manager.
  \item
    The set manager.
  \item
    The address space manager.
  \item
    The segment manager.
\end{itemize}

The id manager is very important because it is very simple and will
introduce you how managers are implemented in kaneton. Every student
should be able to understand it before starting other managers.

Moreover, many parts of the kernel will lead student to use the sets,
so the sets interface must be clearly understood.

The as and segment managers introduce calls to machine-dependent code, this
mecanism must be well understood by every students.

For the two first managers, you have to follow precisely the bevavior
as described in the kaneton reference documentation. For the memory
management, you must be totally compliant with the given interface and
the described behavior, but algorithm, especially for the machine-dependent
code, are design-free.

%
% assignments
%

\subsection{Assignments}

In this section, we will detail each manager.

%
% id manager
%

\subsubsection{id manager}

XXX

%
% set manager
%

\subsubsection{set manager}

XXX

The manager's code is given, see \textit{core/kaneton/set/set.c} and
\textit{core/include/kaneton/set.h}. Take a look at the information
section in the header of the manager's code: each steps of the creation
of a set is described.

The student will have to write the entire code for the \textbf{array}
and \textbf{ll} implementations.

The functions below are explained independently of the data structure
which is noted \textbf{<I>}.

XXX

%
% address space manager
%

\subsubsection{address space manager}

XXX

	    ---
	    For \textbf{k2}, you don't have to deal with tasks, so you must
	    ignore the \textit{task} arguments.
	    ---

%
% segment manager.
%

\subsubsection{segment manager}

XXX


	   ---
	    This function concerns \textbf{k1} pre-reserved segments.
	    All of these must be passed to the inject function at kernel
	    boot time.
	    ---

%
% advanced topics
%

\subsubsection{Advanced Topics}

Students can implement better allocation algorithms for the segment manager,
like a Buddy System etc..

%
% ia32
%

\subsection{Intel Architecture 32-bit Implementation}

Students will have to develop the architecture-dependent part of the
address space manager and segment manager.

For the segment manager, the only work of the architecture-dependent source
code is to get control over the protected mode so to rebuild the Global
Descriptor Table for the kaneton needs.

For the address space manager, the only work the architecture do is to
manager Page Directories, allocating and releasing them each time an
address space object is created, destroyed, modified etc..

The corresponding source files are:

\begin{itemize}
  \item
    \textit{core/kaneton/arch/[architecture]/segment.c}.
  \item
    \textit{core/kaneton/arch/[architecture]/as.c}.
\end{itemize}

The corresponding include files are:

\begin{itemize}
  \item
    \textit{core/include/arch/[architecture]/kaneton/segment.h}.
  \item
    \textit{core/include/arch/[architecture]/kaneton/as.h}.
\end{itemize}

Students must implement all dependent functions they need. Remember that
functions in \textit{libia32} should be reused.

Take a look at the kaneton reference paper for more information about
machine-dependent calls.

\subsubsection{Tips}

\begin{itemize}
  \item
    Segment sizes are aligned on page size. Assuming you are not using
    four megabytes pages, this value will be 4096.

    Remember to use the \textbf{PAGESZ} define.
  \item
    You must create two segments, one for code, the other for data
    per task class: core, driver, service, and user program.

    Each one must have the appropriate privilege level and permissions.
\end{itemize}

%
% main procedure
%

\subsection{Main Procedure}

As the first managers are now working, it is time to write the kernel
main procedure.

This  one is  contained in \textit{core/kaneton/kaneton.c}. Students
must first initialize the \textbf{init} variable from the one given by
the bootloader.

Then, they  must initialize the differents manager. Do not forget that
it is better to initialize the set manager before as and segment,
since these managers rely on sets!

Do not forget to initialize the kernel standard library.

You must initialize the \textbf{printf} function passing pointers to the
console manager functions to \textbf{printf\_init()}.

In addition, you must initialize the kernel \textbf{malloc} function which
is used for early initial stages, until as, segment and region are running.
You must use the \textbf{alloc\_init()} function, passing the survival area.

To finish, you  have to create the kernel address space and to inject
the pre-allocated segments in.

Finally, all managers must be destroyed calling the clean() functions for
each one.

You might place code for your tests between these two phases. Do not
forget to disable it for the final tarball.

XXX set\_ll, segment
