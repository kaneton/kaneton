%%
%% licence       kaneton licence
%%
%% project       kaneton
%%
%% file          /home/buckman/kaneton/view/papers/assignments/k2.tex
%%
%% created       matthieu bucchianeri   [tue feb  7 11:49:56 2006]
%% updated       matthieu bucchianeri   [wed feb 15 23:33:37 2006]
%%

%
% k2
%

\section{k2}

The \textbf{k2} project consists in the development of parts of
the microkernel:

\begin{itemize}
  \item
    The id manager
  \item
    The set manager
  \item
    The address space manager
  \item
    The segment manager
  \item
    A  simple  console manager  (should  be  copy  \& paste  from  your
    bootloader's code)
\end{itemize}

The id  manager is very important  because it is very  simple and will
introduce you how managers  are implemented in kaneton. Every students
should be able  to understand it before starting  other managers (sets
and memory).

as  and  segment  introduce  calls  to  machine-dependent  code,  this
mecanism  must be well  understood by  every students.  Moreover, many
parts of  the kernel will  lead student to  use the sets, so  the sets
interface must be clearly understood.

For the two first managers,  you have to follow precisely the bevavior
as described  in the kaneton  reference documentation. For  the memory
management, you must be totally compliant with the given interface and
the   described   behavior,   but   algorithm  (especially   for   the
machine-dependent code) are design-free.

To  finish, notice  that  the current  implementation  of the  console
driver is not clean and will be changed in future kaneton projects.

%
% id manager
%

\subsection{id manager}

\subsubsection{Description}

The id manager is used to generate identifiers from an identifier object.

Once the  identifier object initialised  -- you got  an \textbf{o\_id}
structure -- the identifiers can be generated.

Remember that identifiers only consist in a 64-bits integer.

\subsubsection{Students work}

Let's take a look at the id manager functions you will implement:

\prototype{t\_error \textbf{id\_show}(o\_id* \textbf{o});}

This function just displays the id object's state.

\prototype{t\_error \textbf{id\_clone}(o\_id* \textbf{o},
                                       t\_id \textbf{old},
                                       t\_id* \textbf{new});}

This function duplicates an id object using the \textbf{o} identifier
object.

\prototype{t\_error \textbf{id\_reserve}(o\_id* \textbf{o},
                                         t\_id* \textbf{i});}

This function reserves an identifier \textbf{i} using the identifier
object \textbf{o}.

\prototype{t\_error \textbf{id\_release}(o\_id* \textbf{o},
                                         t\_id \textbf{i});}

This function releases an identifier.

\prototype{t\_error \textbf{id\_build}(o\_id* \textbf{o});}

This function initialises an identifier object.

\prototype{t\_error \textbf{id\_destroy}(o\_id* \textbf{o});}

This function destroys an identifier object.

\prototype{t\_error \textbf{id\_init}(void);}

This function just initialises the id manager.

\prototype{t\_error \textbf{id\_clean}(void);}

This function just cleans the id manager.

Do not forget to take a look  at the header file for the id manager in
\textit{core/include/kaneton/id.h}, to  learn about the \textbf{o\_id}
structure.

Like  every  other  managers  in kaneton,  the  \textbf{id\_init}  and
\textbf{id\_clean}  functions  must  be  called  by  the  kernel  main
routine on startup and shutdown.

%
% set manager
%

\subsection{set manager}

\subsubsection{Description}

The set  manager is used to  store data. Every kernel  managers use it
instead of building data structures by their own.

The set mecanism is divided in  two parts: the set manager and the set
implementations.

While the set implementations are specific to each type of set (array,
linked-list\ldots), the set manager code  is generic and work with all
sets.

\subsubsection{Students work}

The  manager code  is given,  see  \textit{core/kaneton/set/set.c} and
\textit{core/include/kaneton/set.h}.  Take a  look at  the information
section in the header of the  manager code: each steps of the creation
of a set is described.

The student will have to  write the entire code for the \textbf{array}
and \textbf{ll} implementations.

The functions below are explained independently of the data structure.

\prototype{t\_error \textbf{set\_type\_[x]}(t\_setid \textbf{setid});}

This function just returns an error if the set corresponding to
\textbf{setid} is not of the type \textbf{[x]}.

\prototype{t\_error \textbf{set\_show\_[x]}(t\_setid \textbf{setid});}

This function displays a entire set.

\prototype{t\_error \textbf{set\_head\_[x]}(t\_setid \textbf{setid},
                                            t\_iterator* \textbf{iterator});}

This function returns an iterator on the first element of the set.

\prototype{t\_error \textbf{set\_tail\_[x]}(t\_setid \textbf{setid},
                                            t\_iterator* \textbf{iterator});}

This function returns an iterator on the last element of the set.

\prototype{t\_error \textbf{set\_prev\_[x]}(t\_setid \textbf{setid},
                                            t\_iterator \textbf{current},
                                            t\_iterator* \textbf{previous});}

This function returns an iterator on the previous element of the set.

\prototype{t\_error \textbf{set\_next\_[x]}(t\_setid \textbf{setid},
                                            t\_iterator \textbf{current},
                                            t\_iterator* \textbf{previous});}

This function returns an iterator on the next element of the set.

\prototype{t\_error \textbf{set\_insert\_head\_[x]}(t\_setid \textbf{setid},
                                                    void* \textbf{data});}

This function inserts an object at the head of the set.

\prototype{t\_error \textbf{set\_insert\_tail\_[x]}(t\_setid \textbf{setid},
                                                    void* \textbf{data});}

This function inserts an object at the tail of the set.

\prototype{t\_error \textbf{set\_insert\_before\_[x]}(t\_setid \textbf{setid},
                                                      t\_iterator \textbf{iterator},
                                                      void* \textbf{data});}

This function inserts an object before the one specified by the iterator.

\prototype{t\_error \textbf{set\_insert\_after\_[x]}(t\_setid \textbf{setid},
                                                     t\_iterator \textbf{iterator},
                                                     void* \textbf{data});}

This function inserts an object after the one specified by the iterator.

\prototype{t\_error \textbf{set\_add\_[x]}(t\_setid \textbf{setid},
                                           void* \textbf{data});}

This function adds an object in the set. See at the end of the section
for more details.

\prototype{t\_error \textbf{set\_remove\_[x]}(t\_setid \textbf{setid},
                                              t\_id \textbf{id});}

This function removes an object identified by \textbf{id} from the
set \textbf{setid}.

\prototype{t\_error \textbf{set\_delete\_[x]}(t\_setid \textbf{setid},
                                              t\_iterator \textbf{iterator});}

This function deletes the object corresponding to the iterator.

\prototype{t\_error \textbf{set\_flush\_[x]}(t\_setid \textbf{setid});}

This function removes every object stored in the set.

\prototype{t\_error \textbf{set\_locate\_[x]}(t\_setid \textbf{setid},
                                              t\_id \textbf{id},
                                              t\_iterator* \textbf{iterator});}

This function returns an iterator on the element corresponding to the
identifier \textbf{id}.

\prototype{t\_error \textbf{set\_object\_[x]}(t\_setid \textbf{setid},
                                              t\_iterator \textbf{iterator},
                                              void** \textbf{data});}

This function returns the object's data corresponding to the iterator.

\prototype{t\_error \textbf{set\_push\_array}(t\_setid \textbf{setid},
                                             void* \textbf{data});}

This function adds an object to a FIFO or LIFO structure.

\prototype{t\_error \textbf{set\_pop\_array}(t\_setid \textbf{setid});}

This function removes the next object of a FIFO or LIFO structure.

\prototype{t\_error \textbf{set\_pick\_array}(t\_setid \textbf{setid},
                                             void** \textbf{data});}

This  function returns the  next object  of a  FIFO or  LIFO structure
without deleting it.

\prototype{t\_error \textbf{set\_release\_[x]}(t\_setid \textbf{setid});}

This function releases a set.

The two last functions are  described more specificaly since they have
different prototypes.

\prototype{t\_error \textbf{set\_reserve\_ll}(t\_opts \textbf{opts},
                                              t\_size \textbf{datasz},
                                              t\_setid* \textbf{setid});}

This function reserves a ll set with options \textbf{opts} which will contain
object of \textbf{datasz} size.

\prototype{t\_error \textbf{set\_reserve\_array}(t\_opts \textbf{opts},
                                                 t\_setsz \textbf{initsz},
                                                 t\_size \textbf{datasz},
                                                 t\_setid* \textbf{setid});}

This function reserves an array set with options \textbf{opts}. This
set will contain objects of \textbf{datasz} size and will initialy
be composed of \textbf{initsz} unused elements.

You can notice  that some operations does not make  sense on some data
structures. For  example, \textbf{push} or \textbf{pop}  will never be
used on arrays or lists, but  only on stacks or pipes. In these cases,
the function must be present but will only return an error.

Most   widely,    some   operations   have   no    sense   with   some
options. Inserting at the beginning  of a sorted array must produce an
error since it can cancel the sorting property.

\subsubsection{Constraints}

To  keep  sets behavior  unified,  here  are  some rules  applying  to
\textbf{array} and \textbf{ll}:

\begin{itemize}

\item \textbf{set\_add\_array} must insert the item at the first free place.

\item All deletion functions on arrays must mark the released place as
free. The only case shifting is made is when the \textbf{organise} option
is set.

\item When inserting an object in a linked-list with the \textbf{add}
operation, the item must be placed in the head of the list.

\item As double linked-list are faster than simple one, you have to
implement this kind of lists.

\end{itemize}

Here are the options your sets must recognize:

\begin{itemize}
\item SET\_OPT\_CONTAINER
\item SET\_OPT\_ALLOC and SET\_OPT\_FREE
\item SET\_OPT\_SORT
\item SET\_OPT\_ORGANISE (for arrays only)
\end{itemize}

Do  not  forget to  uncomment  the  step  5 of  \textbf{set\_init}  in
\textit{core/kaneton/set/set.c}, which creates the set container.

%
% address space manager
%

\subsection{address space manager}

\subsubsection{Description}

The address space manager just  manages the address spaces. An address
space is a container describing the addressable memory.

An address space is composed of a set of segments and a set of regions.

\subsubsection{Students work}

Students have to write the whole  code of this manager. No need to say
that intensive use of  sets (previously described) is necessary.  This
manager will introduce you how to program with the set manager.

\prototype{t\_error \textbf{as\_show}(t\_asid \textbf{asid});}

This function shows a precise address space displaying information
on it.

\prototype{t\_error \textbf{as\_dump}(void);}

This function dumps all the address space managed by the address
space manager.

\prototype{t\_error \textbf{as\_give}(t\_asid \textbf{asid},
                                      t\_tskid \textbf{tskid});}

This function gives an address space to another task.

\prototype{t\_error \textbf{as\_clone}(t\_tskid \textbf{tskid},
                                       t\_asid \textbf{old},
                                       t\_asid* \textbf{new});}

This function clones an address space taking care of cloning everything
necessary.

\prototype{t\_error \textbf{as\_reserve}(t\_tskid \textbf{tskid},
                                         t\_asid* \textbf{asid});}

This function reserves an address space object for the task \textbf{tskid}.

\prototype{t\_error \textbf{as\_release}(t\_asid \textbf{asid});}

This function just releases an address space.

\prototype{t\_error \textbf{as\_get}(t\_asid \textbf{asid},
                                    o\_as** \textbf{o});}

This function should only be used by the segment and region managers.

This function just returns the address space object corresponding to
the address space identifier.

\prototype{t\_error \textbf{as\_init}(void);}

This function initialises the address space manager.

\prototype{t\_error \textbf{as\_clean}(void);}

This function cleans the address space manager.

%
% ia32
%

\subsubsection{IA-32 implementation}

Students will  have to develop the architecture-dependent  part of the
address space manager.

The corresponding file is \textit{core/kaneton/arch/\textbf{arch-name}/as.c}.

Students have  to think about what  is an address space  from the ia32
point of view.  Next, they will fill the \textbf{do\_as} structure in
\textit{core/include/arch/\textbf{arch-name}/kaneton/as.h}

To  finish,   they  must   implement  all  dependent   functions  they
need. Remember that functions in \textit{libia32} should be reused.

Take a look at the  kaneton reference paper for more information about
machine-dependent calls.

%
% segment manager.
%

\subsection{segment manager}

\subsubsection{Description}

The segment manager manages physical  memory. Its goals are to reserve
and release physical pages, merge or split allocated pages and finally
to read or write directly into physical memory.

\subsubsection{Students work}

Students have to write everything in this manager. Some functions will
be added in k3.

\prototype{t\_error \textbf{segment\_show}(t\_segid \textbf{segid});}

This function displays information on specified segment.

\prototype{t\_error \textbf{segment\_dump}(void);}

This function displays all segments in the segment container.

\prototype{t\_error \textbf{segment\_clone}(t\_asid \textbf{asid},
                                            t\_segid \textbf{old},
                                            t\_segid* \textbf{new});}

This function clones a segment copying its content.

\prototype{t\_error \textbf{segment\_inject}(o\_segment* \textbf{seg},
                                             t\_asid \textbf{asid});}

This  function   injects  a  pre-allocated  segment   in  the  segment
container.  This function  concerns k1  pre-reserved segments.  All of
these must be passed to the inject function at kernel boot time.

\prototype{t\_error \textbf{segment\_give}(t\_asid \textbf{asid},
                                           t\_segid \textbf{segid});}

This function gives an address space from one process to another.

\prototype{t\_error \textbf{segment\_resize}(t\_segid \textbf{segid},
                                             t\_psize \textbf{new\_size},
                                             t\_segid \textbf{new\_seg});}

This function resizes a segment.

\prototype{t\_error \textbf{segment\_split}(t\_segid \textbf{segid},
                                            t\_psize \textbf{sz1},
                                            t\_segid* \textbf{s1},
                                            t\_segid* \textbf{s2});}

This function split a segment into two segments.

\prototype{t\_error \textbf{segment\_coalesce}(t\_segid \textbf{s1},
                                               t\_segid \textbf{s2},
                                               t\_segid* \textbf{new\_seg});}

This function merges two segments into a single one.

\prototype{t\_error \textbf{segment\_reserve}(t\_asid \textbf{asid},
                                              t\_psize \textbf{size},
                                              t\_perms \textbf{perms},
                                              t\_segid* \textbf{segid});}

This function reserves a segment of specified size.

\prototype{t\_error \textbf{segment\_release}(t\_segid \textbf{segid});}

This function releases a segment.

\prototype{t\_error \textbf{segment\_catch}(t\_asid \textbf{asid},
                                            t\_segid \textbf{segid});}

This function  is used to  force a segment  to be given to  an address
space.  ``catcheable''  segments  are   reserved  by  the  kernel  for
architecture specific managers. Catcheable segments are defined in the
\textit{kaneton.conf} file.

\prototype{t\_error \textbf{segment\_perms}(t\_segid \textbf{segid},
                                            t\_perms \textbf{perms});}

This function changes permissions for a segment.

\prototype{t\_error \textbf{segment\_type}(t\_segid \textbf{segid},
                                           t\_type \textbf{type});}

This function retrieves the type of a segment.

\prototype{t\_error \textbf{segment\_flush}(t\_asid \textbf{asid});}

This function  flushes the  segment container, releasing  all segments
for all address spaces.

\prototype{t\_error \textbf{segment\_get}(t\_segid \textbf{segid},
                                          o\_segment** \textbf{o});}

This  function  gets  the  segment  object  associated  to  a  segment
identifier.

\prototype{t\_error \textbf{segment\_init}(void);}

This function initialises the segment manager.

\prototype{t\_error \textbf{segment\_clean}(void);}

This function cleans the segment manager.

In  addition,  students  have  to  implement  a  first-fit  allocation
algorithm for physical pages.

\prototype{t\_error \textbf{segment\_fit}(o\_as* \textbf{as},
                                          t\_psize \textbf{size},
                                          t\_paddr* \textbf{address});}

This function searchs  the given address space for  a segment of given
size.

This function must  be placed in a separated file: take  a look at the
\textbf{\_SEGMENT\_FILE\_} variable in you \textit{user.conf} file.

As there  can be more than  one algorithm available, do  not forget to
use  the  \textbf{SEGMENT\_FIT} define  in  your \textit{conf.h}  file
(default  value is \textbf{FIT\_FIRST}.  More values  can be  added in
\textit{core/include/kaneton/kaneton.h}

\subsubsection{IA-32 implementation}

Students  have to  implement all  the architecture-dependent  code for
Intel.

Depending  on the  chosen variation  around IA-32,  there  are several
rules to follow:

\begin{itemize}

\item ia32-virtual:
\begin{itemize}

\item Segment  sizes are  aligned on page  size. Assuming you  are not
using PSE  (4 Mbyte pages), this  value will be 4096.  Remember to use
the \textbf{PAGESZ} define.

\item You must create two segments  (one for code, the other for data)
per task class:  kernel, driver, service, and user  program.  Each one
must have the appropriate privilege level and permissions.

\end{itemize}

\item ia32-segment:
\begin{itemize}

\item Obi-Wan Kenobi

\end{itemize}
\end{itemize}

%
% console manager
%

\subsection{console manager}

\subsubsection{Students work}

In fact only two functions are required for the console driver.

\prototype{void \textbf{cons\_attr}(t\_uint8 \textbf{attr});}

This is  the first \textbf{required} function.  This function modifies
display attributes for next characters printing.

\prototype{int \textbf{cons\_print\_char}(char \textbf{c});}

This  is  the second  \textbf{required}  function.  It  adds a  single
character with the currently defined attribute.

Next  functions   are  not  necessary,  but  you   must  provide  some
equivalent.

\prototype{void \textbf{cons\_clear}(void);}

This function clears the console.

\prototype{void \textbf{cons\_scroll}(t\_uint16 \textbf{lines});}

This function scrolls by given number of lines.

\prototype{void \textbf{cons\_print\_string}(char* \textbf{string});}

This function prints a string.

\prototype{void \textbf{cons\_msg}(char \textbf{indicator},
                                   char* \textbf{fmt},
                                   ...);}

This function is used to display formatted and coloured messages.

\prototype{t\_error \textbf{cons\_init}(void);}

\prototype{t\_error \textbf{cons\_clean}(void);}

%
% klibc introduction
%

\subsection{klibc introduction}

%
% modification to kernel main
%

\subsection{kernel main procedure}

As the first managers are now  working, it is time to write the kernel
main procedure.

This  one is  contained  in \textit{core/kaneton/kaneton.c}.  Students
must first initialize the \textbf{init} variable from the one given by
the bootloader. Then, they  must initialize the differents manager. Do
not forget that  it is better to initialize the  set manager before as
and segment, since these managers rely on sets !

Do not  forget to  initialize the kernel  standard library.   You must
initialize  the  \textbf{printf}  function  passing  pointers  to  the
console  manager functions to  \textbf{print\_init}. In  addition, you
must initialize the kernel  \textbf{malloc} function which is used for
early initial stages  (until as, segment and region  are running, this
will  be  possible  in  k3).  You must  use  the  \textbf{alloc\_init}
function,  passing  the  survey   area  pre-allocated  in  k1  (fields
\textbf{alloc} and \textbf{allocsz} of \textbf{t\_init}).

Finally, all  managers must be destroyed  calling the \textbf{\_clean}
functions for each one (in the good order !).

You might place  code for your tests between these  two phases. Do not
forget to disable it for the final tarball.

%
% bonus
%

\subsection{k2 bonus}

Students can implement a better allocation algorithm for
segments. Here are some examples:

\begin{itemize}

\item Best-fit (slower but less external fragmentation)
\item Buddy systems
\item Slab allocator
\item \ldots

\end{itemize}
