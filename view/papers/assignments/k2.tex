%%
%% licence       kaneton licence
%%
%% project       kaneton
%%
%% file          /home/mycure/kaneton/view/papers/assignments/k2.tex
%%
%% created       matthieu bucchianeri   [tue feb  7 11:49:56 2006]
%% updated       julien quintard   [sun may  7 13:47:45 2006]
%%

%
% k2
%

\chapter{k2}

The \textbf{k2} project consists in the development of parts of
the kaneton core leading to a complete memory management.

These microkernel's parts include:

\begin{itemize}
  \item
    A set manager's implementation.
  \item
    The segment manager.
  \item
    The region manager.
\end{itemize}

Since the kaneton microkernel heavily use the set manager to store data,
the student should perfectly understand its design and implementation.
For this reason, the first assignment will be to implement a set
implementation.

The segment manager is crucial since it will be used everywhere to allocate
memory.

Finally the region manager is also very important since it provide a complete
interface to manipulate virtual memory.

Note that the student must also fill in the architecture-dependent code
to make the full memory management working. This means that the student
should perfectly understand the segment, region and address space manager
to be able to develop the whole architecture implementation.

\newpage

%
% informations
%

\section{Informations}

\begin{tabular}{p{7cm}l}
Duration: & One week \\
File name: & \textit{[group]}-k2.tar.gz \\
In charge: & Julien Quintard \\
Newgroup: & kaneton-students@googlegroups.com \\
Languages: & Assembly and C \\
Students per group: & Three \\
\end{tabular}

%
% assignments
%

\section{Assignments}

In this section, we will detail each manager.

%
% set manager
%

\subsection{set manager}

The set manager provides an interface to create and manipule sets. These
sets can be used to store data without taking care of anything.

The manager's code is given, see \textit{kaneton/core/set/set.c} and
\textit{kaneton/include/kaneton/set.h}. Take a look at the information
section in the header of the manager's code: every step of the set creation
is described.

In this project, the student will have to write the entire code for
the \textbf{l}inked-\textbf{l}ist set implementation.

Notice that, the set implementation interface must be respected.

The student so has to fill in the file \textit{kaneton/core/set/set\_ll.c}

%
% segment manager
%

\subsection{segment manager}

The segment manager provides a complete interface to manipulate physical
memory including reserving, modifying permissions, releasing physical
memory areas called \textbf{segments}.

The student has to write the entire segment manager. Needless to say, the
student's segment manager must be compliant with the segment manager
interface.

The segment manager's main file is \textit{kaneton/core/segment/segment.c}

%
% region manager
%

\subsection{region manager}

The region manager provides everything necessary to manage virtual
memory areas called \textbf{regions}.

The student has to write the entire region manager including any
machine-dependent code.

The region manager's main file is \textit{kaneton/core/region/region.c}

%
% advanced topics
%

\section{Advanced Topics}

We advise students to first implement very simple algorithms for the segment
and region manager.

Nevertheless, once working, students might be able to implement better
allocation algorithms for the segment and region manager, like Buddy Systems,
advanced data structures etc..

%
% ia32
%

\section{Intel Architecture 32-bit}

For the segment manager, the only work of the architecture-dependent source
code is to get control over the protected mode so to rebuild the Global
Descriptor Table for the kaneton needs.

The corresponding source files are:

\begin{itemize}
  \item
    \textit{core/kaneton/arch/[architecture]/segment.c}.
  \item
    \textit{core/kaneton/arch/[architecture]/as.c}.
\end{itemize}

The corresponding include files are:

\begin{itemize}
  \item
    \textit{core/include/arch/[architecture]/kaneton/segment.h}.
  \item
    \textit{core/include/arch/[architecture]/kaneton/as.h}.
\end{itemize}

Students must implement all dependent functions they need. Remember that
functions in \textit{libia32} should be reused.

Take a look at the kaneton reference paper for more information about
machine-dependent calls.

\subsubsection{Tips}

\begin{itemize}
  \item
    Segment sizes are aligned on page size. Assuming you are not using
    four megabytes pages, this value will be 4096.

    Remember to use the \textbf{PAGESZ} define.
  \item
    You must create two segments, one for code, the other for data
    per task class: core, driver, service, and user program.

    Each one must have the appropriate privilege level and permissions.
\end{itemize}

%
% main procedure
%

\subsection{Main Procedure}

As the first managers are now working, it is time to write the kernel
main procedure.

This  one is  contained in \textit{core/kaneton/kaneton.c}. Students
must first initialize the \textbf{init} variable from the one given by
the bootloader.

Then, they  must initialize the differents manager. Do not forget that
it is better to initialize the set manager before as and segment,
since these managers rely on sets!

Do not forget to initialize the kernel standard library.

You must initialize the \textbf{printf} function passing pointers to the
console manager functions to \textbf{printf\_init()}.

In addition, you must initialize the kernel \textbf{malloc} function which
is used for early initial stages, until as, segment and region are running.
You must use the \textbf{alloc\_init()} function, passing the survival area.

To finish, you  have to create the kernel address space and to inject
the pre-allocated segments in.

Finally, all managers must be destroyed calling the clean() functions for
each one.

You might place code for your tests between these two phases. Do not
forget to disable it for the final tarball.

XXX set\_ll, segment
