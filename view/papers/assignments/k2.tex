%%
%% licence       kaneton licence
%%
%% project       kaneton
%%
%% file          /home/mycure/kaneton/view/papers/assignments/k2.tex
%%
%% created       matthieu bucchianeri   [tue feb  7 11:49:56 2006]
%% updated       julien quintard   [fri mar  3 22:30:57 2006]
%%

%
% k2
%

\section{k2}

%
% informations
%

\subsection{Informations}

\begin{tabular}{p{7cm}l}
Deadline: & XXX, 23h42 \\
Duration: & Two weeks \\
File name: & k1.tar.gz \\
In charge of: & Julien Quintard - \small{quinta\_j@epita.fr} \\
Newsgroups: & epita.cours.kaneton \\
Languages: & Assembly and C \\
Architectures: & Intel Architecture 32-bit \\
Students per group: & Three \\
\end{tabular}

%
% overview
%

\subsection{Overview}

The \textbf{k2} project consists in the development of parts of
the kaneton core including:

\begin{itemize}
  \item
    The id manager.
  \item
    The set manager.
  \item
    The address space manager.
  \item
    The segment manager.
\end{itemize}

The id manager is very important because it is very simple and will
introduce you how managers are implemented in kaneton. Every student
should be able to understand it before starting other managers.

Moreover, many parts of the kernel will lead student to use the sets,
so the sets interface must be clearly understood.

The as and segment managers introduce calls to machine-dependent code, this
mecanism must be well understood by every students.

For the two first managers, you have to follow precisely the bevavior
as described in the kaneton reference documentation. For the memory
management, you must be totally compliant with the given interface and
the described behavior, but algorithm, especially for the machine-dependent
code, are design-free.

%
% assignments
%

\subsection{Assignments}

In this section, we will detail each manager.

%
% id manager
%

\subsubsection{id manager}

The id manager is used to generate unique identifiers from an identifier
object.

Once the identifier object initialised, you can use your
identifier object \textbf{o\_id} to generate unique identifiers.

---
Remember that, in the current kaneton implementation, identifiers only
consist in 64-bit integers. Therefore, the id manager does not take care
of identifiers recycling.

Moreover, the identifier manager interface was designed to permit
identifiers recycling.
---

Now, let's take a look at the id manager functions you will implement.

\prototype{t\_error \textbf{id\_show}(o\_id* \textit{o});}
	  {This function just displays an id object's state.}

\prototype{t\_error \textbf{id\_clone}(o\_id* \textit{o},
                                       t\_id \textit{old},
                                       t\_id* \textit{new});}
	  {This function duplicates an id object using the \textit{o}
	    identifier object.}

\prototype{t\_error \textbf{id\_reserve}(o\_id* \textit{o},
                                         t\_id* \textit{u});}
	  {This function reserves an identifier \textit{u} using the
	    identifier object \textit{o}.}

\prototype{t\_error \textbf{id\_release}(o\_id* \textit{o},
                                         t\_id \textit{u});}
	  {This function releases the identifier \textit{u}.}

\prototype{t\_error \textbf{id\_build}(o\_id* \textit{o});}
	  {This function initializes an identifier object.}

\prototype{t\_error \textbf{id\_destroy}(o\_id* \textit{o});}
	  {This function destroys an identifier object.}

\prototype{t\_error \textbf{id\_init}(void);}
	  {This function just initializes the id manager.}

\prototype{t\_error \textbf{id\_clean}(void);}
	  {This function just cleans the id manager.}

Like every other managers in kaneton, the \textbf{id\_init()} and
\textbf{id\_clean()} functions must be called by the kernel main
routine on startup and shutdown.

%
% set manager
%

\subsubsection{set manager}

The set  manager is used to store data. Every kernel managers use it
instead of building data structures by their own.

The set mecanism is divided in two parts: the set manager and the set
implementations.

While the set implementations are specific to each type of set: array,
linked-list, etc.., the set manager code is generic and work with all
sets.

The manager's code is given, see \textit{core/kaneton/set/set.c} and
\textit{core/include/kaneton/set.h}. Take a look at the information
section in the header of the manager's code: each steps of the creation
of a set is described.

The student will have to write the entire code for the \textbf{array}
and \textbf{ll} implementations.

The functions below are explained independently of the data structure
which is noted \textbf{<I>}.

\prototype{t\_error \textbf{set\_type\_<I>}(u\_set \textit{u});}
	  {This function just returns an error if the set corresponding
	    to \textit{u} is not of the type \textbf{<I>}.}

\prototype{t\_error \textbf{set\_show\_<I>}(u\_set \textit{u});}
	  {This function displays an entire set.}

\prototype{t\_error \textbf{set\_head\_<I>}(u\_set \textit{u},
                                            t\_iterator* \textit{iterator});}
	  {This function returns an iterator on the first element
	    of the set.}

\prototype{t\_error \textbf{set\_tail\_<I>}(u\_set \textit{u},
                                            t\_iterator* \textit{iterator});}
	  {This function returns an iterator on the last element of
	    the set.}

\prototype{t\_error \textbf{set\_prev\_<I>}(u\_set \textit{u},
                                            t\_iterator \textit{current},
                                            t\_iterator* \textit{previous});}
	  {This function returns an iterator on the previous element
	    of the set.}

\prototype{t\_error \textbf{set\_next\_<I>}(u\_set \textit{u},
                                            t\_iterator \textit{current},
                                            t\_iterator* \textit{previous});}
	  {This function returns an iterator on the next element
	    of the set.}

\prototype{t\_error \textbf{set\_insert\_head\_<I>}(u\_set \textit{u},
                                                    void* \textit{data});}
	  {This function inserts an object at the head of the set.}

\prototype{t\_error \textbf{set\_insert\_tail\_<I>}(u\_set \textit{u},
                                                    void* \textit{data});}
	  {This function inserts an object at the tail of the set.}

\prototype{t\_error \textbf{set\_insert\_before\_<I>}(u\_set \textit{u},
                                                      t\_iterator
                                                        \textit{iterator},
                                                      void* \textit{data});}
	  {This function inserts an object before the one specified
	    by the iterator.}

\prototype{t\_error \textbf{set\_insert\_after\_<I>}(u\_set \textit{u},
                                                     t\_iterator
                                                       \textit{iterator},
                                                     void* \textit{data});}
	  {This function inserts an object after the one specified
	    by the iterator.}

\prototype{t\_error \textbf{set\_add\_<I>}(u\_set \textit{u},
                                           void* \textit{data});}
	  {This function adds a data object in the set.}

\prototype{t\_error \textbf{set\_remove\_<I>}(u\_set \textit{u},
                                              t\_id \textit{id});}
	  {This function removes an object identified by \textit{id}
	    from the set \textit{u}.}

\prototype{t\_error \textbf{set\_delete\_<I>}(u\_set \textit{u},
                                              t\_iterator \textit{iterator});}
	  {This function deletes the object corresponding to the iterator.}

\prototype{t\_error \textbf{set\_flush\_<I>}(u\_set \textit{u});}
	  {This function removes every object stored in the set.}

\prototype{t\_error \textbf{set\_locate\_<I>}(u\_set \textit{u},
                                              t\_id \textit{id},
                                              t\_iterator* \textit{iterator});}
	  {This function returns an iterator on the element corresponding
	    to the identifier \textit{id}.}

\prototype{t\_error \textbf{set\_object\_<I>}(u\_set \textit{u},
                                              t\_iterator \textit{iterator},
                                              void** \textit{data});}
	  {This function returns the data object corresponding to
	    the iterator.}

\prototype{t\_error \textbf{set\_push\_<I>}(u\_set \textit{u},
                                            void* \textit{data});}
	  {This function adds an object to a FIFO or LIFO structure.}

\prototype{t\_error \textbf{set\_pop\_<I>}(u\_set \textit{u});}
	  {This function removes the next object of a FIFO or LIFO structure.}

\prototype{t\_error \textbf{set\_pick\_<I>}(u\_set \textit{u},
                                            void** \textit{data});}
	  {This function returns the next object of a FIFO or LIFO structure
	    without deleting it.}

\prototype{t\_error \textbf{set\_release\_<I>}(u\_set \textit{u});}
	  {This function releases a set.}

The two last functions are  described more specificaly since they have
different prototypes.

\prototype{t\_error \textbf{set\_reserve\_ll}(t\_opts \textit{opts},
                                              t\_size \textit{datasz},
                                              u\_set* \textit{u});}
	  {This function reserves a \textbf{ll} set with options
	    \textit{opts} which will contain object of \textit{datasz} size.}

\prototype{t\_error \textbf{set\_reserve\_array}(t\_opts \textit{opts},
                                                 t\_setsz \textit{initsz},
                                                 t\_size \textit{datasz},
                                                 u\_set* \textit{u});}
	  {This function reserves an \textbf{array} set with options
	    \textit{opts}. This set will contain objects of \textit{datasz}
	    size and will initialy be composed of \textit{initsz} unused
	    elements.}

You can notice that some operations does not make sense on some data
structures. For example, \textbf{set\_push()} or \textbf{set\_pop()}
will never be used on arrays or lists, but only on stacks or pipes.

In these cases, the function must be present but will only return an error.

Most widely, some operations have no sense with some options.

Inserting at the beginning of a sorted array must produce an error
since it can cancel the sorting property.

To keep sets behavior unified, here are some rules applying to
\textbf{array} and \textbf{ll}:

\begin{itemize}
  \item
    \textbf{set\_add\_array()} must insert the item at the first free place.
  \item
    All deletion functions on arrays must mark the released place as
    free.

    The only case shifting is made is when the \textit{organise} option
    is set. Then, each time a deletion function is called, the data structure
    is re-organized to become simpler.
  \item
    When inserting an object in a linked-list with the \textbf{set\_add()}
    operation, the item must be placed in the head of the list.
  \item
    As double linked-list are faster than simple one, you have to
    implement this kind of lists.

    So, the \textbf{ll} data structure is in fact a double linked-list.
\end{itemize}

Here are the options your sets must recognize:

\begin{itemize}
  \item
    The \textit{SET\_OPT\_CONTAINER} option is only used by the set
    manager to build the very first set which then becomes the set
    container.
  \item
    The \textit{SET\_OPT\_ALLOC} tells the set manager to always allocate
    memory and to copy into this memory the content of the given object.
  \item
    The \textit{SET\_OPT\_FREE} tells the set manager to think about
    releasing the objects.

    This option is mutually exclusive with the \textit{alloc} option.
  \item
    The \textit{SET\_OPT\_SORT} tells the set manager to always keep
    the set sorted by identifiers.
  \item
    The \textit{SET\_OPT\_ORGANISE} option is only used to re-organize
    the data structure to keep it as simple as possible.

    This option may have no sens with certain data structures.
\end{itemize}

%
% address space manager
%

\subsubsection{address space manager}

The address space manager just manages the address spaces. An address
space is a container describing the addressable memory.

An address space is composed of a set of segments and a set of regions.

Students have to write the whole code of this manager. No need to say
that intensive use of  sets is necessary. This manager will introduce
you how to program with the set manager.

\prototype{t\_error \textbf{as\_show}(u\_as \textit{u});}
	  {This function shows a precise address space displaying information
	    on it.}

\prototype{t\_error \textbf{as\_dump}(void);}
	  {This function dumps all the address space managed by the address
	    space manager.}

\prototype{t\_error \textbf{as\_clone}(u\_task \textit{task},
                                       u\_as \textit{old},
                                       u\_as* \textit{new});}
	  {This function clones an address space taking care of cloning
	    everything necessary.

	    ---
	    For \textbf{k2}, you don't have to deal with tasks, so you must
	    ignore the \textit{task} arguments.
	    ---}

\prototype{t\_error \textbf{as\_reserve}(u\_task \textit{task},
                                         u\_as* \textit{u});}
	  {This function reserves an address space object for the
	    task \textit{task}.}

\prototype{t\_error \textbf{as\_release}(u\_as \textit{u});}
	  {This function just releases an address space.}

\prototype{t\_error \textbf{as\_get}(u\_as \textit{u},
                                     o\_as** \textit{o});}
	  {This function should only be used by the segment and region
	    managers.

	    This function just returns the address space object
	    corresponding to the address space identifier.}

\prototype{t\_error \textbf{as\_init}(void);}
	  {This function initializes the address space manager.}

\prototype{t\_error \textbf{as\_clean}(void);}
	  {This function cleans the address space manager.}

%
% segment manager.
%

\subsubsection{segment manager}

The segment manager manages physical memory. Its goals are to reserve
and release physical memory areas, merge or split allocated areas and finally
to read or write directly into physical memory.

Students have to write everything in this manager.

---
Some functions will be added in \textbf{k3}.
---

\prototype{t\_error \textbf{segment\_show}(u\_segment \textit{u});}
	  {This function displays information on specified segment.}

\prototype{t\_error \textbf{segment\_dump}(void);}
	  {This function displays all segments in the segment container.}

\prototype{t\_error \textbf{segment\_clone}(u\_as \textit{as},
                                            u\_segment \textit{old},
                                            u\_segment* \textit{new});}
	  {This function clones a segment copying its content.}

\prototype{t\_error \textbf{segment\_inject}(u\_as \textit{as},
                                             o\_segment* \textit{o});}
	  {This function injects a pre-allocated segment in the segment
	    container.

	    ---
	    This function concerns \textbf{k1} pre-reserved segments.
	    All of these must be passed to the inject function at kernel
	    boot time.
	    ---}

\prototype{t\_error \textbf{segment\_give}(u\_segment \textit{u},
                                           u\_as \textit{as});}
	  {This function gives a segment from one address space to another.}

\prototype{t\_error \textbf{segment\_resize}(u\_segment \textit{u},
                                             t\_psize \textit{size},
                                             u\_segment* \textit{new});}
	  {This function resizes a segment.}

\prototype{t\_error \textbf{segment\_split}(u\_segment \textit{u},
                                            t\_psize \textit{size},
                                            u\_segment* \textit{left},
                                            u\_segment* \textit{rigth});}
	  {This function split a segment into two segments.}

\prototype{t\_error \textbf{segment\_coalesce}(u\_segment \textit{left},
                                               u\_segment \textit{right},
                                               u\_segment* \textit{u});}
	  {This function merges two segments into a single one.}

\prototype{t\_error \textbf{segment\_reserve}(u\_as \textit{as},
                                              t\_psize \textit{size},
                                              t\_perms \textit{perms},
                                              u\_segment* \textit{u});}
	  {This function reserves a segment of specified size.}

\prototype{t\_error \textbf{segment\_release}(u\_segment \textit{u});}
	  {This function releases a segment.}

\prototype{t\_error \textbf{segment\_catch}(u\_as \textit{as},
                                            u\_segment \textit{u});}
	  {This function is used to force a segment to be given to an address
	    space. \textbf{catcheable} segments are reserved by the module
	    service for architecture specific servers. Catcheable segments
	    are defined in the \textit{kaneton.conf} file.}

\prototype{t\_error \textbf{segment\_perms}(u\_segment \textit{u},
                                            t\_perms \textit{perms});}
	  {This function changes permissions for a segment.}

\prototype{t\_error \textbf{segment\_type}(u\_segment \textit{u},
                                           t\_type \textit{type});}
	  {This function changes the type of a segment.}

\prototype{t\_error \textbf{segment\_flush}(u\_as \textit{as});}
	  {This function removes every segment that belongs to the
	    address space specified.}

\prototype{t\_error \textbf{segment\_get}(u\_segment \textit{u},
                                          o\_segment** \textit{o});}
	  {This function gets the segment object associated to a segment
	    identifier.}

\prototype{t\_error \textbf{segment\_init}(void);}
	  {This function initializes the segment manager.}

\prototype{t\_error \textbf{segment\_clean}(void);}
	  {This function cleans the segment manager.}

---
In addition, students have to implement a first-fit allocation
algorithm for physical pages.

\prototype{t\_error \textbf{segment\_fit}(o\_as* \textbf{as},
                                          t\_psize \textbf{size},
                                          t\_paddr* \textbf{address});}
	  {This function searchs the given address space for a segment
	    of given size.

	    This function must be placed in a separated file: take a look
	    at the \textit{\_SEGMENT\_FILE\_} variable in you
	    \textit{user.conf} file.

	    As there can be more than one algorithm available, do not forget
	    to use the \textit{SEGMENT\_FIT} define in your \textit{conf.h}
	    file.

	    More values can be added in
	    \textit{core/include/kaneton/kaneton.h}}

---

%
% bonus
%

\subsubsection{Bonus}

Students can implement a better allocation algorithm for segments,
like Buddy Systems.

%
% ia32
%

\subsection{Intel Architecture 32-bit Implementation}

Students will have to develop the architecture-dependent part of the
address space manager and segment manager.

For the segment manager, the only work of the architecture-dependent source
code is to get control over the protected mode so to rebuild the Global
Descriptor Table for the kaneton needs.

For the address space manager, the only work the architecture do is to
manager Page Directories, allocating and releasing them each time an
address space object is created, destroyed, modified etc..

The corresponding source files are:

\begin{itemize}
  \item
    \textit{core/kaneton/arch/[architecture]/segment.c}.
  \item
    \textit{core/kaneton/arch/[architecture]/as.c}.
\end{itemize}

The corresponding include files are:

\begin{itemize}
  \item
    \textit{core/include/arch/[architecture]/kaneton/segment.h}.
  \item
    \textit{core/include/arch/[architecture]/kaneton/as.h}.
\end{itemize}

Students must implement all dependent functions they need. Remember that
functions in \textit{libia32} should be reused.

Take a look at the kaneton reference paper for more information about
machine-dependent calls.

\subsubsection{Tips}

\begin{itemize}
  \item
    Segment sizes are aligned on page size. Assuming you are not using
    four megabytes pages, this value will be 4096.

    Remember to use the \textbf{PAGESZ} define.
  \item
    You must create two segments, one for code, the other for data
    per task class: core, driver, service, and user program.

    Each one must have the appropriate privilege level and permissions.
\end{itemize}

%
% main procedure
%

\subsection{Main Procedure}

As the first managers are now working, it is time to write the kernel
main procedure.

This  one is  contained in \textit{core/kaneton/kaneton.c}. Students
must first initialize the \textbf{init} variable from the one given by
the bootloader.

Then, they  must initialize the differents manager. Do not forget that
it is better to initialize the set manager before as and segment,
since these managers rely on sets!

Do not forget to initialize the kernel standard library.

You must initialize the \textbf{printf} function passing pointers to the
console manager functions to \textbf{printf\_init()}.

In addition, you must initialize the kernel \textbf{malloc} function which
is used for early initial stages, until as, segment and region are running.
You must use the \textbf{alloc\_init()} function, passing the survival area.

To finish, you  have to create the kernel address space and to inject
the pre-allocated segments in.

Finally, all managers must be destroyed calling the clean() functions for
each one.

You might place code for your tests between these two phases. Do not
forget to disable it for the final tarball.
