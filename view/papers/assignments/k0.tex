%%
%% licence       kaneton licence
%%
%% project       kaneton
%%
%% file          /home/mycure/kaneton/view/papers/assignments/k0.tex
%%
%% created       matthieu bucchianeri   [tue feb  7 11:49:23 2006]
%% updated       julien quintard   [sat feb 25 17:02:07 2006]
%%

%
% k0
%

\section{k0}

%
% overview
%

\subsection{Overview}

The \textbf{k0} project consists in the development of the bootstrap.

This project is very specific and will not be re-used in the future
projects.

The only  goal is to develop  a bootstrap in assembly,  loading an ELF
binary object from  floppy drive into main memory  and finally jumping
on the binary entry point.

Generally, and on many architectures, the bootstrap also installs
a more evolved memory addressing model.

%
% ia32
%

\subsection{IA-32 implementation}

The student will have to install and activate the protected mode before
lauching the binary object.

Moreover, the project's diffculty resides in the development of an
application first evolving in real mode, then in protected mode, all
in a single assembly file.

With this project the students will learn how to use the BIOS interrupts
to perform complex tasks and to develop a very low-level program dealing
with microprocessors crashes.
