%%
%% licence       kaneton licence
%%
%% project       kaneton
%%
%% file          /home/mycure/kaneton/view/papers/assignments/k0.tex
%%
%% created       matthieu bucchianeri   [tue feb  7 11:49:23 2006]
%% updated       julien quintard   [fri mar  3 10:06:51 2006]
%%

%
% k0
%

\section{k0}

%
% informations
%

\subsection{Informations}

\begin{tabular}{p{7cm}l}
Deadline: & XXX, 23h42 \\
Duration: & one week \\
File name: & k0.tar.gz \\
In charge of: & Julien Quintard - \small{quinta\_j@epita.fr} \\
Newsgroups: & epita.kaneton \\
Languages: & Assembly and C \\
Architectures: & Intel Architecture 32-bit \\
Students per group: & three \\
\end{tabular}

%
% overview
%

\subsection{Overview}

The \textbf{k0} project consists in the development of the bootstrap.

This project is very specific and will not be reused in the future
projects.

Indeed, we will use a multibootloader like \textit{grub} or \textit{lilo}
as the bootstrap in the next steps.

The main goal of the bootstrap is to launch the bootloader in a better
execution environment, with a better memory addressing model etc..

%
% assignments
%

\subsection{Assignments}

The only  goal is to develop  a bootstrap in assembly,  loading an ELF
binary object from  floppy drive into main memory  and finally jumping
on the binary entry point.

Generally, and on many architectures, the bootstrap also installs
a more evolved memory addressing model.

The project's diffculty resides in the development of an
application first evolving in a basic addressing model, then in
a more evolved one, all in a single assembly file.

Another difficulty is to deal with microprocessor crashes.

%
% ia32
%

\subsection{Intel Architecture 32-bit Implementation}

First of all, the students will certainly want to use the BIOS interrupts
to perform the most complex tasks like loading the ELF binary object
from the floppy drive into main memory.

The only alternative to this is to develop your own floppy device driver.

Then the students will have to install and activate the protected mode.

At this point, the microprocessor is evolving into its 32-bit addressing
mode.

Finally, the binary object just needs to be launched.

The code provided must be located in the directory
\textit{core/bootstrap/arch/[architecture]/}.
