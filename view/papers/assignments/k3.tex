%%
%% licence       kaneton licence
%%
%% project       kaneton
%%
%% file          /home/buckman/kaneton/view/papers/assignments/k3.tex
%%
%% created       matthieu bucchianeri   [fri feb 10 13:52:29 2006]
%% updated       matthieu bucchianeri   [fri feb 10 14:57:06 2006]
%%

\section{k3}

k3 goals are:

\begin{itemize}

\item To implement virtual memory (region manager)

\item To complete as and segment with virtual memory related functions

\item To implement a few drivers

\end{itemize}

\subsection{region manager}

\subsection{as manager}

Students must finish the as manager adding the following functions:

\prototype{t\_error \textbf{as\_paddr}(t\_asid \textbf{asid},
                                       t\_regid \textbf{regid},
                                       t\_vaddr	\textbf{virtual},
                                       t\_paddr* \textbf{physical});}

This function translates a virtual address to a physical address.

\prototype{t\_error \textbf{as\_vaddr}(t\_asid \textbf{as},
                                       t\_segid \textbf{segid},
                                       t\_paddr \textbf{physical},
                                       t\_vaddr* \textbf{virtual});}

This function translates a physical address to a virtual address.

\subsection{segment manager}

\subsubsection{Functions to add}

In this part  of kaneton, students have to  finish the segment manager
adding the following functions:

\prototype{t\_error \textbf{segment\_read}(t\_segid \textbf{segid},
                                           t\_paddr \textbf{offs},
                                           const void* \textbf{buff},
                                           t\_psize \textbf{sz});}

This function reads raw data from a segment into a buffer.

\prototype{t\_error \textbf{segment\_write}(t\_segid \textbf{segid},
                                            t\_paddr \textbf{offs},
                                            void* \textbf{buff},
                                            t\_psize \textbf{sz});}

This function writes data to a segment.

\prototype{t\_error \textbf{segment\_copy}(t\_segid \textbf{dst},
                                           t\_paddr \textbf{offsd},
                                           t\_segid \textbf{src},
                                           t\_paddr \textbf{offss},
                                           t\_psize \textbf{sz});}

This function copies data from one segment to another.

\subsubsection{IA-32 implementation}

These three functions  were not asked for k2  since directly accessing
physical memory with paging mode enabled is not possible on Intel.

Students have to write these functions even if they chose ia32-segment
(which code will be very simple).
