%%
%% licence       kaneton licence
%%
%% project       kaneton
%%
%% file          /home/buckman/kaneton/view/papers/kaneton/5-coding.tex
%%
%% created       matthieu bucchianeri   [mon jan 30 17:32:57 2006]
%% updated       matthieu bucchianeri   [fri feb 10 17:57:27 2006]
%%

%
% coding style
%

\chapter{Coding style}

\newpage

The \kaneton  project developers  try to follow  a coding  style. This
coding style was introduced to normalize the source code, leading to a
more readable source code.

Nevertheless, you  can adapt this coding  style to your own  but try to
follow the rules.

\section*{Headers}

Each file must start with an header formatted as shown below:

\begin{verbatim}
/*
 * licence       kaneton licence
 *
 * project       kaneton
 *
 * file          /home/buckman/kaneton/core/kaneton/as/as.c
 *
 * created       julien quintard   [fri feb 11 02:23:41 2005]
 * updated       matthieu bucchianeri   [mon jan 30 20:30:57 2006]
 */
\end{verbatim}

An emacs  file for generating  and updating this  header automatically
can be found in \textit{tools/emacs/header.el}. Additionally, you need
to set two environment variables:

\begin{itemize}
\item \textbf{EC\_LICENCE} must be set to ``kaneton licence''
\item \textbf{EC\_DEVELOPER} must be set to your first name and last name
\end{itemize}

Please, do not use nicknames in headers.

\section*{Naming convenions}

To keep the code as clear as possible, there are several convetions on
type, functions and variables naming.

\subsection*{Variable name}

Variable  names   must  be  expressed  in  english   with  lower  case
letters. Here are a few rules you are encouraged to follow:

\begin{itemize}
\item \textbf{sz} suffix for variables representing a size
\item \textbf{n} prefix for variables representing a number of objects
\item \textbf{id} suffix for identifiers
\item \ldots
\end{itemize}

Remember that the type name is also used to define a variable. No need
to repeat it in the variable name.

\subsection*{Function name}

Function names  must be expressed  in english with lower  case letters
and must  be prefixed by  the file name  they are implemented  in. For
example, a function part of the address space manager must be prefixed
by \textbf{as\_}.

These  names must be  chosen carefully:  they must  explicitely define
what the function does.

\subsection*{Type name}

As variables  and functions, type  names must be expressed  in english
with lower case letters.

Here are the prefixes you must use when writing your own types:

\begin{itemize}
\item \textbf{m\_} for managers main structures
\item \textbf{o\_} for  objects (an object is a  structure which first
64-bits are an identifier)
\item \textbf{i\_} for interfaces (an interface is a set of function pointers)
\item \textbf{d\_} for architecture-dependent structures
\item \textbf{s\_} for general purpose structures
\item \textbf{t\_} for basic and general purpose typedefs
\end{itemize}

Notice  that \textbf{d\_}  can be  combined with  other  prefixes, for
example \textbf{do\_} for a dependent object.

\section*{Includes}

To keep the  code clear and compact, developer only  need to include a
minimal number of header files:

\begin{itemize}
\item \textit{kaneton.h} for the operating system declarations
\item \textit{klibc.h} for the kaneton specific C library
\end{itemize}

These files  are in the include  path, so do not  use relative include
path.

All include  files must be  protected against multiple  inclusion. The
guard to use  must be named using the  directory name, one underscore,
the file name,  one underscore and a capital H.  For example, the file
\textit{core/include/kaneton/segment.h} will be guarded as follow:

\begin{verbatim}
#ifndef KANETON_SEGMENT_H
#define KANETON_SEGMENT_H	1

...

#endif
\end{verbatim}

In addition,  for architecture-dependent  files, the guard  must begin
with the architecture name (ex: \textbf{IA32\_KANETON\_SEGMENT\_H}).

\section*{Types}

You  may use  as  soon as  possible  standard types:  \textbf{t\_uint8},
\textbf{t\_sint32}, \ldots{}

\section*{Return values}

Every function must report whether it successed or failed. So function
return type should be \textbf{t\_error}.

A function  will return  \textbf{ERROR\_NONE} on success  and anything
else on error (for example: \textbf{ERROR\_UNKNOWN}).

\section*{Indentation}

There are several indentation rules in \kaneton.

\begin{enumerate}

\item  Field names  of structs  and unions  must be  aligned  with the
struct or union name

\begin{verbatim}
struct       s_set
{
  t_setid    setid;
  t_setsz    size;
  t_type     type;
};
\end{verbatim}

or

\begin{verbatim}
typedef struct
{
  o_id       id;
  t_staid    stats;
  t_setid    container;
}            m_as;
\end{verbatim}

\item Defines and variable must be aligned as shown below:

\begin{verbatim}
#define TASK_PRIOR_CORE     230
#define TASK_HPRIOR_CORE    250
#define TASK_LPRIOR_CORE    210

m_task*                     task;
t_tskid                     ktask = ID_UNUSED;
\end{verbatim}

This rule also applies for variables declarations in functions.

\item Function prototype and body should look like this:

\begin{verbatim}
t_error             stats_function(t_staid          staid,
                                   char*            function,
                                   t_stats_func**   f)
{
  t_sint64          slot = -1;
  t_sint64          i;

  ...
}
\end{verbatim}

Notice  that  parameters name  are  aligned  between  each other,  and
variable names are aligned with function name and between each other.

Try  to respect  this alignment  between functions  in a  single file:
function names may be all aligned and parameter names also.

\end{enumerate}

\section*{Comments}

As \kaneton  is intended  to be a  pedagogical project with  clear and
understandable source code.  No need to say that  comments take a very
important part of this objective.

Every file must  begin with a comment describing what  is done in this
code. And  every function must be  preceded by a  comment defining its
behavior.

To prevent direct comments in the code, we used ``steps'':

\begin{itemize}

\item Each critical  code section on a function is  preceded by a step
number

\item The function header comment contains steps descriptions.

\end{itemize}

There is an example below:

\begin{verbatim}
/*
 * this function shows the usage of comments and steps.
 *
 * steps:
 *
 * 1) compute the index.
 * 2) make the operation.
 * 3) check the result.
 */
t_error         test_foobar(int      a,
                            int      b,
                            int*     c)
{
  int           idx;

  /*
   * 1)
   */

  idx = text_make_index(a, b);

  /*
   * 2)
   */

  idx = idx * a + b;

  /*
   * 3)
   */

  if (idx < 0)
    return ERROR_UNKNOWN;

  *c = idx;

  return ERROR_NONE;
}
\end{verbatim}

\section*{File structure}

\kaneton files are divided in multiple sections. Section are delimited
as shown below:

\begin{verbatim}
/*
 * ---------- includes ------------------------------------------------
 */
\end{verbatim}

Possible sections in a file are:

\begin{itemize}

\item headers: information,  dependencies, defines, types, prototypes,
macros\ldots

\item sources: information, extern, globals, includes, functions\ldots

\item makefiles: dependencies, directives, variables, rules\ldots

\end{itemize}

% end of include file
