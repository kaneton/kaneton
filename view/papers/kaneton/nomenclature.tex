%%
%% licence       kaneton licence
%%
%% project       kaneton
%%
%% file          /home/mycure/kaneton/view/papers/kaneton/nomenclature.tex
%%
%% created       julien quintard   [tue mar  7 14:19:17 2006]
%% updated       julien quintard   [thu mar  9 14:37:59 2006]
%%

%
% nomenclature
%

\chapter{Nomenclature}

In this chapter we will detail the kaneton nomenclature which is very
specific to the project.

Then, the reader will be able to understand the next chapters since the
kaneton documents heavily used the kaneton nomenclature.

\newpage

%
% text
%

Like any other important project, the kaneton microkernel project
uses many terms to define very precise things.

The kaneton nomenclature was introduced to make communication between
developers easier. Indeed it is somethimes hard to speak widely using
terms like \textit{architecture-independent soure code},
\textit{architecture-dependent source code}, \textit{contiguous area
of used physical memory}, \textit{contiguous area of free virtual memory},
etc..

The reader should notice that these terms are complex and a bit confused
when used together in the same sentence.

Then, kaneton people decided to introduce a well defined nomenclature
to make things clear.

We will so describe this nomenclature in the next sections.

%
% general
%

\section{General}

core machdep

object, identifier, capability, manager

core, driver, service, program

%
% core
%

\section{Core}

The whole kaneton core is composed of managers.

A \textbf{manager}

--

les objets:

segment region identifier task thread as

%
% external
%

\section{External}

?

---

We also use terms to design these two different pieces of code, the
\textbf{core} and the \textbf{machdep} respectively. These terms are not
quite defined and are not very clear and expressive but we chose them.
