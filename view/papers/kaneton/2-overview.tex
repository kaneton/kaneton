%%
%% licence       kaneton licence
%%
%% project       kaneton
%%
%% file          /home/buckman/kaneton/view/papers/kaneton/2-overview.tex
%%
%% created       matthieu bucchianeri   [mon jan 30 17:09:45 2006]
%% updated       matthieu bucchianeri   [fri feb 10 16:24:57 2006]
%%

%
% overview of the kernel
%

\chapter{Overview of the kernel}

\newpage

\kaneton  micro-kernel  is  only  the \textbf{core}  of  an  operating
system.  Main  task  like   hardware  drivers  or  user  services  are
implemented as  \textbf{servers}. So the  micro-kernel only has  a few
functionalities:\\

\begin{itemize}
\item Memory management
\item Process management
\item Communication
\item Events and I/O
\end{itemize}

In  this chapter, we  will describe  briefly these  tasks and  all the
associated modules and managers.

\section{Memory management}

Handling  the  memory  --  from  virtual  address  space  to  physical
addressing  --   is  done   by  three  major   managers:  \textbf{as},
\textbf{segment} and \textbf{region}.

\subsection*{as, the address space manager}

In  \kaneton, we  call  an  \textbf{address space}  a  list of  memory
location referenced by a task. Each task has its own address space.

\subsection*{segment, the physical memory manager}

\subsection*{region, memory mapping manager}

\section{Process management}

\section{Communication}

\section{Events and I/O}

\section{Other managers}

% end of include file
