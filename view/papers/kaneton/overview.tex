%%
%% licence       kaneton licence
%%
%% project       kaneton
%%
%% file          /home/mycure/kaneton/view/papers/kaneton/overview.tex
%%
%% created       matthieu bucchianeri   [mon jan 30 17:09:45 2006]
%% updated       julien quintard   [thu mar  2 13:12:22 2006]
%%

%
% overview
%

\chapter{Overview}

XXX ce chapitre va vous aider a reconnaitre les fonctionnalites principale
XXX d'un kernel dans kaneton.

The kaneton microkernel is only the core of an operating system.
Main tasks like hardware drivers or user services are implemented as
\textbf{servers}. So the microkernel only has a few functionalities to
provide:

\begin{itemize}
  \item
    Memory management.
  \item
    Process management.
  \item
    Communication.
  \item
    Events.
\end{itemize}

In this chapter we will describe briefly these tasks and all the
associated managers.

%
% memory management
%

\section{Memory Management}

Handling the memory -- from virtual address space to physical
addressing -- is done by three major managers, the \textbf{as},
\textbf{segment} and \textbf{region} managers.

%
% as
%

\subsection{as}

The address space manager just manages the different address spaces
used by the kaneton tasks.

In kaneton, we call an \textbf{as - address space} a list of memory
locations referenced by a task. Each task has its own address space.

%
% segment
%

\subsection{segment}

The segment manager just manages the segments reserved by
the different kaneton entities including the kernel, the drivers etc..

In kaneton terms a \textbf{segment} is a contiguous area of reserved
physical memory.

%
% region
%

\subsection{region}

The region manager keeps track of regions used to map segments for
each address space reserved on the system.

In kaneton, a \textbf{region} is contiguous area of virtual memory
mapping a segment's part.

%
% process management
%

\section{Process Management}

XXX

%
% communication
%

\section{Communication}

XXX

%
% events
%

\section{Events}

XXX
