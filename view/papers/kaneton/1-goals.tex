%%
%% licence       kaneton licence
%%
%% project       kaneton
%%
%% file          /home/buckman/kaneton/view/papers/kaneton/1-goals.tex
%%
%% created       matthieu bucchianeri   [mon jan 30 17:09:17 2006]
%% updated       matthieu bucchianeri   [mon jan 30 17:09:51 2006]
%%

%
% goals of the kaneton project
%

\chapter{Goals of the \kaneton project}

\newpage

\paragraph{}
This   document   describes   the  \kaneton   micro-kernel   reference
implementation.

\paragraph{}
The \kaneton micro-kernel is  studied during the EPITA System, Network
and  Security specialization.   The  project comes  with many  courses
leading student to learn about architectures and kernels and to design
and develop important parts of a micro-kernel.

\paragraph{}
From this fact, \kaneton has been designed in a very particular way to
keep the  design as  simpler as possible  to be understandable  by the
students.

\paragraph{}
\kaneton is not based on any previous kernel code: it is written from
nothing. As we  wanted the kernel to be  portable on any architecture,
we paid  particular attention  in dividing the  code into  two precise
sections: the independent code and the architecture-dependent code.

\paragraph{}
Designing this portability  system, we intended to port  the kernel on
many architectures, so we  had to keep the architecture-dependent code
as  small  as possible.   Moreover,  the  only  requirement about  the
hardware is  the minimal amount of  memory, which must be  at least 32
megabytes.

\paragraph{}
As we wanted the project to be very pedagogic, \kaneton source code is
clear  and heavily  commented. Each  manager  or service  has its  own
directory and  headers. Function name  and prototypes are  unified and
explicit. Each function is preceded by a header explaining the task it
performs and detailing the different steps.

% end of include file
