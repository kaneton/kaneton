%%
%% licence       kaneton licence
%%
%% project       kaneton
%%
%% file          /home/buckman/kaneton/view/papers/kaneton/1-goals.tex
%%
%% created       matthieu bucchianeri   [mon jan 30 17:09:17 2006]
%% updated       matthieu bucchianeri   [fri feb 10 16:24:41 2006]
%%

%
% goals of the kaneton project
%

\chapter{Goals of the \kaneton project}

\newpage

This   document   describes   the  \kaneton   micro-kernel   reference
implementation.

The \kaneton micro-kernel is  studied during the EPITA System, Network
and  Security specialization.   The  project comes  with many  courses
leading student to learn about architectures and kernels and to design
and develop important parts of a micro-kernel.

From this fact, \kaneton has been designed in a very particular way to
keep the  design as  simpler as possible  to be understandable  by the
students.

\kaneton is not based on any previous kernel code: it is written from
nothing. As we  wanted the kernel to be  portable on any architecture,
we paid  particular attention  in dividing the  code into  two precise
sections: the independent code and the architecture-dependent code.

Designing this portability  system, we intended to port  the kernel on
many architectures, so we  had to keep the architecture-dependent code
as  small  as possible.   Moreover,  the  only  requirement about  the
hardware is  the minimal amount of  memory, which must be  at least 32
megabytes.

As we wanted the project to be very pedagogic, \kaneton source code is
clear  and heavily  commented. Each  manager  or service  has its  own
directory and  headers. Function name  and prototypes are  unified and
explicit. Each function is preceded by a header explaining the task it
performs and detailing the different steps.

\section*{Project history}

\subsection*{Year 2004}

The  project  was  introduced  by  students willing  to  learn  kernel
architectures.

\subsection*{Year 2005}

The  first complete  kaneton  project was  scheduled  from January  to
December.

\subsection*{Year 2006}

The  kaneton project comes  with a  development environment.  Then the
students  have to  write parts  of the  microkernel without  having to
build a development environment from scratch which takes much time.

Moreover, courses  were added to the  EPITA specialization curriculum:
microprocessor's architectures, kernels history\ldots{}

% end of include file
