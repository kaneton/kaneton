%%
%% licence       kaneton licence
%%
%% project       kaneton
%%
%% file          /home/buckman/kaneton/view/papers/kaneton/kaneton.tex
%%
%% created       julien quintard   [thu dec  8 00:26:00 2005]
%% updated       matthieu bucchianeri   [wed jan 25 17:19:47 2006]
%%

%
% class
%

\documentclass[12pt,a4wide]{report}

%
% packages
%

\usepackage[english]{babel}
\usepackage{a4wide}
\usepackage[Lenny]{fncychap}
\usepackage{fancyheadings}
\usepackage{multicol}
\usepackage{indentfirst}
\usepackage{graphicx}
\usepackage{color}
\usepackage{xcolor}
\usepackage{verbatim}
\usepackage{xspace}

\pagestyle{fancy}

\setlength{\footrulewidth}{0.3pt}
\setlength{\parindent}{0.3cm}
\setlength{\parskip}{2ex plus 0.5ex minus 0.2ex}

%
% verbatim font
%

\definecolor{verbatimcolor}{rgb}{0,0.4,0}

\makeatletter
\renewcommand{\verbatim@font}
  {\ttfamily\footnotesize\color{verbatimcolor}\selectfont}
\makeatother

%
% kaneton
%

\newcommand{\kaneton}{kaneton\xspace}

%
% header
%

\rfoot{\scriptsize{The kaneton microkernel project}}
\lhead{\scriptsize{The kaneton microkernel project}}

\date{\scriptsize{\today}}

%
% title
%

\title{\huge{The kaneton micro-kernel reference}
\\\vspace{2cm}\includegraphics{logo.jpg}}

%
% authors
%

\author{\small{Julien Quintard}, \small{Matthieu Bucchianeri},
\small{Renaud Voltz}\vspace{2cm}}

%
% document
%

\begin{document}

%
% title
%

\maketitle

%
% --------- text --------------------------------------------------------------
%

%
% authors
%

\newpage

\paragraph{}
The \kaneton project was introduced at EPITA by two students:

\begin{itemize}
\item Julien Quintard \footnote{quinta\_j@epita.fr}
\item Jean-Pascal Billaud \footnote{billau\_j@epita.fr}
\end{itemize}

\paragraph{}
This document especially describes the kaneton reference micro-kernel
developed by:

\begin{itemize}
\item Julien Quintard
\item C\'edric Aubouy \footnote{aubouy\_c@epita.fr}
\item Matthieu Bucchianeri \footnote{bucchi\_m@epita.fr}
\item Renaud Voltz \footnote{voltz\_r@epita.fr}
\end{itemize}

\paragraph{}
But many people contributed to this project and we thank them.

%
% toc
%

\tableofcontents

%
% goals of the kaneton project
%

\chapter{Goals of the \kaneton project}

\newpage

\paragraph{}
This   document   describes   the  \kaneton   micro-kernel   reference
implementation.

\paragraph{}
The \kaneton micro-kernel is  studied during the EPITA System, Network
and  Security specialization.   The  project comes  with many  courses
leading student to learn about architectures and kernels and to design
and develop important parts of a micro-kernel.

\paragraph{}
From this fact, \kaneton has been designed in a very particular way to
keep the  design as  simpler as possible  to be understandable  by the
students.

\paragraph{}
\kaneton is not based on any previous kernel code: it is written from
nothing. As we  wanted the kernel to be  portable on any architecture,
we paid  particular attention  in dividing the  code into  two precise
sections: the independent code and the architecture-dependent code.

\paragraph{}
Designing this portability  system, we intended to port  the kernel on
many architectures, so we  had to keep the architecture-dependent code
as  small  as possible.   Moreover,  the  only  requirement about  the
hardware is  the minimal amount of  memory, which must be  at least 32
megabytes.

\paragraph{}
As we wanted the project to be very pedagogic, \kaneton source code is
clear  and heavily  commented. Each  manager  or service  has its  own
directory and  headers. Function name  and prototypes are  unified and
explicit. Each function is preceded by a header explaining the task it
performs and detailing the different steps.

%
% overview of the kernel
%

\chapter{Overview of the kernel}

%
% development environment
%

\chapter{Development environment}

%
% description of the source tree
%

\chapter{Description of the source tree}

%
% coding style
%

\chapter{Coding style}

%
% architecture-independant code
%

\chapter{Architecture-independant code}

%
% tools
%

\chapter{Tools}

%
% test suite
%

\chapter{Test suite}

%
% glossary
%

\chapter{Glossary}


\end{document}
\label{}
\ref{}
\pageref{}
