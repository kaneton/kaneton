%%
%% licence       kaneton licence
%%
%% project       kaneton
%%
%% file          /home/mycure/kaneton/view/papers/kaneton/kaneton.tex
%%
%% created       julien quintard   [thu dec  8 00:26:00 2005]
%% updated       julien quintard   [thu dec  8 00:37:00 2005]
%%

%
% template
%

%%
%% copyright     (c) julien quintard
%%
%% project       kaneton
%%
%% file          /home/mycure/kaneton/view/templates/paper.tex
%%
%% created       julien quintard   [sat nov 19 18:11:23 2005]
%% updated       julien quintard   [tue dec 13 01:15:46 2005]
%%

%
% class
%

\documentclass[10pt,a4wide]{article}

%
% packages
%

\usepackage[english]{babel}
\usepackage{a4wide}
\usepackage{fancyheadings}
\usepackage{multicol}
\usepackage{indentfirst}
\usepackage{graphicx}
\usepackage{color}
\usepackage{xcolor}
\usepackage{verbatim}

\pagestyle{fancy}

\setlength{\footrulewidth}{0.3pt}
\setlength{\parindent}{0.3cm}
\setlength{\parskip}{2ex plus 0.5ex minus 0.2ex}

%
% verbatim font
%

\definecolor{verbatimcolor}{rgb}{0,0.4,0}

\makeatletter
\renewcommand{\verbatim@font}
  {\ttfamily\footnotesize\color{verbatimcolor}\selectfont}
\makeatother

%
% header
%

\rfoot{\scriptsize{The kaneton microkernel project}}

\date{\scriptsize{\today}}


%
% header
%

\lhead{\scriptsize{The kaneton microkernel reference}}

%
% title
%

\title{The kaneton microkernel reference}

%
% authors
%

\author{\small{Julien Quintard}}

%
% document
%

\begin{document}

%
% title
%

\maketitle

%
% --------- text --------------------------------------------------------------
%



%
% the project
%

\section{The Project}

This document describes the kaneton microkernel reference.

kaneton is a pedagogic microkernel developed by EPITA students during
their specialisation year in System, Network and Security.

This project consists in the design and development of parts of an
operating system including the micro-kernel and the servers.

The kaneton operating system, especially the kaneton micro-kernel was
designed in a very specific way to keep the design the simplest as possible
to be understable by the students.

Nevertheless, the kaneton project and especially the kaneton reference
operating system was implementing using modern tools and modern
programming techniques.

The kaneton microkernel is divided into managers which perform very
specific tasks and make the design simpler to understand.

We decided to develop a microkernel because this type of kernel is very
modular, secure, strong and clear.

All these characteristics make the microkernel a perfect development
project for students.

The goals of the project are to lead the student to understand one or more
processor architectures, an entire microkernel design and implementation
and finally the principles and paradigms of a distributed systems.

The kaneton project was introduced at EPITA by two students
	Julien Quintard
	  \footnote{quinta\_j@epita.fr} and
	Jean-Pascal Billaud
	  \footnote{billau\_j@epita.fr}.

This document especially describes the kaneton reference microkernel
developed by
	Julien Quintard,
	C\'edric Aubouy
	  \footnote{aubouy\_c@epita.fr},
	Matthieu Bucchianeri
	  \footnote{bucchi\_m@epita.fr}.

But many people contributed to this project and we thank them.



%
% the development environment
%

\section{The Development Environment}

XXX

%
% XXX
%

%
% the bootstrap
%

\section{The Bootstrap}

The bootstrap is a little code which is launch by the BIOS.

Its role is to launch the bootloader. To do this, it has to read the
bootloader on a device which can be the floppy, the hard drive,
the network etc..

\subsection{Intel 32-bit Architecture}

For the Intel Architecture 32-bit (IA32), we decided to use the
	GRUB MultiBoot boot loader
	  \footnote{http://www.gnu.org/software/grub/}.

Indeed, the GRUB bootloader handles many different boot devices with
many different hardware devices: floppy, hard drive, network etc..

Moreover, the GRUB bootloader provide the possibility to load
modules in main memory. This feature is very intersting for a micro-kernel
because we can now tell GRUB to load every micro-kernel's servers into
main memory.

The EPITA Computer System Laboratory currently develop an Intel Architecture
32-bit bootstrap. For more information, take a look to the Laboratory
website: \textbf{http://www.lse.epita.fr}.

In technical terms, the bootstrap just load and lanch the bootloader
which will be able to initialise the kernel.



\section{The Bootloader}

The bootloader's goal is to initialise the kernel environment.

In kaneton, the kernel, when launched, has to run in a very specific
environment.

First, the virtual memory has to be initialised. Second, a special
information structure called \textbf{init} has to be provided to
the kernel when started.

So, the bootloader has to fill in these requirements to be able
to correctly laucnh the kernel.

\subsection{Intel 32-bit Architecture}

\end{document}
\label{}
\ref{}
\pageref{}

XXX machdep a la fin car depend de la partie generale contrairement a l autre

XXX header, commit norme, coding style, dev-env: makefiles/shell, moulinette,
    etc..