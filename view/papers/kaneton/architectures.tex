%%
%% licence       kaneton licence
%%
%% project       kaneton
%%
%% file          /home/mycure/kaneton/view/papers/kaneton/architectures.tex
%%
%% created       julien quintard   [sun apr 23 17:08:41 2006]
%% updated       julien quintard   [sun apr 23 17:08:41 2006]
%%

%
% architectures
%

\chapter{Architectures}

This chapter will overview the different architectures supported by
the kaneton microkernel reference.

\newpage

%
% text
%

The kaneton microkernel references currently supports the following
architectures.

Notice that, sometimes, the kaneton microkernel can support multiples
\textit{architectures} of a single microprocessor's architecture.
These multiples \textit{architectures} generally exploit different
facilities of the same microprocessor's architecture.

There is a paper specific to every supported architecture. These papers
are available on the kaneton official website
  \footnote{http://www.kaneton.org}.

%
% intel architecture 32-bit
%

\section{Intel Architecture 32-bit}

The kaneton microkernel reference was first developed on the Intel Architecture
32-bit since this architecture is very popular and cheap.

The following architecture implementations are supported by the kaneton
microkernel.

%
% ia32-virtual
%

\subsection{ia32-virtual}

This architecture uses a flat segmentation model with paging enabled
allowing multiple virtual address space and so virtual address space
protections.

Nevertheless, this architecture implementation is kept as simple as possible
since it is used as the basis of the operating system courses.

One of the particularities of this architecture is that the core has its
own address space. Then, each time a system call occurs, an address space
switch is performed to work in the core's address space.

This particularity leads to bad performances since each time an address
space switch occurs, the microprocessor's caches are flushed.

%
% ia32-optimized
%

\subsection{ia32-optimized}

This architecture also relies on the Intel Architecture 32-bit but uses
every optimisations provided by the microprocessor's architecture.

\notice{This architecture is not implemented yet.}
