%%
%% licence       kaneton licence
%%
%% project       kaneton
%%
%% file          /home/buckman/kaneton/view/papers/kaneton/4-source.tex
%%
%% created       matthieu bucchianeri   [mon jan 30 17:32:40 2006]
%% updated       matthieu bucchianeri   [fri feb 10 18:27:14 2006]
%%

%
% description of the source tree
%

\chapter{Description of the source tree}

\newpage

This section describes \kaneton source tree in details:

\begin{verbatim}
 check/
 core/
 drivers
 env/
 export/
 libs/
 programs/
 services/
 tools/
\end{verbatim}

\section{check/}

The \textbf{check}  subdirectory containes \kaneton  test suite. Check
chapter 9 -- \textbf{Test suite} for more information.

\section{core/}

This subdirectory is the most important of the project. It contains
all the microkernel code, the bootloader and the bootstrap.

\subsection*{core/bootstrap/}

This directory contains bootstraps for all supported architectures.

\begin{verbatim}
 core/bootstrap/arch/
                     ia32-virtual
                     ultrasparc
                     ...
\end{verbatim}

\subsection*{core/bootloader/}

This directory contains bootloaders for all supported architectures.

\begin{verbatim}
 core/bootloader/arch/
                      ia32-virtual
                      ultrasparc
                      ...
\end{verbatim}

\subsection*{core/kaneton/}

This place containts \kaneton source code. It is divided as shown below:

\begin{verbatim}
 core/kaneton/
              arch/
              as/
              conf/
              debug/
              id/
              region/
              schedule/
              segment/
              set/
              stats/
              task/
              thread/
\end{verbatim}

Notice that there is one subdirectory per manager.

The  \textbf{conf} sudirectory  contains links  to user  profile files
(\textit{conf.c} and \textit{conf.h}).

\textbf{debug} contains  a console driver  (only for the  kernel early
boot, later, a console service  is launched) and a serial driver (used
by the test tools).

All  architecture  dependent code  is  placed  in \textbf{arch}.  This
directory contains subdirectories for each architecture.

\begin{verbatim}
 core/kaneton/arch/
                   ia32-virtual
                   ultrasparc
                   ...
\end{verbatim}

Each   of    these   subdirectories   contains    files   implementing
architecture-dependent code for each manager.

\section{drivers/}

% end of include file
