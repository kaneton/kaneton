%%
%% licence       kaneton licence
%%
%% project       kaneton
%%
%% file          /home/mycure/kaneton/view/papers/kaneton/goals.tex
%%
%% created       matthieu bucchianeri   [mon jan 30 17:09:17 2006]
%% updated       julien quintard   [thu mar  2 01:21:16 2006]
%%

%
% goals
%

\chapter{Goals}

The kaneton microkernel is studied during the EPITA System, Network
and  Security specialization. The project comes with many courses
leading student to learn about architectures and kernels and to design
and develop important parts of a microkernel.

From this fact, kaneton has been designed in a very particular way to
keep the design as simple as possible to be understandable by the
students.

kaneton is not based on any previous kernel code: it is written from
scratch. As we wanted the kernel to be portable on any architecture,
we paid particular attention in dividing the code into two precise
sections: the architecture-independent code and the
architecture-dependent code.

---

We also use terms to design these two different pieces of code, the
\textbf{core} and the \textbf{arch} respectively. These terms are not
quite defined and are not very clear and expressive but we chose them.

---

Designing this portability system, we intended to port the kernel on
many architectures, so we had to keep the architecture-dependent code
as small as possible. Moreover, the only requirement about the
hardware is the minimal amount of memory, which must be at least 32
megabytes.

As we wanted the project to be very pedagogic, kaneton source code is
clear and heavily commented. Each manager has its own directory and
headers. Function names and prototypes are unified and explicit.
Each function is preceded by a header explaining the task it performs
and detailing the different steps.

%
% history
%

\section{History}

%
% year 2006
%

\subsection{Year 2006}

The kaneton project comes with a development environment. Then the
students have to write parts of the microkernel without having to
build a development environment from scratch which takes much time.

Moreover, courses were added to the EPITA specialization curriculum
including microprocessor's architectures, kernels history etc..

%
% year 2005
%

\subsection{Year 2005}

The first complete kaneton project was scheduled from January to December.

%
% year 2004
%

\subsection{Year 2004}

The project was introduced by students willing learn kernel architectures.
