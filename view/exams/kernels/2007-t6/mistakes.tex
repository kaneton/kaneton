\section*{Exercice 2 : Gestionnaire d'interruption (3 points)}

{\bf C\^ot\'e noyau}

Un SIGL a essay\'e d'optimiser le handler d'exceptions par d\'efaut dans la version 1.0 de Linux. Trouvez ses erreurs et proposez une correction.

{\bf Note:} Linux 1.0 est sortie en Mars 1994. \`A l'\'epoque Linux n'avait pas \'et\'e port\'e sur d'autres arcitectures que IA32.

\begin{verbatim}
/* This is the default interrupt ``handler'' :-) */
int_msg:
        .asciz ``Unknown interrupt\n''

.align 2
ignore_int:

        //code with sides effects before context saving


        cld                         ; ne pas prendre en compte
        pushl %eax
        pushl %ecx
        pushl %edx
        push %ds
        push %es
        push %fs

        movl $(KERNEL_CS),%eax
        mov %ax,%ds
        mov %ax,%es
        mov %ax,%fs

        pushl $int_msg
        call _printk

        pop %fs
        pop %es
        pop %ds
        popl %edx
        popl %ecx
        popl %eax

        ret

//pop microprocessor info
\end{verbatim}

\begin{itemize}
\item Vous consid\'erez qu'aucun autre registre que ceux qui apparaissent dans ce code ne sera utilis\'e pendant le traitement du handler.
\end{itemize}

\begin{correction}
\begin{verbatim}
/* This is the default interrupt ``handler'' :-) */
int_msg:
        .asciz ``Unknown interrupt\n''

.align 2
ignore_int:
        cld
        pushl %eax
        pushl %ecx
        pushl %edx
        push %ds
        push %es
        push %fs

        movl $(KERNEL_DS),%eax
        mov %ax,%ds
        mov %ax,%es
        mov %ax,%fs

        pushl $int_msg
        call _printk
        popl %eax

        pop %fs
        pop %es
        pop %ds
        popl %edx
        popl %ecx
        popl %eax

        iret
\end{verbatim}
\end{correction}




