\section*{Exercice 4 : Sleep/Wakeup (5.5 points)}

{\bf \emph{sleep}/\emph{wakeup} :}

Le m�canisme \emph{sleep} /\emph{wakeup} existe depuis les plus vieux
Unix.

\emph{sleep} est utilis� pour endormir un thread et \emph{wakeup} pour
le r\'eveiller. Le deux fonctions prennent en param\`etre un identifiant
unique. Un appel \`a \emph{wakeup} avec un identifiant donn� reveille
\textbf{tous} les threads ayant appel\'e \emph{sleep} avec ce m\^eme
identifiant. La fonction \emph{wakeup\_once} ne r\'eveille qu'un seul des
threads bloqu�s par \emph{sleep} (le premier s'�tant endormi).

{\bf Exemple :}

\begin{center}
\begin{tabular}{| l | l | l | l |}
thread 1 & thread 2 & thread 3 & thread 4\\
\ldots & \ldots & \ldots & \ldots \\
sleep(42); & \ldots & \ldots & \ldots \\
\emph{sleeping} & \ldots & \ldots & \ldots \\
\emph{sleeping} & sleep(42); & sleep(123);  & \ldots \\
\emph{sleeping} & \emph{sleeping} & \emph{sleeping} & \ldots \\
\emph{sleeping} & \emph{sleeping} & \emph{sleeping} & \ldots \\
\emph{sleeping} & \emph{sleeping} & \emph{sleeping} & wakeup(42); \\
\ldots & \ldots & \emph{sleeping} & \ldots \\
\ldots & \ldots & \emph{sleeping} & \ldots \\
\end{tabular}
\end{center}

Dans cet exemple, le thread 1 puis le thread 2 s'endorment avec comme
identifiant 42. Le thread 3 s'endort avec l'identifiant 123. C'est le
thread 4 qui r\'eveille les threads 1 et 2.

\begin{description}
\item {\bf Impl�mentation dans kaneton}

\begin{enumerate}
\item
  Quelles donn\'ees allez-vous devoir associer \`a chaque identifiant
  pass\'e \`a vos fonctions ?
  \\
  \'Ecrivez la structure en C correspondante.

\item
  \'Ecrivez le code de la fonction \emph{sleep}.\\
  {\emph{void} \bf sleep}({\em t\_id id)}

\item
  \'Ecrivez le code de la fonction \emph{wakeup\_once}.\\
  {\emph{void} \bf wakeup\_once}({\em t\_id id)}

\item
  \'Ecrivez le code de la fonction \emph{wakeup}.\\
  {\emph{void} \bf wakeup}({\em t\_id id)}

\end{enumerate}

\end{description}
