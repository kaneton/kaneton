\section*{Exercice 4 : Sleep/Wakeup (X points)}

{\bf \emph{sleep}/\emph{wakeup} :}

Le m�canisme \emph{sleep} /\emph{wakeup} existe depuis les plus vieux
Unix.

\emph{sleep} est utilis� pour endormir un thread et \emph{wakeup} pour
le reveiller. Le deux fonctions prennent en argument un identifiant
unique. Un appel a \emph{wakeup} avec un identifiant donn� reveille
\textbf{tous} les threads ayant appel�s \emph{sleep} avec ce m�me
identifiant. La fonction \emph{wakeup\_once} ne reveille qu'un seul de
threads bloqu�s par \emph{sleep} (le premier s'�tant endormi).

\begin{description}
\item {\bf Impl�mentation dans kaneton}

\begin{enumerate}
\item
  Quelles donn�es allez-vous devoir associer � chaque identifiant
  passe � vos fonctions ?
  \\
  Ecrivez la structure en C correspondante.

\item
  Ecrivez le code de la fonction \emph{sleep}.\\
  {\emph{void} \bf sleep}({\em t\_id id)}

\item
  Ecrivez le code de la fonction \emph{wakeup\_once}.\\
  {\emph{void} \bf wakeup\_once}({\em t\_id id)}

\item
  Ecrivez le code de la fonction \emph{wakeup}.\\
  {\emph{void} \bf wakeup}({\em t\_id id)}

\end{enumerate}

\end{description}
