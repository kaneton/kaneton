\section*{Exercice 4 : Sleep/Wakeup (X points)}

{\bf \emph{sleep}/\emph{wakeup} :}

Le m�canisme \emph{sleep} /\emph{wakeup} existe depuis les plus vieux
Unix.

\emph{sleep} est utilis� pour endormir un thread et \emph{wakeup} pour
le reveiller. Le deux fonctions prennent en argument un identifiant
unique. Un appel a \emph{wakeup} avec un identifiant donn� reveille
\textbf{tous} les threads ayant appel�s \emph{sleep} avec ce m�me
identifiant. La fonction \emph{wakeup\_once} ne reveille qu'un seul de
threads bloqu�s par \emph{sleep} (le premier s'�tant endormi).

{\bf Exemple :}

\begin{center}
\begin{tabular}{| l | l | l | l |}
thread 1 & thread 2 & thread 3 & thread 4\\
\ldots & \ldots & \ldots & \ldots \\
sleep(42); & \ldots & \ldots & \ldots \\
\emph{sleeping} & \ldots & \ldots & \ldots \\
\emph{sleeping} & sleep(42); & sleep(123);  & \ldots \\
\emph{sleeping} & \emph{sleeping} & \emph{sleeping} & \ldots \\
\emph{sleeping} & \emph{sleeping} & \emph{sleeping} & \ldots \\
\emph{sleeping} & \emph{sleeping} & \emph{sleeping} & wakeup(42); \\
\ldots & \ldots & \emph{sleeping} & \ldots \\
\ldots & \ldots & \emph{sleeping} & \ldots \\
\end{tabular}
\end{center}

Dans cet exemple, le thread 1 puis le thread 2 s'endorment avec comme
identifiant 42. Le thread 3 s'endort avec l'identifiant 123. C'est le
thread 4 qui reveille les threads 1 et 2.

\begin{description}
\item {\bf Impl�mentation dans kaneton}

\begin{enumerate}
\item
  Quelles donn�es allez-vous devoir associer � chaque identifiant
  passe � vos fonctions ?
  \\
  Ecrivez la structure en C correspondante.

  \begin{correction}
    \begin{verbatim}
typedef struct
{
  t_id  id;

  i_set queue;
}       o_wait;
    \end{verbatim}
  \end{correction}

\item
  Ecrivez le code de la fonction \emph{sleep}.\\
  {\emph{void} \bf sleep}({\em t\_id id)}

  \begin{correction}
    \begin{verbatim}
void          sleep(t_id id)
{
  i_thread    thid;
  o_wait      *wait;

  thread_current(&thid);
  if (set_get(waits, id, (void**)&wait) != ERROR_NONE)
  {
    wait = malloc(sizeof (o_wait));
    wait->id = id;
    set_add(waits, wait);
    set_reserve(pipe, SET_OPT_NONE, sizeof (o_wait), &wait->queue);
  }
  set_push(wait->queue, &thid);
  thread_state(thid, SCHED_STATE_STOP);
}
    \end{verbatim}
  \end{correction}

\item
  Ecrivez le code de la fonction \emph{wakeup\_once}.\\
  {\emph{void} \bf wakeup\_once}({\em t\_id id)}

  \begin{correction}
    \begin{verbatim}
void          wakeup_once(t_id id)
{
  i_thread    thid;
  o_wait      *wait;

  if (set_get(waits, id, (void**)&wait) != ERROR_NONE)
    return;
  if (set_pick(wait->queue, &thid) != ERROR_NONE)
    return;
  set_pop(wait->queue);
  thread_state(thid, SCHED_STATE_RUN);
}
    \end{verbatim}
  \end{correction}


\item
  Ecrivez le code de la fonction \emph{wakeup}.\\
  {\emph{void} \bf wakeup}({\em t\_id id)}

  \begin{correction}
    \begin{verbatim}
void          wakeup(t_id id)
{
  i_thread    thid;
  o_wait      *wait;

  if (set_get(waits, id, (void**)&wait) != ERROR_NONE)
    return;
  while (set_pick(wait->queue, &thid) != ERROR_NONE)
  {
    set_pop(wait->queue);
    thread_state(thid, SCHED_STATE_RUN);
  }
}
    \end{verbatim}
  \end{correction}
\end{enumerate}

\end{description}
