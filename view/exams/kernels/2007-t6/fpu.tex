\section*{Exercice 5 : Unit� � virgule flottante (X points)}

{\bf x87 FPU :}

L'unit� � virgule flottante (FPU) du Pentium, la x87 FPU, est un
composant integr� au microprocesseur et offrant le support des nombres
flottant IEEE 754.

La FPU contient les registres suivants :

\begin{itemize}
\item
  Status Register : contient les flags (Z, N\ldots)
\item
  Control Register : contient le masque d'exception
\item
  Tag Word : indique l'�tat des registres R0-R7
\item
  Instruction Pointer : pointeur vers la derni�re instruction
  flottante ex�cut�e
\item
  Instruction Opcode : opcode de la derni�re instruction flottante
  ex�cut�e
\item
  Data Pointer : pointeur vers la derni�re op�rande en charg�e en
  m�moire
\item
  R0-R7 : huit registres de calcul (aussi appel�s ST(0)-ST(7))
\end{itemize}

Les instructions suivante (disponible sous forme de macros) manipulent le contexte de la FPU :

\begin{itemize}
\item
  \verb|FSAVE(ptr)| : sauvegarde le contexte FPU � l'adresse
  \emph{ptr}
\item
  \verb|FRSTOR(ptr)| : restaure le contexte FPU depuis l'adresse
  \emph{ptr}
\end{itemize}

Lorsque le flag TS du registre CR0 est � 1 et qu'une instruction
flottante est ex�cut�e, une exception 7 (Device Not Available) est
lev�e. Le flag TS est mis � 0 par la macro CLTS() et mis � 1 par
STS().

Dans l'exercice qui suit, nous vous demandons d'utiliser ce m�canisme
d'exception pour effectuer les changements de contexte uniquement au
moment opportun.

\begin{description}
\item {\bf Impl�mentation du changement de contexte pour la FPU}

\begin{enumerate}
\item
  Quatre threads tournent sur notre syst�me. T1, T2, T3 et T4 sont
  ex�cut�s les uns � la suite des autres (Round-Robin).

  Les threads T1, T2 et T4 effectuent des calculs flottants.

  A partir du sch�ma ci-dessous, identifiez � quels instants vous
  allez devoir changer le contexte FPU.

  \begin{center}
    \includegraphics[width=\linewidth]{figures/cs-fpu}
  \end{center}

  Deduisez-en la valeur du flag TS au fil de l'ex�cution de chacun des
  threads.

  \begin{correction}
  \begin{center}
    \includegraphics[width=\linewidth]{figures/cs-fpu-correct}
  \end{center}
  \end{correction}

\item
  La structure permettant de stocker le contexte FPU vous est
  donn�e. Elle se nomme \emph{t\_x87\_context}. Elle contient tous les
  registres cit�s plus haut. Nous ne nous interesserons pas � son
  contenu d�taill�, mais sachez que la structure est compatible avec
  \verb|FSAVE| et \verb|FRSTOR|.

  Dans quel objet kaneton allez-vous stocker le contexte FPU d'un
  thread ? Dans quelle fonction de l'interface kaneton allez-vous
  initialiser ce contexte ?

  \textbf{Dans la suite de l'excercice, vous considererez que les
  structures de contexte FPU sont correctement remplies.}

  \begin{correction}

    Nous allons rajouter une instance de \emph{t\_x87\_context} dans
    la partie d�pendante de l'object \emph{o\_thread}.

    Cette structure devra �tre initialis�e au m�me moment que le reste
    du contexte, dans \emph{ia32\_thread\_init}.

  \end{correction}

\item
  Ecrivez le code du gestionnaire d'interruption Device Not
  Available. C'est bien evidemment dans ce dernier que vous devez
  �changer les contextes FPU.

  \begin{correction}
    \begin{verbatim}
void          ia32_sched_switch_fpu(i_event id)
{
  o_thread    *o;
  o_thread    *old;

  thread_get(sched->current, &o);
  thread_get(sched->current_fpu, &old);
  FSAVE(&old->machdep.fpu_context);
  FRSTOR(&o->machdep.fpu_context);
  CLTS();
  sched->current_fpu = sched->current;
}
    \end{verbatim}
  \end{correction}

  Indiquez � quel endroit du code (aussi bien celui que vous venez
  d'�crire que celui des autres fonctions du scheduler) vous devez
  mettre � jour TS.

  \begin{correction}

    Il faut ajouter un STS() dans \emph{ia32\_sched\_switch}.

  \end{correction}

\end{enumerate}
\end{description}
