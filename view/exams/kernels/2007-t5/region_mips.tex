\subsection{Impl�mentation de \emph{region} sur MIPS}

Dans cet exercice, vous allez implementer le manager \emph{region} de
kaneton pour un microprocesseur de la famille MIPS (vu en cours).

Contrairement aux microprocesseurs IA-32 qui implementent un
algorithme de remplacement automatique des entr�es TLB bas� sur un
arbre (la page-directory et les page-tables), le microprocesseur MIPS
R3000 n'offre aucune aide au programmeur en cas de TLB-miss.

Lorsque la MMU demande une traduction d'adresse et qu'aucune entr�e
dans les TLB n'est satisfaisante, une exception est g�n�r�e. C'est
alors au noyau de remplir les TLB comme il convient, ou de d�clarer
une erreur fatale.

Notez aussi que la taille des pages sur le MIPS R3000 est fix�e � 4
kilo-octets.

\subsubsection{Modification du manager \emph{as} (1 points)}

Les TLB contiennent des entr�es provenant de tous les espaces
d'adressage confondus. Ainsi, pour indiquer � quel espace d'adressage
une entr�e TLB appartient, un champ identifiant de 6 bits est pr�sent
dans chacune des entr�es.

\begin{enumerate}
\item Allez-vous devoir faire des modifications dans la partie d�pendante de
\emph{as} ?\\
Si oui, lesquelles ?

\item En quoi le fait d'encoder le champ identifiant sur 6 bits peut-il
poser probl�me ?
\end{enumerate}

\textbf{Pour la suite de l'exercice, vous ne devez pas considerer le
        probl�me identifi� ci-dessus.}


\subsubsection{Implementation du gestionnaire de TLB-miss (3 points)}

Pour cette question, nous vous fournissons des fonctions de la
biblioth�que MIPS. Ces fonctions (de bas-niveau) sont d�crites en
annexe.

Ecrivez le code du gestionnaire de TLB-miss. Vous pouvez bien entendu
vous appuyer sur les fonctions donn�es ci-dessus et sur toutes les
autres fonctions de l'interface kaneton.

Le prototype du gestionnaire de TLB-miss est le suivant :

\function{tlb\_miss}{(i\_as asid)}
{
\begin{itemize}
\item
  \emph{asid} est l'identifiant de l'espace d'adressage sur lequel
  s'est produit l'erreur
\end{itemize}
}

\subsubsection{Implementation de \emph{mipsr3000\_region\_release} (3 points)}

Ecrivez le code d�pendant de la fonction \emph{region\_release}. Vous
pouvez utiliser les fonctions de l'interface kaneton et celles de la
biblioth�que MIPS.

\subsubsection{Am�lioration des performances (bonus)}

Indiquez pourquoi votre code n'est pas suffisament
performant. Proposez une (ou plusieurs) solution(s) pour gagner en
performances.
