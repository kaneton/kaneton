\section*{Exercice 3 : Pagination sur MIPS (6 points)}

Dans cet exercice, vous allez implementer le manager \emph{region} de
kaneton pour un microprocesseur de la famille MIPS (vu en cours).

Contrairement aux microprocesseurs IA-32 qui implementent un
algorithme de remplacement automatique des entr�es TLB bas� sur un
arbre (la page-directory et les page-tables), le microprocesseur MIPS
R3000 n'offre aucune aide au programmeur en cas de TLB-miss.

Lorsque la MMU demande une traduction d'adresse et qu'aucune entr�e
dans les TLB n'est satisfaisante, une exception est d\'eclench\'ee. C'est
alors au noyau de remplir les TLB comme il convient, ou de d�clarer
une erreur fatale.

{\bf Note :} la taille des pages sur le MIPS R3000 est fix�e � 4ko.


\begin{description}
\item {\bf Impl\'ementation du gestionnaire de TLB-miss}

\begin{enumerate}
\item Les TLB contiennent des entr�es provenant de tous les espaces
d'adressage confondus. Ainsi, pour indiquer � quel espace d'adressage
une entr�e TLB appartient, chaque entr\'ee contient un identifiant
cod\'e sur 6 bits.\\
\\
Allez-vous devoir faire des modifications dans la partie d�pendante de
\emph{as} ?\\
Si oui, lesquelles ?\\
\item En quoi le fait d'encoder le champ identifiant sur 6 bits peut-il
poser probl�me ?\\
\textbf{Pour la suite de l'exercice, vous ne devez pas consid\'erer le
        probl�me identifi� ci-dessus.}\\


\item Ecrivez le code du gestionnaire de TLB-miss.\\
Vous vous appuyerez sur les fonctions de la biblioth\`eque MIPS et de
l'interface kaneton (voir annexe).\\
\\
Le prototype du gestionnaire de TLB-miss est le suivant :

\function{tlb\_miss}{(i\_as asid)}
{
\begin{itemize}
\item
  \emph{asid} est l'identifiant de l'espace d'adressage sur lequel
  s'est produit l'erreur
\end{itemize}
}

\item Ecrivez le code d�pendant de la fonction {\em region\_release}. Vous
pouvez utiliser les fonctions de l'interface kaneton et celles de la
biblioth�que MIPS.\\
\end{enumerate}


\item {\bf Am�lioration des performances (bonus)}

Indiquez pourquoi votre code n'est pas suffisament
performant. Proposez une (ou plusieurs) solution(s) pour gagner en
performances.

\end{description}
