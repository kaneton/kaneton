\section*{Exercice 3 : Pagination (7 points)}

\begin{description}
\item {\bf A - Question de cours (1.5 points)}
\begin{enumerate}
\item D\'ecrivez le principe g\'en\'eral de la traduction d'adresses en m\'emoire pagin\'ee.\\
 Vous d\'etaillerez notamment le fonctionnement interne de la MMU.

\begin{correction}

La pagination consiste � diviser la m�moire physique en zones de
taille �gale nomm�es pages. Sur certaines architectures, il peut y
avoir plusieurs tailles de page.

Chaque processus dispose alors d'un ensemble de pages qu'il peut
adresser, on l'appelle l'espace d'adressage (ou espace d'adressage
virtuel).

L'unit� de gestion de m�moire (MMU) du microprocesseur traduit les
adresses virtuelles en adresses physiques, en suivant certaines r�gles
d�crites par le microprocesseur lui m�me ou par le noyau.

On appelle TLB un cache de traduction d'adresse. Ce cache d�crit la
correspondance d'une page dans la m�moire virtuelle d'un espace
d'adressage avec une page en m�moire physique. Ce cache est interrog�
� chaque traduction par la MMU.

\end{correction}

\end{enumerate}
\end{description}

\begin{description}
\item {\bf B - Impl\'ementation de la pagination sur MIPS (5.5 points)}\\
\\
Dans cet exercice, vous allez implementer le manager \emph{region} de
kaneton pour un microprocesseur de la famille MIPS (vu en cours).\\
\\
Contrairement aux microprocesseurs IA-32 qui implementent un
algorithme de remplacement automatique des entr�es TLB bas� sur un
arbre (la page-directory et les page-tables), le microprocesseur MIPS
R3000 n'offre aucune aide au programmeur en cas de TLB-miss.\\
\\
Lorsque la MMU demande une traduction d'adresse et qu'aucune entr�e
dans les TLB n'est satisfaisante, une exception est d\'eclench\'ee. C'est
alors au noyau de remplir les TLB comme il convient, ou de d�clarer
une erreur fatale.\\
\\
{\bf Note :} la taille des pages sur le MIPS R3000 est fix�e � 4ko.\\

\begin{enumerate}
\item Les TLB contiennent des entr�es provenant de tous les espaces
d'adressage confondus. Ainsi, pour indiquer � quel espace d'adressage
une entr�e TLB appartient, chaque entr\'ee contient un identifiant
cod\'e sur 6 bits.\\
\\
Allez-vous devoir faire des modifications dans la partie d�pendante de
\emph{as} ?\\
Si oui, lesquelles ?\\

\begin{correction}

Nous allons devoir attribuer des identifiants sur 6 bits pour chaque
espace d'adressage. Il faut donc ajouter un champ \emph{id} dans la
partie d�pendante de l'objet \emph{o\_as}.

Cette valeur sera attribu�e par le code d�pendant de \emph{as\_reserve}.

\end{correction}

\item En quoi le fait d'encoder le champ identifiant sur 6 bits peut-il
poser probl�me ?\\
\textbf{Pour la suite de l'exercice, vous ne devez pas consid\'erer le
        probl�me identifi� ci-dessus.}\\

\begin{correction}

6 bits permettent 64 valeurs, ce qui implique qu'il ne peut y avoir
plus de 64 espaces d'adressages (donc 64 t�ches) simultan�ment. Au
dela, il faut alors user d'astuces pour g�rer le TLB.

\end{correction}

\item Ecrivez le code du gestionnaire de TLB-miss.\\
Vous vous appuyerez sur les fonctions de la biblioth\`eque MIPS et de
l'interface kaneton (voir annexe).\\
\\
Le prototype du gestionnaire de TLB-miss est le suivant :

\function{tlb\_miss}{(i\_as asid)}
{
\begin{itemize}
\item
  \emph{asid} est l'identifiant de l'espace d'adressage sur lequel
  s'est produit l'erreur
\end{itemize}
}


\begin{correction}

\end{correction}

\item Ecrivez le code d�pendant de la fonction {\em region\_release}. Vous
pouvez utiliser les fonctions de l'interface kaneton et celles de la
biblioth�que MIPS.\\

\begin{correction}

\end{correction}

\end{enumerate}


\item {\bf C - Am�lioration des performances (bonus)}

\begin{enumerate}
\item Indiquez pourquoi votre code n'est pas suffisament performant.\\
  Proposez une (ou plusieurs) solution(s) pour gagner en
performances.

\begin{correction}

\end{correction}

\end{enumerate}

\end{description}
