\section*{Exercice 4 : Copy-on-write (6 points)}

Le copy-on-write est une astuce d'optimisation tr\`es utilis�e dans
les noyaux modernes dont le principe est le suivant :\\
\\
 lorsqu'on clone un espace d'adressage (par exemple dans le cas d'un appel syst�me
\emph{fork}), les zones r�serv�es en m�moire physique ne sont pas
dupliqu�es. Seules les correspondances (mappings) entre la m�moire
virtuelle et la m�moire physique sont recopi�es.

Par la suite, c'est uniquement lors d'une �criture dans une zone
m�moire que le segment de m�moire physique est recopi� (afin qu'il y
en ait un exemplaire par espace d'adressage).

Dans cet exercice, nous simplifions le probl�me en consid�rant
qu'\textbf{un espace d'adressage ne pourra �tre dupliqu� qu'une unique
fois}.\\
\begin{description}
\item {\bf Impl�mentation du gestionnaire de page-fault}

\begin{enumerate}
\item Comment faut-il modifier les objets {\em o\_as}, {\em o\_segment} et
{\em o\_region} si l'on veut impl\'ementer le copy-on-write dans kaneton ?

\item Quels autres changements devrez-vous Vous indiquerez aussi les modifications � faire dans les op�rations
existantes pour prendre en compte ou remplir les champs ajout�s.




\item Ecrivez le code du gestionnaire de page-fault. Vous pouvez employer
toutes les fonctions de l'interface kaneton. Votre code doit �tre {\bf ind�pendant
de l'architecture}.\\
\\
{\bf Note:} si votre code n\'ecessite une partie d�pendante de
l'architecture (pour mettre-�-jour des TLB par exemple), vous
pr\'eciserez \textbf{sans l'impl�menter} le comportement de ce code.

Le prototype du gestionnaire de page-fault est le suivant :

\function{page\_fault}{(i\_as asid, t\_vaddr address, t\_bool is\_write\_fault)}{}
\begin{itemize}
\item \emph{asid} est l'identifiant de l'espace d'adressage sur lequel
  s'est produit l'erreur
\item \emph{address} est l'adresse virtuelle ayant provoqu� l'erreur
\item \emph{is\_write\_fault} indique si l'erreur est due � une tentative
  d'�criture � une adresse marqu�e en lecture seule.\\
\end{itemize}
\end{enumerate}


\item {\bf Impl�mentation sans limitations (bonus)}

\begin{enumerate}
\item En quoi le cas ou un espace d'adressage peut �tre dupliqu� plusieurs
fois est-il plus complexe � g�rer ?

\item Proposez les modifications pour g�rer le copy-on-write quelque soit le
nombre de clonages de l'espace d'adressage effectu�.
\end{enumerate}
\end{description}
