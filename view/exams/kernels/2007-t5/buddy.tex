\section*{Exercice 2 : Binary buddy system (3 points)}

Un consid�re un allocateur de m�moire grain-fin de type \emph{malloc}
bas� sur un buddy system binaire d'ordre 12 (qui peut allouer des blocs
d'une taille maximale de 2$^{12}$ octets).

L'allocateur dispose initialement d'un bloc libre de 4 ko.

Le programme effectue successivement des appels � \emph{malloc} avec
les tailles suivantes : 126, 510, 1200, 10, 1000 et 498 octets.

\begin{enumerate}
\item Donnez l'�tat de l'allocateur, c'est-�-dire l'ensemble des blocs
libres et occup�s apr�s chacune de ces allocations.

\item Calculez le taux de fragmentation de la m\'emoire apr\'es ces 6
allocations.

\item On veut maintenant allouer un bloc de 2000 octets.
Que se passe-t-il lors cette derni�re allocation ? Que pr�conisez-vous ?
\end{enumerate}
