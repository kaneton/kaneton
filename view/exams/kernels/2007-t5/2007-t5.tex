%%
%% licence       kaneton licence
%%
%% project       kaneton
%%
%% file          /home/buckman/crypt/kaneton/view/exams/kernels/2007-t5/2007-t5.tex
%%
%% created       julien quintard   [fri dec  2 22:25:51 2005]
%% updated       matthieu bucchianeri   [mon feb 12 23:37:26 2007]
%%

%
% template
%

%%
%% licence       kaneton licence
%%
%% project       kaneton
%%
%% file          /home/mycure/kaneton/view/templates/exam.tex
%%
%% created       julien quintard   [fri dec  2 22:20:57 2005]
%% updated       julien quintard   [mon feb 20 23:01:50 2006]
%%

%
% compile mode
%

%
% ---------- header -----------------------------------------------------------
%
% project       kaneton
%
% license       kaneton
%
% file          /home/mycure/kane...w/talk/presentations/kaneton/kaneton.tex
%
% created       julien quintard   [mon may 14 21:02:29 2007]
% updated       julien quintard   [sun feb  6 20:50:37 2011]
%

%
% ---------- setup ------------------------------------------------------------
%

%
% path
%

\def\path{../../..}

%
% template
%

%
% ---------- header -----------------------------------------------------------
%
% project       kaneton
%
% license       kaneton
%
% file          /home/mycure/kaneton/view/template/talk.tex
%
% created       julien quintard   [wed may 16 18:17:37 2007]
% updated       julien quintard   [fri may 23 19:28:10 2008]
%

%
% ---------- class ------------------------------------------------------------
%

\documentclass[8pt]{beamer}

%
% ---------- common -----------------------------------------------------------
%

\input{\path/package/opk/presentation.tex}


%
% title
%

\title{kaneton}

%
% document
%

\begin{document}

%
% title frame
%

\begin{frame}
  \titlepage
\end{frame}

%
% outline frame
%

\begin{frame}
  \frametitle{Outline}

  \tableofcontents
\end{frame}

%
% ---------- text -------------------------------------------------------------
%


%
% overview
%

\section{Overview}

% 1)

\begin{frame}
  \frametitle{Introduction}

  \term{kaneton} is an educational project intended for students to undertake
  in order to learn about operating system internals.
\end{frame}

% 2)

\begin{frame}
  \frametitle{History}

  \begin{itemize}
    \item[2004]
      \name{Julien Quintard} and \name{Jean-Pascal Billaud} decide
      to introduce an optional low-level programming course to first-year
      enginnering students, now known as \name{kastor};
    \item[2004]
      The course having been well received, \name{SRS - Syst\`emes, R\'eseaux,
      S\'ecurit\'e} students ask them to give such an introductory course the
      same year.
    \item[2005]
      The authorization is given to them to teach a kernel development course
      to \name{SRS} students, from January to October. They therefore decide
      to provide students with the design of a microkernel and let the students
      develop it from scratch, their way. \name{kaneton} is born.
    \item[2006]
      After \name{Jean-Pascal Billaud} fled to \name{VMWare}, \name{Julien
      Quintard} started developing a reference implementation and gave
      students this year a skeleton they had to complete. In addition,
      \name{Cedric Aubouy} and \name{Renaud Lienhart} joined the teaching
      team this year.

      \-

      Besides, the \name{LSE - Laboratoire Syst\`eme EPITA} joined the project
      by putting two students on the development of the \name{kaneton}
      research implementation. \name{Matthieu Bucchianeri} and \name{Renaud
      Voltz} thus joined the project.
  \end{itemize}
\end{frame}


% 3)

\begin{frame}
  \frametitle{History}

  \begin{itemize}
    \item[2007]
      This year, \name{Matthieu Bucchianeri} and \name{Renaud Voltz} took
      over the project for a year by lecturing the course and managing the
      project.

      \-

      \name{Julian Pidancet} and \name{Pierre Duteil} joined the project
      as part of the \name{LSE} but \name{Pierre Duteil} had to leave the
      project. Therefore, \name{Elie Bleton}, who was working at the
      \name{LRDE} before, joined the project.
    \item[2008]
      \name{Julian Pidancet} and \name{Elie Bleton} took over this year
      while \name{Laurent Lec} and \name{Nicolas Grandemange} joined as part
      of the \name{LSE}.

      \-

      At the end of this year, after problems with some students as well as
      conflicts with the \name{LSE}, \name{kaneton} maintainers decided not
      to work with the laboratory anymore.
    \item[2009]
      \name{EPITA} alumni were contacted and joined the educational project
      including \name{Francois Goudal}, \name{Benoit Marcot},
      \name{Enguerrand Raymond}, \name{Jean Guyader} but also
      \name{Fabien Le-Mentec}, an \name{EPITECH} alumnus.
  \end{itemize}
\end{frame}

% 4)

\begin{frame}
  \frametitle{Model}

  The project consists for students to fill in some missing parts of the
  kernel.

  \-

  However, note that, unlike \name{Tiger}, the missing parts will never
  be two lines long.

  \-

  Indeed, in \name{kaneton}, students are asked to implement a feature, say,
  providing memory management. Thus, students are free, in a certain way,
  to implement such a feature as they wish.

  \-

  Since the testing usually consists in verifying that the kernel is able
  to provide the functionality, students should be, most of the time, able
  to implement whatever algorithms \etc{} they wish.
\end{frame}

% 5)

\begin{frame}
  \frametitle{People}

  Let's present the people working on the educational project from where they
  studied to what they are now doing:

  \begin{itemize}
    \item
      \name{Francois Goudal};
    \item
      \name{Benoit Marcot};
    \item
      \name{Jean Guyader};
    \item
      \name{Baptiste Afsa}
    \item
      \name{Louis Vatier}; and
    \item
      \name{Julien Quintard}.
  \end{itemize}
\end{frame}

% 6)

\begin{frame}
  \frametitle{Project}

  \name{kaneton} is an important assignment of the \name{SRS}/\name{GISTR}
  curriculum and, as such, must be taken seriously.

  \-

  Especially, in the last years, \name{EPITA} decided to reduce the duration
  of the project to \term{three} months.

  \-

  As such, the other assignments imposed by the specializations in this
  period have been reduced so that students can focus on \name{kaneton}.
\end{frame}

%
% design
%

\section{Design}

% 1)

\begin{frame}
  \frametitle{Overview}

  The kaneton kernel is very different from the kernels you might be
  familiar with, especially the well-known \name{Windows}, \name{Linux},
  \name{BSD} and so forth.
\end{frame}

% 2)

\begin{frame}
  \frametitle{Microkernel}

  First, kaneton is a microkernel, making it modular from the design
  perspective as well as providing properties such as security.
\end{frame}

% 3)

\begin{frame}
  \frametitle{Distributed Computing}

  kaneton has been designed from the ground up for providing the operarting
  system advanced distributed computing features.
\end{frame}

% 4)

\begin{frame}
  \frametitle{Portability}

  kaneton has been designed with portability in mind, especially through
  a specific portability system that perfectly fits the kernel design.
\end{frame}

% 5)

\begin{frame}
  \frametitle{Organisation}

  Besides being a microkernel, kaneton is well organised in the inside,
  splitting functionalites into objects and managers.
\end{frame}

%
% stages
%

\section{Stages}

% 1)

\begin{frame}
  \frametitle{k0}

  The first project, named \term{k0}, consists for students to learn
  about low-level programming.

  \-

  This project comes with a lecture regarding the boot system as well
  as a practical session.

  \-

  \name{Francois Goudal} will be in charge of this stage which will last for
  a week.
\end{frame}

% 2)

\begin{frame}
  \frametitle{k1}

  \term{k1} consists for students to provide the kaneton microkernel the
  capacity to handle events.

  \-

  During this stage, a lecture on general kernel principles and a lecture on
  interrupts will be taught.

  \-

  \name{Julien Quintard} will be in charge of this stage which will last
  for a single week.

  \-

  Note that, starting with \name{k1}, the student snapshot will be used which
  provide students a development environment, making kernel development easier.
\end{frame}

% 3)

\begin{frame}
  \frametitle{k2}

  \term{k2} consists for students to provide the kaneton microkernel a
  memory management unit so that applications as well as the kernel itself
  can reserve, share \etc{} memory.

  \-

  During this stage, a lecture on portability as well as lectures on
  memory management will be taught.

  \-

  \name{Francois Goudal} will be in charge of this stage which will last
  for three weeks.
\end{frame}

% 4)

\begin{frame}
  \frametitle{k3}

  In \term{k3}, students will have to provide kaneton with execution contexts
  such that the kernel can execute multiple threads at the \textit{same} time.

  \-

  Lectures, during this stage, will discuss topics such as interrupts,
  concurrency, multi-processing, scheduling \etc{}

  \-

  \name{Benoit Marcot} will be in charge of this stage which will last for
  three weeks.
\end{frame}

% 5)

\begin{frame}
  \frametitle{Evaluation}

  For every stage, students will have the possibility to test their
  implementation by running, a limited number of times, the test suite used
  for evaluating their work.

  \-

  Besides, at the end of each stage, after submission, the kaneton test system
  will run the test suite and issue a mark according to the test results.

  \-

  Additionally, an exam will take place at the end of the semester to make
  sure that the notions tackled throughout the course are well understood
  by every student.
\end{frame}

%
% tools
%

\section{Tools}

% 1)

\begin{frame}
  \frametitle{Overview}

  The kaneton educational project relies on tools, sometimes developed
  internally.
\end{frame}

% 2)

\begin{frame}
  \frametitle{Web Site}

  The web site contains the documentation including design papers,
  the assignments \etc{} but also hosts the wiki which should be
  the starting point for every student seeking information.

  \-

  Noteworthy is that the wiki contains courses regarding the
  inline assembly, linking, pre-processing and so on. Students
  are invited to read them all as they will come handy when
  developing the kaneton stages.

  \-

  \name{Julien Quintard} should be contacted for requests regarding the
  web site and wiki.
\end{frame}

% 3)

\begin{frame}
  \frametitle{Snapshot}

  The student snapshot has been automatically generated from the current
  kaneton implementation.

  \-

  \name{Francois Goudal} is in charge of this process, hence should be
  contacted if you believe there is a mistake.
\end{frame}

% 4)

\begin{frame}
  \frametitle{Cheat}

  Every student's kaneton implementation will be tested to make sure that
  students did not cheat by relying on implementations by previous or
  current students.

  \-

  \name{Julien Quintard} is in charge of this tool.
\end{frame}

% 5)

\begin{frame}
  \frametitle{Test}

  Students' implementation will be tested in a real environment by applying
  a complete test suite; hence, validating the implementation's behaviour.

  \-

  \name{Jean Guyader} is responsible of this tool and should be contacted
  if necessary.
\end{frame}

%
% information
%

\section{Information}

% 1)

\begin{frame}
  \frametitle{Support}

  \begin{enumerate}
    \item
      \term{Website}

      \-

      You will find on \location{http://kaneton.opaak.org} documents regarding
      the project from the design to the implementation;
    \item
      \term{Wiki}

      \-

      The wiki \location{http://wiki.opaak.org} is the best way to get
      technical information as well as to help other students by adding
      and/or improving pages' contents;
    \item
      \term{Mailing-List}

      \-

      The kaneton educational students mailing-list
      \location{students@kaneton.opaak.org} will be used by teachers as
      an official means for communicating with students.

      \-

      Therefore, every student should subscribe to this mailing-list by sending
      an email to \location{students+subscribe@kaneton.opaak.org}.

      \-

      It is not allowed to post code on the mailing list, or give pointers to
      code in the snapshot that would provide obvious solution to somebody's
      question.
  \end{enumerate}
\end{frame}

% 2)

\begin{frame}
  \frametitle{Groups}

  Except for \name{k0} which is an individual project, the other projects
  from \name{k1} to \name{k3} are done in groups of \term{two} students.

  \-

  Every group is expected to send an email to
  \location{admin@opaak.org}.

  \-

  Note that we will use students' \name{EPITA} email addresses. As such,
  make sure that you check this email box.
\end{frame}

% 3)

\begin{frame}
  \frametitle{Reliance}

  As for \name{Tiger}, every stage depends on the previous one, except
  for \name{k0}.

  \-

  As such, test suites from the previous stages will also be used for both
  testing and marking.

  \-

  Students should therefore make sure to use their test permissions for making
  sure to fix the bugs of previous stages so that such bugs do not impact
  on the current stage results, hence mark.
\end{frame}

% 4)

\begin{frame}
  \frametitle{Machine}

  This year, the machine used by the kaneton educational project will consists
  of the \term{IBM-PC} platform coupled with the \term{IA-32} microprocessor
  architecture \ie{} the most common hardware system on the market.

  \-

  Although it is always best to test your implementation on a real machine,
  it takes time to reboot a real computer. You should therefore use an
  emulator such a \name{QEMU} or \name{Bochs} as they will enable you to
  test your kernel very quickly but they will also let you develop on
  a non-\name{IBM-PC}/\name{IA-32} machine such as a \name{Mac} for example.
\end{frame}

%
% conclusion
%

\section{Conclusion}

% 1)

\begin{frame}
  \frametitle{Concepts}

  Throughout the project, you will learn so many things from terminology,
  to how a computer boots, how the kernel controls the hardware and how it
  provides abstractions as basic as execution contexts.

  \-

  At the end of the project, you will definitely know that nothing is magic
  but purely logic and often actually very simple.
\end{frame}

% 2)

\begin{frame}
  \frametitle{Implementation}

  Although, starting the project by learning how to make a computer execute
  your code, you will end up, after three months, with a running kernel
  and operating system capable of executing programs, the whole on real
  hardware like the machine you have at home.
\end{frame}

% 3)

\begin{frame}
  \frametitle{Changes}

  Over the years, the project has greatly evolved, from a no-implementation
  project, to a reference-based project.

  \-

  However, being a project developed by volunteers willing to dedicate some
  time so that other students can learn, many things are missing and/or
  can be improved including the lectures but also the project implementation.

  \-

  In conclusion, keep in mind that the project exists only because of people
  willing to transfer their knowledge and please respect their effort.
\end{frame}

% 4)

\begin{frame}
  \frametitle{Fun}

  But most of all, kaneton should be about learning through fun!
\end{frame}

% 5)

\begin{frame}
  \frametitle{Reminder}

  Remember to perform the following tasks:

  \begin{itemize}
    \item
      Send an email to \location{admin@opaak.org} regarding the composition
      of your group, before \textbf{Wednesday 16th 2pm} or you will be put
      in a group by force;
    \item
      Subscribe to the students mailing-list
      \location{students@kaneton.opaak.org} by sending an email to
      \location{students+subscribe@kaneton.opaak.org};
    \item
      Watch closely the \name{Wiki} at \location{http://wiki.opaak.org} by
      subscribing the \name{RSS} feed for example;
    \item
      We advise SRS/GISTR lab roots to set up a \name{Xen}-based environment
      as testing on emulators only will become difficult over time;
    \item
      Students must have a ``rack'' containing a \name{POSIX}-compilant
      operating system for the \name{k0} practical session.
  \end{itemize}
\end{frame}

\end{document}


%
% class
%

\documentclass[10pt,a4wide]{article}

%
% packages
%

\usepackage[english]{babel}
\usepackage[T1]{fontenc}
\usepackage{a4wide}
\usepackage{graphicx}
\usepackage{fancyheadings}
\usepackage{multicol}
\usepackage{indentfirst}
\usepackage{color}
\usepackage{ifthen}
\usepackage{comment}
\usepackage{verbatim}
\usepackage{aeguill}

\pagestyle{fancy}

\setlength{\footrulewidth}{0.3pt}
\setlength{\parindent}{0.3cm}
\setlength{\parskip}{2ex plus 0.5ex minus 0.2ex}

%
% correction environment
%

\newenvironment{correction}%
   {
     \ifthenelse
	 {
	   \equal{\kaneton-latex}{subject}
	 }
	 {
	   \comment
	 }
	 {
	   \textbf{\color{red}{ ----- correction}}
	 }
   }%
   {
     \ifthenelse
	 {
	   \equal{\kaneton-latex}{subject}
	 }
	 {
	   \endcomment
	 }
	 {
	   \textbf{\color{red}{ ----- /correction}}
	 }
   }

%
% header
%

\rfoot{\scriptsize{Exam}}

\date{\scriptsize{\today}}


%
% header
%

\lhead{\scriptsize{2007}}

%
% title
%

\title{Examen de noyaux et syst�mes d'exploitation (kaneton)}

%
% authors
%

\author{\small{Matthieu Bucchianeri, Renaud Voltz}}

%
% document
%

\begin{document}

%
% title
%

\maketitle

%
% --------- information -------------------------------------------------------
%

\begin{center}

\textbf{Ordinateurs, PDA et t�l�phones non autoris�s}

\textbf{Documents autoris�s}

\textbf{Dur�e 2 heures}

\scriptsize{Une copie mal pr�sent�e et bourr�e de fautes d'orthographe
            sera p�nalis�e.}

\scriptsize{Le bar�me est donn� � titre indicatif, en cas de modifications,
            aucune contestation ne sera possible}

\end{center}

%
% --------- text --------------------------------------------------------------
%

\vspace{3cm}

\section{Exercice 1 : G�n�ralit�s -- \emph{8 points}}

Ces petits exercices vous offrent la possibilit� de nous montrer votre
bonne connaissance et compr�hension du cours.

\section*{Exercice 1 : Assembleur (3 points)}

Le code assembleur suivant est execut� sur un microprocesseur Intel
80386 en mode r�el.

Indiquez, pour chaque instruction, ce qu'elle fait.

\begin{verbatim}
MOV AX, 0x1200
MOV ES, AX
MOV BX, 0x4000
MOV AL, [ES:BX]
TEST AL, AL
JZ end
ADD AL, 42
MOV AH, 0xE
INT 0x10
\end{verbatim}

\subsection{ICU en cascade (3 points)}

Les microprocesseurs de la famille IA-32 ne disposent que d'un fil
d'IRQ nomm� \textbf{INTA}.

Afin de supporter plusieurs sources d'interruptions externes, on
utilise un controlleur d'interruption appel� PIC (\emph{Programmable
Interrupt Controller}, chip 8259). Ce PIC dispose de 8 entr�es
(\textbf{IR$_{0}$} - \textbf{IR$_{7}$}) et d'une unique sortie
\textbf{INT}. Il dispose aussi d'un fil \textbf{INTACK} qui sert �
l'acquittement par le microprocesseur, d�clencheant la mise sur le bus
de donn�es du num�ro de l'IRQ re�ue.

On considerera que cette valeur mise sur le bus peut �tre
automatiquement additionn�e � une autre par le PIC (ne vous souciez
donc pas de cette derni�re).

On souhaite utiliser 2 PIC en cascade pour disposer de plus de 8
lignes d'IRQ. Proposez un montage r�pondant � ce probl�me.

\subsection{Binary buddy system (2 points)}

Un consid�re un allocateur de m�moire grain-fin de type \emph{malloc}
bas� sur un buddy system binaire d'ordre 12 (qui peut allouer des blocs
d'une taille maximale de 2$^{12}$ octets).

L'allocateur dispose initialement d'un bloc libre de 4 ko.

Le programme effectue successivement des appels � \emph{malloc} avec
les tailles suivantes : 126, 510, 1200, 10, 1000 et 498 octets.

\begin{enumerate}
\item Donnez l'�tat de l'allocateur, c'est-�-dire l'ensemble des blocs
libres et occup�s apr�s chacune de ces allocations.

\item On veut maintenant allouer un bloc de 2000 octets.
Que se passe-t-il lors cette derni�re allocation ? Que pr�conisez-vous ?
\end{enumerate}


\section{Exercice 2 : Gestion de m�moire -- \emph{12 points}}

Les deux parties suivantes sont ind�pendantes.

Vous trouverez en annexe la description de l'interface kaneton et des
objets qu'il vous sera n�cessaire d'utiliser.

Lorsqu'il vous est demand� d'�crire du code, vous pourrez n�gliger la
gestion des erreurs \textbf{uniquement dans les cas o� elle n'apporte
rien au probl�me}. Comprenez par l� qu'on consid�re (par exemple) qu'un
\emph{segment\_get} sur un identifiant de segment valide ne renverra
pas d'erreur. En revanche, vous \textbf{devez gerer les erreurs qui
peuvent conduire � des segfault}.

\subsection{Impl�mentation de \emph{region} sur MIPS}

Dans cet exercice, vous allez implementer le manager \emph{region} de
kaneton pour un microprocesseur de la famille MIPS (vu en cours).

Contrairement aux microprocesseurs IA-32 qui implementent un
algorithme de remplacement automatique des entr�es TLB bas� sur un
arbre (la page-directory et les page-tables), le microprocesseur MIPS
R3000 n'offre aucune aide au programmeur en cas de TLB-miss.

Lorsque la MMU demande une traduction d'adresse et qu'aucune entr�e
dans les TLB n'est satisfaisante, une exception est g�n�r�e. C'est
alors au noyau de remplir les TLB comme il convient, ou de d�clarer
une erreur fatale.

Notez aussi que la taille des pages sur le MIPS R3000 est fix�e � 4
kilo-octets.

\subsubsection{Modification du manager \emph{as} (1 points)}

Les TLB contiennent des entr�es provenant de tous les espaces
d'adressage confondus. Ainsi, pour indiquer � quel espace d'adressage
une entr�e TLB appartient, un champ identifiant de 6 bits est pr�sent
dans chacune des entr�es.

Allez-vous devoir faire des modifications dans la partie d�pendante de
\emph{as} ? Si oui, lesquelles ?

En quoi le fait d'encoder le champ identifiant sur 6 bits peut-il
poser probl�me ?

\textbf{Pour la suite de l'exercice, vous ne devez pas considerer le
        probl�me identifi� ci-dessus.}

\subsubsection{Implementation du gestionnaire de TLB-miss (3 points)}

Pour cette question, nous vous fournissons des fonctions de la
biblioth�que MIPS. Ces fonctions (de bas-niveau) sont d�crites en
annexe.

Ecrivez le code du gestionnaire de TLB-miss. Vous pouvez bien entendu
vous appuyer sur les fonctions donn�es ci-dessus et sur toutes les
autres fonctions de l'interface kaneton.

Le prototype du gestionnaire de TLB-miss est le suivant :

\function{tlb\_miss}{(i\_as asid)}
{
\begin{itemize}
\item
  \emph{asid} est l'identifiant de l'espace d'adressage sur lequel
  s'est produit l'erreur
\end{itemize}
}

\subsubsection{Implementation de \emph{mipsr3000\_region\_release} (3 points)}

Ecrivez le code d�pendant de la fonction \emph{region\_release}. Vous
pouvez utiliser les fonctions de l'interface kaneton et celles de la
biblioth�que MIPS.

\subsubsection{Am�lioration des performances (bonus)}

Indiquez pourquoi votre code n'est pas suffisament
performant. Proposez une (ou plusieurs) solution(s) pour gagner en
performances.


\section*{Exercice 4 : Copy-On-Write (7 points)}

{\bf Rappel :}

Le copy-on-write est une astuce d'optimisation tr\`es utilis�e dans
les noyaux modernes dont le principe est le suivant.

Lorsqu'on clone un espace d'adressage (par exemple dans le cas d'un appel syst�me
\emph{fork}), les zones r�serv�es en m�moire physique ne sont pas
dupliqu�es. Seules les correspondances (mappings) entre la m�moire
virtuelle et la m�moire physique sont recopi�es.

Par la suite, c'est uniquement lors d'une �criture dans une zone
m�moire que le segment de m�moire physique est recopi� (pour qu'il y
en ait un exemplaire par espace d'adressage).

Dans cet exercice, nous simplifions le probl�me en consid�rant
qu'\textbf{un espace d'adressage ne pourra �tre dupliqu� qu'une unique
fois}.\\

\begin{description}
\item {\bf A - Impl�mentation du gestionnaire de page-fault (5.5 points)}

\begin{enumerate}
\item Comment faut-il modifier les objets {\em o\_as}, {\em o\_segment} et
{\em o\_region} si l'on veut impl\'ementer le copy-on-write dans kaneton ?\\
\item Quels autres changements devrez-vous effectuer dans les fonctions
de l'interface kaneton  pour prendre en compte ou remplir les champs ajout�s.\\

\begin{correction}

Il faut ajouter aux objets \emph{o\_segment} un compteur de
r�ference. Ce dernier sera mis � 0 lors de \emph{segment\_reserve} et
incremente et decremente lorsqu'une region sera reserv�e ou
lib�r�e. Un \emph{semgent\_release} n'aura d'effet que si ce compteur
de r�ference est � 0.

Nous aurons aussi besoin de pr�ciser dans l'objet segment les
identifiants des 2 regions (et leur address space respectifs) qui se
partagent le segment.

En ce qui concerne les regions, nous allons ajouter dans
\emph{o\_region} un bool�en indiquant si la region est en
copy-on-write. Ce bool�en sera mis � 0 par \emph{region\_reserve}, et
mis a un par \emph{as\_clone} lors du clonage de la region. Cette
derni�re fonction ne fera plus appel � \emph{segment\_clone} pour
dupliquer la m�moire physique.

\end{correction}

\item \'Ecrivez le code du gestionnaire de page-fault. Vous pouvez employer
toutes les fonctions de l'interface kaneton. Votre code doit �tre {\bf ind�pendant
de l'architecture}.\\
\\
{\bf Note:} si votre code n\'ecessite une partie d�pendante de
l'architecture (pour mettre-�-jour des TLB par exemple), vous
pr\'eciserez \textbf{sans l'impl�menter} le comportement de ce code.

Le prototype du gestionnaire de page-fault est le suivant :\\
\\
{\bf page\_fault}({\em i\_as asid, t\_vaddr address, t\_bool is\_write\_fault)}\\
\begin{itemize}
\item \emph{asid} : identifiant de l'espace d'adressage sur lequel s'est
  produite l'erreur.
\item \emph{address} : adresse virtuelle ayant provoqu� l'erreur.
\item \emph{is\_write\_fault} : indique si l'erreur est d\^ue � une tentative
  d'�criture � une adresse marqu�e en lecture seule.\\
\end{itemize}

\begin{correction}
XXX
\end{correction}

\end{enumerate}

\item {\bf B - Question de cours (1.5 points)}

\begin{enumerate}
\item Expliquez les diff\'erences et les points communs entre le Copy-On-Write
et l'Allocation-On-Demand.\\

\begin{correction}

Le copy-on-write et l'allocation-on-demand sont deux astuces pour
am�liorer la gestion de la m�moire en minimisant le nombre de page
physique allou�es.

Les deux techniques se basent sur la g�n�ration volontaire de
page-fault. Alors que le premier utilise le marquage lecture-seule
d'une entr�e TLB, le second repose sur l'absence d'une entr�e.

Alors que le copy-on-write recopie les pages partag�es mais d�j�
existantes lors de leur premiere modification, l'allocation � la
demande alloue les pages physique lors de leur premier acc�s.

\end{correction}

\end{enumerate}

\item {\bf C - Impl�mentation sans limitations (bonus)}

\begin{enumerate}
\item En quoi le cas ou un espace d'adressage peut �tre dupliqu� plusieurs
fois est-il plus complexe � g�rer ?

\item Proposez les modifications pour g�rer le copy-on-write quelque soit le
nombre de clonages d'un espace d'adressage.

\begin{correction}
XXX
\end{correction}

\end{enumerate}
\end{description}


\section{kaneton as manager}

{\large {\bf as object}}

\begin{verbatim}
       typedef struct
         {
           i_as         asid;

           i_task       tskid;

           i_set        segments;
           i_set        regions;

           machdep_data(o_as);
       }                o_as;
\end{verbatim}

\subsection*{as interface}

\function{as\_vaddr}{(i\_as \argument{id},
                      t\_paddr \argument{physical},
                      t\_vaddr \argument{virtual})}
	 {
	   This function translates a physical address into its virtual
	   address.
	 }

\function{as\_paddr}{(i\_as \argument{id},
                      t\_vaddr \argument{virtual},
                      t\_paddr \argument{physical})}
	 {
	   This function translates a virtual address into its physical
	   address.
	 }

\function{as\_clone}{(i\_task \argument{task},
                      i\_as \argument{old},
                      i\_as* \argument{new})}
	 {
	   This function clones an address space taking care of cloning
	   all the necessary: segments, regions etc\ldots
	 }

\function{as\_reserve}{(i\_task \argument{task},
                        i\_as* \argument{id})}
	 {
	   This function reserves an address space object for the
	   task \argument{task} object.

	   The reserved address space object's identifier is returned
	   in \argument{id}.
	 }

\function{as\_release}{(i\_as \argument{id})}
	 {
	   This function just releases the address space \argument{id}.
	 }

\function{as\_get}{(i\_as \argument{id},
                    o\_as** \argument{o})}
	 {
	   This function should only be used by the as manager, the segment
	   manager and the region manager. It just returns the address space
	   object corresponding to the address space identifier \argument{id}.
	 }



\section{kaneton segment manager}

\subsection*{segment object}

\begin{verbatim}
      typedef struct
        {
          i_segment    segid;

          i_as         asid;

          t_paddr      address;
          t_psize      size;

          t_perms      perms;

          machdep_data(o_segment);
      }                o_segment;
\end{verbatim}

\subsection*{segment interface}

\function{segment\_clone}{(i\_as \argument{as},
                           i\_segment \argument{old},
                           i\_segment* \argument{new})}
	 {
	   This function clones a segment which will then belongs to
	   the address space object \argument{as}. Cloning a segment
	   consists in reserving a new segment with the
	   exact same properties and copying its content.
	 }

\function{segment\_read}{(i\_segment \argument{id},
                          t\_paddr \argument{offset},
                          void* \argument{buffer},
                          t\_psize \argument{size})}
	 {
	   This function reads \argument{size} bytes at offset
	   \argument{offset} from the segment \argument{id}.
	 }

\function{segment\_write}{(i\_segment \argument{id},
                           t\_paddr \argument{offset},
                           const void* \argument{buffer},
                           t\_psize \argument{size})}
	 {
	   This function writes the data of \argument{buffer} into the
	   segment \argument{id}.
	 }

\function{segment\_copy}{(i\_segment \argument{dst},
                          t\_paddr \argument{offd},
                          i\_segment \argument{src},
                          t\_paddr \argument{offs},
                          t\_psize \argument{size})}
	 {
	   This function copies data from segment \argument{src} to
	   segment \argument{dst}.
	 }

\function{segment\_reserve}{(i\_as \argument{as},
                             t\_psize \argument{size},
                             t\_perms \argument{perms},
                             i\_segment* \argument{id})}
	 {
	   This function reserves a segment with specified properties.
	 }

\function{segment\_release}{(i\_segment \argument{id})}
	 {
	   This function releases the segment \argument{id}.
	 }

\function{segment\_perms}{(i\_segment \argument{id},
                           t\_perms \argument{perms})}
	 {
	   This function changes the permissions of the segment \argument{id}.
	 }

\section{kaneton region manager}

\subsection*{region object}

\begin{verbatim}
      typedef struct
        {
          i_region     regid;

          i_segment    segid;

          t_vaddr      address;
          t_paddr      offset;
          t_vsize      size;
          t_opts       opts;

          machdep_data(o_region);
      }                o_region;
\end{verbatim}

\subsection*{region interface}

\function{region\_reserve}{(i\_as \argument{as},
                            i\_segment \argument{segment},
                            t\_paddr \argument{offset},
                            t\_opts \argument{opts},
                            t\_vaddr \argument{address},
                            t\_vsize \argument{size},
                            i\_region* \argument{id})}
	 {
	   This function reserves a region with specified properties.
	 }

\function{region\_release}{(i\_as \argument{as},
                            i\_region \argument{id})}
	 {
	   This function releases the region \argument{id} that belongs
	   to the address space object \argument{as}.
	 }


\newpage

\section{kaneton libmips}

\subsection*{Interface}

\functionn{t\_vaddr}{mips\_bad\_vaddr}{()}
{
  Get the virtual address that caused the fault.
}

\function{mips\_set\_tlb}{(t\_uint32 \argument{entry},
			   t\_uint32 \argument{id},
			   t\_vaddr \argument{vaddr},
			   t\_paddr \argument{paddr},
			   t\_perms  \argument{perms})}
{
  This function fills the entry of index \argument{entry} of the TLB
  with a translation of virtual address \argument{vaddr} in address
  space \argument{id} to the physical address \argument{paddr}.

  Permissions \argument{perms} are associated with the translation.
}

\function{mips\_get\_tlb}{(t\_uint32 \argument{entry},
			   t\_vaddr* \argument{vaddr},
			   t\_uint32* \argument{id})}
{
  This function get the virtual address \argument{vaddr} and the
  address space identifier \argument{id} translated by the entry
  \argument{entry} of the TLB.
}

\function{mips\_del\_tlb}{(t\_uint32 \argument{entry})}
{
  Mark the \argument{entry} of the TLB as invalid (remove it).
}

\functionn{t\_uint32}{mips\_number\_entry}{()}
{
  Get the total number of entries in the TLB.
}

\functionn{t\_uint32}{mips\_random\_entry}{()}
{
  Get the index of a random entry in the TLB that can be replaced.
}

\end{document}

