%
% ---------- header -----------------------------------------------------------
%
% project       kaneton
%
% license       kaneton
%
% file          /home/mycure/kaneton/view/talk/presentations/k4/k4.tex
%
% created       julien quintard   [sat nov 29 20:40:35 2008]
% updated       julien quintard   [mon feb  2 14:41:36 2009]
%

%
% ---------- setup ------------------------------------------------------------
%

%
% path
%

\def\path{../../..}

%
% template
%

%
% ---------- header -----------------------------------------------------------
%
% project       kaneton
%
% license       kaneton
%
% file          /home/mycure/kaneton/view/template/lecture.tex
%
% created       julien quintard   [wed may 16 18:17:26 2007]
% updated       julien quintard   [sun may 18 23:23:40 2008]
%

%
% class
%

\documentclass[8pt]{beamer}

%
% packages
%

\usepackage{pgf,pgfarrows,pgfnodes,pgfautomata,pgfheaps,pgfshade}
\usepackage[T1]{fontenc}
\usepackage{colortbl}
\usepackage{times}
\usepackage{amsmath,amssymb}
\usepackage{graphics}
\usepackage{graphicx}
\usepackage{color}
\usepackage{xcolor}
\usepackage[english]{babel}
\usepackage{enumerate}
\usepackage[latin1]{inputenc}
\usepackage{verbatim}
\usepackage{aeguill}

%
% style
%

\usepackage{beamerthemesplit}
\setbeamercovered{dynamic}

%
% verbatim stuff
%

\definecolor{verbatimcolor}{rgb}{0.00,0.40,0.00}

\makeatletter

\renewcommand{\verbatim@font}
  {\ttfamily\footnotesize\selectfont}

\def\verbatim@processline{
  {\color{verbatimcolor}\the\verbatim@line}\par
}

\makeatother

%
% -
%

\renewcommand{\-}{\vspace{0.4cm}}

%
% date
%

\date{\today}

%
% logos
%

\pgfdeclareimage[interpolate=true,width=34pt,height=18pt]
                {epita}{\path/logo/epita}
\pgfdeclareimage[interpolate=true,width=49pt,height=18pt]
                {upmc}{\path/logo/upmc}
\pgfdeclareimage[interpolate=true,width=25pt,height=18pt]
                {lse}{\path/logo/lse}

\newcommand{\logos}
  {
    \pgfuseimage{epita}
  }

%
% institute
%

\institute
{
  \inst{1} kaneton microkernel project
}

%
% table of contents at the beginning of each section
%

\AtBeginSection[]
{
  \begin{frame}<beamer>
   \frametitle{Outline}
    \tableofcontents[current]
  \end{frame}
}

%
% table of contents at the beginning of each subsection
%

\AtBeginSubsection[]
{
  \begin{frame}<beamer>
   \frametitle{Outline}
    \tableofcontents[current,currentsubsection]
  \end{frame}
}


%
% title
%

\title{k0}

%
% document
%

\begin{document}

%
% title frame
%

\begin{frame}
  \titlepage
\end{frame}

%
% outline frame
%

\begin{frame}
  \frametitle{Outline}

  \tableofcontents
\end{frame}

%
% ---------- text -------------------------------------------------------------
%

%
% introduction
%

\section{Introduction}

% 1)

\begin{frame}
  \frametitle{Overview}

  \name{k0} is the first stage of the kaneton project.

  \-

  Unlike the other future stages, it is not mandatory for the next stages, since GRUB will be used to boot the kaneton snapshot.

\end{frame}

% 2)

\begin{frame}
  \frametitle{Objective}

  The goal of this project is for students to practise low level programming and assembly.

  \-

  This is also the only stage where the real-mode of the Intel CPU will be used, so it gives a full approach on this CPU.

\end{frame}

%
% assignments
%

\section{Assignments}

% 1)

\begin{frame}
  \frametitle{String display}

  The first exercise will consist in writing assembly code to display characters and strings on the screen.

  \-

  It will make you use the BIOS calls.

\end{frame}

% 2)

\begin{frame}
  \frametitle{Registers dump}

  The goal of this exercise is to make you reuse your previous code, and make some more complex things, like converting an integer to a string.

  \-

  The goal is to make you practise the Intel assembly language.

  \-

  You will also have to write a very basic allocation function

\end{frame}

% 3)

\begin{frame}
  \frametitle{Keyboard inputs}

  The goal of this exercise is to practise with BIOS calls, and to write keyboard routines

  \-

  You will have to write two routines, one to get one char, and one to get a line.
\end{frame}

% 4)

\begin{frame}
  \frametitle{Floppy drive}

  This exercise is again using BIOS calls to drive another peripherial : the floppy disk drive. The goal of this exercise is to be reused for the next exercises.

  \-

  You will have to write some code to read the second sector of the floppy (the first one has already been read and loaded in memory by the BIOS).

\end{frame}

% 5)

\begin{frame}
  \frametitle{Operating mode switching}

  This exercise consists in writing a routine that switches the CPU from real mode to protected mode.

  \-

  You will then have to write protected-mode (32-bit) code that prints a string on the screen. Note that BIOS calls can't be used from protected mode.

\end{frame}

% 6)

\begin{frame}
  \frametitle{ELF Loader}

  In this final exercise, we will provide you a basic ELF binary. You will have to put it in the sector 2 of the floppy disk. Your code will have to load this elf binary in the memory, and jump on its entry point, to execute it.

  \-

  This final exercise reuses some of the code you have been writing in previous exercises.

\end{frame}

%
% evaluation
%

\section{Evaluation}

% 1)

\begin{frame}
  \frametitle{Test suite}

  This stage is checked by a test suite.

  \-

  You must provide the symbols that are specified in the assignments, and respect the interface.

  Be careful not to include any of your own test code, since it will conflict with our own tests.

\end{frame}

%
% conclusion
%

\section{Conclusion}

% 1)

\begin{frame}
  \frametitle{Schedule}

  Practical session tomorrow, from 19:00 to 00:00.

  \-

  You will have to finish the exercises before Sunday, 15 at 23:42.

  \-

  Your tarball will have to be uploaded on the Kaneton Epita intranet in due time.
\end{frame}

%
% bibliography
%

\begin{frame}
  \frametitle{Bibliography}

  \bibliographystyle{amsplain}
  \bibliography{\path/bibliography/bibliography}
\end{frame}

\end{document}
