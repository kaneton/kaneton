%
% ---------- header -----------------------------------------------------------
%
% project       kaneton
%
% license       kaneton
%
% file          /home/mycure/kane...w/talk/presentations/kaneton/kaneton.tex
%
% created       julien quintard   [mon may 14 21:02:29 2007]
% updated       julien quintard   [thu feb 11 22:19:59 2010]
%

%
% ---------- setup ------------------------------------------------------------
%

%
% path
%

\def\path{../../..}

%
% template
%

%
% ---------- header -----------------------------------------------------------
%
% project       kaneton
%
% license       kaneton
%
% file          /home/mycure/kaneton/view/template/talk.tex
%
% created       julien quintard   [wed may 16 18:17:37 2007]
% updated       julien quintard   [fri may 23 19:28:10 2008]
%

%
% ---------- class ------------------------------------------------------------
%

\documentclass[8pt]{beamer}

%
% ---------- common -----------------------------------------------------------
%

%
% ---------- header -----------------------------------------------------------
%
% project       kaneton
%
% license       kaneton
%
% file          /home/mycure/kaneton/view/package/opk/presentation.tex
%
% created       julien quintard   [wed may 16 18:17:26 2007]
% updated       julien quintard   [wed apr 22 20:50:07 2009]
%

%
% ---------- packages ---------------------------------------------------------
%

%
% packages
%

\usepackage{pgf,pgfarrows,pgfnodes,pgfautomata,pgfheaps,pgfshade}
\usepackage[T1]{fontenc}
\usepackage{colortbl}
\usepackage{times}
\usepackage{amsmath,amssymb}
\usepackage{multicol}
\usepackage{graphics}
\usepackage{graphicx}
\usepackage{color}
\usepackage{xcolor}
\usepackage[english]{babel}
\usepackage{enumerate}
\usepackage[latin1]{inputenc}
\usepackage{verbatim}
\usepackage{aeguill}

%
% ---------- date -------------------------------------------------------------
%

%
% date
%

\date{\scriptsize{\today}}

%
% ---------- configuration ----------------------------------------------------
%

%
% style
%

\usepackage{beamerthemesplit}
\setbeamercovered{dynamic}

%
% table of contents at the beginning of each section
%

\AtBeginSection[]
{
  \begin{frame}<beamer>
   \frametitle{Outline}
    \tableofcontents[current]
  \end{frame}
}

%
% table of contents at the beginning of each subsection
%

\AtBeginSubsection[]
{
  \begin{frame}<beamer>
   \frametitle{Outline}
    \tableofcontents[current,currentsubsection]
  \end{frame}
}

%
% ---------- commands ---------------------------------------------------------
%

%
% -
%

\renewcommand{\-}{\vspace{0.4cm}}

%
% note
%
% this command is used for making little notes, tips, hints etc.
%
% #1:   text
%

\renewcommand\note[1]
{
  \textbf{\color{c_note}Note}

  \begin{tabular}{p{0.2cm}p{13.8cm}}
  & #1
  \end{tabular}
}

%
% example
%
% this command is used for creating examples.
%
% #1:   text
%

\renewcommand\example[1]
{
  \textit{Example:}

  \begin{tabular}{p{0.2cm}p{13.8cm}}
  & {\footnotesize{\texttt{#1}}}
  \end{tabular}
}

%
% ---------- common -----------------------------------------------------------
%

%
% ---------- header -----------------------------------------------------------
%
% project       kaneton
%
% license       kaneton
%
% file          /home/mycure/kaneton/view/package/opk/common.tex
%
% created       julien quintard   [fri may 23 18:51:54 2008]
% updated       julien quintard   [wed feb 23 15:56:42 2011]
%

%
% ---------- author -----------------------------------------------------------
%
% this command set the authors.
%

\author
  {%
    kaneton people
  }

%
% ---------- configuration ----------------------------------------------------
%

%
% verbatim
%

\makeatletter
  \def\verbatim@font{\footnotesize\ttfamily\hyphenchar\font\m@ne\@noligs}

  \def\verbatim@processline
    {\hskip15ex{\color{c_verbatim}\the\verbatim@line}\par}
\makeatother

%
% ---------- colours ----------------------------------------------------------
%

\definecolor{c_function}{rgb}{0.40,0.00,0.00}
\definecolor{c_command}{rgb}{0.00,0.00,0.40}
\definecolor{c_verbatim}{rgb}{0.00,0.40,0.00}
\definecolor{c_code}{rgb}{0.00,0.40,0.00}
\definecolor{c_note}{rgb}{0.87,0.84,0.02}
\definecolor{c_location}{rgb}{0.70,0.40,0.20}
\definecolor{c_type}{rgb}{0.30,0.50,0.70}
\definecolor{c_argument}{rgb}{0.20,0.30,0.80}
\definecolor{c_return}{rgb}{0.70,0.70,0.00}

%
% ---------- environments -----------------------------------------------------
%

%
% details
%

\newenvironment{details}
   {
     \ifthenelse
	 {
	   \equal{\mode}{private}
	 }
	 {%
	   \textbf{ \color{red}{[details]}}
	 }
	 {%
	   \comment
	 }
   }
   {
     \ifthenelse
	 {
	   \equal{\mode}{private}
	 }
	 {%
	   \textbf{ \color{red}{[/details]}}
	 }
	 {%
	   \endcomment
	 }
   }

%
% correction
%

\newenvironment{correction}
   {
     \ifthenelse
	 {
	   \equal{\mode}{private}
	 }
	 {%
	   \textbf{ \color{red}{[correction]}}
	 }
	 {%
	   \comment
	 }
   }
   {
     \ifthenelse
	 {
	   \equal{\mode}{private}
	 }
	 {%
	   \textbf{ \color{red}{[/correction]}}
	 }
	 {%
	   \endcomment
	 }
   }

%
% ---------- commands ---------------------------------------------------------
%

%
% term
%
% this command is used for introducing new important words.
%
% #1:   text
%

\newcommand\term[1]
  {%
    \textbf{#1}%
  }

%
% name
%
% this command is used when refering to a already introduced term or
% a somewhat special word.
%
% #1:   text
%

\newcommand\name[1]
  {%
    \textit{#1}%
  }

%
% code
%
% this command is used for words which represent function names or code
% e.g \code{show()} etc.
%
% #1:   text
%

\newcommand\code[1]
  {%
    {\footnotesize\texttt{\color{c_code}#1}}%
  }

%
% reference
%
% this command is used for referencing a figure or section or anything else.
% this command should encapsulate the whole text e.g
% \reference{Figure \ref{XXX}}, \reference{Chapter \ref{XXX}} etc.
%
% #1:   section/figure/etc.
%

\newcommand\reference[1]
  {%
    \textit{#1}%
  }

%
% location
%
% this command is used for describing a location: path, URL etc.
%
% #1:   location
%

\newcommand\location[1]
  {%
    {\footnotesize\texttt{\color{c_location}#1}}%
  }

%
% function
%
% this special command is used for describing a function.
%
% #1:   return type
% #2:   function name
% #3:   arguments list
% #4:   description text
%

\newcommand\function[4]
  {
    \begin{tabular}{p{1.5cm}p{14.5cm}}
      {\footnotesize\texttt{\color{c_return}#1}} &
      \texttt{\color{c_function}#2}(#3)
    \end{tabular}

    \begin{tabular}{p{2.2cm}p{11.8cm}}
      & #4
    \end{tabular}
  }

%
% type
%
% this command should only be used in the function() command and is used
% for describing a type.
%
% #1:   type name
%

\newcommand\type[1]
  {%
    {\footnotesize\texttt{#1}}%
  }

%
% argument
%
% this command is also dependent from the function() command and is used for
% describing arguments.
%
% #1:   argument name
%

\newcommand\argument[1]
  {%
    {\footnotesize\texttt{\color{c_argument}#1}}%
  }

%
% command
%
% this command is equivalent to the function one but targets shell commands,
% assembly labels etc.
%
% #1:   command line
% #2:   description text
%

\newcommand\command[2]
  {
    \begin{tabular}{p{0.2cm}p{13.8cm}}
    & \texttt{\color{c_command}#1}
    \end{tabular}

    \begin{tabular}{p{1cm}p{13cm}}
    & #2
    \end{tabular}
}

%
% question
%
% this command is used in feedback documents for creating questions.
%
% #1:   text
% #2:   space needed for people to answer
%

\newcommand\question[2]
{
  \textbf{\subsubsection*{#1}}

  \vspace{#2}
}

%
% latex
%
% this command latexify the given text by putting a backslah in front of it.
%
% #1:   command name
%

\newcommand\latex[1]
{%
  $\backslash$#1%
}

%
% ie, etc, eg ...
%

\newcommand\ie[0]{\textit{i.e.}}
\newcommand\eg[0]{\textit{e.g.}}
\newcommand\etc[0]{\textit{etc.}}
\newcommand\aka[0]{\textit{a.k.a.}}




%
% title
%

\title{kaneton}

%
% document
%

\begin{document}

%
% title frame
%

\begin{frame}
  \titlepage
\end{frame}

%
% outline frame
%

\begin{frame}
  \frametitle{Outline}

  \tableofcontents
\end{frame}

%
% ---------- text -------------------------------------------------------------
%


%
% overview
%

\section{Overview}

% 1)

\begin{frame}
  \frametitle{Introduction}

  \term{kaneton} is an educational project intended for students to undertake
  in order to learn about operating system internals.
\end{frame}

% 2)

\begin{frame}
  \frametitle{History}

  \begin{itemize}
    \item[2004]
      \name{Julien Quintard} and \name{Jean-Pascal Billaud} decide
      to introduce an optional low-level programming course to first-year
      enginnering students, now known as \name{kastor};
    \item[2004]
      The course having been well received, \name{SRS - Syst\`emes, R\'eseaux,
      S\'ecurit\'e} students ask them to give such an introductory course the
      same year.
    \item[2005]
      The authorization is given to them to teach a kernel development course
      to \name{SRS} students, from January to October. They therefore decide
      to provide students with the design of a microkernel and let the students
      develop it from scratch, their way. \name{kaneton} is born.
    \item[2006]
      After \name{Jean-Pascal Billaud} fled to \name{VMWare}, \name{Julien
      Quintard} started developing a reference implementation and gave
      students this year a skeleton they had to complete. In addition,
      \name{Cedric Aubouy} and \name{Renaud Lienhart} joined the teaching
      team this year.

      \-

      Besides, the \name{LSE - Laboratoire Syst\`eme EPITA} joined the project
      by putting two students on the development of the \name{kaneton}
      research implementation. \name{Matthieu Bucchianeri} and \name{Renaud
      Voltz} thus joined the project.
  \end{itemize}
\end{frame}


% 3)

\begin{frame}
  \frametitle{History}

  \begin{itemize}
    \item[2007]
      This year, \name{Matthieu Bucchianeri} and \name{Renaud Voltz} took
      over the project for a year by lecturing the course and managing the
      project.

      \-

      \name{Julian Pidancet} and \name{Pierre Duteil} joined the project
      as part of the \name{LSE} but \name{Pierre Duteil} had to leave the
      project. Therefore, \name{Elie Bleton}, who was working at the
      \name{LRDE} before, joined the project.
    \item[2008]
      \name{Julian Pidancet} and \name{Elie Bleton} took over this year
      while \name{Laurent Lec} and \name{Nicolas Grandemange} joined as part
      of the \name{LSE}.

      \-

      At the end of this year, after problems with some students as well as
      conflicts with the \name{LSE}, \name{kaneton} maintainers decided not
      to work with the laboratory anymore.
    \item[2009]
      \name{EPITA} alumni were contacted and joined the educational project
      including \name{Francois Goudal}, \name{Benoit Marcot},
      \name{Enguerrand Raymond}, \name{Jean Guyader} but also
      \name{Fabien Le-Mentec}, an \name{EPITECH} alumnus.
  \end{itemize}
\end{frame}

% 4)

\begin{frame}
  \frametitle{Model}

  The project consists for students to fill in some missing parts of the
  kernel.

  \-

  However, note that, unlike \name{Tiger}, the missing parts will never
  be two lines long.

  \-

  Indeed, in \name{kaneton}, students are asked to implement a feature, say,
  providing memory management. Thus, students are free, in a certain way,
  to implement such a feature as they wish.

  \-

  Since the testing usually consists in verifying that the kernel is able
  to provide the functionality, students should be, most of the time, able
  to implement whatever algorithms \etc{} they wish.
\end{frame}

% 5)

\begin{frame}
  \frametitle{People}

  Let's present the people working on the educational project from where they
  studied to what they are now doing:

  \begin{itemize}
    \item
      \name{Francois Goudal};
    \item
      \name{Benoit Marcot};
    \item
      \name{Enguerrand Raymond};
    \item
      \name{Jean Guyader};
    \item
      \name{Fabien Le-Mentec}; and
    \item
      \name{Julien Quintard}.
  \end{itemize}
\end{frame}

% 6)

\begin{frame}
  \frametitle{Project}

  \name{kaneton} is an important assignment of the \name{SRS}/\name{GISTR}
  curriculum and, as such, must be taken seriously.

  \-

  Especially, in the last years, \name{EPITA} decided to reduce the duration
  of the project to \term{three} months.

  \-

  As such, the other assignments imposed by the specializations in this
  period have been reduced so that students can focus on \name{kaneton}.
\end{frame}

%
% design
%

\section{Design}

% 1)

\begin{frame}
  \frametitle{Overview}

  The kaneton kernel is very different from the kernels you might be
  familiar with, especially the well-known \name{Windows}, \name{Linux},
  \name{BSD} and so forth.
\end{frame}

% 2)

\begin{frame}
  \frametitle{Microkernel}

  First, kaneton is a microkernel, making it modular from the design
  perspective as well as providing properties such as security.
\end{frame}

% 3)

\begin{frame}
  \frametitle{Distributed Computing}

  kaneton has been designed from the ground up for providing the operarting
  system advanced distributed computing features.
\end{frame}

% 4)

\begin{frame}
  \frametitle{Portability}

  kaneton has been designed with portability in mind, especially through
  a specific portability system that perfectly fits the kernel design.
\end{frame}

% 5)

\begin{frame}
  \frametitle{Organisation}

  Besides being a microkernel, kaneton is well organised in the inside,
  splitting functionalites into objects and managers.
\end{frame}

%
% stages
%

\section{Stages}

% 1)

\begin{frame}
  \frametitle{k0}

  The first project, named \term{k0}, consists for students to learn
  about low-level programming.

  \-

  This project comes with a lecture regarding the boot system as well
  as a practical session.

  \-

  \name{Francois Goudal} will be in charge of this stage which will last for
  a week.
\end{frame}

% 2)

\begin{frame}
  \frametitle{k1}

  \term{k1} consists for students to provide the kaneton microkernel a
  memory management unit so that applications as well as the kernel itself
  can reserve, share \etc{} memory.

  \-

  During this stage, a lecture on general kernel principles, a lecture on
  portability as well as lectures on memory management will be taught.

  \-

  \name{Francois Goudal} will be in charge of this stage which will last
  for three weeks.

  \-

  Note that, starting with \name{k1}, the student snapshot will be used which
  provide students a development environment, making kernel development easier.
\end{frame}

% 3)

\begin{frame}
  \frametitle{k2}

  In \term{k2}, students will have to provide kaneton with execution contexts
  such that the kernel can execute multiple threads at the \textit{same} time.

  \-

  Lectures, during this stage, will discuss topics such as interrupts,
  concurrency, multi-processing, scheduling \etc{}

  \-

  \name{Benoit Marcot} will be in charge of this stage which will last for
  three weeks.
\end{frame}

% 4)

\begin{frame}
  \frametitle{k3}

  In \term{k3}, students will have to implement servers on top of their freshly
  developed microkernel.

  \-

  These developments will include device drivers as well as services.

  \-

  Lectures discussing virtualization, devices, communication, file systems,
  security as well as advanced research kernels will be taught during this
  stage.

  \-

  \name{Julien Quintard \& Jean Guyader} will be in charge of this stage
  which will last for three weeks.
\end{frame}

% 5)

\begin{frame}
  \frametitle{Evaluation}

  For every stage, students will have the possibility to test their
  implementation by running, a limited number of times, the test suite used
  for evaluating their work.

  \-

  Besides, at the end of each stage, after submission, the kaneton test system
  will run the test suite and issue a mark according to the test results.

  \-

  Note however that, every test suite run decreases your maximum mark by
  one point. Therefore, a student or group of students, relying on the test
  suite six times will, at best, achieve a 14/14 mark.

  \-

  Additionally, an exam will take place at the end of the semester to make
  sure that the notions tackled throughout the course are well understood
  by every student.

  \-

  Students can test and submit their work through the \name{EPITA}-specific
  intranet: \location{http://epita.opaak.org}.
\end{frame}

%
% tools
%

\section{Tools}

% 1)

\begin{frame}
  \frametitle{Overview}

  The kaneton educational project relies on tools, sometimes developed
  internally.
\end{frame}

% 2)

\begin{frame}
  \frametitle{Web Site}

  The web site contains the documentation including design papers,
  the assignments \etc{} but also hosts the wiki which should be
  the starting point for every student seeking information.

  \-

  \name{Julien Quintard} should be contacted for requests regarding the
  web site and wiki.
\end{frame}

% 3)

\begin{frame}
  \frametitle{Snapshot}

  The student snapshot has been automatically generated from the current
  kaneton implementation.

  \-

  \name{Francois Goudal} is in charge of this process, hence should be
  contacted if you believe there is a mistake.
\end{frame}

% 4)

\begin{frame}
  \frametitle{Cheat}

  Every student's kaneton implementation will be tested to make sure that
  students did not cheat by relying on implementations by previous or
  current students.

  \-

  \name{Julien Quintard} is in charge of this tool.
\end{frame}

% 5)

\begin{frame}
  \frametitle{Test}

  Students' implementation will be tested in a real environment by applying
  a complete test suite; hence, validating the implementation's behaviour.

  \-

  \name{Jean Guyader} is responsible of this tool and should be contacted
  if necessary.
\end{frame}

%
% information
%

\section{Information}

% 1)

\begin{frame}
  \frametitle{Support}

  As a student, there are three ways of getting support when undertaking
  the kaneton educational project:

  \-

  \begin{enumerate}
    \item
      \term{Website}

      \-

      You will find on \location{http://kaneton.opaak.org} documents regarding
      the project from the design to the implementation;
    \item
      \term{Wiki}

      \-

      The wiki \location{http://wiki.opaak.org} is the best way to get
      technical information as well as to help other students by adding
      and/or improving pages' contents;
    \item
      \term{Mailing-List}

      \-

      The kaneton educational students mailing-list
      \location{students@kaneton.opaak.org} will be used by teachers as
      an official means for communicating with students.

      \-

      Therefore, every student should subscribe to this mailing-list by sending
      an email to \location{students+subscribe@kaneton.opaak.org}.

      \-

      It is not allowed to post code on the mailing list, or give pointers to
      code in the snapshot that would provide obvious solution to somebody's
      question.

  \end{enumerate}
\end{frame}

% 2)

\begin{frame}
  \frametitle{Groups}

  Except for \name{k0} which is an individual project, the other projects
  from \name{k1} to \name{k3} are done in groups of \term{two} students.

  \-

  Every group is expected to send an email to
  \name{Julien Quintard}\footnote{\location{julien.quintard@cl.cam.ac.uk}}
  before the $18$th of February with the group composition.

  \-

  Note that we will use students' \name{EPITA} email addresses. As such,
  make sure that you check this email box.
\end{frame}

% 3)

\begin{frame}
  \frametitle{Reliance}

  As for \name{Tiger}, every stage depends on the previous one, except
  for \name{k0}.

  \-

  As such, test suites from the previous stages will also be used for both
  testing and marking.

  \-

  Students should therefore make sure to use their test permissions for making
  sure to fix the bugs of previous stages so that such bugs do not impact
  on the current stage results, hence mark.
\end{frame}

% 4)

\begin{frame}
  \frametitle{Machine}

  This year, the machine used by the kaneton educational project will consists
  of the \term{IBM-PC} platform coupled with the \term{IA-32} microprocessor
  architecture \ie{} the most common hardware system on the market.

  \-

  Although it is always best to test your implementation on a real machine,
  it takes time to reboot a real computer. You should therefore use an
  emulator such a \name{QEmu} or \name{Bochs} as they will enable you to
  test your kernel very quickly but they will also let you develop on
  a non-\name{IBM-PC}/\name{IA-32} machine such as a \name{Mac} for example.

  \-

  Our own automatic test suite doesn't test your tarballs on a real physical
  machine, because we needed to be able to run tests automatically and in
  parallel, so you can get your traces fast enough.
  We use hardware virtualization for that, based on Xen HVM, which is the
  closest way of testing without using a real machine (better than Qemu).
  If you have the possibility to run Xen HVMs on your computer, or in your lab
  we strongly suggest that you try your kernel on it too, before uploading.
  Note that you will always need Qemu to develop, since Xen HVM doesn't provide
  a good way to debug. So you should only try it on a Xen HVM when it works on
  Qemu, just to make sure.

\end{frame}

% 5)

\begin{frame}
  \frametitle{Requirements}

  As a reminder, students must have a rack containing a \name{POSIX} compliant
  operating system.

  \-

  This rack should be installed and ready so that you use it for the
  \name{k0} practical session.
\end{frame}

%
% conclusion
%

\section{Conclusion}

% 1)

\begin{frame}
  \frametitle{Concepts}

  Throughout the project, you will learn so many things from terminology,
  to how a computer boots, how the kernel controls the hardware and how it
  provides abstractions as basic as execution contexts.

  \-

  At the end of the project, you will definitely know that nothing is magic
  but purely logic and often actually very simple.
\end{frame}

% 2)

\begin{frame}
  \frametitle{Implementation}

  Although, starting the project by learning how to make a computer execute
  your code, you will end up, after three months, with a running kernel
  and operating system capable of executing programs, the whole on real
  hardware like the machine you have at home.
\end{frame}

% 3)

\begin{frame}
  \frametitle{Changes}

  Over the years, the project has greatly evolved, from a no-implementation
  project, to a reference-based project.

  \-

  However, being a project developed by volunteers willing to dedicate some
  time so that other students can learn, many things are missing and/or
  can be improved including the lectures but also the project implementation.

  \-

  In conclusion, keep in mind that the project exists only because of people
  willing to transfer their knowledge and please respect their effort.
\end{frame}

% 4)

\begin{frame}
  \frametitle{Fun}

  But most of all, kaneton should be about learning through fun!
\end{frame}

% 5)

\begin{frame}
  \frametitle{Reminder}

  Remember to perform the following tasks:

  \begin{itemize}
    \item
      Send an email to
      \name{Julien Quintard}\footnote{\location{julien.quintard@cl.cam.ac.uk}}
      regarding the composition of your group;
    \item
      Subscribe to the students mailing-list
      \location{students@kaneton.opaak.org} by sending an email to
      \location{students+subscribe@kaneton.opaak.org};
    \item
      Watch closely the \name{Wiki} at \location{http://wiki.opaak.org} by
      subscribing the \name{RSS} feed for example.
    \item
      Feel free to edit and add content to the \name{Wiki}, this is
      \textbf{your} system-related zone.
  \end{itemize}
\end{frame}

\end{document}
