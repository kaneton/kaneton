%
% ---------- header -----------------------------------------------------------
%
% project       kaneton
%
% license       kaneton
%
% file          /home/mycure/kane...w/talk/presentations/kaneton/kaneton.tex
%
% created       julien quintard   [mon may 14 21:02:29 2007]
% updated       julien quintard   [thu jan 29 18:14:34 2009]
%

%
% ---------- setup ------------------------------------------------------------
%

%
% path
%

\def\path{../../..}

%
% template
%

%
% ---------- header -----------------------------------------------------------
%
% project       kaneton
%
% license       kaneton
%
% file          /home/mycure/kaneton/view/template/talk.tex
%
% created       julien quintard   [wed may 16 18:17:37 2007]
% updated       julien quintard   [fri may 23 19:28:10 2008]
%

%
% ---------- class ------------------------------------------------------------
%

\documentclass[8pt]{beamer}

%
% ---------- common -----------------------------------------------------------
%

%
% ---------- header -----------------------------------------------------------
%
% project       kaneton
%
% license       kaneton
%
% file          /home/mycure/kaneton/view/package/opk/presentation.tex
%
% created       julien quintard   [wed may 16 18:17:26 2007]
% updated       julien quintard   [wed apr 22 20:50:07 2009]
%

%
% ---------- packages ---------------------------------------------------------
%

%
% packages
%

\usepackage{pgf,pgfarrows,pgfnodes,pgfautomata,pgfheaps,pgfshade}
\usepackage[T1]{fontenc}
\usepackage{colortbl}
\usepackage{times}
\usepackage{amsmath,amssymb}
\usepackage{multicol}
\usepackage{graphics}
\usepackage{graphicx}
\usepackage{color}
\usepackage{xcolor}
\usepackage[english]{babel}
\usepackage{enumerate}
\usepackage[latin1]{inputenc}
\usepackage{verbatim}
\usepackage{aeguill}

%
% ---------- date -------------------------------------------------------------
%

%
% date
%

\date{\scriptsize{\today}}

%
% ---------- configuration ----------------------------------------------------
%

%
% style
%

\usepackage{beamerthemesplit}
\setbeamercovered{dynamic}

%
% table of contents at the beginning of each section
%

\AtBeginSection[]
{
  \begin{frame}<beamer>
   \frametitle{Outline}
    \tableofcontents[current]
  \end{frame}
}

%
% table of contents at the beginning of each subsection
%

\AtBeginSubsection[]
{
  \begin{frame}<beamer>
   \frametitle{Outline}
    \tableofcontents[current,currentsubsection]
  \end{frame}
}

%
% ---------- commands ---------------------------------------------------------
%

%
% -
%

\renewcommand{\-}{\vspace{0.4cm}}

%
% note
%
% this command is used for making little notes, tips, hints etc.
%
% #1:   text
%

\renewcommand\note[1]
{
  \textbf{\color{c_note}Note}

  \begin{tabular}{p{0.2cm}p{13.8cm}}
  & #1
  \end{tabular}
}

%
% example
%
% this command is used for creating examples.
%
% #1:   text
%

\renewcommand\example[1]
{
  \textit{Example:}

  \begin{tabular}{p{0.2cm}p{13.8cm}}
  & {\footnotesize{\texttt{#1}}}
  \end{tabular}
}

%
% ---------- common -----------------------------------------------------------
%

%
% ---------- header -----------------------------------------------------------
%
% project       kaneton
%
% license       kaneton
%
% file          /home/mycure/kaneton/view/package/opk/common.tex
%
% created       julien quintard   [fri may 23 18:51:54 2008]
% updated       julien quintard   [wed feb 23 15:56:42 2011]
%

%
% ---------- author -----------------------------------------------------------
%
% this command set the authors.
%

\author
  {%
    kaneton people
  }

%
% ---------- configuration ----------------------------------------------------
%

%
% verbatim
%

\makeatletter
  \def\verbatim@font{\footnotesize\ttfamily\hyphenchar\font\m@ne\@noligs}

  \def\verbatim@processline
    {\hskip15ex{\color{c_verbatim}\the\verbatim@line}\par}
\makeatother

%
% ---------- colours ----------------------------------------------------------
%

\definecolor{c_function}{rgb}{0.40,0.00,0.00}
\definecolor{c_command}{rgb}{0.00,0.00,0.40}
\definecolor{c_verbatim}{rgb}{0.00,0.40,0.00}
\definecolor{c_code}{rgb}{0.00,0.40,0.00}
\definecolor{c_note}{rgb}{0.87,0.84,0.02}
\definecolor{c_location}{rgb}{0.70,0.40,0.20}
\definecolor{c_type}{rgb}{0.30,0.50,0.70}
\definecolor{c_argument}{rgb}{0.20,0.30,0.80}
\definecolor{c_return}{rgb}{0.70,0.70,0.00}

%
% ---------- environments -----------------------------------------------------
%

%
% details
%

\newenvironment{details}
   {
     \ifthenelse
	 {
	   \equal{\mode}{private}
	 }
	 {%
	   \textbf{ \color{red}{[details]}}
	 }
	 {%
	   \comment
	 }
   }
   {
     \ifthenelse
	 {
	   \equal{\mode}{private}
	 }
	 {%
	   \textbf{ \color{red}{[/details]}}
	 }
	 {%
	   \endcomment
	 }
   }

%
% correction
%

\newenvironment{correction}
   {
     \ifthenelse
	 {
	   \equal{\mode}{private}
	 }
	 {%
	   \textbf{ \color{red}{[correction]}}
	 }
	 {%
	   \comment
	 }
   }
   {
     \ifthenelse
	 {
	   \equal{\mode}{private}
	 }
	 {%
	   \textbf{ \color{red}{[/correction]}}
	 }
	 {%
	   \endcomment
	 }
   }

%
% ---------- commands ---------------------------------------------------------
%

%
% term
%
% this command is used for introducing new important words.
%
% #1:   text
%

\newcommand\term[1]
  {%
    \textbf{#1}%
  }

%
% name
%
% this command is used when refering to a already introduced term or
% a somewhat special word.
%
% #1:   text
%

\newcommand\name[1]
  {%
    \textit{#1}%
  }

%
% code
%
% this command is used for words which represent function names or code
% e.g \code{show()} etc.
%
% #1:   text
%

\newcommand\code[1]
  {%
    {\footnotesize\texttt{\color{c_code}#1}}%
  }

%
% reference
%
% this command is used for referencing a figure or section or anything else.
% this command should encapsulate the whole text e.g
% \reference{Figure \ref{XXX}}, \reference{Chapter \ref{XXX}} etc.
%
% #1:   section/figure/etc.
%

\newcommand\reference[1]
  {%
    \textit{#1}%
  }

%
% location
%
% this command is used for describing a location: path, URL etc.
%
% #1:   location
%

\newcommand\location[1]
  {%
    {\footnotesize\texttt{\color{c_location}#1}}%
  }

%
% function
%
% this special command is used for describing a function.
%
% #1:   return type
% #2:   function name
% #3:   arguments list
% #4:   description text
%

\newcommand\function[4]
  {
    \begin{tabular}{p{1.5cm}p{14.5cm}}
      {\footnotesize\texttt{\color{c_return}#1}} &
      \texttt{\color{c_function}#2}(#3)
    \end{tabular}

    \begin{tabular}{p{2.2cm}p{11.8cm}}
      & #4
    \end{tabular}
  }

%
% type
%
% this command should only be used in the function() command and is used
% for describing a type.
%
% #1:   type name
%

\newcommand\type[1]
  {%
    {\footnotesize\texttt{#1}}%
  }

%
% argument
%
% this command is also dependent from the function() command and is used for
% describing arguments.
%
% #1:   argument name
%

\newcommand\argument[1]
  {%
    {\footnotesize\texttt{\color{c_argument}#1}}%
  }

%
% command
%
% this command is equivalent to the function one but targets shell commands,
% assembly labels etc.
%
% #1:   command line
% #2:   description text
%

\newcommand\command[2]
  {
    \begin{tabular}{p{0.2cm}p{13.8cm}}
    & \texttt{\color{c_command}#1}
    \end{tabular}

    \begin{tabular}{p{1cm}p{13cm}}
    & #2
    \end{tabular}
}

%
% question
%
% this command is used in feedback documents for creating questions.
%
% #1:   text
% #2:   space needed for people to answer
%

\newcommand\question[2]
{
  \textbf{\subsubsection*{#1}}

  \vspace{#2}
}

%
% latex
%
% this command latexify the given text by putting a backslah in front of it.
%
% #1:   command name
%

\newcommand\latex[1]
{%
  $\backslash$#1%
}

%
% ie, etc, eg ...
%

\newcommand\ie[0]{\textit{i.e.}}
\newcommand\eg[0]{\textit{e.g.}}
\newcommand\etc[0]{\textit{etc.}}
\newcommand\aka[0]{\textit{a.k.a.}}




%
% title
%

\title{kaneton}

%
% document
%

\begin{document}

%
% title frame
%

\begin{frame}
  \titlepage
\end{frame}

%
% outline frame
%

\begin{frame}
  \frametitle{Outline}

  \tableofcontents
\end{frame}

%
% ---------- text -------------------------------------------------------------
%


%
% overview
%

\section{Overview}

% 1)

\begin{frame}
  \frametitle{Introduction}

  \term{kaneton} is an educational project intended for students to undertake
  in order to learn about operating system internals.

  \-

  Note that a research project is also tied to the educational one. Indeed,
  people work in order to improve the microkernel and build a useable
  distributed operating system on top of it.
\end{frame}

% 2)

\begin{frame}
  \frametitle{History}

  \begin{itemize}
    \item[2004]
      \name{Julien Quintard} and \name{Jean-Pascal Billaud} [EPITA'06] decide
      to introduce an optional low-level programming course to first-year
      enginnering students, now known as \name{kastor};
    \item[2004]
      The course having been well received, \name{SRS - Syst\`emes, R\'eseaux,
      S\'ecurit\'e} students ask them to give such an introduction course the
      same year.
    \item[2005]
      The authorization is given to them to teach an kernel development course
      to \name{SRS} students, from January to October. They therefore decided
      to provide students with the design of a microkernel and let the students
      develop it from scratch, their way. \name{kaneton} was born.
    \item[2006]
      After \name{Jean-Pascal Billaud} fled to \name{VMWare}, \name{Julien
      Quintard} started developing a reference implementation and gave
      students this year a skeleton they had to complete. In addition,
      \name{Cedric Aubouy} and \name{Renaud Lienhart} joined the teaching
      team this year.

      \-

      Besides, the \name{LSE - Laboratoire Syst\`eme EPITA} joined the project
      by putting two students on the development of the \name{kaneton}
      research implementation. \name{Matthieu Bucchianeri} and \name{Renaud
      Voltz} thus joined the project.
  \end{itemize}
\end{frame}

% 3)

\begin{frame}
  \frametitle{History}

  \begin{itemize}
    \item[2007]
      This year, \name{Matthieu Bucchianeri} and \name{Renaud Voltz} took
      over the project for a year by lecturing the course and managing the
      project.

      \-

      \name{Julian Pidancet} and \name{Pierre Duteil} joined the project
      as part of the \name{LSE} but \name{Pierre Duteil} had to leave the
      project. Therefore, \name{Elie Bleton}, who was working at the
      \name{LRDE} before, joined the project.
    \item[2008]
      \name{Julian Pidancet} and \name{Elie Bleton} took over this year
      while \name{Laurent Lec} and \name{Nicolas Grandemange} joined as part
      of the \name{LSE}.

      \-

      At the end of this year, after problems with students and the \name{LSE},
      \name{kaneton} maintainers decided not to work with the \name{LSE}
      anymore.
    \item[2009]
      \name{EPITA} alumni were contacted and joined the educational project
      including \name{Francois Goudal}, \name{Benoit Marcot},
      \name{Jean Guyader} and \name{Frank Curo} but also
      \name{Fabien Le-Mentec}, an \name{EPITECH} alumnus.
  \end{itemize}
\end{frame}

% 4)

\begin{frame}
  \frametitle{Changes}

  Over the years, the project has greatly evolved, from a no-implementation
  project, to a reference-based project but also through the assignments.

  \-

  However, being a project developed by volunteers willing to dedicate some
  time so that other students can learn, many things are missing and/or
  can be improved including the lectures but also the project implementation.

  \-

  In conclusion, keep in mind that the project exist only because of people
  willing to pass their knowledge to students.
\end{frame}

% 5)

\begin{frame}
  \frametitle{Model}

  The project consists for students to fill in some missing parts of the
  kernel.

  \-

  However, note that, unlike \name{Tiger}, the missing parts will never
  be two lines long.

  \-

  Indeed, in \name{kaneton}, students are asked to implement a feature, say,
  providing memory management. Thus, students are free, in a certain way,
  to implement such a feature the way they want.

  \-

  Since the testing usually consists in verifying that the kernel is able
  to provide the functionality, students should be, most of the time, able
  to implement whatever algorithms \etc{} we wish.
\end{frame}

% 6)

\begin{frame}
  \frametitle{People}

  Let's present the people working on the educational project from where they
  studied to what they are now doing:

  \begin{itemize}
    \item
      \name{Francois Goudal};
    \item
      \name{Benoit Marcot};
    \item
      \name{Jean Guyader};
    \item
      \name{Fabien Le-Mentec};
    \item
      \name{Frank Curo};
    \item
      \name{Anthony Guduszeit}; and
    \item
      \name{Julien Quintard}.
  \end{itemize}
\end{frame}

% 7)

\begin{frame}
  \frametitle{Project}

  \name{kaneton} is an important assignments in the \name{SRS}/\name{GISTR}
  curriculum and, as such, must be taken seriously.

  \-

  Especially, in the last years, \name{EPITA} decided to reduce the duration
  of the project to $3$ months.

  \-

  As such, the other assignments imposed by the specializations in this
  period have been reduced so that students can focus on \name{kaneton}.
\end{frame}

% 8)

\begin{frame}
  \frametitle{opaak}

  The kaneton project is part of the opaak trilogy which consists
  of \term{kastor}, \term{kaneton} and \term{kayou}.

  \-

  While kayou is a distributed operating system built upon the kaneton
  microkernel; kastor is a standalone project that you may have heard of
  or even undertook.

  \-

  Note however that also some concepts studied through kastor will be
  discussed in kaneton, the kaneton educational project is far more
  complicated as it goes deeper in hardware and software problematics.
\end{frame}

%
% design
%

\section{Design}

% 1)

\begin{frame}
  \frametitle{Overview}

  The kaneton kernel is very different from the kernels you might be
  familiar with, especially the well-known \name{Windows}, \name{Linux},
  \name{BSD} and so forth.
\end{frame}

% 2)

\begin{frame}
  \frametitle{Microkernel}

  First, kaneton is a microkernel, making it modular from the design
  perspective as well as providing properties such as security.
\end{frame}

% 3)

\begin{frame}
  \frametitle{Distributed Computing}

  kaneton has been designed from the ground up for providing the operarting
  system advanced distributed computing features.
\end{frame}

% 4)

\begin{frame}
  \frametitle{Portability}

  kaneton has been designed with portability in mind, especially through
  a specific portability system that perfectly fits the kernel design.
\end{frame}

% 5)

\begin{frame}
  \frametitle{Organisation}

  Besides being a microkernel, kaneton is well organised in the inside,
  splitting functionalites into objects and managers.
\end{frame}

%
% stages
%

\section{Stages}

% 1)

\begin{frame}
  \frametitle{k0}

  The first project, named \term{k0}, consists for students to learn
  about low-level programming.

  \-

  This project comes with a lecture regarding the boot system as well
  as practical sessions.

  \-

  \name{Francois Goudal} will be in charge of this stage which will last for
  a week.
\end{frame}

% 2)

\begin{frame}
  \frametitle{k1}

  \term{k1} consists for students to provide the kaneton microkernel a
  memory management unit so that applications as well as the kernel itself
  can reserve, share \etc{} memory.

  \-

  During this stage, a lecture on general kernel principles, a lecture on
  portability as well as lectures on memory management will be taught.

  \-

  \name{Francois Goudal} will be in charge of this stage which will last
  for three weeks.

  \-

  Note that, starting with \name{k1}, the student snapshot will be used.
\end{frame}

% 3)

\begin{frame}
  \frametitle{k2}

  In \term{k2}, students will have to provide kaneton with execution contexts
  such that the kernel can execute multiple threads at the \textit{same} time.

  \-

  Lectures, during this stage, will discuss topics such as interrupts,
  concurrency, multi-processing, scheduling \etc{}

  \-

  \name{Benoit Marcot} will be in charge of this stage which will last for
  three weeks.
\end{frame}

% 4)

\begin{frame}
  \frametitle{k3}

  In \term{k3}, students will have to implement a device drivers, being
  a disk driver, network driver \etc{} XXX

  \-

  Lectures discussing virtualization, devices, communication \etc{} will
  be taught during this stage.

  \-

  \name{Jean Guyader} will be in charge of this stage which will last for
  two weeks.
\end{frame}

% 5)

\begin{frame}
  \frametitle{k4}

  For \term{k4}, students will have to design and develop a file system
  on top of a disk driver.

  \-

  During this stage, teachers will teach students about security, file systems
  but also the \name{Windows NT} kernel.

  \-

  \name{Julien Quintard} will be in charge of this stage which will last for
  three weeks.
\end{frame}

% 6)

\begin{frame}
  \frametitle{Evaluation}

  For every stage, students will have the possibility to test their
  implementation by running, a limited number of times, the test suite used
  for evaluating their work.

  \-

  Besides, at the end of each stage, after submission, the kaneton test system
  will run the test suite and issue a mark according to the test results.

  \-

  Additionally, an exam will take place at the end of the semester to make
  sure that notions tackled in the course are understood by every student.
\end{frame}

%
% tools
%

\section{Tools}

% 1)

\begin{frame}
  \frametitle{Overview}

  The kaneton educational project relies on tools, sometimes developed
  internally, enabling students to make the best of such tools within the
  project time frame.
\end{frame}

% 2)

\begin{frame}
  \frametitle{Web Site}

  The web site contains the documentation including design papers,
  the assignments \etc{} but also hosts the wiki which should be
  the starting point for every students seeking information.

  \-

  \name{Julien Quintard} should be contacted for requests regarding the
  web sites.
\end{frame}

% 3)

\begin{frame}
  \frametitle{Snapshot}

  The student snapshot has been automatically generated from the current
  kaneton implementation.

  \-

  \name{Francois Goudal} is in charge of this process, hence should be
  contacted if you believe there is a mistake.
\end{frame}

% 4)

\begin{frame}
  \frametitle{Cheat}

  Every student's kaneton implementation will be tested to make sure that
  students did not cheat by relying on implementations by previous or
  current students.

  \-

  \name{Julien Quintard} is in charge of this tool.
\end{frame}

% 5)

\begin{frame}
  \frametitle{Test}

  Students' implementation will be tested in a real environment by applying
  a complete test suite; hence, validating the implementation's behaviour.

  \-

  \name{Jean Guyader} is responsible of this tool and should be contacted
  if necessary.
\end{frame}

%
% information
%

\section{Information}

XXX site kaneton.org + mailing-list + wiki
XXX groups de deux sauf pour k0. chaque tranche notee separement mais stage N
  relies on N - 1 so don't get distanced and fix your bugs.
XXX plateforme/architecture utilisee: ibm-pc ia32
XXX tester sur machine reelle pas emulateur comme qemu ou bochs
XXX besoin d'un rack linux opur k0 et d'un systeme unix pour le projet

%
% conclusion
%

\section{Conclusion}

have fun
apreo: discuter avec les profs!

\end{document}
