%
% ---------- header -----------------------------------------------------------
%
% project       kaneton
%
% license       kaneton
%
% file          /home/mycure/kaneton/view/talk/presentations/k4/k4.tex
%
% created       julien quintard   [sat nov 29 20:40:35 2008]
% updated       julien quintard   [mon feb  2 14:41:36 2009]
%

%
% ---------- setup ------------------------------------------------------------
%

%
% path
%

\def\path{../../..}

%
% template
%

%%
%% copyright     (c) julien quintard
%%
%% project       kaneton
%%
%% file          /home/mycure/kaneton/view/templates/lecture.tex
%%
%% created       julien quintard   [sat nov 19 17:13:03 2005]
%% updated       julien quintard   [fri dec  2 22:36:34 2005]
%%

%
% class
%

\documentclass[8pt]{beamer}

%
% packages
%

\usepackage{pgf,pgfarrows,pgfnodes,pgfautomata,pgfheaps,pgfshade}
\usepackage{colortbl}
\usepackage{times}
\usepackage{amsmath,amssymb}
\usepackage{graphics}
\usepackage{graphicx}
\usepackage{color}
\usepackage{xcolor}
\usepackage[english]{babel}
\usepackage{enumerate}
\usepackage[latin1]{inputenc}

%
% style
%

\usepackage{beamerthemesplit}
\setbeamercovered{dynamic}

%
% verbatim font
%

\definecolor{verbatimcolor}{rgb}{0,0.4,0}

\makeatletter
\renewcommand{\verbatim@font}
  {\ttfamily\footnotesize\color{verbatimcolor}\selectfont}
\makeatother

%
% new line
%

\newcommand{\nl}[0]{\vspace{0.4cm}}

%
% date
%

\date{\today}

%
% logos
%

\pgfdeclareimage[interpolate=true,width=34pt,height=18pt]
                {epita}{../../logos/epita}
\pgfdeclareimage[interpolate=true,width=49pt,height=18pt]
                {upmc}{../../logos/upmc}
\pgfdeclareimage[interpolate=true,width=25pt,height=18pt]
                {lse}{../../logos/lse}

\newcommand{\logos}
  {
    \pgfuseimage{epita}
  }

%
% institute
%

\institute
{
  \inst{1} kaneton microkernel project
}

%
% table of contents at the beginning of each section
%

\AtBeginSection[]
{
  \begin{frame}<beamer>
   \frametitle{Outline}
    \tableofcontents[current]
  \end{frame}
}

%
% table of contents at the beginning of each subsection
%

\AtBeginSubsection[]
{
  \begin{frame}<beamer>
   \frametitle{Outline}
    \tableofcontents[current,currentsubsection]
  \end{frame}
}


%
% title
%

\title{k4}

%
% document
%

\begin{document}

%
% title frame
%

\begin{frame}
  \titlepage
\end{frame}

%
% outline frame
%

\begin{frame}
  \frametitle{Outline}

  \tableofcontents
\end{frame}

%
% ---------- text -------------------------------------------------------------
%

%
% introduction
%

\section{Introduction}

% 1)

\begin{frame}
  \frametitle{Overview}

  \name{k4} is a stage that can be considered as \term{open}.

  \-

  Indeed, total freedom is given to students regarding the design and
  implementation of this project.
\end{frame}

% 2)

\begin{frame}
  \frametitle{Objective}

  The goal of this project is for students to design and implement either a
  file system or a network stack depending on their choice for \name{k3}.
\end{frame}

%
% assignments
%

\section{Assignments}

% 1)

\begin{frame}
  \frametitle{Design}

  The first task consists for students to design the system by studying
  the different algorithms and techniques.
\end{frame}

% 2)

\begin{frame}
  \frametitle{Interface}

  The students file system or network stack is not meant to comply to any
  pre-defined interface.

  \-

  As such, students are invited to study the limitations of well-known
  interfaces such as \name{POSIX}'s in order to define a new one. Additional
  points will be granted according to the solution they propose.
\end{frame}

% 3)

\begin{frame}
  \frametitle{Implementation}

  Once the system designed, students will have to implement it by
  taking care of providing a few illustrating applications.

  \-

  Indeed, no test application is provided as total freedom is given to
  students regarding the interface.
\end{frame}

%
% evaluation
%

\section{Evaluation}

% 1)

\begin{frame}
  \frametitle{Viva}

  During the viva, students will have to first describe their choices in
  terms of design before detailling what problems they encountered afterwards
  regarding the previously sketched design.

  \-

  The file system/network stack interface and user-level applications should
  also be discussed, though quite briefly.

  \-

  The viva will last for $10$ minutes with possibly $5$ minutes for questions.
\end{frame}

%
% conclusion
%

\section{Conclusion}

% 1)

\begin{frame}
  \frametitle{Freedom}

  The main objective of this stage is to give students total control over
  a well-defined functionality.

  \-

  Students should therefore take advantage of this freedom to study and
  think about what well-known systems provide and whether such systems are
  actually good or bad.
\end{frame}

%
% bibliography
%

\begin{frame}
  \frametitle{Bibliography}

  \bibliographystyle{amsplain}
  \bibliography{\path/bibliography/bibliography}
\end{frame}

\end{document}
