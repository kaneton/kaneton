%%
%% licence       kaneton licence
%%
%% project       kaneton
%%
%% file          /home/mycure/kaneton/view/papers/assignments/k2.tex
%%
%% created       matthieu bucchianeri   [tue feb  7 11:49:56 2006]
%% updated       julien quintard   [sat jun 17 13:48:10 2006]
%%

%
% k2
%

\chapter{k2}

The \textbf{k2} project consists in the development of parts of
the kaneton core leading to a complete memory management.

These microkernel's parts include:

\begin{itemize}
  \item
    A set manager's implementation.
  \item
    The segment manager.
  \item
    The region manager.
\end{itemize}

Since the kaneton microkernel heavily use the set manager to store data,
the student should perfectly understand its design and implementation.
For this reason, the first assignment will be to implement a set
implementation.

The segment manager is crucial since it will be used everywhere to allocate
memory.

Finally the region manager is also very important since it provide a complete
interface to manipulate virtual memory.

Note that the student must also fill in the architecture-dependent code
to make the full memory management working. This means that the student
should perfectly understand the segment, region and address space manager
to be able to develop the whole architecture implementation.

\newpage

%
% informations
%

\section{Informations}

\begin{tabular}{p{7cm}l}
Duration: & One week \\
File name: & \textit{[group]}-k2.tar.gz \\
In charge: & Julien Quintard \\
Newgroup: & kaneton-students@googlegroups.com \\
Languages: & Assembly and C \\
Students per group: & Three \\
Target directories:
  & \textit{kaneton/core/set/} \\
  & \textit{kaneton/core/segment/} \\
  & \textit{kaneton/core/region/} \\
\end{tabular}

%
% assignments
%

\section{Assignments}

In this part of the kaneton project some important notions will be
studied.

First, the students will have to write entire managers. This point
is very important since one of the fundamental concept of kaneton microkernel
is to be subdivided into managers.

Second, the kaneton microkernel was designed to be ported on many
architectures. Indeed, the kaneton core uses specific macro functions
to import and call architecture dependent stuff. The student will also
have to understand this concept and to study how it is implemented in the
kaneton microkernel, this feature being heavily used in the region
manager since the big part of the region manager is generally located in
the architecture-dependent code.

In the following sections, we will detail each manager.

%
% set manager
%

\subsection{set manager}

The set manager provides an interface to create and manipule sets. These
sets can be used to store data without taking care of anything.

The manager's code is given, see \textit{kaneton/core/set/set.c} and
\textit{kaneton/include/kaneton/set.h}. Take a look at the information
section in the header of the manager's code: every step of the set creation
is described.

In this project, the student will have to write the entire code for
the \textbf{l}inked-\textbf{l}ist set implementation.

Notice that, the set implementation interface must be respected.

The student so has to fill in the file \textit{kaneton/core/set/set\_ll.c}

%
% segment manager
%

\subsection{segment manager}

The segment manager provides a complete interface to manipulate physical
memory including reserving, modifying, releasing physical memory areas
called \textbf{segments}.

The student has to write the entire segment manager. Needless to say, the
student's segment manager must be compliant with the segment manager
interface.

Moreover, the architecture-dependent code will have to be written using
kaneton internal portability facilities.

Note that some functions of the segment manager are provided including
\textit{segment\_resize()}, \textit{segment\_split()},
\textit{segment\_coalesce()}, \textit{segment\_read()},
\textit{segment\_write()}, \textit{segment\_copy()} and
\textit{segment\_catch()}.

%
% region manager
%

\subsection{region manager}

The region manager provides everything necessary to manage virtual
memory areas called \textbf{regions}.

The student has to write the entire region manager including any
machine-dependent code.

Moreover, the architecture-dependent code will have to be written.

Note that some functions of the region manager are provided including
\textit{region\_resize()}, \textit{region\_split()} and
\textit{region\_coalesce()}.

%
% advanced topics
%

\section{Advanced Topics}

We advise students to first implement very simple algorithms for the segment
and region manager.

Nevertheless, once working, students might be able to implement better
allocation algorithms for the segment and region manager, like Buddy Systems,
advanced data structures etc..

%
% ia32
%

\section{Intel Architecture 32-bit}

For the segment manager, the only work of the architecture-dependent source
code is to get control over the protected mode so to rebuild the Global
Descriptor Table for the kaneton needs.

Students must implement all dependent functions they need.

Remember that the architecture-dependent code should heavily use the
\textit{libia32} functions.

Take a look at the kaneton reference paper for more information about
machine-dependent calls.

%
% Tips
%

\subsubsection{Tips}

\begin{itemize}
  \item
    Segment sizes are aligned on page size. Assuming you are not using
    four megabytes pages, this value will be 4096.

    Remember to use the \textbf{PAGESZ} macro.
  \item
    You must create two segments per task class: core, driver, service
    and user program.

    One of the segment is for the code while the other is for the data.

    Each one must have the appropriate privilege level and permissions.
\end{itemize}
