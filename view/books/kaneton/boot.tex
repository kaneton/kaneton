%%
%% licence       kaneton licence
%%
%% project       kaneton
%%
%% file          /home/mycure/kaneton/view/papers/kaneton/boot.tex
%%
%% created       matthieu bucchianeri   [mon jan 30 17:33:29 2006]
%% updated       julien quintard   [sun apr  9 22:14:20 2006]
%%

%
% boot
%

\chapter{Boot}

In this chapter we will describe the kaneton boot system, more precisly
the kaneton bootloader.

Note that many kaneton core design choices make the bootloader understanding
difficult to students. We will try to explain the design choices to make
things clear.

\newpage

%
% text
%

%
% bootstrap
%

\section{Bootstrap}

We call \textbf{boostrap} the piece of code which is first launched by the
machine and then which launchs the bootloader.

The main goal of the bootstrap is to launch the bootloader in better
circumstances.

In kaneton, another bootloader generally plays the bootstrap role.

%
% bootloader
%

\section{Bootloader}

In kaneton, one of the \textbf{bootloader}'s role is to prepare a correct
execution context for the core.

To do so, the bootloader has to install the proper memory addressing model
the kernel is expecting to, depending on the architecture.

Another bootloader's role is to provide the kernel enough information
on the current microprocessor's architecture.

To do so, the bootloader has to fill in an initialization structure
\textit{t\_init} which then will be passed to the core when launched.

This structure will inform the core of the location of some
fundamental objects like the kernel code, the kernel stack, the
initialization structure, the modules etc..

Moreover, and for design reasons, the bootloader has to specify the
pre-reserved core segments, the pre-reserved core regions, the alloc
survival area and other specific kaneton design components so the core
can properly and correctly initialize its data structures based on the
architecture via the initialization structure.

We will now detail these components to better understand the design
choices and their resulting effects.

%
% modules
%

\subsubsection{Modules}

We call \textbf{module} any additionnal file passed at the boot time.

A module can be a picture, a binary object, a configuration file
or anything else.

Do not consider a module in the other kernels terms. A kaneton module
can be anything while the term \textit{module} is generally used to describe
a kernel module, a dynamic part of code which can be attached to the
running kernel.

In kaneton, modules are used to pass the kaneton configuration file
which contains the kaneton parameters and describes the very first
kaneton servers to launch. The kaneton servers' binary objects to
launch are also passed as modules.

Of course the bootstrap has to handle kind of modules, otherwise it will
not be capable of passing them to the bootloader. For this reason,
a multi-bootloader is generally used as kaneton boostrap.

The bootloader must build a special data structure containing everything
necessary to get access to the modules including: number of modules,
modules' information, modules' content etc.. Then the core will be
able to use it to retrieve the modules.

In fact, the core itself will not use the modules. Indeed, a dedicated
server will be launched by the kernel and will get access to the modules.
Then this dedicated server will launch the very first servers described
by the kaneton configuration file.

To do so, the kernel will first launch this dedicated server and then
will explicitly give the area containing the modules to this dedicated
server.

It is obviously simpler to give a single area containing the modules
stuff rather than giving an area containing the modules information, then
an area per module and finally another area containing the modules' string
names.

For these reasons, a very specific area must be built to contain
to whole stuff in relation with the modules so the kernel just has to
give this area to the dedicated server.

The modules area is of the following form:

\begin{enumerate}
  \item
    A \textit{t\_modules} structure specifying some general information on
    the modules like the number of modules.
  \item
    An array of variable length modules, each module being composed of:

    \begin{enumerate}
      \item
	A descriptor \textit{t\_module} containing the module's size and
	a pointer to the module's name.
      \item
	The module content.
      \item
	The module's string name terminated by a zero character.
    \end{enumerate}
\end{enumerate}

Be careful, the array of modules is a variable length item array.
Module data and name are variable length blocks.

The dedicated server will refer to the module descriptors to browse
throught the modules.

The calculation used to compute the next module address is:

\begin{verbatim}
sizeof(t_module) + module->size + strlen(module->name) + 1
\end{verbatim}

\notice{A smart idea would be to add a \textit{next} field in the
  \textit{t\_module} structure to directly go through the next module.
  Obviously we could add another interesting field: \textit{content}
  referencing the module's data.}

%
% core segments
%

\subsubsection{Core Segments}

A \textbf{segment} is a contiguous area of used physical memory.

Notice that a segment in the kaneton terms has nothing to do with
a segment in some architectures terms.

In fact, the kaneton microkernel has to provide physical memory operations to
reserve, release physical memory areas or segments. To do so, the kaneton
core is composed of a segment manager which provides an interface to
manipulate the segments.

As an example, let's consider the segment reservation which is a very
common segment operation. This operation looks for a contiguous area
of unused physical memory in its internal data structure and then
returns an identifier on this area.

Now, consider that, at the core boot time, the segment manager has
an empty data structure. Then the segment manager can choose any
unused area and return an identifier on it. By \textit{any unused area} we
mean any unused physical memory area including the kernel code area, the kernel
stack area, the modules area, the architecture dependent areas like maybe the
video mapped memory area and the BIOS mapped memory area etc..

You should understand that the kaneton core needs to know which physical
memory area should not be reserved or in other terms, are pre-reserved.

So, the bootloader must build a data structure containing the set of
pre-reserved core segments. By pre-reserved segments we mean every
fundamental and distinctive areas. We will call this area which
contains these specific segments the \textit{core segments area}.

This area will be used by the task manager to properly initialize
the kernel's address space used by the kernel task object.

In other words, these core segments describe the initial state of the
segment manager, so the initial state of the physical memory held by the
core.

Therefore it will be impossible, for example, to reserve an area
already held for something else because the segment manager internal
data structure now contains the pre-reserved core segments held by the
core.

The core segments are segment objects which are partially
initialized. Indeed, some segment object properties cannot be set
by the bootloader and will be set by the segment manager.

\notice{Once the segment manager is initialized, the core segments
  get useless, so they can be released.}

%
% core regions
%

\subsubsection{Core Regions}

A \textbf{region} is a contiguous area of used virtual memory.

A region maps the whole or part of a segment.

As for the core segments, the bootloader needs \textit{core regions area}.

The core regions describe the initial state of the region manager, or
the initial core's virtual memory state.

The kernel's task manager will so go through this data structure
and properly rebuild the virtual memory from this data structure to
initialize the kernel's address space.

The core regions are region objects which are partially initialized
as the core segments.

\notice{Once the regions manager is initialized, the core regions
  get useless, so they can be released.}

%
% survival area
%

\subsubsection{Survival Area}

Like every program, kernels need memory to build and maintain internal
data structures. These data structures can for example be used to keep
track of the alive processes or to keep track of the memory state.

As an example, kernels typically have a kind of manager which keeps track
of the physical memory state. This manager internally uses a data
structure to know which memory areas are for example used.

During operating systems history, different data structures were used for
this purpose from the most basic one; the bitmap which just tells whether
a physical memory page is used or not; to more complex one like the buddy
system or complex data structures like the b-tree.

Let's consider a relatively simple data structure which only keeps track
of used physical memory area in a linked-list. Each element of the
linked-list provides enough information to describe a used physical memory
area including the base address, the area size, the permissions etc..

Note that, in kaneton terms, this manager is called the segment manager.

Therefore, to maintain such a data structure this manager needs memory.
Nevertheless, it seems impossible to allocate memory since this manager
is actually initializing its internal data structure to provide this
functionality.

You can notice that this manager needs memory while it is its role to provide
an interface to manage memory.

This mutual dependency generally leads to static and specific implementations.

Indeed, kernels usually use a very specific data structure to install
the physical memory manager. Then the other parts of the kernels can allocate
memory without taking care of anything.

But the price to pay is to implement a very specific and restrictive
data structure to provide the very first functionality: the memory management.

Then, the kernels' physical memory managers are typically ugly for this
precise reason.

We wanted kaneton to be elegant in every points. So we had to choose another
solution to this typical problem to permit kaneton to evolve in a very
coherent way where everything is dynamic and based on other kaneton
managers.

In kaneton, the data structures are managed by the set manager. This
manager uses the \textit{malloc}() function to allocate fine-grained memory.
This set manager is used by every kaneton manager to store data.

Now let's come back to our physical memory manager problem. Recall that, in
kaneton, the physical memory manager is the segment manager.

We wanted to build the segment manager over the set manager so the segment
manager can use dynamic data structures and not ad-hoc data structure
solutions.

The segment manager uses the set manager which is based on the
\textit{malloc}() function. Then the \textit{malloc}() function needs to
allocate physical memory areas to provide fine-grained allocation in them.
Once again, there is an evil mutual dependency between \textit{malloc}()
and the segment manager.

To break this mutual dependency we decided to give the \textit{malloc}()
function a pre-reserved area so it can use it to provide fine-grained
memory until the segment and region managers are initialized.

Once the segment and region managers are initialized, the
\textit{malloc}() function will just have to reserve physical memory to
perform the same work than every typical malloc() function do.

Since, at the core boot time, the segment manager is not initialized yet,
the core itself cannot decide where to put this pre-reserved area.

Therefore, the kernel relies on the bootloader to find a pre-reserved
area of physical memory for the \textit{malloc}() function. This area will
then be called the \textit{survival area}.

%
% initialization structure
%

\subsubsection{Initialization Structure}

As mentioned before, the bootloader has to build an initialization structure
and to pass it to the kernel so the kernel can locate the different
fundamental elements in main memory.

The initialization structure called \textit{t\_init} is defined in the
file \textit{kaneton/include/core/init.h}.

Note that every size fields in the initialization structure represent
sizes aligned on an architecture page size \textit{PAGESZ}.

The initialization structure contains fields to locate the different
fundamental objects in main memory like the kernel code,
the initialization structure itself, the module area, the core segments
area, the core regions area, the kernel stack and the survival
area.

The initialization structure could also contain machine-dependent fields.

Look at the file \textit{kaneton/include/arch/[architecture]/core/init.h}
for more information about the machine-dependent initialization structure
called \textit{a\_init}.
