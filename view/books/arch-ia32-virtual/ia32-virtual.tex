%%
%% licence       kaneton licence
%%
%% project       kaneton
%%
%% file          /home/buckman/kaneton/view/books/ia32-virtual/ia32-virtual.tex
%%
%% created       matthieu bucchianeri   [sat sep  2 11:23:55 2006]
%% updated       matthieu bucchianeri   [sun sep  3 12:02:35 2006]
%%

%
% template
%

%%
%% licence       kaneton licence
%%
%% project       kaneton
%%
%% file          /home/mycure/kaneton/view/templates/book.tex
%%
%% created       julien quintard   [wed mar  1 23:45:22 2006]
%% updated       julien quintard   [thu may  4 12:36:54 2006]
%%

%
% class
%

\documentclass[10pt,a4wide]{book}

%
% packages
%

\usepackage[english]{babel}
\usepackage[T1]{fontenc}
\usepackage{a4wide}
\usepackage{fancyheadings}
\usepackage{multicol}
\usepackage{indentfirst}
\usepackage{graphicx}
\usepackage{color}
\usepackage{xcolor}
\usepackage{verbatim}

\usepackage{aeguill}

\usepackage[Lenny]{../../../tools/latex/fncychap}

\pagestyle{fancy}

\setlength{\footrulewidth}{0.3pt}
\setlength{\parindent}{0.3cm}
\setlength{\parskip}{2ex plus 0.5ex minus 0.2ex}

%
% logos
%

\newcommand{\logos}
  {
    \begin{center}
      \includegraphics[scale=0.8]{../../logos/kaneton.pdf}
    \end{center}
  }

%
% colors
%

\definecolor{functioncolor}{rgb}{0.40,0.00,0.00}
\definecolor{commandcolor}{rgb}{0.00,0.00,0.40}
\definecolor{verbatimcolor}{rgb}{0.00,0.40,0.00}
\definecolor{noticecolor}{rgb}{0.87,0.84,0.02}

%
% function
%

\newcommand\function[3]{
  \begin{tabular}{p{0.2cm}p{13.8cm}}
  & {\color{functioncolor}\textbf{#1}}#2
  \end{tabular}

  \begin{tabular}{p{1cm}p{13cm}}
  & #3
  \end{tabular}}

%
% align
%

\newcommand\align[1]{
  \\ & \hspace{#1}}

%
% argument
%

\newcommand\argument[1]{\textit{#1}}

%
% command
%

\newcommand\command[2]{
  \begin{tabular}{p{0.2cm}p{13.8cm}}
  & {\color{commandcolor}\textbf{#1}}
  \end{tabular}

  \begin{tabular}{p{1cm}p{13cm}}
  & #2
  \end{tabular}}

%
% notice
%

\newcommand\notice[1]{
  {\color{noticecolor}\textbf{Notice}}

  \begin{tabular}{p{0.2cm}p{13.8cm}}
  & #1
  \end{tabular}}

%
% example
%

\newcommand\example[1]{
  \textit{Example:}

  \begin{tabular}{p{0.2cm}p{13.8cm}}
  & \textit{#1}
  \end{tabular}}

%
% warning XXX
%

%
% verbatim stuff
%

\makeatletter

\renewcommand{\verbatim@font}
  {\ttfamily\footnotesize\color{verbatimcolor}\selectfont}

\def\verbatim@processline{\hskip15ex\the\verbatim@line\par}

\makeatother

%
% header
%

\rhead{}
\rfoot{\scriptsize{The kaneton microkernel project}}

\date{\scriptsize{\today}}


%
% header
%

\lhead{\scriptsize{The kaneton microkernel project reference}}
\rhead{IA-32 implementation}

%
% title
%

\title{The kaneton microkernel reference\\IA-32 (with virtual memory)
	implementation
       \logos}

%
% authors
%

\author{\small{Julien Quintard},
        \small{Matthieu Bucchianeri} and
        \small{Renaud Voltz}}

%
% document
%

\begin{document}

%
% title
%

\maketitle

%
% --------- text --------------------------------------------------------------
%

This document describes the kaneton microkernel reference project over
Intel IA-32 architecture.

This document should be used by every student willing implement the
kaneton microkernel as people looking for more details on the kaneton
microkernel implementation on Intel architecture.

All the kaneton documents are available on
the official website
  \footnote{http://www.kaneton.org}.

\textbf{Prerequisites}: this document details the kaneton implementation
for IA-32 architecture. A knowledge of this architecture's basics is
strongly recommended.

%
% toc
%

\tableofcontents

%
% chapters
%

%%
%% licence       kaneton licence
%%
%% project       kaneton
%%
%% file          /home/mycure/kaneton/view/papers/kaneton/overview.tex
%%
%% created       matthieu bucchianeri   [mon jan 30 17:09:45 2006]
%% updated       julien quintard   [thu mar  2 13:12:22 2006]
%%

%
% overview
%

\chapter{Overview}

XXX ce chapitre va vous aider a reconnaitre les fonctionnalites principale
XXX d'un kernel dans kaneton.

The kaneton microkernel is only the core of an operating system.
Main tasks like hardware drivers or user services are implemented as
\textbf{servers}. So the microkernel only has a few functionalities to
provide:

\begin{itemize}
  \item
    Memory management.
  \item
    Process management.
  \item
    Communication.
  \item
    Events.
\end{itemize}

In this chapter we will describe briefly these tasks and all the
associated managers.

%
% memory management
%

\section{Memory Management}

Handling the memory -- from virtual address space to physical
addressing -- is done by three major managers, the \textbf{as},
\textbf{segment} and \textbf{region} managers.

%
% as
%

\subsection{as}

The address space manager just manages the different address spaces
used by the kaneton tasks.

In kaneton, we call an \textbf{as - address space} a list of memory
locations referenced by a task. Each task has its own address space.

%
% segment
%

\subsection{segment}

The segment manager just manages the segments reserved by
the different kaneton entities including the kernel, the drivers etc..

In kaneton terms a \textbf{segment} is a contiguous area of reserved
physical memory.

%
% region
%

\subsection{region}

The region manager keeps track of regions used to map segments for
each address space reserved on the system.

In kaneton, a \textbf{region} is contiguous area of virtual memory
mapping a segment's part.

%
% process management
%

\section{Process Management}

XXX

%
% communication
%

\section{Communication}

XXX

%
% events
%

\section{Events}

XXX

%
% ---------- header -----------------------------------------------------------
%
% project       kaneton
%
% license       kaneton
%
% file          /home/enguerrand/.../book/architecture/mips64/bootloader.tex
%
% created       enguerrand raymond   [sun oct  5 16:54:35 2008]
% updated       enguerrand raymond   [wed may 13 21:09:14 2009]
%

%
% booloader
%

\chapter{Bootloader}

This chapter describes the bootloading phase of kaneton on MIPS 64 systems. Each processor bootstrap are described. For the moment, the kaneton bootloader and the kernel will be loaded at this address. Later a more real firmware will be developed to load from floppy disk the kernel and these modules

\subsection{Bootloader operations}

Schema


The microprocessor initalization corresponds to a reset or a soft reset exception, see chapter 5 in \cite{MIPS64:MIPS64V3} for a complete CPU state description after these exceptions occur.

%%
%% licence       kaneton licence
%%
%% project       kaneton
%%
%% file          /home/buckman/kaneton/view/books/arch-ia32-virtual/memory.tex
%%
%% created       matthieu bucchianeri   [sat sep  2 11:40:35 2006]
%% updated       matthieu bucchianeri   [sat nov  4 18:38:29 2006]
%%

%
% memory management
%

\chapter{Memory management}

\newpage

%
% overview of the ia-32 mmu
%

\section{Overview of the IA-32 MMU}

%
% privilege-check capable segmentation model
%

\section{Privilege-check capable segmentation model}

%
% raw accesses to kaneton segments
%

\section{Raw accesses to kaneton segments}

Direct accesses to memory segments (in kaneton terms) is not possible
on IA-32, as it needs to bypass the MMU, which is impossible.

The \textit{segment\_read}, \textit{segment\_write} and
\textit{segment\_copy} dependent code is implemented using the region
manager. We create a temporary mapping using \textit{region\_reserve},
then we copy the data to a kernel buffer or directly from one segment
to another, and to finish we call \textit{region\_release}.

As a result, the three operations described above are very slow on
IA-32 architecture, and their use is not recommended.

%
% creating address spaces
%

\section{Creating address spaces}

Creating an address space is not just allocating a few structures and
filling it. It also involves preparing the address space for later
operations.

The case of the kernel address space must be handled separately, as it
is more complex and critical.

%
% creating the kernel address space
%

\subsection{Creating the kernel address space}

%
% creating a classical address space
%

\subsection{Creating a classical address space}

When creating a new address space, we must install a few basic
structures into this one:

\begin{enumerate}
\item A page-directory is built. This is done reserving a one-page
segment and the calling \textit{pd\_build} ;
\item The global memory areas must be injected and mapped: the
\textit{.handler} section, the TSS, both GDT and IDT, and to finish, a
ring 0 stack (see the \textbf{Task management} chapter for more
information about these) ;
\item The page-directory base register is given to all threads, so
when switching from one thread to another, the address space is
automatically switched.
\end{enumerate}

The need of mapping some common regions into each address space comes
from the absense of Task State Segment, an high-level context
switching mecanism implemented on IA-32. Instead, the context
switching is done by hand. Structures like the interrupt vector and a
kernel stack are needed when doing this context switch. They must be
mapped permanently, because the address space is switched to the
kernel address space only at some advanced stages of the switching
algorithm. But these structures are mapped with the ``Supervisor''
flag enabled, so even the owner of the address space cannot access it
in normal running conditions. Please refer to the \textbf{Task
management} chapter for further information.

The third point is done getting the task associated with the new
address space and browsing its thread set. For each thread, we fill
the register CR3 (the page-directory base register) with the
page-directory base, computed with \textit{pd\_get\_cr3}. See
\textbf{Address space switching} section in \textbf{Task management}
chapter.

%
% operations on regions
%

\section{Operations on regions}

The region manager defines three operations on regions:

\begin{itemize}
\item The reservation ;
\item The releasing ;
\item The resizing.
\end{itemize}

We must add the permission changing operation, coming from the segment
manager (see the corresponding paragraph for more information).

Other operations (splitting and coalescing) don't affect architecture
specific structure, and have no implementation issues on IA-32.

%
% reserving a region
%

\subsection{Reserving a region}

Reserving a region is the operation that create in a given address
space a mapping between one or more physical pages and the same amount
of virtual pages.

The only action performed to reserve a region is to fill the
page-tables tree, creating one or more page-tables and adding a few
entries into these.

As region is a high-level frontend, it only performs a mapping of
contiguous areas. The algorithm for contiguous mapping is quite simple:

\begin{enumerate}
\item We compute the \textit{pde\_start} and \textit{pde\_end}
values. These values give the interval of page-directory entries we
will need to loop thought. We use a macro named \textit{PDE\_ENTRY},
that returns for a given address the index of the page-directory entry
that leads to the good page-table ;
\item Identically, we compute \textit{pte\_start} and
\textit{pte\_end}, indicating the index of the first page-table entry
to add into the first page-table, and the last page-table entry into
the last page-table ;
\item Now, we will need to loop throught the page-directory. So, first
of all, we must map it into the current address space (the kernel
address space). This is done using the
\textit{ia32\_region\_map\_chunk} function. This particular function
uses the \textbf{mirroring technique}, explained below this algorithm ;
\item We loop throught \textit{pde\_start} and \textit{pde\_end}
page-directory entries ;
  \begin{enumerate}
  \item If the page-table we need to modify does not exist, we create
  it. This is done by reserving a segment and adding the corresponding
  page-directory entry ;
  \item Next, as we mapped the page-directory, we also need to map the
  page-table into the kernel address space in order to change it. This
  is done the same way as previously ;
  \item Now, we are able to add some page-table entries. We loop
  throught the good interval, which is determined as follow:
    \begin{itemize}
    \item If the current page-table is the first-one, then we start at
    \textit{pte\_start}, otherwise, we start at 0 ;
    \item If the current page-table is the last-one, then we stop our
    loop at \textit{pte\_end}, otherwise, we use the maximum value
    (1024), to fill all the page-table entries ;
    \end{itemize}
    \item We fill these entries. Each one maps exactly one page ;
    \item Before moving on to the next page-table, we take care of
    unmapping the previous one ;
  \end{enumerate}
\item Once the whole process is complete, we unmap the page-directory.
\end{enumerate}

The mirroring technique permits to access the current page-directory
and page-tables hierarchy in the kernel address space. To access a
page-table, this one must be mapped. And to map it, we must access
another page-table to create a mapping. As you can see, there is a
cyclic problem here.

The solution is to consider the page-directory as a page-table. One
entry in the page-directory points to the page-directory itself. This
way, accessing a page-table is done by reading or writing to a precise
address computed as follow:

$page\_table = ENTRY\_ADDR(PD\_MIRROR, page\_table\_index)$

The \textit{PD\_MIRROR} constant correspond to the index of the entry
in the page-directory that points to the page-directory. To access the
page-directory itself, the formula is the following:

$page\_directory = ENTRY\_ADDR(PD\_MIRROR, PD\_MIRROR)$

With this technique, we are able to modify the page-directory and the
page-tables without mapping them each time. Both
\textit{ia32\_region\_map\_chunk} and
\textit{ia32\_region\_unmap\_chunk} rely on the mirroring technique.

%
% releasing a region
%

\subsection{Releasing a region}

Releasing a region is exactly the same processing as for reserving
one, but instead of filling the page-table entries, we remove it.

\begin{enumerate}
\item We compute \textit{pde\_start}, \textit{pde\_end}, \textit{pte\_start}
and \textit{pte\_end} the same way ;
\item We map the page-directory into the current address space ;
\item We loop throught \textit{pde\_start} and \textit{pde\_end}
page-directory entries ;
  \begin{enumerate}
  \item We map the current page-table into the kernel address space in order
    to change it ;
  \item Now, we are able to add some page-table entries. We loop
  throught the good interval, which is determined as follow:
    \begin{itemize}
    \item If the current page-table is the first-one, then we start at
    \textit{pte\_start}, otherwise, we start at 0 ;
    \item If the current page-table is the last-one, then we stop our
    loop at \textit{pte\_end}, otherwise, we use the maximum value
    (1024), to fill all the page-table entries ;
    \end{itemize}
    \item We remove the page-table entries ;
    \item Before moving on to the next page-table, we take care of
    unmapping the current one ;
    \item If we unmapped all the entries present in the page-table, we
    remove it and release the associated segment ;
  \end{enumerate}
\item Once the whole process is complete, we unmap the page-directory.
\end{enumerate}

%
% changing permissions of a region
%

\subsection{Changing permissions of a region}

On IA-32, we manage permissions with regions and not with
segments. The \textit{segment\_perms} machine dependent code is very
similar to the two previous one.

The algorithm is quite the same, except that we modify each page-table
entry only changing the permission flags.

%
% resizing of a region
%

\subsection{Resizing of a region}

The current way of resizing a region is not optimized and rely on
other operations of the region manager for simplicity reasons.

When shrinking a region, the independent code changes the
corresponding \textit{o\_region} object size field. The work done by
the dependent code on IA-32 is to remove the invalidated page-table
entries. The current implementation inject a temporary ``virtual''
region corresponding to the memory chunk to unmap, and the directly
calls \textit{ia32\_region\_release}.

When a region needs to be enlarged, the process is very similar: we
create a region directly after the region we need to enlarge. We use
\textit{ia32\_region\_reserve} to fill the page-tables. Then, we
coalesce the two region into one unique.

%
% translation cache flushing
%

\subsection{Translation cache flushing}

The translation caches (also known as Translation Lookaside Buffers,
TLB), are the low latency memories used to make the translation of
virtual addresses into physical addresses.

In fact, the whole page-tables tree is not browsed each time the
microprocessor accesses a word in memory. Doing so will leads the
translation mecanism to be very slow (as each memory access need two
other accesses, one to read the page-directory and another to read a
page-table entry). Each time such translation is done, the
corresponding page-table entry is stored into the TLB. So, next accesses
to the same page will not need to go thought the paging tree.

But what appends when we add or remove a page translation while the TLB
already have a translation for the same address? Let's imagine the
following situation: we map the page A to the page frame B. The
microprocessor accesses a word in A. The TLB is filled with the
translation A $\rightarrow$ B. Next, we change the mapping so A
$\rightarrow$ C. Now, the CPU accesses a word in A. The translation
will be fetched from the TLB, still having A $\rightarrow$ B.

So, when such mapping change occurs, we need to invalidate one or more
TLB entries. Additionnaly, when switching from one address space to
another, we must flush the entire TLB.

On IA-32, flushing the whole TLB is done automatically when switching
address space (changing the value in the page-directory base
register). As all the functions modifying the virtual addresses
mapping are executed in kernel-land, we must flush some TLB entries
only when modifying mapping into the kernel address space. When
changing some mappings into a task address space, it is not necessary
to flush the TLB, as switching back to the task will invalidate the
whole caches.

%%
%% licence       kaneton licence
%%
%% project       kaneton
%%
%% file          /home/buckman/kaneton/view/books/ia32-virtual/event.tex
%%
%% created       matthieu bucchianeri   [sat sep  2 11:40:54 2006]
%% updated       matthieu bucchianeri   [sun sep  3 12:24:43 2006]
%%

%
% event management
%

\chapter{Event management}

%
% overview of exceptions, interrupts and hardware interrupts
%

\section{Overview of exceptions, interrupts and hardware interrupts}


%%
%% licence       kaneton licence
%%
%% project       kaneton
%%
%% file          /home/buckman/kaneton/view/books/ia32-virtual/task.tex
%%
%% created       matthieu bucchianeri   [sat sep  2 11:41:15 2006]
%% updated       matthieu bucchianeri   [sun sep  3 12:19:13 2006]
%%

%
% task management
%

\chapter{Task management}

%
% overview of execution contexts on ia-32
%

\section{Overview of execution contexts on IA-32}

%
% reserving a task
%

\section{Reserving a task}

%
% setting up the execution context
%

\section{Setting up the execution context}

%
% preparing a new thread
%

\section{Preparing a new thread}

%
% scheduling and context switch
%

\section{Scheduling and context switch}

%%
%% licence       kaneton licence
%%
%% project       kaneton
%%
%% file          /home/buckman/kaneton/view/books/ia32-virtual/io.tex
%%
%% created       matthieu bucchianeri   [sat sep  2 11:39:52 2006]
%% updated       matthieu bucchianeri   [sun sep  3 12:21:36 2006]
%%

%
% io management
%

\chapter{I/O management}

%
% overview of io on ia-32
%

\section{Overview of I/Os on IA-32}

%
% allowing or denying ios
%

\section{Allowing or denying I/Os}

%%
%% licence       kaneton licence
%%
%% project       kaneton
%%
%% file          /home/mycure/kaneton/view/papers/kaneton/bibliography.tex
%%
%% created       julien quintard   [mon may  8 18:35:35 2006]
%% updated       julien quintard   [mon may  8 20:38:56 2006]
%%

%
% bibliograpy
%

\chapter{Bibliography}

This chapter contains the bibliography.

%
% text
%

\begin{thebibliography}{0}
  \bibitem{AST-SCO}
    \textbf{Structured Computer Organization};
    by
    \textit{Andrew S. Tanenbaum}
  \bibitem{AST-CN}
    \textbf{Computer Networks};
    by
    \textit{Andrew S. Tanenbaum}
  \bibitem{AST-OSDI}
    \textbf{Operating Systems: Design and Implementation};
    by
    \textit{Andrew S. Tanenbaum, Albert S Woodhull}
  \bibitem{AST-MOS}
    \textbf{Modern Operating Systems};
    by
    \textit{Andrew S. Tanenbaum}
  \bibitem{AST-DOS}
    \textbf{Distributed Operating Systems};
    by
    \textit{Andrew S. Tanenbaum}
  \bibitem{AST-DSPP}
    \textbf{Distributed Systems: Principles and Paradigms};
    by
    \textit{Andrew S. Tanenbaum, Maarten van Steen}
  \bibitem{NAL-DA}
    \textbf{Distributed Algorithms};
    by
    \textit{Nancy A. Lynch}
\end{thebibliography}


\end{document}
