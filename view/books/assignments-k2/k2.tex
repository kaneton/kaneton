%%
%% licence       kaneton licence
%%
%% project       kaneton
%%
%% file          /home/buckman/kaneton/view/books/assignments-k1/k1.tex
%%
%% created       matthieu bucchianeri   [tue feb  7 11:49:56 2006]
%% updated       matthieu bucchianeri   [sun feb 11 22:54:06 2007]
%%

%
% k2
%

\chapter{K2: event management}

%
% informations
%

\begin{tabular}{p{7cm}l}
Duration: & 2 weeks \\
Directory name: & kaneton/ \\
In charge: & Matthieu Bucchianeri \& Renaud Voltz\\
Mailing-list: & kaneton-students@googlegroups.com \\
Languages: & C \\
Students per group: & 2 (same groups as for K1) \\
\end{tabular}

\section{Abstract}

K2 project consists in developing XXX

\begin{enumerate}
  \item
    {\bf The event manager}\\
    XXX
  \item
    {\bf The time manager}\\
    XXX
\end{enumerate}

XXX

%
% event manager
%

\newpage

\section{\textbf{event} manager}

\begin{itemize}
  \item {\bf Overview}\\
    XXX
  \item {\bf Assignments}\\
    XXX
  \item {\bf Interface}\\
    XXX
  \item {\bf {Files}}\\

    \begin{tabular}{| l | l |}
      \hline
      machine-independent & {\em kaneton/core/event/event.c}\\
      machine-dependent & {\em kaneton/core/arch/ibm-pc.ia32-virtual/event.c}\\
      libarch & {\em libs/libia32/pmode/idt.c}\\
      & {\em libs/libia32/include/pmode/idt.h}\\
      & {\em libs/libia32/interrupt/*.c}\\
      & {\em libs/libia32/include/interrupt/*.h}\\\hline
    \end{tabular}
\end{itemize}


%
% tine manager
%

\newpage

\section{\textbf{time} manager}
\begin{itemize}
  \item {\bf Overview}\\
    XXX
  \item {\bf Assignments}\\
    XXX
  \item {\bf Interface}\\
    XXX
  \item {\bf Files}\\

    \begin{tabular}{| l | l |}
      \hline
      machine-independent & {\em kaneton/core/time/timer.c}\\
      machine-dependent & {\em kaneton/core/arch/ibm-pc.ia32-virtual/timer.c}\\
      libarch & {\em libs/libia32/time/timer.c}\\
      &  {\em libs/libia32/include/time/pit.h}\\\hline
    \end{tabular}

\end{itemize}

%
% advanced topics
%

\newpage

\section{Bonuses}

kaneton microkernel is first of all a pedagogical project which do not
aims at being optimized. That is why, when nothing is specified, you
always will implement the simplest algorithms.\\
\\
Nevertheless, we will always encourage students who want to write
additional bonuses, as far as they respect the following rules:

\begin{enumerate}
  \item Bonuses will be evaluated only if a basic implementation is
  actually working.
  \item Bonuses must be either picked from the following list, or
  accepted by the kaneton team.\\
\end{enumerate}

Bonuses ideas:
\begin{itemize}
\item XXX
\item XXX
\end{itemize}
