%%
%% licence       kaneton licence
%%
%% project       kaneton
%%
%% file          /home/mycure/kaneton/view/lectures/kaneton/kaneton.tex
%%
%% created       julien quintard   [sun dec  4 16:40:51 2005]
%% updated       julien quintard   [sun dec  4 17:06:23 2005]
%%

%
% template
%

%%
%% copyright     (c) julien quintard
%%
%% project       kaneton
%%
%% file          /home/mycure/kaneton/view/templates/lecture.tex
%%
%% created       julien quintard   [sat nov 19 17:13:03 2005]
%% updated       julien quintard   [fri dec  2 22:36:34 2005]
%%

%
% class
%

\documentclass[8pt]{beamer}

%
% packages
%

\usepackage{pgf,pgfarrows,pgfnodes,pgfautomata,pgfheaps,pgfshade}
\usepackage{colortbl}
\usepackage{times}
\usepackage{amsmath,amssymb}
\usepackage{graphics}
\usepackage{graphicx}
\usepackage{color}
\usepackage{xcolor}
\usepackage[english]{babel}
\usepackage{enumerate}
\usepackage[latin1]{inputenc}

%
% style
%

\usepackage{beamerthemesplit}
\setbeamercovered{dynamic}

%
% verbatim font
%

\definecolor{verbatimcolor}{rgb}{0,0.4,0}

\makeatletter
\renewcommand{\verbatim@font}
  {\ttfamily\footnotesize\color{verbatimcolor}\selectfont}
\makeatother

%
% new line
%

\newcommand{\nl}[0]{\vspace{0.4cm}}

%
% date
%

\date{\today}

%
% logos
%

\pgfdeclareimage[interpolate=true,width=34pt,height=18pt]
                {epita}{../../logos/epita}
\pgfdeclareimage[interpolate=true,width=49pt,height=18pt]
                {upmc}{../../logos/upmc}
\pgfdeclareimage[interpolate=true,width=25pt,height=18pt]
                {lse}{../../logos/lse}

\newcommand{\logos}
  {
    \pgfuseimage{epita}
  }

%
% institute
%

\institute
{
  \inst{1} kaneton microkernel project
}

%
% table of contents at the beginning of each section
%

\AtBeginSection[]
{
  \begin{frame}<beamer>
   \frametitle{Outline}
    \tableofcontents[current]
  \end{frame}
}

%
% table of contents at the beginning of each subsection
%

\AtBeginSubsection[]
{
  \begin{frame}<beamer>
   \frametitle{Outline}
    \tableofcontents[current,currentsubsection]
  \end{frame}
}


%
% title
%

\title{kaneton}

%
% authors
%

\author
{
  Julien~Quintard\inst{1}
}

%
% document
%

\begin{document}

%
% title frame
%

\begin{frame}
  \titlepage

  \begin{center}
    \logos
  \end{center}
\end{frame}

%
% outline frame
%

\begin{frame}
  \frametitle{Outline}
  \tableofcontents
\end{frame}

%
% overview
%

\section{Overview}

% 1)

\begin{frame}
  \frametitle{Description}

  \begin{itemize}
    \item
      About \textbf{11} hours of course.
    \item
      Evaluated via the kaneton project.
  \end{itemize}
\end{frame}

% 2)

\begin{frame}
  \frametitle{Overview}

  Through this course we will study kernel designs from a critical
  point of view.

  \nl

  There exists many different kernels in many different kernel categories.

  \nl

  All these kernels are different, but what are these differencies.

  \nl

  This course will try to answer these questions through the
  kaneton microkernel design.

  \nl

  This course is divided in sessions, each session dealing with the
  related project. Each session tries to give students a complete view
  on kernel designers choices.
\end{frame}

%
% k0
%

\section{k0}

\begin{frame}
  \frametitle{Overview}

  
\end{frame}

\end{document}
