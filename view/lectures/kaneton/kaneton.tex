%%
%% licence       kaneton licence
%%
%% project       kaneton
%%
%% file          /home/mycure/kaneton/view/lectures/kaneton/kaneton.tex
%%
%% created       julien quintard   [sun dec  4 16:40:51 2005]
%% updated       julien quintard   [mon dec 26 21:43:46 2005]
%%

%
% template
%

%%
%% copyright     (c) julien quintard
%%
%% project       kaneton
%%
%% file          /home/mycure/kaneton/view/templates/lecture.tex
%%
%% created       julien quintard   [sat nov 19 17:13:03 2005]
%% updated       julien quintard   [fri dec  2 22:36:34 2005]
%%

%
% class
%

\documentclass[8pt]{beamer}

%
% packages
%

\usepackage{pgf,pgfarrows,pgfnodes,pgfautomata,pgfheaps,pgfshade}
\usepackage{colortbl}
\usepackage{times}
\usepackage{amsmath,amssymb}
\usepackage{graphics}
\usepackage{graphicx}
\usepackage{color}
\usepackage{xcolor}
\usepackage[english]{babel}
\usepackage{enumerate}
\usepackage[latin1]{inputenc}

%
% style
%

\usepackage{beamerthemesplit}
\setbeamercovered{dynamic}

%
% verbatim font
%

\definecolor{verbatimcolor}{rgb}{0,0.4,0}

\makeatletter
\renewcommand{\verbatim@font}
  {\ttfamily\footnotesize\color{verbatimcolor}\selectfont}
\makeatother

%
% new line
%

\newcommand{\nl}[0]{\vspace{0.4cm}}

%
% date
%

\date{\today}

%
% logos
%

\pgfdeclareimage[interpolate=true,width=34pt,height=18pt]
                {epita}{../../logos/epita}
\pgfdeclareimage[interpolate=true,width=49pt,height=18pt]
                {upmc}{../../logos/upmc}
\pgfdeclareimage[interpolate=true,width=25pt,height=18pt]
                {lse}{../../logos/lse}

\newcommand{\logos}
  {
    \pgfuseimage{epita}
  }

%
% institute
%

\institute
{
  \inst{1} kaneton microkernel project
}

%
% table of contents at the beginning of each section
%

\AtBeginSection[]
{
  \begin{frame}<beamer>
   \frametitle{Outline}
    \tableofcontents[current]
  \end{frame}
}

%
% table of contents at the beginning of each subsection
%

\AtBeginSubsection[]
{
  \begin{frame}<beamer>
   \frametitle{Outline}
    \tableofcontents[current,currentsubsection]
  \end{frame}
}


%
% title
%

\title{kaneton}

%
% authors
%

\author
{
  Julien~Quintard\inst{1}
}

%
% document
%

\begin{document}

%
% title frame
%

\begin{frame}
  \titlepage

  \begin{center}
    \logos
  \end{center}
\end{frame}

%
% outline frame
%

\begin{frame}
  \frametitle{Outline}
  \tableofcontents
\end{frame}

%
% overview
%

\section{Overview}

% 1)

\begin{frame}
  \frametitle{Description}

  \begin{itemize}
    \item
      About \textbf{11} hours of course.
    \item
      Evaluated via the kaneton project.
  \end{itemize}
\end{frame}

% 2)

\begin{frame}
  \frametitle{Overview}

  Through this course we will study kernel designs from a critical
  point of view.

  \nl

  There exists many different kernels in many different kernel categories.

  \nl

  All these kernels are different, but what are these differencies.

  \nl

  This course will try to answer these questions through the
  kaneton microkernel design.

  \nl

  This course is divided in sessions, each session dealing with the
  related project. Each session tries to give students a complete view
  on kernel designers choices.
\end{frame}

% 3)

\begin{frame}
  \frametitle{Architectures}

  This project should be develop on any architecture, the students group
  choosing its architecture.

  \nl

  Nevertheless, this is not possible for some reasons.

  \nl

  The architecture used for the kaneton project is the Intel 32-bit
  architecture.
\end{frame}

%
% k0
%

\section{k0}

% 1)

\begin{frame}
  \frametitle{Overview}

  The goal of the \textbf{k0} project is to develop a boostrap.

  \nl

  This project was introduced in the kaneton project only for
  external Intel architecture reasons.

  \nl

  Indeed, the students did not understand the different memory models
  or only in theory. To make this more understandable, we decided to
  introduce a new project where students will have to deal with very
  low programming and the very first memory model: the Intel real mode.
\end{frame}

% 2)

\begin{frame}
  \frametitle{Assignments}

  The students just have to develop a boostrap.

  \nl

  The boostrap has three phases:

  \begin{enumerate}[<+->]
    \item
      Displays some messages using BIOS interrupts facilities.
    \item
      Installs the new memory model.
    \item
      Launches the bootloader.
  \end{enumerate}
\end{frame}

%
% k1
%

\section{k1}

% 1)

\begin{frame}
  \frametitle{Overview}

  The goal of the \textbf{k1} project is to develop a bootloader.

  \nl

  The bootloader's role is to install an execution environment for the
  kernel's futur execution.

  \nl

  In other terms, it is for example better to install the virtual memory
  and then to launch the kernel. Then the kernel will be executed with the
  virtual memory available.

  \nl

  In fact, the bootloader deals with low level programming and generally
  installs everything needed by the kernel. Then the kernel can be launched
  and can run without taking care of low-level stuff.
\end{frame}

% 2)

\begin{frame}
  \frametitle{Info Structure}

  We decided to create a gate between the bootloader and the kernel as
  multibootloaders like grub and lilo already do.

  \nl

  Indeed, such multibootloaders provide a multiboot info structure containing
  every low-level information like the amount of physical memory, the type
  of the kernel binary, the list of the modules etc..

  \nl

  We wanted the students to understand the programming in a very strict
  environment where no memory allocator is present, where we can read and
  write the entire memory while this must be done very carefully.

  \nl

  To introduce this, we decided to build an information structure provided
  by the bootloader to the kernel. Then, everyone can write a new
  bootloader of course being compliant with this structure.
\end{frame}

% 3)

\begin{frame}
  \frametitle{Assignments}

  The students will have to deal with very low-level programming:

  \begin{itemize}[<+->]
    \item
      Install the protected mode.
    \item
      Install the virtual memory.
    \item
      Deal with the stack.
  \end{itemize}

  and to develop in a very strict environment.

  \nl

  Indeed, the information structure is extremely compact. So each group
  will have to build this structure which contains specific substructures
  like segments, regions which are dynamic data structure and modules
  which have to be located in a specific place.
\end{frame}

\end{document}
