%%
%% licence       kaneton licence
%%
%% project       kaneton
%%
%% file          /home/buckman/kaneton/view/lectures/lse01/lse01.tex
%%
%% created       julien quintard   [fri dec  2 07:53:59 2005]
%% updated       matthieu bucchianeri   [tue apr 11 00:07:37 2006]
%%

%
% template
%

%
% ---------- header -----------------------------------------------------------
%
% project       kaneton
%
% license       kaneton
%
% file          /home/mycure/kaneton/view/template/lecture.tex
%
% created       julien quintard   [wed may 16 18:17:26 2007]
% updated       julien quintard   [sun may 18 23:23:40 2008]
%

%
% class
%

\documentclass[8pt]{beamer}

%
% packages
%

\usepackage{pgf,pgfarrows,pgfnodes,pgfautomata,pgfheaps,pgfshade}
\usepackage[T1]{fontenc}
\usepackage{colortbl}
\usepackage{times}
\usepackage{amsmath,amssymb}
\usepackage{graphics}
\usepackage{graphicx}
\usepackage{color}
\usepackage{xcolor}
\usepackage[english]{babel}
\usepackage{enumerate}
\usepackage[latin1]{inputenc}
\usepackage{verbatim}
\usepackage{aeguill}

%
% style
%

\usepackage{beamerthemesplit}
\setbeamercovered{dynamic}

%
% verbatim stuff
%

\definecolor{verbatimcolor}{rgb}{0.00,0.40,0.00}

\makeatletter

\renewcommand{\verbatim@font}
  {\ttfamily\footnotesize\selectfont}

\def\verbatim@processline{
  {\color{verbatimcolor}\the\verbatim@line}\par
}

\makeatother

%
% -
%

\renewcommand{\-}{\vspace{0.4cm}}

%
% date
%

\date{\today}

%
% logos
%

\pgfdeclareimage[interpolate=true,width=34pt,height=18pt]
                {epita}{\path/logo/epita}
\pgfdeclareimage[interpolate=true,width=49pt,height=18pt]
                {upmc}{\path/logo/upmc}
\pgfdeclareimage[interpolate=true,width=25pt,height=18pt]
                {lse}{\path/logo/lse}

\newcommand{\logos}
  {
    \pgfuseimage{epita}
  }

%
% institute
%

\institute
{
  \inst{1} kaneton microkernel project
}

%
% table of contents at the beginning of each section
%

\AtBeginSection[]
{
  \begin{frame}<beamer>
   \frametitle{Outline}
    \tableofcontents[current]
  \end{frame}
}

%
% table of contents at the beginning of each subsection
%

\AtBeginSubsection[]
{
  \begin{frame}<beamer>
   \frametitle{Outline}
    \tableofcontents[current,currentsubsection]
  \end{frame}
}


%
% title
%

\title{kaneton LSE internal presentation}

%
% authors
%

\author
{
  Matthieu~Bucchianeri and Renaud~Voltz\inst{1}
}

%
% figures
%

%
% document
%

\begin{document}

%
% title frame
%

\begin{frame}
  \titlepage

  \begin{center}
    \logos
  \end{center}
\end{frame}

%
% outline frame
%

\begin{frame}
  \frametitle{Outline}
  \tableofcontents
\end{frame}

%
% recalls
%

\section{Recalls}

%
% the bootloader
%

\subsection{The bootloader}

% 1)

\begin{frame}
  \frametitle{The bootloader}

  Goals:

  \nl

  \begin{itemize}[<+->]
  \item Setup a basic runtime environment for the kernel:
    \begin{itemize}[<+->]
    \item Relocate the kernel code
    \item Build the ``init'' record with all necessary information
    \item Prepare protected mode and paging for the kernel
    \item Make a basic identity mapping of necessary areas
    \item Install the kernel stack
    \item Mark prereserved segments of memory
    \end{itemize}
  \item Scheme of memory organization
  \end{itemize}

\end{frame}

%
% sets
%

\subsection{Sets}

% 1)

\begin{frame}
  \frametitle{Sets}
\end{frame}

%
% objects hierarchy
%

\subsection{Objects hierarchy}

% 1)

\begin{frame}
  \frametitle{Objects hierarchy}
\end{frame}

%
% memory management
%

\section{Memory management}

%
% summary
%

\subsection{Summary}

% 1)

\begin{frame}
  \frametitle{Summary of memory management}

  Goals:

  \nl

  \begin{itemize}[<+->]
  \item Allocate physical memory for processes
  \item Map this physical memory in a process address space
  \item Manipulate easily address spaces
  \end{itemize}

  \nl

  Components:

  \nl

  \begin{itemize}[<+->]
  \item Address space (bound to a task)
  \item Segments (physical memory chunks)
  \item Regions (virtual mapping of segments)
  \end{itemize}

\end{frame}

%
% address spaces
%

\subsection{Address spaces}

% 1)

\begin{frame}
  \frametitle{Description}

  An address space:

  \nl

  \begin{itemize}[<+->]
  \item Is bound to one task only
  \item Describe a task's physical and virtual memory
  \item Is able to make translations between physical and virtual addresses
  \item Is able to clone itself
  \end{itemize}

  \nl

  To summarize:

  \nl

  \begin{itemize}[<+->]
  \item A set of segments
  \item A set of regions
  \end{itemize}

\end{frame}

% 2)

\begin{frame}
  \frametitle{Intel IA32 implementation}

  With Intel architecture, there are two main ways to implement address spaces:

  \nl

  \begin{itemize}
  \item Segmentation
  \item Paging
  \end{itemize}

\end{frame}

% 3)

\begin{frame}
  \frametitle{Intel IA32 implementation\\Segmentation}

  \begin{itemize}[<+->]
  \item Implement some kind of ``windows'' of addressable memory (with a base and a limit)
  \item Selecting a ``window'' is made via segment selector registers
  \end{itemize}

  \nl

  Pros:

  \begin{itemize}[<+->]
  \item Quick: translation is done adding the base of a segment to the linear address and checking the limit and privileges
  \end{itemize}

  Cons:

  \begin{itemize}[<+->]
  \item Code must be position independent
  \item Always have to manipulate segment selectors registers
  \item Address space is not 4 Gb large and limited to real amount of physical memory
  \item Ares are contiguous
  \end{itemize}

\end{frame}

% 4)

\begin{frame}
  \frametitle{Intel IA32 implementation\\Paging}

  \begin{itemize}[<+->]
  \item A virtual address is divided in three fields: the page directory index, the page table index and the offset
  \item Translation is done using tables: one page directory and multiple page tables
  \end{itemize}

  \nl

  Pros:

  \begin{itemize}[<+->]
  \item 4 Gb of addressable memory
  \item Only one register to change when switching address space (PDBR)
  \item Contiguous data in the address space may not be contiguous in physical memory
  \end{itemize}

  Cons:

  \begin{itemize}[<+->]
  \item More complex (page directory \& page table manipulation)
  \item A bit slower: translations need to follow links in page tables (a cache called the Translation Lookaside Buffer is used to minimize memory accesses)
  \end{itemize}

\end{frame}

% 4)

\begin{frame}
  \frametitle{Intel IA32 implementation\\Details on the ia32-virtual profile}

  For the moment, the only available solution is using the paging mecanism.

  \nl

  So an address space is represented only by its page-directory base register.
  Reserving an address space means:

  \begin{itemize}[<+->]
  \item Reserving a segment for the page directory
  \item Mapping the created page-directory to clear it
  \item Filling an internal field holding the PDBR
  \end{itemize}

  \nl

  Address space switching will be detailed later in this lecture.

\end{frame}

% 5)

\begin{frame}
  \frametitle{The special case of the kernel address space}

  The case of the kernel address space must be handled in a special way. We must reuse the page-directory created by the bootloader but we must clean everything we've mapped and that is not longer useful.

  Additionally, we must ensure that currently used page-tables (we don't know how many they are since it depends on kernel code and modules size) will not be erased so we must inject them into the kernel address space.

  And to finish, we must add the so called ``mirroring'' entry, used to access page tables.

\end{frame}

%
% segments
%

\subsection{Segments}

% 1)

\begin{frame}
  \frametitle{Description}
\end{frame}

% 2)

\begin{frame}
  \frametitle{Operations}
\end{frame}

% 3)

\begin{frame}
  \frametitle{Prereserved segments and catcheable segments}
\end{frame}

%
% regions
%

\subsection{Regions}

% 1)

\begin{frame}
  \frametitle{Description}
\end{frame}

% 2)

\begin{frame}
  \frametitle{Operations}
\end{frame}

% 3)

\begin{frame}
  \frametitle{Mapping process on IA32}
\end{frame}

%
% higher level functionalities
%

\subsection{Higher-level functionalities}

% 1)

\begin{frame}
  \frametitle{The map manager}
\end{frame}

% 2)

\begin{frame}
  \frametitle{kernel malloc function}
\end{frame}

%
% integration with tasks
%

\subsection{Integration with tasks}

% 1)

\begin{frame}
  \frametitle{Address space switch}
\end{frame}

% 2)

\begin{frame}
  \frametitle{Accessing an address space from the kernel}
\end{frame}

%
% events management
%

\section{Events management}

%
% test suite
%

\section{Test suite}

%
% testing students tarballs
%

\subsection{Testing students tarballs}

% 1)

\begin{frame}
  \frametitle{Steps in testing a student tarball}
\end{frame}

%
% the test suite
%

\subsection{The test suite}

% 1)

\begin{frame}
  \frametitle{The test suite}
\end{frame}

\end{document}
