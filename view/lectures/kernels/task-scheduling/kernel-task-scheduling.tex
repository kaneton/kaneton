%%
%% licence       kaneton licence
%%
%% project       kaneton
%%
%% file          /home/buckman/kaneton/view/lectures/kernels/task-scheduling/kernel-task-scheduling.tex
%%

%
% template
%

%
% ---------- header -----------------------------------------------------------
%
% project       kaneton
%
% license       kaneton
%
% file          /home/mycure/kaneton/view/template/lecture.tex
%
% created       julien quintard   [wed may 16 18:17:26 2007]
% updated       julien quintard   [sun may 18 23:23:40 2008]
%

%
% class
%

\documentclass[8pt]{beamer}

%
% packages
%

\usepackage{pgf,pgfarrows,pgfnodes,pgfautomata,pgfheaps,pgfshade}
\usepackage[T1]{fontenc}
\usepackage{colortbl}
\usepackage{times}
\usepackage{amsmath,amssymb}
\usepackage{graphics}
\usepackage{graphicx}
\usepackage{color}
\usepackage{xcolor}
\usepackage[english]{babel}
\usepackage{enumerate}
\usepackage[latin1]{inputenc}
\usepackage{verbatim}
\usepackage{aeguill}

%
% style
%

\usepackage{beamerthemesplit}
\setbeamercovered{dynamic}

%
% verbatim stuff
%

\definecolor{verbatimcolor}{rgb}{0.00,0.40,0.00}

\makeatletter

\renewcommand{\verbatim@font}
  {\ttfamily\footnotesize\selectfont}

\def\verbatim@processline{
  {\color{verbatimcolor}\the\verbatim@line}\par
}

\makeatother

%
% -
%

\renewcommand{\-}{\vspace{0.4cm}}

%
% date
%

\date{\today}

%
% logos
%

\pgfdeclareimage[interpolate=true,width=34pt,height=18pt]
                {epita}{\path/logo/epita}
\pgfdeclareimage[interpolate=true,width=49pt,height=18pt]
                {upmc}{\path/logo/upmc}
\pgfdeclareimage[interpolate=true,width=25pt,height=18pt]
                {lse}{\path/logo/lse}

\newcommand{\logos}
  {
    \pgfuseimage{epita}
  }

%
% institute
%

\institute
{
  \inst{1} kaneton microkernel project
}

%
% table of contents at the beginning of each section
%

\AtBeginSection[]
{
  \begin{frame}<beamer>
   \frametitle{Outline}
    \tableofcontents[current]
  \end{frame}
}

%
% table of contents at the beginning of each subsection
%

\AtBeginSubsection[]
{
  \begin{frame}<beamer>
   \frametitle{Outline}
    \tableofcontents[current,currentsubsection]
  \end{frame}
}


%
% title
%

\title{Kernels - Task and scheduling}

%
% authors
%

\author
{
  Matthieu~Bucchianeri and Renaud~Voltz\inst{1}
}

%
% figures
%

%
%\pgfdeclareimage[interpolate=true,width=188pt,height=97pt]
%                {sample}
%		{figures/sample}

%
% document
%

\begin{document}

%
% title frame
%

\begin{frame}
  \titlepage

  \begin{center}
    \logos
  \end{center}
\end{frame}

%
% outline frame
%

\begin{frame}
  \frametitle{Outline}
  \tableofcontents
\end{frame}

%
% multitasking basics
%

\section{Multitasking basics}

%
% execution context
%

\section{Execution context}

% -)

\begin{frame}
  \frametitle{Definition}

  The \textbf{execution context} or \textbf{processor context} is the
  image of the processor's state at a given moment.

  Execution context includes:

  \begin{itemize}
  \item
    The instruction pointer
  \item
    The stack pointer
  \item
    The flags
  \item
    Processor's running mode (privilege level, interrupt state,
    etc\ldots)
  \item
    The address space
  \item
    Other general purpose or dedicated registers
  \end{itemize}

  \-

  Let's see some real examples of execution contexts. For the moment,
  we omit FPU context.

\end{frame}

% -)

\begin{frame}
  \frametitle{Example of context: MIPS}

  MIPS execution context consists in a few registers:

  \begin{itemize}
  \item
    \$pc
  \item
    \$hi, \$lo, \$r0-\$r31
  \item
    ASid register (in CP0)
  \end{itemize}

\end{frame}

% -)

\begin{frame}
  \frametitle{Example of context: IA-32}

  On IA-32 the processor context includes:

  \begin{itemize}
  \item
    EIP
  \item
    ESP
  \item
    EFLAGS
  \item
    The segment selectors (CS, DS, ES, FS, GS and SS)
  \item
    CR3 (PDBR)
  \item
    General registers: EAX, EBX, ECX, EDX, ESI, EDI, EBP
  \end{itemize}

  \-

  Why GDTR, IDTR or CR0 are not present? These registers ARE NOT part
  of the processor context since the kernel is used to configure them
  correctly. In addition, they cannot be part of the context while
  they are privileged registers not addressable by user.

\end{frame}

% -)

\begin{frame}
  \frametitle{Example of context: SPARC}

  SPARC is an example of original context because of the register
  window technique:

  \begin{itemize}
  \item
    PC
  \item
    PSR (contains flags)
  \item
    G0-G7, Y
  \item
    I0-I7, L0-L7 \& O0-O7 of \emph{all windows used by the current thread}
  \end{itemize}

  \-

  Context switching on such microprocessors becomes more difficult.

\end{frame}

% -)

\begin{frame}
  \frametitle{Context saving \& restoring}

  Context saving is the action of creating in memory an image of the
  CPU context. Context restoring is the way of taking such image and
  setting the CPU context according to it.

  \-

  The difficulty in context saving and restoring is to deal with
  unadressable registers. For example,on IA-32, EIP is not
  addressable. Getting/setting its value is done using tricky code
  (call to fake label, manual stacking of new EIP value followed by a
  RET\ldots)

\end{frame}

% -)

\begin{frame}
  \frametitle{Context switching}

  Context switching is a term involving two operations:

  \begin{itemize}
  \item
    Saving the current context
  \item
    Restoring another context
  \end{itemize}

  \-

  Context switching is the main operation in the kernel's
  scheduler. This part of the code is critical for performances.

\end{frame}

% -)

\begin{frame}[containsverbatim]
  \frametitle{Context switching - direct method}

  The simplest way to switch context is to store the current context
  in a dedicated structure instanciated for each thread and to restore
  the context from such structure.

  \begin{verbatim}
t_context    thread_A, thread_B;

thread_A.r0 <- r0
thread_A.r1 <- r1
...
thread_A.r15 <- r15
...
r0 <- thread_B.r0
r1 <- thread_B.r1
...
r15 <- thread_B.r15
  \end{verbatim}

  \begin{itemize}
  \item[+]
    Space efficient
  \item[-]
    Variable addresses computation: cost time and registers
  \end{itemize}

\end{frame}

% -)

\begin{frame}[containsverbatim]
  \frametitle{Context switching - stack method}

  Another way is to use the stack. Context to save is pushed onto the
  stack, then the stack is switched to the new elected thread's stack,
  which contains the previously saved context.

  \begin{verbatim}
push r0
push r1
...
push r15
sp <- thread_A.sp
pop r15
...
pop r1
pop r0
  \end{verbatim}

  \begin{itemize}
  \item[+]
    Time efficient, no computation, no useless memory accesses
  \item[-]
    Requires one supervisor stack per thread (because of stack
    switches when privilege level changes)
  \end{itemize}

\end{frame}

% -)

\begin{frame}[containsverbatim]
  \frametitle{Context switching - resuming execution}

  In the two previous examples, we omitted to talk about the program
  counter. We saved/restored all registers, except the PC, as it needs
  special code to be handled. Let's imagine the naive solution:

  \begin{verbatim}
    push pc
    ...
    sp <- new_stack
    ...
    pop pc
  \end{verbatim}

  This code does not work. The stacked address for PC is the
  instruction following the \verb|push pc|. Resuming at this address
  is incorrect. One simple and efficient solution is to store the
  address of the PC where to resume execution.

  \begin{verbatim}
    push label_resume
    ...
    sp <- new_stack
    ...
    pop pc
label_resume:
  \end{verbatim}

\end{frame}

% -)

\begin{frame}
  \frametitle{Exercise: simple stack-based context switching on
  ARM7TDMI}

  In this example, you will write some code switching context on an
  ARM7TDMI. The exercise is simplified as possible.

  \begin{itemize}
  \item
    General purpose registers are r0 to r12, plus r14
  \item
    r15 is the PC
  \item
    r13 is the SP
  \item
    Flags will be omitted to simplify the exercise\ldots
  \item
    The function \emph{sched\_next} takes in r0 the address of the
    current stack and returns in r0 the pointer to the stack of the
    newly elected thread
  \end{itemize}

  XXX assembleur

\end{frame}

% -)

\begin{frame}
  \frametitle{Exercise: correction}

  XXX

\end{frame}

% -)

\begin{frame}
  \frametitle{Hardware context switching on IA-32}

  IA-32 provides facilities to switch execution contexts. TSS are
  structures containing all the processor's registers.

  XXX decrire meca exact

\end{frame}

% -)

\begin{frame}
  \frametitle{Hardware vs software context switching}
\end{frame}

% -)

\begin{frame}
  \frametitle{FPU/SIMD context}
\end{frame}

% -)

\begin{frame}
  \frametitle{Naive FPU context switching}
\end{frame}

% -)

\begin{frame}
  \frametitle{Better solution\ldots}
\end{frame}

% -)

\begin{frame}
  \frametitle{Optimal FPU context switching}
\end{frame}

% -)

\begin{frame}
  \frametitle{Multiprocessor case}
\end{frame}

%
% scheduling
%

\section{Scheduling}

% -)

\begin{frame}
  \frametitle{Definition}
\end{frame}

% -)

\begin{frame}
  \frametitle{History}
\end{frame}

% -)

\begin{frame}
  \frametitle{Preemptive vs collaborative execution}
\end{frame}

% -)

\begin{frame}
  \frametitle{Thread state automaton}
\end{frame}

% -)

\begin{frame}
  \frametitle{Algorithm: non-preemptive First-In First-Out}
\end{frame}

% -)

\begin{frame}
  \frametitle{Algorithm: Round Robin}
\end{frame}

% -)

\begin{frame}
  \frametitle{Execution priority}
\end{frame}

% -)

\begin{frame}
  \frametitle{Algorithm: Round Robin with fixed priorities and aging}
\end{frame}

% -)

\begin{frame}
  \frametitle{Dynamic priorization}
\end{frame}

% -)

\begin{frame}
  \frametitle{Algorithm: Multilevel Feedback Queue}
\end{frame}

% -)

\begin{frame}
  \frametitle{Real-time scheduling}
\end{frame}

%
% task management
%

\section{Task management}

% -)

\begin{frame}
  \frametitle{Tasks in the kernel}
\end{frame}

% -)

\begin{frame}
  \frametitle{Process Control Block (PCB)}
\end{frame}

% -)

\begin{frame}
  \frametitle{Example of PCB: XXX}
\end{frame}

% -)

\begin{frame}
  \frametitle{Address space mapping}
\end{frame}

% -)

\begin{frame}
  \frametitle{Example of typical address space: Linux}
\end{frame}

% -)

\begin{frame}
  \frametitle{Task lifecycle}
\end{frame}

% -)

\begin{frame}
  \frametitle{Example of primitives: Windows NT}
\end{frame}

%
% multithreading
%

\section{Multithreading}

%
% multiprocessor systems
%

\section{Multiprocessor systems}

%
% bibliography
%

\section{Bibliography}

\begin{thebibliography}{4}

%  \bibitem{ID}
%    Sample
%    \newblock Sample

\end{thebibliography}

\end{document}
