%%
%% licence       kaneton licence
%%
%% project       kaneton
%%
%% file          /home/buckman/kaneton/view/lectures/kernels/memory-management/kernel-memory-management.tex
%%

%
% template
%

%
% ---------- header -----------------------------------------------------------
%
% project       kaneton
%
% license       kaneton
%
% file          /home/mycure/kaneton/view/template/lecture.tex
%
% created       julien quintard   [wed may 16 18:17:26 2007]
% updated       julien quintard   [sun may 18 23:23:40 2008]
%

%
% class
%

\documentclass[8pt]{beamer}

%
% packages
%

\usepackage{pgf,pgfarrows,pgfnodes,pgfautomata,pgfheaps,pgfshade}
\usepackage[T1]{fontenc}
\usepackage{colortbl}
\usepackage{times}
\usepackage{amsmath,amssymb}
\usepackage{graphics}
\usepackage{graphicx}
\usepackage{color}
\usepackage{xcolor}
\usepackage[english]{babel}
\usepackage{enumerate}
\usepackage[latin1]{inputenc}
\usepackage{verbatim}
\usepackage{aeguill}

%
% style
%

\usepackage{beamerthemesplit}
\setbeamercovered{dynamic}

%
% verbatim stuff
%

\definecolor{verbatimcolor}{rgb}{0.00,0.40,0.00}

\makeatletter

\renewcommand{\verbatim@font}
  {\ttfamily\footnotesize\selectfont}

\def\verbatim@processline{
  {\color{verbatimcolor}\the\verbatim@line}\par
}

\makeatother

%
% -
%

\renewcommand{\-}{\vspace{0.4cm}}

%
% date
%

\date{\today}

%
% logos
%

\pgfdeclareimage[interpolate=true,width=34pt,height=18pt]
                {epita}{\path/logo/epita}
\pgfdeclareimage[interpolate=true,width=49pt,height=18pt]
                {upmc}{\path/logo/upmc}
\pgfdeclareimage[interpolate=true,width=25pt,height=18pt]
                {lse}{\path/logo/lse}

\newcommand{\logos}
  {
    \pgfuseimage{epita}
  }

%
% institute
%

\institute
{
  \inst{1} kaneton microkernel project
}

%
% table of contents at the beginning of each section
%

\AtBeginSection[]
{
  \begin{frame}<beamer>
   \frametitle{Outline}
    \tableofcontents[current]
  \end{frame}
}

%
% table of contents at the beginning of each subsection
%

\AtBeginSubsection[]
{
  \begin{frame}<beamer>
   \frametitle{Outline}
    \tableofcontents[current,currentsubsection]
  \end{frame}
}


%
% title
%

\title{Kernels - Memory management}

%
% authors
%

\author
{
  Matthieu~Bucchianeri and Renaud~Voltz\inst{1}
}

%
% figures
%

%
%\pgfdeclareimage[interpolate=true,width=188pt,height=97pt]
%                {sample}
%		{figures/sample}

%

\pgfdeclareimage[interpolate=true,width=233pt,height=100pt]
                {paging-step1}
		{figures/paging-step1}
\pgfdeclareimage[interpolate=true,width=233pt,height=100pt]
                {paging-step2}
		{figures/paging-step2}
\pgfdeclareimage[interpolate=true,width=233pt,height=100pt]
                {paging-step3}
		{figures/paging-step3}
\pgfdeclareimage[interpolate=true,width=233pt,height=100pt]
                {paging-error-step1}
		{figures/paging-error-step1}
\pgfdeclareimage[interpolate=true,width=233pt,height=100pt]
                {paging-error-step2}
		{figures/paging-error-step2}
\pgfdeclareimage[interpolate=true,width=233pt,height=100pt]
                {paging-miss-step1}
		{figures/paging-miss-step1}
\pgfdeclareimage[interpolate=true,width=233pt,height=100pt]
                {paging-miss-step2}
		{figures/paging-miss-step2}

% tlb

\pgfdeclareimage[interpolate=true,width=187pt,height=100pt]
                {detailed-tlb}
		{figures/detailed-tlb}
\pgfdeclareimage[interpolate=true,width=270pt,height=120pt]
                {mips-tlb}
		{figures/mips-tlb}
\pgfdeclareimage[interpolate=true,width=268pt,height=100pt]
                {direct-mapped}
		{figures/direct-mapped}

% smp

\pgfdeclareimage[interpolate=true,width=210pt,height=180pt]
                {smp-coherency-step1}
		{figures/smp-coherency-step1}
\pgfdeclareimage[interpolate=true,width=210pt,height=180pt]
                {smp-coherency-step2}
		{figures/smp-coherency-step2}
\pgfdeclareimage[interpolate=true,width=210pt,height=180pt]
                {smp-coherency-step3}
		{figures/smp-coherency-step3}


%
% document
%

\begin{document}

%
% title frame
%

\begin{frame}
  \titlepage

  \begin{center}
    \logos
  \end{center}
\end{frame}

%
% outline frame
%

\begin{frame}
  \frametitle{Outline}
  \tableofcontents
\end{frame}

%
% physical memory & virtual memory
%

\section{Physical memory \& virtual memory}


%
% segmentation
%

\section{Segmentation}

%
% paging
%

\section{Paging}

% -)

\begin{frame}
  \frametitle{Translation Lookaside Buffers}

  \begin{itemize}
    \item
    When the CPU encounters a memory access instruction such as
    a load or a store, the source or destination \textbf{virtual
    address} is translated into a \textbf{physical address}
    \item
    This translation is done using a translation cache: the
    Translation Lookaside Buffers (TLB).
    \item
    There are one or more TLB into a single MMU: for example, we
    found two TLB on UltraSPARC architectures: one for instruction
    fetches and the other for data accesses. On IA-32 and MIPS
    architecture, there is only one mixed TLB.
    \item
    These caches are often \textbf{full associative} or high degree
    \textbf{set associative}, for performances reasons. But their size
    is very small:
    \begin{tabular}{|c|c|c|}
    \hline
    Microprocessor & TLB & Entries \\
    \hline
    Pentium (non-MMX) & Instruction & 32 entries, 4-way set associative \\
    \hline
    Pentium (non-MMX) & Data & 64 entries, 4-way set associative \\
    \hline
    Pentium 4 & Instruction & 128 entries, 4-way set associative \\
    \hline
    Pentium 4 & Data & 64 entries, full associative \\
    \hline
    UltraSPARC IIi & Instruction & 64 entries, full associative \\
    \hline
    UltraSPARC IIi & Data & 64 entries, full associative \\
    \hline
    MIPS R8000 & Mixed & 384 entries, 3-way set associative \\
    \hline
    \end{tabular}
  \end{itemize}
\end{frame}

% -)

\begin{frame}
  \frametitle{TLB Organization}

    \begin{itemize}
      \item A TLB is a cache where each entry is made of two parts :

      \begin{center}
        \pgfuseimage{detailed-tlb}
      \end{center}

      \item A cache-hit occurs when:
      \begin{enumerate}
      \item
        V = 1
      \item
        ID = current ASID \textbf{or} G = 1
      \item
        VA $\leq$ requested address $<$ VA + SZ
      \end{enumerate}
    \end{itemize}
\end{frame}

% -)

\begin{frame}
  \frametitle{Scenario of page-hit}

  \begin{center}
    \pgfuseimage{paging-step1}
  \end{center}

\end{frame}

% -)

\begin{frame}
  \frametitle{Scenario of page-hit}

  \begin{center}
    \pgfuseimage{paging-step2}
  \end{center}

\end{frame}

% -)

\begin{frame}
  \frametitle{Scenario of page-hit}

  \begin{center}
    \pgfuseimage{paging-step3}
  \end{center}

\end{frame}

% -)

\begin{frame}
  \frametitle{Scenario of page-hit with error}

  \begin{center}
    \pgfuseimage{paging-error-step1}
  \end{center}

\end{frame}

% -)

\begin{frame}
  \frametitle{Scenario of page-hit with error}

  \begin{center}
    \pgfuseimage{paging-error-step2}
  \end{center}

\end{frame}

% -)

\begin{frame}
  \frametitle{Scenario of page-miss}

  \begin{center}
    \pgfuseimage{paging-miss-step1}
  \end{center}

\end{frame}

% -)

\begin{frame}
  \frametitle{Scenario of page-miss}

  \begin{center}
    \pgfuseimage{paging-miss-step2}
  \end{center}

\end{frame}

% -)

\begin{frame}
  \frametitle{What to do on a TLB-miss ?}

  When an error or TLB-miss occurs, an exception is thrown. There are
  three possible actions:

  \begin{itemize}
  \item
    Stopping the program and reporting the error to the user
    (\emph{Segmentation fault}
  \item
    Modifying the TLB entry to avoid the error (changing rights or
    ownership) and resuming the program
  \item
    Filling the missing entry and resuming the program
  \end{itemize}

  \-

  In the next slides, we will focus on the third point. This is not an
  error, remember that the TLB is a cache, meaning that it cannot host
  all the required entries.

\end{frame}

% -)

\begin{frame}
  \frametitle{Manual TLB entry replacement}

  Processors with very basic MMU let the kernel fill the TLB.

  \-

  On TLB-miss, an exception is thrown and the involved virtual address
  is passed through a register. The exception handler must look in its
  own structures an perform the TLB entry replacement by hand.

  \-

  A TLB replacement involves:

  \begin{itemize}
  \item
    Selecting an unlikely to be used entry to remove
  \item
    Building and adding the new entry
  \end{itemize}

\end{frame}

% -)

\begin{frame}
  \frametitle{Choosing a good entry to replace}

  Sometimes, the TLB cache is full, meaning that all of the few
  entries are used. Adding a new entry forces to remove an existing
  one.

  \-

  Some TLB implements a LRU (\emph{Least Recently Used}) algorithm and
  put in a register the number of the entry most unlikely to be used.

  \-

  Otherwise, the kernel can implement its own aging algorithms using
  the TLB entries facilities (accessed bit, user free
  bits\ldots). Another easier solution is to replace a random entry,
  this is fast but not really efficient.

\end{frame}

% -)

\begin{frame}
  \frametitle{MIPS example}

  \begin{center}
    \pgfuseimage{mips-tlb}
  \end{center}

  MIPS TLB entries are mapping two pages each, depending on the last
  bit of the virtual address (odd or even pages). Many page size are
  available, from 4 Kb to 256 Mb.

\end{frame}

% -)

\begin{frame}
  \frametitle{MIPS example}

  MMU registers list:

  \begin{itemize}
  \item
    ASID register: specified the current ASID value
  \item
    Index register: index of the TLB to use (for modifying or removing)
  \item
    Random register: random index of a TLB entry that can be replaced
    (the first dozen entries are protected)
  \item
    Tag register (called EntryHi): used to set the Tag word of an entry
  \item
    Entry registers (called EntryLo): used to set the Entry word of an entry
  \item
    PageMask register: used to specify the size of the page for the
    TLB entry
  \item
    Bad virtual address (BadVAddr) register: reports the address that
    caused page-miss of page-error
\end{itemize}

\end{frame}

% -)

\begin{frame}
  \frametitle{Exercise: tree-based replacement algorithm}

  Write a pseudo-algorithm for replacing TLB entries with a binary
  page tree.

  \-

  Your algorithm, given a virtual address, must fill the TLB with the
  correct entry or report an error if no translation can be done.

  \-

  Each bit of a virtual address correspond to a direction to take in
  the tree. A bit set tells to take the right child and a bit reset
  the left child.

  \-

  The leaf contains the translation. The page size is determined by
  the depth of a leaf.

\end{frame}

% -)

\begin{frame}
  \frametitle{Exercise: correction}

  \begin{enumerate}
  \item
    Pagesz $\leftarrow$ 31
  \item
    Get BadVAddr
  \item
    For each bit from left to right
    \begin{enumerate}
    \item
      Goto appropriate child
    \item
      In non-existent child, throw error
    \item
      On a leaf:
      \begin{itemize}
      \item
        Index $\leftarrow$ Random
      \item
        EntryHi $\leftarrow$ Build tag (Vaddr$_{31-pagesz}$, ASID)
      \item
        EntryLo $\leftarrow$ Build entry (Paddr, Writable)
      \item
        Pagemask $\leftarrow$ $2^{pagesz}-1$
      \end{itemize}
    \item
      Otherwise, Pagesz $\leftarrow$ Pagesz - 1
    \end{enumerate}
  \end{enumerate}

\end{frame}

% -)

\begin{frame}
  \frametitle{Semi-automatic TLB replacement}

  Some kind of MMU offers a semi-automatic TLB replacement mecanism.

  \-

  The principle of semi-automatic TLB management consist in a software
  cache in main memory. When no entry is found in the MMU's TLB, a
  preprogrammed replacement routine is called and search the software
  cache.

  \-

  If no entry is found in the software table, then an exception is
  raised to the kernel.

\end{frame}

% -)

\begin{frame}
  \frametitle{UltraSPARC example}

  On UltraSPARC, the software cache is called TSB (\emph{Translation
  Storage Buffer}) and is a direct-mapped cache of TTE
  (\emph{Translation Table Entry}). Searching the TSB involves only
  one memory access, this is quite acceptable.

  \-

  \begin{center}
    \pgfuseimage{direct-mapped}
  \end{center}

  \-

  The MMU is given only the base and the size of the TSB.

\end{frame}

% -)

\begin{frame}
  \frametitle{Full-automatic TLB replacement}



\end{frame}

% -)

\begin{frame}
  \frametitle{IA-32 example}

\end{frame}

% -)

\begin{frame}
  \frametitle{Exercise: contiguous mapping with page-directory \&
  page-tables}

  \begin{itemize}
  \item
    pde\_start $\leftarrow$ PDE\_ENTRY(Vaddr) \\
    pde\_end $\leftarrow$ PDE\_ENTRY(Vaddr + Size) \\
    pte\_start $\leftarrow$ PDE\_ENTRY(Vaddr) \\
    pte\_end $\leftarrow$ PDE\_ENTRY(Vaddr + Size)
  \item
    i $\leftarrow$ pde\_start to pde\_end
    \begin{itemize}
    \item
      if pd\_get\_table(i) not exist then pd\_add\_table(i, new table)
    \item
      table $\leftarrow$ pd\_get\_table(i)
    \item
      j $\leftarrow$ (pte\_start if i == pde\_start, 0 otherwise) to (pte\_end if i == pde\_end, 1024 otherwise)
      \begin{itemize}
      \item
        pt\_add\_page(table, j, translation)
      \end{itemize}
    \end{itemize}
  \end{itemize}

\end{frame}

% -)

\begin{frame}
  \frametitle{Cycles in virtual memory management}

  There are a few common cycles when dealing with virtual memory.

  \begin{itemize}
  \item
    When a TLB-miss handler is called, it may generate another
    TLB-miss when accessing the kernel's structures
  \item
    When filling page-directory \& page-tables to map a region of
    memory, the page directory and/or page-tables needs to be mapped
  \end{itemize}

  \-

  Sometimes, the CPU provides facilities to avoid this cycle: a region
  in virtual memory called a bypass region exists and permits to
  bypass the MMU (this means that putting the memory management
  structures in this area will never generate TLB-miss).

  \-

  Others CPU do not. So we have to show imagination.

\end{frame}

% -)

\begin{frame}
  \frametitle{Mirroring on IA-32}

\end{frame}

% -)

\begin{frame}
  \frametitle{TLB coherency}

\end{frame}

% -)

\begin{frame}
  \frametitle{TLB incoherency example}

\end{frame}

% -)

\begin{frame}
  \frametitle{Coherency on multiprocessor platform}

  \begin{center}
    \pgfuseimage{smp-coherency-step1}
  \end{center}

\end{frame}

% -)

\begin{frame}
  \frametitle{Coherency on multiprocessor platform}

  \begin{center}
    \pgfuseimage{smp-coherency-step2}
  \end{center}

\end{frame}

% -)

\begin{frame}
  \frametitle{Coherency on multiprocessor platform}

  \begin{center}
    \pgfuseimage{smp-coherency-step3}
  \end{center}

\end{frame}

% -)

\begin{frame}
  \frametitle{Unmapping}

  Unmapping memory pages is very easy:

  \begin{itemize}
  \item
    Remove the translation into the kernel's or MMU's structures
  \item
    Do not forget to invalidate the corresponding entry in TLB if
    needed
  \end{itemize}

\end{frame}

%
% allocation algorithms
%

\section{Allocation algorithms}

% -)

\begin{frame}
  \frametitle{Physical memory allocation}

\end{frame}

% ... XXX user, copy on write, stack expansion, page-on-demand etc..

%
% swapping
%

\section{Swapping}

%
% bibliography
%

\section{Bibliography}

\begin{thebibliography}{4}

%  \bibitem{ID}
%    Sample
%    \newblock Sample

\end{thebibliography}

\end{document}
