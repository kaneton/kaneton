%%
%% licence       kaneton licence
%%
%% project       kaneton
%%
%% file          /home/buckman/kaneton/view/lectures/kernels/memory-management/kernel-memory-management.tex
%%

%
% template
%

%%
%% copyright     (c) julien quintard
%%
%% project       kaneton
%%
%% file          /home/mycure/kaneton/view/templates/lecture.tex
%%
%% created       julien quintard   [sat nov 19 17:13:03 2005]
%% updated       julien quintard   [fri dec  2 22:36:34 2005]
%%

%
% class
%

\documentclass[8pt]{beamer}

%
% packages
%

\usepackage{pgf,pgfarrows,pgfnodes,pgfautomata,pgfheaps,pgfshade}
\usepackage{colortbl}
\usepackage{times}
\usepackage{amsmath,amssymb}
\usepackage{graphics}
\usepackage{graphicx}
\usepackage{color}
\usepackage{xcolor}
\usepackage[english]{babel}
\usepackage{enumerate}
\usepackage[latin1]{inputenc}

%
% style
%

\usepackage{beamerthemesplit}
\setbeamercovered{dynamic}

%
% verbatim font
%

\definecolor{verbatimcolor}{rgb}{0,0.4,0}

\makeatletter
\renewcommand{\verbatim@font}
  {\ttfamily\footnotesize\color{verbatimcolor}\selectfont}
\makeatother

%
% new line
%

\newcommand{\nl}[0]{\vspace{0.4cm}}

%
% date
%

\date{\today}

%
% logos
%

\pgfdeclareimage[interpolate=true,width=34pt,height=18pt]
                {epita}{../../logos/epita}
\pgfdeclareimage[interpolate=true,width=49pt,height=18pt]
                {upmc}{../../logos/upmc}
\pgfdeclareimage[interpolate=true,width=25pt,height=18pt]
                {lse}{../../logos/lse}

\newcommand{\logos}
  {
    \pgfuseimage{epita}
  }

%
% institute
%

\institute
{
  \inst{1} kaneton microkernel project
}

%
% table of contents at the beginning of each section
%

\AtBeginSection[]
{
  \begin{frame}<beamer>
   \frametitle{Outline}
    \tableofcontents[current]
  \end{frame}
}

%
% table of contents at the beginning of each subsection
%

\AtBeginSubsection[]
{
  \begin{frame}<beamer>
   \frametitle{Outline}
    \tableofcontents[current,currentsubsection]
  \end{frame}
}


%
% title
%

\title{Kernels - Memory management}

%
% authors
%

\author
{
  Matthieu~Bucchianeri and Renaud~Voltz\inst{1}
}

%
% figures
%

%
%\pgfdeclareimage[interpolate=true,width=188pt,height=97pt]
%                {sample}
%		{figures/sample}

%

\pgfdeclareimage[interpolate=true,width=233pt,height=100pt]
                {paging-step1}
		{figures/paging-step1}
\pgfdeclareimage[interpolate=true,width=233pt,height=100pt]
                {paging-step2}
		{figures/paging-step2}
\pgfdeclareimage[interpolate=true,width=233pt,height=100pt]
                {paging-step3}
		{figures/paging-step3}

% tlb

\pgfdeclareimage[interpolate=true,width=187pt,height=100pt]
                {detailed-tlb}
		{figures/detailed-tlb}

%
% document
%

\begin{document}

%
% title frame
%

\begin{frame}
  \titlepage

  \begin{center}
    \logos
  \end{center}
\end{frame}

%
% outline frame
%

\begin{frame}
  \frametitle{Outline}
  \tableofcontents
\end{frame}

%
% physical memory & virtual memory
%

\section{Physical memory \& virtual memory}


%
% segmentation
%

\section{Segmentation}

%
% paging
%

\section{Paging}

% -)

\begin{frame}
  \frametitle{Translation Lookaside Buffers}

  \begin{itemize}
    \item
    When the CPU encounters a memory access instruction such as
    a load or a store, the source or destination \textbf{virtual
    address} is translated into a \textbf{physical address}
    \item
    This translation is done using a translation cache: the
    Translation Lookaside Buffers (TLB).
    \item
    There are one or more TLB into a single MMU: for example, we
    found two TLB on UltraSPARC architectures: one for instruction
    fetches and the other for data accesses. On IA-32 and MIPS
    architecture, there is only one mixed TLB.
    \item
    These caches are often \textbf{full associative} or high degree
    \textbf{set associative}, for performances reasons. But their size
    is very small:
    \begin{tabular}{|c|c|c|}
    \hline
    Microprocessor & TLB & Entries \\
    \hline
    Pentium (non-MMX) & Instruction & 32 entries, 4-way set associative \\
    \hline
    Pentium (non-MMX) & Data & 64 entries, 4-way set associative \\
    \hline
    Pentium 4 & Instruction & 128 entries, 4-way set associative \\
    \hline
    Pentium 4 & Data & 64 entries, full associative \\
    \hline
    UltraSPARC IIi & Instruction & 64 entries, full associative \\
    \hline
    UltraSPARC IIi & Data & 64 entries, full associative \\
    \hline
    MIPS R8000 & Mixed & 384 entries, 3-way set associative \\
    \hline
    \end{tabular}
  \end{itemize}
\end{frame}

% -)

\begin{frame}
  \frametitle{TLB Organization}

    \begin{itemize}
      \item A TLB is a cache where each entry is made of two parts :

      \begin{center}
        \pgfuseimage{detailed-tlb}
      \end{center}

      \item A cache-hit occurs when:
      \begin{enumerate}
      \item
        V = 1
      \item
        ID = current ASID \textbf{or} G = 1
      \item
        VA $\leq$ requested address $<$ VA + SZ
      \end{enumerate}
    \end{itemize}
\end{frame}

% -)

\begin{frame}
  \frametitle{Scenario of page-hit}

  \begin{center}
    \pgfuseimage{paging-step1}
  \end{center}

\end{frame}

% -)

\begin{frame}
  \frametitle{Scenario of page-hit}

  \begin{center}
    \pgfuseimage{paging-step2}
  \end{center}

\end{frame}

% -)

\begin{frame}
  \frametitle{Scenario of page-hit}

  \begin{center}
    \pgfuseimage{paging-step3}
  \end{center}

\end{frame}

% -)

\begin{frame}
  \frametitle{Scenario of page-hit with error}



\end{frame}

% -)

\begin{frame}
  \frametitle{Scenario of page-miss}

\end{frame}

% -)

\begin{frame}
  \frametitle{What to do on TLB-miss ?}

\end{frame}

% -)

\begin{frame}
  \frametitle{Manual TLB entry replacement}

\end{frame}

% -)

\begin{frame}
  \frametitle{MIPS example}

\end{frame}

% -)

\begin{frame}
  \frametitle{Exercise: tree-based replacement algorithm}

\end{frame}

% -)

\begin{frame}
  \frametitle{Semi-automatic TLB replacement}

\end{frame}

% -)

\begin{frame}
  \frametitle{UltraSPARC example}

\end{frame}

% -)

\begin{frame}
  \frametitle{Full-automatic TLB replacement}

\end{frame}

% -)

\begin{frame}
  \frametitle{IA-32 example}

\end{frame}

% -)

\begin{frame}
  \frametitle{Exercise: contiguous mapping with page-directory \&
  page-tables}

\end{frame}

% -)

\begin{frame}
  \frametitle{TLB coherency}

\end{frame}

% -)

\begin{frame}
  \frametitle{TLB incoherency example}

\end{frame}

% -)

\begin{frame}
  \frametitle{Unmapping}

\end{frame}

%
% allocation algorithms
%

\section{Allocation algorithms}

% -)

\begin{frame}
  \frametitle{Physical memory allocation}

\end{frame}

% ... XXX user etc..

%
% swapping
%

\section{Swapping}

%
% bibliography
%

\section{Bibliography}

\begin{thebibliography}{4}

%  \bibitem{ID}
%    Sample
%    \newblock Sample

\end{thebibliography}

\end{document}
