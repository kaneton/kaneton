\pgfdeclareimage[interpolate=true,width=180pt]
                {io_mapped}
                {figures/io_mapped}

\pgfdeclareimage[interpolate=true,width=200pt]
                {io_port}
                {figures/io_port}

\pgfdeclareimage[interpolate=true,width=320pt]
                {io_hybrid}
                {figures/io_hybrid}

%
%
%

\begin{frame}
  \frametitle{I/O device}

  An I/O device is generally composed of a control register and a data buffer.

  \nl

  Example of I/O devices: Hard disk, keyboard, Interrupt Controller, serial
  port\ldots

  \nl

  To address the I/O devices, the microprocessor can use one of the 3
  architectures:

  \begin{itemize}
    \item memory-mapped I/O\\
      The hardware device's memory is mapped in RAM.
    \item I/O ports\\
    \item hybrid\\
  \end{itemize}

\end{frame}

%
%
%

\begin{frame}
  \frametitle{Memory-mapped I/O}

  \begin{center}
    \pgfuseimage{io_mapped}
  \end{center}

  {\bf Advantages}\\
  \begin{itemize}
    \item restricted instruction set
    \item No special protection mechanism is needed, since protection is
      already implemented at the memory level.
  \end{itemize}

  {\bf Disadvantages}\\
  \begin{itemize}
    \item Caching. Solution: disable caching (for a page)
  \end{itemize}

\end{frame}

%
%
%

\begin{frame}
  \frametitle{I/O ports}

  \begin{center}
    \pgfuseimage{io_port}
  \end{center}

  {\bf Advantages}\\
  \begin{itemize}
    \item Can accept a large number of devices without affecting main memory.
  \end{itemize}

  {\bf Disadvantages}\\
  \begin{itemize}
    \item Enlarge the instruction set with special read and write instructions.
  \end{itemize}

\end{frame}

%
%
%

\begin{frame}
  \frametitle{Hybrid I/O}

  \begin{center}
    \pgfuseimage{io_hybrid}
  \end{center}

\end{frame}

%
%
%

\begin{frame}
  \frametitle{I/O performing}

  I/O are asynchronous:

  \begin{itemize}
    \item in emission: a hardware device may send data at any time
    \item in reception: the kernel cannot find out when the device is ready to
      receive data
  \end{itemize}

  \nl

  There are 2 methods to handle asynchronous events:

  \begin{itemize}
    \item Programmed I/O (polling)
    \item Interrupt-based I/O
  \end{itemize}

\end{frame}

%
%
%

\begin{frame}
  \frametitle{Programmed I/O}



\end{frame}


%
%
%

\begin{frame}
  \frametitle{Interrupt-based I/O}



\end{frame}

%
%
%

\begin{frame}
  \frametitle{Direct Memory Access (DMA)}

  To support DMA, the hardware must have a DMA controller. This controller
  provides the following registers:

  \begin{itemize}
    \item The address register determiness the transfert's memory address.
    \item count register specifies the transfer size.
    \item control register contains the I/O port and the transfert direction.
  \end{itemize}

  \nl

  {\bf Advantages}\\
  \begin{itemize}
    \item DMA minimize the microprocessor load
    \item decrease the interrupt rate
  \end{itemize}

  {\bf Disadvantages}\\
  \begin{itemize}
    \item The DMA controller is much slower than the CPU. Also, when the CPU,
      it could assume the transfert faster.
    \item 
  \end{itemize}

\end{frame}
