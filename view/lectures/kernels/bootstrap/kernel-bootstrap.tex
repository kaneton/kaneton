%%
%% licence       kaneton licence
%%
%% project       kaneton
%%
%% file          /home/buckman/kaneton/view/lectures/kernels/bootstrap/kernel-bootstrap.tex
%%

%
% template
%

%%
%% copyright     (c) julien quintard
%%
%% project       kaneton
%%
%% file          /home/mycure/kaneton/view/templates/lecture.tex
%%
%% created       julien quintard   [sat nov 19 17:13:03 2005]
%% updated       julien quintard   [fri dec  2 22:36:34 2005]
%%

%
% class
%

\documentclass[8pt]{beamer}

%
% packages
%

\usepackage{pgf,pgfarrows,pgfnodes,pgfautomata,pgfheaps,pgfshade}
\usepackage{colortbl}
\usepackage{times}
\usepackage{amsmath,amssymb}
\usepackage{graphics}
\usepackage{graphicx}
\usepackage{color}
\usepackage{xcolor}
\usepackage[english]{babel}
\usepackage{enumerate}
\usepackage[latin1]{inputenc}

%
% style
%

\usepackage{beamerthemesplit}
\setbeamercovered{dynamic}

%
% verbatim font
%

\definecolor{verbatimcolor}{rgb}{0,0.4,0}

\makeatletter
\renewcommand{\verbatim@font}
  {\ttfamily\footnotesize\color{verbatimcolor}\selectfont}
\makeatother

%
% new line
%

\newcommand{\nl}[0]{\vspace{0.4cm}}

%
% date
%

\date{\today}

%
% logos
%

\pgfdeclareimage[interpolate=true,width=34pt,height=18pt]
                {epita}{../../logos/epita}
\pgfdeclareimage[interpolate=true,width=49pt,height=18pt]
                {upmc}{../../logos/upmc}
\pgfdeclareimage[interpolate=true,width=25pt,height=18pt]
                {lse}{../../logos/lse}

\newcommand{\logos}
  {
    \pgfuseimage{epita}
  }

%
% institute
%

\institute
{
  \inst{1} kaneton microkernel project
}

%
% table of contents at the beginning of each section
%

\AtBeginSection[]
{
  \begin{frame}<beamer>
   \frametitle{Outline}
    \tableofcontents[current]
  \end{frame}
}

%
% table of contents at the beginning of each subsection
%

\AtBeginSubsection[]
{
  \begin{frame}<beamer>
   \frametitle{Outline}
    \tableofcontents[current,currentsubsection]
  \end{frame}
}


%
% title
%

\title{Kernels - Bootstrap}

%
% authors
%

\author
{
  Matthieu~Bucchianeri and Renaud~Voltz\inst{1}
}

%
% figures
%

%
%\pgfdeclareimage[interpolate=true,width=188pt,height=97pt]
%                {sample}
%		{figures/sample}

% openboot

\pgfdeclareimage[interpolate=true,width=232pt,height=140pt]
                {devtree}
		{figures/devtree}
\pgfdeclareimage[interpolate=true,width=222pt,height=180pt]
                {ofw}
		{figures/ofw}

%
% document
%

\begin{document}

%
% title frame
%

\begin{frame}
  \titlepage

  \begin{center}
    \logos
  \end{center}
\end{frame}

%
% outline frame
%

\begin{frame}
  \frametitle{Outline}
  \tableofcontents
\end{frame}

%
% sun's openboot
%

\begin{frame}
  \frametitle{OpenBoot (Sun)}

  OpenBoot (or OpenFirmware) is the software embedded in all Sun's
  SPARC based stations.

  \-

  OpenBoot offers many services (as the BIOS does) necessary to the
  bootup phase. For example:

  \begin{itemize}
  \item
    Device-tree exploration (\emph{sibling}, \emph{child},
    \emph{getprop}\ldots)
  \item
    Device I/O (\emph{open}, \emph{read}, \emph{write}\ldots)
  \item
    MMU (\emph{map\_phys}, \emph{unmap\_phys}, \emph{itlb\_load},
    \emph{dtlb\_load}\ldots)
  \item
    Environment (boot path, boot device, boot arguments\ldots)
  \item
    Time (\emph{milliseconds})
  \end{itemize}

\end{frame}

% -)

\begin{frame}
  \frametitle{OpenBoot device-tree example}

  \begin{center}
  \pgfuseimage{devtree}
  \end{center}

\end{frame}

% -)

\begin{frame}
  \frametitle{OpenBoot call}

  Calling OpenBoot is as simple as jumping to the so-called
  \emph{firmware entry-point} (given in register \%g7 at boot time)
  with a structure in first argument (register \%o0). This structure is
  as follow:

  \begin{itemize}
  \item
    A pointer to a string indicating the name of the function to call.
  \item
    The number of arguments \emph{N}
  \item
    The number of return values \emph{M}
  \item
    \emph{N} 64-bit arguments
  \item
    \emph{M} 64-bit slots for results
  \end{itemize}

\end{frame}

% -)

\begin{frame}
  \frametitle{OpenBoot example}

  \begin{center}
  \pgfuseimage{ofw}
  \end{center}

\end{frame}

%
% sgi's arc
%

\begin{frame}
  \frametitle{ARCS (SGI)}

\end{frame}

%
% bibliography
%

\section{Bibliography}

\begin{thebibliography}{4}

%  \bibitem{ID}
%    Sample
%    \newblock Sample

\end{thebibliography}

\end{document}
