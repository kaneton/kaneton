%%
%% licence       kaneton licence
%%
%% project       kaneton
%%
%% file          /home/buckman/kaneton/view/lectures/seminar_2006/seminar.tex
%%
%% created       julien quintard   [fri dec  2 07:53:59 2005]
%% updated       matthieu bucchianeri   [sun dec 10 00:47:04 2006]
%%

%
% template
%

%
% ---------- header -----------------------------------------------------------
%
% project       kaneton
%
% license       kaneton
%
% file          /home/mycure/kaneton/view/template/lecture.tex
%
% created       julien quintard   [wed may 16 18:17:26 2007]
% updated       julien quintard   [sun may 18 23:23:40 2008]
%

%
% class
%

\documentclass[8pt]{beamer}

%
% packages
%

\usepackage{pgf,pgfarrows,pgfnodes,pgfautomata,pgfheaps,pgfshade}
\usepackage[T1]{fontenc}
\usepackage{colortbl}
\usepackage{times}
\usepackage{amsmath,amssymb}
\usepackage{graphics}
\usepackage{graphicx}
\usepackage{color}
\usepackage{xcolor}
\usepackage[english]{babel}
\usepackage{enumerate}
\usepackage[latin1]{inputenc}
\usepackage{verbatim}
\usepackage{aeguill}

%
% style
%

\usepackage{beamerthemesplit}
\setbeamercovered{dynamic}

%
% verbatim stuff
%

\definecolor{verbatimcolor}{rgb}{0.00,0.40,0.00}

\makeatletter

\renewcommand{\verbatim@font}
  {\ttfamily\footnotesize\selectfont}

\def\verbatim@processline{
  {\color{verbatimcolor}\the\verbatim@line}\par
}

\makeatother

%
% -
%

\renewcommand{\-}{\vspace{0.4cm}}

%
% date
%

\date{\today}

%
% logos
%

\pgfdeclareimage[interpolate=true,width=34pt,height=18pt]
                {epita}{\path/logo/epita}
\pgfdeclareimage[interpolate=true,width=49pt,height=18pt]
                {upmc}{\path/logo/upmc}
\pgfdeclareimage[interpolate=true,width=25pt,height=18pt]
                {lse}{\path/logo/lse}

\newcommand{\logos}
  {
    \pgfuseimage{epita}
  }

%
% institute
%

\institute
{
  \inst{1} kaneton microkernel project
}

%
% table of contents at the beginning of each section
%

\AtBeginSection[]
{
  \begin{frame}<beamer>
   \frametitle{Outline}
    \tableofcontents[current]
  \end{frame}
}

%
% table of contents at the beginning of each subsection
%

\AtBeginSubsection[]
{
  \begin{frame}<beamer>
   \frametitle{Outline}
    \tableofcontents[current,currentsubsection]
  \end{frame}
}


%
% title
%

\title{Pr\'{e}sentation de kaneton\\http://www.kaneton.org}

%
% authors
%

\author
{
  Renaud~Voltz,
  Matthieu~Bucchianeri,
  Julien~Quintard
}

%
% figures
%

% 10 frames for a microkernel example

\pgfdeclareimage[interpolate=true,width=290pt,height=165pt]
                {microkernel_example_01}
		{figures/microkernel_example_01}
\pgfdeclareimage[interpolate=true,width=290pt,height=165pt]
                {microkernel_example_02}
		{figures/microkernel_example_02}
\pgfdeclareimage[interpolate=true,width=290pt,height=165pt]
                {microkernel_example_03}
		{figures/microkernel_example_03}
\pgfdeclareimage[interpolate=true,width=290pt,height=165pt]
                {microkernel_example_04}
		{figures/microkernel_example_04}
\pgfdeclareimage[interpolate=true,width=290pt,height=165pt]
                {microkernel_example_05}
		{figures/microkernel_example_05}
\pgfdeclareimage[interpolate=true,width=290pt,height=165pt]
                {microkernel_example_06}
		{figures/microkernel_example_06}
\pgfdeclareimage[interpolate=true,width=290pt,height=165pt]
                {microkernel_example_07}
		{figures/microkernel_example_07}
\pgfdeclareimage[interpolate=true,width=290pt,height=165pt]
                {microkernel_example_08}
		{figures/microkernel_example_08}
\pgfdeclareimage[interpolate=true,width=290pt,height=165pt]
                {microkernel_example_09}
		{figures/microkernel_example_09}
\pgfdeclareimage[interpolate=true,width=290pt,height=165pt]
                {microkernel_example_10}
		{figures/microkernel_example_10}

% memory management

\pgfdeclareimage[interpolate=true,width=179pt,height=190pt]
                {vmem}
		{figures/vmem}
\pgfdeclareimage[interpolate=true,width=137pt,height=190pt]
                {vmem_overlap}
		{figures/vmem-overlap}
\pgfdeclareimage[interpolate=true,width=137pt,height=190pt]
                {vmem_contiguous}
		{figures/vmem-contiguous}
\pgfdeclareimage[interpolate=true,width=137pt,height=190pt]
                {vmem_sharing}
		{figures/vmem-sharing}

% process scheduling

\pgfdeclareimage[interpolate=true,width=150pt,height=190pt]
                {mfq1}
		{figures/sched1}
\pgfdeclareimage[interpolate=true,width=150pt,height=190pt]
                {mfq2}
		{figures/sched2}
\pgfdeclareimage[interpolate=true,width=150pt,height=190pt]
                {mfq3}
		{figures/sched3}
\pgfdeclareimage[interpolate=true,width=150pt,height=190pt]
                {mfq4}
		{figures/sched4}
\pgfdeclareimage[interpolate=true,width=150pt,height=190pt]
                {mfq5}
		{figures/sched5}
\pgfdeclareimage[interpolate=true,width=150pt,height=190pt]
                {mfq6}
		{figures/sched6}
\pgfdeclareimage[interpolate=true,width=150pt,height=190pt]
                {mfq7}
		{figures/sched7}

%
% document
%

\begin{document}

%
% title frame
%

\begin{frame}
  \titlepage

  \begin{center}
    \logos
  \end{center}
\end{frame}

%
% outline frame
%

\begin{frame}
  \frametitle{Contenu}
  \tableofcontents
\end{frame}

%
% overview
%

\section{Pr\'{e}sentation}

% 1)

\begin{frame}
  \frametitle{Les gens qui travaillent sur kaneton}

  \textbf{Design}

  \begin{itemize}
    \item
      Julien Quintard
    \item
      Jean-Pascal Billaud
  \end{itemize}

  \nl

  \textbf{Implementation}

  \begin{itemize}
    \item
      Renaud Voltz
    \item
      Matthieu Bucchianeri
    \item
      Enguerrand Raymond
  \end{itemize}

  \nl

  \textbf{Autres gens ayant contribu\'{e}s significativement}

  \begin{itemize}
    \item
      C\'{e}dric Aubouy
  \end{itemize}
\end{frame}

% 2)

\begin{frame}
  \frametitle{Description}

  kaneton est un micro-noyau \`{a} but p\'{e}dagogique
  d\'{e}velopp\'{e} par le LSE.

  \-

  \begin{itemize}
  \item
    Un design compr\'{e}hensible
  \item
    Un code clair et comment\'{e}
  \item
    Facilement maintenable et portable
  \item
    Destin\'{e} au distribu\'{e}
  \end{itemize}

\end{frame}

% 3)

\begin{frame}

  kaneton est aussi un projet en sp\'{e}cialisation SRS et GISTR.\\

  Deux cours :

  \-

  \begin{itemize}
  \item
    Noyaux et syst\`{e}mes d'exploitation
  \item
    Architecture des microprocesseurs
  \end{itemize}

  \-

  Cinq projets en SRS :

  \-

  \begin{itemize}
  \item
    K0 : le bootstrap
  \item
    K1 : la m\'{e}moire
  \item
    K2 : les \'{e}v\'{e}nements
  \item
    K3 : les process et l'ordonnancement
  \item
    K4 : les IPC
  \end{itemize}

  \-

  Deux projets en GISTR :

  \-

  \begin{itemize}
  \item
    KG0 : les \'{e}v\'{e}nements et les I/O
  \item
    KG1 : les process et l'ordonnancement
  \end{itemize}
\end{frame}

% 4)

\begin{frame}
  \frametitle{La pr\'{e}sentation de ce soir}

  \begin{itemize}
  \item
    Introduit le design de kaneton
  \item
    Pr\'{e}sente ce qui \`{a} \'{e}t\'{e} cod\'{e}
  \item
    Liste de ce qu'il reste \`{a} faire
  \end{itemize}
\end{frame}

%
% microkernels
%

\section{Les micro-noyaux}

% 5)

\begin{frame}
  \frametitle{Les micro-noyaux}

  \begin{itemize}
  \item
    $\leftrightarrow$ noyaux monolithiques
  \item
    $\leftrightarrow$ exo-noyaux
  \item
    Conception modulaire : les services
  \item
    Fonctionnement en \emph{userland}
  \item
    Communication inter-processus (IPC)
  \end{itemize}

\end{frame}

% 6)

\begin{frame}
  \frametitle{Microkernel example}

  \begin{center}
    \pgfuseimage{microkernel_example_01}
  \end{center}
\end{frame}

% 7)

\begin{frame}
  \frametitle{Microkernel example}

  \begin{center}
    \pgfuseimage{microkernel_example_02}
  \end{center}
\end{frame}

% 8)

\begin{frame}
  \frametitle{Microkernel example}

  \begin{center}
    \pgfuseimage{microkernel_example_03}
  \end{center}
\end{frame}

% 9)

\begin{frame}
  \frametitle{Microkernel example}

  \begin{center}
    \pgfuseimage{microkernel_example_04}
  \end{center}
\end{frame}

% 10)

\begin{frame}
  \frametitle{Microkernel example}

  \begin{center}
    \pgfuseimage{microkernel_example_05}
  \end{center}
\end{frame}

% 11)

\begin{frame}
  \frametitle{Microkernel example}

  \begin{center}
    \pgfuseimage{microkernel_example_06}
  \end{center}
\end{frame}

% 12)

\begin{frame}
  \frametitle{Microkernel example}

  \begin{center}
    \pgfuseimage{microkernel_example_07}
  \end{center}
\end{frame}

% 13)

\begin{frame}
  \frametitle{Microkernel example}

  \begin{center}
    \pgfuseimage{microkernel_example_08}
  \end{center}
\end{frame}

% 14)

\begin{frame}
  \frametitle{Microkernel example}

  \begin{center}
    \pgfuseimage{microkernel_example_09}
  \end{center}
\end{frame}

% 15)

\begin{frame}
  \frametitle{Microkernel example}

  \begin{center}
    \pgfuseimage{microkernel_example_10}
  \end{center}
\end{frame}

% 16)

%
% kaneton design
%

\section{Le design de kaneton}

% 17)

\begin{frame}
  \frametitle{Fonctions remplies par kaneton}

\end{frame}

\begin{frame}
  \frametitle{Le \emph{set} manager}

\end{frame}

% -)

\begin{frame}
  \frametitle{Gestion de la m\'{e}moire}

  kaneton offre des primitives pour g\'{e}rer la m\'{e}moire physique et la
  m\'{e}moire virtuelle \`{a} travers le segment manager, le region manager et
  enfin l'as manager.

\end{frame}

% -)

\begin{frame}
  \frametitle{as, segment et region}

  \begin{center}
    \pgfuseimage{vmem}
  \end{center}
\end{frame}

% -)

\begin{frame}
  \frametitle{Le mapping}

  \begin{center}
    \pgfuseimage{vmem_overlap}
  \end{center}
\end{frame}

% -)

\begin{frame}
  \frametitle{L'allocation non contig\"{u}e}

  \begin{center}
    \pgfuseimage{vmem_contiguous}
  \end{center}
\end{frame}

% -)

\begin{frame}
  \frametitle{Le partage}

  \begin{center}
    \pgfuseimage{vmem_sharing}
  \end{center}
\end{frame}

% -)

\begin{frame}
  \frametitle{Pourquoi une telle gestion de la m\'{e}moire ?}

  \begin{itemize}
  \item
    Proteger les donn\'{e}es entres les processus\ldots
  \item
    \ldots{} tout en permettant le partage de certaines zones
  \item
    Optimiser l'allocation de la m\'{e}moire physique
  \item
    Permettre l'utilisation de programmes contenant des adresses absolues
  \item
    Avoir \`{a} disposition plus de m\'{e}moire qu'il n'y en a (\emph{swapping})
  \end{itemize}

\end{frame}

% -)

\begin{frame}
  \frametitle{Ev\`{e}nements et I/O}

\end{frame}

\begin{frame}
  \frametitle{T\^{a}ches et threads}

\end{frame}

% -)

\begin{frame}
  \frametitle{L'ordonnanceur}

  kaneton dispose d'un ordonnanceur permettant l'ex\'{e}cution
  simultan\'{e}e de plusieurs threads.

  Cet ordonnanceur est pr\'{e}emptif, \'{e}quitable et avec gestion de
  priorit\'{e}s. L'algorithme utilis\'{e} est un \emph{multilevel
  feedback queues} (\`{a} l'instar de GNU/Linux).

\end{frame}

% -)

\begin{frame}
  \frametitle{Le multilevel feedback queue}

  \begin{center}
    \pgfuseimage{mfq1}
  \end{center}
\end{frame}

% -)

\begin{frame}
  \frametitle{Le multilevel feedback queue}

  \begin{center}
    \pgfuseimage{mfq2}
  \end{center}
\end{frame}

% -)

\begin{frame}
  \frametitle{Le multilevel feedback queue}

  \begin{center}
    \pgfuseimage{mfq3}
  \end{center}
\end{frame}

% -)

\begin{frame}
  \frametitle{Le multilevel feedback queue}

  \begin{center}
    \pgfuseimage{mfq4}
  \end{center}
\end{frame}

% -)

\begin{frame}
  \frametitle{Le multilevel feedback queue}

  \begin{center}
    \pgfuseimage{mfq5}
  \end{center}
\end{frame}

% -)

\begin{frame}
  \frametitle{Le multilevel feedback queue}

  \begin{center}
    \pgfuseimage{mfq6}
  \end{center}
\end{frame}

% -)

\begin{frame}
  \frametitle{Le multilevel feedback queue}

  \begin{center}
    \pgfuseimage{mfq7}
  \end{center}
\end{frame}

% -)

\begin{frame}
  \frametitle{Le multilevel feedback queue}

  \begin{center}
    \pgfuseimage{mfq1}
  \end{center}
\end{frame}

% -)

\begin{frame}
  \frametitle{Le support multiprocesseur}

  Nous pr\'{e}voyons que kaneton puisse tirer parti des diff\'{e}rents
  microprocesseurs dans un syst\`{e}me multiprocesseur.

  Dans un syst\`{e}me multiprocesseur symé\'{e}trique (SMP),
  l'ordonnanceur disposera d'une file de threads actifs pour chaque
  microprocesseur.

  Le cpu manager permettra d'effectuer du \emph{load balancing} et de
  la migration de tâches d'un processeur \`{a} un autre.

  Le gestionnaire de m\'{e}moire devra lui aussi subir des
  modifications afin d'assurer dans tout le syst\`{e}me la
  coh\'{e}rence des espaces d'adressages.

\end{frame}

\begin{frame}
  \frametitle{Les IPC}

\end{frame}

%
% avancement
%

\section{Avancement}

\begin{frame}
  \frametitle{Les managers ind\'{e}pendants}

\end{frame}

\begin{frame}
  \frametitle{Le syst\`{e}me de portabilit\'{e}}

\end{frame}

\begin{frame}
  \frametitle{Le code d\'{e}pendant de l'architecture}

\end{frame}

\begin{frame}
  \frametitle{Le portage IA-32}

\end{frame}

\begin{frame}
  \frametitle{Les tests}

\end{frame}

%
% projets
%

\section{Projets}

\begin{frame}
  \frametitle{Les choses \`{a} faire}

  \begin{itemize}
  \item
    Le portage sur architecture MIPS
  \item
    Le multiprocesseur sym\'{e}trique (SMP)
  \item
    Les pilotes
  \item
    Une couche r\'{e}seau
  \item
    Un Virtual Filesystem Switch (VFS)
  \item
    Une librairie d'\'{e}mulation UNIX
  \item
    Les services distribu\'{e}s
  \end{itemize}

\end{frame}

%
% ce que nous cherchons
%

\section{Ce que nous cherchons}

\end{document}
