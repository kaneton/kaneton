%
% ---------- header -----------------------------------------------------------
%
% project       kaneton
%
% license       kaneton
%
% file          /home/mycure/kane...view/lecture/inline-assembly/linking.tex
%
% created       julien quintard   [fri oct 24 17:31:58 2008]
% updated       julien quintard   [thu feb 12 23:24:13 2009]
%

%
% ---------- setup ------------------------------------------------------------
%

%
% path
%

\def\path{../..}

%
% template
%

%
% ---------- header -----------------------------------------------------------
%
% project       kaneton
%
% license       kaneton
%
% file          /home/mycure/kaneton/view/template/lecture.tex
%
% created       julien quintard   [wed may 16 18:17:26 2007]
% updated       julien quintard   [sun may 18 23:23:40 2008]
%

%
% class
%

\documentclass[8pt]{beamer}

%
% packages
%

\usepackage{pgf,pgfarrows,pgfnodes,pgfautomata,pgfheaps,pgfshade}
\usepackage[T1]{fontenc}
\usepackage{colortbl}
\usepackage{times}
\usepackage{amsmath,amssymb}
\usepackage{graphics}
\usepackage{graphicx}
\usepackage{color}
\usepackage{xcolor}
\usepackage[english]{babel}
\usepackage{enumerate}
\usepackage[latin1]{inputenc}
\usepackage{verbatim}
\usepackage{aeguill}

%
% style
%

\usepackage{beamerthemesplit}
\setbeamercovered{dynamic}

%
% verbatim stuff
%

\definecolor{verbatimcolor}{rgb}{0.00,0.40,0.00}

\makeatletter

\renewcommand{\verbatim@font}
  {\ttfamily\footnotesize\selectfont}

\def\verbatim@processline{
  {\color{verbatimcolor}\the\verbatim@line}\par
}

\makeatother

%
% -
%

\renewcommand{\-}{\vspace{0.4cm}}

%
% date
%

\date{\today}

%
% logos
%

\pgfdeclareimage[interpolate=true,width=34pt,height=18pt]
                {epita}{\path/logo/epita}
\pgfdeclareimage[interpolate=true,width=49pt,height=18pt]
                {upmc}{\path/logo/upmc}
\pgfdeclareimage[interpolate=true,width=25pt,height=18pt]
                {lse}{\path/logo/lse}

\newcommand{\logos}
  {
    \pgfuseimage{epita}
  }

%
% institute
%

\institute
{
  \inst{1} kaneton microkernel project
}

%
% table of contents at the beginning of each section
%

\AtBeginSection[]
{
  \begin{frame}<beamer>
   \frametitle{Outline}
    \tableofcontents[current]
  \end{frame}
}

%
% table of contents at the beginning of each subsection
%

\AtBeginSubsection[]
{
  \begin{frame}<beamer>
   \frametitle{Outline}
    \tableofcontents[current,currentsubsection]
  \end{frame}
}


%
% title
%

\title{Program linking and object files}

%
% document
%

\begin{document}

%
% title frame
%

\begin{frame}
  \titlepage
\end{frame}

%
% outline frame
%

\begin{frame}
  \frametitle{Outline}

  \tableofcontents
\end{frame}

%
% figures
%

\pgfdeclareimage[interpolate=true,width=320pt]
                {buildcycle}
                {figures/buildcycle}

\pgfdeclareimage[interpolate=true,width=370pt]
                {elf}
                {figures/elf}

%
% ---------- text -------------------------------------------------------------
%

%
% introduction
%

\section{Introduction}

% 1)

\begin{frame}
  \frametitle{Overview}

  This course targets \textbf{executable formats} in an operating system.

  \-

  This course will answer questions such as: how source-code is turned into
  object-code and what are the steps from source-code compiling to
  runtime execution.

\end{frame}

% 2)

\begin{frame}
  \frametitle{Assumptions}

  A good knowledge of \textbf{ABI} (Application Binary Interface) should be helpful. This
  course will deal with stack, arguments passing, local and gloable variables,
  \etc{}

  \-

  Having, at least once in a lifetime, taken a look at a \textit{.o} file should make things much
  clearer.
\end{frame}



%
% overview
%

\section{Computer architecture}

% 1)

\begin{frame}
  \frametitle{Von Neumann's architecture :: Basics}

  Todays code and data programs structures are a direct application of the \textbf{Von Neumann's architecture}.

  \-

  This architecture falls into four main parts:

  \begin{itemize}
    \item
      The ALU (Arithmetic Logic Unit) : Achieve basic operations such as addition and incrementation.
    \item
      The Control Unit : In charge of scheduling instructions.
    \item
      The memory : Contains both program and data.
    \item
      Inputs and outputs.
  \end{itemize}

\end{frame}

% 2)

\begin{frame}
  \frametitle{Von Neumann's architecture :: Advanced concepts}

A program is read from a mass memory (\textit{i.e.} non volatile storage such as
tape and hard-drive), and is loaded into the execution memory. This is
the main difference with the former Harvard architecture where code and
data are physically splitted.

\-

Initially, regarding code as data allowed code modification at runtime. For
instance, a loop instruction was buit from an opcode plus the index. Later,
with the rise of registers inside processors, this feature became
deprecated.

\-

Nevertheless, this architecture endures for a quite obvious reason: it
permits the use of \textbf{compilers}.

\end{frame}



%
% from source-code to runtime execution
%

\section{From source-code to runtime execution}

% 1)

\begin{frame}
  \frametitle{The linking phase :: The linker}

Linking is the process of combining various pieces of code and data together
to form a single executable that can be loaded in memory.

\-

The basic concepts of linking remain the same, regardless of the operating system,
processor architecture or object file format being used.

\-

The linker is a program that takes one or more objects generated by a compiler and combines them
into a single executable program.

\end{frame}

% 2)

\begin{frame}
  \frametitle{The linking phase :: Symbol resolution}

Assemblers and compilers produce \textbf{relocatable object files}. Relocatable means functions
and variables are not binded to any address. In a relocatable object file, address are symbols
(i.e. strings like assembly line: \textit{call foo}).

\-

Linkers combine these object files together to generate executable object files, by
turning these symbols into address. In other words, the linker assigns runtime
addresses to each section and each symbol. At this point, the code (functions) and data
(static and global variables) will have unique \textbf{runtime addresses}.

\-

During the link-editing of an object, any relocation information supplied with the input
relocatable objects is applied to the output file.However, when creating a dynamic executable or
shared object, many of the relocations cannot be completed at link-edit time. These relocations
require logical addresses that are known only when the objects are loaded into memory. In these
cases, the link-editor generates new relocation records as part of the output file image. The
loader (runtime linker) must then process these new relocation records.


\end{frame}

% 3)

\begin{frame}
  \frametitle{The loading phase}

\textbf{Program loading} refers to copying a program image from hard disk to the main memory in
order to put the program in a ready-to-run state.

\-

The linker creates a program image, the loader is in charge to load this image in the
system memory (virtual memory the system is running in protected mode), and to manage
all other images present.

\-

In some cases, program loading also
might involve allocating storage space or mapping virtual addresses to disk pages.

\end{frame}

% 4)

\begin{frame}
  \frametitle{Build cycle}

 \begin{center}
   \pgfuseimage{buildcycle}
  \end{center}

\end{frame}



%
% object file
%

\section{Object file}

% 1)

\begin{frame}
  \frametitle{Object file :: Executable format}

An object file format is a computer file format used for the storage of
object code and related data typically produced by a compiler or assembler.

\-

There are many different object file formats; originally each type of
computer had its own unique format, but with the advent of Unix and
other portable operating systems, some formats, such as COFF and ELF,
have been defined and used on different kinds of systems.

\-

\begin{itemize}
    \item
      ELF (modern UNIX, Linux, Solaris, \etc{})
    \item
	COFF (System V)
    \item
	PE (stands for Portable Executable, Windows NT)
    \item
	DWARF
        \item
a.out
  \end{itemize}

\end{frame}

% 2

\begin{frame}
  \frametitle{Object file :: File structure}

Most object file
formats are structured as blocks of data, each block containing a
certain type of data. These blocks can be paged in as needed by the
virtual memory system, needing no further processing to be ready to use.

\-

The linker, with the help of a \textbf{linker script}, builds executable object files.

\-

The simplest object file format is the DOS \textit{.COM} format,
which is simply a file of raw bytes that is always loaded at a fixed
location. Other formats are more elaborate: they may contains relocation
and debugging information (COFF, ELF, \etc{}).

\-

Types of data supported by typical object file formats:

\begin{itemize}
    \item
	BSS (\textit{Block Started by Symbol})
    \item
	Text segement
    \item
	Data segment
   \end{itemize}

\end{frame}



%
% memory block
%

\section{Memory block}

% 1)

\begin{frame}
  \frametitle{Memory block :: The text segment}

A.k.a text segment, text, code, text section (mis-use), code section (mis-use), \etc{}

\-

Refers to a portion of an object file that contains executable instructions
(\textit{i.e.} the machine-code). This is why object files are usually called
binaries.

\-

Text segment is the only essential element in an object file.

\-

It has a fixed size and is usually read-only. If the text section is
not read-only, then the particular architecture allows self-modifying
code.

\-

As a memory region, a code segment resides in the lower parts of
memory or at its very bottom, in order to prevent heap and stack
overflows from overwriting it.

\end{frame}

% 2)

\begin{frame}
  \frametitle{Memory block :: The data segment}

The data segment in an object file contains the global variables that
have been initialized by the programmer. It has a fixed size, since all
of the data in this section is set by
the programmer before the program is loaded.

\-

However, it is not read-only,
since the values of the variables can be altered at runtime.
This is in contrast to the Rodata (constant, read-only data) section,
as well as the code segment (also known as text segment).

\end{frame}

% 3)

\begin{frame}
  \frametitle{Memory block :: The bss}

It is often referred to as the
"bss section" or "bss segment". ``bss'' stands for Block Started by Symbol.

\-

The ``bss'' contains uninitialized static variables. The program loader initializes the
memory allocated for the bss section when it loads the program (usually,
the bss segment is zero-filled).

\-

For embedded kernel developpment, the ``bss'' is usually used to reserve memory space
for the heap and the stack.

\end{frame}



%
% case study
%

\section{Case studies}

% 1)

\begin{frame}
  \frametitle{Case studies :: The ELF object file format}

The segments contain information that is necessary for runtime execution
of the file, while sections contain important data for linking and
relocation. Sections are defined in the linker script, while segments are
the products of object-code and a linker script.

\-

Unlike many proprietary executable file formats, ELF is very flexible and
extensible,and it is not bound to any particular processor or architecture. This has allowed
it to be adopted by many different operating systems on many different platforms.

\end{frame}

% 2)

\begin{frame}
  \frametitle{Case studies :: The ELF object file format}


 \begin{center}
   \pgfuseimage{elf}
  \end{center}


\end{frame}

% 3)

\begin{frame}[containsverbatim]
  \frametitle{Case studies :: Writing a linker script for the WindRiver linker}

\begin{verbatim}
$cat basic.c
#pragma use_section .stack
int stack[256];

#pragma use_section .code
int main()
{
  return 0;
}
\end{verbatim}

\end{frame}

% 4)

\begin{frame}[containsverbatim]
  \frametitle{Case studies :: Writing a linker script for the WindRiver linker}

\begin{verbatim}
MEMORY
{
  flash: org = 0xF0003000, len = 0x20000
  sram: org = 0x00023000, len = 0x10000
}

SECTIONS
{
  GROUP : {
    .code : {
      *(.text)
    }
  } > flash

  GROUP : {
    .var : {
      *(.data)
    }

    .stack : {
      *(.bss)
    }
  } > sram
}

$ dld -o basic.bin -l basic.o basic.dld
\end{verbatim}

\end{frame}

% 5)

\begin{frame}[containsverbatim]
  \frametitle{Case studies :: Writing a linker script for the WindRiver linker}

The linker command outputs a link map, referencing for each functions and variables its address.

\begin{verbatim}
$ dld -o basic.bin -l basic.o basic.dld > basic.map
\end{verbatim}

A map file looks like:

\begin{verbatim}
g_task_pending          0x00023000      0x00000010
g_an_int                0x00023010      0x00000004

SC_krnTaskStart         0xf0003000      0x00000400
SC_krnTaskStop          0xf0003400      0x00000200
SC_krnTaskDelete        0xf0003600      0x00000800
...
\end{verbatim}
\end{frame}



%
% exercises
%

\section{Exercises}

% 1)

\begin{frame}[containsverbatim]
  \frametitle{Exercise :: 1}

The following source code is linked as a UNIX standard object file.

\begin{verbatim}
#include <stdio.h>

int foo;
int bar = 90;

int main()
{
  int local1;
  int local2 = 67;

  foo = 4;
  printf("%d %d %d", local2, foo, bar);

  return 0;
}
\end{verbatim}

\begin{enumerate}
    \item
    What is the UNIX standard object file format?
    \item
    How many sections will be used?
    \item
    How many segments will be produced?
    \item
    Where these functions and variables will be located in the object file?
   \end{enumerate}

\end{frame}

% 2)

\begin{frame}[containsverbatim]
  \frametitle{Exercise :: 2}

Loading this code as an ELF object file, what will be the output?

\begin{verbatim}
#include <stdio.h>

int bar = 90;

int main()
{
  printf("%d\n", bar);

  return 0;
}
\end{verbatim}


\end{frame}

% 3)

\begin{frame}[containsverbatim]
  \frametitle{Exercise :: 3}

Loading this code as a raw binary object file, what will be the output?

\begin{verbatim}
#include <stdio.h>

int bar = 90;

int main()
{
  printf("%d\n", bar);

  return 0;
}
\end{verbatim}

\end{frame}



%
% conclusion
%

\section{Conclusion}

% 1)

\begin{frame}
  \frametitle{Conclusion}

  In this course, basic concepts of linking phase has been shown. Besides
  linking tool, modern operating systems using virtual memory and paging
  may need extra tools like loaders, although they are not mandatory.

  \-

  Much more theory on linkers and loaders is available. Take a look
  at whitepapers on the subjects. Advanced linking concepts :

  \begin{itemize}
    \item
        Relocation algorithms
    \item
        Relaxation
    \item
        Library
    \item
        Dynamic linking
    \item
      \etc{}
  \end{itemize}

\end{frame}



%
% bibliography
%

\section{Bibliography}

\begin{thebibliography}{3}
  \bibitem{Hp-UX}
    HP-UX Linker and Libraries User's Guide, HP 9000 Computers, HP

  \bibitem{Sun}
    Linkers and Libraries Guide, Sun Microsystems
\end{thebibliography}

\end{document}
