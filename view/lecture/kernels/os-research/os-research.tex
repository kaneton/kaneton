%
% ---------- header -----------------------------------------------------------
%
% project       kaneton
%
% license       kaneton
%
% file          /home/mycure/kane...ecture/kernels/windows-nt/windows-nt.tex
%
% created       fabien le mentec
% updated       julien quintard   [wed apr 22 11:29:29 2009]
%

%
% ---------- setup ------------------------------------------------------------
%

%
% path
%

\def\path{../../..}

%
% template
%

%
% ---------- header -----------------------------------------------------------
%
% project       kaneton
%
% license       kaneton
%
% file          /home/mycure/kaneton/view/template/lecture.tex
%
% created       julien quintard   [wed may 16 18:17:26 2007]
% updated       julien quintard   [sun may 18 23:23:40 2008]
%

%
% class
%

\documentclass[8pt]{beamer}

%
% packages
%

\usepackage{pgf,pgfarrows,pgfnodes,pgfautomata,pgfheaps,pgfshade}
\usepackage[T1]{fontenc}
\usepackage{colortbl}
\usepackage{times}
\usepackage{amsmath,amssymb}
\usepackage{graphics}
\usepackage{graphicx}
\usepackage{color}
\usepackage{xcolor}
\usepackage[english]{babel}
\usepackage{enumerate}
\usepackage[latin1]{inputenc}
\usepackage{verbatim}
\usepackage{aeguill}

%
% style
%

\usepackage{beamerthemesplit}
\setbeamercovered{dynamic}

%
% verbatim stuff
%

\definecolor{verbatimcolor}{rgb}{0.00,0.40,0.00}

\makeatletter

\renewcommand{\verbatim@font}
  {\ttfamily\footnotesize\selectfont}

\def\verbatim@processline{
  {\color{verbatimcolor}\the\verbatim@line}\par
}

\makeatother

%
% -
%

\renewcommand{\-}{\vspace{0.4cm}}

%
% date
%

\date{\today}

%
% logos
%

\pgfdeclareimage[interpolate=true,width=34pt,height=18pt]
                {epita}{\path/logo/epita}
\pgfdeclareimage[interpolate=true,width=49pt,height=18pt]
                {upmc}{\path/logo/upmc}
\pgfdeclareimage[interpolate=true,width=25pt,height=18pt]
                {lse}{\path/logo/lse}

\newcommand{\logos}
  {
    \pgfuseimage{epita}
  }

%
% institute
%

\institute
{
  \inst{1} kaneton microkernel project
}

%
% table of contents at the beginning of each section
%

\AtBeginSection[]
{
  \begin{frame}<beamer>
   \frametitle{Outline}
    \tableofcontents[current]
  \end{frame}
}

%
% table of contents at the beginning of each subsection
%

\AtBeginSubsection[]
{
  \begin{frame}<beamer>
   \frametitle{Outline}
    \tableofcontents[current,currentsubsection]
  \end{frame}
}


%
% title
%

\title{Operating Systems Research}

%
% document
%

\begin{document}

%
% title frame
%

\begin{frame}
  \titlepage
\end{frame}

%
% outline frame
%

\begin{frame}
  \frametitle{Outline}

  \tableofcontents
\end{frame}

%
% Introduction
%

\section{Introduction}

\begin{frame}
 \frametitle{About}

 \begin{itemize}
   \item
     A course on practical operating systems issues
   \item
     Pointers on ongoing research, recent projects
   \item
     What you may want to work on, advices on how to do it
 \end{itemize}
\end{frame}

\begin{frame}
 \frametitle{Plan}

 \begin{itemize}
   \item
     computing landscape
     \begin{itemize}
       \item hardware issues
       \item software issues
     \end{itemize}
   \item
     kernel and OS mechanisms addressing those issues
     \begin{itemize}
       \item Singularity
       \item Helios
       \item Barrelfish
       \item coping with existing software
     \end{itemize}
   \item
     approach when doing practical OS related research
     \begin{itemize}
       \item summary
       \item perspectives
     \end{itemize}
   \item
     references
     \begin{itemize}
       \item research groups
       \item projects
       \item conferences
       \item jobs, internships
     \end{itemize}
 \end{itemize}
\end{frame}


%
% Computing landscape
%

%
% Kernel and OS mechanisms addressing those issues
%

\section{Kernel and OS mechanisms}

% singularity

\begin{frame}
  \frametitle{Singularity - 0}

  Kernel of a full operating system research framework from Microsoft.

  \begin{itemize}
    \item answers the following question: "what would a software plateform look like if it was designed from scratch, with the primaty goal of improved depenability and trustwothiness"
    \item ie. without the legacy of the 70's
    \item aims to rethink the software stack to avoid unexpected interactions between applications
    \item unexpected interactions lead to software vulnerabilities and lack of robustness
  \end{itemize}

\end{frame}

\begin{frame}
  \frametitle{Singularity - 1}
  3 key architectureal features
  \begin{itemize}
    \item support for execution
      \begin{itemize} \item software isolated processes (SIPs) \end{itemize}
    \item support for communication
      \begin{itemize} \item contract based channels \end{itemize}
    \item support for system verification
      \begin{itemize} \item manifest based programs \end{itemize}
  \end{itemize}
\end{frame}

\begin{frame}
  \frametitle{Singularity - 2}
  SIPs
  \begin{itemize}
    \item single AS, multithreaded software isolated processes
    \item all communication done via message passing
      \begin{itemize} \item ie. no CreateRemoteXxx() \end{itemize}
    \item lightweight creation costs, management
      \begin{itemize}
	\item part of the work is done statically
	\item no TLB/cache flush as context is switched
	\item costs: \textasciitilde 0.5 fork() on Linux
      \end{itemize}
    \item no code modification
      \begin{itemize}
	\item plugins run in a separated process with its own security descriptor
	\item thus there is no need for a conccurent security model (ie. think about active x...)
	\item memory accesses are checked statically
      \end{itemize}
  \end{itemize}
\end{frame}

\begin{frame}
  \frametitle{Singularity - 3}
  Contract based channels
  \begin{itemize}
    \item every communication between SIPs occur through channels
    \item 2 endpoints typed message queue
      \begin{itemize} \item // todo: channel example \end{itemize}
    \item declarative, implemented in Sing\#
      \begin{itemize}
        \item declarative means less prone to programmer's error, thus improved security
        \item implied documentation
      \end{itemize}
    \item described as a state automata
  \end{itemize}
\end{frame}


\begin{frame}
  \frametitle{Singularity - 4}
  Manifest based programs
  \begin{itemize}
    \item a manifest describes a program and its interactions with the system
    \begin{itemize}
      \item which resources it needs, which restrictions it has over the system
      \item drives the SIPs code placement // todo: show the code
      \item channels interconnections
    \end{itemize}
    \item it describes an expected behavior
    \item the kernel itself has a manifest // todo: show the code
  \end{itemize}
\end{frame}


\begin{frame}
  \frametitle{Singularity - 5}
  Singularity kernel
 
 \begin{itemize}
   \item single address space microkernel, SIPs execute outside the kernel
     \begin{itemize} \item but 192 ABI functions \end{itemize}
   \item 90\% implemented in Sing\#
     \begin{itemize} \item garbage collection, introspection... \end{itemize}
   \item rigorous ABI, no ioctl like calls
   \item follow the least priviledge principle
   \begin{itemize}
     \item initially, a process can only manipulate its own state, and create child SIPs
     \item according to its manifest, it can be created with existing endpoints
     \item it can receive endpoints via a channel
     \begin{itemize}
       \item this last point allows for statically checked use of a resource
       \item for instance, notepad.exe using a Winsock socket...
     \end{itemize}
   \end{itemize}
  \item the ABI is versionned, allowing to handle backward compatibility
    \begin{itemize} \item no need for awful versionned structures, thus improved security \end{itemize}
 \end{itemize}
\end{frame}


\begin{frame}
  \frametitle{Singularity - 6}
  The exchange heap
 
 \begin{itemize}
   \item the only area where data can be exchanged between SIPs
   \item ownership of a block is transfered via a channel, then released to the destination SIP
   \item a heap block is accessible by a single thread at a time
   \item questions
     \begin{itemize}
       \item how to implement broadcast like message, data duplication? (since only one dest)
       \item what if 2 thread want to communicate data to 2 different endpoints
       \item for instance network and fs: the paper says there is only one heap block per SIP at a time
     \end{itemize}
 \end{itemize}
\end{frame}


\begin{frame}
  \frametitle{Singularity - 7}
  Protection domains
 
  \begin{itemize}
    \item even if SIPs are isolated in software, the Singularity framework allows for hardware protection, ie. address spaces
    \item called augmented SIPs
    \item "in case software based verification fails" :)
    \item protection domains can be selected at will, allowing for different favors of separation schemes to be used (ie. microkernel, monolithic...)
  \end{itemize}
\end{frame}


\begin{frame}
  \frametitle{Singularity - 8}
  Singularity framework / Research Development Kit
  \begin{itemize}
    \item the whole framework is open source and documented
    \item http://singularity.codeplex.com/
    \item asplos2008\_singularity\_rdk\_tutorial.pdf
  \end{itemize}
\end{frame}


\begin{frame}
  \frametitle{Singularity - 9}
  Conclusion
\end{frame}


\begin{frame}
  \frametitle{Singularity - 10}
  References
  \begin{itemize}
    \item http://research.microsoft.com/en-us/projects/singularity/
  \end{itemize}
\end{frame}


% Helios

\begin{frame}
\frametitle{Helios - 0}
Issue addressed by Helios
\begin{itemize}
 \item more and more tasks offloaded to heterogeneous processing units
 \item programmable hardware (FPGAs), DPSs, GPUs...
 \item Helios places an application on the unit it is the more affine with
\end{itemize}
\end{frame}


\begin{frame}
\frametitle{Helios - 1}
Sattellite kernels
\begin{itemize}
 \item a sattellite kernel is instanciated for each processor: cpu, programmable device...
 \item implemented as a process, each with its own scheduler.
 \item requierments: timer, interrupt controller, exception handling.
 \item each kernel contains: scheduler, memory manager.
 \item the coordinator contains the namespace manager (queryable via remote message passing)
\end{itemize}
\end{frame}


\begin{frame}
\frametitle{Helios - 2}
Affinity model
  \begin{itemize}
    \item processes can specify processes they are bound to
    \item for instance, network stack bound to network interface 
    \item affinity is used to choose a kernel the process will run on
  \end{itemize}
\end{frame}


\begin{frame}
\frametitle{Helios - 3}
Implementation
  \begin{itemize}
    \item modification of the singularity os (singularity rdk). written in Sing\#
    \item Application first compiled into the CIL bytecode, then into the ISA of the particular processor
    \item if vendor refuses to ship the cil bytecode, it can provide fat binaries
  \end{itemize}
\end{frame}


\begin{frame}
\frametitle{Helios - 4}
Discussion
  \begin{itemize}
    \item no automated way is proposed to decide about the affinity
    \begin{itemize}
      \item 2 processors, 1 with a FPU
      \item 2 applications, 1 using the FPU
      \item how to choose the more affine unit
    \end{itemize}
    \item requirements to be considered as a processing unit are too strong
    \begin{itemize}
      \item GPUs have no programmable timers
      \item Cryptographic coprocessors are blackboxes
    \end{itemize}
    \item application must be redesigned in term of processes
    \begin{itemize} \item break the multithread programming model \end{itemize}
  \end{itemize}
\end{frame}


\begin{frame}
\frametitle{Helios - 5}
References
  \begin{itemize}
    \item www.sigops.org/sosp/sosp09/papers/nightingale-sosp09.pdf
    \item research.microsoft.com/en-us/groups/os/singularity/
  \end{itemize}
\end{frame}


% Barrelfish

\begin{frame}
\frametitle{References}
\begin{itemize}
  \item http://www.barrelfish.org/barrelfish\_sosp09.pdf
  \item http://www.barrelfish.org/
  \item http://research.microsoft.com/en-us/groups/camsys/default.aspx
  \item http://www.systems.ethz.ch/
\end{itemize}
\end{frame}


% Coping with existing software

\section{Coping with existing software}

\begin{frame}
  \frametitle{Coping with existing software}
  Sometimes not possible to start from scratch to resolve an issue and have to cope with
  existing software. In this section we see research projects dealing with such issues.
\end{frame}


\begin{frame}
  \frametitle{shadowdrivers - 0}
  \begin{itemize}
    \item addresses the application state recovery problem
    \begin{itemize} \item if an error occurs, restore the application last valid state \end{itemize}
    \item shadowdrivers is a way to improve operating system stability by hiding to the userland application a driver has crashed
    \item it adds a transparent local redudancy
    \item similar to fault tolerance schemes used in distributed systems, but assumes non malicious drivers
    \item however, the shadow driver is not a full replica, implements only the needed services
  \end{itemize}
\end{frame}


\begin{frame}
  \frametitle{shadowdrivers - 1}
  Discussion
  \begin{itemize}
    \item assume it is possible to detect driver crashes
    \item either not possible, or too late (ie. kernel has already crashed)
    \item a well written application has to handle errors. the problem should not be solved at the driver level
    \item anoter approach: the operating system provides a snapshot/recovery API for the userland applications to snapshot and recover their states in case of crash
      \begin{itemize} \item but not transparent, and this is the goal of such a project \end{itemize}
    \item this approach would be well suited for a stackable model like the NT one
    \item works even between drivers, since drivers can be seen as clients of one another
  \end{itemize}
\end{frame}


\begin{frame}
 \frametitle{nooks}
 nooks is another project aiming at improving device driver reliability

 \begin{itemize}
   \item defines an interposition layer between drivers (ie. kernel extension) and kernel (linux)
   \item the framework implements the following layers:

   \begin{itemize}

     \item isolation
     \begin{itemize}
       \item memory management is modified so that the kernel is mapped read only in the extensions
       \item extension procedure calls (XPC) to transfer control flow from extensions to kernel
     \end{itemize}

     \item interposition
     \begin{itemize}
       \item kernel and extension wrappers (ie. indirect calls as modules are loaded)
       \item some wrapper create a shadow copy of the params, which is then synchronized as the XPC returns
     \end{itemize}

     \item object tracking
     \begin{itemize}
       \item instrument kernel functions allocating objects to keep track of all the objects
       \item different scheme used to stay performant
       \item object garbage collection
     \end{itemize}

     \item recovery
     \begin{itemize}
       \item detects potential bugs: invalid parameters detection, deadlocks, invalid memory deallocation...
       \item on bug, reload the extension without the application knowing it
     \end{itemize}

    \end{itemize}
  \end{itemize}
\end{frame}


\begin{frame}
  \frametitle{References}
  \begin{itemize}
    \item http://www.cs.washington.edu/homes/levy/shadowdrivers.pdf
    \item http://www.cs.washington.edu/homes/levy/, for more work about OS reliability
    \item http://nooks.cs.washington.edu/
  \end{itemize}
\end{frame}


%
% Approach
%

\section{Approach}

\begin{frame}
  \frametitle{Approach - 0}
  Operating systems research is very large, difficult to have a clear view

  \begin{itemize}
    \item
      many heterogeneous topics
    \item
      lot of research groups, rarely working together
    \item
      even in the same lab, it is possible teams do not collaborate (esp. true if not in same country)
    \item
      lot of unused (useless?) and redundant work
    \item
      emphasis on producing papers
    \item
      but there still exists labs working on real things
      \begin{itemize}
	\item
	  http://qnx.com
	\item
          http://www.ok-labs.com
        \item
	  http://ertos.nicta.com.au
        \item
	  http://www.xtreemos.eu
	\item
	  \textit{check their opportunities page if interested}
      \end{itemize}
  \end{itemize}
\end{frame}

\begin{frame}
  \frametitle{Approach - 1}
  From this course point of view, research must be practical.
  There should at least be a prototype. Some advices:
  \begin{itemize}
    \item
      generally, targeting an existing software is better than reimplementing from scratch
    \item
      do something useful for NT, it is likely someone form MS will get in touch with you
    \item
      for instance: http://dokan-dev.net/en/
    \item
      main reason: it is nearly impossible to get ride of backward compatibility(existing user base) and very difficult to cope with legacy (badly designed APIs...)
    \item
      the real challenge: implementing new ideas that are backward compatible
      with previous software (esp. challenging in proprietary software)
    \item
      common point between Qemu, VMWare, NaCL: they preserve what already exists
    \item
      not to say that working on something new is a bad idea
  \end{itemize}
\end{frame}

\begin{frame}
  \frametitle{Approach - 2}
  To illustrate, here is a system project example
  \begin{itemize}
  \item
    automatically (ie. not by hand) monitors PCI bus requests on a NT system
    \begin{itemize} \item you have which what register is hit, the access mode... at a given time T \end{itemize}
  \item
    automatically monitor application behavior
    \begin{itemize} \item you have the function called by an application at T \end{itemize}
  \item
    correlate the events together
    \begin{itemize} \item you have an automated way to tell what a PCI register is used for \end{itemize}
  \item
    usage
    \begin{itemize}
      \item automated PCI driver specifications (couple it with a DSL: http://code.google.com/p/rathaxes/)
      \item propietary driver reverse engineering
      \item virtualisation: add missing functions, correct bugs...
    \end{itemize}
  \end{itemize}
\end{frame}

\end{document}
