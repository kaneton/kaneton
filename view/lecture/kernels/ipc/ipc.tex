%
% ---------- header -----------------------------------------------------------
%
% project       kaneton
%
% license       kaneton
%
% file          /home/mycure/kane.../kernels/prerequisites/prerequisites.tex
%
% created       julien quintard   [wed may 16 19:28:59 2007]
% updated       julien quintard   [wed may 16 19:29:16 2007]
%

%
% ---------- setup ------------------------------------------------------------
%

%
% path
%

\def\path{../../..}

%
% template
%

%
% ---------- header -----------------------------------------------------------
%
% project       kaneton
%
% license       kaneton
%
% file          /home/mycure/kaneton/view/template/lecture.tex
%
% created       julien quintard   [wed may 16 18:17:26 2007]
% updated       julien quintard   [sun may 18 23:23:40 2008]
%

%
% class
%

\documentclass[8pt]{beamer}

%
% packages
%

\usepackage{pgf,pgfarrows,pgfnodes,pgfautomata,pgfheaps,pgfshade}
\usepackage[T1]{fontenc}
\usepackage{colortbl}
\usepackage{times}
\usepackage{amsmath,amssymb}
\usepackage{graphics}
\usepackage{graphicx}
\usepackage{color}
\usepackage{xcolor}
\usepackage[english]{babel}
\usepackage{enumerate}
\usepackage[latin1]{inputenc}
\usepackage{verbatim}
\usepackage{aeguill}

%
% style
%

\usepackage{beamerthemesplit}
\setbeamercovered{dynamic}

%
% verbatim stuff
%

\definecolor{verbatimcolor}{rgb}{0.00,0.40,0.00}

\makeatletter

\renewcommand{\verbatim@font}
  {\ttfamily\footnotesize\selectfont}

\def\verbatim@processline{
  {\color{verbatimcolor}\the\verbatim@line}\par
}

\makeatother

%
% -
%

\renewcommand{\-}{\vspace{0.4cm}}

%
% date
%

\date{\today}

%
% logos
%

\pgfdeclareimage[interpolate=true,width=34pt,height=18pt]
                {epita}{\path/logo/epita}
\pgfdeclareimage[interpolate=true,width=49pt,height=18pt]
                {upmc}{\path/logo/upmc}
\pgfdeclareimage[interpolate=true,width=25pt,height=18pt]
                {lse}{\path/logo/lse}

\newcommand{\logos}
  {
    \pgfuseimage{epita}
  }

%
% institute
%

\institute
{
  \inst{1} kaneton microkernel project
}

%
% table of contents at the beginning of each section
%

\AtBeginSection[]
{
  \begin{frame}<beamer>
   \frametitle{Outline}
    \tableofcontents[current]
  \end{frame}
}

%
% table of contents at the beginning of each subsection
%

\AtBeginSubsection[]
{
  \begin{frame}<beamer>
   \frametitle{Outline}
    \tableofcontents[current,currentsubsection]
  \end{frame}
}



%
% title
%

\title{Inter Process Communication}

%
% document
%

\begin{document}

%
% title frame
%

\begin{frame}
  \titlepage
\end{frame}

%
% figures
%

\begin{frame}
        \tableofcontents
\end{frame}
  % -)

\section{Intro}
\subsection{Needs of IPCs}
\begin{frame}
\frametitle{Needs of IPCs}
\begin{itemize}
\item In a micro kernel everything is a different task.
\item Drivers run inside task.
\item There is a need for communication between those entities.
\end{itemize}
\end{frame}

\begin{frame}
\frametitle{History}
\begin{itemize}
\item Since the micro kernel research exist everyone tries to come out
with the best and the fast IPC implement.
\item A micro kernel will spend most of its time doing ipcs
\end{itemize}
\end{frame}

\begin{frame}
\frametitle{Why IPC are slow}
\begin{itemize}
\item IPC is about communication between two process that run in user
space.
\item The kernel has to allow them to communicate together.
\item The IPC are forced to pass by the kernel, otherwise there is some
security breach.
\end{itemize}
\end{frame}

\begin{frame}
\frametitle{Smalest set of operation to perform an IPC}
\begin{itemize}
\item Thread A
\begin{enumerate}
\item load ID of B
\item set msg leght to 0
\item call kernel
\end{enumerate}
\item Kernel
\begin{enumerate}
\item access Thread B
\item Switch stack pointer
\item Switch address space
\item load if of A
\item return to user
\end{enumerate}
\item Thread B
\begin{enumerate}
\item Received msg
\end{enumerate}
\end{itemize}
\end{frame}

\begin{frame}
\frametitle{Where is the time spend?}
\begin{itemize}
\item Let T the time of the all sequence.
\item 70\% of T is spent by context switching.
\item An address space switching take a lot of time.
\item Switching of address space flush the TLBs.
\end{itemize}
\end{frame}

\section{IPCs}
\section{Pipe}
\begin{frame}
\frametitle{Pipe}

\begin{itemize}
\item This is a synchronous IPC paradigm.
\item Can be repesented as a file, or just an fd.
\item FIFO.
\item The buffer is stored in the kernel.
\end{itemize}
\end{frame}

\section{Socket}
\begin{frame}
\frametitle{Socket}
\begin{itemize}
\item Socket is used for wide range communication.
\item A lot of RPC use TCP/IP so they use sockets.
\item MPI use sockets.
\item Socket are slow, and got a big latency but they work on remote
systems.
\end{itemize}
\end{frame}

\begin{frame}
\frametitle{RPC over Sockets}
\begin{itemize}
\item XML-RPC
\item SOAP
\item Sun RPC (NFS)
\item Java RMI
\item Corba
\item And so many more ...
\end{itemize}
\end{frame}

\subsection{Shared memory}
\begin{frame}
\begin{itemize}
\item The kernel allocate a physical page.
\item The same physical page is mapped in both address spaces.
\item Semaphore can be used inside the region.
\item This is a fastest way to exanche data from one process to another.
\item This is a 0 copy paradigm.
\item This mecanism is a bit intrusive and can lead to security issues.
\end{itemize}
\end{frame}

\subsection{Message queue}
\begin{frame}
\frametitle{Message queue}
Message queues are basically a ring buffer over shared mem.
\end{frame}


\section{Kaneton IPC}
\subsection{Init}
\begin{frame}
\frametitle{Init}
\begin{itemize}
\item Each thread own a mailbox
\item The mailbox is used to store the messages in asynchronous mode.
\end{itemize}
\end{frame}

\subsection{Synchronous}
\begin{frame}
\frametitle{Synchronous}
\begin{itemize}
\item message\_transmit send a synchronous message, wait until the
received thread comes.
\item message\_received block is there is no pending message, otherwise
give the message.
\end{itemize}
\end{frame}

\subsection{Asynchronous}
\begin{frame}
\frametitle{Asynchronous}
\begin{itemize}
\item message\_send send an asynchronous message.
\item message\_throw send an asynchronous message. The buffer is not
copied directly. to garanty than the caller need to use the request
argument to make the copy has been performed. On non smp system this
function perform like message\_send.
\item message\_poll non blocking function, works also for syncronous
message.
\end{itemize}
\end{frame}

\end{document}
