%
% ---------- header -----------------------------------------------------------
%
% project       kaneton
%
% license       kaneton
%
% file          /home/mycure/kane...re/kernels/generalities/generalities.tex
%
% created       julien quintard   [wed may 16 19:28:59 2007]
% updated       julien quintard   [wed apr 22 11:12:55 2009]
%

%
% ---------- setup ------------------------------------------------------------
%

%
% title
%

\title{Kernel Generalities}

%
% document
%

\begin{document}

%
% title frame
%

\begin{frame}
  \titlepage
\end{frame}

%
% outline frame
%

\begin{frame}
  \frametitle{Outline}

  \tableofcontents
\end{frame}

%
% figures
%

% bsd

\pgfdeclareimage[interpolate=true,width=157pt,height=120pt]
                {bsd}
		{\figurepath/figures/bsd}

% kaneton

\pgfdeclareimage[interpolate=true,width=145pt,height=120pt]
                {kaneton}
		{\figurepath/figures/kaneton}

% lseos

\pgfdeclareimage[interpolate=true,width=174pt,height=120pt]
                {lseos}
		{\figurepath/figures/lseos}

% hexo

\pgfdeclareimage[interpolate=true,width=176pt,height=120pt]
                {hexo}
		{\figurepath/figures/hexo}

% nt

\pgfdeclareimage[interpolate=true,width=194pt,height=120pt]
                {nt}
		{\figurepath/figures/nt}

% k

\pgfdeclareimage[interpolate=true,width=151pt,height=120pt]
                {k}
		{\figurepath/figures/k}

\pgfdeclareimage{l4ipc}
                {\figurepath/figures/l4ipc}

\pgfdeclareimage[height=180pt]
                {xenarchi}
                {\figurepath/figures/xenarchi}

%
% roles of a kernel
%

\section{Quick History}

\begin{frame}
\frametitle{Quick history}

        \begin{itemize}
                \item 60s - IBM - First main frame
                \item 70s - Unix - Servers operating system 
                \item 80s - Microsoft  - Personal computer (PC)
                \item 90s - Linux / Minix / GNU - Open source
                \item 2000 - VMware / Goggle - Virtualisation / Distribution 
        \end{itemize}

\end{frame}

\begin{frame}
\frametitle{Evolution ...}
The way to use computer has changed through the years.

  \-

The kernel and operating systems had to evolve according to those new
use cases.

  \-

Kernel came from massively multi-user to mono-user and then end up begin a mix with virtualisation and distribution.
\end{frame}

\section{Introduction}

% -)

\begin{frame}
  \frametitle{Goals}

  The kernel is the central part of an operating system. The kernel is the entity offering abstraction of the hardware to the applications, including:

  \begin{itemize}
  \item
    Memory management
  \item
    Process management
  \item
    I/Os and events (both leading to drivers)
  \item
    Inter-Process Communication
  \end{itemize}

  In addition, the kernel must ensure the security of the resources.

  \-

  The final goal of a kernel is to run user programs, to enable them
  to interact with hardware and between each others, what constitutes
  an operating system.

\end{frame}

% -)

\begin{frame}
  \frametitle{A good kernel is\ldots}

  \begin{itemize}
  \item
    Fast. Performances are critical.
  \item
    Reliable. When a kernel crashes, the whole machine does.
  \item
    Maintainable. Other developers must be able to write their own drivers or services.
  \item
    Fault tolerant. In terms of hardware faults and software faults.
  \item
    Secure.
  \item
    Portable. To work on different hardware with less possible porting effort.
  \item
    Distributed? Some says a kernel has to handle distribution and looks like a single instance even if it's running on thousand of machines.
  \end{itemize}

\end{frame}

% -)

\begin{frame}
  \frametitle{Kernel-space \& User-land}

  Kernel-space is the execution environment of the kernel. All operations are permitted. Hardware can be accessed and controlled. Microprocessor internal structures and control registers too. The kernel can access all the tasks address spaces.

  \-

  User-land is the environment of classical programs, with many restrictions. Hardware access is forbidden. Configuration facilities are denied. A program can only access its own address space.

  \-

  The more code you have in kernel-space, the more risks of crashes you
  have.

\end{frame}

%
% kernel choices 
%

\section{Kernel choices}

\subsection{Mono/Multi user}
\begin{frame}
  \frametitle{Mono/Multi user/task}
  \begin{itemize}
        \item Usually this problem comes along with the multi-tasking.
        If we decide to go for the multi-tasking we might want to have multi-user support to provide security.
        \item Some systems don't need multi-user/multi-tasking. They are often using for embedded/Real-Time applications.
        \item Multi-user support involves to thing about a system to protect data from one user against another. It could be ACL, capabilities, \ldots
  \end{itemize}
\end{frame}


\subsection{MM: Segmentation/Pagination}
\begin{frame}
  \frametitle{MM: Segmentation/Pagination}
  \begin{itemize}
        \item Segmentation/Pagination are generally used to improved
        security and flexibility.
        \item Walks throw the page tables takes a bit of time and
        it's not deterministic. Therefore it can't be used of hard
        real-time.
        \item To support the pagination you have to have a MMU on your
        platform. The MMU is usually part of the processor because it
        needs to be called every time the execution access the memory.
  \end{itemize}
\end{frame}

\subsection{Scheduling Offline/Online}
\begin{frame}
  \frametitle{Scheduling: Offline/Online}
        Online:
        \begin{itemize}
                \item The online scheduling is the usual way to schedule nowadays. The scheduler is dynamic and allocate the time slices on demand.
                \item It's much more flexible than the scheduling offline.
                \item The scheduler itself consumes some CPU time.
        \end{itemize}
        \-
        Offline:
        \begin{itemize}
                \item The order in which the tasks will be executed is pre-computed.
                \item You need a limited number of tasks.
                \item It's not dynamic.
                \item Fit perfectly in a lockdown environment (real-time again?)
        \end{itemize}
\end{frame}

%
% monolithic kernels
%

\section{Kernel design}
\subsection{Monolithic kernels}

%
% monolithic kernels
%

\begin{frame}
  \frametitle{Monolithic kernels}

  Monolithic kernels includes everything into the kernel, even drivers, file systems, networking\ldots

  \-

  \emph{This approach well known as ``The Big Mess''. The structure is that there is no structure.} -- Tanenbaum

  \-

  Pros:

  \begin{itemize}
  \item
    High performances: only function calls
  \end{itemize}

  \-

  Cons:

  \begin{itemize}
  \item
    Dangerous: one bug in a non-critical service may lead the system
    to crash
  \item
    Not easy to understand and maintain
  \end{itemize}

\end{frame}

%
% bsd
%

\begin{frame}
  \frametitle{Example: BSD}

  4.4BSD kernel includes all functionalities in kernel-land: from process management to network layer, through file systems.

  \begin{center}
    \pgfuseimage{bsd}
  \end{center}

  The kernel is about 200 000 lines long.

\end{frame} 

%
% microkernels
%

\subsection{Micro-kernels}

%
% microkernels
%

\begin{frame}
  \frametitle{Micro-kernels}

  In micro-kernels, only the critical functionalities are running in kernels-space. The other services are running as user programs (having extended privileges).

  \-

  Pros:

  \begin{itemize}
  \item
    Small amount of code in kernel-space: less bugs and risks of crashes.
  \item
    Clear design, easy to understand
  \item
    Services can be started and stopped: reduce system load, run concurrent services\ldots
  \end{itemize}

  \-

  Cons:

  \begin{itemize}
  \item
    Lots of IPC: slower performances
  \end{itemize}

\end{frame}

%
% kaneton
%

\begin{frame}
  \frametitle{Example: kaneton}

  kaneton provide a dozen of critical managers: memory, process,
  I/O\ldots{} Only these functionalities are running in kernel-land.

  \begin{center}
    \pgfuseimage{kaneton}
  \end{center}

  Advanced functionalities (file systems, network\ldots) are provided by services in userland. IPC are omnipresent.

\end{frame}

\begin{frame}
  \frametitle{Example: L4}
  \begin{itemize}
        \item The kernel only contains 3 things:
        \begin{itemize}
                \item Threads
                \item Memory mapping: map, unmap and grant
                \item IPC
        \end{itemize}
        \item Concept of recursive address space, $\sigma{}_0$ 
        \item IPC communication: Chiefs and clans.
  \end{itemize}
\end{frame}
\begin{frame}
  \frametitle{L4 IPCs: Chiefs and clans}
  \begin{center}
  \pgfuseimage{l4ipc} \\
  Chiefs and Clans in L4. \\ Example of communication between to sub entities.
  \end{center}
\end{frame}

\begin{frame}
        \frametitle{Xen the open source hypervisor}
        \begin{itemize}
                \item Xen boots and then start Linux, Solaris or Netbsd dom0.
                \item Dom0 is a module of xen.
                \item Each domain (virtual machine) is a module.
                \item In the future dom0 could be split in different domain/module (stub-domain).
                \item There are communication mechanisms between the different layers (xenstore, ring buffers, \ldots:).
        \end{itemize}
\end{frame}
\begin{frame}
\frametitle{The xen architecture}
\begin{center}
\pgfuseimage{xenarchi}


The xen architecture

\end{center}
\end{frame}

%
% nanokernels
%

\subsection{Nano-kernels}

%
% nanokernels
%

\begin{frame}
  \frametitle{Nano-kernels}

  The only difference with micro-kernels is that more and more services are pushed out of kernel-space. The kernel code is then reduced.

\end{frame}

%
% lse/os
%

\begin{frame}
  \frametitle{Example: LSE/OS}

  LSE/OS has a nano-kernel conception to keep the code as tiny as possible, preventing bugs into the kernel leading to global crashed of the system. Only the core is running in kernel-land.

  \begin{center}
    \pgfuseimage{lseos}
  \end{center}

  LSE/OS pushes out of the kernel the timer and interrupt controller modules and offers minimal services for task and memory management.

\end{frame}

%
% exokernels
%

\subsection{Exo-kernels}

%
% exokernels
%

\begin{frame}
  \frametitle{Exo-kernels}

  Exo-kernels are a very young class of kernel, still under research effort. Principles:

  \begin{itemize}
  \item
    A few abstractions as possible
  \item
    Abstractions are libraries
  \item
    Build other abstractions on existing one
  \item
    Programs use directly these abstractions for performances
  \end{itemize}

  \-

  No commercial operating systems are based on exo-kernel.

  \-

  Pros:

  \begin{itemize}
  \item
    Fully customizable, by adding and removing unnecessary libraries
  \item
    Good performances: function calls like in monolithic kernels
  \item
    Clear design: modules are libraries
  \end{itemize}

\end{frame}

%
% hexo
%

\begin{frame}
  \frametitle{Example: HEXO}

  HEXO is a massively parallel heterogeneous multiprocessor exo-kernel.

  \begin{center}
    \pgfuseimage{hexo}
  \end{center}

  HEXO is build over two abstractions: CPU-specific and
  Platform-specific. Higher-level abstractions can be written easily over HEXO's primitives.

\end{frame}

%
% hybrid kernels
%

\subsection{Hybrid kernels}

%
% hybrid kernels
%

\begin{frame}
  \frametitle{Hybrid kernels}

  The hybrid kernels rely on all the previously studied kernel models.

  \begin{itemize}
  \item
    Structure similar to micro-kernels, with services
  \item
    Most of the code is running in kernel-land to improve performance, like in monolithic kernels
  \end{itemize}

  Many people consider this class of kernels as marketing argument, because there is no great innovation.

  \-

  Pros:

  \begin{itemize}
  \item
    All in kernel-space: less IPCs, better performances
  \item
    Micro-kernel design: clear and easy to maintain
  \end{itemize}

  \-

  Cons:

  \begin{itemize}
  \item
    Risks of crashes like with monolithic designs
  \end{itemize}

\end{frame}

%
% windows nt
%

\begin{frame}
  \frametitle{Example: Windows NT}

  \begin{center}
    \pgfuseimage{nt}
  \end{center}

  The NT kernel is a mix of exo-kernels (HAL), micro-kernels (services) and monolithic kernels (everything in kernel-land). On Windows NT, it is fun to notice that even the GUI services are part of the kernel.

\end{frame}

%
% specific kernels
%

\subsection{Specific kernels}

%
% specific kernels
%

\begin{frame}
  \frametitle{Specific kernels}

  Specific kernels are dedicated to specific domains and applications.

  \begin{itemize}
  \item
    Some of the classical functionalities are not provided
  \item
    Important drivers are directly included into the kernel
  \item
    No portability
  \item
    Reduced set of system calls
  \item
    Non-standard API (specific API)
  \end{itemize}

  \-

  For example, the operating system of a washing machine does not need virtual memory, scheduler or file systems.

\end{frame}

%
% k
%

\begin{frame}
  \frametitle{Example: Kastor}

  Kastor is a specific kernel providing a reduced set of functionality. Kastor is intended to run small games.

  \begin{center}
    \pgfuseimage{k}
  \end{center}

  Kastor does not provide some classical functionalities such as virtual memory or process management, but is provides some main functions for its specific domain like video or sound drivers.

\end{frame}

\section{Conclusion}
\begin{frame}
  \frametitle{Conclusion}
        There is a lot of different type of kernel out there.

        \-

        A lot of systems need their own specific design especially in real-time or embedded system.

        \-

        The future of the kernel and computing in general turns out to be cloud computing and virtualisation.
        There is still lot of effort that needs to be made to achieve those goals of: distribution, flexibility and compatibility.
\end{frame}

\end{document}

