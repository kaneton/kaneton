%
% ---------- header -----------------------------------------------------------
%
% project       kaneton
%
% license       kaneton
%
% file          /home/mycure/kane...view/lecture/kernels/devices/devices.tex
%
% created       julien quintard   [wed may 16 19:28:59 2007]
% updated       julien quintard   [wed apr 22 11:11:34 2009]
%

%
% ---------- setup ------------------------------------------------------------
%


%
% title
%

\title{Devices}

%
% document
%

\begin{document}

%
% title frame
%

\begin{frame}
  \titlepage
\end{frame}

%
% outline frame
%

\begin{frame}
  \frametitle{Outline}

  \tableofcontents
\end{frame}

  % -)

\section{Hardware memory access}
\begin{frame}
  \frametitle{Minimal components and buses}
  \begin{center}
    \includegraphics[width=150pt,height=150pt]{\figurepath/figures/arch-basic}
  \end{center}
\end{frame}


\subsection{Memory accesses}
\begin{frame}
  \frametitle{Memory accesses}
  \begin{center}
    \begin{overlayarea}{10cm}{10cm}
      \only<1>{\includegraphics[width=240pt,height=200pt]{\figurepath/figures/memory-access-step1}}
      \only<2>{\includegraphics[width=240pt,height=200pt]{\figurepath/figures/memory-access-step2}}
      \only<3>{\includegraphics[width=240pt,height=200pt]{\figurepath/figures/memory-access-step3}}
      \only<4>{\includegraphics[width=240pt,height=200pt]{\figurepath/figures/memory-access-step4}}
    \end{overlayarea}
  \end{center}
\end{frame}

% -)

\subsection{Memory controller}
\begin{frame}
  \frametitle{Memory controller}
  \begin{center}
    \only<1>{\includegraphics[width=300pt,height=110pt]{\figurepath/figures/memory-controller}}
  \end{center}
\end{frame}

% -)

\subsection{Misaligned Memory access}
\begin{frame}
\frametitle{Misaligned memory access}
  \begin{itemize}
    \item Memory access are not every time aligned.
    \item For instance "movl (\$0xABCD57), \%eax".
    \item Some architecture doesn't support unaligned memory access, the
    processor raise an exception (Sparc).
    \item Hopefully (or not) x86 supports unaligned memory access.
  \end{itemize}
\end{frame}

\begin{frame}
  \frametitle{Misaligned memory access}
  \begin{center}
\begin{overlayarea}{10cm}{10cm}
    \only<1>{\includegraphics[width=300pt,height=150pt]{\figurepath/figures/unaligned-step1}}
    \only<2>{\includegraphics[width=300pt,height=150pt]{\figurepath/figures/unaligned-step2}}
    \only<3>{\includegraphics[width=300pt,height=150pt]{\figurepath/figures/unaligned-step3}}
    \only<4>{\includegraphics[width=300pt,height=150pt]{\figurepath/figures/unaligned-step4}}
\end{overlayarea}
  \end{center}
\end{frame}

\section{Devices}

\subsection{What is a device?}
\begin{frame}
\begin{itemize}
  \item A device is a piece of hardware connected on a computer via a
        bus.
  \item It can be a wired bus or over the air, like Bluetooth.
  \item A device can make actions and has different status.
  \item To store all those information device use registers.
  \item Devices need interrupts in order to notify the computer. It's
  for synchronous actions.
\end{itemize}
\end{frame}
\subsection{Register}
\begin{frame}
\frametitle{Devices Registers}
\begin{itemize}
        \item Registers can be access using an address on the device.
        \item Sometimes the device memory is mapped on the computer physical memory, sometime it required to use some I/O instructions.
        \item There is 2 types of registers:
        \begin{itemize}
          \item Actions registers, often write only (but can be RW), writing to those registers do an action.
          \item Status registers, read only. This is often the result of an action.
        \end{itemize}
\end{itemize}
\end{frame}


\section{Different way to get in...}
\subsection{PIO}

\begin{frame}
        \frametitle{PIO}
        \begin{itemize}
        \item Programmed Input Output.
        \item Use instruction in{b,w,l} and out{b,w,l}.
        \item Very slow.
        \item Takes a lot of CPU loads.
        \item The processor cannot do anything else when data are transferring.
        \end{itemize}
\end{frame}

\begin{frame}
\begin{center}
\begin{overlayarea}{10cm}{10cm}
\only<1>{\includegraphics[height=225pt]{\figurepath/figures/ioport_1.pdf}}
\only<2>{\includegraphics[height=225pt]{\figurepath/figures/ioport_2.pdf}}
\only<3>{\includegraphics[height=225pt]{\figurepath/figures/ioport_3.pdf}}
\only<4>{\includegraphics[height=225pt]{\figurepath/figures/ioport_4.pdf}}
\end{overlayarea}
\end{center}
\end{frame}

\subsection{MMIO}
\begin{frame}
        \frametitle{MMIO}
        \begin{itemize}
        \item Memory Mapped Input Output.
        \item It's a physical address but it's not RAM.
        \item Can be mapped using pagination or segmentation.
        \item Can be cached, but be careful.
        \end{itemize}
\end{frame}

\begin{frame}
\begin{center}
\begin{overlayarea}{10cm}{10cm}
\only<1>{\includegraphics[height=150pt]{\figurepath/figures/mmio_1.pdf}}
\only<2>{\includegraphics[height=150pt]{\figurepath/figures/mmio_2.pdf}}
\only<3>{\includegraphics[height=150pt]{\figurepath/figures/mmio_3.pdf}}
\only<4>{\includegraphics[height=150pt]{\figurepath/figures/mmio_4.pdf}}
\only<5>{\includegraphics[height=150pt]{\figurepath/figures/mmio_5.pdf}}
\only<6>{\includegraphics[height=150pt]{\figurepath/figures/mmio_6.pdf}}
\only<7>{\includegraphics[height=150pt]{\figurepath/figures/mmio_7.pdf}}
\only<8>{\includegraphics[height=150pt]{\figurepath/figures/mmio_8.pdf}}
\only<9>{\includegraphics[height=150pt]{\figurepath/figures/mmio_9.pdf}}
\only<10>{\includegraphics[height=150pt]{\figurepath/figures/mmio_10.pdf}}
\end{overlayarea}
\end{center}
\end{frame}

\subsection{DMA}
\begin{frame}
\frametitle{DMA}
        \begin{itemize}
        \item Direct Memory Access
        \item Background transfer from phys memory to phys memory.
        \item An interrupt is raised when the transfer is done.
        \item Modern way to do big transfer.
        \end{itemize}
\end{frame}

\begin{frame}
\begin{center}
\begin{overlayarea}{10cm}{10cm}
\only<1>{\includegraphics[height=150pt,width=302pt]{\figurepath/figures/dma-step1}}
\only<2>{\includegraphics[height=150pt,width=302pt]{\figurepath/figures/dma-step2}}
\only<3>{\includegraphics[height=150pt,width=302pt]{\figurepath/figures/dma-step3}}
\end{overlayarea}
\end{center}
\end{frame}

\section{Driver example, i8042}

\begin{frame}
\frametitle{Description}
\begin{itemize}
\item i8042 is the standard keyboard controller.
\item i8042 got 2 IO ports, 0x60 and 0x64.
\item 0x60 is the data port, 0x64 is the cmd port.
\item i8042 wired on the IRQ 1.
\end{itemize}
\end{frame}

\begin{frame}
\frametitle{Steps}
\begin{enumerate}
\item Received and interrupt.
\item Read status register (cmd).
\item Read data register.
\end{enumerate}
\end{frame}

\begin{frame}[fragile]
\frametitle{Example i8042}
\begin{verbatim}
void irq1_handler(void)
{
    uint8_t st;

    st = inb(0x64);
    if (st & 0x10 || st & 0x20)
       inb(0x60)
    else
      printf("Not a i8042 event\n");
}
\end{verbatim}
\end{frame}


\section{PCI Bus}
\subsection*{Background}
\begin{frame}
\frametitle{Background}
\begin{itemize}
\item PCI, Peripheral Component Interconnect
\item Different version exists:
\begin{itemize}
\item PCI 2.2 32 bits, 33MHz, 133 Mo/s.
\item PCI 2.2 64 bits, 64 MHz, 528 Mo/s.
\item PCI-X 64 bits, 133 MHz, 1066 Mo/s.
\end{itemize}
\item The bandwidth is shared between the devices on the same bus. That
why AGP has been created, to dedicate some bandwidth to the graphic
card.
\item In PCI express the bandwidth is not shared anymore. 
\item PCI is even used for internal device connexion in the southbridge like (network card or sound card).
\end{itemize}
\end{frame}

\subsection{BDF, Bus Device Function}
\begin{frame}[fragile]
\frametitle{Bus Device Function}
This is the naming convention for pci device.

\-

00:12.4:
\begin{itemize}
\item Bus: 00
\item Device: 12
\item Function: 4
\end{itemize}

\-

A same device can expose multiple functions; it's often the case for usb controllers:

\begin{verbatim}
00:1d.0 USB Controller [0c03]:USB UHCI Controller #1 [8086:27c8]
00:1d.1 USB Controller [0c03]:USB UHCI Controller #2 [8086:27c9]
00:1d.2 USB Controller [0c03]:USB UHCI Controller #3 [8086:27ca]
00:1d.3 USB Controller [0c03]:USB UHCI Controller #4 [8086:27cb]
00:1d.7 USB Controller [0c03]:USB2 EHCI Controller [8086:27cc]
\end{verbatim}

\end{frame}

\subsection{Config Space and BAR}
\begin{frame}
\frametitle{Config space and BAR}
In order to configure, enumerate, play with the pci devices. Each device got
there own configuration table, called Config space.

\-

The pci config space exposes the different MMIO region and I/O port that a specific device uses. By default the pci space starts at 0x80000000, but it can be remapped.

\-

The pci BAR (Base address register), is an offset where to find the memory region of the devices. This is where are the registers.

\end{frame}

\begin{frame}
\begin{center}
\includegraphics[height=150pt]{\figurepath/figures/config_space}
\end{center}
\end{frame}

\begin{frame}[fragile]
\frametitle{ATI graphic card exemple}
\begin{verbatim}
01:00.0 0300: 1002:7104 (prog-if 00 [VGA controller])
  Subsystem: 1002:0b32
  Control: I/O+ Mem+ BusMaster+ SpecCycle- MemWINV- VGASnoop- ParErr- Stepping- SERR- FastB2B- DisINTx-
  Status: Cap+ 66MHz- UDF- FastB2B- ParErr- DEVSEL=fast >TAbort- <TAbort- <MAbort- >SERR- <PERR- INTx-
  Latency: 0, Cache Line Size: 64 bytes
  Interrupt: pin A routed to IRQ 16
  Region 0: Memory at d0000000 (64-bit, prefetchable) [size=256M]
  Region 2: Memory at efde0000 (64-bit, non-prefetchable) [size=64K]
  Region 4: I/O ports at dc00 [size=256]
  Expansion ROM at efe00000 [disabled] [size=128K]
\end{verbatim}

\begin{verbatim}
00: 02 10 04 71 07 00 10 00 00 00 00 03 10 00 80 00
10: 0c 00 00 d0 00 00 00 00 04 00 de ef 00 00 00 00
20: 01 dc 00 00 00 00 00 00 00 00 00 00 02 10 32 0b
30: 00 00 e0 ef 50 00 00 00 00 00 00 00 0b 01 00 00
40: 00 00 00 00 00 00 00 00 00 00 00 00 02 10 32 0b
50: 01 58 02 06 00 00 00 00 10 80 01 00 a0 0f 64 08
60: 16 09 00 00 01 0d 00 00 40 00 01 11 00 00 00 00
70: 00 00 00 00 00 00 00 00 00 00 00 00 00 00 00 00
80: 05 00 80 00 00 00 00 00 00 00 00 00 00 00 00 00
90: 00 00 00 00 00 00 00 00 00 00 00 00 00 00 00 00
a0: 00 00 00 00 00 00 00 00 00 00 00 00 00 00 00 00
b0: 00 00 00 00 00 00 00 00 00 00 00 00 00 00 00 00
c0: 00 00 00 00 00 00 00 00 00 00 00 00 00 00 00 00
d0: 00 00 00 00 00 00 00 00 00 00 00 00 00 00 00 00
e0: 00 00 00 00 00 00 00 00 00 00 00 00 00 00 00 00
f0: 00 00 00 00 00 00 00 00 00 00 00 00 00 00 00 00
\end{verbatim}
\end{frame}

\section{USB}
\subsection*{Background}
\begin{frame}
\frametitle{Background}
\begin{itemize}
\item Universal Serial Bus.
\item Support hot swapping since its creation.
\item Goes from 1.5 MBit/s up to 480 MBits (and 4.8 GBit/s in the futur).
\item Nowaday almost every type of devices got a usb version.
\end{itemize}
\end{frame}

\begin{frame}
\frametitle{Normes}
\begin{itemize}
\item USB 1.1 (1998): 2 speeds available, low speed (1.5MBit/s) and
full speed (12 MBit/s).
\item USB 2.0 (2000): High speed (480 MBit/s).
\item USB 3.0 (2008): Super speed (4.8 Gbit/s). Planned to 2010.
\end{itemize}
\end{frame}

\subsection{Protocol}
\begin{frame}
\frametitle{Protocol}
\begin{itemize}
\item The bandwidth is temporarily shared between all the devices on the
same bus.
\item The time is divided in frame and micro-frame.
\item The host control the communication with a token.
\item Each device got a device number. This number is used as a address.
This number is coded under 7 bits, so the maximum number of device on
the same bus would be $2^7$
\end{itemize}
\end{frame}
\subsection{Transfer modes}
\begin{frame}
\frametitle{Transfer modes}
\begin{itemize}
\item Command transfer, initialisation packets.
\item Interrupt transfer, this is not a proper interrupt, but this message insure a low latency, it's used by mouse and keyboard.
\item Isochronous transfer, this is for big transfer in real time, like video and audio stream (webcam).
\item Mass transfer, used for large number of data (mass storage device).
\end{itemize}
\end{frame}

\begin{frame}
\frametitle{Enumeration process}
\begin{itemize}
\item When a new device is plugged in the resistor change.
\item The device use the address 0 at the beginning.
\item Then the device send a bunch of characteristic about his class and about what the device is suppose to do.
\end{itemize}
\end{frame}

\subsection{USB classes}
\begin{frame}
\frametitle{USB classes}
\begin{center}
\includegraphics[width=300pt]{\figurepath/figures/usb_classes}
\end{center}
\end{frame}

\subsection{UHCI}
\begin{frame}
\frametitle{UHCI}
\begin{itemize}
\item This is a spec from Intel for usb controllers.
\item This spec describe the controllers registers.
\end{itemize}
\end{frame}

\subsection{Future}
\begin{frame}
\frametitle{Future}
\begin{itemize}
\item Wireless usb, but it's almost dead since Intel left the consortium.
\item For wired link, USB 3.0 will be the new standards in the next years.
\end{itemize}
\end{frame}


\end{document}


