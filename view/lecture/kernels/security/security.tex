%
% ---------- header -----------------------------------------------------------
%
% project       kaneton
%
% license       kaneton
%
% file          /home/mycure/kane...ew/lecture/kernels/security/security.tex
%
% created       julien quintard   [fri oct 24 17:31:58 2008]
% updated       julien quintard   [wed apr 22 11:14:27 2009]
%

%
% ---------- setup ------------------------------------------------------------
%

%
% path
%

\def\path{../../..}

%
% template
%

%
% ---------- header -----------------------------------------------------------
%
% project       kaneton
%
% license       kaneton
%
% file          /home/mycure/kaneton/view/template/lecture.tex
%
% created       julien quintard   [wed may 16 18:17:26 2007]
% updated       julien quintard   [sun may 18 23:23:40 2008]
%

%
% class
%

\documentclass[8pt]{beamer}

%
% packages
%

\usepackage{pgf,pgfarrows,pgfnodes,pgfautomata,pgfheaps,pgfshade}
\usepackage[T1]{fontenc}
\usepackage{colortbl}
\usepackage{times}
\usepackage{amsmath,amssymb}
\usepackage{graphics}
\usepackage{graphicx}
\usepackage{color}
\usepackage{xcolor}
\usepackage[english]{babel}
\usepackage{enumerate}
\usepackage[latin1]{inputenc}
\usepackage{verbatim}
\usepackage{aeguill}

%
% style
%

\usepackage{beamerthemesplit}
\setbeamercovered{dynamic}

%
% verbatim stuff
%

\definecolor{verbatimcolor}{rgb}{0.00,0.40,0.00}

\makeatletter

\renewcommand{\verbatim@font}
  {\ttfamily\footnotesize\selectfont}

\def\verbatim@processline{
  {\color{verbatimcolor}\the\verbatim@line}\par
}

\makeatother

%
% -
%

\renewcommand{\-}{\vspace{0.4cm}}

%
% date
%

\date{\today}

%
% logos
%

\pgfdeclareimage[interpolate=true,width=34pt,height=18pt]
                {epita}{\path/logo/epita}
\pgfdeclareimage[interpolate=true,width=49pt,height=18pt]
                {upmc}{\path/logo/upmc}
\pgfdeclareimage[interpolate=true,width=25pt,height=18pt]
                {lse}{\path/logo/lse}

\newcommand{\logos}
  {
    \pgfuseimage{epita}
  }

%
% institute
%

\institute
{
  \inst{1} kaneton microkernel project
}

%
% table of contents at the beginning of each section
%

\AtBeginSection[]
{
  \begin{frame}<beamer>
   \frametitle{Outline}
    \tableofcontents[current]
  \end{frame}
}

%
% table of contents at the beginning of each subsection
%

\AtBeginSubsection[]
{
  \begin{frame}<beamer>
   \frametitle{Outline}
    \tableofcontents[current,currentsubsection]
  \end{frame}
}


%
% title
%

\title{Security}

%
% document
%

\begin{document}

%
% title frame
%

\begin{frame}
  \titlepage
\end{frame}

%
% outline frame
%

\begin{frame}
  \frametitle{Outline}

  \tableofcontents
\end{frame}

%
% figures
%

\pgfdeclareimage[interpolate=true,width=180pt]
                {amoeba-capability}
                {figures/amoeba-capability}

\pgfdeclareimage[interpolate=true,height=170pt]
                {amoeba-attenuation}
                {figures/amoeba-attenuation}

%
% ---------- text -------------------------------------------------------------
%

%
% introduction
%

\section{Introduction}

% 1)

\begin{frame}
  \frametitle{Overview}

  This course targets security in operating systems. As such, intra-kernel
  security will not be discussed.

  \-

  This course will focus on describing techniques for controlling the access to
  well-defined objects; these objects being created and managed by the kernel,
  a server, a service or whatever independent entity.
\end{frame}

% 2)

\begin{frame}
  \frametitle{Assumptions}

  Obviously, providing security would not make much sense in an unsafe
  system.

  \-

  The course therefore considers that the studied operating system
  is safe enough so that security makes sense.

  \-

  More precisely, in systems such as critical, real-time \etc{},
  ensuring safety would be far more important than providing access control
  because without strong safety guarantees, security would not have any impact.
\end{frame}

% 3)

\begin{frame}
  \frametitle{Topic}

  Note that none of the following topics will be discussed throughout this
  course as the focus is \term{access control}:

  \-

  \begin{itemize}
    \item
      Unit testing;
    \item
      Model checking;
    \item
      Secure architecture;
    \item
      Vulnerabilities;
    \item
      \etc{}
  \end{itemize}
\end{frame}

%
% overview
%

\section{Overview}

% 1)

\begin{frame}
  \frametitle{Why}

  Controlling access to objects is quite obvious although it can be for
  different reasons: privacy, security, accountability \etc{}
\end{frame}

% 2)

\begin{frame}
  \frametitle{What}

  The type of objects the system provides access control for varies from:

  \begin{itemize}
    \item
      Files;
    \item
      Kernel data;
    \item
      Network accesses;
    \item
      Shared memory;
    \item
      Devices;
    \item
      \etc{}
  \end{itemize}
\end{frame}

%
% terminology
%

\section{Terminology}

% 1)

\begin{frame}
  \frametitle{Introduction}

  This section aims at providing the student the vocabulary related to
  access control, this terminology being the one used in the literature.
\end{frame}

% 2)

\begin{frame}
  \frametitle{Subject/Initiator}

  The \term{subject}/\term{initiator} represents the entity trying to access
  an object.

  \-

  According to the security poclicy and the rights the subject has on the
  object, the requested operation is granted or rejected.
\end{frame}

% 3)

\begin{frame}
  \frametitle{Object/Target}

  The \term{object}/\term{target} represents the information that is kept
  from being seen from unauthorised subjects through a security policy.
\end{frame}

% 4)

\begin{frame}
  \frametitle{Policy}

  The security \term{policy} defines a set a rules on which the
  managers rely to decide whether or not an access is granted.
\end{frame}

% 5)

\begin{frame}
  \frametitle{Manager}

  A \term{manager} represents an entity providing access to targets.

  \-

  Managers follow the rules defined by the security policy in use.

  \-

  In an operating system, the kernel is likely to be the only manager although
  in microkernel-based operating systems, every server may also provide access
  control.
\end{frame}

% 6)

\begin{frame}
  \frametitle{Operation}

  The \term{operation} represents the action requested by a subject on
  a target.

  \-

  The manager receiving the request checks whether the subject has the right
  to perform this operation and acts accordingly.
\end{frame}

% 7)

\begin{frame}
  \frametitle{Example}

  Let's take an example to put the terminology into use although this
  example is not related to operating systems.

  \-

  On \name{Facebook}, users can decide to restrict their profile's access
  being to friends, friends of friends, everyone \etc{}

  \-

  In such a context:

  \begin{itemize}
    \item
      Users are initiators;
    \item
      Browsing your profile is the operation;
    \item
      Your profile is the subject;
    \item
      \name{Facebook}'s website is the manager; and
    \item
      The security policy allows you to define who is authorised to access
      your information
  \end{itemize}
\end{frame}

%
% history
%

\section{History}

% 1)

\begin{frame}
  \frametitle{Overview}

  This section is intended to give a glimpse of history to students.
\end{frame}

% 2)

\begin{frame}
  \frametitle{Multics :: Overview}

  \name{\term{Multics} - Multiplexed Information and Computing Service}---
  started in $1964$ by the \name{MIT}, \name{AT\&T Bell Labs} and
  \name{General Electric}---was first released in $1969$.

  \-

  \name{Multics} was the first system to emphasize security from the
  design perspective.

  \-

  In spite of this, \name{Multics} extreme complexity opened the way to many
  security flaws.

  \-

  However, break-ins became very rare with the second generation of hardware
  that enabled \term{ring-oriented} security.
\end{frame}

% 3)

\begin{frame}
  \frametitle{Multics :: Objectives}

  The goals of \name{Multics} were to provide:

  \begin{itemize}
    \item
      High availability; and
    \item
      Scalability, by adding resources such as main memory, storage capacity
      \etc{};
  \end{itemize}
\end{frame}

% 4)

\begin{frame}
  \frametitle{Multics :: Security}

  \name{Multics} achieved access control through the use of \name{ACL - Access
  Control List}s on every file providing flexible information sharing and
  complete privacy when required.

  \-

  Through such a technique, fine grained access control was provided to users,
  which was quite revolutionary at the time.
\end{frame}

% 5)

\begin{frame}
  \frametitle{CAP Computer :: Overview}

  The Cambridge \term{CAP Computer} was an experimental computer built
  in the $1970$s at the \name{University of Cambridge, Computer Laboratory}.
\end{frame}

% 6)

\begin{frame}
  \frametitle{CAP Computer :: Objectives}

  The objective of this system was to provide complete security.
\end{frame}

% 7)

\begin{frame}
  \frametitle{CAP Computer :: Security}

  The \name{CAP Computer} achieved security through the use of capabilities
  both in hardware and software.
\end{frame}

% 8)

\begin{frame}
  \frametitle{UNIX :: Overview}

  \term{UNIX}, started in $1969$ with \name{UNIX-PDP7}, was developed by
  \name{Ken Thompson} with the help of \name{Dennis Ritchie} as an alternative
  to \name{Multics}.

  \-

  Although the first objective was to make a game run faster, \name{UNIX}
  evolved to a complete operating system and was rewritten in \name{C}
  in $1973$.
\end{frame}

% 9)

\begin{frame}
  \frametitle{UNIX :: Objectives}

  The goals of \name{UNIX} were to provide:

  \begin{itemize}
    \item
      Portability;
    \item
      Multi-tasking;
    \item
      Multi-user interaction; and
    \item
      Time sharing.
  \end{itemize}
\end{frame}

% 10)

\begin{frame}
  \frametitle{UNIX :: Security}

  The \name{UNIX} operating system introduced the \name{``Everything is file''}
  philosophy.

  \-

  \name{UNIX} also introduced a new access control scheme consisting of a
  set of permissions for three types of entities: \name{user} which represents
  the file owner, \name{group} which specifies the group the file belongs
  to and \name{other} which represents everybody else.

  \-

  In addition, every set of permission is composed of three bits \name{rwx}
  for \name{read}, \name{write} and \name{execution}.

  \-

  This scheme provided a perfect trade-off between the granularity on one
  side and the size taken by access control metadata on the other side.
\end{frame}

% 11)

\begin{frame}
  \frametitle{Linux}

  \term{Linux} started in $1991$ by \name{Linus Torvalds} mainly based on
  the \name{MINIX} operating system.

  \-

  \name{Linux} being a clone of \name{UNIX}, the access control scheme
  remains based on the \name{UNIX} permissions.
\end{frame}

% 12)

\begin{frame}
  \frametitle{Windows}

  \term{Windows} did not provide any access control for the first versions
  of its operating system mainly because there was no need for such control
  since these versions of the operating system were mono-user.

  \-

  Starting with \name{Windows NT} \ie{} \name{Windows XP}, \name{Windows 2000}
  \etc{}, every file was given an \name{Access Control List} manageable by
  its owner.
\end{frame}

%
% principles
%

\section{Principles}

% 1)

\begin{frame}
  \frametitle{Overview}

  In this section, we define a few security principles:

  \begin{enumerate}
    \item
      \term{Security Through Obscurity/Transparency};
    \item
      \term{Security Through Design}; and
    \item
      \term{Least Privilege}.
  \end{enumerate}
\end{frame}

% 2)

\begin{frame}
  \frametitle{Security Through Obscurity/Transparency :: Bad/Good}

  Security through obscurity is a principle which attempts to use secrecy of
  design, implementation \etc{} to provide or enhance security.

  \-

  A system relying on security through obscurity may have theoretical or
  actual security vulnerabilities, but its owners or designers believe that
  the flaws are not known, and that attackers are unlikely to find them.

  \-

  Keeping details of the system secret does not make it more robust, but
  indeed increases the effort an attacker has to put in order to find such
  security flaws.

  \-

  The related security through transparency philosophy suggests that security
  flaws should be disclosed as soon as possible.
\end{frame}

% 3)

\begin{frame}
  \frametitle{Security Through Obscurity/Transparency :: Not too Bad/Good}

  Assuming that perfect or ``unbroken'' solutions may be difficult to obtain
  and that relying solely on security through obscurity is a very poor design
  choice, keeping secret some of the details of an otherwise well-engineered
  system may be a reasonable tactic of defense.

  \-

  For example, the goal might be to simply reduce the short-run risk of
  exploitation of a vulnerability in the main components of the system while
  a security fix is implemented.

  \-

  A variant, especially in computer networks, is to detect attackers trying
  to identify potential security flaws. Since this stage is necessary, the
  system can use it to detect and block attackers.
\end{frame}

% 4)

\begin{frame}
  \frametitle{Security Through Obscurity/Transparency :: Conclusion}

  All in all, there is a general consensus that security through obscurity
  should never be used as a primary security measure.

  \-

  It is, at best, a secondary measure; and disclosure of the obscurity should
  not result in a compromise.

  \-

  In contrary, domains such as cryptogragraphy rely on the
  \name{full disclosure} philosophy so that, given everything but the
  cryptographic key, the system remains secure.

  \-

  In conclusion, although the techniques stand in contrast with each other,
  many real-world projects include elements of both strategies.
\end{frame}

% 5)

\begin{frame}
  \frametitle{Security By Design}

  Security by design means that the software has been designed from the ground
  up to be secure.

  \-

  Malicious practices are taken for granted and care is taken to minimize
  impact when a security vulnerability is discoverd.

  \-

  The most common security design practice when it comes to user inputs is
  to avoid buffer overflows and format string vulnerabilities, among others.
\end{frame}

% 6)

\begin{frame}
  \frametitle{Least Privilege :: Overview}

  The principle of least privilege requires that every module such as a
  process, a user or a program must be able to access only the information
  and resources that are necessary to its purpose.

  \-

  For example, in most operating systems such as \name{Linux} and
  \name{Windows}, the file system code, running in \name{kernel mode}, has
  maximum privileges; therefore there is no security enforcement.

  \-

  The same goes for code running in \name{kernel mode} when it comes to
  accessing the hardware. The \name{CAP Computer} was the first system
  to introduce access control at the hardware level.

  \-

  For example, a user $U$ with \name{read/write} access on a file
  $F$ violates the least privilege principle if $U$ can complete his
  tasks with only the \name{read} permission.
\end{frame}

% 7)

\begin{frame}
  \frametitle{Least Privilege :: Mitigation}

  Although the principle of least privilege is recognised as being a
  good design principle, true least privilege is, in practice, complicated
  to enforce.

  \-

  Indeed, there is no method to evaluate a process to define the least
  amount of privilege it will ever need.

  \-

  Therefore, in practice, the best that can be done is to restrict the
  privileges to eliminate some we casually predict that it will never need.

  \-

  For example, we know that \name{Notepad} should never access the network.

  \-

  In conclusion, this principle is theoretically very interesting but
  difficult to put into practice.
\end{frame}

%
% categories
%

\section{Categories}

% 1)

\begin{frame}
  \frametitle{Overview}

  In this section, we briefly define the two categories of access control,
  namely:

  \begin{itemize}
    \item
      \name{\term{MAC} - Mandatory Access Control}; and
    \item
      \name{\term{DAC} - Discretionary Access Control}.
  \end{itemize}
\end{frame}

% 2)

\begin{frame}
  \frametitle{MAC}

  In military and other high-security computer systems, a policy of
  \name{\term{MAC} - Mandatory Access Control} is used for restricting the
  dissemination of information

  \-

  Objects, such as files, are assigned \term{security labels} that restrict who
  is allowed to access them.

  \-

  Users of the system are assigned \term{clearance levels} such that a user
  with a clearance level of classified may read documents that are classified
  and unclassified but not secret or top-secret. Interestingly, the same user
  may actually create secret or top-secret documents without being able to
  access them subsequently.

  \-

  Therefore, with \name{MAC}, the user does not have control over her own
  documents. The system mandates who can read and write everything with the
  objective to limit the dissemination of information.

  \-

  Finally, to demote information from secret to classified, for instance,
  special privileges are required.
\end{frame}

% 3)

\begin{frame}
  \frametitle{DAC}

  Commercial operating systems like \name{UNIX} and \name{Windows} take a
  totally different approach using \name{\term{DAC} - Discretionary Access
  Control} by putting control of an object into the hands of the entity
  that created it.

  \-

  Following this philosophy, the object's owner is implicity granted
  permissions to read, write but also to administrate the permissions on
  this file.

  \-

  Although many operating systems relying on \name{DAC} implement the concept
  of user one way or another, the standard does not say anything about
  owners making this notion theoretically distinguishable from \name{DAC}.
\end{frame}

% 4)

\begin{frame}
  \frametitle{Conclusion}

  Although \name{MAC}-enabled systems allow policy administrators to
  implement organization-wide security policies, systems can implement both
  \name{MAC} and \name{DAC} simultaneously.

  \-

  In such a system for example, \name{DAC} would refer to one category of
  access control where subjects can transfer permissions to one another.

  \-

  One the other side, \name{MAC} would refer to a second category of access
  control that impose constraints upon the first.

  \-

  As an example, \name{kaneton} provides both categories. Indeed, the
  capabilities enable tasks to implement a \name{DAC} policy while the kernel
  implements the \name{MAC} policy in order to give access to devices, mapped
  memory regions \etc{} to precise tasks \eg{} the \name{VGA} mapped-memory to
  the \name{VGA} driver \etc{}
\end{frame}

%
% models
%

\section{Models}

% 1)

\begin{frame}
  \frametitle{Overview}

  In this section, the different models of access control are presented,
  detailed and illustrated by a few examples.
\end{frame}

% 2)

\begin{frame}
  \frametitle{Owner}

  The most basic access control model must be the one used by a system where
  the object owner/creator only has the right to access it.

  \-

  Obviously, such a scheme is not much of interest as it lacks everything.
\end{frame}

% 3)

\begin{frame}
  \frametitle{UNIX Permissions :: Overview}

  The \name{UNIX} permissions where introduced with the \name{UNIX}
  operating system.

  \-

  At that time, computers had little main memory and little storage capacity.
  As such, designing the file system's access control was not as easy as
  thought.

  \-

  The solution picked was, at the time, a very good trade-off between
  expressivity and complexity especially in terms of storage capacity.

  \-

  Indeed, while a user cannot choose explicitely which users can access one of
  her file and with which permissions, the \name{UNIX} scheme relies on the
  concept that a file belongs both to a user, its owner---probably its
  creator---and a group.
\end{frame}

% 4)

\begin{frame}[containsverbatim]
  \frametitle{UNIX Permissions :: Details}

  As such, each file has, basically, $9$ bits of permissions, three sets of
  $3$ bits \name{rwx} for the owner, group and others, meaning users that are
  not the owner and that are not member of the group.

  \-

  \begin{center}
    \begin{tabular}{ccc}
      \textbf{u}ser & \textbf{g}roup & \textbf{o}ther \\
      \verb|r w x| & \verb|r w x| & \verb|r w x| \\
    \end{tabular}
  \end{center}

  \-

  Note that, in \name{UNIX}, the system administrator is the only user allowed
  to create groups, making difficult for users to organise people by
  themselves.

  \-

  Furthermore, the group space is \term{flat} meaning that there is no
  hierarchy \ie{} a group cannot be a member of another group. This detail
  also complicates organising users in large systems.

  \-

  In conclusion, this scheme may lack flexibility especially when it comes to
  managing fine grain accesses but also for large systems where organising
  users becomes difficult.
\end{frame}

% 5)

\begin{frame}
  \frametitle{ACL :: Overview}

  An \name{\term{ACL} - Access Control List} is a list of permissions
  attached to an object.

  \-

  The list specifies who or what is allowed to access the object and what
  operations are permitted. Typically, each \name{ACL} entry specifies a
  subject and a list of allowed operations.
\end{frame}

% 6)

\begin{frame}
  \frametitle{ACL :: Limitation}

  The question of how access control lists are edited depends on the system.

  \-

  Systems said to have discretionary access control let the creator or owner
  of the object manage the list of permissions while systems with mandatory
  access control enforce system-wide restrictions, overriding the permissions
  stated in the \name{ACL}s.

  \-

  \name{ACL}s suffer from the same limitation than \name{UNIX} permissions.
  Indeed, in the context of large systems with many users, it can become
  cumbersome to assign permissions to every individual.
\end{frame}

% 7)

\begin{frame}
  \frametitle{ACL :: Example}

  The \name{NTFS - NT File System} provides \name{ACL}s with extended
  operations such as editing, listing a directory \etc{} in addition to the
  well-known read, write and execute.

  \-

  Most of the \name{UNIX}-like operating systems such as \name{Linux},
  \name{Solaris} \etc{} supported the so-called \name{POSIX.1e} standard.
  However, since it has been abandoned, many operating systems now implement
  the \name{ACL} scheme provided by \name{NFS - Network File System} \name{v4}.
\end{frame}

% 8)

\begin{frame}
  \frametitle{Confused Deputy Problem :: Overview}

  Let's consider a service waiting for two arguments: the name of an input
  file and the name of an output file.

  \-

  The service provided consists in compiling the first file, generating the
  second one.

  \-

  Assuming that the service has the permission to read and write the
  \location{/etc/passwd} file while the user does not; what would happen
  if this file name was given as output?

  \-

  The service would process the given input file and overwrite the
  \location{/etc/passwd} file which the user did not have the permission
  to write.
\end{frame}

% 9)

\begin{frame}
  \frametitle{Confused Deputy Problem :: Explanation}

  When the file name was passed from the client to the server, the permission
  did not go along with it.

  \-

  Therefore, the permission was increased by the system silently and
  automatically.

  \-

  This is a way to perform a \term{privilege escalation} attack.

  \-

  This kind of attacks is well-known on \name{UNIX}-like operating systems
  via \name{setuid} executables.

  \-

  Note that in that case, both \name{UNIX} permissions and \name{ACL}s
  suffer from this problem.

  \-

  The solution would be for the service to check if the user has the right to
  access the given files but such a solution would require explicit attention
  to security by the server and most services would not take that extra
  security measure.
\end{frame}

% 10)

\begin{frame}
  \frametitle{Capabilities :: Overview}

  A \term{capability} is a communicable, unforgeable token of authority.

  \-

  A capability references an object along with an associated set of access
  rights.

  \-

  Therefore, processes willing to perform an action, instead of identifying
  the object, pass the capability as an argument. Thus, the receiving process
  will verify the validity of this capability before extracting both
  the object identifier and the access rights.

  \-

  Capabilities are also interesting for providing the holder the possibility
  to \term{attenuate} a capability \ie{} to restrict a given capability's
  permissions, creating a new capability that could then be delegated and/or
  used.
\end{frame}

% 11)

\begin{frame}
  \frametitle{Capabilities :: Least Privilege}

  Capability-based system refer to the principle of sharing capability with
  each other according to the principle of least privilege.

  \-

  Indeed, in the example of the confused deputy problem, the client would
  provide the service two capabilities: one for the input file and the other
  one for the output file.

  \-

  Thus, the service would not have any way of overwriting the
  \location{/etc/passwd} unless the output capability references this file;
  which is unlikely as we assumed the client did not have any permission
  on this file.

  \-

  In conclusion, we can clearly see that the service, by using the received
  capabilities, cannot by tricked in performing illegal operations.
\end{frame}

% 12)

\begin{frame}
  \frametitle{Capabilities :: Resemblance}

  Although most operating systems implement a facility which resembles
  capabilities, they typically do not provide enough support to allow
  for exchange of capabilities among possibly mutually untrusting entities.

  \-

  In contrast, a capability-based system is designed with that goal in mind.

  \-

  For example, the \name{POSIX 1e/2c} capabilities are coarse-grained
  privileges that cannot be transferred between processes; hence are not
  capabilities in the strict sense given by the underlying concept.
\end{frame}

% 13)

\begin{frame}
  \frametitle{Capabilities :: Examples}

  Capabilities have been implemented in many systems, even languages.

  \-

  The secure distributed computing language \term{E} was
  designed\footnote{The capability concept is closely related to the notion
  of object-oriented programming.} with capabilities in mind.

  \-

  Research operating systems such as the historical suite \term{KeyKOS},
  \term{EROS}, \term{CapROS} and \term{Coyotos} all used capabilities from
  the ground up.

  \-

  The \name{IBM}'s operating system \term{System 250} integrated the notion
  of capability addressing. Processes needed a capability to access a region
  of memory. Unfortunately, such a technique could drastically slow down the
  system depending on the hardware, making the system less portable.

  \-

  Another example is \name{Google} \term{Caja} which allows websites to
  exchange information with a fine-grain access. Imagine a user drag-dropping
  her \name{GMail} contact list on \name{Facebook}, creating a capability
  so that the \name{Facebook} website can access these contacts on the
  \name{GMail} website without interfering with anything else on the user's
  \name{GMail} account.
\end{frame}

% 14)

\begin{frame}
  \frametitle{Capabilities :: Amoeba :: Overview}

  Like \name{kaneton}, the distributed operating system \name{Amoeba}, designed
  by \name{Pr.} \name{Andrew Tanenbaum} is based on the concept of
  capabilities.

  \-

  \begin{center}
    \pgfuseimage{amoeba-capability}
  \end{center}

  \-

  When an object is created, the server picks a random \name{Check} field and
  stores it both in the new capability and inside its own table. Besides,
  all the \name{Rights} bits are initially on, and it is this \name{owner
  capability} that is returned to the client.

  \-
  
  Then, when the capability is sent back to the server in a request to perform
  an operation, the \name{Check} field is verified by the process explained
  next.
\end{frame}

% 15)

\begin{frame}
  \frametitle{Capabilities :: Amoeba :: Attenuation}

  To create a restricted capability, a client pass its capability back to the
  server, along with a bit mask for the new rights.

  \-

  The server takes the original---randomly picked---\name{Check} field from
  its table and \textit{EXCLUSIVE OR}s it with the new rights and finally
  runs the result through a one-way function.

  \-

  Note that the new rights must obviously be a subset of the given
  capability's rights.

  \-

  Once the new capability created, the process may send it to another process,
  if it wishes.

  \-

  Finally, note that the meaning of the \name{Rights} field depends on the
  nature of the object.

  \-

  The verification process consists in retreiving the original \name{Check}
  field, \textit{EXCLUSIVE OR}ing it with the \name{Rights} field, then
  through the one-way function and checking if it produces the capability's
  \name{Check} field.
\end{frame}

% 16)

\begin{frame}
  \frametitle{Capabilities :: Amoeba :: Attenuation}

  \begin{center}
    \pgfuseimage{amoeba-attenuation}
  \end{center}
\end{frame}

% 17)

\begin{frame}
  \frametitle{Capabilities :: Conclusion}

  The \name{kaneton} microkernel adopted a different approach. Indeed, the
  format of capabilities is not statically defined such that every server
  manages its capabilities its own way with its own specific format.

  \-

  Unfortunately, capabilities are not really popular in common operating
  systems.

  \-

  One reason might be that such operating systems were adapted from time
  to time to provide extended security mechanisms.

  \-

  Since modifying an existing system is difficult, such operating systems
  tend to include \name{ACL}s as it does not require a complete review of the
  system design.
\end{frame}

% 18)

\begin{frame}
  \frametitle{Certificates :: Overview}

  Certification is another way to provide access control.

  \-

  We will see two types of certificates:

  \begin{itemize}
    \item
      \name{\term{IC} - Identity Certificate};
    \item
      \name{\term{AC} - Authorization Certificate}.
  \end{itemize}

  \-

  Note that a certificate typically encapsulates a validity period as well
  as, obviously, the digital signature.
\end{frame}

% 19)

\begin{frame}
  \frametitle{Certificates :: Identity Certificate}

  \name{Identity Certificate}s also known as \name{\term{PKC} - Public Key
  Certificate}s are digitally signed certificates that encapsulate the
  identity of the certificate holder \ie{} caller.

  \-

  This type of certificate is well-used in distributed systems including
  on the \name{Internet} since it allows server to identify the client that
  is requesting an operation.

  \-

  Note that the access permissions are usually included in the certificate
  but could be stored elsewhere, in an \name{Access Control List} for instance.

  \-

  The holder identity can be referenced through a name but also via a
  public key.
\end{frame}

% 20)

\begin{frame}
  \frametitle{Certificates :: Authorization Certificate}

  In contrary, an \name{Authorization Certificate} is a digital document
  that describes permissions to access a service.

  \-

  Interestingly, this certificate can be delegated since it does not
  identify the holder.

  \-

  Such certificates are similar to capabilities as they store the permissions
  to access an object and a reference to the object, not the actual object
  itself.
\end{frame}

% 21)

\begin{frame}
  \frametitle{Certificates :: Singularity}

  \term{Singularity} is a microkernel-based research operating system designed
  and implemented by \name{\term{MSR} - Microsoft Research}.

  \-

  \name{Singularity} differs from all other operating system kernels in the
  way that all the processes, including the kernel, share the same address
  space.

  \-

  Indeed, considering that address space switching, more specifically flushing
  translation caches, is what takes time when it comes to context switching;
  \name{Microsoft} researchers decided to remove this isolation mechanism
  and rather let all processes live in the same space.

  \-

  Since no isolation is provided, the system must ensure that processes will
  not access unauthorised memory regions. To guarantee that, \name{Singularity}
  relies on software isolation.

  \-

  Indeed, static analysis is used for ensuring that invariants are verified.
  In order to prove such components behaviours, a new language was introduced:
  \name{Sing\#}.
\end{frame}

% 22)

\begin{frame}
  \frametitle{Certificates :: Singularity}

  However, how the kernel, given a binary object, could know if the code
  has been verified.

  \-

  Indeed, if the code has not been, it could tamper the whole memory; hence
  take complete control over the system.

  \-

  Although this has not been explicitely detailed by the authors,
  a way of achieving this might be to rely on certification.

  \-

  Imagine that the kernel would reject running any binary that has not been
  attached a properly signed certificate. Such a certificate could be signed
  by a \name{\term{CA} - Certificate Authority} run by \name{Microsoft},
  directly by the system administrator, or through a local compiling service.
\end{frame}

% 23)

\begin{frame}
  \frametitle{Certificate :: Conclusion}

  Certification basically provides the same kind of expressivity that
  capabilities.

  \-

  The difference lies in the computation time and space required by
  certification.

  \-

  Indeed, issuing and verifying a digital signature takes far more time
  than running a one-way function.

  \-

  In addition, public keys must be very large compared to the hash resulting
  from the one-way function.

  \-

  One difference in favour of certification is that the system does not
  need to keep anything in its own internal tables but the its key pair used
  for issuing certificates.

  \-

  All in all, capabilities are probably more suited to operating systems
  because of their small size while certification are more likely to be used
  in distributed systems.
\end{frame}

%
% further
%

\section{Further}

% 1)

\begin{frame}
  \frametitle{RBAC :: Overview}

  \name{\term{RBAC} - Role-Based Access Control} is a newer alternative
  approach to \name{Mandatory Access Control} and \name{Discretionary
  Access Control}.

  \-

  \name{RBAC} is a neutral and flexible access control model sufficiently
  powerful to simulate \name{DAC} and \name{MAC}.

  \-

  Prior to the development of \name{RBAC}, \name{MAC} and \name{DAC} were
  considered to be the only known models for access control. Research in the
  late $1990$s demonstrated that \name{RBAC} falls in neither category.
\end{frame}

% 2)

\begin{frame}
  \frametitle{RBAC :: Details}

  First, services need to classify clients into named roles. For instance,
  a hospital would classify users into \name{surgeon}, \name{doctor},
  \name{nurse} and \name{patient}.

  \-

  Then, access permissions are assigned to roles rather that directly to
  principals.

  \-

  Therefore, managing individual rights becomes a matter of simply assigning
  the appropriate roles to the user, which is quite straightforward in
  an organization where administrator already perform tasks such as adding
  a user, changing a user's department \etc{}

  \-

  All entities have to do is to authenticate so that they prove they belong
  to a role.

  \-

  Note that obviously, a subject can have multiple roles, a role can have
  multiple subjects, a role can have multiple permissions and a permission
  can be assigned to different roles.
\end{frame}

% 3)

\begin{frame}
  \frametitle{RBAC :: Conclusion}

  \name{RBAC} is especially interesting as it eases administration since
  roles change far less frequently than individuals join or leave
  organisations.

  \-

  Authentication can be done through certification or by using capabilities.
  Indeed, \name{RBAC} information such as the role and possibly some
  parameters could be included in a capability.

  \-

  Examples of security solutions implementing \name{RBAC} include
  \name{SELinux}, \name{Microsoft Active Directory}, \name{grsecurity} \etc{}
\end{frame}

%
% conclusion
%

\section{Conclusion}

% 1)

\begin{frame}
  \frametitle{Conclusion}

  Although some access control models are more attractive than others,
  designers should always remember that choosing the right solution always
  depends on the system's constraints.

  \-

  Therefore, relying on \name{UNIX} permissions might still be an acceptable
  solution for certain systems while certification might be too expensive
  to implement on other systems.
\end{frame}

%
% bibliography
%

\begin{frame}
  \frametitle{Bibliography}

  \bibliographystyle{amsplain}
  \bibliography{\path/bibliography/bibliography}
\end{frame}

\end{document}

Exam: repertoires apache/ subversion/ et trac/. chacun contient des sous-rep
  pour les users. on veut restreindre l'acces des repertoires users aux autres
  users *mais* on veut que: (i) le rep apache d'un user soit accessible
  en lecture par le user apache (ii) le rep subversion d'un user soit
  accessible en lecture/ecriture par le user subversion et en lecture par
  le user apache (iii) le repertoire trac d'un user soit accessible en
  lecture/ecriture par le user apache.
  de plus bien sur chaque user doit pouvoir acceder son propre repertoire
  personnel de chaque repertoire apache, subversion et trac et les autres
  doivent en etre exclus.
  proposez une solution, la plus simple possible qui fournisse tout cela.
  cet exemple est tres concret!

Exam: disons qu'on veut un systeme avec des capabilities mais l'on ne veut
  pas que les processes puisse se refiler des capabilities pour les utiliser.
  en d'autre termes on veut que seuls les gens qui ont crees des objets
  puissent utiliser leurs cap mais pas un autre process qui utilise une cap
  qu'il n'a pas cree -> principal-specific capability (on ajoute juste l'id
  de la tache dans la cap et on hash ca avec les rights)
  de plus ca protege du vol. par contre ca suppose que l'on peut detecter
  l'appelant (dans le cas d'un OS c'est bon mais ce ne serait pas forcement
  le cas dans un systeme reparti).
