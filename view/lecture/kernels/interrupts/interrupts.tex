%
% ---------- header -----------------------------------------------------------
%
% project       kaneton
%
% license       kaneton
%
% file          /home/mycure/kane...ecture/kernels/interrupts/interrupts.tex
%
% created       julien quintard   [wed may 16 19:17:31 2007]
% updated       julien quintard   [wed may 16 19:17:32 2007]
%

\pgfdeclareimage[interpolate=true,width=180pt]
                {stack}
                {figures/stack}

\pgfdeclareimage[interpolate=true,width=140pt]
                {rings}
                {figures/rings}

\pgfdeclareimage[interpolate=true,width=190pt]
                {isr_kernel}
                {figures/isr_kernel}

\pgfdeclareimage[interpolate=true,width=190pt]
                {isr_user}
                {figures/isr_user}

\pgfdeclareimage[interpolate=true,width=300pt]
                {event}
                {figures/event}

\pgfdeclareimage[interpolate=true,width=230pt]
                {ivt}
                {figures/ivt}

\pgfdeclareimage[interpolate=true,width=270pt]
                {pic}
                {figures/pic}

\pgfdeclareimage[interpolate=true,width=220pt]
                {m68kivt}
                {figures/m68kivt}

\pgfdeclareimage[interpolate=true,width=310pt]
                {nested}
                {figures/nested}

\section{The execution stack}

%
%
%

\begin{frame}
  \frametitle{The execution stack}

  The stack is a LIFO structure, mainly used to:
  \begin{itemize}
    \item save the instruction to return to after a subroutine call
    \item pass parameters to a subroutine
    \item allocate space for subroutines local variables
  \end{itemize}

  \nl

  On more specific occasions, the stack can also be used to backup an
  environment state.

\end{frame}

%
%
%

\begin{frame}
\frametitle{The stack's structure}

  The stack is composed of frames: everytime a subroutine is called, a new
  frame is pushed. Also the frame for the executing subroutine is always on
  the top of the stack. When the subroutine returns, this frame is poped.

  \nl

  Thus, the current stack state is described by the contents of 2 registers:
  \begin{enumerate}
    \item the {\em stack pointer} contains the address of the top of the stack
    \item the {\em base pointer} contains the base address of the current stack
      frame
  \end{enumerate}

  \begin{center}
    \pgfuseimage{stack}
  \end{center}

\end{frame}

%
%
%

\begin{frame}
  \frametitle{Execution stack \& multithreading}

  Tasks cannot share their execution stack:

  \begin{itemize}
    \item {\bf For safety reasons}\\
      The execution stack contains private data for the running task.
      Corrupting those data results in corrupting the whole task's behavior.

      \nl

    \item {\bf For practical reasons}\\
      Let us consider the following scenario:

      \begin{enumerate}
        \item {\em task1} pushes {\em data1}
	\item scheduler transfers execution from {\em task1} to {\em task2}
	\item {\em task2} pushes {\em data2}
	\item scheduler transfers execution from {\em task2} to {\em task1}
	\item {\em task1} pops {\em data2}
      \end{enumerate}

      \nl

      At step 5, {\em task1} expects to pop data1 it pushed at step 1. But if
      {\em task1} and {\em task2} both share the same execution stack,
      {\em task1} will actually pop {\em data2} that {\em task2} pushed at step
      3. This results in a double issue:

      \begin{itemize}
	\item {\em task1} will continue its execution with a corrupted value.
	\item {\em task1} should not be able to access {\em task2}'s private data.
      \end{itemize}

  \end{itemize}

\end{frame}

\section{Privilege levels}

%
%
%

\begin{frame}
  \frametitle{Privilege rings}

  \begin{center}
    \pgfuseimage{rings}
  \end{center}

  \nl

  {\bf Fundamental rule:}\\
  An application cannot increase its own privileges.

\end{frame}

%
%
%

\begin{frame}
  \frametitle{Privilege levels}

  \begin{enumerate}
  \item {\bf Kernelland}\\
    This privilege level is often called {\em supervisor mode} by microprocessor
    fabricants. In modern operating systems, this ring is dedicated to the kernel.

    \nl

  \item {\bf Userland}\\
    Also called {\em user mode}, this level is opposed to Kernelland since it
    imposes privilege restrictions. Userland is dedicated to user programs
    which should not modify the system or other programs behavior. It is
    characterized by the following restrictions:

    \begin{itemize}
      \item restricted instruction set
      \item restricted memory access
      \item restricted I/O access
    \end{itemize}
  \end{enumerate}

  \nl

  Other intermediate levels can be attributed to special applications such as
  drivers or trusted operating system services.

  \nl

  In an operating system which uses more than 2 privilege levels, Userland is
  considered as the less privileged one.

\end{frame}

\section{Events}

%
%
%

\begin{frame}
  \frametitle{Overview}

  Interrupts were introduced to replace polling which is based on active
  loops. Active loops are very performant when they handle lots of events, but
  waste the CPU time when they are waiting for a condition that does not
  happen.

  \nl

  Interrupts make it possible to make a process sleep until an event wakes it
  up. Thus microprocessor cycles are saved to serve other processes. On certain
  embedded architectures, interrupts are also a easy way to save energy.

  \nl

  Interrupts are a critical part of code in the kernels implementation:

  \begin{itemize}
    \item The code is non-portable accros compilers and platforms
    \item the code is difficult to debug
    \item kernel performances highly depends on interrupt handling implementation
  \end{itemize}

\end{frame}


%
%
%

\begin{frame}
  \frametitle{Definition of an event}

  {\bf Definition:}\\
  Events indicate to the processor that something happened in the system
  and requires kernel attention. They typically result in a forced transfer of
  execution from the currently running task to a special kernel routine  called
  an {\em interrupt service routine} (ISR) or an {\em interrupt handler}.\\

  \nl

  Events are categorized in 3 classes:

  \begin{itemize}
    \item {\bf Exceptions}\\
      Raised by the processor itself when detecting an internal error.\\
      Example: {\em divide by zero}, {\em page fault}, {\em invalide opcode}\ldots
    \item {\bf Software interrupts (or trap)}\\
      Caused by the running program when executing a special ASM instruction.
    \item {\bf Hardware interrupts (or IRQ)}\\
      Provoked by external hardware devices.
  \end{itemize}

  \nl

  From the microprocessor point of view, exceptions and software interrupts are
  seen as internal interrupts, whereas IRQ are treated as external interrupts.

\end{frame}

%
%
%

\begin{frame}
  \frametitle{The Interrupt Service Routine (ISR)}

  The {\em Interrupt Service Routine} (ISR) is a kernel procedure which is
  called to handle an event. The kernel must provide a specific ISR for each
  event which may occur.

   \nl

   Each ISR is characterized by:

  \begin{enumerate}
    \item the interrupt vector to which it is associated
    \item the privilege level which is required to invoke this ISR.
  \end{enumerate}

  \nl


  ISR are kernel routines (running with kernel's privileges) that can be
  executed on Userland programs needs, when they raise an interrupt.

  \nl

  Some ISR are famous:

  \begin{itemize}
    \item Exceptions ISR: page fault handler
    \item Software Interrupts ISR: system calls
    \item IRQ ISR: drivers I/O routines
  \end{itemize}

\end{frame}

%
%
%

\begin{frame}
  \frametitle{Events \& privilege level switching}

  \begin{enumerate}
    \item ISR interrupts a Kernelland routine:
      \begin{center}
        \pgfuseimage{isr_kernel}
      \end{center}

    \item ISR interrupts an Userland routine:
      \begin{center}
        \pgfuseimage{isr_user}
      \end{center}
  \end{enumerate}

\end{frame}

%
%
%

\begin{frame}
  \frametitle{Events \& privilege level switching}

  An event interrupts the currently running program and invokes the ISR
  execution. As an ISR is always executed in Kernelland, two cases may happen:

  \begin{enumerate}
    \item {\bf ISR interrupts a Kernelland routine}
    The processor is already in {\em superisor mode} when the execution is
    transfered to the ISR.

    \item {\bf ISR interrupts an Userland routine}
    The processor is operating in {\em user mode} when it detects the event.
    In such a situation, the microprocessor behaves like this:

    \begin{itemize}
      \item switch from {\em user mode} to {\em supervisor mode}
      \item execute the ISR in {\em supervisor mode}
      \item switch from {\em supervisor mode} to {\em user mode}
    \end{itemize}

  \end{enumerate}

  \nl

  {\bf Questions for both cases:}

  \begin{itemize}
    \item How does the processor find the ISR for a given interrupt ?
    \item How does the processor resume the interrupted routine ?
  \end{itemize}

\end{frame}

%
%
%

\begin{frame}
  \frametitle{The interrupt vector}

  Every event in the system is given its own identifier called an
  {\bf interrupt vector}. The {\em interrupt vector} specifies the source of
  the interrupt, and therefore, it can be used to determine which ISR should
  be executed.

  \nl

  In the case of an exception or a software interrupt, the microprocessor
  deduces the {\em interrupt vector} internally.

  \nl

  Concerning the IRQ,
  Some embedded microprocessors provide a set of interrupt lines, to which
  hardware devices are directly wired. Such microprocessors associate an
  {\em interrupt vector} for each IRQ line.

  \nl

  Other microprocessors do not provide any unit to handle multiple IRQ lines.
  Those microprocessors communicate with an external chip called an
  {\bf Interrupt Controller}.


\end{frame}

%
%
%

\begin{frame}
  \frametitle{The interrupt controller}

  The Interrupt Controller multiplexes the hardware devices interrupts requests
  towards the single microprocessor IRQ line. The {\em interrupt vector} is
  transfered via the data bus:

  \nl

  \begin{center}
    \pgfuseimage{pic}
  \end{center}

\end{frame}

%
%
%

\begin{frame}
  \frametitle{Finding the ISR}

  An {\em interrupt vector} is loaded in a microprocessor internal register
  everytime an event is raised. Depending on wether this register is visible
  by the software, the kernel handles events 2 different manners:

  \nl

  \begin{enumerate}
    \item If the {\em interrupt vector register} is not visible\\
      The kernel has no way to determine the source of the event. Also, the
      kernel is unable to chose the corresponding ISR. Therefore, the ISR
      dispatch must be assumed by the microprocessor itself.

      \nl

    \item If the {\em interrupt vector register} is visible\\
      The microprocessor loads its PC with the base address of a generic ISR.
      Then, the kernel can load the {\em interrupt vector} from the
      microprocessor register, and easily select the ISR to handle the event.
      The dispatch is made by the kernel.
  \end{enumerate}

\end{frame}


%
%
%

\begin{frame}
  \frametitle{Interrupt Vector Table}

  On architectures such as IA-32, event identification is done internally. That
  means that the microprocessor does not provide a way to tell the kernel which
  {\em interrupt vector} has been detected. Instead, the microprocessor
  determines itself which ISR must be executed:

  \nl

  \begin{enumerate}
    \item the microprocessor gets the {\em interrupt vector}
    \item the microprocessor determines the corresponding ISR
    \item the microprocessor loads the ISR's first instruction into the
      {\em Programm Counter}
    \item the ISR handles the event
  \end{enumerate}

  \nl

  That implies that the microprocessor can find the addresse of each ISR. To do
  so, the microprocessor consults an {\bf interrupt vector table}, whose base
  address must be stored in a dedicated register.

\end{frame}

%
%
%

\begin{frame}
  \frametitle{Interrupt Vector Table}

  \begin{center}
    \pgfuseimage{ivt}
  \end{center}

\end{frame}

%
%
%

\begin{frame}
  \frametitle{the Cause register}

  On other architectures, like MIPS, the microprocessor delegates the event
  handling to the kernel:

  \nl

  \begin{itemize}
    \item the microprocessor stores the {\em interrupt vector} in the
    {\em Cause} register
    \item the microprocessor loads the {\em Program Counter} with the
      address of the generic handler's first instruction
    \item the generic handler retrieves the {\em interrupt vector} from
      the {\em Cause} register
    \item the generic handler execute the appropriated ISR
  \end{itemize}

  \nl

  This implies that the {\em Cause} register is visible from the kernel.
\end{frame}


%
%
%

\begin{frame}
  \frametitle{}

  SCHEMA SANS IVT (MIPS)

\end{frame}

%
%
%

\begin{frame}
  \frametitle{Returning from an interrupt}

  resuming the interrupted program (or not)


\end{frame}

%
%
%

\begin{frame}
  \frametitle{The execution context}

  The {\bf execution context} is the state of the program execution at a given
  time.

  \nl

  Concretly, an image of the {\em execution context} can be obtained by saving
  the state of:

  \begin{itemize}
    \item the stack
    \item the memory mapping
    \item microprocessor registers
  \end{itemize}

  \nl

  Loading an {\em execution context} from an image, back to the microprocessor
  registers, makes the program resume from where the context image was first
  saved.

  \nl

  Thus it becomes possible to interrupt a program, backup its {\em execution
  context}, execute an ISR and resume the interrupted program in a transparent
  manner by simply restoring its last {\em execution context}.

\end{frame}

%
%
%

\begin{frame}
  \frametitle{Backing up the context}

  {\bf Q:} Where is saved the context ?\\
  {\bf A:} The easiest way to backup the context is to push it on the stack.

  \nl

  {\bf Q;} Why not in memory ?\\
  {\bf A:} Allocation times would be too long. Allocation algorithms may
  corrupt the context before it was actually saved !

  \nl

  {\bf Q:} Why on the stack ?\\
  {\bf A:} In the case of nested interrupts, a LIFO structure is naturally
  appropriate.

  \nl

  {\bf Q:} Whose stack ?\\
  {\bf A:} The {\em execution context} should be saved and restored by whoever is
  interrupting the program. So on the ISR's stack.

  \nl

  {\bf Q:} When should the context be saved ?\\
  {\bf A:} As soon as the program is interrupted to ensure that no other
  operation corrupts the {\em execution context} before it is saved. In the same
  way, no operation should be executed between the context restoration and the
  return from interrupt.

\end{frame}

%
%
%

\begin{frame}
  \frametitle{Event mechanism summary}

  \pgfuseimage{event}

\end{frame}

%
%
%

\begin{frame}
  \frametitle{Event mechanism summary}

  The previous figure illustrates what happens when a program is interrupted:

  \nl

  \begin{enumerate}
    \item The microprocessor performs a privilege switch from User to Supervisor
      mode:
      \begin{itemize}
        \item Current stack switches from the program (user) stack to the
	  interrupt (kernel) stack.
        \item The {\em Program Counter} and the {\em Program Status Word} are
	 saved on the kernel stack.
      \end{itemize}
    \item The event handler saves the task's execution context on the kernel
      stack.
    \item The event handler execute the appropriate ISR.
    \item The event handler restores the task's execution context from the
      kernel stack.
    \item The event handler provokes a {\em Return From Interrupt}. Therefore,
      the microprocessor switches back to User mode:
    \begin{itemize}
      \item The last {\em Program Status Word} and the {\em Program Counter}
        of the interrupted program are restored with the values saved on the
	stack at step 1.
      \item Current stack switches back to the interrupted program's one.
    \end{itemize}
  \end{enumerate}

  \nl

  Steps 1 and 5 are automatically performed by the microprocessor. Some
  microprocessors also handle steps 2 and 4.

\end{frame}

%
%
%

\begin{frame}
  \frametitle{Exercise}

  Motorola M68000 specifications:

  \begin{itemize}
    \item Operating modes
      \begin{itemize}
        \item User
	\item Supervisor
      \end{itemize}

    \item Registers:
      \begin{itemize}
        \item {\em D0}-{\em D7}: data registers
        \item {\em A0}-{\em A7}: address registers (A7 is the stack pointer)
        \item {\em PC}: program counter
        \item {\em CCR}
      \end{itemize}

    \item ISR dispatch is organized through an interrupt vector located at
      address 0x0. This 1024-bytes table looks like the one below:

      \nl
      \begin{center}
      \pgfuseimage{m68kivt}
      \end{center}
  \end{itemize}

\end{frame}

%
%
%

\begin{frame}
  \frametitle{Exercise}

  \nl

  When interrupted, the M68000 automatically saves the {\em CCR} and the
  {\em PC} onto the Supervisor stack {\em A7'}. The return from interrupt is
  assumed by the {\tt RTI} instruction.

  \nl

  Write the code which could permit to 

\end{frame}

%
%
%

\begin{frame}[containsverbatim]
  \frametitle{Solution}

  \begin{verbatim}
event_init:
       moveq.l  #1, d0                  ; arg1 = 1 (interrupt id)
       lea      handler_1, a0           ; arg2 = handler_1 (interrupt handler)
       jsr      init_handler            ; call init_handler() subroutine
       rts

isr_1:
       ...                              ; ISR treatment
       rts

init_isr:
       movem.l  d0/a1, -(a7)            ; save a1 on the stack
       lsl.l    #2, d0                  ; a1 <- a1 * 4
       move.l   d0, a1                  ; 
       move.l   a0, 8(a1)               ; fill the interrupt vector entry with the handler address
       movem.l  (a7)+,d0/a1             ; restore a1 from the stack
       rts

handler_1:
       movem.l  d0-d7/aO-a6, -(a7)      ; save the execution context on the stack
       jsr      isr_1                   ; call the Interrupt Service Routine
       movem.l  (a7)+, dO-d7/a0-a6      ; restore the execution context from the stack
       rti                              ; return from interrupt
  \end{verbatim}

\end{frame}

%
%
%

\begin{frame}
  \frametitle{Stack size}

  \begin{quote}
    With experience, one learns the standard, scientific way to compute the
    proper size for a stack: Pick a size at random and hope.

    \begin{flushright}
     Jack Ganssle
    \end{flushright}
  \end{quote}


  \nl

  \begin{itemize}

    \item {\bf Testing}\\
      Required stack size can easily be approximated by testing the stack needs
      in real conditions. Testing often underestimates the real needs.

    \nl

    \item {\bf Analyzis}\\
      On the opposite, analyzis approach is much more complex and often
      overestimates the real needs.
  \end{itemize}

\end{frame}

%
%
%

\begin{frame}
  \frametitle{Latency}

  Interrupt latency is the amount of time elapsed:

  \begin{itemize}
  \item From the moment the microprocessor detects an interrupt, until the
    execution of the ISR's first instruction.
  \item From the moment the ISR has finished executing, until the interrupted
    porgram is finally resumed.
  \end{itemize}

  \nl

  During these times, neither the microprocessor nor the kernel do actually
  handle the event. They just install a safe environment to run the ISR and to
  resume the interrupted program. In other words, latency is the amount of
  {\bf lost time} when an interrupt is handled.

  \nl

  Part of the latency is due to inevitable microprocessor internal operations
  (such as privilege/stack switches, privilege checking\ldots). But the rest of
  the latency is caused by the event handler and must be minimalized by the
  programmer.

\end{frame}

%
%
%

\begin{frame}
  \frametitle{Overload}

  Interrupt handling overloads the system when:

  \begin{enumerate}
    \item latency is too strong / interrupt handling is too slow
    \item too many interrupts occur
  \end{enumerate}

  \nl

  In both cases, a system overload leads to serious issues:

  \begin{itemize}
    \item missing interrupts
    \item starving user programs
    \item making interrupt handling durations unpredictable for real-time systems
  \end{itemize}

  \nl

  reducing overload:
  \begin{itemize}
    \item optimizing interrupt handlers and ISR
    \item limitating the events rate (using DMA or buffering)
    \item using polling
  \end{itemize}

\end{frame}

%
%
%

\begin{frame}
  \frametitle{Nested events}

  Some processors support nested interrupts. An ISR that interrupts a program
  could itself be interrupted by the ISR for a higher-priority interrupt:
  \begin{center}
    \pgfuseimage{nested}
  \end{center}

  \nl

  {\bf Rule:} an ISR cannot be interrupted by an event for a lower or equal
  pritority level.

\end{frame}

%
%
%

\begin{frame}
  \frametitle{Nested events \& real-time}

  Real-time systems commonly implement nested events.

  \nl

  {\bf Advantages}

  \begin{itemize}
    \item Make the system more reactive, especially for high-priority
      interrupts.
    \item Allow to use events in the ISR.
  \end{itemize}

  \nl

  {\bf Disadvantages}

  \begin{itemize}
    \item Consumes larger amounts of stack.
    \item Considering an IRQ which needs to be handled in a time T, what
      would happen if the ISR for this interrupt were itself interrupted by an
      event of a higher priority ?
  \end{itemize}

\end{frame}
