%
% ---------- header -----------------------------------------------------------
%
% project       kaneton
%
% license       kaneton
%
% file          /home/mycure/kane.../kernels/prerequisites/prerequisites.tex
%
% created       julien quintard   [wed may 16 19:28:59 2007]
% updated       julien quintard   [wed may 16 19:29:16 2007]
%

%
% ---------- setup ------------------------------------------------------------
%

%
% path
%

\def\path{../../../..}

%
% template
%

%
% ---------- header -----------------------------------------------------------
%
% project       kaneton
%
% license       kaneton
%
% file          /home/mycure/kaneton/view/template/lecture.tex
%
% created       julien quintard   [wed may 16 18:17:26 2007]
% updated       julien quintard   [sun may 18 23:23:40 2008]
%

%
% class
%

\documentclass[8pt]{beamer}

%
% packages
%

\usepackage{pgf,pgfarrows,pgfnodes,pgfautomata,pgfheaps,pgfshade}
\usepackage[T1]{fontenc}
\usepackage{colortbl}
\usepackage{times}
\usepackage{amsmath,amssymb}
\usepackage{graphics}
\usepackage{graphicx}
\usepackage{color}
\usepackage{xcolor}
\usepackage[english]{babel}
\usepackage{enumerate}
\usepackage[latin1]{inputenc}
\usepackage{verbatim}
\usepackage{aeguill}

%
% style
%

\usepackage{beamerthemesplit}
\setbeamercovered{dynamic}

%
% verbatim stuff
%

\definecolor{verbatimcolor}{rgb}{0.00,0.40,0.00}

\makeatletter

\renewcommand{\verbatim@font}
  {\ttfamily\footnotesize\selectfont}

\def\verbatim@processline{
  {\color{verbatimcolor}\the\verbatim@line}\par
}

\makeatother

%
% -
%

\renewcommand{\-}{\vspace{0.4cm}}

%
% date
%

\date{\today}

%
% logos
%

\pgfdeclareimage[interpolate=true,width=34pt,height=18pt]
                {epita}{\path/logo/epita}
\pgfdeclareimage[interpolate=true,width=49pt,height=18pt]
                {upmc}{\path/logo/upmc}
\pgfdeclareimage[interpolate=true,width=25pt,height=18pt]
                {lse}{\path/logo/lse}

\newcommand{\logos}
  {
    \pgfuseimage{epita}
  }

%
% institute
%

\institute
{
  \inst{1} kaneton microkernel project
}

%
% table of contents at the beginning of each section
%

\AtBeginSection[]
{
  \begin{frame}<beamer>
   \frametitle{Outline}
    \tableofcontents[current]
  \end{frame}
}

%
% table of contents at the beginning of each subsection
%

\AtBeginSubsection[]
{
  \begin{frame}<beamer>
   \frametitle{Outline}
    \tableofcontents[current,currentsubsection]
  \end{frame}
}


%
% title
%

\title{Kernel Generalities}

%
% document
%

\begin{document}

%
% title frame
%

\begin{frame}
  \titlepage
\end{frame}

%
% figures
%

% bsd

\pgfdeclareimage[interpolate=true,width=157pt,height=120pt]
                {bsd}
		{figures/bsd}

% kaneton

\pgfdeclareimage[interpolate=true,width=145pt,height=120pt]
                {kaneton}
		{figures/kaneton}

% lseos

\pgfdeclareimage[interpolate=true,width=174pt,height=120pt]
                {lseos}
		{figures/lseos}

% hexo

\pgfdeclareimage[interpolate=true,width=176pt,height=120pt]
                {hexo}
		{figures/hexo}

% nt

\pgfdeclareimage[interpolate=true,width=194pt,height=120pt]
                {nt}
		{figures/nt}

% k

\pgfdeclareimage[interpolate=true,width=151pt,height=120pt]
                {k}
		{figures/k}

\pgfdeclareimage{l4ipc}
                {figures/l4ipc}

\pgfdeclareimage{xenarchi}
                {figures/xenarchi}

\tableofcontents

%
% roles of a kernel
%

\section{Quick History}

\begin{frame}
\frametitle{Quick history}

        \begin{itemize}
                \item 60s - IBM - First main frame
                \item 70s - Unix - Server operating system 
                \item 80s - Microsoft  - Personnal compuer (PC)
                \item 90s - Linux / Minix / GNU - Open source
                \item 2000 - VmWare / Goggle - Virtualisation / Distribution 
        \end{itemize}

\end{frame}

\begin{frame}
\frametitle{Evolution ...}
The way to use computer has changed through the year.

  \-

The kernel and operating systems as to adapt over the years.

  \-

Kernel went from massively multi-user to mono-user and then end up begin
a mix of that with virtualisation and distrubution.
\end{frame}

\section{Introduction}

% -)

\begin{frame}
  \frametitle{Goals}

  The kernel is the central part of an operating system. The kernel is
  the entity offering abstraction of the hardware to the applications,
  including:

  \begin{itemize}
  \item
    Memory management
  \item
    Process management
  \item
    I/Os and events (both leading to drivers)
  \item
    Inter-Process Communication
  \end{itemize}

  In addition, the kernel must ensure the security of the resources.

  \-

  The final goal of a kernel is to run user programs, to enable them
  to interact with hardware and between each others, what constitues
  an operating system.

\end{frame}

% -)

\begin{frame}
  \frametitle{A good kernel is\ldots}

  \begin{itemize}
  \item
    Fast. Performances are critical.
  \item
    Reliable. When a kernel crashes, the whole machine does.
  \item
    Maintainable. Other developers must be able to write their own
    drivers or services.
  \item
    Fault tolerant. Either in terms of hardware faults and software
    faults.
  \item
    Secure.
  \item
    Portable. To work on different hardware with less possible porting
    effort.
  \item
    Distributed ? Some says a kernel has to handle distribution and looks
    like a single instance even if it's running on thousand of machines.
  \end{itemize}

\end{frame}

% -)

\begin{frame}
  \frametitle{Kernel-space \& User-land}

  Kernel-space is the execution environment of the kernel. All
  operations are permitted. Hardware can be accessed and
  controlled. Microprocessor internal stuctures and control registers
  too. The kernel can access all the tasks address spaces.

  \-

  User-land is the environment of classical programs, with many
  restrictions. Hardware access is forbidden. Configuration facilities
  are denied. A program can only access its own address space.

  \-

  The more code you have in kernel-space, the more risks of crashes you
  have.

\end{frame}

%
% kernel choisses 
%

\section{Kernel choisses}

\section{Mono/Multi user}
\begin{frame}
  \frametitle{Mono/Multi user}
  \begin{itemize}
        \item Usally this problem comes allong with the multi-tasking.
        If we decide to go for the multi-tasking we might want to as
        multi-user support to provide security.
        \item Some systems don't need multi-user/multi-tasking. They are
        ofen use for embedded/Real-Time application.
        \item Multi-user support involve to thing about a system to
        protect data from one user against another. It could be
        ACL/capabalities/\ldots
  \end{itemize}
\end{frame}


\subsection{MM: Segmentation/Pagination}
\begin{frame}
  \frametitle{MM: Segmentation/Pagination}
  \begin{itemize}
        \item Segmentation/Pagination are generally used to improved
        security and flexibility.
        \item The walking throw the page tables takes a bit of time and
        it's not deterministic. Therefor it can't be used of hard
        real-time.
        \item To support the pagination you have to have a MMU on your
        platform.
  \end{itemize}
\end{frame}

\subsection{Scheduling Offline/Online}
\begin{frame}
  \frametitle{Scheduling: Offline/Online}
        \begin{itemize}
        \item Online:
        \begin{itemize}
                \item The online scheduling is the usal way to schedule
                nowadays. The scheduler is dynamic and alloce the
                timeslices on demand.
                \item It's much more flexible than the scheduling
                offline.
                \item The scheduler itself consumes some CPU time.
        \end{itemize}
        \item Offline:
        \begin{itemize}
                \item The order in which the tasks will be executed is
                pre-computed.
                \item You need a limited number of taks.
                \item It's not dymaic.
                \item Fill perfectly in a lockdown system (real-time
                again?)
        \end{itemize}
        \end{itemize}
\end{frame}

%
% monolithic kernels
%

\section{Kernel design}
\subsection{Monolithic kernels}

%
% monolithic kernels
%

\begin{frame}
  \frametitle{Monolithic kernels}

  Monolithic kernels includes everything into the kernel, even
  drivers, filesystems, networking\ldots

  \-

  \emph{This approach well known as ``The Big Mess''. The structure is
  that there is no structure.} -- Tanenbaum

  \-

  Pros:

  \begin{itemize}
  \item
    High performances: only function calls
  \end{itemize}

  \-

  Cons:

  \begin{itemize}
  \item
    Dangerous: one bug in a non-critical service may lead the system
    to crash
  \item
    Not easy to understand and maintain
  \end{itemize}

\end{frame}

%
% bsd
%

\begin{frame}
  \frametitle{Example: BSD}

  4.4BSD kernel include all functionnalities in kernel-land: from
    process management to network layer, through filesystems.

  \begin{center}
    \pgfuseimage{bsd}
  \end{center}

  The kernel is about 200.000 lines long.

\end{frame}

%
% microkernels
%

\subsection{Micro-kernels}

%
% microkernels
%

\begin{frame}
  \frametitle{Micro-kernels}

  In micro-kernels, only the critical functionnalities are running in
  kernels-space. The other services are running as user programs
  (having extended privileges).

  \-

  Pros:

  \begin{itemize}
  \item
    Small amount of code in kernel-space: less bugs and risks of
    crashes
  \item
    Clearest design, easy to understand
  \item
    Services can be started and stopped: reduce system load, run
    concurrent services\ldots
  \end{itemize}

  \-

  Cons:

  \begin{itemize}
  \item
    Lots of IPC: slower performances
  \end{itemize}

\end{frame}

%
% kaneton
%

\begin{frame}
  \frametitle{Example: kaneton}

  kaneton provide a dozen of critical managers: memory, process,
  I/O\ldots{} Only these functionnalities are running in kernel-land.

  \begin{center}
    \pgfuseimage{kaneton}
  \end{center}

  Advanced functionnalities (filesystems, network\ldots) are provided
  by services in userland. IPC are omnipresent.

\end{frame}

\begin{frame}
  \frametitle{Example: L4}
  \begin{itemize}
        \item The kernel only contains 3 things:
        \begin{itemize}
                \item Threads
                \item Memory mapping: map, unmap and grant
                \item IPC
        \end{itemize}
        \item Concept of recursive address space, $\sigma{}_0$ 
        \item IPC communication: Chiefs and clan.
  \end{itemize}
\end{frame}
\begin{frame}
  \frametitle{L4 IPCs: Chiefs and clans}
  \begin{center}
  \pgfuseimage{l4ipc}
  Chiefs and Clans in L4. 
  \end{center}
\end{frame}

\begin{frame}
        \frametitle{Xen the open source hypervisor}
        \begin{itemize}
                \item Xen boots and then start linux/solaris/netbsd dom0.
                \item Dom0 is a module of the kernel xen.
                \item Each domain (virtual machine) is a module.
                \item In the futur dom0 could be split in different
                volume (stub-domain).
                \item There is communication mechanisme between the
                different layer (xenstore, ring buffers).
        \end{itemize}
\end{frame}
\begin{frame}
\frametitle{The xen architecture}
\begin{center}
\pgfuseimage{xenarchi}


The xen architecture

\end{center}
\end{frame}

%
% nanokernels
%

\subsection{Nano-kernels}

%
% nanokernels
%

\begin{frame}
  \frametitle{Nano-kernels}

  The only difference with micro-kernels is that more and more
  services are pushed out of kernel-space. The kernel code is then
  reduced.

\end{frame}

%
% lse/os
%

\begin{frame}
  \frametitle{Example: LSE/OS}

  LSE/OS has a nano-kernel conception to keep the code as tiny as
  possible, preventing bugs into the kernel leading to global crashed
  of the system. Only the core is running in kernel-land.

  \begin{center}
    \pgfuseimage{lseos}
  \end{center}

  LSE/OS pushes out of the kernel the timer and interrupt controller
  modules and offers minimal services for task and memory management.

\end{frame}

%
% exokernels
%

\subsection{Exo-kernels}

%
% exokernels
%

\begin{frame}
  \frametitle{Exo-kernels}

  Exo-kernels is a very young class of kernel, still under research
  effort. Principles:

  \begin{itemize}
  \item
    A few abstraction as possible
  \item
    Abstractions are libraries
  \item
    Build other abstractions on existing one
  \item
    Programs use directly these abstractions for performances
  \end{itemize}

  \-

  No commercial operating systems are based on exo-kernel.

  \-

  Pros:

  \begin{itemize}
  \item
    Fully customizable, by adding and removing unnecessary libraries
  \item
    Good performances: function calls like in monolithic kernels
  \item
    Clear design: modules are libraries
  \end{itemize}

\end{frame}

%
% hexo
%

\begin{frame}
  \frametitle{Example: HEXO}

  HEXO is a massively parallel heterogeneous multiprocessor
  exo-kernel.

  \begin{center}
    \pgfuseimage{hexo}
  \end{center}

  HEXO is build over two abstractions : CPU-specific and
  Platform-specific. Higher level abstractions can be wrote easily
  over HEXO's primitives.

\end{frame}

%
% hybrid kernels
%

\subsection{Hybrid kernels}

%
% hybrid kernels
%

\begin{frame}
  \frametitle{Hybrid kernels}

  The hybrid kernels relies on all the previously studied kernel
  models.

  \begin{itemize}
  \item
    Structure similar to micro-kernels, with services
  \item
    Most of the code is running in kernel-land to improve performance,
    like in monolithic kernels
  \end{itemize}

  This class of kernels is considered as ``marketing argument'' by
  many people, because there is no great innovation.

  \-

  Pros:

  \begin{itemize}
  \item
    All in kernel-space: less IPCs, better performances
  \item
    Micro-kernel design: clear and easy to maintain
  \end{itemize}

  \-

  Cons:

  \begin{itemize}
  \item
    Risks of crashes like with monolithic designs
  \end{itemize}

\end{frame}

%
% windows nt
%

\begin{frame}
  \frametitle{Example: Windows NT}

  \begin{center}
    \pgfuseimage{nt}
  \end{center}

  The NT kernel is a mix of exo-kernels (HAL), micro-kernels
  (services) and monolithic kernels (everything in kernel-land). On
  Windows NT, it is fun to notice that even the GUI services are part
  of the kernel.

\end{frame}

%
% specific kernels
%

\subsection{Specific kernels}

%
% specific kernels
%

\begin{frame}
  \frametitle{Specific kernels}

  Specific kernels are dedicated to specific domains and applications.

  \begin{itemize}
  \item
    Some of the classical functionnalities are not provided
  \item
    Important drivers are directly included into the kernel
  \item
    No portability
  \item
    Reduced set of system calls
  \item
    Non-standard API (specific API)
  \end{itemize}

  \-

  For example, the operating system of a washing machine does not need
  virtual memory, scheduler or filesystems.

\end{frame}

%
% k
%

\begin{frame}
  \frametitle{Example: K}

  K is a specific kernel providing a reduced set of functionnality. K
  is intended to run small games.

  \begin{center}
    \pgfuseimage{k}
  \end{center}

  K does not provide some classical functionnalities such as virtual
  memory or process management, but is provides some main functions
  for its specific domain like video or sound drivers.

\end{frame}

\end{document}
