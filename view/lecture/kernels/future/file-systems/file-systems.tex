%
% ---------- header -----------------------------------------------------------
%
% project       kaneton
%
% license       kaneton
%
% file          /home/mycure/kane...els/future/file-systems/file-systems.tex
%
% created       julien quintard   [tue jan 13 05:55:34 2009]
% updated       julien quintard   [tue apr  7 00:09:28 2009]
%

%
% ---------- setup ------------------------------------------------------------
%

%
% path
%

\def\path{../../../..}

%
% template
%

%%
%% copyright     (c) julien quintard
%%
%% project       kaneton
%%
%% file          /home/mycure/kaneton/view/templates/lecture.tex
%%
%% created       julien quintard   [sat nov 19 17:13:03 2005]
%% updated       julien quintard   [fri dec  2 22:36:34 2005]
%%

%
% class
%

\documentclass[8pt]{beamer}

%
% packages
%

\usepackage{pgf,pgfarrows,pgfnodes,pgfautomata,pgfheaps,pgfshade}
\usepackage{colortbl}
\usepackage{times}
\usepackage{amsmath,amssymb}
\usepackage{graphics}
\usepackage{graphicx}
\usepackage{color}
\usepackage{xcolor}
\usepackage[english]{babel}
\usepackage{enumerate}
\usepackage[latin1]{inputenc}

%
% style
%

\usepackage{beamerthemesplit}
\setbeamercovered{dynamic}

%
% verbatim font
%

\definecolor{verbatimcolor}{rgb}{0,0.4,0}

\makeatletter
\renewcommand{\verbatim@font}
  {\ttfamily\footnotesize\color{verbatimcolor}\selectfont}
\makeatother

%
% new line
%

\newcommand{\nl}[0]{\vspace{0.4cm}}

%
% date
%

\date{\today}

%
% logos
%

\pgfdeclareimage[interpolate=true,width=34pt,height=18pt]
                {epita}{../../logos/epita}
\pgfdeclareimage[interpolate=true,width=49pt,height=18pt]
                {upmc}{../../logos/upmc}
\pgfdeclareimage[interpolate=true,width=25pt,height=18pt]
                {lse}{../../logos/lse}

\newcommand{\logos}
  {
    \pgfuseimage{epita}
  }

%
% institute
%

\institute
{
  \inst{1} kaneton microkernel project
}

%
% table of contents at the beginning of each section
%

\AtBeginSection[]
{
  \begin{frame}<beamer>
   \frametitle{Outline}
    \tableofcontents[current]
  \end{frame}
}

%
% table of contents at the beginning of each subsection
%

\AtBeginSubsection[]
{
  \begin{frame}<beamer>
   \frametitle{Outline}
    \tableofcontents[current,currentsubsection]
  \end{frame}
}


%
% title
%

\title{File Systems}

%
% document
%

\begin{document}

%
% title frame
%

\begin{frame}
  \titlepage
\end{frame}

%
% outline frame
%

\begin{frame}
  \frametitle{Outline}

  \tableofcontents
\end{frame}

%
% ---------- text -------------------------------------------------------------
%

%
% introduction
%

\section{Introduction}

% 1)

\begin{frame}
  \frametitle{Overview}

  This course is intended to present file systems through their history,
  the principles, the access control and so forth.
\end{frame}

% 2)

\begin{frame}
  \frametitle{Need}

  Both operating systems and users need to store data in a permanent way
  \ie{} so that data can be retrieve after the computer has been shut down.
\end{frame}

% 3)

\begin{frame}
  \frametitle{Equipment}

  Storage devices were early integrated to computers to provide a way
  to store information starting with punched cards but more recently with
  hard drives, floppies and so on.
\end{frame}

% 4)

\begin{frame}
  \frametitle{Software}

  Providing a storage capacity such as a hard disk does not suffice. Indeed,
  software must be provided to manage the space provided and possibly organise
  data, in other words, abstract the hardware into something meaningful to
  users and applications.

  \-

  As such, many systems rely on the \textbf{file} abstraction, the most popular
  being \textit{UNIX} because of its motto \textit{``Everything is file''}.
\end{frame}

XXX

definition: used to store, catalog and retrieve information from a storage
  stable or not

abstraction: information is abstracted into files i.e data gathered logically
  together

objective: keep track of the files stored, including the data but also info
  on the file itself: file name, etc.. besides the storage capacity needs to
  be manage to find space etc.

%
% dinosaurs
%

\section{Dinosaurs}

% dectape

\subsection{DECTape}

% 1)

\begin{frame}
  \frametitle{Overview}

  In the $60s$ early days of computing, computers of the size of refrigerators,
  hence far smaller than room-wide maiframes, used tapes for storing
  data in a permanent way.

  \-

  At this age of computing, every computer came with its own operating system
  and each operating system had its way for accessing the permanently stored
  data.

  \-

  Therefore, the \textit{DEC} \textit{PDP-11} minicomputer, for instance,
  had a few instructions in its instruction set for reading and writing blocks
  of the inserted tape.

  \-

  Note that, no abstraction was provided and the operating system and
  applications, which were tightly related at this time, had to consider
  the data directly.
\end{frame}

% cp/m

\subsection{CP/M}

% 1)

\begin{frame}
  \frametitle{Overview}

  \textit{CP/M - Control Program for Microcomputers} was invented by
  \textit{Gary Kildall} in $1973$.

  \-

  This operating system was able to control at least one floppy disk
  and provided a way to store information in what was called \textbf{files}.

  \-

  However, although the system provided a way to store files in a flat
  organisation, the name \textit{file system} was not used yet.
\end{frame}

% 2)

\begin{frame}
  \frametitle{Files}

  \textit{CP/M} actually introduced the file abstraction concept.

  \-

  Files were limited to eight characters and were given a \textbf{extension}
  composed of three characters following a period.

  \-

  This file extension was used for determining the file type.

  \-

  Although this can be seen as extremely limitating, it was not the case
  back then since computers suffered the same kind of limitations anyway.
\end{frame}

%
% monkeys
%

\section{Monkeys}

% fat

\subsection{FAT}

\begin{frame}
  \frametitle{QDOS}

\end{frame}

\begin{frame}
  \frametitle{DOS}

\end{frame}

\begin{frame}
  \frametitle{FAT}

\end{frame}

%
% cats
%

\section{Cats}


XXX

%
% support
%

\section{Support}

% 1)

\begin{frame}
  \frametitle{Overview}

  Although the storage support does not matter much it obviously impact on
  the file system design.

  \-

  For instance, \textit{Google} relies on the \textit{GFS - Google File System}
  which has been designed to handle more reads than writes.

  \-

  Besides, according to the storage support speed and capacity, not all
  file systems will be equally efficient.
\end{frame}

% 2)

\begin{frame}
  \frametitle{List}

  The storages that are commonly used are listed below:

  \begin{itemize}
    \item
      \textit{RAM - Random Access Memory};
    \item
      \textit{Flash};
    \item
      \textit{Disk};
    \item
      \textit{CD-ROM};
    \item
      \etc{}
  \end{itemize}
\end{frame}

%
% terminology
%

\section{Terminology}

% 1)

\begin{frame}
  \frametitle{Overview}

  XXX
\end{frame}

o Terminology
  - Names
  - Metadata/Superblock
  - Ino
  - Inodes
  - Blocks
  - Clusters
  - Groups
  - Links
  - etc.

%
% principles
%

\section{Principles}

% 1)

\begin{frame}
  \frametitle{Overview}

  XXX
\end{frame}

o Overview
  - General organisation
  - VFS abstraction

%
% file systems
%

\section{File Systems}

% 1)

\begin{frame}
  \frametitle{Overview}

  XXX
\end{frame}

% centralised

\subsection{Centralised}

% XXX
\subsection{CP/M}

% 1)


 o CP/M: in 1973
 o FAT
  - fat allocation table
  - blocks chain
  - limitations: names, file size, file system size
 o ext2 (FFS/UFS)
  - groups
  - inode tables
  - block tables
  - fsck
  - limitations: names, directory size, file system size, consistency(fsck)
 o ext3
  - journaling
 o NTFS
  - advanced data structures (compared to fat)
  - reliability(journaling)
  - ACLs
  - sparse files
  - compression
  - hard/symbolic links
  - encryption
 o reiserfs
  - by (murderer) Hans Reiser
  - journaling
  - very efficient for small files (directly included in the directory inode's block)
  - good performance
  - b+tree
 o JFS
  - from IBM
  - 64-bit
  - journaling
  - b+tree
  - dynamic inode allocation
  - compression
 o XFS
  - quite similar to JFS
 o ZFS
  - from Sun
  - 128-bit
  - storage pools
  - copy-on-write transactional model
  - etc.

% networked

\subsection{Networked}

% 1)

\begin{frame}
  \frametitle{Overview}

  XXX
\end{frame}

 o NFS
 o Plan9

%
% access control
%

\section{Access Control}

% 1)

\begin{frame}
  \frametitle{Overview}

  XXX
\end{frame}

o Access Control
 - reference to the Security lecture
 o Unix permissions
 o ACL
 o Advanced modeles?

%
% advanced features
%

\section{Advanced Features}

% 1)

\begin{frame}
  \frametitle{Overview}

  XXX
\end{frame}

o Advanced Features
 o Union mounts
 o Versioning
   - Elephant(research)
 o LVM - Logical Volume Manager

%
% conclusion
%

\section{Conclusion}

% 1)

\begin{frame}
  \frametitle{Conclusion}

  XXX
\end{frame}

o Conclusion
 o Rethinking the file abstraction -> naming service + object oriented
   - Spring Name Service

%
% bibliography
%

\begin{frame}
  \frametitle{Bibliography}

  \bibliographystyle{amsplain}
  \bibliography{\path/bibliography/bibliography}
\end{frame}

\end{document}
