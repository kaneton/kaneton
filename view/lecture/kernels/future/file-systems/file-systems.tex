%
% ---------- header -----------------------------------------------------------
%
% project       kaneton
%
% license       kaneton
%
% file          /home/mycure/kane...els/future/file-systems/file-systems.tex
%
% created       julien quintard   [tue jan 13 05:55:34 2009]
% updated       julien quintard   [tue apr 21 19:16:45 2009]
%

%
% ---------- setup ------------------------------------------------------------
%

%
% path
%

\def\path{../../../..}

%
% template
%

%
% ---------- header -----------------------------------------------------------
%
% project       kaneton
%
% license       kaneton
%
% file          /home/mycure/kaneton/view/template/lecture.tex
%
% created       julien quintard   [wed may 16 18:17:26 2007]
% updated       julien quintard   [sun may 18 23:23:40 2008]
%

%
% class
%

\documentclass[8pt]{beamer}

%
% packages
%

\usepackage{pgf,pgfarrows,pgfnodes,pgfautomata,pgfheaps,pgfshade}
\usepackage[T1]{fontenc}
\usepackage{colortbl}
\usepackage{times}
\usepackage{amsmath,amssymb}
\usepackage{graphics}
\usepackage{graphicx}
\usepackage{color}
\usepackage{xcolor}
\usepackage[english]{babel}
\usepackage{enumerate}
\usepackage[latin1]{inputenc}
\usepackage{verbatim}
\usepackage{aeguill}

%
% style
%

\usepackage{beamerthemesplit}
\setbeamercovered{dynamic}

%
% verbatim stuff
%

\definecolor{verbatimcolor}{rgb}{0.00,0.40,0.00}

\makeatletter

\renewcommand{\verbatim@font}
  {\ttfamily\footnotesize\selectfont}

\def\verbatim@processline{
  {\color{verbatimcolor}\the\verbatim@line}\par
}

\makeatother

%
% -
%

\renewcommand{\-}{\vspace{0.4cm}}

%
% date
%

\date{\today}

%
% logos
%

\pgfdeclareimage[interpolate=true,width=34pt,height=18pt]
                {epita}{\path/logo/epita}
\pgfdeclareimage[interpolate=true,width=49pt,height=18pt]
                {upmc}{\path/logo/upmc}
\pgfdeclareimage[interpolate=true,width=25pt,height=18pt]
                {lse}{\path/logo/lse}

\newcommand{\logos}
  {
    \pgfuseimage{epita}
  }

%
% institute
%

\institute
{
  \inst{1} kaneton microkernel project
}

%
% table of contents at the beginning of each section
%

\AtBeginSection[]
{
  \begin{frame}<beamer>
   \frametitle{Outline}
    \tableofcontents[current]
  \end{frame}
}

%
% table of contents at the beginning of each subsection
%

\AtBeginSubsection[]
{
  \begin{frame}<beamer>
   \frametitle{Outline}
    \tableofcontents[current,currentsubsection]
  \end{frame}
}


%
% title
%

\title{File Systems}

%
% document
%

\begin{document}

%
% title frame
%

\begin{frame}
  \titlepage
\end{frame}

%
% outline frame
%

\begin{frame}
  \frametitle{Outline}

  \tableofcontents
\end{frame}

%
% figures
%

\pgfdeclareimage[interpolate=true,width=220pt]
                {fat}
                {figures/fat}

\pgfdeclareimage[interpolate=true,width=200pt]
                {ext2-overview}
                {figures/ext2-overview}

\pgfdeclareimage[interpolate=true,width=240pt]
                {ext2-inode}
                {figures/ext2-inode}

%
% ---------- text -------------------------------------------------------------
%

%
% introduction
%

\section{Introduction}

% 1)

\begin{frame}
  \frametitle{Overview}

  This course is intended to present file systems through their history,
  the principles, the access control and so forth.
\end{frame}

% 2)

\begin{frame}
  \frametitle{Need}

  Both operating systems and users need to store data in a permanent way
  \ie{} so that data can be retrieve after the computer has been shut down.
\end{frame}

% 3)

\begin{frame}
  \frametitle{Equipment}

  Storage devices were early integrated to computers to provide a way
  to store information starting with punched cards but more recently with
  hard drives, floppies and so on.
\end{frame}

% 4)

\begin{frame}
  \frametitle{Software}

  Providing a storage capacity such as a hard disk does not suffice. Indeed,
  software must be provided to manage the space provided and possibly organise
  data, in other words, abstract the hardware into something meaningful to
  users and applications.

  \-

  As such, many systems rely on the \textbf{file} abstraction, the most popular
  being \textit{UNIX} because of its motto \textit{``Everything is file''}.
\end{frame}

%
% support
%

\section{Support}

% 1)

\begin{frame}
  \frametitle{Overview}

  Although the storage support does not matter much it obviously impact on
  the file system design.

  \-

  For instance, \textit{Google} relies on the \textit{GFS - Google File System}
  which has been designed to handle more reads than writes.

  \-

  Besides, according to the storage support speed and capacity, not all
  file systems will be equally efficient.
\end{frame}

% 2)

\begin{frame}
  \frametitle{List}

  The storages that are commonly used are listed below:

  \begin{itemize}
    \item
      \textit{RAM - Random Access Memory};
    \item
      \textit{Flash};
    \item
      \textit{Disk};
    \item
      \textit{CD-ROM};
    \item
      \etc{}
  \end{itemize}
\end{frame}

XXX

definition: used to store, catalog and retrieve information from a storage
  stable or not

abstraction: information is abstracted into files i.e data gathered logically
  together

objective: keep track of the files stored, including the data but also info
  on the file itself: file name, etc.. besides the storage capacity needs to
  be manage to find space etc.

%
% dinosaurs
%

\section{Dinosaurs}

% dectape

\subsection{DECTape}

% 1)

\begin{frame}
  \frametitle{Overview}

  In the $60s$ early days of computing, computers of the size of refrigerators,
  hence far smaller than room-wide maiframes, used tapes for storing
  data in a permanent way.

  \-

  At this age of computing, every computer came with its own operating system
  and each operating system had its way for accessing the permanently stored
  data.

  \-

  Therefore, the \textit{DEC} \textit{PDP-11} minicomputer, for instance,
  had a few instructions in its instruction set for reading and writing blocks
  of the inserted tape.

  \-

  Note that, no abstraction was provided and the operating system and
  applications, which were tightly related at this time, had to consider
  the data directly.
\end{frame}

% cp/m

\subsection{CP/M}

% 1)

\begin{frame}
  \frametitle{Overview}

  \textit{CP/M - Control Program for Microcomputers} was invented by
  \textit{Gary Kildall} in $1973$.

  \-

  This operating system was able to control at least one floppy disk
  and provided a way to store information in what was called \textbf{files}.

  \-

  However, although the system provided a way to store files in a flat
  organisation, the name \textit{file system} was not used yet.
\end{frame}

% 2)

\begin{frame}
  \frametitle{Files}

  \textit{CP/M} actually introduced the file abstraction concept.

  \-

  Files were limited to eight characters and were given a \textbf{extension}
  composed of three characters following a period.

  \-

  This file extension was used for determining the file type.

  \-

  Although this can be seen as extremely limitating, it was not the case
  back then since computers suffered the same kind of limitations anyway.
\end{frame}

%
% monkeys
%

\section{Monkeys}

% fat

\subsection{FAT}

% 1)

\begin{frame}
  \frametitle{QDOS}

  Other people soon got the same idea as \textit{Kildall} and \textit{Tim
  Patterson} wrote his own operating system called \textit{QDOS - Quick
  and Dirty Operating System}.

  \-

  \textit{QDOS} was a quickly-writte clone of \textit{CP/M} for $16-bit$
  computers because it needed an operating system for such machines and
  \textit{CP/M} did not support it.

  \-

  \textit{QDOS} had a slightly different file system although it provided
  the same features such as a flat naming model, file extensions \etc{}

  \-

  The \textit{QDOS} file system used an organisation method called
  \textit{FAT - File Allocation Table} so the file system itself was given
  the incredibly sticky and imaginative name of \textit{FAT}.
\end{frame}

% 2)

\begin{frame}
  \frametitle{DOS}

  \textit{Bill Gates} bought \textit{QDOS} for $\$50,000$ and renamed it
  \textit{MS-DOS} so that he was now able to sell it to \textit{IBM} and
  every company making \textit{IBM} clones.
\end{frame}

% 3)

\begin{frame}
  \frametitle{Metadata}

  Metadata literally means \textit{``data about data''} and that's exactly
  what it is.

  \-

  While metadata may sound relatively recent and modern, all file systems
  right from the very beginning had to store at least some metadata along.

  \-

  However, we can consider \textit{CP/M} file system, hence \textit{FAT},
  to be the first to introduce metadata the way file systems now use it.

  \-

  A file system is given a storage space. At the very beginning of this
  storage, a data structure is recorded containing general information
  about the file system: the \textbf{metadata}.

  \-

  Obviously, depending on the file systems, the metadata give more or less
  information such as the size of the storage space, the block size,
  location of other data structures \etc{}
\end{frame}

% 4)

\begin{frame}
  \frametitle{FAT}

  As for \textit{CP/M} file system, \textit{FAT} names the files according
  to a $8-3$ file name/extension scheme.

  \-

  Besides, the way file are stored was designed around the physical structure
  of floppy disks, the primary storage device of the day.

  \-

  The \textit{File Allocation Table} describes which area of the disk were
  allocated to files, which were free space and which were damaged and
  unused, known as \textit{``bad sectors''}.

  \-

  Because each floppy had very little space \ie{} $8 sectors x 40 tracks x
  512 bytes = 160 KB$, the table itself needed to be very small. To keep it
  small, the disk was divided into \textbf{clusters}, a cluster---groups
  of contiguous sectors---hence becoming the smallest storage unit.

  \-

  In order to cope with files bigger than a cluster, the file system uses
  a \textbf{block chaining} technique by relying on the \textit{FAT} to
  know what is the next block that contains the data related to a file.

  \-

  Note however that while increasing the cluster size decreases the table
  size, it also increases internal fragmentation, hence the waste of
  storage space.
\end{frame}

% 5)

\begin{frame}
  \frametitle{Versions}

  The first version of the \textit{FAT} file system was \textit{FAT-12}
  because it used a $12-bit$ number for counting the clusters: $2^{12} x
  8KB cluster = 32MB$. The file system was therefore limited to a total
  storage capacity of $32MB$.

  \-

  Although this sounded reasonable at the time, it was without counting on
  the amazing evolution of computing. \textit{FAT-12} quickly become
  far too limited for modern storage devices such as hard disks.

  \-

  \textit{FAT-16}($2 GB$) and even \textit{FAT-32}($8 TB$) were later released
  for dealing with the increasing storage capacity. It is however interesting
  to note that, although a plain text string indicates the file system type
  \textit{``FAT16''}, \textit{``FAT32''} \etc{}, it cannot be used by
  the file system driver to detect the \textit{FAT} version and a sophisticate
  computation must be performed according to the storage capacity, cluster
  size \etc{} This anecdocte explains the \textit{dirty} name-part.

  \-

  Later came \textit{VFAT - Virtual FAT} that was able to handle long
  file names up to $255$ characters. However, for compatibility reason, a
  short name was also created: \textit{``TETON~1.TXT''} for instance.
\end{frame}

% 6)

\begin{frame}
  \frametitle{Directories}

  Id did not really matter that \textit{FAT} did not provide any way to
  organise files because, on a floppy disk, only a few files could be stored
  anyway.

  \-

  However, since \textit{IBM} was about to release the \textit{PC-XT} with
  an optional $20 MB$ hard disk, it would be better if this enormous space
  could be organised.

  \-

  Therefore, the \textbf{directory} concept was introduced, using the
  backslash character as a separator.
\end{frame}

% 7)

\begin{frame}
  \frametitle{Example}

  \begin{center}
    \pgfuseimage{fat}
  \end{center}
\end{frame}

% 8)

\begin{frame}
  \frametitle{Summary}

  Although \textit{FAT} has been used for decades, it is nonetheless a
  very strange piece of software.

  \-

  Among its limitations can be counted:

  \begin{itemize}
    \item
      The name length constraint;
    \item
      The file size;
    \item
      The storage space size limit;
    \item
      The internal fragmentation;
    \item
      The poor data structure; and
    \item
      The lack of features including late directories \etc{}
  \end{itemize}
\end{frame}

% ufs

\subsection{UFS}

% 1)

\begin{frame}
  \frametitle{Overview}

  The \textit{UFS - Unix File System}, also known as the \textit{Berkeley}
  \textit{FFS - Fast File System}, came out of the research community devoted
  to \textit{UNIX}.

  \-

  Among notable improvements and differences with other file systems are
  the case sensitive names, $32-bit$ pointers and \textbf{cylinder}-grouping
  to improve disk head movement time.

  \-

  Another interesting thing was introduced to check the file system
  consistency. Indeed, since file metadata can be corrupted if the computer
  crashes \etc{}, the tool \textit{FSCK - File System Check} was introduced.

  \-

  Finally, \textit{UFS} introduces the \textbf{links}, both hard and symbolic.

  \-

  Many \textit{UNIX} vendors and other operating systems implemented their
  own version of \textit{UFS}, often making them incompatible with each
  other.
\end{frame}

% partitioning

\subsection{Partitioning}

XXX

% ext2

\subsection{ext2}

% 1)

\begin{frame}
  \frametitle{Overview}

  \textit{Ext2} is basically a clone of \textit{UFS} in the very exact way
  to \textit{Linux} is a clone of \textit{UNIX}.

  \-

  \textit{Ext2} was designed as a successor of \textit{ext} to bring a new
  file system to the \textit{Linux} operating system.
\end{frame}

% 2)

\begin{frame}
  \frametitle{Design}

  The concepts explained below are the same in \textit{UFS} but are described
  for \textit{ext2}.

  \-

  Firstly, the file system metadata are known to be stored in the
  \textbf{superblock}, a special block, at the beginning of the partition.

  \-

  Secondly, files metadata are stored in what are known as \textbf{Inode}s
  and every \textit{Inode} is identified by a \textit{\textbf{INO} - Inode No}.

  \-

  The partition is divided into \textbf{blocks}, usually of $8 KB$.
  Besides, in order to keep the file metadata close to the data itself,
  both are stored in the same \textbf{group} \ie{} \textit{UFS}'s
  containers.
\end{frame}

% 3)

\begin{frame}
  \frametitle{Details}

  \textit{Ext2} keeps track of allocated blocks inside a group by relying
  on a very simple data structure: a bitmap.

  \-

  Likewise, a bitmap is used to know which are the \textit{inode} used and
  which are the one free. Indeed, \textit{inode}s are pre-allocated according
  to the file system size.

  \-

  Those \textit{inode}s are kept in an \textit{inode} table following the
  \textit{inode} bitmap.
\end{frame}

% 4)

\begin{frame}
  \frametitle{Example}

  \begin{center}
    \pgfuseimage{ext2-overview}
  \end{center}
\end{frame}

% 5)

\begin{frame}
  \frametitle{Example}

  \begin{center}
    \pgfuseimage{ext2-inode}
  \end{center}
\end{frame}

% 6)

\begin{frame}
  \frametitle{Example}

  As an example, let's consider a file \textit{teton.txt} located in the root
  directory.

  \-

  Assuming that the system knows the associated \textit{ino}, it will first
  start by computing the group in charge of this \textit{ino}.

  \-

  Since the metadata contains information regarding the number of groups,
  the number of inodes and so forth, the system can easily know in which
  group lies an inode according to its \textit{ino}.

  \-

  Once the group located, the system will retrieve the \textit{inode} from
  the \textit{inode} table and then access the data through its pointers,
  being direct or indirects.
\end{frame}

% 7)

\begin{frame}
  \frametitle{Size}

  So, how big can a file be in \textit{ext2}?

  \-

  Let's do the math since we know that the standardised block size is $8 KB$.

  \-

  \begin{eqnarray*}
    size & = & 8192 + \\
         &   & \frac{8192}{4} x 8192 + \\
         &   & \frac{8192}{4}^{2} x 8192 + \\
         &   & \frac{8192}{4}^{3} x 8192 + \\
         & = & 70403120701440 B \\
         & = & 64 TB \\
  \end{eqnarray*}
\end{frame}

% 8)

\begin{frame}
  \frametitle{Fault Tolerance}

  \textit{ext2} provides some fault tolerance features.

  \-

  For example, in order to deal with corrupted metadata, the superblock is
  replicated at fixed positions in the partition.

  \-

  As such, if the superblock is detected as being corrupted because the magic
  number is invalid, another superblock is used and the first one is updated,
  getting back to a consistent state.
\end{frame}

% 9)

\begin{frame}
  \frametitle{Summary}

  Although \textit{ext2} has been used for years and is loved by many for
  being very stable, it still lack a few things though it is undeniably,
  a clean file system design.

  \begin{itemize}
    \item
      The file system consistency is not ensured and must be checked on
      a regular basis through \textit{fsck};
    \item
      Many components are static such as the number of inodes, the number of
      blocks \etc{} such that it is impossible to have a single big file
      or millions of small files;
    \item
      Some data structures are still very slow, especially the directory
      entries that must be explored sequentially to find a particular entry;
    \item
      Both internal and external fragmentations are still major problems
      in \textit{ext2}.
  \end{itemize}
\end{frame}

% 10)

\begin{frame}
  \frametitle{Journaling}

  Updating file systems to reflect changes to files and directories usually
  requires many separate write operations.

  \-

  This introduces a race condition for which an interruption like a power
  failure between writes can leave data structures in an invalid intermediate
  state.

  \-

  For example, deleting a file on a \textit{ext2} involves two steps: removing
  its directory entry and marking space for the file and its inode as free in
  the free space map.

  \-

  To avoid both leaking space and having to walk through the entire data
  structure with \textit{fsck}, \textbf{journaling} can be used.

  \-

  A journaled file system maintains a journal of the changes it intends to
  make, ahead of time. After a crash, recovery simply involves replaying
  changes from this journal until the file system is consistent again.
\end{frame}

%
% cats
%

\section{Cats}

% reiserfs

\subsection{ReiserFS}

% 1)

\begin{frame}
  \frametitle{Overview}

  Because no file system provided a way to prevent corruptions, \textit{Hans
  Reiser} decided to create his own file system and very modestly name
  it \textbf{ReiserFS}.

  \-

  \textit{ReiserFS} added not only journaling but attempted to improve
  many aspects of \textit{ext2} such as dynamic inodes management, tail
  packing, efficient space management, improved directory handling \etc{}

  \-

  Although \textit{ReiserFS} performance has been measured to be $10-15\%$
  faster than \textit{ext2}, its extreme complexity and lack of compatibility
  between versions were largely criticized.
\end{frame}

% 2)

\begin{frame}
  \frametitle{Superblock}

  The superblock is located at byte $65536$, leaving the first $64K$ unused.

  \-

  The superblock is not replicated making the file system vulnerable to
  metadata corruption.
\end{frame}

% 3)

\begin{frame}
  \frametitle{Blocks}

  The partition can contain a maximum of $2^{32}$ blocks so that the
  block size might be set according to the file system size factor.

  \-

  Directly following the superblock is a bitmap of free blocks that is used
  for finding space. As such, if the block size is $4,096K$, a block bitmap
  keeps track of $32,768$ blocks.

  \-

  More bitmap blocks might be necessary to reference all the file system
  blocks. Those blocks would simply follow the first bitmap block.
\end{frame}

% 4)

\begin{frame}
  \frametitle{Tree}

  \textit{ReiserFS} relies on the \textbf{b+tree} data structure for keeping
  information. \textit{b+tree}s are very efficient data structures that allows
  for searches, insertions and deletions in logarithmic time; hence are very
  suitable for file systems.

  \-

  This special and unique data structure is actually called \textbf{s+tree}
  in the \textit{ReiserFS} documentation because it has been specifically
  implemented so that balancing can be deferred until data is commit to the
  disk; hence improving the performance.
\end{frame}

% 5)

\begin{frame}
  \frametitle{Tree}

  The tree is composed of internal and leaf nodes such that every node is
  a disk block.

  \-

  Each object, called an \textbf{item}, such as files, directories \etc{} is
  assigned a unique \textbf{key}. Such keys can very much be compared as
  \textit{ino} in \textit{ext2} for instance.

  \-

  Such keys are actually represented by a tuple $(directory-id, object-id,
  offset, type)$ so that files belonging to the same directory but also blocks
  of the same file lie in the same sub-tree.

  \-

  Note that to handle large files, several internal blocks are required to
  provide enough pointers since a internal block can only contain
  $(block-size - 48) / 4$ pointers to data blocks.
\end{frame}

% 6)

\begin{frame}
  \frametitle{Tail Packing}

  Since most small files are very common in current operating systems,
  \textit{ReiserFS} has been designed for reducing internal fragmentation
  by storing those small files at the end of not-fully used blocks.

  \-

  This technique is sometimes called \textbf{tail packing}, \textbf{block
  sub-allocation} or even \textbf{tail merging}

  \-

  In some scenarios, where the majority of files are smaller than half
  the block size such as source code files, configuration files \etc{}, tail
  packing can both increase storage efficiency but also increase performance
  due to space locality by reading contiguous blocks containing multiple
  small files.

  \-

  Note that these advantages can be negated by the increased complexity
  of implementation as it is the case for \textit{ReiserFS}.
\end{frame}

% 7)

\begin{frame}
  \frametitle{Journal}

  The \textbf{journal} is composed of few contiguous blocks that follow the
  blocks bitmap.

  \-

  The journal describes the \textbf{transactions}---operations respecting the
  \textit{ACID - Atomicity, Consistency, Isolation and Durability}
  properties---performed on the file system. Therefore, the file system is
  kept in a consistent state at all times.

  \-

  At a later point, the journal is flushed and, if everything is successful,
  marked as such so that operations are made effective.

  \-

  Note that originally, journaling was applied only on metadata but
  \textit{Reiser 4} introduced journaling for both metadata and data.
\end{frame}

% 8)

\begin{frame}
  \frametitle{Journal}

  The journal is a circular array that contains \textbf{transactions} needed
  to be commit on the disk.

  \-

  The first block contains the journal header that specify where the
  first transaction to process is located in the journal.

  \-

  Every transaction is assigned a transaction identifier and a transaction
  is considered as to be processed if its identifier is lower than the current
  identifier which is stored in the journal header.
\end{frame}

% 9)

\begin{frame}
  \frametitle{Transaction}

  XXX
\end{frame}

% 10)

\begin{frame}
  \frametitle{Epilogue}

  The company responsible for \textit{ReiserFS} decided to discontinue
  \textit{ReiserFS} stopping maintaining it and to move to \textit{Reiser 4}.

  \-

  Although this has been criticized a lot by the community especially
  because of the file system complexity; hence difficulty to maintain it,
  an unexpected event put an end to the project.

  \-

  Indeed, in $2006$, \textit{Hans Reiser}'s wife disappeared and \textit{Hans
  Reiser} was arrested for murder. Although he denied for two years,
  he finally led the police to his wife body and was jailed for life in
  $2008$.
\end{frame}

% vfs

\subsection{VFS}

% 1)

\begin{frame}
  \frametitle{Introduction}

  Until now, several file systems have been described. Therefore, it is
  reasonable to think that an operating system could use different
  file systems, potentially at the same time.

  \-

  Indeed, a partition could use \textit{ext2} while another could use
  \textit{Reiser 4} for example.

  \-

  How could the operating system make sure that applications work
  independently of the file system in use?
\end{frame}

% 2)

\begin{frame}
  \frametitle{Overview}

  The purpose of a \textit{VFS - Virtual File System} is to allow client
  applications to access different types of concrete file systems in a uniform
  way by providing an abstract layer.

  \-

  The first \textit{VFS} mechanism in \textit{UNIX}-like systems was introduced
  by \textit{Sun Microsystems} in \textit{SunOS} $2.0$ in $1985$.

  \-

  It allowed client applications to transparently access either the local
  file system or a network storage through \textit{NFS - Network File System}.

  \-

  On \textit{Windows} systems, the same concept is provided under the
  name \textit{IFS - Installable File System}.
\end{frame}

% 3)

\begin{frame}
  \frametitle{Interface}

  The \textit{VFS} provides an standardized interface composed of
  \texttt{open()}, \texttt{stat()}, \texttt{chmod()} \etc{}

  \-

  When an operation is performed, given a pathname, the \textit{VFS} kernel
  layer stats by looking in the \textit{dcache} for a \textit{dentry}, loading
  \textit{inode} from the disk into \textit{vnode} and a new \textit{dentry}
  to complete the lookup process if required.

  \-

  To lookup for an \textit{inode}, the \textit{VFS} calls \texttt{lookup()}
  provided by a file system specific implementation such as \textit{ext2}.

  \-

  Once the \textit{dentry} has been located, performing the operation is quite
  straightforward since everything is standardized. Thus, the \textit{VFS}
  can perform \texttt{stat()} by returning the userspace applications the
  information it needs.
\end{frame}

% 4)

\begin{frame}
  \frametitle{Implementations}

  The actual file systems register to the \textit{VFS} at the startup by
  providing a \texttt{file\_system\_type} structure describing the file
  system.

  \-

  This structure includes a file system name, a function pointer on a
  method to read the superblock and some flags.

  \-

  The \texttt{read\_super()} method is responsible for determining if the
  device contains a file system of the given type.

  \-

  This method returns a \texttt{super\_block} structure that contains
  information among which is a \texttt{super\_operations} structure that
  defines the file system implementation interface with function pointers
  to \texttt{read\_inode()}, \texttt{write\_super}, \texttt{clear\_inode}
  and so on.

  \-

  Likewise, a \texttt{inode\_operations} and a \texttt{file\_operations} are
  provided containing function pointers such as \texttt{create()},
  \texttt{link()}, \texttt{mknode}, \texttt{truncate} but also \texttt{read},
  \texttt{write}, \texttt{lock} \etc{}
\end{frame}

% 5)

\begin{frame}
  \frametitle{FUSE}

  Recently, the same concept was taken to user-land so that file systems
  can run in user-land.

  \-

  Indeed, system calls are redirected and passed to the user-land through
  a virtual device acting as a communication channel between the kernel
  and user-land driver.

  \-

  This technique is very useful for developing file systems as there is
  no need to restart the kernel after every modification; hence providing the
  very same advantages as micro-kernels.
\end{frame}

% 6)

\begin{frame}
  \frametitle{Conclusion}

  Although the \textit{VFS} may seem very complicated when described in
  books, it is actually extremely logic and simple.

  \-

  Indeed, the \textit{VFS} does nothing but providing a file system abstraction
  that every actual file system must match.
\end{frame}

XXX
 o NTFS
  - advanced data structures (compared to fat)
  - reliability(journaling)
  - ACLs
  - sparse files
  - compression
  - hard/symbolic links
  - encryption
 o JFS
  - from IBM
  - 64-bit
  - journaling
  - b+tree
  - dynamic inode allocation
  - compression
 o XFS
  - quite similar to JFS
 o ZFS
  - from Sun
  - 128-bit
  - storage pools
  - copy-on-write transactional model
  - etc.

o Access Control
 - reference to the Security lecture
 o Unix permissions
 o ACL
 o Advanced modeles?

o Advanced Features
 o Union mounts
 o Versioning
   - Elephant(research)
 o LVM - Logical Volume Manager

%
% conclusion
%

\section{Conclusion}

% 1)

\begin{frame}
  \frametitle{Conclusion}

  XXX
\end{frame}

o Conclusion
 o Rethinking the file abstraction -> naming service + object oriented
   - Spring Name Service

%
% bibliography
%

\begin{frame}
  \frametitle{Bibliography}

  \bibliographystyle{amsplain}
  \bibliography{\path/bibliography/bibliography}
\end{frame}

\end{document}

XXX algo d'allocation: first fit, next fit, last fit, best fit, worst fit
XXX partitions
