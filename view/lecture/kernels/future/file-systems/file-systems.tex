%
% ---------- header -----------------------------------------------------------
%
% project       kaneton
%
% license       kaneton
%
% file          /home/mycure/kane...els/future/file-systems/file-systems.tex
%
% created       julien quintard   [tue jan 13 05:55:34 2009]
% updated       julien quintard   [mon mar 30 20:35:10 2009]
%

%
% ---------- setup ------------------------------------------------------------
%

%
% path
%

\def\path{../../../..}

%
% template
%

%%
%% copyright     (c) julien quintard
%%
%% project       kaneton
%%
%% file          /home/mycure/kaneton/view/templates/lecture.tex
%%
%% created       julien quintard   [sat nov 19 17:13:03 2005]
%% updated       julien quintard   [fri dec  2 22:36:34 2005]
%%

%
% class
%

\documentclass[8pt]{beamer}

%
% packages
%

\usepackage{pgf,pgfarrows,pgfnodes,pgfautomata,pgfheaps,pgfshade}
\usepackage{colortbl}
\usepackage{times}
\usepackage{amsmath,amssymb}
\usepackage{graphics}
\usepackage{graphicx}
\usepackage{color}
\usepackage{xcolor}
\usepackage[english]{babel}
\usepackage{enumerate}
\usepackage[latin1]{inputenc}

%
% style
%

\usepackage{beamerthemesplit}
\setbeamercovered{dynamic}

%
% verbatim font
%

\definecolor{verbatimcolor}{rgb}{0,0.4,0}

\makeatletter
\renewcommand{\verbatim@font}
  {\ttfamily\footnotesize\color{verbatimcolor}\selectfont}
\makeatother

%
% new line
%

\newcommand{\nl}[0]{\vspace{0.4cm}}

%
% date
%

\date{\today}

%
% logos
%

\pgfdeclareimage[interpolate=true,width=34pt,height=18pt]
                {epita}{../../logos/epita}
\pgfdeclareimage[interpolate=true,width=49pt,height=18pt]
                {upmc}{../../logos/upmc}
\pgfdeclareimage[interpolate=true,width=25pt,height=18pt]
                {lse}{../../logos/lse}

\newcommand{\logos}
  {
    \pgfuseimage{epita}
  }

%
% institute
%

\institute
{
  \inst{1} kaneton microkernel project
}

%
% table of contents at the beginning of each section
%

\AtBeginSection[]
{
  \begin{frame}<beamer>
   \frametitle{Outline}
    \tableofcontents[current]
  \end{frame}
}

%
% table of contents at the beginning of each subsection
%

\AtBeginSubsection[]
{
  \begin{frame}<beamer>
   \frametitle{Outline}
    \tableofcontents[current,currentsubsection]
  \end{frame}
}


%
% title
%

\title{File Systems}

%
% document
%

\begin{document}

%
% title frame
%

\begin{frame}
  \titlepage
\end{frame}

%
% outline frame
%

\begin{frame}
  \frametitle{Outline}

  \tableofcontents
\end{frame}

%
% ---------- text -------------------------------------------------------------
%

%
% introduction
%

\section{Introduction}

% 1)

\begin{frame}
  \frametitle{}

  This course is intended to present file systems through their history,
  the principles, the access control and so forth.
\end{frame}

% 2)

\begin{frame}
  \frametitle{Need}

  
\end{frame}

o Introduction
 o What do people need?
 o What does the hardware provide?
 o What should the software do to abstract the hardware into something meaningful.
   - storing and organising data abstracted into files.

%
% support
%

\section{Support}

% 1)

\begin{frame}
  \frametitle{Overview}

  XXX
\end{frame}

o Storage
 - the storage does not matter much but can obviously impact on the file system design.
   - RAM
   - Flash
   - Disk
   - Network
   - etc.

%
% terminology
%

\section{Terminology}

% 1)

\begin{frame}
  \frametitle{Overview}

  XXX
\end{frame}

o Terminology
  - Names
  - Metadata/Superblock
  - Ino
  - Inodes
  - Blocks
  - Clusters
  - Groups
  - Links
  - etc.

%
% principles
%

\section{Principles}

% 1)

\begin{frame}
  \frametitle{Overview}

  XXX
\end{frame}

o Overview
  - General organisation
  - VFS abstraction

%
% file systems
%

\section{File Systems}

% 1)

\begin{frame}
  \frametitle{Overview}

  XXX
\end{frame}

% centralised

\subsection{Centralised}

% 1)

\begin{frame}
  \frametitle{Overview}

  XXX
\end{frame}

 o CP/M: in 1973
 o FAT
  - fat allocation table
  - blocks chain
  - limitations: names, file size, file system size
 o ext2 (FFS/UFS)
  - groups
  - inode tables
  - block tables
  - fsck
  - limitations: names, directory size, file system size, consistency(fsck)
 o ext3
  - journaling
 o NTFS
  - advanced data structures (compared to fat)
  - reliability(journaling)
  - ACLs
  - sparse files
  - compression
  - hard/symbolic links
  - encryption
 o reiserfs
  - by (murderer) Hans Reiser
  - journaling
  - very efficient for small files (directly included in the directory inode's block)
  - good performance
  - b+tree
 o JFS
  - from IBM
  - 64-bit
  - journaling
  - b+tree
  - dynamic inode allocation
  - compression
 o XFS
  - quite similar to JFS
 o ZFS
  - from Sun
  - 128-bit
  - storage pools
  - copy-on-write transactional model
  - etc.

% networked

\subsection{Networked}

% 1)

\begin{frame}
  \frametitle{Overview}

  XXX
\end{frame}

 o NFS
 o Plan9

%
% access control
%

\section{Access Control}

% 1)

\begin{frame}
  \frametitle{Overview}

  XXX
\end{frame}

o Access Control
 - reference to the Security lecture
 o Unix permissions
 o ACL
 o Advanced modeles?

%
% advanced features
%

\section{Advanced Features}

% 1)

\begin{frame}
  \frametitle{Overview}

  XXX
\end{frame}

o Advanced Features
 o Union mounts
 o Versioning
   - Elephant(research)
 o LVM - Logical Volume Manager

%
% conclusion
%

\section{Conclusion}

% 1)

\begin{frame}
  \frametitle{Conclusion}

  XXX
\end{frame}

o Conclusion
 o Rethinking the file abstraction -> naming service + object oriented
   - Spring Name Service

%
% bibliography
%

\begin{frame}
  \frametitle{Bibliography}

  \bibliographystyle{amsplain}
  \bibliography{\path/bibliography/bibliography}
\end{frame}

\end{document}
