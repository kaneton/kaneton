%
% ---------- header -----------------------------------------------------------
%
% project       kaneton
%
% license       kaneton
%
% file          /home/texane/proj/present/kaneton/view/lecture/kernels/future/windows-nt/windows-nt.tex
%
% created       fabien le mentec
% updated       fabien le mentec
%

%
% ---------- setup ------------------------------------------------------------
%

%
% path
%

\def\path{../../../..}

%
% template
%

%
% ---------- header -----------------------------------------------------------
%
% project       kaneton
%
% license       kaneton
%
% file          /home/mycure/kaneton/view/template/lecture.tex
%
% created       julien quintard   [wed may 16 18:17:26 2007]
% updated       julien quintard   [sun may 18 23:23:40 2008]
%

%
% class
%

\documentclass[8pt]{beamer}

%
% packages
%

\usepackage{pgf,pgfarrows,pgfnodes,pgfautomata,pgfheaps,pgfshade}
\usepackage[T1]{fontenc}
\usepackage{colortbl}
\usepackage{times}
\usepackage{amsmath,amssymb}
\usepackage{graphics}
\usepackage{graphicx}
\usepackage{color}
\usepackage{xcolor}
\usepackage[english]{babel}
\usepackage{enumerate}
\usepackage[latin1]{inputenc}
\usepackage{verbatim}
\usepackage{aeguill}

%
% style
%

\usepackage{beamerthemesplit}
\setbeamercovered{dynamic}

%
% verbatim stuff
%

\definecolor{verbatimcolor}{rgb}{0.00,0.40,0.00}

\makeatletter

\renewcommand{\verbatim@font}
  {\ttfamily\footnotesize\selectfont}

\def\verbatim@processline{
  {\color{verbatimcolor}\the\verbatim@line}\par
}

\makeatother

%
% -
%

\renewcommand{\-}{\vspace{0.4cm}}

%
% date
%

\date{\today}

%
% logos
%

\pgfdeclareimage[interpolate=true,width=34pt,height=18pt]
                {epita}{\path/logo/epita}
\pgfdeclareimage[interpolate=true,width=49pt,height=18pt]
                {upmc}{\path/logo/upmc}
\pgfdeclareimage[interpolate=true,width=25pt,height=18pt]
                {lse}{\path/logo/lse}

\newcommand{\logos}
  {
    \pgfuseimage{epita}
  }

%
% institute
%

\institute
{
  \inst{1} kaneton microkernel project
}

%
% table of contents at the beginning of each section
%

\AtBeginSection[]
{
  \begin{frame}<beamer>
   \frametitle{Outline}
    \tableofcontents[current]
  \end{frame}
}

%
% table of contents at the beginning of each subsection
%

\AtBeginSubsection[]
{
  \begin{frame}<beamer>
   \frametitle{Outline}
    \tableofcontents[current,currentsubsection]
  \end{frame}
}


%
% title
%

\title{Windows NT}

%
% document
%

\begin{document}

%
% title frame
%

\begin{frame}
  \titlepage
\end{frame}

%
% outline frame
%

\begin{frame}
  \frametitle{Outline}

  \tableofcontents
\end{frame}

%
% Overview
%

\section{Overview}

% 1)

\begin{frame}
  \frametitle{History}

  \begin{itemize}
    \item
      Windows NT is a kernel designed in the 90s by Microsoft
    \item
      The very core concepts did not change since then
    \item
      The team led by David Cutler, previously working at DEC on VAX/VMS
  \end{itemize}

\end{frame}

% 2)

\begin{frame}
  \frametitle{Design}

  NT kernel was far ahead of his time regarding today s general purpose kernels evolution

  \begin{itemize}
    \item
      monolithic kernel
    \item
      drivers communicate by passing messages
    \item
      stackable model
    \item
      asynchronous IO
    \item
      multithreaded
    \item
      ``hybrid'' is a marketing term (from Torvald: \textit{http://en.wikipedia.org/wiki/Hybrid\_kernel})
  \end{itemize}

\end{frame}

% 3)

\begin{frame}[containsverbatim]
  \frametitle{Kernel As Seen By The Userland (1)}
   
  Look at the following userland code

  \begin{verbatim}

  HANDLE Handles[3];

  // object model
  Handles[0] = CreateFile(...);
  Handles[1] = CreateThread(...);
  Handles[2] = CreateEvent(...);
  Handles[3] = CreateMutex(...);

  // asynchronous io model
  Overlapped.hEvent = Handles[2];
  ReadFile(Handles[0], ..., &Overlapped);

  // homogeneous model
  WaitForMultipleObjects(Handles, 4, ...);

  // common interface
  CloseHandle(Handles[0]);
  CloseHandle(Handles[1]);
  CloseHandle(Handles[2]);
  CloseHandle(Handles[3]);

  \end{verbatim}
  
\end{frame}

\begin{frame}
  \frametitle{Kernel As Seen By The Userland (2)}

 It tells us long about the NT design

 \begin{itemize}
   \item
     there is a design :)
   \item
     NT has been designed around the object model
   \item
     everything is object, manipulated by HANDLE in userland
   \item
     the scheduler acts as a central place for object synchronization
   \item
     IO requests are asynchronous
 \end{itemize}

\end{frame}


%
% Object Manager
%

\section{Object Manager}

% 1)

\begin{frame}
  \frametitle{Introduction}

 The goal of every operating system is to manage \textbf{resources}, may
 they be physical or virtual.

  \-

 The NT kernel approach is to represent resources as objects managed by a
 subsystem, the \textbf{Object Manager}.

\end{frame}

% 2)

\begin{frame}
  \frametitle{Issues}

  the Object Manager addresses the following issues

  \begin{itemize}
    \item
      resource naming
    \item
      resource referencing
    \item
      resource access control
 \end{itemize}
\end{frame}

% 3)

\begin{frame}
  \frametitle{Resources As Objects}

  NT is very wide regarding what it calls a resource

  \begin{itemize}
    \item
      file (FILE\_OBJECT), device (DEVICE\_OBJECT), driver (DRIVER\_OBJECT) ...
    \item
      process (KPROCESS), thread (KTHREAD)
    \item
      synchronization: mutex (KMUTEX), event (KEVENT), callbacks ...
    \item
      even types are objects (OBJECT\_TYPE)
  \end{itemize}

\end{frame}

% 4)

\begin{frame}
  \frametitle{Objects Representation}

  In memory object instances are splitted into a \textbf{header}, common to every
  object type, and a type specific \textbf{body}. Since the header is common,
  NT can manipulate every resource using the same interface. This header contains:

  \begin{itemize}
    \item
      a pointer to the object OBJECT\_TYPE
    \item
      methods: Dump, Open, Close, Delete, Parse, Security ...
    \item
      security descriptor pointer
    \item
      reference counting information 
  \end{itemize}
\end{frame}

% 5)

\begin{frame}
  \frametitle{Naming}

  Every system needs to bind a \textbf{resource} to a \textbf{name}. The NT object
  manager uses a tree naming scheme. Conceptually, the manager is thus very similar
  to a filesystem as far as naming is concerned.

  \begin{itemize}
    \item
      everything is stored  under a root directory
    \item
      (unfollowed) convention: one directory per object type (``/Driver/'', ``/Callback/'', ``/Device/'' ...)
    \item
      subdirectory structure is type specific

      \begin{itemize}
        \item
          ``/Device/HarddiskVolume0/Dr0''
        \item
          ``/Driver/Ntfs''
        \item
          ``/Callback/PowerState''
      \end{itemize}

    \item
      special directories: ``/GLOBAL??/'' seen in userland as ``//./'')
    \item
      symbolic links allowed: ``//./C:'' points to ``/Device/HarddiskVolume0''
  \end{itemize}

\end{frame}


% 6)

\begin{frame}
  \frametitle{Referencing}

  Object lifetime relies upon a \textbf{reference count} stored in the object header.
  Obtaining an object reference can be done by \textbf{address}, \textbf{name} or by
  \textbf{handle}.

  \begin{itemize}
    \item
      most of the time, kernel and drivers reference objects by address

      \begin{itemize}
        \item
          FILE\_OBJECT*, DEVICE\_OBJECT*, DRIVER\_OBJECT*...
        \item
          ObjectHeader = CONTAINER\_OF(Address, OBJECT\_HEADER, ObjectBody)
        \item
          no lookup, access is done in o(1)
      \end{itemize}

    \item
      userland (Windows Native API) references objects by HANDLE
      \begin{itemize}
        \item
          NtCreateFile, NtCreateProcess, NtClose ...
        \item
          syscall has to find the object given its HANDLE
      \end{itemize}
  \end{itemize}

\end{frame}


% 7)

\begin{frame}
  \frametitle{Scalability}

  As seen, object referencing is done in O(1) by the kernel. But the
  userland manipulates HANDLEs, and this handle has to be resolved in
  the corresponding object at every request (system call).

  \begin{itemize}
    \item
       one handle table per process plus one for the system
    \item
      three level index table
  \end{itemize}
\end{frame}

% 8)

\begin{frame}
  \frametitle{Access Control}

  Every object has a \textbf{security descriptor}. Security descriptors are covered in
  another section, but the object manager is the place where security controls actually
  occur.

  \begin{itemize}
    \item
      the basic function is ObRerefenceObject
    \item
      according to argument passed, perform access control or not
   \item
     generally, objects comming from the kernel are trusted and access control
     is only done on HANDLEs provided by the userland.
  \end{itemize}
\end{frame}

% 9)

\begin{frame}
  \frametitle{Putting It All Together}

  \begin{itemize}
    \item
      userland process
      \begin{itemize}
        \item
          CreateFile(``C://foo'');
        \item
          real system call NtCreateFile(``//./C:/foo'');
      \end{itemize}

    \item
      kernel syscall implementation
      \begin{itemize}
        \item
          ObReferenceObjectByName(``C://foo'')
      \end{itemize}

    \item
      object manager
      \begin{itemize}
        \item
          translates ``//./C:/foo'' to ``/GLOBAL??/C:/foo''
        \item
          resolves the symlink to ``/Device/HarddiskVolume0/foo''
        \item
          translates ``//./C:/foo'' to ``/GLOBAL??/C:/foo''
        \item
          gets a DEVICE\_OBJECT on ``/Device/HarddiskVolume0''
        \item
          sends an IRP\_MJ\_CREATE irp to io manager for the device object,  with ``/foo'' as argument
      \end{itemize}

    \item
      io manager
      \begin{itemize}
        \item
          finds the top most driver associated to this device object, (ie. ntfs)
        \item
          calls the driver dispatch routine for IRP\_MJ\_CREATE
      \end{itemize}

  \end{itemize}

\end{frame}

%
% security
%

\section{Security}


% 1)

\begin{frame}
  \frametitle{Security Building Blocks}

  NT security model relies 3 building blocks to address the
  \textbf{identification} and \textbf{permission} issues.

  \begin{itemize}

    \item
      security identifier (SID)
      \begin{itemize}
      \item
        couple (domain, account)
      \item
        given to the user at login time
      \item
        every object has an owner and a group SID
      \end{itemize}

    \item
      Access Control Entry (ACE)
      \begin{itemize}
        \item
          triple (access - deny, permissions, SID)
        \item
          grant or refuse a permission mask to a given SID. An audit feature is available too.
        \item
          permission: read, write, syncrhonize, delete, traverse...
      \end{itemize}

    \item
      Access Control List (ACL)
      \begin{itemize}
        \item
          list of ACEs
        \item
          Discretionnary Access Control List (DACL) defines object accessibility
        \item
          System Access Control List (SACL) for object audit purpose
      \end{itemize}

  \end{itemize}

\end{frame}

% 2)

\begin{frame}[containsverbatim]
  \frametitle{Security As Seen By Userland}

  \begin{itemize}
    \item
      HANDLE CreateEvent(\textbf{LPSECURITY\_ATTRIBUTES}, ...);
    \item
      HANDLE CreateMutex(\textbf{LPSECURITY\_ATTRIBUTES}, ...);
    \item
      HANDLE CreateFile(... , \textbf{LPSECURITY\_ATTRIBUTES}, ...);
    \item
      HANDLE CreateThread(\textbf{LPSECURITY\_ATTRIBUTES}, ...);
  \end{itemize}

\end{frame}

% 3)

\begin{frame}
  \frametitle{Security Lifecycle}

  At login, every process(hence, every thread) gets an access token containing its SIDs.
  When a thread requests access to an object, 2 passes algorithm:

  \begin{itemize}
    \item
      first, if the thread's SID is found in a denying ACE, access is denied
    \item
      then, an intersection is done between the requested permissions and the allowing ACEs permission
  \end{itemize}
\end{frame}

% 4)

\begin{frame}
  \frametitle{Insecurity of Windows NT}

  While NT core pieces were designed to besecure, \textbf{conceptual} and \textbf{implementation}
  holes remain, making the whole kernel unsecure. Most of the kernel exploitable bugs come from:

  \begin{itemize}
    \item
      old and unmaintained code. For instance, the graphical user interface is implemented as a kernel driver
    \item
      NT programming model complexity. Many security issues found in third party drivers
  \end{itemize}

\end{frame}


%
% IO Manager
%

\section{IO Manager}

% 1

\begin{frame}
 \frametitle{Role}
 \begin{itemize}
  \item
    management of communication between drivers
 \end{itemize}
\end{frame}

% 2

\begin{frame}
 \frametitle{Model}
 \begin{itemize}
  \item
    based on IO request packets (IRPs)
  \item
    layered model
  \item
    fully asynchronous
 \end{itemize}
\end{frame}

% 3

\begin{frame}
 \frametitle{Packet based communication}
 \begin{itemize}
  \item
    once sent to a device driver, an IRP walks the stack down
  \item
    at the end of the stack, completion routines are called in order
  \item
    not as in a microkernel
    \begin{itemize}
      \item
        every driver is in the same address space (kernel)
      \item
        not designed for isolation, but modularity
    \end{itemize}
  \item
    GUI layer does not fit this model
 \end{itemize}
\end{frame}

% 4

\begin{frame}
 \frametitle{Layered model}
 \begin{itemize}
   \item
     device drivers are linked together, forming driver stacks
   \item
     a stack models the way IO processing occurs
   \item
     nodes can be dynamically added or removed
     \begin{itemize}
       \item
         allow for dynamic stackable filesystems, VPNs, ciphering solutions
     \end{itemize}
 \end{itemize}
\end{frame}

% 5

\begin{frame}
 \frametitle{Asynchronous model}
 \begin{itemize}
   \item
     the IRP can be pend at any stage of the processing
   \item
     asynchronous IOs trivially implemented in the userland
     \begin{itemize}
       \item
         in fact, blocking IOs are implemented on top of AIOs since the very beginning
     \end{itemize}
 \end{itemize}
\end{frame}



%
% Programming For Windows NT
%

\section{Programming For Windows NT}

% 1

\begin{frame}
  \frametitle{Windows Driver Model}

 This section introduces building blocks used to implement NT device drivers. \\
 It covers the following topics:
 \begin{itemize}
 \item
   driver model
 \item
   execution
 \item
   io
 \item
   synchronization
 \item
   memory 
 \item
   development tools
 \end{itemize}

\end{frame}

% 2

\begin{frame}[containsverbatim]
 \frametitle{driver model - 1}

 Drivers have to implement a documented interface

 \begin{itemize}
  \item exported entry point

  \begin{verbatim}
   NTSTATUS DriverEntry(PDRIVER_OBJECT, PUNICODE_STRING) { ... }
  \end{verbatim}

  \item unload routine
   \begin{verbatim}
    DriverObject->DriverUnload = DriverUnload;
   \end{verbatim}

  \item A driver serves IO requests by filling a table of function pointers
   \begin{verbatim}
    DriverObject->MajorFunction[IRP_MJ_READ] = myDeviceRead;
   \end{verbatim}

  \item fast IO routines
  \begin{itemize}
   \item bypass the dispatch table for improved performances
   \item used only by file systems
  \end{itemize}

 \end{itemize}

\end{frame}

% 3

\begin{frame}
 \frametitle{driver model - 2}

 Drivers instanciate one object per device

 \begin{itemize}
  \item IoCreateDevice()
  \item device object visible in the object namespace
  \item can be opened from userland to retrieve a handle
 \end{itemize}

 Once instantiated, device object is inserted into a stack

 \begin{itemize}
  \item stack example
   \begin{itemize}
    \item socket (afd.sys)
    \item TDI (tdi.sys)
    \item TCPIP (tcpip.sys)
    \item NDIS (ndis.sys)
    \item ethernet controller (rtle8023.sys)
   \end{itemize}

 \end{itemize}

\end{frame}

% 4

\begin{frame}
 \frametitle{execution primitives - 1}

 Thread objects

 \begin{itemize}
  \item KeXxxThread routines
  \item no actual difference between user and system threads
  \item system thread address space is the kernel one
 \end{itemize}

\end{frame}

% 5

\begin{frame}
 \frametitle{execution primitives - 2}

 Asynchronous Procedure Calls (APCs)

 \begin{itemize}
  \item KeXxxApc routines
  \item used by the kernel to implement the asynchronous io model as seen by the userland
  \item per thread APC queue, routine executed in the thread context
  \item userland thread must be alertable
  \item scheduled as the thread returns from kernel
 \end{itemize}

\end{frame}

% 6

\begin{frame}
 \frametitle{execution primitives - 3}

 Deferred Procedure Calls (DPCs)

 \begin{itemize}
  \item KeXxxDpc routines
  \item one DPC queue for the entire system
  \item executed when the irql drops below DISPATCH\_LEVEL
 \end{itemize}

\end{frame}

% 7

\begin{frame}
 \frametitle{IO programming - 1}

 Every cpu has an Interrupt Request Level (IRQL)

 \begin{itemize}
  \item KeXxxIrql routines
  \item hardware level a cpu is operating at
  \item do not confuse with thread priorities
  \item implemented in software on most architecture
  \item levels lower than the current one are masked off
 \end{itemize}

\end{frame}

% 8

\begin{frame}
 \frametitle{IO programming - 2}

 Common IRQLs (from lower to higher)

 \begin{itemize}
  \item PASSIVE\_LEVEL: user threads, most kernel mode operations
  \item APC\_LEVEL: APCs, page fault
  \item DISPATCH\_LEVEL: thread scheduler, DPCs
  \item DEVICE\_INTERRUPT\_LEVEL(DIRQL): device interrupts
 \end{itemize}

\end{frame}

% 9

\begin{frame}
 \frametitle{IO programming - 3}

 Requesting a device object

 \begin{itemize}
  \item IoCallDriver(DeviceObject, Irp);
 \end{itemize}

\end{frame}

% 10

\begin{frame}
 \frametitle{IO programming - 4}

 Userland communication

 \begin{itemize}
  \item NtDeviceIoControl
  \item different buffering modes based on the control code
  \item user address validation
  \begin{itemize}
   \item common source for kernel vulnerabilities
   \item ProbeForRead(), ProbeForWrite()
   \item \_try / \_except {} blocks
  \end{itemize}
 \end{itemize}

\end{frame}

% 11

\begin{frame}
 \frametitle{IO programming - 5}

 Handling interrupts

 \begin{itemize}
  \item KeXxxInterrupt routines

  \item connect to interrupt vector, possibly shared
   \begin{itemize}
   \item IoConnectInterrupt(IsrRoutine)
  \end{itemize}

  \item bottom / top halves model
  \begin{itemize}
   \item isr performs the small work (read or write registers...)
   \item isr then schedule a dpc with IoRequestDpc() and returns
   \item dpc completes the interrupt processing
  \end{itemize}

 \end{itemize}

\end{frame}

% 12

\begin{frame}
 \frametitle{conccurency and synchronization - 1}

 Spinlock

 \begin{itemize}
  \item KeXxxSpinLock routines
  \item on up systems, irql raised to DISPATCH\_LEVEL
  \item on mp systems, same as above plus loop until acquired
 \end{itemize}

\end{frame}

% 13

\begin{frame}
 \frametitle{conccurency and synchronization - 2}

 Mutexes

 \begin{itemize}
  \item implemented at scheduler level
  \item costly, but blocking calls allowed while held
  \item use fast mutexes for better performance
  \begin{itemize}
   \item ExAcquireFastMutex(), ExReleaseFastMutex()
   \item atomic test, fast for first owner
   \item event queue otherwise
  \end{itemize}
 \end{itemize}

\end{frame}

% 14

\begin{frame}
 \frametitle{conccurency and synchronization - 3}

 Shared resources

 \begin{itemize}
  \item ExAcquireResourceSharedLite(), ExAcquireResourceExclusiveLite()
  \item single writer, multiple readers
  \item recurrent pattern in most data structure
 \end{itemize}

\end{frame}

% 15

\begin{frame}
 \frametitle{conccurency and synchronization - 4}

 InterlockedXXX routines

 \begin{itemize}
  \item InterlockedCompareExchange (compare and swap)
  \item implemented in hardware whenever possible
  \item allow for efficient, non blocking data structure implementation
  \item ExInterlockedPushEntryList()
 \end{itemize}

\end{frame}

% 16

\begin{frame}
 \frametitle{conccurency and synchronization - 5}

 Choosing the right synchronization primitive is very important

 \begin{itemize}
  \item highly impact on the overall system performance
  \item even more important mp systems
 \end{itemize}

\end{frame}

% 17

\begin{frame}
 \frametitle{memory - 1}

 Kernel virtual memory allocator

 \begin{itemize}
  \item ExAllocatePool functions family

  \item PagedPool
  \begin{itemize}
   \item may page fault on access
   \item thus may need a disk io to be served
   \item thus may block
   \item thus must not be accessed at DISPATCH\_LEVEL
   \item for instance, when a spinlock is held
  \end{itemize}

  \item NonPagedPool
  \begin{itemize}
   \item can be accessed at any IRQL
  \end{itemize}

 \end{itemize}

\end{frame}

% 18

\begin{frame}
 \frametitle{memory - 2}

 Physical memory

 \begin{itemize}
  \item MmAllocatePagesForMdl
  \item return a descriptor for further virtual access
 \end{itemize}

\end{frame}

% 19

\begin{frame}
 \frametitle{development tools - 1}

 Windows Driver Kit (WDM)

 \begin{itemize}
  \item collection of .h and .lib for driver development
  \item driver samples
 \end{itemize}

\end{frame}

% 20

\begin{frame}
 \frametitle{development tools - 2}

 Windows Research Kit (WRK)

 \begin{itemize}
  \item kernel and tools source code
  \item design books (from the 90s)
 \end{itemize}

\end{frame}

% 21

\begin{frame}
 \frametitle{development tools - 3}

 Debugging

 \begin{itemize}

  \item Windows debugging tools
  \begin{itemize}
   \item windbg 
   \item dbgview
  \end{itemize}

  \item Windows symbol packages

  \item Driver verifier
   \begin{itemize}
    \item allows for runtime checks of common errors
    \item creates exceptionnal conditions
    \item fault injection
   \end{itemize}
  \item

 \end{itemize}

\end{frame}

% 22

\begin{frame}
 \frametitle{development tools - 4}

 Other tools

 \begin{itemize}
  \item DriverLoader (osronline)
  \item DeviceTree (osronline)
 \end{itemize}
\end{frame}

% 23

\begin{frame}
 \frametitle{references}

 \begin{itemize}
  \item http://www.microsoft.com/whdc
  \item http://osronline.com/
  \item http://msdn.microsoft.com/en-us/library/aa972908.aspx
 \end{itemize}

\end{frame}


%
% Bibliography
%

\section{Bibliography}

\end{document}
