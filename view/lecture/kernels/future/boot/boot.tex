%
% ---------- header -----------------------------------------------------------
%
% project       kaneton
%
% license       kaneton
%
% file          /Users/francois/k...iew/lecture/kernels/future/boot/boot.tex
%
% created       julien quintard   [fri oct 24 17:31:58 2008]
% updated          [mon jan 26 21:56:52 2009]
%

%
% ---------- setup ------------------------------------------------------------
%

%
% path
%

\def\path{../../../..}

%
% template
%

%
% ---------- header -----------------------------------------------------------
%
% project       kaneton
%
% license       kaneton
%
% file          /home/mycure/kaneton/view/template/lecture.tex
%
% created       julien quintard   [wed may 16 18:17:26 2007]
% updated       julien quintard   [sun may 18 23:23:40 2008]
%

%
% class
%

\documentclass[8pt]{beamer}

%
% packages
%

\usepackage{pgf,pgfarrows,pgfnodes,pgfautomata,pgfheaps,pgfshade}
\usepackage[T1]{fontenc}
\usepackage{colortbl}
\usepackage{times}
\usepackage{amsmath,amssymb}
\usepackage{graphics}
\usepackage{graphicx}
\usepackage{color}
\usepackage{xcolor}
\usepackage[english]{babel}
\usepackage{enumerate}
\usepackage[latin1]{inputenc}
\usepackage{verbatim}
\usepackage{aeguill}

%
% style
%

\usepackage{beamerthemesplit}
\setbeamercovered{dynamic}

%
% verbatim stuff
%

\definecolor{verbatimcolor}{rgb}{0.00,0.40,0.00}

\makeatletter

\renewcommand{\verbatim@font}
  {\ttfamily\footnotesize\selectfont}

\def\verbatim@processline{
  {\color{verbatimcolor}\the\verbatim@line}\par
}

\makeatother

%
% -
%

\renewcommand{\-}{\vspace{0.4cm}}

%
% date
%

\date{\today}

%
% logos
%

\pgfdeclareimage[interpolate=true,width=34pt,height=18pt]
                {epita}{\path/logo/epita}
\pgfdeclareimage[interpolate=true,width=49pt,height=18pt]
                {upmc}{\path/logo/upmc}
\pgfdeclareimage[interpolate=true,width=25pt,height=18pt]
                {lse}{\path/logo/lse}

\newcommand{\logos}
  {
    \pgfuseimage{epita}
  }

%
% institute
%

\institute
{
  \inst{1} kaneton microkernel project
}

%
% table of contents at the beginning of each section
%

\AtBeginSection[]
{
  \begin{frame}<beamer>
   \frametitle{Outline}
    \tableofcontents[current]
  \end{frame}
}

%
% table of contents at the beginning of each subsection
%

\AtBeginSubsection[]
{
  \begin{frame}<beamer>
   \frametitle{Outline}
    \tableofcontents[current,currentsubsection]
  \end{frame}
}


%
% title
%

\title{Boot}

%
% document
%

\begin{document}

%
% title frame
%

\begin{frame}
  \titlepage
\end{frame}

%
% outline frame
%

\begin{frame}
  \frametitle{Outline}

  \tableofcontents
\end{frame}

%
% ---------- text -------------------------------------------------------------
%

%
% introduction
%

\section{Introduction}

\begin{frame}
  \frametitle{Introduction}

  A machine will not start an operating system automatically. There are a few steps before a kernel starts.

  \-

  This process is called the bootstrap. It is very specific to the machine, each architecture has its own bootstrap method.

  \-

  Through the bootstrap, the machine will provide some basic features that will be used to load the kernel, or by the kernel itself to drive the machine.

\end{frame}

\section{Generalities}
\subsection{CPU startup}
% 1)

\begin{frame}
  \frametitle{CPU startup}

  The CPU contains several registers that defines its behaviour.

  \-

  \begin{itemize}
  \item Program Counter
  \item Status registers
  \item \ldots
  \end{itemize}

  \-

  These registers must have a known value on reset.

  \-

  The caches must be initialized, there musn't be any random valid entries there.

  \-

  These initializations are purely hardware, a reset pin on the CPU is used to trigger this mechanism.

\end{frame}

\subsection{Firmware}

\begin{frame}
  \frametitle{Firmware}

  The CPU initializes its PC to a constant address on a reset (Reset vector). There must be some binary code here so that the CPU can execute some consistant instructions.

  \-

  This code is platform specific and is not likely to change. It is therefore in general stored in ROM chips, or in Flash EEPROM chips.

\end{frame}       

\begin{frame}
  \frametitle{Firmware role}

  In general, the firmware has several purposes :
  
  \-

  \begin{itemize}
  \item Initializing some peripherials
  \item Installing some services to ease the task of the user programs
  \item Finding a user program to run
  \item Loading the user program on the boot device and running it
  \end{itemize}

\end{frame}

\begin{frame}
  \frametitle{Peripherial initialization}

  The peripherials have their own configuration registers. They must also be configured on reset. The firmware will then configure the devices it needs :

  \-

  \begin{itemize}
  \item Video card
  \item Keyboard controller
  \item Hard disk controller
  \item {<old school>Floppy disk controller</old school>}
  \item USB Host controller
  \item Network interface
  \item \ldots
  \end{itemize}

  \-

  Since the peripherials might be different from one platform to another, and since the same peripherials can be used in several different ways with a CPU, the firmware contains basic drivers. These are specific to these peripherials and to the way they are used in a specific machine. A firmware will in general belong to one specific platform and won't probably work on another without modifications.

\end{frame}

\begin{frame}
  \frametitle{Services installation}

  To ease the task of the user program, the drivers contained in the firmware will be made available to the user program, so that the user program can have a standard way to access the base peripherials of every platform.

  \-

  The firmware will setup the RAM and the CPU so that a user program can call generic subprograms that will handle basic hardware operations.

  \-

  This interface has to be specified for a given architecture, so that a program written for this architecture can run on all compatible platforms.

\end{frame}

\begin{frame}
  \frametitle{Examples :: Openboot}

  OpenBoot (or OpenFirmware) is the firmware embedded in all Sun's
  SPARC based stations.

  \-

  OpenBoot offers many services necessary to the
  bootup phase. For example:

  \begin{itemize}
  \item
    Device-tree exploration (\emph{sibling}, \emph{child},
    \emph{getprop}\ldots)
  \item
    Device I/O (\emph{open}, \emph{read}, \emph{write}\ldots)
  \item
    MMU operations (\emph{map\_phys}, \emph{unmap\_phys}, \emph{itlb\_load},
    \emph{dtlb\_load}\ldots)
  \item
    Environment (boot path, boot device, boot arguments\ldots)
  \item
    Time (\emph{milliseconds})
  \end{itemize}

  \-

  All these functionnalities make OpenBoot a tiny OS. In addition,
  OpenBoot is able to run bytecode scripts (in Forth language), to
  load ELF files from disk or network (TFTP) and to debug programs
  (disassembly, registers dump, calls trace\ldots).

\end{frame}

\begin{frame}
  \frametitle{OpenBoot service call}

  Calling OpenBoot is as simple as jumping to the so-called
  \emph{firmware entry-point} (given in register \%g7 at boot time)
  with a structure in first argument (register \%o0). This structure is
  as follow:

  \begin{itemize}
  \item
    A pointer to a string indicating the name of the function to call.
  \item
    The number of arguments \emph{N}
  \item
    The number of return values \emph{M}
  \item
    \emph{N} 64-bit arguments
  \item
    \emph{M} 64-bit slots for results
  \end{itemize}

\end{frame}

\begin{frame}
  \frametitle{Examples :: ARCS}

  ARCS is the firmware used in all SGI MIPS-based stations. ARCS has
  also been used on a few Pentium-based stations, in replacement of
  the traditional BIOS.

  \-

  ARCS offers:

  \begin{itemize}
  \item
    Device-tree functions (\emph{GetPeer}, \emph{GetChild},
    \emph{GetComponent}\ldots)
  \item
    Device I/O (\emph{open}, \emph{read}, \emph{write}\ldots)
  \item
    Filesystem supports, FAT and ISO9660 (\emph{mount}, \emph{open},
    \emph{read}\ldots)
  \item
    Environment variables
  \item
    Time (\emph{GetTime})
  \end{itemize}

  \-

  ARCS is also able to load ELF files, from disk or network.

\end{frame}

\begin{frame}
  \frametitle{ARCS System Parameter Block}

  The System Parameter Block (SPB) is a structure filled by ARCS before giving the 
  hand to the boot program. The block contains notably the Firmware Vector, which is 
  an array of function pointers. Calling ARCS is as simple as calling the wanted function 
  from this vector. 

\end{frame}


\begin{frame}
  \frametitle{Examples :: (U)EFI}

  EFI (Extensible Firmware Interface) is a firmware specification made by Intel, and used in Itanium (IA-64) platforms. Apple also uses EFI in their new generation of Intel IA-32 machines.

  \-

  EFI has been designed to replace the BIOS that was far too limited, especially for High-end server platforms.

  \-

  An EFI firmware is mainly written in C while the BIOSes were written in ASM, and it uses 32-bit code where a BIOS used 16-bit code.

\end{frame}

\begin{frame}
  \frametitle{(U)EFI}

  EFI is still not included in standard PCs since it would break the backwards compatibility, which is a huge advantage of this architecture, compared to all others.

  \-

  EFI is extensible, it is composed of modules that can be loaded, and user can easily add or remove modules. The EFI standard describes a protocol to share services between the modules.
  
  \-

  An EFI firmware provides some advanced runtime services to the operating system, such as time, environment, device drivers for video, disk, network, filesystems, \ldots that can be called through a simple C function call from the operating system.

  \-

  EFI also provides a boot manager, that allow to select the boot device and directly start an operating system (the bootloader in that case is an EFI module), and a shell that allows to perform some basic actions.

\end{frame}

\subsection{Bootloader}

\begin{frame}
  \frametitle{Goal of the bootloader}

  A kernel, in order to start, generally requires a specific setup of the machine.

  \-

  It cannot rely on the firmware for that.

  \-

  A specific small program called bootloader will prepare the CPU for the kernel, find and load the kernel, and finally transfer the execution to the kernel.

  \-

\end{frame}

\begin{frame}
  \frametitle{Loading the kernel}

  The first task of the bootloader is to locate the kernel image. There are several possibilities: the kernel might be stored as raw data on some special blocks of a disk, it can also be stored as a file, on a filesystem, it could even be available on the network, through a TFTP server. For that reason, bootloaders may need to have some filesystem drivers, or network interface drivers with a small IP stack.

  \-

  Once the kernel has been located, it must be loaded in memory. The kernel image can take several forms. It can be an ELF binary, it can be a Windows PE file, it can be a block of raw data to load at a specific address, it can be one of the previous compressed with gzip, \ldots The bootloader must know the file format of the kernel it loads.

  \-
  
  When the kernel is loaded, the bootloader must jump on its entry point. So the entry point must be specified, either as something constant, or through an address provided in the kernel image.

\end{frame}

\begin{frame}
  \frametitle{Providing system information to the kernel}

  To boot correctly, a kernel may need to get some information from the system or from the user:

  \begin{itemize}
  \item RAM size, boot device
  \item boot options (graphic mode, root partition \ldots)
  \item modules information
  \end{itemize}

  \-

  The bootloader can pass information through to ways:

  \begin{itemize}
  \item the microprocessor registers
  \item an info structure written in memory. Most often, the structure base address is stored in a register. In some specific cases, this structure can also be found with a magic number.
  \end{itemize}

\end{frame}

\begin{frame}
  \frametitle{Bootloader specificities}

  In general, each kernel has its own dedicated bootloader, for each architecture it supports. For example : Lilo for Linux on x86, Silo for Linux on Sparc, \ldots.

  \-

  Some attempts have been made to make a standard. Grub for instance proposes the Multi-Boot Loader format. So Grub can boot every kernel that complies with the Multi-Boot Loader standard, like Linux, and FreeBSD for example.

\end{frame}

\section{x86 specificities}
\subsection{CPU startup}

\begin{frame}
  \frametitle{x86 CPU startup}

  The x86 CPU reset configures the CPU in Real-mode (16 bits, no segmentation, no paging, no privileges). This has been made to keep the compatibility within all the Intel x86 processors family.

  \- 

  The Reset vector value is 0xFFFFFFF0. This physical address is initialized by the motherboard so that it contains a jump to the firmware entry point.

\end{frame}

\subsection{Firmware}

\begin{frame}
  \frametitle{The BIOS}

  The firmware, on x86 platforms, is called BIOS (Basic Input/Output System). It is stored on a non volatile memory, and mapped by the hardware into the physical address space.

  \-

  The BIOS is a very primitive firmware, since they tried to keep the compatibility with old machines, so it didn't evolve much for years.

  \-

  The BIOS will first call the Video card's own BIOS. On x86, video cards have their own firmware that consists in x86 executable code.

  \-

  After this, the BIOS will initialize several peripherials and setup interrupt handlers to provide services (called BIOS Calls), so the user program can use these peripherials without any specific driver, by putting parameters in some registers and triggering a software interrupt.

\end{frame}

\begin{frame}
  \frametitle{The BIOS calls}

  The BIOS offers services to the bootloader/operating system. It setups some routines through a software interrupt interface.

  \-

  Several interrupt handlers are installed by the BIOS, each one concerns a type of operation, such as video, disk, floppy, \ldots.

  \-

  The actual operation is specified through the AX register, and optional parameters are passed through the other general purpose registers.

  \-

  Since the BIOS is 16-bit code, these BIOS services can only be called when the CPU is running in Real-mode. Since all modern operating systems are now running in protected mode, they can't use these BIOS calls. The VM86 mode allows to run 16-bit tasks in protected mode and is sometimes used to call BIOS calls from the protected mode, but in general, the kernels have their own device drivers and communicate directly with all the peripherials. So the BIOS calls are not really helpful to the kernels, and they are in general only used in the early boot stage, by the bootloader, before it switches to protected mode.

\end{frame}

\subsection{Bootloader}

\begin{frame}
  \frametitle{Boot sector}

  The BIOS, on the ia32 architecture, provides a way to boot from several devices, such as floppy disks or hard disks.

  \-

  1st sector of the devices (1 sector = 512 bytes)

  \-

  The bootloader will install itself in this sector.

\end{frame}

\begin{frame}
  \frametitle{Multi-stage bootloader}

  512 bytes is very small.

  \-

  Boot sector code = stage1.

  \-

  stage2+ are stored somewhere else

  \-

  stage1 role is to load stage2 (stage2 can be much bigger)

  \-

  Grub : 3 stages

\end{frame}

\begin{frame}
  \frametitle{The Multiboot standard}

  The Multiboot standard specifies an interface between a boot loader and a operating system.

  \-

  It aims at unifying kernel boot procedures in order to easily allow multiple OSes to coexist on the same machine.

  \-

  Multiboot specifications impose :

  \begin{itemize}
  \item an OS image format (to know how to load it and how to jump on it)
  \item Machine state
  \item Boot information format (structure of data passed to the kernel)
  \end{itemize}

\end{frame}

\begin{frame}
  \frametitle{Multiboot header}
  kernel image must :

  \begin{itemize}
  \item be an ordinary 32-bit executable file in the standard format for that particular operating system
  \item contain an additional header called Multiboot header. The Multiboot header must be contained completely within the first 8192 bytes of the OS image, and must be longword (32-bit) aligned.
  \end{itemize}

  \begin{tabular}{|l|l|l|l|l|}
  \hline
  Offset & Type & Field & Description & NameNote\\\hline\hline
  0 & u32 & magic & multiboot magic & required\\\hline
  4 & u32 & flags & flags & required\\\hline
  8 & u32 & checksum & checksum & required\\\hline
  12 & u32 & header\_addr & multiboot header addr & if flags[16] is set\\\hline
  16 & u32 & load\_addr & .text physical addr & if flags[16] is set\\\hline
  20 & u32 & load\_end\_addr & .data end & if flags[16] is set\\\hline
  24 & u32 & bss\_end\_addr & .bss physical end & if flags[16] is set\\\hline
  28 & u32 & entry\_addr & entry point physical addr & if flags[16] is set\\\hline
  32 & u32 & mode\_type  & graphic mode & if flags[2] is set\\\hline
  36 & u32 & width & number of columns & if flags[2] is set\\\hline
  40 & u32 & height & number of lines & if flags[2] is set\\\hline
  44 & u32 & depth & depth & if flags[2] is set\\\hline
  \end{tabular}

\end{frame}

\section{Conclusion}
\begin{frame}
  \frametitle{Conclusion}

  There is no standard bootstrap, each architecture has its own standard, that provides a way of setting up the machine, and providing services to the kernel.

  \-

  A bootloader is always required to fill the hole between the machine firmware and the operating system kernel.

  \-

  A bootloader is a quite small piece of code, but it is mandatory. A kernel cannot start without a working bootloader.

  \-

  To avoid the need of writing a new bootloader each time people is making a new kernel, some standards have been setup, but they are still not used by all the main operating systems.

\end{frame}

%
% bibliography
%

\begin{frame}[allowframebreaks]
  \frametitle{Bibliography}

  \bibliographystyle{amsplain}
  \bibliography{\path/bibliography/bibliography}
\end{frame}

\end{document}
