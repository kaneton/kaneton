%
% ---------- header -----------------------------------------------------------
%
% project       kaneton
%
% license       kaneton
%
% file          /home/mycure/kane...ure/kernels/future/security/security.tex
%
% created       julien quintard   [fri oct 24 17:31:58 2008]
% updated       julien quintard   [sat nov 29 20:41:38 2008]
%

%
% ---------- setup ------------------------------------------------------------
%

%
% path
%

\def\path{../../../..}

%
% template
%

%
% ---------- header -----------------------------------------------------------
%
% project       kaneton
%
% license       kaneton
%
% file          /home/mycure/kaneton/view/template/lecture.tex
%
% created       julien quintard   [wed may 16 18:17:26 2007]
% updated       julien quintard   [sun may 18 23:23:40 2008]
%

%
% class
%

\documentclass[8pt]{beamer}

%
% packages
%

\usepackage{pgf,pgfarrows,pgfnodes,pgfautomata,pgfheaps,pgfshade}
\usepackage[T1]{fontenc}
\usepackage{colortbl}
\usepackage{times}
\usepackage{amsmath,amssymb}
\usepackage{graphics}
\usepackage{graphicx}
\usepackage{color}
\usepackage{xcolor}
\usepackage[english]{babel}
\usepackage{enumerate}
\usepackage[latin1]{inputenc}
\usepackage{verbatim}
\usepackage{aeguill}

%
% style
%

\usepackage{beamerthemesplit}
\setbeamercovered{dynamic}

%
% verbatim stuff
%

\definecolor{verbatimcolor}{rgb}{0.00,0.40,0.00}

\makeatletter

\renewcommand{\verbatim@font}
  {\ttfamily\footnotesize\selectfont}

\def\verbatim@processline{
  {\color{verbatimcolor}\the\verbatim@line}\par
}

\makeatother

%
% -
%

\renewcommand{\-}{\vspace{0.4cm}}

%
% date
%

\date{\today}

%
% logos
%

\pgfdeclareimage[interpolate=true,width=34pt,height=18pt]
                {epita}{\path/logo/epita}
\pgfdeclareimage[interpolate=true,width=49pt,height=18pt]
                {upmc}{\path/logo/upmc}
\pgfdeclareimage[interpolate=true,width=25pt,height=18pt]
                {lse}{\path/logo/lse}

\newcommand{\logos}
  {
    \pgfuseimage{epita}
  }

%
% institute
%

\institute
{
  \inst{1} kaneton microkernel project
}

%
% table of contents at the beginning of each section
%

\AtBeginSection[]
{
  \begin{frame}<beamer>
   \frametitle{Outline}
    \tableofcontents[current]
  \end{frame}
}

%
% table of contents at the beginning of each subsection
%

\AtBeginSubsection[]
{
  \begin{frame}<beamer>
   \frametitle{Outline}
    \tableofcontents[current,currentsubsection]
  \end{frame}
}


%
% title
%

\title{Security}

%
% document
%

\begin{document}

%
% title frame
%

\begin{frame}
  \titlepage
\end{frame}

%
% outline frame
%

\begin{frame}
  \frametitle{Outline}

  \tableofcontents
\end{frame}

%
% ---------- text -------------------------------------------------------------
%

%
% introduction
%

\section{Introduction}

% 1)

\begin{frame}
  \frametitle{Overview}

  This course targets security in operating systems.

  \-

  Note that providing security inside a kernel is different.

  \-

  We will focus on describing tools for controlling the access to well-defined
  objects, these objects being created and managed by the kernel, a server,
  a service or whatever independent entity.
\end{frame}

% 2)

\begin{frame}
  \frametitle{Assumptions}

  Obviously, providing security would not make much sense in an unsafe
  system.

  \-

  We will therefore consider in this course that the studied operating system
  is safe enough so that security makes sense.

  \-

  More precisely, in systems such as critical, real-time \etc{} systems,
  ensuring safety would be far more important than providing access control
  because without strong safety guarantees, security would have an impact.
\end{frame}

% 3)

\begin{frame}
  \frametitle{Summary}

  Therefore, this course will targets \term{access control} and none of
  the following topics will be discussed:

  \-

  \begin{itemize}
    \item
      Unit testing;
    \item
      Model checking;
    \item
      Secure architecture;
    \item
      Vulnerabilities;
    \item
      \etc{}
  \end{itemize}
\end{frame}

%
% terminology
%

\section{Terminology}

% 1)

\begin{frame}
  \frametitle{Subject/Initiator}

  The \term{subject}/\term{initiator} represents the entity trying to access
  an object.

  \-

  According to the security model and the rights the subject has on the
  object, the request operation is granted.
\end{frame}

% 2)

\begin{frame}
  \frametitle{Object/Target}

  The \term{object}/\term{target} represents the information that is kept
  from being seen from unauthorised subjects through a security model.
\end{frame}

% 3)

\begin{frame}
  \frametitle{Model}

  The security model defines a set a rules and techniques on which the
  managers rely to control the access granted to subjects.
\end{frame}

% 4)

\begin{frame}
  \frametitle{Manager}

  A \term{manager} represents an entity providing access to targets.

  \-

  Managers follow the rules defined by the security model in use.
\end{frame}

% 5)

\begin{frame}
  \frametitle{Operation}

  The \term{operation} represents the action requested by a subject on
  a target.

  \-

  The manager receiving the request checks if the subject has the right
  to perform this operation and acts accordingly.
\end{frame}

% 6)

\begin{frame}
  \frametitle{Example}

  XXX file system?
\end{frame}

%
% overview
%

\section{Overview}

% 1)

\begin{frame}
  \frametitle{Overview}

  o Why?
    - obvious
  o What?
    - any ``object'' being file, kernel data, network access, shared memory etc.
\end{frame}

%
% history
%

\section{History}

% 1)

\begin{frame}
  \frametitle{Overview}

  - Just to give a glimpse of history
  o Multics: started 1964 with a first release in 1969.
             was the first system to emphasize security from the design perspective. In spite of this,
             Multics extreme complexity opened the way to many security flaws.
             however break-ins became very rare with the second generation of hardware: ring-oriented security
    o Goals: high availability, scalability (by adding resources such as main memory, storage capacity etc.)
             -> ACLs on every file
  o Cambridge CAP Computer: in the 1970s
    o Goals: complete security
             -> Capabilities both in hardware and software
  o UNIX: started in 1969 (with UNIX-PDP7)
    o Goals: portable, multi-tasking, multi-user, time sharing .. everything is file
             -> UNIX permissions: u/g/o
  o Linux: started in 1991 ... until now
    o Goals: Free and Open UNIX
             -> UNIX permissions
\end{frame}

%
% principles
%

\section{Principles}

% 1)

\begin{frame}
  \frametitle{Overview}

  o Design/Obscurity
    - Security through obscurity
        - controversial principle, especially used by governements and organizations which consists in
          relying on secrecy for ensuring security
    - Security by design: opposite to security through obscurity
  o Least Privilege
\end{frame}

%
% types
%

\section{Types}

% 1)

\begin{frame}
  \frametitle{Overview}

  o MAC
  o DAC
\end{frame}

%
% models
%

\section{Models}

% 1)

\begin{frame}
  \frametitle{Overview}

  - A computer security model is a scheme for specifying and enforcing security policies.
  o Unix Permissions: which is a specificity of ACLs
  o ACL - Access Control List
    - Insecure in many ways: Confused deputy problem
    o Examples:
      o NTFS
  o Capabilities
    - Unfortunately, only implemented in research operating systems while, *at best*, commercial
      systems use ACLs. A reason might be that fixing security with ACLs does not require
      a complete review of the system design.
    o Hardware/Software/Language
    o Examples
      o CAP
      o Amoeba
      o kaneton
      o Language E
      o TrustedBSD
  o Certificates
    o Certified Capabilities
    o Examples:
      o Singularity
  o MLS - Multi-Level Security
  o RBAC - Role-Based Access Control
  o LBAC - Lattice-Based Access Control
\end{frame}

%
% conclusion
%

\section{Conclusion}

% 1)

\begin{frame}
  \frametitle{Overview}

  XXX
\end{frame}

%
% bibliography
%

\begin{frame}
  \frametitle{Bibliography}

  \bibliographystyle{amsplain}
  \bibliography{\path/bibliography/bibliography}
\end{frame}

\end{document}
