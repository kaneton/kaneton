%
% ---------- header -----------------------------------------------------------
%
% project       kaneton
%
% license       kaneton
%
% file          /home/mycure/kane...ure/kernels/future/security/security.tex
%
% created       julien quintard   [fri oct 24 17:31:58 2008]
% updated       julien quintard   [sat jan 10 21:01:50 2009]
%

%
% ---------- setup ------------------------------------------------------------
%

%
% path
%

\def\path{../../../..}

%
% template
%

%%
%% copyright     (c) julien quintard
%%
%% project       kaneton
%%
%% file          /home/mycure/kaneton/view/templates/lecture.tex
%%
%% created       julien quintard   [sat nov 19 17:13:03 2005]
%% updated       julien quintard   [fri dec  2 22:36:34 2005]
%%

%
% class
%

\documentclass[8pt]{beamer}

%
% packages
%

\usepackage{pgf,pgfarrows,pgfnodes,pgfautomata,pgfheaps,pgfshade}
\usepackage{colortbl}
\usepackage{times}
\usepackage{amsmath,amssymb}
\usepackage{graphics}
\usepackage{graphicx}
\usepackage{color}
\usepackage{xcolor}
\usepackage[english]{babel}
\usepackage{enumerate}
\usepackage[latin1]{inputenc}

%
% style
%

\usepackage{beamerthemesplit}
\setbeamercovered{dynamic}

%
% verbatim font
%

\definecolor{verbatimcolor}{rgb}{0,0.4,0}

\makeatletter
\renewcommand{\verbatim@font}
  {\ttfamily\footnotesize\color{verbatimcolor}\selectfont}
\makeatother

%
% new line
%

\newcommand{\nl}[0]{\vspace{0.4cm}}

%
% date
%

\date{\today}

%
% logos
%

\pgfdeclareimage[interpolate=true,width=34pt,height=18pt]
                {epita}{../../logos/epita}
\pgfdeclareimage[interpolate=true,width=49pt,height=18pt]
                {upmc}{../../logos/upmc}
\pgfdeclareimage[interpolate=true,width=25pt,height=18pt]
                {lse}{../../logos/lse}

\newcommand{\logos}
  {
    \pgfuseimage{epita}
  }

%
% institute
%

\institute
{
  \inst{1} kaneton microkernel project
}

%
% table of contents at the beginning of each section
%

\AtBeginSection[]
{
  \begin{frame}<beamer>
   \frametitle{Outline}
    \tableofcontents[current]
  \end{frame}
}

%
% table of contents at the beginning of each subsection
%

\AtBeginSubsection[]
{
  \begin{frame}<beamer>
   \frametitle{Outline}
    \tableofcontents[current,currentsubsection]
  \end{frame}
}


%
% title
%

\title{Security}

%
% document
%

\begin{document}

%
% title frame
%

\begin{frame}
  \titlepage
\end{frame}

%
% outline frame
%

\begin{frame}
  \frametitle{Outline}

  \tableofcontents
\end{frame}

%
% figures
%

\pgfdeclareimage[interpolate=true,width=180pt]
                {amoeba-capability}
                {figures/amoeba-capability}

%
% ---------- text -------------------------------------------------------------
%

%
% introduction
%

\section{Introduction}

% 1)

\begin{frame}
  \frametitle{Overview}

  This course targets security in operating systems.

  \-

  Note that providing security inside the kernel is a different topic.

  \-

  We will focus on describing techniques for controlling the access to
  well-defined objects; these objects being created and managed by the kernel,
  a server, a service or whatever independent entity.
\end{frame}

% 2)

\begin{frame}
  \frametitle{Assumptions}

  Obviously, providing security would not make much sense in an unsafe
  system.

  \-

  We will therefore consider in this course that the studied operating system
  is safe enough so that security makes sense.

  \-

  More precisely, in systems such as critical, real-time \etc{},
  ensuring safety would be far more important than providing access control
  because without strong safety guarantees, security would not have an impact.
\end{frame}

% 3)

\begin{frame}
  \frametitle{Summary}

  This course will therefore target \term{access control} and none of
  the following topics will be discussed:

  \-

  \begin{itemize}
    \item
      Unit testing;
    \item
      Model checking;
    \item
      Secure architecture;
    \item
      Vulnerabilities;
    \item
      \etc{}
  \end{itemize}
\end{frame}

%
% terminology
%

\section{Terminology}

% 1)

\begin{frame}
  \frametitle{Subject/Initiator}

  The \term{subject}/\term{initiator} represents the entity trying to access
  an object.

  \-

  According to the security model and the rights the subject has on the
  object, the requested operation is granted.
\end{frame}

% 2)

\begin{frame}
  \frametitle{Object/Target}

  The \term{object}/\term{target} represents the information that is kept
  from being seen from unauthorised subjects through a security model.
\end{frame}

% 3)

\begin{frame}
  \frametitle{Policy}

  The security \term{policy} defines a set a rules and techniques on which the
  managers rely to control the access granted to subjects.
\end{frame}

% 4)

\begin{frame}
  \frametitle{Manager}

  A \term{manager} represents an entity providing access to targets.

  \-

  Managers follow the rules defined by the security policy in use.
\end{frame}

% 5)

\begin{frame}
  \frametitle{Operation}

  The \term{operation} represents the action requested by a subject on
  a target.

  \-

  The manager receiving the request checks whether the subject has the right
  to perform this operation and acts accordingly.
\end{frame}

% 6)

\begin{frame}
  \frametitle{Example}

  On \name{Facebook}, users can decide to restrict their profile's access
  being to friends, friends of friends, everyone \etc{}

  \-

  In such a context:

  \begin{itemize}
    \item
      Users are initiators;
    \item
      Browsing your profile is the operation;
    \item
      Your profile is the subject;
    \item
      \name{Facebook}'s website is the manager; and
    \item
      The security policy allows you to define who is authorised to access
      your information
  \end{itemize}
\end{frame}

%
% overview
%

\section{Overview}

% 1)

\begin{frame}
  \frametitle{Why}

  Controlling access to objects is quite obvious although it can be for
  different reasons: privacy, security \etc{}
\end{frame}

% 2)

\begin{frame}
  \frametitle{What}

  The type of objects the system provides access control for varies from:

  \begin{itemize}
    \item
      Files;
    \item
      Kernel data;
    \item
      Network accesses;
    \item
      Shared memory;
    \item
      Devices;
    \item
      \etc{}
  \end{itemize}
\end{frame}

%
% history
%

\section{History}

% 1)

\begin{frame}
  \frametitle{Overview}

  This section is intended to give a glimpse of history to students.
\end{frame}

% 2)

\begin{frame}
  \frametitle{Multics}

  \name{\term{Multics} - Multiplexed Information and Computing Service}
  started in $1964$ by the \name{MIT}, \name{AT\&T Bell Labs} and
  \name{General Electric} was first released in $1969$.

  \-

  \name{Multics} was the first system to emphasize security from the
  design perspective.

  \-

  In spite of this, \name{Multics} extreme complexity opened the way to many
  security flaws.

  \-

  However, break-ins became very rare with the second generation of hardware
  that enabled \term{ring-oriented} security.
\end{frame}

% 3)

\begin{frame}
  \frametitle{Multics}

  The goals of \name{Multics} were to provide:

  \begin{itemize}
    \item
      High availability;
    \item
      Scalability, by adding resources such as main memory, storage capacity
      \etc{};
  \end{itemize}
\end{frame}

% 4)

\begin{frame}
  \frametitle{Multics}

  \name{Multics} achieved access control through the use of \name{ACL - Access
  Control List}s on every file providing flexible information sharing and
  complete privacy when required.

  \-

  Through such a technique, fine grained access control was provided to users.
\end{frame}

% 5)

\begin{frame}
  \frametitle{CAP Computer}

  The Cambridge \term{CAP Computer} was an experimental computer built
  in the $1970$s at the \name{University of Cambridge, Computer Laboratory}.
\end{frame}

% 6)

\begin{frame}
  \frametitle{CAP Computer}

  The objective of this system was to provide complete security.
\end{frame}

% 7)

\begin{frame}
  \frametitle{CAP Computer}

  The \name{CAP Computer} achieved security through the use of capabilities
  both in hardware and software.
\end{frame}

% 8)

\begin{frame}
  \frametitle{UNIX}

  \term{UNIX}, started in $1969$ with \name{UNIX-PDP7} was developed by
  \name{Ken Thompson} with the help of \name{Dennis Ritchie} as an alternative
  to \name{Multics}.

  \-

  Although the first objective was to make a game run faster, \name{UNIX}
  evolved to a complete operating system and was rewritten in \name{C}
  in $1973$.
\end{frame}

% 9)

\begin{frame}
  \frametitle{UNIX}

  The goals of \name{UNIX} were to provide:

  \begin{itemize}
    \item
      Portability;
    \item
      Multi-tasking;
    \item
      Multi-user interaction;
    \item
      Time sharing;
  \end{itemize}

  \-

  Most of these concepts were provided through the concept of \term{file}.
\end{frame}

% 10)

\begin{frame}
  \frametitle{UNIX}

  \name{UNIX} introduced a new way of providing access control.

  \-

  First, since, in \name{UNIX}, everything is a file, this new scheme applies
  to files.

  \-

  This scheme consists of three bits \term{rwx} for \name{read}, \name{write}
  and \name{execution} while every set of bits is provided for the following
  entities \term{ugo} - \name{user}, \name{group} and \name{other}.

  \-

  This scheme provided a perfect trade-off between the size taken by access
  control metadata and the granularity allowed.
\end{frame}

% 11)

\begin{frame}
  \frametitle{Linux}

  \term{Linux} started in $1991$ by \name{Linus Torsvald} mainly based on
  \name{MINIX}.

  \-

  \name{Linux} being a clone of \name{UNIX}, the access control scheme
  remains based on the \name{UNIX} permissions.
\end{frame}

%
% principles
%

\section{Principles}

% 1)

\begin{frame}
  \frametitle{Overview}

  In this section, we define a few security principles

  \begin{enumerate}
    \item
      \term{Security Through Obscurity/Transparency}
    \item
      \term{Security Through Design};
    \item
      \term{Least Privilege}.
  \end{enumerate}
\end{frame}

% 2)

\begin{frame}
  \frametitle{Security Through Obscurity/Transparency}

  Security through obscurity is a principle which attempts to use secrecy of
  design, implementation \etc{} to provide or enhance security.

  \-

  A system relying on security through obscurity may have theoretical or
  actual security vulnerabilities, but its owners or designers believe that
  the flaws are not known, and that attackers are unlikely to find them.
\end{frame}

% 3)

\begin{frame}
  \frametitle{Security Through Obscurity/Transparency}

  Keeping details of the system secret does not make it more robust, but
  indeed increases the effort an attacker has to put in order to find such
  security flaws.

  \-

  The related security through transparency philosophy suggests that security
  flaws should be disclosed as soon as possible.
\end{frame}

% 4)

\begin{frame}
  \frametitle{Security Through Obscurity/Transparency}

  Assuming that perfect or ``unbroken'' solutions may be difficult to obtain
  and that relying solely on security through obscurity is a very poor design
  choice, keeping secret some of the details of an otherwise well-engineered
  system may be a reasonable tactic of defense.

  \-

  For example, the goal might be to simply reduce the short-run risk of
  exploitation of a vulnerability in the main components of the system while
  a security fix is implemented.

  \-

  A variant, especially in computer networks, is to detect attackers trying
  to identify potential security flaws. Since this stage is necessary, the
  system can use it to detect and block attackers.
\end{frame}

% 5)

\begin{frame}
  \frametitle{Security Through Obscurity/Transparency}

  All in all, there is a general consensus that security through obscurity
  should never be used as a primary security measure.

  \-

  It is, at best, a secondary measure; and disclosure of the obscurity should
  not result in a compromise.

  \-

  In contrary, domains such as cryptogragraphy rely on the
  \name{full disclosure} philosophy so that, given everything but the
  cryptographic key, the system remains secure.

  \-

  In conclusion, although the techniques stand in contrast with each other,
  many real-world projects include elements of both strategies.
\end{frame}

% 6)

\begin{frame}
  \frametitle{Security By Design}

  Security by design means that the software has been designed from the ground
  up to be secure.

  \-

  Malicious practices are taken for granted and care is taken to minimize
  impact when a security vulnerability is discoverd.

  \-

  The most common security design practices when it comes to user inputs is
  to avoid buffer overflows and format string vulnerabilities, among others.
\end{frame}

% 7)

\begin{frame}
  \frametitle{Least Privilege}

  The principle of least privilege requires that every module such as a
  process, a user or a program must be able to access only such information
  and resources that are necessary to its purpose.

  \-

  For example, in most operating systems such as \name{Linux} and
  \name{Windows}, the code running in \name{kernel mode} runs with maximum
  privileges; therefore there is no security enforcement.

  \-

  The same goes for code running in \name{kernel mode} when it comes to
  accessing the hardware. The \name{CAP Computer} was the first system
  to introduce access control at the hardware level.

  \-

  For example, given a user $U$ with \name{read/write} access on a file
  $F$ violates the least privilege principle if $U$ can complete his
  tasks with only the \name{read} permission.
\end{frame}

% 8)

\begin{frame}
  \frametitle{Least Privilege}

  Although the principle of least privilege is recognised as being a
  good design principle, true least privilege is, in practice, impossible
  to enforce.

  \-

  Indeed, there is no method to evaluate a process to define the least
  amount of privilege it will ever need.

  \-

  Therefore, in practice, the best that can be done is to restrict the
  privileges to eliminate some we casually predict that it will never need.

  \-

  For example, we know that \name{Notepad} should never access the network.

  \-

  In conclusion, this principle is theoretically very interesting but
  difficult to put in practice.
\end{frame}

%
% categories
%

\section{Categories}

% 1)

\begin{frame}
  \frametitle{Overview}

  In this section, we briefly define the two categories of access control,
  namely:

  \begin{itemize}
    \item
      \name{\term{MAC} - Mandatory Access Control};
    \item
      \name{\term{DAC} - Discretionary Access Control}.
  \end{itemize}
\end{frame}

% 2)

\begin{frame}
  \frametitle{MAC}

  \name{MAC} referes to a category of access control by which \textit{``the
  operating system constrains the ability of a subject to access an object.''}.

  \-

  With \name{MAC}, the security policy is centrally controlled by an
  administrator, in that case, the operating system.

  \-

  Indeed, users do not have the ability to override the policy by, for
  example, passing the access permission to another user.

  \-

  \name{MAC} is particularly interesting for implementing an organization-wide
  security policy that is guaranteed, in principle, to be enforced by all
  users.

  \-

  Note that, historically, \name{MAC} has been closely associated with
  military systems, referred as \textit{``a means of restricting access to
  objects based on the sensitivity of the information''}.
\end{frame}

% 3)

\begin{frame}
  \frametitle{DAC}

  \name{DAC} define access control as a \textit{``means of restricting access
  to objects based on the identity of subjects and/or groups to which they
  belong''}.

  \-

  The controls are \term{discretionary} in the sense that a subject with a
  certain access permission is capable of passing this permission to any other
  subject.

  \-

  The permissions are often controlled by the owner of the objects though
  this is not mandatory. Indeed, the notion of owner is not tightly linked
  to the \name{DAC} policy.

  \-

  The traditional \name{UNIX} system of users, groups and \name{rwx}
  permissions is an example of \name{DAC}.
\end{frame}

% 4)

\begin{frame}
  \frametitle{Conclusion}

  Note that systems can implement both \name{MAC} and \name{DAC}
  simultaneously.

  \-

  In such a system for example, \name{DAC} would refer to one category of
  access control where subjects can transfer permissions to one another.

  \-

  One the other side, \name{MAC} would refer to a second category of access
  control that impose constraints upon the first.

  \-

  As an example, \name{kaneton} provides both categories. Indeed, the
  capabilities implement the \name{DAC} policy while the kernel implements
  the \name{MAC} policy in order to give access to devices, mapped memory
  regions \etc{} to precise tasks.
\end{frame}

%
% models
%

\section{Models}

% 1)

\begin{frame}
  \frametitle{Overview}

  In this section, the different models of access control are presented,
  detailled and illustrated by a few examples.
\end{frame}

% 2)

\begin{frame}
  \frametitle{Owner}

  The most basic access control model must be the one used by a system where
  the object owner/creator only has the right to access it.

  \-

  Obviously, such a scheme is not much of interest.
\end{frame}

% 3)

\begin{frame}
  \frametitle{UNIX Permissions}

  The \name{UNIX} permissions where introduced with the introduction of
  the \name{UNIX} operating system.

  \-

  At that time, computers had little main memory and little storage capacity.
  As such, designing the file system's access control was not as easy as
  thought.

  \-

  The solution picked was, at the time, a very good trade-off between
  expressivity and complexity especially in terms of storage capacity.

  \-

  Indeed, while a user cannot choose explicitely which users can access one of
  her file and with which permissions, the \name{UNIX} scheme relies on the
  concept that a file belongs both to a user, its owner---probably its
  creator---and a group.
\end{frame}

% 4)

\begin{frame}[containsverbatim]
  \frametitle{UNIX Permissions}

  As such, each file has, basically, $9$ bits of permissions, three sets of
  $3$ bits \name{rwx} for the owner, group and others, meaning users that are
  not the owner and that are not member of the group.

  \-

  \begin{center}
    \begin{tabular}{ccc}
      \textbf{u}ser & \textbf{g}roup & \textbf{o}ther \\
      \verb|r w x| & \verb|r w x| & \verb|r w x| \\
    \end{tabular}
  \end{center}

  \-

  Note that, in \name{UNIX}, the system administrator is the only user allowed
  to create groups, making difficult for users to organise users by themselves.

  \-

  Furthermore, the group space is \term{flat} meaning that there is no
  hierarchy \ie{} a group cannot be a member of another group. This detail
  also complicates organising users in large systems.

  \-

  In conclusion, this system may lack flexibility especially when it comes to
  managing fine grain accesses but also for large systems where organising
  users becomes difficult.
\end{frame}

% 5)

\begin{frame}
  \frametitle{ACL :: Overview}

  An \name{\term{ACL} - Access Control List} is a list of permissions
  attached to an object.

  \-

  The list specifies who or what is allowed to access the object and what
  operations are permitted. Typically, each \name{ACL} entry specifies a
  subject and a list of allowed operations.
\end{frame}

% 6)

\begin{frame}
  \frametitle{ACL :: Limitation}

  The question of how access control lists are edited depends on the system.

  \-

  Systems said to have discretionary access control let the creator or owner
  of the object manage the list of permissions while systems with mandatory
  access control enforce system-wide restrictions, overriding the permissions
  stated in the \name{ACL}s.

  \-

  \name{ACL} suffer from the same limitation than \name{UNIX} permissions.
  Indeed, in the context of large systems with many users, it can become
  cumbersome to assign permissions to every individual \etc{}
\end{frame}

% 7)

\begin{frame}
  \frametitle{ACL :: Example}

  The \name{NTFS - NT File System} provides \name{ACL}s with extended
  operations such as editing, listing a directory \etc{} in addition to the
  well-known read, write and execute.

  \-

  Most of the \name{UNIX}-like operating systems such as \name{Linux},
  \name{Solaris} \etc{} support the so-called \name{POSIX.1e} standard although
  it has been abandoned.

  \-

  Therefore, many operating systems now implement the \name{ACL} system
  provided by \name{NFS - Network File System} \name{v4}.
\end{frame}

% 8)

\begin{frame}
  \frametitle{Confused Deputy Problem :: Overview}

  Let's consider a service waiting for two arguments: the name of an input
  file and the name of an output file.

  \-

  The service provided consists in compiling the first file, generating the
  second one.

  \-

  Assuming that the service has the permission to read and write the
  \location{/etc/passwd} file while the user does not; what would happen
  if this file was given as output?

  \-

  The service will process the given input file and overwrite the
  \location{/etc/passwd} file which the user did not have the permission
  to write.
\end{frame}

% 9)

\begin{frame}
  \frametitle{Confused Deputy Problem :: Explanation}

  When the file name was passed from the client to the server, the permission
  did not go along with it.

  \-

  Therefore, the permission was increased by the system silently and
  automatically.

  \-

  This is a way to perform a \name{privilege escalation} attack.

  \-

  This kind of attacks is well-known on \name{UNIX}-like operating systems
  via \name{setuid} executables.

  \-

  Note that in that case, both \name{UNIX} permissions and \name{ACL}s
  suffer from this problem.

  \-

  The solution would be for the service to check if the user has the right to
  access the given files but such a solution would require explicit attention
  to security by the server and most services would take that extra measure.
\end{frame}

% 10)

\begin{frame}
  \frametitle{Capabilities :: Overview}

  A \term{capability} is a communicable, unforgeable token of authority.

  \-

  A capability references an object along with an associated set of access
  rights.

  \-

  Therefore, processes willing to perform an action, instead of identifying
  the object, pass the capability as an argument. Thus, the receiving process
  will verify the validity of this capability before extracting both
  the object identifier and the access rights.

  \-

  An interesting concept of capabilities is also to provide a capability
  holder the possibility to \term{attenuate} it \ie{} to restrict the
  a given capability's permissions, creating a new capability that could
  then be distributed.
\end{frame}

% 11)

\begin{frame}
  \frametitle{Capabilities :: Least Privilege}

  Capability-based system refer to the principle of sharing capability with
  each other according to the principle of least privilege.

  \-

  Indeed, in the example of the confused deputy problem, the client could
  provide the service two capabilities: one for the input file and the other
  one for the output file.

  \-

  Thus, the service would not have any way of overwriting the
  \location{/etc/passwd} unless the input capability references this file;
  which is unlikely as we assumed the client did not have the permission
  to read this file.

  \-

  In conclusion, we can clearly see that the service, by using the received
  capabilities, cannot by tricked in performing illegal operations.
\end{frame}

% 12)

\begin{frame}
  \frametitle{Capabilities :: Resemblance}

  Although most operating systems implement a facility which resembles
  capabilities, they typically do not provide enough support to allow
  for exchange of capabilities among possibly mutually untrusting entities.

  \-

  In contrast, a capability-based system is designed with that goal in mind.

  \-

  Indeed, for example, the \name{POSIX 1e/2c} capabilities are coarse-grained
  privileges that cannot be transferred between processes; hence are not
  capabilities.
\end{frame}

% 13)

\begin{frame}
  \frametitle{Capabilities :: Examples}

  Capabilities have been implemented in many systems, even languages.

  \-

  The secure distributed computing language \term{E} was designed with
  capabilities in mind. Indeed, the capability concept is closely related
  to the notion of object-oriented programming.

  \-

  Operating systems such as the historical suite \name{KeyKOS}, \name{EROS},
  \name{CapROS} and \name{Coyotos} all used capabilities from the ground up.

  \-

  The \name{IBM}'s operating system \name{System 250} integrated the notion
  of capability addressing. Processes needed a capability to access a region
  of memory. Unfortunately, such a technique could drastically slow down the
  system.

  \-

  Another example is \name{Google} \name{Caja} which allows websites to
  exchange information with a fine-grain access. Imagine a user drag-dropping
  her \name{GMail} contact list on \name{Facebook}, creating a capability
  so that the \name{Facebook} website can access these contacts on the
  \name{GMail} website without interfering with anything else on the user's
  \name{GMail} account.
\end{frame}

% 14)

\begin{frame}
  \frametitle{Capabilities :: Amoeba}

  Like \name{kaneton}, the distributed operating system \name{Amoeba}, designed
  by \name{Pr.} \name{Andrew Tanenbaum} is based on the concept of
  capabilities.

  \-

  \begin{center}
    \pgfuseimage{amoeba-capability}
  \end{center}

  \-

  When an object is created, the server pichs a random \name{Check} field and
  store it both in the new capability and inside its own table. Besides,
  all the \name{Rights} bits are initially on, and it is this \name{owner
  capability} that is returned to the client.

  \-
  
  Then, when the capability is sent back to the server in a request to perform
  an operation, the \name{Check} field is verified.
\end{frame}

XXX diff avec certificats: simple niveau computation: hash vs signature verif

XXX
    - Unfortunately, only implemented in research operating systems while, *at best*, commercial
      systems use ACLs. A reason might be that fixing security with ACLs does not require
      a complete review of the system design.

      o Amoeba
      o kaneton
      o Google Caja
  o Certificates
    o Certified Capabilities
    o Examples:
      o Singularity
  o RBAC - Role-Based Access Control
XXX

%
% conclusion
%

\section{Conclusion}

% 1)

\begin{frame}
  \frametitle{Overview}

  XXX
\end{frame}

%
% bibliography
%

\begin{frame}
  \frametitle{Bibliography}

  \bibliographystyle{amsplain}
  \bibliography{\path/bibliography/bibliography}
\end{frame}

\end{document}

Exam: repertoires apache/ subversion/ et trac/. chacun contient des sous-rep
  pour les users. on veut restreindre l'acces des repertoires users aux autres
  users *mais* on veut que: (i) le rep apache d'un user soit accessible
  en lecture par le user apache (ii) le rep subversion d'un user soit
  accessible en lecture/ecriture par le user subversion et en lecture par
  le user apache (iii) le repertoire trac d'un user soit accessible en
  lecture/ecriture par le user apache.
  de plus bien sur chaque user doit pouvoir acceder son propre repertoire
  personnel de chaque repertoire apache, subversion et trac et les autres
  doivent en etre exclus.
  proposez une solution, la plus simple possible qui fournisse tout cela.
  cet exemple est tres concret!
