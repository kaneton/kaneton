%
% ---------- header -----------------------------------------------------------
%
% project       kaneton
%
% license       kaneton
%
% file          /home/mycure/kane...ernels/virtualization/virtualization.tex
%
% created       julien quintard   [wed may 16 19:28:59 2007]
% updated       julien quintard   [tue may  5 10:49:43 2009]
%

%
% ---------- setup ------------------------------------------------------------
%

%
% path
%

\def\path{../../..}
\def\figurepath{.}

%
% template
%

%
% ---------- header -----------------------------------------------------------
%
% project       kaneton
%
% license       kaneton
%
% file          /home/mycure/kaneton/view/template/lecture.tex
%
% created       julien quintard   [wed may 16 18:17:26 2007]
% updated       julien quintard   [sun may 18 23:23:40 2008]
%

%
% class
%

\documentclass[8pt]{beamer}

%
% packages
%

\usepackage{pgf,pgfarrows,pgfnodes,pgfautomata,pgfheaps,pgfshade}
\usepackage[T1]{fontenc}
\usepackage{colortbl}
\usepackage{times}
\usepackage{amsmath,amssymb}
\usepackage{graphics}
\usepackage{graphicx}
\usepackage{color}
\usepackage{xcolor}
\usepackage[english]{babel}
\usepackage{enumerate}
\usepackage[latin1]{inputenc}
\usepackage{verbatim}
\usepackage{aeguill}

%
% style
%

\usepackage{beamerthemesplit}
\setbeamercovered{dynamic}

%
% verbatim stuff
%

\definecolor{verbatimcolor}{rgb}{0.00,0.40,0.00}

\makeatletter

\renewcommand{\verbatim@font}
  {\ttfamily\footnotesize\selectfont}

\def\verbatim@processline{
  {\color{verbatimcolor}\the\verbatim@line}\par
}

\makeatother

%
% -
%

\renewcommand{\-}{\vspace{0.4cm}}

%
% date
%

\date{\today}

%
% logos
%

\pgfdeclareimage[interpolate=true,width=34pt,height=18pt]
                {epita}{\path/logo/epita}
\pgfdeclareimage[interpolate=true,width=49pt,height=18pt]
                {upmc}{\path/logo/upmc}
\pgfdeclareimage[interpolate=true,width=25pt,height=18pt]
                {lse}{\path/logo/lse}

\newcommand{\logos}
  {
    \pgfuseimage{epita}
  }

%
% institute
%

\institute
{
  \inst{1} kaneton microkernel project
}

%
% table of contents at the beginning of each section
%

\AtBeginSection[]
{
  \begin{frame}<beamer>
   \frametitle{Outline}
    \tableofcontents[current]
  \end{frame}
}

%
% table of contents at the beginning of each subsection
%

\AtBeginSubsection[]
{
  \begin{frame}<beamer>
   \frametitle{Outline}
    \tableofcontents[current,currentsubsection]
  \end{frame}
}


%
% title
%

\title{Advanced x86}

%
% table of contents at the beginning of each section
%

\AtBeginSection[]
{
  \begin{frame}<beamer>
    \frametitle{Outline}

    \begin{multicols}{2}
      \tableofcontents[current,hidesubsections]
    \end{multicols}
  \end{frame}
}

%
% table of contents at the beginning of each subsection
%

\AtBeginSubsection[]
{
  \begin{frame}<beamer>
    \frametitle{Outline}

    \begin{multicols}{2}
      \tableofcontents[current,hidesubsections]
    \end{multicols}
  \end{frame}
}

%
% document
%

\begin{document}

%
% title frame
%

\begin{frame}
  \titlepage
\end{frame}

%
% outline frame
%

\begin{frame}
  \frametitle{Outline}

  \begin{multicols}{2}
    \tableofcontents
  \end{multicols}
\end{frame}

  % -)


\section{CPU}

\subsection{Execution and privilege levels}
\subsubsection{x86 architecture overview}
\subsubsection{Execution modes}
\begin{frame}
        \frametitle{Execution modes}
        \begin{itemize}
        \item Protected Mode
                \begin{itemize}
                \item 32 bits
                \item 4 privelege levels (from 0 to 3)
                \item Nominal mode for the CPU
                \item Gives access to special functionalities: MMU,
                      syscalls, task management, interupt management, ...
                \end{itemize}
        \item Long Mode
                \begin{itemize}
                \item 64 bits
                \item Similar to the protected mode
                \item Execution bit
                \end{itemize}
        \item Real Mode
                \begin{itemize}
                \item Mode in which the CPU is in when powered on
                \item The BIOS and the bootloader runs in Real mode
                \item no privilege separation (1 level)
                \end{itemize}
        \item SMM Mode
                \begin{itemize}
                \item System Management Mode
                \item "Maintenance" mode
                \item TOTAL control of the platform
                \item no privilege separation (1 level)
                \item Access to advance platform functionaly such as
                      power management, brightness control, custom OEM
                      keys, ...
                \end{itemize}
        \end{itemize}
\end{frame}

\subsubsection{Privilege levels}
        \begin{frame}
        \frametitle{Privilege levels (intro)}
        \begin{itemize}
                \item Started with 80286 (80s)
                \item 4 rings 0, 1, 2 and 3. 0 is the most privileged.
                \item Separation between Kernel, Drivers, Services and
                      Applications but most OSes use only 2 (0 and 3).
                \item Some feature of the CPU are only accessible
                within the highest privileged mode such as the MMU, I/O,
                ...
                \item CPL (Current Privileged Level) indicate the mode
                      in which the cpu is currently running.
                \item In comparason the ARM CPU only have two privileged
                      level.
        \end{itemize}
        \end{frame}

        \begin{frame}
        \frametitle{Changing privilege level}
        \begin{itemize}
                \item Privileged Level is defined in a flag on the Code
                Segment Selector
                \item A system even cause the privelege level to change
                      like a interruption or an exception
                \item Each interruption or exception handling code is
                      associated with a "to-be-run-in" flag that
                      indicate in which mode the CPU must switch to
                      handle the event.
                \item The "to-be-run-in" flags lives in the IDT
                (Interupt Description Table)
                \item A syscal is software exception raised by a less
                      privileged level to request an action.
                \item Starting with the Pentium II we access to a
                      delicated instructions to make a syscal
                      (SYSENTER/SYSEXIT SYSCALL/SYSRET)
        \end{itemize}
        \end{frame}
\subsubsection{Device access control}
        \begin{frame}
        \frametitle{Device access control}
        \begin{itemize}
                \item I/O instructions like in/out only works in ring0
                \item However the CPU expose a bitmap to let the OS
                control which port will be accessible from different
                priveleged level.
                \item The bitmap is configurable throught the iopl
                instruction.
                \item On linux the syscal iopl(2) expose this
                functionality to userspace (only to root)
        \end{itemize}
        \end{frame}

\subsection{Memory Management}
        \begin{frame}
        \frametitle{Memoy Management Unit}
        \begin{itemize}
                \item The MMU is a componant of the system that can
                translate system addresse (address in your code) into
                bus address (address on the bus address of the CPU)
                \item The MMU can be programmed only from ring 0
                \item Isolate the different address spaces running on
                the same system
                \item A virtual address is the address that the process will see.
                \item A physical address is the address that the comes
                out of the MMU. The CPU use the physical address on its
                bus address.

        \end{itemize}
        \end{frame}
\subsubsection{Segmentation}
        \begin{frame}
        \frametitle{Segmentation}
        \begin{itemize}
                \item Started with 80286
                \item Segments are used to split memory into contigus
                chunks.
                \item A segment is composed of:
                        \begin{itemize}
                        \item Type (code, data or special)
                        \item And offset (in RAM) and a size
                        \item Privilege (RPL: Requested privilege level)
                        \item Access Control (read or/and write)
                        \end{itemize}
                \item ring0 can change the segment with registers like
                cs, ds, ss
                \item Each process will only see inside their segment
                \item The address inside a segment will be translated be
                the MMU before it hits the bus address
                \item The transaltion is such as bus@ = @ + seg.offset
        \end{itemize}
        \end{frame}
\subsubsection{Paging}
        \begin{frame}
        \frametitle{Paging}
        \begin{itemize}
                \item Started with 80386
                \item Fixe fragmentation caused by segmentation
                \item Split memory into pages (4K, 2M, 1G)
                \item Each address space is defined by a set of
                tables that maps to virtual pages to physical pages
                \item Different mode (32b, PAE, 64b).
        \end{itemize}
        \end{frame}

        \begin{frame}
        \frametitle{Paging fact and sizes}
        \begin{itemize}
                \item 32 bits
                        \begin{itemize}
                        \item Virtual address: 32 bits
                        \item Physical address: 32 bits -> Max 4G
                        \item 2 levels of page table
                        \end{itemize}
                \item PAE (Physical Address Extension)
                        \begin{itemize}
                        \item Virtual address: 32 bits
                        \item Physical address: 36 bits -> Max 64G
                        \item 3 levels of page table
                        \end{itemize}
                \item 64 bits
                        \begin{itemize}
                        \item Virtual address: 48 bits
                        \item Physical address: 36-48 bits (Depends on
                        the CPU)
                        \item 4 levels of page table
                        \end{itemize}
        \end{itemize}
        \end{frame}
\subsubsection{e820}
        \begin{frame}
        \frametitle{e820}
        \begin{itemize}
                \item Table that descibe the physical layout of the RAM.
                \item e820 is setup by the BIOS
                \item e820 could change is you: select some BIOS options,
                add some RAM, add a device, ...
                \item All the memory regions are not necessary RAM
                \item On x86 the first 1M is reserved for the system.
                0-0xfffff
                        \begin{itemize}
                        \item BIOS (usually 0xf0000)
                        \item BIOS extensions (PXE ROM, RAID, ...) 0xe0000
                        \item VGA buffers (graphic:0xa0000 text:0xb7000)
                        \item VGA BIOS 0xc0000
                        \end{itemize} 
                \item Some devices are programme through mmio so they
                need a memory region: LAPIC,IOAPIC,HPET,...
                \item PCI bus (usually 0xf0000000, below 4G)
        \end{itemize}
        \end{frame}

        \begin{frame}[fragile]
        \frametitle{e820 exemple (Lenovo x220 8Go of RAM)}
\begin{verbatim}
00000000-0000ffff : reserved
00010000-0009d7ff : System RAM
0009d800-0009ffff : reserved
000a0000-000bffff : PCI Bus 0000:00
000c0000-000c7fff : Video ROM
000c8000-000cbfff : pnp 00:00
000cc000-000cffff : pnp 00:00
000d0000-000d3fff : pnp 00:00
000d4000-000d7fff : pnp 00:00
000d8000-000dbfff : pnp 00:00
000dc000-000dffff : pnp 00:00
000e0000-000fffff : reserved
00100000-1fffffff : System RAM
[...]
dae9f000-daf9efff : ACPI Non-volatile Storage
daf9f000-daffefff : ACPI Tables
dafff000-daffffff : System RAM
db000000-df9fffff : reserved
dfa00000-febfffff : PCI Bus 0000:00
fec00000-fec00fff : reserved
fed00000-fed003ff : HPET 0
fed08000-fed08fff : reserved
fed10000-fed19fff : reserved
fed1c000-fed1ffff : reserved
fed40000-fed4bfff : PCI Bus 0000:00
fee00000-fee00fff : Local APIC
ffd20000-ffffffff : reserved
100000000-21e5fffff : System RAM
21e600000-21e7fffff : reserved
21e800000-21fffffff : RAM buffer
\end{verbatim}
\end{frame}

\subsection{Long mode (64 bits)}
        \begin{frame}
        \frametitle{Long mode (64 bits)}
        \begin{itemize}
                \item NX (No eXecution) or XD (eXecution Disabled)
                \item 64 bits physical address
                \item Twice as much register compared with 32 bit mode
                \item ASLR (Address Space Layout Randomization)
                \item New relatif addressing mode usefull for PIE
                (Position Independant Executable)
                \item However segmentation support is reduced
                \item Segment are useful to isolate ring 0, 1 and 2.
                Paging only support to protection level.
                \item SMEP (Supervisor Mode Execution Protection) to
                protect the kernel from running malisious userspace code
        \end{itemize}
        \end{frame}

\subsection{Virtualization extensions}
        \begin{frame}
        \frametitle{Virtualization extensions}
        \begin{itemize}
                \item Started with Pentium 4 and AMD athlon (2000,
                1998?) %FIXME
                \item VMX,vt-x (Intel) / SVM, Pacifica (AMD)
                \item Two mode when running in ring0 vmx and vmx-root
                \item vmx-root is the hypervisor mode (Virtual Machine
                Monitor)
                \item vmx is the "virtualized" mode. It' the mode in
                which the guests are running
                \item vmx has a control block assosiated with the guest
                environement (VMCB, VM control Block). It contains
                things like:
                        \begin{itemize}
                        \item Guest's config registers: CR3, GDTR, IDTR
                        \item Guest's EAX-EDX, EIP, EBP, ESP
                        \item Guest's exit code
                        \end{itemize}
               \item HAP: Hardware Assisted Paging to virtualize the
               memory (EPT, Nested Paging)
               \item IOMMU to virtualize the DMA access
        \end{itemize}
        \end{frame}

\subsection{TXT}
        \begin{frame}
        \frametitle{TXT (Trusted eXecution Technology}
        \begin{itemize}
                \item Based the the MLE (Measured Launch Environement)
                \item Protected Execution, SMX mode (Safer Mode
                eXtensions)
                \item Data scealing %FIXME
                \item Secure handling of Input/Output
                \item Attestion of the platform. Is it the computer I
                think it is?
                \item Hash of the firmwares: PXE ROM, BIOS, ACPI tables, BIOS options)
                \item Integrity check of the system before launch.
                \item Can take advantage of the TPM (Trusted Paltform
                Module)
        \end{itemize}
        \end{frame}

\subsection{SMM}
        \begin{frame}
        \frametitle{SMM (System Management Mode)}
        \begin{itemize}
                \item Exists since 386 (1985)
                \item A SMI (System Management Interupt) will cause the
                system to switch to SMM mode
                \item But what is a SMI? It could be:
                        \begin{itemize}
                        \item Hardware Event: timer fire, termal sensor
                        reach a given temperature, ...
                        \item Software Event: I/O port, MMIO write to a
                        give address
                        \end{itemize}
                \item SMM code is embedded in the BIOS by the motherboad manufacturer
                \item The SMM RAM (SRAM) leaves in memory
                \item SMM as access to all the RAM (above 4Go)
                \item SMM can have unexpected side effect
        \end{itemize}
        \end{frame}

%       {SMI handling flow}
        \begin{frame}
                \frametitle{SMI handling flow}
                \begin{itemize}
                        \item A SMI is generated
                        \item The CPU switch to SMM
                        \item The CPU save the registers in SMM RAM
                        (SRAM)
                        \item The base on the SRAM can be found on the
                        CPU in the special register SMBASE
                        \item The CPU jumps at SMBASE + 0x8000 and
                        starts executing
                        \item Do stuff. Can modify the state
                        \item Restore the state
                        \item Return from SMI
                \end{itemize}
        \end{frame}

        \begin{frame}
                \frametitle{SMRAM protection}
                \begin{itemize}
                        \item SMRAMC (SMRAM control) and ESMRAMC (Extened SRAM
                        Control)
                        \begin{itemize}
                                \item D\_OPEN, set to 1 allow access of
                                the SMRAM from ring0
                                \item D\_LCK, once set to 1 it can't
                                be changed until next reboot
                                \item D\_LCK is set in the BIOS
                        \end{itemize}
                        \item SMM is a very atractive place to put a
                        rootkit because it runs outside of the OS
                        environement
                        \item Once in SMM the OS execution is frozen
                        \item Even with these protections in place we
                        can note some successful atack
                        \begin{itemize}
                                \item Modification of the SMRAM through
                                the GART (graphic IOMMU). Laurent Absil
                                - Loic Duflot, 2007
                                \item Bug in the Q35 BIOS. Rafal
                                Wojtczuk - Joanna Rutkowska, 2008
                                \item SMRAM modification via defered %FIXME
                                cache flush. ANSSI, 2009
                        \end{itemize}
                \end{itemize}
        \end{frame}

\section{Chipset}
        \begin{frame}
                \frametitle{PC chipset}
                \begin{itemize}
                        \item Split between two componant northbridge and southbridge
                        \item A CPU only have two buses (address and
                        data), the chipset to the multiplexing
                        \item The chipset implement buses protocol like
                        PCI, PCIe, AGP
                        \item Have some integrated functionality
                        (network card, sound, gpu, ...)
                \end{itemize}
        \end{frame}

        %{NorthBridge}
        \begin{frame}
                \frametitle{NorthBridge}
                \begin{itemize}
                        \item Historical concept
                        \item Include the memory controller
                        \item AGP was plugged into the NorthBridge to
                        gain bandwidth.
                        \item Previous generation include Intel GPU (with GTT)
                        \item On the Intel platform the Northbridge is
                        dead now. It's been integrated fully in the CPU
                        (including the GPU !!!)
                \end{itemize}
        \end{frame}


        \begin{frame}
                \frametitle{SouthBridge}
                \begin{itemize}
                        \item Buses controler
                        \item Integrated devices
                        \item (Intel) On today's platform it's called
                        the PCH (Platform Control Hub).
                \end{itemize}
        \end{frame}

\subsection{I/O on x86}
\subsubsection{Types of I/O}
        \begin{frame}
                \frametitle{Types of I/O}
                \begin{itemize}
                        \item PIO: Port-Mapped I/O (instruction in/out)
                        \item MMIO: Memory Mapped I/O (read/write on
                              the address)
                        \item DMA: Direct Memory Access. Asynchronous
                              transfert mecanism.
                        \item Interruption: Event. can't carry extra
                        data. Limited number.
                        \item MSI: Message signaled interrupt. PCI
                        interrupt. Calls an interrupt vector.
                \end{itemize}
        \end{frame}

\subsubsection{Buses}
        \begin{frame}
                \frametitle{Buses}
                \begin{itemize}
                        \item ISA, AGP, PCI, PCIe, PCI/PCIe bridge
                        \item SMBus: System Management Bus
                        \item LPCBus (Low Pin Count): Keyboard/Mouse (PS2), TPM, BIOS
                        (EPROM)
                \end{itemize}
        \end{frame}

%\subsection{Security concerns}
%\subsubsection{DMA attacks}
%\subsubsection{Firewire}
%\subsubsection{Interupt}
%\subsubsection{Devices}
%
%\subsection{Security improvement}
%\subsubsection{IOMMU}
%\subsubsection{ACS}
%\subsubsection{MSI}

\subsection{ACPI}
        \begin{frame}
                \frametitle{ACPI}
                \begin{itemize}
                        \item Partnership betwen HP, Intel, Microsoft
                        and Phoenix (1996)
                        \item Set of table to describe and configure the
                        platform.
                        \item Use of a specific language, ASL
                        \item The "complied" ASL code is storage on
                        tables

                        \item Mainly used to describe what need to be
                        done for a state transition.
                                \begin{itemize}
                                        \item What needs to be done to
                                        this device on sleep?
                                        \item What needs to be done to
                                        this device on wakeup?
                                \end{itemize}
                \end{itemize}
        \end{frame}

\subsubsection{ASL}
        \begin{frame}
                \frametitle{ASL}
                \begin{itemize}
                        \item OO language (Package, Object,
                        Method/Field).
                        \item Let's you describe registry and action on
                        a device
                        \item Control flow like condition and loops
                        \item To support ACPI the kernel needs to
                        embedded an ASL interpretor.
                \end{itemize}
        \end{frame}

        \begin{frame}[fragile]
        \frametitle{ASL example}
        \begin{block}{Device TEST}
        The device has one register CNT (1 byte) and two methods to
        respectively set to zero and increment this register.
        We use the Instruction OperationRegion to create the register.
        \end{block}
\begin{verbatim}
Device(TEST){
        OperationRegion(REG, SystemMemory, 0x0000400, 0x1)
        Field (REG, ByteAcc, Nolock, Preserve) { CNT, 0 }

        Method (_SAT, 1, NotSerialized)
        {
          Store (0x0, \\_SB.PCI0.TEST.CNT)
        }

        Method (_PSR, 1, NotSerialized)
        {
          Increment (\\_SB.PCI0.TEST.CNT)
        }
}
\end{verbatim}
\end{frame}

\subsubsection{ACPI tables}
        \begin{frame}
                \frametitle{ACPI tables}
                \begin{itemize}
                        \item RSDT: Root System Describtion Table
                        \item DSDT: Differentiated System Description
                        Talbe
                        \item SSDT: Secondary System Description Table
                \end{itemize}
        \end{frame}
\subsubsection{Interupts}

\subsection{TPM}
\subsubsection{Use case}
        \begin{frame}
                \frametitle{Use case}
                \begin{itemize}
                        \item Safe storage. Internaly use RSA key to safely store
                        data.
                        \item Integrity. Check if the platform was in
                        the same state when the data was sealed
                        \item Sealing. Seal a secret with the current
                        state of the platform.
                        \item Attestation. Expose the state of the
                        platform so a remote entity can validate the
                        state (Total or partial attestation).
                \end{itemize}
        \end{frame}

\subsubsection{Architecture}
        \begin{frame}
                \frametitle{Keys}
                \begin{itemize}
                        \item SRK: Storage Root Key. Key Pair. The
                        public key is used to encrypt other key. The
                        private key never leave the TPM.
                        \item AIK: Attestation Identity Keys. Used to
                        signed the PCRs in case of attestation.
                        \item EK: Endorsement Key. Comes with the TPM.
                \end{itemize}
        \end{frame}

        \begin{frame}[fragile]
                \frametitle{PCRs: Platform Configuration Registers}
                \begin{itemize}
                        \item 0-3: Describe the state of the BIOS
                        \item 4: Used by the bootloader
                        \item 14: Multiboot modules
                        \item The rest are general purpose register.
                \end{itemize}
                
                \begin{block}{Operation on a register}
                The only operation available on a PCR is the extend
                operation defined as such:
                \begin{center}
                \begin{verbatim}
                PCR[i] = SHA1( PCR[i] | SHA1(data) )
                \end{verbatim}
                \end{center}
                \end{block}
\end{frame}

\subsubsection{Root of Trust}
%\subsubsection{SRTM}
%\subsubsection{DRTM}
\subsubsection{Attack}

\end{document}


