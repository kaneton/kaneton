%%
%% licence       kaneton licence
%%
%% project       kaneton
%%
%% file          /home/mycure/kaneton/view/books/kaneton/coding-style.tex
%%
%% created       matthieu bucchianeri   [mon jan 30 17:32:57 2006]
%% updated       julien quintard   [tue aug  1 16:00:01 2006]
%%

%
% coding style
%

\chapter{Coding style}

We will detail in this chapter the coding style used to develop the
kaneton microkernel reference implementation.

\newpage

%
% case
%

The kaneton project developers try to follow a coding style. This
coding style was introduced to normalize the source code, leading to a
more readable source code.

Nevertheless, you can adapt this coding style to your own but try to
follow the rules.

\section{Case}

The whole kaneton source code is written using lower case letters.

Moreover, every text including comments etc. must be written using
lower case letters

%
% headers
%

\section{Headers}

Each file must start with a header formatted as shown below:

\begin{verbatim}
/*
 * licence       kaneton licence
 *
 * project       kaneton
 *
 * file          /home/mycure/core/kaneton/kaneton/as/as.c
 *
 * created       julien quintard   [fri feb 11 02:23:41 2005]
 * updated       matthieu bucchianeri   [mon jan 30 20:30:57 2006]
 */
\end{verbatim}

An emacs configuration file for automatically generating and updating
this header can be found in \textit{tools/emacs}.

Additionally, you need to set two environment variables to generate
a correct kaneton header:

\begin{itemize}
  \item
    \textit{EC\_LICENCE} must be set to ``kaneton licence''.
  \item
    \textit{EC\_DEVELOPER} must be set to your first name and last name.
\end{itemize}

Please, do not use nicknames in headers.

%
% naming convention
%

\section{Naming Convenions}

To keep the code as clear as possible, there are several conventions on
types, functions and variables naming.

%
% variables
%

\subsection{Variables}

Here are a few rules you are encouraged to follow:

\begin{itemize}
  \item
    \textit{sz} suffix for variables representing a size.

    \begin{verbatim}
      #define PAGESZ          4096

      int                     modsz;
    \end{verbatim}
  \item
    \textit{n} prefix for variables representing a number of objects.

    \begin{verbatim}
      int                     nclusters;
    \end{verbatim}
  \item
    etc.
\end{itemize}

Moreover, the types are used as pre-names:

\begin{verbatim}
t_vaddr                 video_vaddr;
\end{verbatim}

This example is not correct, instead prefer:

\begin{verbatim}
t_vaddr                 video;
\end{verbatim}

%
% functions
%

\subsection{Functions}

Function names must be prefixed by the file name, context name they are
implemented in.

For example, a function part of the address space manager must be prefixed
by \textit{as\_}.

These names must be chosen carefully: they must explicitely define
what the function does without being too long.

%
% types
%

\subsection{Types}

Here are the prefixes you must use when writing your own types:

\begin{itemize}
  \item
    \textit{m\_} for managers main structures.
  \item
    \textit{o\_} for kaneton objects.
  \item
    \textit{d\_} for dispatch interfaces.
  \item
    \textit{a\_} for architecture-dependent structures.
  \item
    \textit{s\_} for general purpose structures.
  \item
    \textit{t\_} for basic and general purpose typedefs.
  \item
    \textit{c\_} for kaneton capabilities.
  \item
    \textit{i\_} for kaneton identifiers.
\end{itemize}

Notice that \textit{a\_} can be combined with other prefixes, for
example \textit{ao\_} for an architecture-dependent object.

%
% includes
%

\section{Includes}

To keep the code clear and compact, developers only need to include a
minimal number of header files:

\begin{itemize}
  \item
    \textit{kaneton.h} for the microkernel declarations.
  \item
    \textit{libc.h} for the kaneton specific C library.
\end{itemize}

These files are located in the include path, so do not use relative include
path.

\begin{verbatim}
#include <libc.h>
#include <kaneton.h>

int             main(int                argc,
                     char**             argv)
{
  [...]

  return (0);
}
\end{verbatim}

All include files must be protected against multiple inclusions. The
guard macro to use must be named using the directory name, one underscore,
the file name, one underscore and a capital ``H''.

For example, the file \textit{kaneton/include/kaneton/segment.h} will be
guarded as follow:

\begin{verbatim}
#ifndef KANETON_SEGMENT_H
#define KANETON_SEGMENT_H	1

[...]

#endif
\end{verbatim}

In addition, for architecture-dependent files, the guard macro must begin
with the architecture name; for example for the Intel architecture:
\textit{IA32\_KANETON\_SEGMENT\_H}.

%
% types
%

\section{Types}

You may use as soon as possible standard types: \textit{t\_uint8},
\textit{t\_sint32}, \textit{t\_uint64} etc.

This nomenclature is more understandable than
\textit{unsigned long long int}.

%
% return values
%

\section{Return Values}

Every function must report whether it successed or failed.

In kaneton, functions' return type must be \textit{t\_error}.

A function will return \textit{ERROR\_NONE} on success and anything
else on error, for example \textit{ERROR\_UNKNOWN} to indicate a non-specific
error.

%
% indentation
%

\section{Indentation}

There are several indentation rules in kaneton.

\begin{enumerate}
  \item
    Field names of structures and unions must be aligned with the
    structure or union name.

    \begin{verbatim}
      struct       s_set
      {
        i_set      id;
        t_setsz    size;
        t_type     type;
      };
    \end{verbatim}

    or

    \begin{verbatim}
      typedef struct
      {
        o_id       id;
        i_stats    stats;
        i_set      ass;
      }            m_as;
    \end{verbatim}
  \item
    Macros and variables must be aligned as shown below:

    \begin{verbatim}
      #define TASK_PRIOR_KERNEL     230
      #define TASK_HPRIOR_KERNEL    250
      #define TASK_LPRIOR_KERNEL    210

      m_task*                     task;
      i_task                      ktask = ID_UNUSED;
    \end{verbatim}

    This rule also applies for variables declarations in functions.
  \item
    Function prototypes and bodies should look like this:

    \begin{verbatim}
      t_error             stats_function(i_stats          id,
                                         char*            function,
                                         t_stats_func**   f)
      {
        t_sint64          slot = -1;
        t_sint64          i;

        [...]
      }
    \end{verbatim}

    Notice that argument names are aligned between each other,
    and variable names are aligned with function name and between
    each other.

    Try to respect this alignment between functions in a single file:
    function names may be all aligned and argument names also.
\end{enumerate}

%
% comments
%

\section{Comments}

As kaneton is intended to be a pedagogical project with clear and
understandable source code; no need to say that comments take a very
important part of this objective.

Every file must begin with a comment describing what is done in this
code via the \textit{information} section.

Moreover, every function must be preceded by a comment defining its
behavior.

For complex functions and yo prevent direct comments in the source code,
we used \textit{steps}:

\begin{itemize}
  \item
    Each critical code section in a function is preceded by a step
    number.
  \item
    The function header comment contains steps descriptions.
\end{itemize}

An example is present below:

\begin{verbatim}
/*
 * this function shows the usage of comments and steps.
 *
 * steps:
 *
 * 1) compute the index.
 * 2) make the operation.
 * 3) check the result.
 */

t_error         test_foobar(int      a,
                            int      b,
                            int*     c)
{
  int           index;

  /*
   * 1)
   */

  index = text_make_index(a, b);

  /*
   * 2)
   */

  index = index * a + b;

  /*
   * 3)
   */

  if (index < 0)
    return (ERROR_UNKNOWN);

  *c = index;

  return (ERROR_NONE);
}
\end{verbatim}

%
% sections
%

\section{Sections}

kaneton files are divided in multiple sections.

Section are delimited as shown below:

\begin{verbatim}
/*
 * ---------- includes ------------------------------------------------
 */
\end{verbatim}

Possible sections in a file are:

\begin{itemize}
  \item
    \textit{header files}: information, dependencies, defines, types,
    prototypes, macros, etc.
  \item
    \textit{source files}: information, extern, globals, includes,
    functions, etc.
  \item
    \textit{make files}: dependencies, directives, variables, rules, etc.
\end{itemize}

Moreover, every important file, for example the main file of each
kaneton manager, have to contain a section \textit{information} describing
the whole manager.

In addition, a section named \textit{assignments} is generally necessary
for manager will be filled in by the students. This section briefly describes
the work to be done by the students.

%
% macros
%

\section{Macros}

The kaneton microkernel uses few fundamental macros lited below:

\begin{itemize}
  \item
    \textit{\_\_\_bootloader} indicates that this source code belongs to
    the bootloader.
  \item
    \textit{\_\_\_kernel} indicates that this source code belongs to the
    microkernel.
  \item
    \textit{\_\_\_kaneton} indicates that this kernel is the kaneton
    microkernel.
  \item
    \textit{\_\_\_wordsz} indicates the word size: 16-bit, 32-bit,
    64-bit etc.
  \item
    \textit{\_\_\_endian} indicates the endianness.
\end{itemize}
