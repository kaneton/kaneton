%
% ---------- header -----------------------------------------------------------
%
% project       kaneton
%
% license       kaneton
%
% file          /home/mycure/kaneton/view/book/core/kaneton.tex
%
% created       julien quintard   [mon may 14 19:56:45 2007]
% updated       julien quintard   [mon dec 17 23:55:38 2007]
%

%
% path
%

\def\path{../..}

%
% template
%

%%
%% licence       kaneton licence
%%
%% project       kaneton
%%
%% file          /home/mycure/kaneton/view/templates/book.tex
%%
%% created       julien quintard   [wed mar  1 23:45:22 2006]
%% updated       julien quintard   [thu may  4 12:36:54 2006]
%%

%
% class
%

\documentclass[10pt,a4wide]{book}

%
% packages
%

\usepackage[english]{babel}
\usepackage[T1]{fontenc}
\usepackage{a4wide}
\usepackage{fancyheadings}
\usepackage{multicol}
\usepackage{indentfirst}
\usepackage{graphicx}
\usepackage{color}
\usepackage{xcolor}
\usepackage{verbatim}

\usepackage{aeguill}

\usepackage[Lenny]{../../../tools/latex/fncychap}

\pagestyle{fancy}

\setlength{\footrulewidth}{0.3pt}
\setlength{\parindent}{0.3cm}
\setlength{\parskip}{2ex plus 0.5ex minus 0.2ex}

%
% logos
%

\newcommand{\logos}
  {
    \begin{center}
      \includegraphics[scale=0.8]{../../logos/kaneton.pdf}
    \end{center}
  }

%
% colors
%

\definecolor{functioncolor}{rgb}{0.40,0.00,0.00}
\definecolor{commandcolor}{rgb}{0.00,0.00,0.40}
\definecolor{verbatimcolor}{rgb}{0.00,0.40,0.00}
\definecolor{noticecolor}{rgb}{0.87,0.84,0.02}

%
% function
%

\newcommand\function[3]{
  \begin{tabular}{p{0.2cm}p{13.8cm}}
  & {\color{functioncolor}\textbf{#1}}#2
  \end{tabular}

  \begin{tabular}{p{1cm}p{13cm}}
  & #3
  \end{tabular}}

%
% align
%

\newcommand\align[1]{
  \\ & \hspace{#1}}

%
% argument
%

\newcommand\argument[1]{\textit{#1}}

%
% command
%

\newcommand\command[2]{
  \begin{tabular}{p{0.2cm}p{13.8cm}}
  & {\color{commandcolor}\textbf{#1}}
  \end{tabular}

  \begin{tabular}{p{1cm}p{13cm}}
  & #2
  \end{tabular}}

%
% notice
%

\newcommand\notice[1]{
  {\color{noticecolor}\textbf{Notice}}

  \begin{tabular}{p{0.2cm}p{13.8cm}}
  & #1
  \end{tabular}}

%
% example
%

\newcommand\example[1]{
  \textit{Example:}

  \begin{tabular}{p{0.2cm}p{13.8cm}}
  & \textit{#1}
  \end{tabular}}

%
% warning XXX
%

%
% verbatim stuff
%

\makeatletter

\renewcommand{\verbatim@font}
  {\ttfamily\footnotesize\color{verbatimcolor}\selectfont}

\def\verbatim@processline{\hskip15ex\the\verbatim@line\par}

\makeatother

%
% header
%

\rhead{}
\rfoot{\scriptsize{The kaneton microkernel project}}

\date{\scriptsize{\today}}


%
% header
%

\lhead{\scriptsize{The kaneton microkernel :: core}}
\rhead{}

%
% title
%

\title{The kaneton microkernel :: core \\
       \version \\
       \logos}

%
% authors
%

\author{\small{Julien Quintard},
        \small{Matthieu Bucchianeri} and
        \small{Renaud Voltz}}

%
% document
%

\begin{document}

%
% title
%

\maketitle

%
% --------- text --------------------------------------------------------------
%

This document describes the kaneton microkernel reference project.

\-

This document should be used by every student willing to implement the
kaneton microkernel and even by people looking for more details on
the kaneton microkernel design and implementation.

\-

All the kaneton documents are available on
  the official website\footnote{http://www.kaneton.org}.

%
% toc
%

\toc

%
% chapters
%

%
% ---------- header -----------------------------------------------------------
%
% project       kaneton
%
% license       kaneton
%
% file          /home/mycure/kaneton/view/book/kaneton/goals.tex
%
% created       julien quintard   [fri jun  1 13:58:12 2007]
% updated       julien quintard   [mon may 19 23:09:48 2008]
%

%
% ---------- goals ------------------------------------------------------------
%

\chapter{Goals}
\label{chapter:goals}

In this chapter we will briefly introduce the kaneton microkernel
through the kaneton microkernel goals.

\newpage

%
% ---------- text -------------------------------------------------------------
%

The project was primarily designed by two students in computer science,
\name{Julien Quintard} and \name{Jean-Pascal Billaud}.

These two students previously actively contributed to the development
of a nanokernel-based operating system project in a French research laboratory.
This system was not powerful enough, especially from the design point of view.

Therefore, the two students started the design of a new microkernel
by their own, called \term{kaneton}, for educational purposes.

The design was based on five fundamental guidelines.

\begin{enumerate}
  \item
    \textbf{Educational}

    \-

    The kaneton project is built to become an educational project. The design
    as well as the implementation must therefore be as understandable as
    possible so that everyone interested in kernel internals can go through the
    documents and source code and actually understand how it works.

    \-

    This \textit{understandable} property can be achieved through a very clear
    and coherent design. Moreover, the implementation should be written using
    modern tools and techniques to make the code as generic as possible and
    easily readable.
  \item
    \textbf{Portability}

    \-

    The microkernel was particularly designed to be portable. The designers
    tried to develop a portability system powerful enough to port kaneton on
    any, existing or not, architectures.
  \item
    \textbf{Maintanability}

    \-

    Although microkernel-based operating systems rely on a modular design,
    kaneton designers also wanted the microkernel itself to be modular and
    maintainable.
  \item
    \textbf{Distributed Computing}

    \-

    The kaneton microkernel must be designed to fit distributed operating
    systems requirements. Indeed, the kaneton microkernel was developed in
    order to design and implement a distributed operating system named
    \term{kayou}.

    \-

    This point led to many specific choices in the kaneton microkernel design.
  \item
    \textbf{Demystification}

    \-

    kaneton people wanted to break some well-known kind of computer
    science rules. Indeed, for instance, many computer scientists consider
    the source code as the actual documentation. Also, for many low-level
    programmers, the kernel boot source code and more generally the
    kernel source code itself cannot be understandable, clear and coherent as
    it is related to low-level programming: microprocessor, devices \etc{}

    kaneton people paid particular attention to the microkernel source code to
    be easily understandable, maintainable and extendable. Moreover, kaneton
    people tried to write documentation for every part of the project.
\end{enumerate}

Notice that building an educational microkernel project is nothing innovative.
Indeed few other projects already exist; the most popular being \name{MINIX}
from \name{Vrije Universiteit}, \name{NachOS} from \name{Berkeley University}
or \name{PintOS} from \name{Stanford University}.

kaneton people tried to design and implement a modern microkernel since, the
original \name{MINIX} microkernel for example, do not use modern development
tools. Moreover, the kaneton source code is heavily commented and use modern
languages techniques while trying to stay easily understandable.

The educational characteristic of kaneton does not constraint it from being
optimised afterwards. kaneton people believe that implementing optimised
algorithms in the first place does not lead to maintainable implementations.

Finally, note that the kaneton project is actually composed of two projects:
the \name{kaneton microkernel \term{educational} project} which provides
everything necessary to students willing to learn about kernels internals;
and the \name{kaneton microkernel \term{research} project} which focuses on
designing and implementing a powerful, reliable, flexible microkernel.
Obviously these two projects are highly related as the kaneton educational
project relies on the implementation of the kaneton research project.

%
% ---------- header -----------------------------------------------------------
%
% project       kaneton
%
% license       kaneton
%
% file          /home/mycure/kaneton/view/book/development/history.tex
%
% created       julien quintard   [fri jun  1 14:33:16 2007]
% updated       julien quintard   [mon jun 18 12:38:23 2007]
%

%
% ---------- history ----------------------------------------------------------
%

\chapter{History}

In this chapter we detail the kaneton history from the first
year with low-level programming introduction to the last kaneton
microkernel implementation.

\newpage

%
% ---------- text -------------------------------------------------------------
%

During the kaneton history, the project evolved and courses were added
to the curriculum to make the whole kaneton project more interesting and
understandable by the students. Moreover, the educational project, which
was already targetting the \textit{EPITA}'s \textit{System, Network and
Security} major, was also used in other contexts, victim of its success
and of the very hard work achieved by kaneton people over the years.

%
% 2004
%

\section{2004}

The first year, a low-level programming introduction course named \textbf{k}
was proposed for the \textit{EPITA} Engineering School's first year students.

About fourteen hours courses were taught introducing the \textit{Intel 32-bit}
microprocessor's external architecture and low-level programming.

The students had to develop small, poor and messy device drivers for the
console and keyboard peripherals. Moreover, a tiny command interpreter was
developed by students so that a kernel action could be triggered by entering
a command.

The course was a bit chaotic but this first shot was a success.

Therefore, the students majoring in \textit{System, Network and Security}
asked the two students \textit{Julien Quintard} and \textit{Jean-Pascal
Billaud} a complete kernel project for their curriculum so that they can
learn more about operating systems internals.

%
% 2005
%

\section{2005}

The two, still students, \textit{Julien Quintard} and \textit{Jean-Pascal
Billaud} then prepared a complete microkernel design the students will have
to implement. This was the premises of the \textbf{kaneton microkernel
educational project}. Additionally, two complete courses on kernel design and
\textit{Intel 32-bit Architecture} programming were prepared.

The project was composed of six steps, from the bootstrap, passing by
the kernel internals including memory management, task management etc.
to the servers with an \textit{IDE} device driver and finally a \textit{FAT}
file system.

Notice that the majority of the students did not success in implementing a
complete scheduling system allowing the creation of user-land task. Indeed,
the best groups achieved in providing the management of kernel-land tasks only.
Therefore, the \textit{IDE} driver, \textit{FAT} file system etc. were
running in the kernel.

Inspite of this, once again, the whole project was a success. However,
kaneton people noticed that the students took much time doing boring work
like filling in header files, dealing with versionning problems, writing
\textit{Make} files and \textit{Shell} scripts etc.

Moreover, the courses were too messy and the students had difficulties
to make the relation between the kaneton design and the microprocessor's
architecture implementation.

As a result, kaneton people decided to start implementing a kaneton microkernel
reference by their own in the \textit{C} language. This implementation will
then be used to compare the behaviour of students' implementation with
the reference. Moreover, this implementation led to the creation of a
new project: the \textbf{kaneton microkernel research project}.

%
% 2006
%

\section{2006}

While \textit{Jean-Pascal Billaud} leaved the project, people joined it
starting with \textit{EPITA} last year students \textit{C\'edric Aubouy},
\textit{Renaud Lienhart} but also \textit{Fabien Le-Mentec} from
\textit{EPITECH} who knew these people from the \textit{EPITA Computer Systems
Laboratory} where they were all working together a year before.
\textit{C\'edric Aubouy} and \textit{Renaud Lienhard} were in charge of the
kernel and \textit{Intel 32-bit Architecture} courses, respectively.
\textit{Julien Quintard} was still in charge of the kaneton educational
project given to the students.

More over two \textit{EPITA} first year students joined the \textit{EPITA
Computer Systems Laboratory}, \textit{Matthieu Bucchianeri} and \textit{Renaud
Voltz}. Indeed, from this date, the \textit{EPITA Computer Systems Laboratory}
was a strong partner of the kaneton microkernel project. These two students
were hired for contributing to the development of the kaneton research project.
Moreover, these students were supposed to teach and supervise the kaneton
educational project the following year.

\textit{Matthieu Bucchianeri} and \textit{Renaud Voltz} did an amazing work
on the kaneton research project implementation. Indeed, most of the
code related to the \textit{Intel 32-bit Architecture} comes from them. In
addition, the test suite as well as many tests were written by them. Thanks
to their work.

This year, kaneton people decided to introduce a development environment,
based on the kaneton research reference implementation, including everything
necessary to set up a collaborative kernel development.

While, previously, the students had to write the entire microkernel and
servers from scratch, this year, students only had to write precise parts
of the microkernel including some set implementations, memory management,
task scheduling etc.

Few mistakes were made especially about the choice of parts the students
had to implement. Indeed, asking the students to implement set implementations
like linked-list, array etc. was a very bad idea. This year, the project
was not completed and students stopped the project before the messaging
system implementation.

A course was also added to the \textit{EPITA} \textit{System, Network
and Security} major's curriculum about microprocessors' internals. This
course was introduced and taught by \textit{Julien Quintard}.

In conclusion, the kaneton educational project was not a real success this year
and needed some modifications. For instance, the course about the \textit{Intel
32-bit Architecture} was too specific and hard to understand but also hard
to teach. Instead, kaneton people decided to introduce a more general course
about kernel principles for the next year.

The kaneton research project implementation, in 2006,
  counted\footnote{Estimations realised with the software \textit{sloccount}.}
about \textit{7,000} lines for the \textit{core} and about \textit{2,000}
lines for the \textit{Intel 32-bit Architecture} implementation.

%
% 2007
%

\section{2007}

People affiliated with the \textit{EPITA Computer Systems Laboratory} joined
the project: \textit{Pierre Duteil} and \textit{Julian Pidancet}. Moreover,
students who implemented the kaneton educational project the previous year
decided to join the project: \textit{Enguerrand Raymond} and \textit{Mathieu
S\'elari\`es}, mainly working on the \textit{MIPS Architecture} portage among
other contributions.

This year, \textit{Matthieu Bucchianeri} and \textit{Renaud Voltz} were in
charge of the educational project by giving the kaneton courses as well
as supervising students' educational implementations.

As the kaneton research implementation was much more advanced as in 2006,
the students were given more code and then focused only on interesting and
system-related parts.

Students totally implemented the physical and virtual memory management, the
event and timers, the thread manager, the scheduler and the messaging system.
As for the previous years, the implementation was based on the \textit{Intel
32-bit Architecture}.

This year, the whole kaneton educational project was also given to students
from the \textit{Realtime \& Embedded Systems} specialization, for a total
of about $50$ students. The project was evaluated using a test suite,
developed for the kaneton research project, of about a hundred tests.

This year, the kaneton educational project was an amazing success as many
students completed a working microkernel, able to run tasks and to implement
some servers running on top of the kaneton microkernel.

The kaneton research implementation has grown to \textit{9,000} lines of source
code for the \textit{core} and \textit{5,500} lines for the microprocessor's
architecture implementation on \textit{Intel 32-bit}. The kaneton research
implementation was able to start modules - as standalone binaries - in
user-land as well as to make them communicate through the kaneton messaging
system.

This year, \textit{Pierre Duteil} leaved the project.

%%
%% licence       kaneton licence
%%
%% project       kaneton
%%
%% file          /home/mycure/kaneton/view/papers/kaneton/overview.tex
%%
%% created       matthieu bucchianeri   [mon jan 30 17:09:45 2006]
%% updated       julien quintard   [thu mar  2 13:12:22 2006]
%%

%
% overview
%

\chapter{Overview}

XXX ce chapitre va vous aider a reconnaitre les fonctionnalites principale
XXX d'un kernel dans kaneton.

The kaneton microkernel is only the core of an operating system.
Main tasks like hardware drivers or user services are implemented as
\textbf{servers}. So the microkernel only has a few functionalities to
provide:

\begin{itemize}
  \item
    Memory management.
  \item
    Process management.
  \item
    Communication.
  \item
    Events.
\end{itemize}

In this chapter we will describe briefly these tasks and all the
associated managers.

%
% memory management
%

\section{Memory Management}

Handling the memory -- from virtual address space to physical
addressing -- is done by three major managers, the \textbf{as},
\textbf{segment} and \textbf{region} managers.

%
% as
%

\subsection{as}

The address space manager just manages the different address spaces
used by the kaneton tasks.

In kaneton, we call an \textbf{as - address space} a list of memory
locations referenced by a task. Each task has its own address space.

%
% segment
%

\subsection{segment}

The segment manager just manages the segments reserved by
the different kaneton entities including the kernel, the drivers etc..

In kaneton terms a \textbf{segment} is a contiguous area of reserved
physical memory.

%
% region
%

\subsection{region}

The region manager keeps track of regions used to map segments for
each address space reserved on the system.

In kaneton, a \textbf{region} is contiguous area of virtual memory
mapping a segment's part.

%
% process management
%

\section{Process Management}

XXX

%
% communication
%

\section{Communication}

XXX

%
% events
%

\section{Events}

XXX

%
% ---------- header -----------------------------------------------------------
%
% project       kaneton
%
% license       kaneton
%
% file          /home/mycure/kaneton/view/book/development/source-tree.tex
%
% created       julien quintard   [thu may 17 22:41:36 2007]
% updated       julien quintard   [thu may 31 08:34:23 2007]
%

%
% ---------- source tree ------------------------------------------------------
%

\chapter{Source Tree}
\label{chapter:source tree}

In this chapter we will briefly describe the kaneton microkernel project
source tree.

\newpage

%
% ---------- text -------------------------------------------------------------
%

The kaneton microkernel reference source tree looks like the following
listing:

\begin{verbatim}
cheat/
configure/
environment/
export/
history/
kaneton/
library/
license/
test/
tool/
transcript/
view/
\end{verbatim}

%
% cheat/
%

\subsection*{cheat/}

Since the kaneton microkernel is implemented by students, the kaneton
people need to check whether students are cheating by re-using parts of
previous years projects or other kernel source codes available on the
\textit{Internet}.

To avoid cheating, kaneton people developed a software checking for
commonalities between different source codes.

This directory contains scripts that performs these verifications. However,
the students work over the years are not stored in this directory but in
the \textit{history/} directory instead.

%
% configure/
%

\subsection*{configure/}

This directory contains everything necessary for configuring its own
kaneton microkernel development environment through the compiling process
to the boot system.

Any new contributor should first look at this directory. However, note that
this directory mainly contains tools targeting final-users rather than
kaneton contributors. Indeed, for instance, the \textit{configure} utility
aims at providing a user-friendly way for configuration but does not take
advantage of the power of the kaneton development environment.

Contributors should then learn about how the development environment works
while final-users should use the \textit{configure} tool.

%
% environment/
%

\subsection*{environment/}

This directory contains everything necessary to the kaneton development
environment.

The kaneton development environment allows different developers to
interact on the development of the same microkernel in a pretty easy way.

The development environment aims at providing developers to possibility to
work in a collaborative manner without interfering with each other. These
developers are likely to run different operating systems on different
microprocessors. In addition, the kaneton microkernel can be targeted for
different microprocessor architectures. The development environment was
introduced to cope with these combinations by providing profiles, each
profile describing the behaviour of a component: underlying operating system,
target architecture, user-specific stuff etc.

As a result, each developer can use a different operating system and
microprocessor architecture with its own specific compiling flags, kaneton
parameters etc. without modifying another developer's configuration.

The development environment is detailed in \textit{Section
\ref{section:environment}}.

%
% export/
%

\subsection*{export/}

The \textit{export/} directory contains scripts used to generate a kaneton
tarball in order to be distributed to the students at the beginning of the
kaneton educational project.

Indeed, these scripts rearrange the kaneton hierarchy hidding some important
directories the students do not need to be aware of. Moreover some source
code parts are removed since the students have to rewrite these pieces
of code as assignments.

These scripts are also used for making backups and distribution tarbalss of
the kaneton microkernel.

%
% history/
%

\subsection*{history/}

The \textit{history/} directory contains the students work over the years
in the universities and schools the kaneton project was used as an operating
system course's implementation material.

The tools of the \textit{cheat/} directory use these students works for
performing cheating verifications.

%
% kaneton/
%

\subsection*{kaneton/}

This directory is the most important of the project since it contains
the whole microkernel source code.

The directory is composed of three important subdirectories: \textit{core/},
\textit{platform/} and \textit{architecture/}. These subdirectories are
described next.

% core/

\subsubsection*{core/}

This directory contains the kaneton core source code.

The directory is divided as shown below:

\begin{verbatim}
as/
region/
sched/
segment/
set/
task/
thread/
[...]
\end{verbatim}

Each directory represents a kaneton core manager. For more information on
the kaneton core, please refer to the appropriate document:
\textit{The kaneton microkernel :: core}

% platform/

\subsubsection*{platform/}

This directory contains everything in relation with what the kaneton
microkernel project calls a \textit{platform}. The platform represents the
board supporting the devices: microprocessor, memory, peripherals etc.

This directory obviously contains subdirectories for each platform
supported by the kaneton microkernel.

% architecture/

\subsubsection*{architecture/}

The \textit{architecture/} directory contains the source-code related to
the microprocessor architectures supported by the kaneton microkernel.

This directory is composed of subdirectories, each one representing a
supported architecture: \textit{ia32}, \textit{mips64} etc. Note that each
architecture can be specialised. For instance, the \textit{ia32/optimised}
architecture represents an optimised implementation of the \textit{Intel IA-32}
microprocessor architecture.

%
% library/
%

\subsection*{library/}

This directory contains the libraries used by the kaneton microkernel itself,
the kaneton microkernel servers or maybe both. This directory especially
contains the standard \textit{kaneton C library}.

%
% license/
%

\subsection*{license/}

This directory contains the licenses used for any program or document
in relation with the kaneton microkernel project. Indeed, the kaneton
microkernel is under the \textit{kaneton license} which is described in
depth in the documents contained in this directory. Note that these licenses
are also available in \textit{Chapter \ref{chapter:licenses}}.

Each student has to read and agree with the kaneton license before
implementing or even using the kaneton microkernel project..

Indeed, every user of the kaneton-related stuff is considered as having
implicitly accepted the kaneton license.

%
% test/
%

\subsection*{test/}

Since the kaneton microkernel is used as a material for operating system
courses, the kaneton microkernel reference, which is the basis of students
work, must be extremely reliable.

The kaneton project therefore contains a set of tools in order to validate
the kaneton reference implementation behaviour. These tools are also used
for evaluating the correctness of the students implementation.

The \textit{test/} directory contains the set of kaneton scripts and tests
for validating a kaneton microkernel implementation.

%
% tool/
%

\subsection*{tool/}

This directory contains additional scripts and configuration files used by
the kaneton development environment or the kaneton developers.

As examples, this directory contains scripts for generating prototypes,
building a boot device etc.

%
% transcript/
%

\subsection*{transcript/}

This directory contains real-time recorded sessions. These sessions can be
replayed in order to present a feature of the development environment or
of the kaneton microkernel.

%
% view/
%

\subsection*{view/}

This directory contains all the kaneton documents including kaneton
administrative documents, examinations, lectures materials, kaneton papers
and books etc.

Additionally, scripts are provided in order to very easily build and
display these documents.
%%
%% copyright quintard julien
%% 
%% kaneton
%% 
%% development-environment.tex
%% 
%% path          /home/mycure/kaneton
%% 
%% made by mycure
%%         quintard julien   [quinta_j@epita.fr]
%% 
%% started on    Tue Jul  5 12:23:08 2005   mycure
%% last update   Sun Oct 23 02:55:45 2005   mycure
%%

%
% class
%

\documentclass[8pt]{beamer}

%
% packages
%

\usepackage{pgf,pgfarrows,pgfnodes,pgfautomata,pgfheaps,pgfshade}
\usepackage{colortbl}
\usepackage{times}
\usepackage{amsmath,amssymb}
\usepackage{graphics}
\usepackage{graphicx}
\usepackage{color}
\usepackage{xcolor}
\usepackage[english]{babel}
\usepackage{enumerate}
\usepackage[latin1]{inputenc}

%
% style
%

\usepackage{beamerthemesplit}
\setbeamercovered{dynamic}

%
% verbatim font
%

\definecolor{verbatimcolor}{rgb}{0,0.4,0}

\makeatletter
\renewcommand{\verbatim@font}
  {\ttfamily\footnotesize\color{verbatimcolor}\selectfont}
\makeatother

%
% new line
%

\newcommand{\nl}[0]{\vspace{0.4cm}}

%
% title
%

\title{Development Environment}

%
% authors
%

\author
{
  Julien~Quintard\inst{1} \\
  {\tiny julien.quintard@gmail.com}
}

\institute
{
  \inst{1} kaneton distributed operating system project
}

%
% date
%

\date{\today}

%
% logos
%

\pgfdeclareimage[interpolate=true,width=34pt,height=18pt]
                {epita}{../../logos/epita}
\pgfdeclareimage[interpolate=true,width=49pt,height=18pt]
                {upmc}{../../logos/upmc}
\pgfdeclareimage[interpolate=true,width=25pt,height=18pt]
                {lse}{../../logos/lse}

%
% table of contents at the beginning of each section
%

\AtBeginSection[]
{
  \begin{frame}<beamer>
   \frametitle{Outline}
    \tableofcontents[current]
  \end{frame}
}

%
% table of contents at the beginning of each subsection
%

\AtBeginSubsection[]
{
  \begin{frame}<beamer>
   \frametitle{Outline}
    \tableofcontents[current,currentsubsection]
  \end{frame}
}

%
% document
%

\begin{document}

%
% title frame
%

\begin{frame}
  \titlepage

  \begin{center}
    \pgfuseimage{epita} \hspace{0.1cm} \pgfuseimage{upmc} \hspace{0.1cm}
    \pgfuseimage{lse} \hspace{0.1cm}
  \end{center}
\end{frame}

%
% outline frame
%

\begin{frame}
  \frametitle{Outline}
  \tableofcontents
\end{frame}

%
% overview
%

\section{Overview}

% 1)

\begin{frame}
  \frametitle{Introduction}

  From the previous years, a development environment was introduced.

  \nl

  The questions are:

  \begin{enumerate}[<+->]
    \item
      Why?
    \item
      What are the advantages and disadvantages of such a
      development environment?
    \item
      How did the other promotions do?
  \end{enumerate}
\end{frame}

% 2)

\begin{frame}
  \frametitle{Explanations}

  Over the years, the kaneton project evolved, starting with a very
  simple introduction to low-level programming, to microkernel
  development and finally to a distributed operating system project.

  \nl

  Going always further implies many modifications in the project
  including:

  \begin{itemize}[<+->]
    \item
      The courses given which now go from the Intel processor to
      the distributed operating system concepts
    \item
      The assignments which always evolve to study advanced topics
    \item
      The context because we now have to provide parts of the microkernel
      to avoid students a development from scratch
    \item
      .. and so the requirements
  \end{itemize}
\end{frame}

% 3)

\begin{frame}
  \frametitle{The Courses}

  The kaneton project now comes with four courses:

  \begin{enumerate}
    \item
      The design of the kaneton distributed operating system including
      the microkernel
    \item
      The Intel processor
    \item
      The kernel concepts
    \item
      The distributed operating system concepts
  \end{enumerate}
\end{frame}

% 4)

\begin{frame}
  \frametitle{The Assignments}

  During the year 2005, the students develop a poor microkernel
  from scratch with few functionalities, a driver and finally a baby
  file system.

  \nl

  We cannot ask the students of the year 2006 to develop the same project
  but to go further to study advanced topics like distributed algorithms.

  \nl

  So, we cannot ask the students to develop every parts of the microkernel
  because this takes much time and implies to not study advanced
  topics.
\end{frame}

% 5)

\begin{frame}
  \frametitle{The Context}

  Providing students parts of the microkernel is not enough.

  \nl

  Indeed, we decided to provide a complete development environment
  including:

  \begin{itemize}
    \item
      Makefiles
    \item
      Shell scripts
    \item
      Papers
    \item
      Tools
    \item
      .. everything you need to start microkernel development
  \end{itemize}
\end{frame}

% 6)

\begin{frame}
  \frametitle{Why?}

  The remaining question is:

  \nl

  \textbf{Why providing such a development environment and not letting us
    develop one ourself?}

  \nl

  The answers simply are:

  \begin{itemize}
    \item
      Developing such a development environment takes much time and
      need experience
    \item
      This development environment include very powerful features:
      multiusers cooperation, different operating systems etc..
    \item
      Finally, students will not be able to create such a complicated
      development tree so it is provided to not waste time.
  \end{itemize}
\end{frame}

% 7)

\begin{frame}
  \frametitle{The Requirements}

  The students starting the kaneton project should think that they
  will learn many many things during the year.

  \nl

  This year, we are trying to lead students to a distributed operating
  system.

  \nl

  This implies more concepts, algorithms and techniques to learn.

  \nl

  To do this we introduced more courses but the students will have
  to work hard to be able to success.
\end{frame}

% 8)

\begin{frame}[containsverbatim]
  \frametitle{Tree}

  \begin{center}

  \begin{verbatim}
    /
      conf/
      core/
      doc/
      drivers/
      env/
      export/
      libs/
      papers/
      programs/
      services/
      tools/
  \end{verbatim}

  \end{center}
\end{frame}

%
% conf
%

\section{conf}

% 1)

\begin{frame}
  \frametitle{Overview}

  The \textbf{conf} directory contains user variables used to parameterise:

  \begin{itemize}
    \item
      the development environment: makefiles, scripts etc..
    \item
      the kernel
  \end{itemize}

  \nl

  This configuration system is very interesting coupled with versionning
  system.

  \nl

  Indeed, you can develop using special compilation flags, specific kernel
  configuration without conflicts with other developers.
\end{frame}

% 2)

\begin{frame}[containsverbatim]
  \frametitle{Tree}

  \begin{verbatim}
    conf/
      mycure/
        conf.c
        conf.h
        kaneton.conf
        modules.conf
        mycure.conf
      pwipwi/
      chiche/
  \end{verbatim}

  This configuration system uses the shell variable \$USER to find
  the main configuration file: \textbf{conf/\$USER/\$USER.conf}.
\end{frame}

% 3)

\begin{frame}
  \frametitle{conf.c}

  This file is not used yet.
\end{frame}

% 4)

\begin{frame}
  \frametitle{conf.h}

  This file contains macros to configure the kernel:

  \begin{itemize}
    \item
      \textbf{CONF\_TITLE}
    \item
      \textbf{CONF\_VERSION}
    \item
      \textbf{CONF\_DEBUG}
    \item
      etc..
  \end{itemize}

  \nl

  This file is included by the kernel code.
\end{frame}

% 5)

\begin{frame}
  \frametitle{kaneton.conf}

  This configuration file is used to pass arguments at the runtime to the
  servers.

  \nl

  This file is also used to configure kernel and servers input variables.
\end{frame}

% 6)

\begin{frame}
  \frametitle{modules.conf}

  This file contains the list of the modules to be loaded by the
  multi-bootloader.

  \nl

  These modules will be passed to the kernel at the boot time.

  \nl

  Be careful, a module here is not a module in the Linux or BSD terms.

  \nl

  A module is simply a file to load.
\end{frame}

% 7)

\begin{frame}
  \frametitle{\$USER.conf}

  Finally the main configuration file contains the configuration
  variables for the development environment.

  \nl

  This file uses the syntax of the make files.

  \nl

  Every variable defined in this file will be used by the makefiles
  and the scripts.
\end{frame}

%
% env
%

\section{env}

% 1)

\begin{frame}
  \frametitle{Overview}

  The \textbf{env} directory contains the different development environments.

  \nl

  This directory is the heart of the kaneton development system.

  \nl

  Indeed, a user can develop the kaneton project on a Mac machine using
  cross compilation for Intel processors ('cause PowerPC processor)
  while another one is using a FreeBSD operating system on an Intel processor.

  \nl

  So, the development environment has to deal with these different operating
  systems and architectures just for the development.
\end{frame}

% 2)

\begin{frame}
  \frametitle{Our System}

  To do this, we decided to introduce an environment system.

  \nl

  Every time a user gets the kaneton development tarball, he first has to
  create his development environment given a couple operating system and
  architecture which leads to an environment.

  \nl

  Once the environment is installed, the user can develop, compile the kernel
  etc.. without problems because everything (makefiles, scripts etc..) use
  the binaries, variables etc.. for his environment.

  \nl

  The environment is specified in the user configuration file.
\end{frame}

% 3)

\begin{frame}[containsverbatim]
  \frametitle{Tree}

  \begin{verbatim}
    env/
      clean.sh
      init.sh
      unix/
        clean.sh
        init.sh
        kaneton.mk
      macos-powerpc.ia32/
  \end{verbatim}

  \nl

  Here the \textbf{unix} is considered as the generic unix
  environment but everyone can add a specific linux, FreeBSD, Solaris etc..
  environment.
\end{frame}

% 4)

\begin{frame}
  \frametitle{init.sh}

  The \textbf{init.sh} shell script is used to install the development
  environment.

  \nl

  This script first gets the configuration variables from the user
  configuration file, then calls the specific \textbf{init.sh} script
  of the given environment.

  \nl

  Finally the script installs some links and initialises the makefile
  dependencies.

  \nl

  The \textit{[environment]}/init.sh shell script is used to install
  specific stuff.
\end{frame}

% 5)

\begin{frame}
  \frametitle{clean.sh}

  The \textbf{clean.sh} shell script just cleans the environment.

  \nl

  This shell script also call the environment specific clean.sh script.
\end{frame}

% 6)

\begin{frame}
  \frametitle{kaneton.mk}

  The \textbf{kaneton.mk} makefile dependency is the heart of the
  kaneton compilation system.

  \nl

  Indeed, every makefile is composed of calls to special routines
  which are implemented by the makefile dependency depending on the
  environment: operating system plus architecture source and destination.

  \nl

  Moreover the \textbf{kaneton.mk} makefile dependency includes the
  user configuration file so each makefile of the system is able to
  use user defined variables.

  \nl

  The kaneton compilation system uses a very special gmake feature:
  the makefile \textbf{call} function.
\end{frame}

% 7)

\begin{frame}[containsverbatim]
  \frametitle{Use}

  \begin{verbatim}
    $ make init
    [+] installing environment

    [+] your current configuration:
    [+]   environment:              unix
    [+]   architecture:             ia32
    [+]   multi-bootloader:         grub

    [...]

    $ make clean
    [+] cleaning environment

    [...]

    $ 
  \end{verbatim}
\end{frame}

%
% tools
%

\section{tools}

% 1)

\begin{frame}
  \frametitle{Overview}

  The \textbf{tools} directory contains programs, scripts, special
  files used by the kaneton project.

  \nl

  For example a script to initialise and install modules on a grub
  bootloader boot device is provided in the subdirectory
  \textit{scripts/multi-bootloaders/grub/}.

  \nl

  The \textbf{tools} directory also contains the ld scripts used
  to correctly compile the bootstrap, the bootloader, the kernel, the
  drivers, the services and the programs.
\end{frame}

% 2)

\begin{frame}[containsverbatim]
  \frametitle{Tree}

  \begin{verbatim}
    tools/
      scripts/
        ld/
          arch/
            ia32/
              bootstrap.lds
              bootloader.lds
              kaneton.lds
              driver.lds
              service.lds
              user.lds
        multi-bootloaders/
          grub/
          lilo/
        prototypes/
          mkp.py
  \end{verbatim}
\end{frame}

% 3)

\begin{frame}[containsverbatim]
  \frametitle{Use}

  \begin{verbatim}
    $ make build
    [+] initialising boot system

    [+] boot system initialised successfully
    $ make install
    [+] initialising boot system

    [+] /tmp/menu.lst
    [+] core/bootloader/bootloader
    [+] core/kaneton/kaneton
    [+] conf/mycure/kaneton.conf
    [+] drivers/cons/cons
    [+] services/dsh/dsh

    [+] boot system initialised successfully
    $ 
  \end{verbatim}
\end{frame}

% 4)

\begin{frame}[containsverbatim]
  \frametitle{Prototypes}

  The compilation system permits to generate the prototypes in a very easy
  and elegant way.

  \begin{verbatim}
    $ make proto
    [PROTOTYPES]            libdata.h
    [PROTOTYPES]            libstring.h
    [PROTOTYPES]            libsys.h
    [PROTOTYPES]            bootloader.h
    [PROTOTYPES]            ia32.h
    [PROTOTYPES]            kaneton.h
    [PROTOTYPES]            as.h
    [PROTOTYPES]            conf.h
    [PROTOTYPES]            serial.h

    [...]

    $ 
  \end{verbatim}
\end{frame}

% 5)

\begin{frame}[containsverbatim]
  \frametitle{Explanations}

  This system is based on tags in the header files which specify
  from which files to extract prototypes.

  \nl

  The tags are of the form:

  \begin{verbatim}
    /*
     * ---------- prototypes -------------------------------------------------
     *
     *      ../../kaneton/set/set.c
     *      ../../kaneton/set/set_array.c
     *      ../../kaneton/set/set_ll.c
     *      ../../kaneton/set/set_bpt.c
     */
  \end{verbatim}
\end{frame}

% 5)

\begin{frame}[containsverbatim]
  \frametitle{Dependencies}

  The compilation system uses full dependencies between files.

  \nl

  To regenerate the dependencies, for example when adding a
  \textit{\#include} c-preprocessor directive in a source file:

  \begin{verbatim}
    $ make dep
    [REMOVE]                .makefile.mk
    [DEPENDENCIES]          dump.c
    [DEPENDENCIES]          alloc.c
    [DEPENDENCIES]          sum2.c

    [...]

    $ 
  \end{verbatim}
\end{frame}

%
% libs
%

\section{libs}

% 1)

\begin{frame}
  \frametitle{Overview}

  The \textbf{libs} directory contains the libraries used by the kaneton
  project like:

  \begin{itemize}
    \item
      libc
    \item
      crt
    \item
      libposix
    \item
      etc..
  \end{itemize}
\end{frame}

%
% core
%

\section{core}

% 1)

\begin{frame}
  \frametitle{Overview}

  The \textbf{core} directory contains the source code for the microkernel
  including the bootstrap, the bootloader and the kernel itsef.

  \nl

  Each part contains an \textbf{arch} directory used for architecture
  dependent soure code.
\end{frame}

% 2)

\begin{frame}[containsverbatim]
  \frametitle{Tree}

  \begin{verbatim}
    core/
      bootstrap/
        arch/
          ia32/ <---;
          machdep --+
      bootloader/
        arch/
      kaneton/
        arch/
        as/
        conf/
        debug/
        id/
        segment/
        set/
        stats/
  \end{verbatim}
\end{frame}

%
% drivers
%

\section{drivers}

% 1)

\begin{frame}
  \frametitle{Overview}

  The \textbf{drivers} directory contains the drivers of the kaneton
  microkernel.

  \nl

  A driver, in the kaneton terms, is a microkernel server which is allowed
  to communicate with hardware devices.
\end{frame}

% 2)

\begin{frame}[containsverbatim]
  \frametitle{Tree}

  \begin{verbatim}
    drivers/
      cons/
        Makefile
        cons.c
      dma/
      kbd/
      ide/
  \end{verbatim}
\end{frame}

%
% services
%

\section{services}

% 1)

\begin{frame}
  \frametitle{Overview}

  The \textbf{services} directory contains the services of the kaneton
  microkernel.

  \nl

  A service, in the kaneton terms, in simply a server which does not
  communicate with the hardware.
\end{frame}

% 2)

\begin{frame}[containsverbatim]
  \frametitle{Tree}

  \begin{verbatim}
    services/
      dsh/
      mod/
        Makefile
        mod.c
        modfs.c
  \end{verbatim}
\end{frame}

%
% programs
%

\section{programs}

% 1)

\begin{frame}
  \frametitle{Overview}

  The \textbf{programs} directory contains the sources of common
  programs.

  \nl

  A program in the kaneton terms is just a non-privilegied
  process.
\end{frame}

% 2)

\begin{frame}[containsverbatim]
  \frametitle{Tree}

  \begin{verbatim}
    programs/
      ls/
      wc/
      cat/
      mount/
      umount/
      gcc/
      emacs/
  \end{verbatim}
\end{frame}

%
% export
%

\section{export}

% 1)

\begin{frame}
  \frametitle{Overview}

  The \textbf{export} directory is used to create kaneton distribution.

  \nl

  This feature is especially used by the maintainers of the kaneton
  project which create very special kaneton distribution for
  the students.
\end{frame}

% 2)

\begin{frame}[containsverbatim]
  \frametitle{Use}

  The only way to export kaneton is to do like this:

  \begin{verbatim}
    $ make export
    [!] usage: exporter.sh [stage]

    available stages: k0 k1 k2 k3 k4 k5 k6 k7 k8 k9 kaneton dist
    $ make export-k3
  \end{verbatim}

  \begin{itemize}
    \item
      \textbf{k[0-9]}: create a special kaneton version for the k[0-9]
      subproject
    \item
      \textbf{kaneton}: create an entire kaneton version for the lastest
      subproject
    \item
      \textbf{dist}: create an entire backup of the kaneton development
      project
  \end{itemize}
\end{frame}

%
% papers
%

\section{papers}

% 1)

\begin{frame}
  \frametitle{Overview}

  The \textbf{papers} directory contains the papers and lectures
  in relation with the kaneton project.

  \nl

  We prefered set the papers directly into the tarball so every student
  can easily read them.
\end{frame}

% 2)

\begin{frame}[containsverbatim]
  \frametitle{Tree}

  \begin{verbatim}
    papers/
      assignments/
      design/
      kaneton/
      seminar/
      lectures/
        kernels/
        inline-assembly/
        c-preprocessor/
        distributed-operating-systems/
        arch-ia32/
  \end{verbatim}
\end{frame}

% 3)

\begin{frame}[containsverbatim]
  \frametitle{Use}

  \begin{verbatim}
    $ make view
    [+] papers:

    [+]   assignments
    [+]   design
    [+]   arch-ia32
    [+]   c-preprocessor
    [+]   distributed-operating-systems
    [+]   inline-assembly
    [+]   kernels
    [+]   development-environment

    [!] usage: viewer.sh [paper]
    $ make view-design
  \end{verbatim}
\end{frame}

%
% doc
%

\section{doc}

% 1)

\begin{frame}
  \frametitle{Overview}

  The \textbf{doc} directory contains every document useful for
  the development of the kaneton project.

  \nl

  This directory will theorically contain documents on the different
  architectures, documents on some hardware devices like ide, usb etc..
\end{frame}

\end{document}

%%
%% licence       kaneton licence
%%
%% project       kaneton
%%
%% file          /home/mycure/kaneton/view/papers/kaneton/nomenclature.tex
%%
%% created       julien quintard   [tue mar  7 14:19:17 2006]
%% updated       julien quintard   [sun apr  9 23:56:13 2006]
%%

%
% nomenclature
%

\chapter{Nomenclature}

In this chapter we will detail the kaneton nomenclature which is very
specific to the project.

Then, the reader will be able to understand the next chapters since the
kaneton documents heavily use the kaneton nomenclature.

Note that certain managers use another nomenclature for their own
purpose.

\newpage

%
% text
%

Like any other important project, the kaneton microkernel project
uses many terms to define very precise things.

The kaneton nomenclature was introduced to make communication between
developers easier. Indeed it is somethimes hard to speak widely using
terms like \textit{architecture-independent soure code},
\textit{architecture-dependent source code},
\textit{contiguous area of used physical memory},
\textit{contiguous area of free virtual memory}, etc..

The reader should notice that these terms are complex and a bit confused
when used together in the same sentence.

Then, kaneton people decided to introduce a well defined nomenclature
to make things clear.

We will so describe this nomenclature in the next sections.

%
% microkernel
%

\section{Microkernel}

The kaneton microkernel source code is composed of two majors parts, the
architecture-independent part and the architecture-dependent part.

The architecture-independent source code is called \textbf{core} and
make mainly reference to the kaneton managers. The architecture-dependent
source code is called \textbf{machdep} for \textit{machine-dependent}.

The kaneton microkernel is composed of \textbf{objects}. Technically,
an object, in kaneton terms, is simply a structure that begins with
a kaneton identifier. All kaneton objects are identifiers by kaneton
identifiers and are manipulated by kaneton capabilities.

An \textbf{identifier} is just a number referencing a kaneton object.
The kaneton microkernel reference implementation uses 64-bit identifiers.

A \textbf{capability} is a kind of cryptographic key used to reference
kaneton objects over the network.

Moreover, the kaneton microkernel is divided into subparts to make the
microkernel very clear to students. The subparts are called \textbf{managers}.

A \textbf{manager} is simply a piece of coherent code which takes care
of a unique kaneton object's type. For example the \textit{segment manager}
provides simple operations to use physical memory.

The whole core is composed managers which manage multiple kaneton objects.

A \textbf{segment} object describes a contiguous area of physical memory.
Each segment has a base address, a size, permissions and other properties.

A \textbf{region} object describes a contiguous area of virtual memory.
Each region references a segment meaning that a part of this segment
is mapped by this region.

An \textbf{as} or \textbf{address space} is an abstraction describing
a set of addressable memory addresses. An address space object is composed
of a set of segments listing the useable physical memory and a set of
regions listing the useable virtual memory.

A \textbf{thread} object is the kaneton active entity. Indeed, a thread
describes an execution context.

The \textbf{scheduler} manages threads using classes, behaviours and
priorities.

A \textbf{task} object is an abstraction describing a complete execution
context with addressable memory and threads. A task is composed of
an address space and a set of threads. Note that kaneton tasks are
native multi-threaded tasks.

An \textbf{event} object describes an external event including hardware
events like interrupts as well as software events also known as system
calls.

The kaneton microkernel uses a very specific object called \textbf{set}.
A set object is an abstract data structure managed by the set manager.
A set is just used to store data without taking care of how it is technically
done. The set concepts was introduced to make the microkernel code as
clear as pseudo code.

The kaneton core was implemented following a very strict source code
nomenclature we will now detail.

Each manager provides an interface to manipulate a kaneton object or
something else. The naming scheme used for these provided functions
is the explained below.

The function \textit{manager}\_\textbf{init}() initializes the whole
manager and the function \textit{manager}\_\textbf{clean}() cleans it.

The function \textit{manager}\_\textbf{show}() displays information
on a precise object while the function \textit{manager}\_\textbf{dump}()
displays information on every objects managed.

The function \textit{manager}\_\textbf{reserve}() reserves an object
given some properties and the function \textit{manager}\_\textbf{release}()
releases it.

The function \textit{manager}\_\textbf{clone}() clones an object. It is
important to understand that cloning an object does not just mean
generating an identical object. Indeed, cloning an object means creating
a new object with the same properties. Notice that the fact of creation
implies a new object so a new identifier.

The function \textit{manager}\_\textbf{give}() gives an object to another
entity.

The function \textit{manager}\_\textbf{flush}() cleans every objects
previously reserved depending on the implementation.

The \textit{manager} must be replaced by the manager name like \textit{as},
\textit{task}, \textit{segment} etc..

Moreover, the kaneton development environment are organized in a very
strict way to help students to find the files they are looking for. Indeed,
given a manager name \textit{manager}, the files in relation with this
manager are located in the directory listed below:

\begin{itemize}
  \item
    \textbf{kaneton/core/}\textit{manager}\textbf{/}\textit{manager}\textbf{.c}
  \item
    \textbf{kaneton/core/arch/}\textit{architecture}\textbf{/}\textit{manager}\textbf{.c}
  \item
    \textbf{kaneton/include/core/}\textit{manager}\textbf{.h}
  \item
    \textbf{kaneton/include/arch/}\textit{architecture}\textbf{/core/}\textit{manager}\textbf{.h}
\end{itemize}

%
% operating system
%

\section{Operating System}

The kaneton microkernel manages four different task classes.

A \textbf{program} is the lowest priviliged task f the system. The common
user programs are the well known UNIX{\copyright} binaries like
\textit{/bin/ls}, \textit{/bin/sh} etc..

A \textbf{service} is a microkernel server which provides a logical
service. For example, a service could be the \textit{Virtual File System}
which dispatches the calls to the filesystems to the correct servers.

A \textbf{driver} is a service which performs hardware communication.
For example, a \textit{Wireless driver} is a driver in the kaneton terms.

Finally, a \textbf{core} is a task which is a kind of super-driver in which
it has full rights on the whole machine.

A \textbf{module} is simply an additional file passed at the boot time.

XXX

The kaneton developers wanted to make things simple to add source code
at the beginning and/or at the end of the functions of a manager. To do
so, some macro functions were introduced.

%%
%% licence       kaneton licence
%%
%% project       kaneton
%%
%% file          /home/mycure/kaneton/view/papers/kaneton/coding-style.tex
%%
%% created       matthieu bucchianeri   [mon jan 30 17:32:57 2006]
%% updated       julien quintard   [thu mar  2 13:57:17 2006]
%%

%
% coding style
%

\chapter{Coding style}

The kaneton project developers try to follow a coding style. This
coding style was introduced to normalize the source code, leading to a
more readable source code.

Nevertheless, you can adapt this coding style to your own but try to
follow the rules.

%
% case
%

\section{Case}

The whole kaneton source code is written using lower case letters.

Moreover, every text including comments etc.. must be written using
lower case letters

%
% headers
%

\section{Headers}

Each file must start with an header formatted as shown below:

\begin{verbatim}
/*
 * licence       kaneton licence
 *
 * project       kaneton
 *
 * file          /home/mycure/kaneton/core/kaneton/as/as.c
 *
 * created       julien quintard   [fri feb 11 02:23:41 2005]
 * updated       matthieu bucchianeri   [mon jan 30 20:30:57 2006]
 */
\end{verbatim}

An emacs configuration file for automatically generating and updating
this header can be found in \textit{tools/emacs}.

Additionally, you need to set two environment variables to generate
a correct kaneton header:

\begin{itemize}
  \item
    \textbf{EC\_LICENCE} must be set to ``kaneton licence''.
  \item
    \textbf{EC\_DEVELOPER} must be set to your first name and last name.
\end{itemize}

Please, do not use nicknames in headers.

%
% naming convention
%

\section{Naming Convenions}

To keep the code as clear as possible, there are several conventions on
types, functions and variables naming.

%
% variables
%

\subsection{Variables}

Here are a few rules you are encouraged to follow:

\begin{itemize}
  \item
    \textbf{sz} suffix for variables representing a size.

    \begin{verbatim}
      #define PAGESZ          4096

      int                     modsz;
    \end{verbatim}
  \item
    \textbf{n} prefix for variables representing a number of objects.

    \begin{verbatim}
      int                     nclusters;
    \end{verbatim}
  \item
    etc..
\end{itemize}

Moreover, the types are used as pre-names:

\begin{verbatim}
t_vaddr                 video_vaddr;
\end{verbatim}

This example is not correct, instead prefer:

\begin{verbatim}
t_vaddr                 video;
\end{verbatim}

%
% functions
%

\subsection{Functions}

Function names must be prefixed by the file name, context name they are
implemented in.

For example, a function part of the address space manager must be prefixed
by \textit{as\_}.

These names must be chosen carefully: they must explicitely define
what the function does without being too long.

%
% types
%

\subsection{Types}

As variables and functions, type names must be expressed in english
with lower case letters.

Here are the prefixes you must use when writing your own types:

\begin{itemize}
  \item
    \textbf{m\_} for managers main structures.
  \item
    \textbf{o\_} for kaneton objects.
  \item
    \textbf{i\_} for interfaces.
  \item
    \textbf{d\_} for architecture-dependent structures.
  \item
    \textbf{s\_} for general purpose structures.
  \item
    \textbf{t\_} for basic and general purpose typedefs.
  \item
    \textbf{c\_} for kaneton capabilities.
  \item
    \textbf{u\_} for kaneton unique identifiers.
\end{itemize}

Notice that \textbf{d\_} can be combined with other prefixes, for
example \textbf{do\_} for a dependent object.

%
% includes
%

\section{Includes}

To keep the code clear and compact, developers only need to include a
minimal number of header files:

\begin{itemize}
  \item
    \textbf{kaneton.h} for the microkernel declarations.
  \item
    \textbf{klibc.h} for the kaneton specific C library.
\end{itemize}

These files are located in the include path, so do not use relative include
path.

\begin{verbatim}
#include <libc.h>
#include <kaneton.h>

int             main(int                argc,
                     char**             argv)
{
  [...]

  return (0);
}
\end{verbatim}

All include files must be protected against multiple inclusions. The
guard macro to use must be named using the directory name, one underscore,
the file name, one underscore and a capital ``H''.

For example, the file \textit{core/include/kaneton/segment.h} will be
guarded as follow:

\begin{verbatim}
#ifndef KANETON_SEGMENT_H
#define KANETON_SEGMENT_H	1

[...]

#endif
\end{verbatim}

In addition, for architecture-dependent files, the guard macro must begin
with the architecture name; for example for the Intel architecture:
\textit{IA32\_KANETON\_SEGMENT\_H}.

%
% types
%

\section{Types}

You may use as soon as possible standard types: \textbf{t\_uint8},
\textbf{t\_sint32}, \textbf{t\_uint64} etc..

This nomenclature is more understandable than
\textbf{unsigned long long int}.

%
% return values
%

\section{Return Values}

Every function must report whether it successed or failed.

In kaneton, functions' return type must be \textbf{t\_error}.

A function will return \textbf{ERROR\_NONE} on success and anything
else on error, for example \textbf{ERROR\_UNKNOWN} to indicate a non-specific
error.

%
% indentation
%

\section{Indentation}

There are several indentation rules in kaneton.

\begin{enumerate}
  \item
    Field names of structures and unions must be aligned with the
    structure or union name.

    \begin{verbatim}
      struct       s_set
      {
        u_set      id;
        t_setsz    size;
        t_type     type;
      };
    \end{verbatim}

    or

    \begin{verbatim}
      typedef struct
      {
        o_id       id;
        u_stats    stats;
        u_set      container;
      }            m_as;
    \end{verbatim}
  \item
    Macros and variables must be aligned as shown below:

    \begin{verbatim}
      #define TASK_PRIOR_CORE     230
      #define TASK_HPRIOR_CORE    250
      #define TASK_LPRIOR_CORE    210

      m_task*                     task;
      u_task                      ktask = ID_UNUSED;
    \end{verbatim}

    This rule also applies for variables declarations in functions.
  \item
    Function prototypes and bodies should look like this:

    \begin{verbatim}
      t_error             stats_function(u_stats          id,
                                         char*            function,
                                         t_stats_func**   f)
      {
        t_sint64          slot = -1;
        t_sint64          i;

        [...]
      }
    \end{verbatim}

    Notice that argument names are aligned between each other,
    and variable names are aligned with function name and between
    each other.

    Try to respect this alignment between functions in a single file:
    function names may be all aligned and argument names also.
\end{enumerate}

%
% comments
%

\section{Comments}

As kaneton is intended to be a pedagogical project with clear and
understandable source code; no need to say that comments take a very
important part of this objective.

Every file must begin with a comment describing what is done in this
code via the \textit{information} section.

Moreover, every function must be preceded by a comment defining its
behavior.

For complex functions and yo prevent direct comments in the source code,
we used \textbf{steps}:

\begin{itemize}
  \item
    Each critical code section in a function is preceded by a step
    number.
  \item
    The function header comment contains steps descriptions.
\end{itemize}

An example is present below:

\begin{verbatim}
/*
 * this function shows the usage of comments and steps.
 *
 * steps:
 *
 * 1) compute the index.
 * 2) make the operation.
 * 3) check the result.
 */

t_error         test_foobar(int      a,
                            int      b,
                            int*     c)
{
  int           index;

  /*
   * 1)
   */

  index = text_make_index(a, b);

  /*
   * 2)
   */

  index = index * a + b;

  /*
   * 3)
   */

  if (index < 0)
    return (ERROR_UNKNOWN);

  *c = index;

  return (ERROR_NONE);
}
\end{verbatim}

%
% sections
%

\section{Sections}

kaneton files are divided in multiple sections.

Section are delimited as shown below:

\begin{verbatim}
/*
 * ---------- includes ------------------------------------------------
 */
\end{verbatim}

Possible sections in a file are:

\begin{itemize}
  \item
    \textbf{header files}: information, dependencies, defines, types,
    prototypes, macros, etc..
  \item
    \textbf{source files}: information, extern, globals, includes,
    functions, etc..
  \item
    \textbf{make files}: dependencies, directives, variables, rules, etc..
\end{itemize}

Moreover, every important file, for example the main file of each
kaneton manager, have to contain a section \textit{information} describing
the whole manager.

In addition, a section named \textit{assignments} is generally necessary
for manager will be filled in by the students. This section briefly describes
the work to be done by the students.

%
% macros
%

\section{Macros}

The kaneton microkernel uses few fundamental macros lited below:

\begin{itemize}
  \item
    \textbf{\_\_\_bootloader} indicates that this source code belongs to
    the bootloader.
  \item
    \textbf{\_\_\_kernel} indicates that this source code belongs to the
    microkernel.
  \item
    \textbf{\_\_\_kaneton} indicates that this kernel is the kaneton
    microkernel.
  \item
    \textbf{\_\_\_wordsz} indicates the word size: 16-bit, 32-bit,
    64-bit etc..
  \item
    \textbf{\_\_\_endian} indicates the endianness.
\end{itemize}

%
% ---------- header -----------------------------------------------------------
%
% project       kaneton
%
% license       kaneton
%
% file          /home/mycure/kaneton/view/book/kaneton/boot.tex
%
% created       julien quintard   [mon dec 17 21:36:28 2007]
% updated       julien quintard   [wed dec 19 21:33:14 2007]
%

%
% ---------- boot -------------------------------------------------------------
%

\chapter{Boot}
\label{chapter:label}

This chapter contains the kaneton boot specifications. Every bootloader
willing to run a kaneton microkernel instance needs to comply to these
specifications.

\newpage

%
% ---------- text -------------------------------------------------------------
%

The kaneton microkernel is not directly launched when the computer is
turned on. Indeed, a \textbf{bootloader} first set up an execution environment
so that the kernel can be launched properly. The bootloader takes some
\textbf{inputs} which represents additional files: configuration files,
execution files etc. For example, the first input the bootloader uses is
the kaneton microkernel binary file which is loaded and launched by the
bootloader itself. In addition, the second input must be the \textit{mod}
service which is launched by the kernel.

The kaneton microkernel is launched with an \textbf{init} structure
as argument. This structure is described next.

\begin{verbatim}
  typedef struct
  {
    t_paddr                       mem;
    t_psize                       memsz;

    t_paddr                       kcode;
    t_psize                       kcodesz;

    t_paddr                       mcode;
    t_psize                       mcodesz;
    t_vaddr                       mlocation;
    t_vaddr                       mentry;

    t_paddr                       init;
    t_psize                       initsz;

    t_inputs*                     inputs;
    t_psize                       inputssz;

    t_uint32                      nsegments;
    s_segment*                    segments;
    t_psize                       segmentssz;

    t_uint32                      nregions;
    s_region*                     regions;
    t_psize                       regionssz;

    t_uint32                      ncpus;
    s_cpu*                        cpus;
    t_psize                       cpussz;
    i_cpu                         bsp;

    t_paddr                       kstack;
    t_psize                       kstacksz;

    t_paddr                       alloc;
    t_psize                       allocsz;

    machine_data(init);
  }                               t_init;
\end{verbatim}

This structure informs the kernel about the memory layout i.e the location
of the different elements in memory as these locations vary according to
the machine.

Note that size fields must be aligned on \texttt{PAGESZ}. Indeed, the
core memory managers behave at the byte level. It is the machine responsibility
to call the core with properly aligned sizes.

%
% core
%

\section{Core}

% memory

\subsection*{Memory}

The \texttt{mem} and \texttt{memsz} fileds specify the offset and the size of
the underlying hardware's RAM.

The \texttt{mem} attribute is very likely to be set to zero but could
vary on specific platforms.

% kernel code

\subsection*{Kernel Code}

The two fields \texttt{kcode} and \texttt{kcodesz} specify the physical memory
location and size of the kernel code.

% mod service

\subsection*{\textit{mod} Service}

The \textit{mod} service is the very first server launched by the kaneton
microkernel. This server is responsible for creating and starting the
other servers.

Fields \texttt{mcode} and \texttt{mcodesz} specify the location of the physical
memory area containing the \textit{mod} service code.

\texttt{mlocation} contains the virtual memory address the code area must
be mapped whilst \texttt{mentry} contains the virtual address of the code's
entry point.

These fields were introduced so that the parsing of the \textit{mod} binary
is performed by the bootloader. This way, the kernel does not have take
care of handling multiples executable file formats like \textit{ELF},
\textit{COFF} etc.

Indeed, the kernel receives the \textit{init} structure, creates a new task,
maps the \textit{mod} service code and points the task's thread to the
entry point.

% init structure

\subsection*{\textit{init} Structure}

The \texttt{init} and \textit{initsz} fields contain the location and size
of the \textit{init} structure itself.

% inputs

\subsection*{Inputs}

Inputs are additional files passed to the \textit{mod} service.

Theses files are gathered together in a single memory area specified through
the \texttt{input} and \texttt{inputsz} fields.

This area first contains metadata with the \texttt{t\_inputs} structure:

\begin{verbatim}
  typedef struct
  {
    t_uint32                      ninputs;
  }                               t_inputs;
\end{verbatim}

The \texttt{ninputs} field of the metadata obviously indicates the number
of inputs. These inputs follow the metadata as explained next.

Inputs are actually organised in an array of elements composed of the input
metadata and the input contents. The input metadata is described by the
\texttt{t\_init} structure:

\begin{verbatim}
  typedef struct
  {
    char*                         name;
    t_psize                       size;
  }                               t_input;
\end{verbatim}

Everything related to inputs is packed in a single location so that passing
these information to the \textit{mod} service is as simple as passing the
address and size of this memory area i.e \texttt{inputs} and \texttt{inputssz}.

% segments

\subsection*{Segments}

Segments passed by the bootloader to the kernel indicate the zones of
physical memory which are already used. Thus, the kaneton microkernel can
initialise its memory managers, especially the segment manager, according
to those zones.

The \texttt{nsegments} attribute indicates the number of segments in the
array located in the memory area specified by \texttt{segments} and
\texttt{segmentssz}.

Elements of the array of segment are of the following type:

\begin{verbatim}
  typedef struct
  {
    t_paddr                       address;
    t_psize                       size;
    t_perms                       perms;
  }                               s_segment;
\end{verbatim}

% regions

\subsection*{Regions}

Regions provided through the \textit{init} structure indicate the kernel
which memory locations are already mapped.

The kernel can use these information for initialising its memory managers,
in this case the region manager, so that data structures are coherent.

As for the segments, the \texttt{nregions} regions are gathered in an
array located at \texttt{regions} of size \texttt{regionssz}.

Every element of the region array are of the following type:

\begin{verbatim}
  typedef struct
  {
    t_uint32                      segment;

    t_vaddr                       address;
    t_paddr                       offset;
    t_vsize                       size;
    t_opts                        opts;
  }                               s_region;
\end{verbatim}

Note that the \texttt{segment} field of this last structure correspond to
an index in the array of segments.

% processors

\subsection*{Processors}

The \texttt{ncpus} field indicates the number of cpu elements in the array
located at \texttt{cpus} of size \texttt{cpussz}. Each element is of the
following form:

\begin{verbatim}
  typedef struct
  {
    i_cpu                         cpuid;
  }                               s_cpu;
\end{verbatim}

Additionally, the \texttt{bsp} field indicates the identifier of the boot
processor.

% kernel stack

\subsection*{Kernel Stack}

The kernel stack is specified through the \texttt{kstack} and
\texttt{kstacksz}.

% allocator's pre-reserved memory

\subsection*{Allocator's Pre-Reserved Memory}

The memory area specified by \texttt{alloc} and \texttt{allocsz} is used
by the kaneton microkernel for performing allocations in order to set up
the memory managers.

Indeed, when the kaneton microkernel starts, the memory managers are
not initialised and hence cannot provide memory management functionalities.
Traditional kernels tend to rely on a specifically designed physical memory
manager for this purpose. This design leads to an ugly implementation.

kaneton people wanted to avoid that and decided to rely on a pre-reserved
memory area provided by the bootloader.

%
% machine
%

\section{Machine}

The reader would have probably noticed the use of the \texttt{machine\_data()}
macro-function in the \textit{init} structure.

Indeed, the \textit{init} structure can include machine-dependent information
that will be later used by the kaneton machine components.

For instance, the \textit{IA-32} kaneton bootloader includes, through
the \texttt{machine\_data()} macro-function, the following information in
the \textit{init} structure:

\begin{verbatim}
  #define         machine_data_init()                                     \
    struct                                                                \
    {                                                                     \
      t_ia32_gdt                  gdt;                                    \
      t_ia32_directory            pd;                                     \
    }
\end{verbatim}

%%
%% licence       kaneton licence
%%
%% project       kaneton
%%
%% file          /home/mycure/kaneton/view/papers/kaneton/core.tex
%%
%% created       matthieu bucchianeri   [mon jan 30 17:33:29 2006]
%% updated       julien quintard   [fri mar 10 01:50:49 2006]
%%

%
% core
%

\chapter{Core}

\newpage

%
% text
%

%
% bootstrap
%

\section{Bootstrap}

XXX

%
% bootloader
%

\section{Bootloader}

XXX

%
% kaneton
%

\section{kaneton}

XXX

%
% id
%

\section{id}

The following rules describes a typical use of id objects in kaneton:

\begin{itemize}
  \item
    The \textbf{init} function of a manager calls \textbf{id\_build}
    to generate a \textbf{o\_id} object.
  \item
    Functions generating new objects will use \textbf{id\_reserve} to
    reserve new identifiers for the created objects.
  \item
    Functions removing objets will call \textbf{id\_destroy} to release
    identifiers of destroyed objets.
  \item
    The \textbf{clean} function of a manager will release the identifier
    generator with \textbf{id\_release}.
\end{itemize}

%%
%% licence       kaneton licence
%%
%% project       kaneton
%%
%% file          /home/mycure/kaneton/view/papers/kaneton/drivers.tex
%%
%% created       julien quintard   [thu may  4 11:58:45 2006]
%% updated       julien quintard   [thu may  4 12:03:34 2006]
%%

%
% drivers
%

\chapter{Drivers}

In this chapter we will overview the kaneton fundamental drivers.

Since drivers are generally highly dependent from the microprocessor's
architecture, we will only describe in this chapter the generic drivers
while specific ones will be describes in architecture-specific papers.

These architecture-specific papers can be found on the official kaneton
website
  \footnote{http://www.kaneton.org}

\newpage

%
% text
%

Recall that, in kaneton terms, a driver is a task which can communicate
with one or more hardware devices.

%%
%% licence       kaneton licence
%%
%% project       kaneton
%%
%% file          /home/mycure/kaneton/view/papers/kaneton/services.tex
%%
%% created       julien quintard   [thu may  4 12:02:18 2006]
%% updated       julien quintard   [mon may  8 17:52:15 2006]
%%

%
% services
%

\chapter{Services}

In this chapter we will describe the fundamental services the kaneton
microkernel needs to become a complete operating system.

\newpage

%
% text
%

Recall that, in kaneton terms, a service is a task which provide a
service but which never communicate with hardware devices.

\notice{Since the kaneton microkernel's development is not finished
  yet, no service was developed.}

%
% ---------- header -----------------------------------------------------------
%
% project       kaneton
%
% license       kaneton
%
% file          /home/mycure/kane...ture/kernels/portability/portability.tex
%
% created       julien quintard   [fri oct 24 17:31:58 2008]
% updated       julien quintard   [fri may 21 15:08:57 2010]
%

%
% ---------- setup ------------------------------------------------------------
%

%
% path
%

\def\path{../../..}

%
% template
%

%%
%% copyright     (c) julien quintard
%%
%% project       kaneton
%%
%% file          /home/mycure/kaneton/view/templates/lecture.tex
%%
%% created       julien quintard   [sat nov 19 17:13:03 2005]
%% updated       julien quintard   [fri dec  2 22:36:34 2005]
%%

%
% class
%

\documentclass[8pt]{beamer}

%
% packages
%

\usepackage{pgf,pgfarrows,pgfnodes,pgfautomata,pgfheaps,pgfshade}
\usepackage{colortbl}
\usepackage{times}
\usepackage{amsmath,amssymb}
\usepackage{graphics}
\usepackage{graphicx}
\usepackage{color}
\usepackage{xcolor}
\usepackage[english]{babel}
\usepackage{enumerate}
\usepackage[latin1]{inputenc}

%
% style
%

\usepackage{beamerthemesplit}
\setbeamercovered{dynamic}

%
% verbatim font
%

\definecolor{verbatimcolor}{rgb}{0,0.4,0}

\makeatletter
\renewcommand{\verbatim@font}
  {\ttfamily\footnotesize\color{verbatimcolor}\selectfont}
\makeatother

%
% new line
%

\newcommand{\nl}[0]{\vspace{0.4cm}}

%
% date
%

\date{\today}

%
% logos
%

\pgfdeclareimage[interpolate=true,width=34pt,height=18pt]
                {epita}{../../logos/epita}
\pgfdeclareimage[interpolate=true,width=49pt,height=18pt]
                {upmc}{../../logos/upmc}
\pgfdeclareimage[interpolate=true,width=25pt,height=18pt]
                {lse}{../../logos/lse}

\newcommand{\logos}
  {
    \pgfuseimage{epita}
  }

%
% institute
%

\institute
{
  \inst{1} kaneton microkernel project
}

%
% table of contents at the beginning of each section
%

\AtBeginSection[]
{
  \begin{frame}<beamer>
   \frametitle{Outline}
    \tableofcontents[current]
  \end{frame}
}

%
% table of contents at the beginning of each subsection
%

\AtBeginSubsection[]
{
  \begin{frame}<beamer>
   \frametitle{Outline}
    \tableofcontents[current,currentsubsection]
  \end{frame}
}


%
% title
%

\title{Portability}

%
% document
%

\begin{document}

%
% title frame
%

\begin{frame}
  \titlepage
\end{frame}

%
% outline frame
%

\begin{frame}
  \frametitle{Outline}

  \tableofcontents
\end{frame}

%
% ---------- text -------------------------------------------------------------
%

%
% introduction
%

\section{Introduction}

\begin{frame}
  \frametitle{Introduction}

  Nowadays, a lot of CPU types exist (IA-32, IA-32\_64, IA-64, ARM, ARM, MIPS, PowerPC, SH, m68k, Blackfin, ...) and a lot of machines uses these CPUs, in different ways.

  \-

  The modern operating systems are trying to cope with that, and they tend to support all these platforms, so that a user can use the same system on all the different machines he uses.

  \-

  Portability requires some design in the kernel, to avoid rewriting everything from scratch and maintaining a separate branch for each supported architecture.

  \-

  This course describes what are the differences between machines, and how to design a kernel so it can be ported on several platforms.

\end{frame}

\section{Architecture}
\subsection{Machine architecture vs CPU architecture}

\begin{frame}
  \frametitle{Machine architecture vs CPU architecture}

  From a kernel point of view, there are two kind of architectures on a machine :

  \begin{itemize}
  \item The CPU Architecture
  \item The Machine/Platform Architecture
  \end{itemize}

  \-

  The following slides will explain what makes a CPU architecture specific, what makes a Machine architecture specific, and how a kernel can be designed so it can be as easily as possible ported on several machines.  

\end{frame}

\subsection{Microprocessor architecture}

\begin{frame}
  \frametitle{Microprocessor architecture}

  The architecture of a microprocessor defines :

  \begin{itemize}
  \item Its intructions set
  \item Its registers
  \item Its operational modes/behaviour
  \end{itemize}

  \-

  It's the interface between the software and the CPU itself, so it's basically what will be documented by the manufacturer in the datasheet of the CPU.

\end{frame}

\begin{frame}
  \frametitle{Microprocessor architecture}
  
  Several microprocessors models can share the same architecture.

  \-

  IA32 (commonly called x86) processors are manufactured by Intel, AMD, Via, \ldots

  \-

  ARM designs CPU architectures, manufacturers can use these specifications to make their own CPU.

\end{frame}

\begin{frame}
  \frametitle{Microprocessor architecture extensions}

  Some manufacturers are expanding one CPU architecture to provide more features, but keeping the main CPU architecture as a base.

  \-

  A code that was made to run on the base architecture will work on all the derivatives architectures, but not the opposite.

  \-

  This approach has been used quite a lot, for example, with IA-32 architecture, and the additional instruction sets (MMX, SSE, SSE2, SSE3 from Intel, 3dNow from AMD)

  \-

  A portable kernel would use these features if they are available, but must provide a software alternative in the other case. That way, the kernel can benefit from the performance gain provided by these instructions set, but doesn't depend on their presence to work.

\end{frame}

\begin{frame}
  \frametitle{CPU Architecture classes}

  Some categories have been made to distinguish families of CPU architectures :

  \begin{itemize}
  \item RISC Architecture: Reduced Instruction Set Computer - This family of CPU architectures focus on providing a really basic and simple set of instructions, and they try to optimize each instruction as much as possible, so that even something quite complex, that will require several instructions, could be faster than on a CPU where one single instruction would have been used. Such an architecture contains generally a lot of registers, so that programs can work as much as possible on data stored in registers instead of doing several memory accesses.

  \item CISC Architecture: Complex Instruction Set Computer - This family of CPU architectures focus more on providing a consequent set of instructions, so that some usual operations, that could be done using several elementary operations, are available through a single instruction. For that reason, the instruction set of a CISC CPU contains quite a lot of operations. This kind of CPU actually quite often contains a RISC CPU and a Microcode that describes how to do each instruction.

  \end{itemize}

\end{frame}

\begin{frame}
  \frametitle{Microprocessor internal and external architecture}

  All this was about external architecture.

  \-

  There is another kind of Microprocessor architecture : the internal architecture.

  \-

  This is how the CPU is actually achieving the implementation of the interface required by the external architecture.

  \-

  This is the CPU manufacturers core business, it's what makes the difference between two CPUs that have the same external architecture, like Intel vs AMD, or Via on IA-32 processors, since it's what makes the difference in terms of performance, power consumption, \ldots

\end{frame}

\subsection{Machine architecture}

\begin{frame}
  \frametitle{Machine architecture}
  
  A microprocessor can't work alone. It has interfaces to communicate with other components : memory buses, peripherial buses, interrupts lines, \ldots

  \-

  This provides a way to connect the memory required by the microprocessor, and some peripherials that can be useful in a machine, such as time-sources/clocks, user interfaces, data storage, network interfaces, \ldots

  \-

  The firmware provided in a machine is also something specific to the machine architecture since it provides services to interface with the hardware to the operating system.

  \-

  All these things are external components that are connected to the CPU in a specific way. A machine architecture describes the set of components that are used, and how they are connected together.

\end{frame}

\begin{frame}
  \frametitle{Machine architecture}

  The IBM-PC architecture is the most known architecture that uses an IA-32 CPU, but there are actually some other. SGI, for instance, made a machine based on an IA-32 CPU, but that was different from the IBM-PC architecture (SGI 320) since it didn't have a BIOS, but a firmware based on their own ARCS firmware they used on their other MIPS machines.

  \-

  More recently, Apple released their own architecture based on IA-32, where they also don't use the BIOS, but instead provide a firmware called EFI.

  \-

  Some other CPU architectures are used in several different machine architectures. For example, the MIPS R5000 CPU was used in the SGI Indy architecture, and in the Sony Playstation 2 architecture. These two machines use the same CPU architecture but they are totally different. An OS written for the SGI Indy won't work on the Sony Playstation 2.

  \-

  Almost every single Windows mobile based smartphone is a different Machine architecture, although they all share the ARM CPU architecture.

\end{frame}

\section{Kernel splitting}
\subsection{Independant part}

\begin{frame}
  \frametitle{Independant part}

  The independant part of a kernel consists in all the code that will work on every machine.

  \-

  It basically consists of all the code handling the kernel's concepts. For example, a task is a high level concept that we can find in almost all kernels. A portable kernel will be able to manage tasks on all the architectures it supports. Some of the task management done by the kernel consists in the manipulation of structures describing the task in a generic way. All this code does not rely on machines specific features, the same code will run and work on all machines. This is an independant code.

  \-

  Of course, creating a task will certainly require some specific actions as well. The goal of a portable kernel will be to isolate as much as possible those parts of the code, and to put as much things as possible in the independant part, to avoid redundancy in the dependant part.

  \-

  No ASM code can be found in the independant part since ASM is by definition dependant of the CPU architecture.

\end{frame}

\subsection{Dependant part}

\begin{frame}
  \frametitle{Dependant part}

  The dependant part of a kernel contains all the code that is aimed to support a specific feature of a CPU, a machine, or a specific hardware.

  \-

  Writing entries in the MMU cache is something specific to the CPU. It's generally done using some assembly instructions that are different from one CPU architecture to another. That's why it's a code that depends on the CPU architecture.

  \-

  Setting up the platform to trigger an interrupt on a regular basis is something specific to the Machine architecture. In general, a machine contains a dedicated hardware, connected to a CPU interrupt line. That's why configuring it is done by some code that depends on the Machine architecture.

  \-

  These examples are not always true : if a CPU embeds its own clock source that can trigger an interrupt internally, then the code to configure it is not Machine architecture dependant, but it is CPU architecture dependant.

\end{frame}

\subsection{Board support package (BSP)}

\begin{frame}
  \frametitle{Dependant part}

  Some kernels, especially the kernels for embedded operating systems, introduced the Board Support Package (BSP) concept.
  
  \-

  For example, Windows CE, Microsoft's kernel for embedded systems, is working on 3 CPU architectures : IA-32, ARM9, SuperH, but it runs on a lot of devices with a lot of various hardware devices. Each device has its own machine architecture, and its own set of peripherials. For that reason, each platform requires some specific code to make the kernel to work on it.

\end{frame}


\section{Kaneton portability}

\begin{frame}
  \frametitle{Kaneton Portability}

  The Kaneton microkernel is designed to be portable. For that reason, the code is splitted in several sections :

  \begin{itemize}
  \item Core
  \item Machine
  \begin{itemize}
  \item Architecture
  \item Platform
  \item Glue
  \end{itemize}
  \end{itemize}

\end{frame}

\subsection{Core}

\begin{frame}
  \frametitle{Kaneton Core}

  In the Kaneton design, the machine independant code is called Core.

  \-

  This section of code contains all the kernel code which is not specific to a machine. It contains the high level functions and code that are used for all the kernel internal operations, such as :
  \begin{itemize}
  \item Allocate physical memory
  \item Allocate and map virtual memory
  \item Create a new task
  \item Send a message between two tasks
  \item \ldots
  \end{itemize}

  Most of these operations don't require to do any CPU specific operations, but some of them do. For example, mapping virtual memory to physical memory requires to configure the MMU through the caches, this is something different for each CPU. For that reason, every function in the Core module of Kaneton will make a call to a potential machine dependant code.

  \-
  
  This is achieved through a macro called machine\_call.

\end{frame}

\subsection{Machine}

\begin{frame}
  \frametitle{Kaneton Architecture}
  
  This section of the Kaneton code contains all the code specific to the CPU architecture.

  \-

  This is mainly the code that will require to use specific assembly calls (inline assembly) or the code that works on data structures imposed by the CPU.

\end{frame}

\begin{frame}
  \frametitle{Kaneton Platform}
  
  This section of the Kaneton code contains all the code specific to the machine architecture.

  \-

  This is the code that will handle some peripherials attached to the CPU in a specific architecture, and that are used by the kernel itself.

\end{frame}

\begin{frame}
  \frametitle{Kaneton Glue}

  In the Kaneton design, the independant code (Core) calls the dependant code.

  \-

  To make it possible, the dependant code must have a generic interface, so that the Core code doesn't contain specifically any call to a function specific to an architecture.

  \-

  In Kaneton, the independant code and the dependant code functions share exactly the same prototypes, so that the Core code can call the Machine code in a generic way.

  \-

  For that reason, a wrapper was required, to implement the correct interface, that would then call the Architecture and the Machine code accordingly. This is the role of Glue.

\end{frame}

\section{Conclusion}

\begin{frame}
  \frametitle{Conclusion}

  In order to make a portable kernel, one must well distinguish what code can be reused for every machine, and what code is specific to the machine he is working on.

  \-

  This problem has now been addressed in most of the modern operating systems kernels (Linux, BSD, Windows CE, \ldots) with more or less style, but it's generally much better when a kernel has been designed to be portable from the beginning.

  \-

  This portability is very important nowadays, since the hardware is evolving fast. Making a new kernel without thinking about portability is a bad strategy since it makes it dependant on the durability of the architecture it is being made for.

  
\end{frame}


%
% bibliography
%

\begin{frame}[allowframebreaks]
  \frametitle{Bibliography}

  \bibliographystyle{amsplain}
  \bibliography{\path/bibliography/bibliography}
\end{frame}

\end{document}

%%
%% licence       kaneton licence
%%
%% project       kaneton
%%
%% file          /home/mycure/kaneton/view/papers/kaneton/architectures.tex
%%
%% created       julien quintard   [sun apr 23 17:08:41 2006]
%% updated       julien quintard   [sun apr 23 17:08:41 2006]
%%

%
% architectures
%

\chapter{Architectures}

This chapter will overview the different architectures supported by
the kaneton microkernel reference.

\newpage

%
% text
%

The kaneton microkernel references currently supports the following
architectures.

Notice that, sometimes, the kaneton microkernel can support multiples
\textit{architectures} of a single microprocessor's architecture.
These multiples \textit{architectures} generally exploit different
facilities of the same microprocessor's architecture.

There is a paper specific to every supported architecture. These papers
are available on the kaneton official website
  \footnote{http://www.kaneton.org}.

%
% intel architecture 32-bit
%

\section{Intel Architecture 32-bit}

The kaneton microkernel reference was first developed on the Intel Architecture
32-bit since this architecture is very popular and cheap.

The following architecture implementations are supported by the kaneton
microkernel.

%
% ia32-virtual
%

\subsection{ia32-virtual}

This architecture uses a flat segmentation model with paging enabled
allowing multiple virtual address space and so virtual address space
protections.

Nevertheless, this architecture implementation is kept as simple as possible
since it is used as the basis of the operating system courses.

One of the particularities of this architecture is that the core has its
own address space. Then, each time a system call occurs, an address space
switch is performed to work in the core's address space.

This particularity leads to bad performances since each time an address
space switch occurs, the microprocessor's caches are flushed.

%
% ia32-optimized
%

\subsection{ia32-optimized}

This architecture also relies on the Intel Architecture 32-bit but uses
every optimisations provided by the microprocessor's architecture.

\notice{This architecture is not implemented yet.}


%%
% ---------- header -----------------------------------------------------------
%
% project       kaneton
%
% license       kaneton
%
% file          /home/mycure/kaneton/view/book/development/test.tex
%
% created       julien quintard   [thu may 24 12:18:23 2007]
% updated       julien quintard   [wed dec  8 22:37:27 2010]
%

%
% ---------- test -------------------------------------------------------------
%

\subsection{Test}
\label{section:test}

The \name{test} tool enables students to test their kaneton implementation
against a set of tests that have been designed by the contributors. Below
is briefly described the terminology used by this tool in order to give
the reader an overview of the general scenario involving students, the
administrator and the server running the test system.

\begin{itemize}
  \item
    A \name{certificate} is used to make sure clients can authenticate the
    test server;
  \item
    Every certificate is sealed by a cryptographic \name{key};
  \item
    Each user is provided with a \name{capability} in order to identify
    herself to the server;
  \item
    These capabilities are sealed by a \name{code} which the server uses
    in order to detect illegally forged capabilities;
  \item
    A \name{configuration} specifies the number of tests a user is allowed
    to requests the server;
  \item
    The user's \name{database} is generated based on a configuration and
    maintains the current user's state on the server including the number
    of tests performed so far, the kaneton implementations submitted for
    evaluation \etc{}
  \item
    A \name{snapshot} is a kaneton implementation in its shipping form;
  \item
    The \name{machine} represents the target
    \name{platform}/\name{architecture} couple on which a snapshot is
    supposed to be tested or evaluated for instance;
  \item
    An \name{image} represents a kaneton snapshot compiled in a bootable
    form;
  \item
    A \name{test} is a function included in the kernel which performs a
    specific set of operations and possibly prints information to the console;
  \item
    The tests are often gathered together in a \name{suite} which represents
    the testing unit students are offered to trigger for their kaneton
    implementation;
  \item
    Once a snapshot is received by the server in order to be tested,
    the system compiles it into an image. The server also takes care to
    include the tests in the compilation process so that they can be triggered.
    These pre-compiled tests are referred to as the \name{bundle};
  \item
    The image can then be tested by triggering the tests of the suite. The
    image is therefore booted in an emulated \name{environment}. This
    environment can sometimes be chosen and offers a trade-off between
    simplicity and realism. The most common environments are \name{QEMU}
    and \name{Xen};
  \item
    Depending on the success of the tests, a set of results is generated
    and compiled in a \name{bulletin} file;
  \item
    Finally, the server retrieves this bulletin, adds some meta information
    such as the date of the test, the environment and machine used \etc{}
    and stores everything in a \name{report}. Note that this report is
    also sent back to the user so she can consult it;
  \item
    Students can also decide to submit their kaneton implementation for
    a specific \name{stage} for future evaluation. Note that suites and
    stages are completely different though they often bear the same names:
    \name{k0}, \name{k1}, \name{k2} etc{};
  \item
    The administrator can decide to evaluate the snapshots which have been
    submitted for a stage by invoking a script which will attribute grades
    according to the \textit{point}s associated with every test.
  \item
    Finally, a \name{statement} is produced containing the grades of every
    student for a given stage.
\end{itemize}

The following describes the \name{test} tool according to the user's role
regarding the system: either the administrator who sets up the system or
a student who uses it in order to improve and/or evaluate his implementation.

%
% administrator
%
\subsubsection{Administrator}

The administrator is responsible for setting up the system but also maintaining
it on a daily basis.

% requirements
\subsubsubsection{Requirements}

The \name{test} tool must be installed on a publicly accessible server since
the server script will be waiting for incoming requests. Note that by default,
the clients assume the test server to be accessible at the address:
\location{https://test.opaak.org:8421}.

Besides, since the purpose of the \name{test} tool is to run the students'
kaneton implementation in emulated environments, both \name{QEMU} and
\name{Xen} should be available though one might want to configure the tool
for supporting a single environment, \name{QEMU} for instance.

Note that the test system has been developed with \name{Python 2.6} and
may be out of date by the time the administrator sets it up. In addition,
the system depends on a variety of \name{Python} packages including
\name{argparse}, \name{yaml}, \name{pyopenssl}, \name{hmac}, \name{pickle},
\name{xmlrpc}, \name{subprocess}, \name{serial} among others.

Finally, the administrator should make sure the following applications
are installed since some test scripts need them: \name{dd}, \name{mkfs.ext2},
\name{mount}, \name{umount}, \name{mutt}, \name{sendmail}, \name{qemu}
and \name{mkisofs}.

% set up
\subsubsubsection{Set Up}

The first step for an administrator consists in generating the necessary
files, especially the certificates, code and capabilities required for
securing the test system.

The \location{test/utilities/} directory contains the scripts that perform
such operations. Note that all the generated files are stored in the
\location{test/store/} directory.

First the \name{CA - Certification Authority}'s and server's certificates
must be generated. The first is used to issue certificates while the latter
is used for clients to identify the server with absolute certainty.

\begin{verbatim}
  $> make certificate
  [+] generating the CA and server's key/certificate pair
  [+] CA key/certificate generated
  [+] server key/certificate generated
  [+] CA and server's key/certificate pair generated and stored
  $> 
\end{verbatim}

The next step consists in generating a code for the administrator to
issue capabilities but also for the server to verify that the received
capabilities have not been illegally forged.

\begin{verbatim}
  $> make code
  [+] generating the server's code
  [+] server code successfuly generated and stored
  $> 
\end{verbatim}

With a server code, the students' and contributor's capabilities can be
built, hence granting them the right the contact the server.

The following generates the contributor's capability. This capability is
special in the way that contributors can perform any operation in a completely
contrain-free manner.

\begin{verbatim}
  $> make capability-contributor
  [+] generating the contributor's capability
  [+] contributor's capability generated and stored
  $> 
\end{verbatim}

In contrast, the following command generates a set of capabilities for the
students belonging to the school referred to as \name{``epita::2010''}. Note
that the script requires the \location{history/epita/2010/} to be populated
with the groups and their \location{people} file.

\begin{verbatim}
  $> make capability-school@epita::2010
  [+] generating students' capabilities
  [+] extracting the students from the history 'epita/2010'
  [+] students information retrieved
  [+] generating the students' capabilities:
  [+]   epita::2010::group11
  [+]   epita::2010::group10
  [+]   epita::2010::group13
  [+]   epita::2010::group12
  [+]   epita::2010::group33
  [+]   epita::2010::group32
  [+]   epita::2010::group17
  [+]   epita::2010::group30
  [+]   epita::2010::group19
  [+]   epita::2010::group18
  [+]   epita::2010::group5
  [+]   epita::2010::group4
  [+]   epita::2010::group7
  [+]   epita::2010::group6
  [+]   epita::2010::group1
  [+]   epita::2010::group3
  [+]   epita::2010::group2
  [+]   epita::2010::group15
  [+]   epita::2010::group9
  [+]   epita::2010::group8
  [+]   epita::2010::group14
  [+]   epita::2010::group31
  [+]   epita::2010::group16
  [+]   epita::2010::group24
  [+]   epita::2010::group25
  [+]   epita::2010::group26
  [+]   epita::2010::group27
  [+]   epita::2010::group20
  [+]   epita::2010::group21
  [+]   epita::2010::group22
  [+]   epita::2010::group23
  [+]   epita::2010::group28
  [+]   epita::2010::group29
  [+]   epita::2010::group34
  [+] students' capabilities generated and stored
  $> 
\end{verbatim}

In addition, the administrator could decide to generate or re-generate
a capability for a specific student of a school. The following shows an
example for such an action.

\begin{verbatim}
  $> make capability-student@epita::2010::group8
  [+] generating the student's capability:
  [+]   epita::2010::group8
  [+] student's capability generated and stored
  $> 
\end{verbatim}

The next step consists in the databases generation. A database contains the
state of a user profile including the number of test requests, the quota for
such tests, the submitted snapshots and so forth. The database files are
absolutely fundamental to the server since such databases are updated after
each client's request.

The syntax for generating databases follows the one for capabilities, as
shown next for the contributor.

\begin{verbatim}
  $> make database-contributor
  [+] generating database from contributor's configuration
  [+] contributor's database generated and stored
  $> 
\end{verbatim}

Once the certificates, code, capabilities and databases generated, the
administrator can move on to the deployment process.

% deployment
\subsubsubsection{Deployment}

The deployment basically consists in copying the \location{test/} environment
to the test server though one might want to copy the whole kaneton environment
or the smallest subset of the \location{test/} directory which should, in this
case, include the following absolutely necessary items:

\begin{itemize}
  \item
    The \location{test/environments/} directory which contains the descriptions
    of the supported test environments;
  \item
    The \location{test/images/} directory which contains a script for
    automatically generating a \name{Debian Live} system which is used
    for compiling a kaneton snapshot into a bootable image;
  \item
    The \location{test/packages/} directory which contains the \name{ktp -
    Kaneton Test Package} required by the server-side standalone scripts
    for manipulating files such as databases, capabilities \etc{} but also
    for performing cryptographic operations and send/receive \name{XMLRPC}
    requests;
  \item
    The \location{test/scripts/} directory which contains the fundamental
    scripts for building bootable images, distributing the capabilities to
    the students through emails, evaluating the submitted snapshots and so on;
  \item
    The \location{test/server/} directory which contains the server script
    for handling the clients' requests;
  \item
    The \location{test/stages/} directory which contains the files requirement
    for evaluating the students' snapshots;
  \item
    The \location{test/store/} directory which contains the generated files
    such as the users' databases, the server's code and certificate; and
  \item
    The \location{test/suites/} directory which contains the files describing
    the tests to be including in a given tests suite.
\end{itemize}

Once copied, the administrator only needs to launch the server script located
in the \location{test/server/} directory, as shown below:

\begin{verbatim}
  $> ./server.py
  [meta] serving on 88.191.84.128:8421
\end{verbatim}

Note that a few additional steps may be required depending on the current state
of the kaneton development.

The first of these steps may consist in generating a \name{Debian Live} system
since this is absolutely required for the test system to work. For more
information regarding the generation of such an image, please refer to the
\location{test/images/} directory.

The second step should consist for the administrator in building the kaneton
tests bundle. The bundle represents a pre-compiled set of tests that is
included in the students' snapshot compilation process. The tests are
pre-compiled in order to prevent leaking information since students could
very well dump the content of those tests and force the compilation to fail,
hence retrieving the source code in the compilation process' error log.

In order to generate such a bundle, the administrator must first activate
the \name{test} module, as show next:

\begin{verbatim}
  _MODULES_               +=              test
\end{verbatim}

Then, the administrator must move to the \location{test/tests/} directory
and launch a compilation process through the following command:

\begin{verbatim}
  $> make
\end{verbatim}

Once generated, the test bundle, located in \location{store/bundle/[machine]/}
must be copied to the server, at the same location.

Finally, for more information on the server script, please refer to the
\location{test/server/} directory.

% scripts
\subsubsubsection{Scripts}

Although the deployment process is pretty straightforward, the administrator
is required to manage the test system through several scripts.

First, the \name{distribute} script must be used by the administrator to send
the capabilities to the respective owners so that the students can use
the test system. Note that this script relies on the \name{Mutt} mailing
system for sending the emails containing the attached capabilities.

\begin{verbatim}
  $> ./distribute.py
  recipients:
    contributor
  $>
\end{verbatim}

While the \name{construct} script enables the administrator to build a
bootable image from a kaneton snapshot, the \name{stress} script takes
a bootable image and triggers the tests belonging to the given test
suite. Note that both scripts are directly used by the server script for
building and testing the received kaneton snapshots.

\begin{verbatim}
  $> ./construct.py --snapshot kaneton.tar.bz2                          \
                    --image kaneton.img                                 \
                    --environment xen                                   \
  the kaneton image has been constructed in 'kaneton.img'
  $> ./stress.py --image kaneton.img                                    \
                 --suite k2                                             \
                 --environment xen                                      \
                 --verbose
  segment
    permissions/01 :: true
  id
    simple :: true
    clone :: true
    multiple :: true
  $> 
\end{verbatim}

Note that the administrator could also test a kaneton image manually,
especially through the following command:

\begin{verbatim}
  $> qemu -fda kaneton.img -curses
\end{verbatim}

Besides, note that an administrator willing to include a new test in the
system would probably want to test it locally first since testing through
the server takes some time. In order to test locally, the administrator
must first activate the bundle module in its user profile
\location{environment/profile/user/\${KANETON\_USER}/\${KANETON\_USER}.conf}:

\begin{verbatim}
  _MODULES_               +=              bundle
\end{verbatim}

Then, the administrator must trigger the test by calling the test function
manually in its kaneton implementation. For instance, in order to trigger
the \name{kaneton/core/task/guest} test, the administrator could add the
following line after \code{kernel\_initialize()} and before running the
test system in \location{kaneton/core/core.c}:

\begin{verbatim}
  [...]

  module_call(console, message,
              '+', "starting the kernel\n");

  assert(kernel_initialize() == STATUS_OK);

  /* XXX[temporary] */
  test_core_task_guest();

  module_call(test, run);

  [...]
\end{verbatim}

Once the kaneton image rebuilt, the administrator can boot it locally
through \name{QEMU} and get the output, hence check that the test went
as excepted:

\begin{verbatim}
  $> qemu -fda environment/profile/user/${KANETON_USER}/${KANETON_USER}.img
\end{verbatim}

Back to the server side, the \name{evaluate} script can be used by the
administrator in order to assign grades to the snapshots submitted by the
students. The script generates a statement containing the results of this
evaluation process.

\begin{verbatim}
  $> ./evaluate.py --stage k2                                           \
                   --pattern "^epita::2010::.*$"
  the statement has been saved in '../store/statement/20101102-223645.db'
  $> 
\end{verbatim}

Finally, the \name{dump} script takes any \name{YAML}-based file and
displays its inner structure in a hierarchical manner.

\begin{verbatim}
  $> ./dump.py --path ../store/statement/20101102-223848.db
  meta:
    reference:              20.0
    stage:                  k2
  data:
    epita::2010::group7:
      date:                 2010/11/02 20:46:44
      grade:                16.0
      snapshot:             20101102-204644
      members:
        email:              admin@opaak.org
        name:               admin
      configurations:
        Xen:
          report:           20101102-224213
          notch:            4
          score:            4
        QEMU:
          report:           20101102-223848
          notch:            4
          score:            0
\end{verbatim}

%
% student
%
\subsubsection{Student}

The student has the possibility to request actions from the test server
through the client script located in \location{test/client/}.

% requirements
\subsubsubsection{Requirements}

Although the client script is integrated in the kaneton environment, it also
makes use of the \name{ktp}. Therefore, as for the server, the client depends
on a variety of \name{Python} packages including \name{yaml}, \name{pyopenssl},
\name{hmac}, \name{pickle}, \name{xmlrpc}, \name{subprocess} among others.

% use
\subsubsubsection{Use}

The client script enables the user to request one of the five operations
described below.

\begin{verbatim}
  $> make
  [!] usage: client.py [command]

  [!] commands:
  [!]   information
  [!]   submit-[stage]
  [!]   test-[environment]::[suite]
  [!]   list
  [!]   display-[identifier]
  [!]   retest-[identifier]
  $>
\end{verbatim}

The \name{information} operation requests the server to return information
on the current state of the user's profile. The information returned range
from the number of tests performed, the quota for every test suite to the
available stages or the snapshots having been previously submitted.

\begin{verbatim}
  $> make information
  [+] configuration:
  [+]   server:                 https://test.opaak.org:8421
  [+]   capability:             /data/mycure/repositories/kaneton/environment/profile/user/julien.quintard/julien.quintard.cap
  [+]   platform:               ibm-pc
  [+]   architecture:           ia32/educational

  [+] information:
  [+]   profile:
  [+]     identifier:           contributor
  [+]     community:            contributors
  [+]     members:
  [+]       name:               admin
  [+]       email:              admin@opaak.org
  [+]   suites:
  [+]                           k1
  [+]                           k3
  [+]                           k2
  [+]                           kaneton
  [+]   stages:
  [+]                           k1
  [+]                           k2
  [+]                           k3
  [+]   environments:
  [+]                           qemu
  [+]                           xen
  [+]   database:
  [+]     reports:
  [+]       xen:
  [+]         ibm-pc.ia32/educational:
  [+]           k3:
  [+]           k2:
  [+]           k1:
  [+]       qemu:
  [+]         ibm-pc.ia32/educational:
  [+]           k3:
  [+]           k2:
  [+]           k1:
  [+]     settings:
  [+]       xen:
  [+]         ibm-pc.ia32/educational:
  [+]           k3:
  [+]             requests:     0
  [+]             quota:        -1
  [+]           k2:
  [+]             requests:     0
  [+]             quota:        -1
  [+]           k1:
  [+]             requests:     0
  [+]             quota:        -1
  [+]       qemu:
  [+]         ibm-pc.ia32/educational:
  [+]           k3:
  [+]             requests:     0
  [+]             quota:        -1
  [+]           k2:
  [+]             requests:     0
  [+]             quota:        -1
  [+]           k1:
  [+]             requests:     0
  [+]             quota:        -1
  $> 
\end{verbatim}

The \name{test} command enables the user to trigger a test suite for the
current kaneton implementation on the specified environment such as \name{QEMU}
or \textit{Xen} for instance.

The server then returns the resulted report which the client stores locally
in \location{test/store/report/}. In addition, the client displays a quick
summary of the report in order for the user to know whether things went
as expected.

\begin{verbatim}
  $> make test-xen::k2
  [+] configuration:
  [+]   server:                 https://test.opaak.org:8421
  [+]   capability:             /data/mycure/repositories/kaneton/environment/profile/user/julien.quintard/julien.quintard.cap
  [+]   platform:               ibm-pc
  [+]   architecture:           ia32/educational

  [+] report(20101103:140601):
  [+]   segment                                                           [1/1]
  [+]   id                                                                [3/3]
  $> 
\end{verbatim}

The \name{list} command enables the user to display the identifiers of the
reports in the local store.

\begin{verbatim}
  $> make list
  [+] reports:
  [+]   20101103:140601:
  [+]     xen :: ibm-pc :: ia32/educational :: k2 :: 2010/11/03 14:06:01
\end{verbatim}

The \name{display} command gives the user the possibility to dump a locally
stored report in a very detailed way.

\begin{verbatim}
  $> make display-20101103:140601
  [+] report:
  [+]   meta:
  [+]     platform:               ibm-pc
  [+]     date:                   2010/11/03 14:06:01
  [+]     architecture:           ia32/educational
  [+]     duration:               63.499
  [+]     suite:                  k2
  [+]     identifier:             20101103:140601
  [+]     environments:
  [+]       stress:               xen
  [+]       construct:            xen
  [+]   data:
  [+]     segment:                                                        [1/1]
  [+]       permissions/01:
  [+]         status: True
  [+]         description: This test creates a task and address space before reserving a segment and changing its permissions.
  [+]         duration: 0.010
  [+]         output: 
  [+]     id:                                                             [3/3]
  [+]       simple:
  [+]         status: True
  [+]         description: This test reserves a single identifier.
  [+]         duration: 0.004
  [+]         output: 
  [+]       clone:
  [+]         status: True
  [+]         description: This test reserves, clones and releases identifiers.
  [+]         duration: 0.005
  [+]         output: 
  [+]       multiple:
  [+]         status: True
  [+]         description: This test reserves thousands of identifiers, checking that no collisions occured.
  [+]         duration: 0.040
  [+]         output: 
  $> 
\end{verbatim}

The \name{submit} command sends the user's snapshot to the server so as to
be evaluated for the given stage.

\begin{verbatim}
  $> make submit-k1
  [+] configuration:
  [+]   server:                 https://test.opaak.org:8421
  [+]   capability:             /data/mycure/repositories/kaneton/environment/profile/user/julien.quintard/julien.quintard.cap
  [+]   platform:               ibm-pc
  [+]   architecture:           ia32/educational

  [+] the snapshot has been submitted successfully
  $> 
\end{verbatim}

Finally, the \name{retest} command provides contributors the possibility to
re-launch the test suite of the given identified test. This command is
especially useful to re-test a snapshot should an unexpected error occur on
the test server.

Indeed since test requests are limited for students, it would be unfair for the
student to be forced to sacrifice a test slot because something went wrong
on the server-side. By requesting a contributor, the student's snapshot can
be re-tested. Once the test complete, an email is sent to the student along
with the attached report.

\begin{verbatim}
  $> make retest-20101103:140601
  [+] configuration:
  [+]   server:                 https://test.opaak.org:8421
  [+]   capability:             /data/mycure/repositories/kaneton/environment/profile/user/julien.quintard/julien.quintard.cap
  [+]   platform:               ibm-pc
  [+]   architecture:           ia32/educational

  [+] the snapshot has been re-tested successfully
  $> 
\end{verbatim}

%
% robot
%
\subsubsection{Robot}

The \name{robot} test tool enables contributors to test the kaneton research
implementation on a regular basis; hence control the status of the development.

The robot basically retrieves the kaneton implementation by checking out the
\name{Subversion} repository. Then, several test suites are triggered through
the test client. Once the reports have been received, a message is built
summarizing the results. This message is then sent to the kaneton contributors
mailing-list.

The deployment of the \name{robot} is quite straigthforward. First, the
\location{test/robot/} directory must be copied to the server. Note that
the \name{robot.py} script depends upon the \name{ktp} package which must
therefore be copied as well.

Then, the \name{SSH} configuration file \name{config} must be placed in
the \location{\${HOME}/.ssh/} directory. Besides, this file should be edited in
order to properly reference the \name{SSH} keys since the default configuration
assumes the kaneton test directory to be located at \location{/kaneton/}.

Finally, the \name{robot.cron} crontab file must be setup through the
following command in order to trigger the robot every night:

\begin{verbatim}
  $> crontab robot.cron
\end{verbatim}

Once again, the administrator should make sure to edit this file should
the robot files not be located in the default location \ie{}
\location{/kaneton/}.

%%
% ---------- header -----------------------------------------------------------
%
% project       kaneton
%
% license       kaneton
%
% file          /home/mycure/kaneton/view/book/development/cheat.tex
%
% created       julien quintard   [thu may 24 11:57:37 2007]
% updated       julien quintard   [fri jun  1 01:03:21 2007]
%

%
% ---------- cheat ------------------------------------------------------------
%

\subsection{Cheat}
\label{section:cheat}

The \textit{cheat} tool checks whether students cheated by using pieces of
code from kaneton projects of the previous years.

The \textit{history/} directory is composed of directories organizing the
kaneton students implementations over the years and for every school and
university the education project was used. Then each subdirectory represents
a year and contains subdirectories for each students group of this year.

Each student group directory contains a \textit{sources/} subdirectory
containing the tarballs of the different kaneton stages: \textit{k0},
\textit{k1}, \textit{k2} and so on; a \textit{fingerprints/} directory
containing an internal source representation used for detecting cheating,
a \textit{tests/} directory containing a summary of the testing results
for each stage and a \textit{cheats/} directory which contains a list of
commonalities with other kaneton implementations of the same and previous
years.

The \textit{cheat} tool takes a year and a stage as arguments. Its first
task is to generate the fingerprints of the other kaneton implementations
for this stage of the same and previous years.

Once the fingerprints are generated, the tool performs the checks by
comparing each pair of kaneton implementations for this stage.

The \textit{cheat} tool is based on another tool which cannot be revealed
here. For more information, please contact your supervisor.


\include{licenses}
%%
%% licence       kaneton licence
%%
%% project       kaneton
%%
%% file          /home/mycure/kaneton/view/papers/kaneton/bibliography.tex
%%
%% created       julien quintard   [mon may  8 18:35:35 2006]
%% updated       julien quintard   [mon may  8 20:38:56 2006]
%%

%
% bibliograpy
%

\chapter{Bibliography}

This chapter contains the bibliography.

%
% text
%

\begin{thebibliography}{0}
  \bibitem{AST-SCO}
    \textbf{Structured Computer Organization};
    by
    \textit{Andrew S. Tanenbaum}
  \bibitem{AST-CN}
    \textbf{Computer Networks};
    by
    \textit{Andrew S. Tanenbaum}
  \bibitem{AST-OSDI}
    \textbf{Operating Systems: Design and Implementation};
    by
    \textit{Andrew S. Tanenbaum, Albert S Woodhull}
  \bibitem{AST-MOS}
    \textbf{Modern Operating Systems};
    by
    \textit{Andrew S. Tanenbaum}
  \bibitem{AST-DOS}
    \textbf{Distributed Operating Systems};
    by
    \textit{Andrew S. Tanenbaum}
  \bibitem{AST-DSPP}
    \textbf{Distributed Systems: Principles and Paradigms};
    by
    \textit{Andrew S. Tanenbaum, Maarten van Steen}
  \bibitem{NAL-DA}
    \textbf{Distributed Algorithms};
    by
    \textit{Nancy A. Lynch}
\end{thebibliography}


\end{document}

--

virer de nomenclature le truc sur le code et le mettre dans coding-style

--

module: n importe quoi au boot

donc il faut qu on trouve un autre nom pour les binaires passifs.

--

un solution pour les verbatim foireux a la couleur ou les fonctions
qui sont decoupees sur deux pages (nom,desc), il suffit de faire une
box autour (tabular) et comme ca il ne le splitte pas
