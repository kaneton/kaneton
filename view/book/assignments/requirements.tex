%
% ---------- header -----------------------------------------------------------
%
% project       kaneton
%
% license       kaneton
%
% file          /home/mycure/kaneton/view/book/assignments/requirements.tex
%
% created       julien quintard   [fri may 23 22:13:42 2008]
% updated       julien quintard   [mon jan 31 15:48:13 2011]
%

%
% ---------- requirements -----------------------------------------------------
%

\chapter{Requirements}
\label{chapter:requirements}

This chapter discusses the requirements for students to undertake the
\name{kaneton} educational project.

\newpage

%
% ---------- text -------------------------------------------------------------
%

The requirements can be divided into two categories: \term{theoretical} and
\term{practical}. Any student lacking one of the following points should
first make sure he/she has a good knowledge of it before starting the
project.

Note however that students can learn everything along the project though we
recommend a minimum of understanding regarding the operating system principles
and programming languages.

The \name{Opaak} \name{Wiki} accessible at \location{http://kaneton.opaak.org/documentation}
contains resources and references which could be useful to the student
willing to document himself.

%
% theoretical
%

\section{Theoretical}

The following topics must be known from every student willing to undertake
the project:

\begin{itemize}
  \item
    \term{Operating Systems Principles} and especially micro-kernel
    architectures;
  \item
    \term{Computer Architectures}.
\end{itemize}

%
% practical
%

\section{Practical}

Since the goal of the project remains to develop a micro-kernel, every
student should be familiar with the programming languages and
development environments the project relies on:

\begin{itemize}
  \item
    \term{UNIX} environment or your operating system environment --- as
    long as it is supported by the kaneton development environment;
  \item
    \term{C} programming language;
  \item
    \term{Assembly} programming language.
\end{itemize}
