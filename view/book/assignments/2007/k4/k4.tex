%%
%% licence       kaneton licence
%%
%% project       kaneton
%%
%% file          view/books/assignments/2007/k4/k4.tex
%%
%% created	 pidancet julian	[tue may29 16:34:56 2006]
%% updated	 pidancet julian	[tue may29 16:34:56 2006]
%%

%
% k2
%

\chapter{K4: IPC Management}

%
% informations
%

\begin{tabular}{p{7cm}l}
Duration: & 2 weeks \\
Directory name: & kaneton/ \\
In charge: Julian Pidancet \& Matthieu Bucchianeri \& Renaud Voltz\\
Mailing-list: & kaneton-students@googlegroups.com \\
Languages: & C, assembly \\
Students per group: & 2 (same groups as for K3) \\
\end{tabular}

\section{Abstract}

K4 project consists in developing the last part of the microkernel : the
IPC (Inter-process Communication). As you know, in a microkernel, contrary
to a monolithic kernel, every kernel services are running independently
from the core, in separate userspace privileged tasks. Though, in order
to keep coherency, tasks need to interact : sending, receiving messages,
and also synchronizing each others. The IPC model chosen for Kaneton is
``message passing''

The concerned parts are:

\begin{enumerate}
  \item
    {\bf The message manager, machine-independant part}\\
    Inter-process messaging primitives. Send and receive functions,
    synchronous and asynchronous.
  \item
    {\bf The message manager, IA-32 part}\\
    Handling soft-interrupt (syscall to the messaging primitives).
\end{enumerate}

\newpage

\section{message manager, \textbf{machine-independant part}}

\begin{itemize}
  \item {\bf Assignments}\\

  In the machine-independant part of the message manager, you will have to
  implement the messaging primitives for sending and receiving messages.
  These primitives will have to be accessible by syscall, to permit
  running tasks to send and receive messages from userspace.

  \item {\bf Interface}\\

\function{message\_async\_send}{(i\_task \argument{sender},
				 i\_node \argument{dest},
				 t\_tag \argument{tag},
				 t\_vaddr \argument{data},
				 t\_size \argument{size})}
	 {

	 }


  \item {\bf {Files}}\\

    \begin{tabular}{| l | l |}
      \hline
      machine-independent & {\em kaneton/core/message/message.c}\\
      &  {\em kaneton/include/core/message.h}\\\hline
    \end{tabular}
\end{itemize}

\newpage

\section{message manager \textbf{IA-32 part}}
\begin{itemize}
  \item {\bf Assignments}\\

    The goal of this part is to permit software to send and receive
    data from userspace.\\

    Your work is splitted in two parts. First part is to write syscall
    handlers for the 4 given primitives. Second part is to write interface
    sending and receiving functions that will provide task-side syscalls.

  \item {\bf {Files}}\\

    \begin{tabular}{| l | l |}
      \hline
      machine-dependent & {\em kaneton/core/arch/ibm-pc.ia32-virtual/message.c}\\
      & {\em kaneton/include/arch/ibm-pc.ia32-virtual/core/message.h}\\\hline
    \end{tabular}

\end{itemize}

%
% advanced topics
%

\newpage

\section{Appendix}

\textbf{Example of event\_reserve}

\begin{verbatim}
void          ia32_pf_handler(t_id         id,
                              t_uint32     error_code)
{
  t_uint32    addr;

  SCR2(addr);
  printf("#PF @ %p\n", addr);

  while (1)
    ;
}

...

event_reserve(14, EVENT_FUNCTION, EVENT_HANDLER(ia32_pf_handler));
\end{verbatim}

\textbf{Example of timer\_reserve}

\begin{verbatim}
void          sched_switch(void)
{
  // FIXME: chiche just stole this code
}

...

timer_reserve(EVENT_FUNCTION, TIMER_HANDLER(sched_switch),
              sched->quantum, TIMER_REPEAT_ENABLE,
              &sched->machdep.timer);
\end{verbatim}
