%%
%% licence       kaneton licence
%%
%% project       kaneton
%%
%% file          /home/mycure/kaneton/view/papers/assignments/k3.tex
%%
%% created       matthieu bucchianeri   [fri feb 10 13:52:29 2006]
%% updated       julien quintard   [sat jun 17 12:54:51 2006]
%%

%
% k3
%

\chapter{k3}

The \textbf{k3} project consists in the development of everything in
relation with the interrupts.

This part could be called the \textit{funny} part.

Indeed, one of the part of the project is to develop some drivers to
make the kaneton microkernel more attractive. Moreover, these developments
could be very helpful for the future developments since the students will
then be able to write an interactive kind of debugger.

Nevertheless, another part of the project is to design and implement
the event manager. This part is by far less funny than the first one
but will lead the students to the reflection in terms of genericity,
portability etc.

\newpage

%
% informations
%

\section{Informations}

\begin{tabular}{p{7cm}l}
Duration: & Two weeks \\
File name: & \textit{[group]}-k3.tar.gz \\
In charge: & Julien Quintard \\
Newgroup: & kaneton-students@googlegroups.com \\
Languages: & Assembly and C \\
Students per group: & Three \\
Target directories:
  & \textit{kaneton/core/event/} \\
  & \textit{kaneton/core/debug/} \\
\end{tabular}

%
% assignments
%

\section{Assignments}

Since one of the assignments in to develop some drivers, first,
the student will have to provide anything necessary to write drivers.

Indeed, devices trigger what are called \textit{IRQ}
\textit{Interrupt Request} which then become interrupts. While the term
used can be slightly different, it always means a device alerting the
microprocessor of an event: a transfer just finished, a key was pressed
etc.

Then, since the interrupts are highly dependent of the microprocessor
architecture, the kaneton core needs a manager to handle a common
concept but from a more abstract view: the event manager.

The student, in this assignment will need to design the event manager's
interface. Once designed, the student will also have to implement it.


--

\subsection{event manager}

The students have to design the event manager. This assignment is very
special compared to the previous ones.

The goal is to design the event manager so it can provide the
interface the end user will require.

An event is an abstraction over the hardware and software interrupts.
An event occurs each time an hardware interrupt is triggered and each
time a program makes a system call.

In this assignments the students must think about the implication of
their design on the whole project. A short document will have to be
provided explaining the choices made, the problems encountered etc.

Finally, the students also have to implement the event manager designed.

\subsubsection{IA-32 implementation}

The students will have to implement the IDT while managing the PIC.

For more information, take a look at the Intel books and Google still
is your best friend.

\subsection{timer manager}

As for the event manager, the students must design and implement the timer
manager.

Nevertheless, this manager is simpler to design since it relies on the
event manager.

\subsubsection{IA-32 implementation}

The students should start to handle some hardware interrupt to be able
to create timers.

\subsection{keyboard driver}

In k3, the students will start to enjoy kaneton :) if it is not the
case yet.

Indeed, the students have to write a kernel-space keyboard driver.
Then it will be possible to read keys from the keyboard.

Of course this part is very specific to the IBM PC on which the students
are implementing kaneton.

\subsection{kernel mini-shell}

As for the keyboard driver, the students will develop a mini kernel-space
shell to allow the stupid GL user to press keys and see what? ... squares
and circles ...

Enjoy the corde!

--- XXX

k3: event manager a modeliser et implanter

    partie fun: les drivers console, shell, keyboard, timer
