%
% ---------- header -----------------------------------------------------------
%
% project       kaneton
%
% license       kaneton
%
% file          /home/mycure/kaneton/view/book/assignments/k3.tex
%
% created       julien quintard   [fri nov 28 05:25:37 2008]
% updated       julien quintard   [mon jan 31 14:35:22 2011]
%

%
% ---------- k3 ---------------------------------------------------------------
%

\chapter{k3}
\label{chapter:k3}

\name{k3} is the final stage and consists for students to experiment on top
of their freshly developed microkernel.

\newpage

%
% ---------- text -------------------------------------------------------------
%

%
% objectives
%

\section{Objectives}

Throughout this project, students will explore the multiple ways of
providing a major operating system feature, being a file system, a
network service or any other feature that has been granted approval by
the teaching board.

Students will have to study the pros and cons of every technique in order to
choose the right algorithms to use for their implementation.

Therefore, in this project, students are free to design and implement whatever
they wish as long as the required feature is eventually provided.

%
% requirements
%

\section{Requirements}

This stage requires a working microkernel.

%
% snapshot
%

\section{Snapshot}

A demonstration application is provided. Note however that this application
contains the bare minimum \ie{} a \textit{CRT - C Runtime} along with a
dummy function.

Finally, a system library is provided for issuing system call in
an easy way.

%
% assignments
%

\section{Assignments}

% design

\subsection*{Design}

Students are asked to study the different algorithms, data structures,
techniques in order to provide the chosen feature.

% implementation

\subsection*{Implementation}

Students are then asked to implement the service of their choice
along with any extra component required.

Finally, students should provide user-level applications for illustration
purposes.

%
% submission
%

\section{Submission}

Every student must submit its kaneton implementation through the Intranet
before the viva.

%
% evaluation
%

\section{Evaluation}

Students will have a 10-minute viva to show the jury what they have
accomplished, what features that have provided, what problems they
have encountered and how they solved them etc.

The viva will take place in a classroom. Therefore students are
asked to bring whatever they need to present and demonstrate
their work including laptops, slides etc.
