%
% ---------- header -----------------------------------------------------------
%
% project       kaneton
%
% license       kaneton
%
% file          /home/mycure/kane...view/book/assignments/future/support.tex
%
% created       julien quintard   [fri nov 28 04:29:40 2008]
% updated       julien quintard   [sat jan 17 13:35:54 2009]
%

%
% ---------- support ----------------------------------------------------------
%

\chapter{Support}
\label{chapter:support}

This chapter lists the ways students can ask for help.

\newpage

%
% ---------- text -------------------------------------------------------------
%

Programming a complete micro-kernel can be very hard for your nerves, and
it is sometimes just as easy to ask someone for advise, opinion, tips \etc{}

The \name{kaneton} people therefore decided to provide students a number of
tools for students to seek information.

%
% mailing-list
%

\section{Mailing-List}

The students mailing-list can be used by students for both communicating with
other students and \name{kaneton} people.

Note that rules must be followed when it comes to using this communication
medium. Indeed, people should be careful regarding the topics addressed
on this mailing-list especially when it comes to giving too much details
related to a specific implementation problem.

Since other students are also available on this mailing-list, you should not
hesitate to ask for help as they probably ran into the same problem at the
time they were working on this part.

Do not hesitate to ask questions as kaneton teachers are always happy
to help people interested in computer systems.

Anyway, it would be good to drop a message in order to inform kaneton
people that you are undertaking the educational project ... and saying
Hello :).

The mailing-list can be accessed at the email address:

\begin{center}
  \location{students@kaneton.opaak.org}
\end{center}

%
% wiki
%

\section{Wiki}

The \name{Wiki} enables students to access information useful to the
\name{kaneton} educational project. Note that students should consider it
at their information space and are obviously welcome to add and modify the
pages.

The \name{kaneton} \name{Wiki} can be accessed at the following address:

\begin{center}
  \location{http://wiki.opaak.org}
\end{center}
