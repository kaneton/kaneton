%
% ---------- header -----------------------------------------------------------
%
% project       kaneton
%
% license       kaneton
%
% file          /home/mycure/kaneton/view/book/assignments/k0.tex
%
% created       julien quintard   [fri nov 28 05:25:37 2008]
% updated       julien quintard   [wed feb 16 11:46:20 2011]
%

%
% ---------- k0 ---------------------------------------------------------------
%

\chapter{k0}
\label{chapter:k0}

\name{k0} is an introduction to low-level programming through the development
of a simplistic boot loader. Note that this stage is not critical because
the following stages will not rely on this boot loader as \name{GRUB - GRand
Unified Bootloader} will be used instead.

Through this stage, the student will manipulate several concepts including:

\begin{enumerate}
  \item
    \term{Computer Boot Process}

    \-

    The several phases a computer runs through in order to check the
    machine's state, to the location of the bootable devices until the
    execution is transferred to the first valid boot sector found.
  \item
    \term{Assembly Language}

    The boot loader will be entirely developed in the Intel \name{x86}
    assembly language.
  \item
    \term{\name{BIOS - Basic Input/Output System} Interrupts}

    The \name{BIOS} provides basic functionalities for reading the floppy,
    displaying text to the screen \etc{} These functionalities will be used
    in the first exercices to get used to the low-level development
    environment.
  \item
    \term{Processor Modes}

    The \name{Intel} microprocessor can run in different modes depending on
    the word length---16 or 32 bits---, the protection capabilities \etc{}

    \-

    While the simplest mode will be used at first, enabling students to
    use the \name{BIOS}' features, a more evolved mode will finally be used
    as required before handing the execution to the kernel \ie{} the
    \name{Protected Mode}.
  \item
    \term{\name{ELF - Executable and Linking Format}}

    The boot loader, written in assembly, is packed in a raw format so that
    to be detected as a valid boot loader by the computer's \name{BIOS}.

    \-

    However, the kernel, probably written in \name{C} or \name{C++}, will
    be packed in an advanced format referred to as \name{ELF}. Although this
    format has numerous capabilities, the boot loader will basically
    find the entry point and jump on it.
\end{enumerate}

\newpage

%
% ---------- text -------------------------------------------------------------
%

%
% development
%

\section{Development}

Students are advised to use \name{GCC - GNU Compiler Collection} to assemble
their source files and \name{GNU LD} to link them into binary files. Note
that students must therefore follow the \name{GAS -GNU Assembler} syntax.

The following example illustrates this two-step process:

\begin{verbatim}
  $> gcc -c bootsect.S -o bootsect.o
  $> ld bootsect.o -o bootsect ---oformat binary ---Ttext 0x7c00
\end{verbatim}

In order to execute your code, students should use the \name{QEMU} emulator,
as follows:

\begin{verbatim}
  $> qemu -fda bootsect
\end{verbatim}

Let us recall that \name{x86} microprocessors start exeucting in 16-bit. A
directive must therefore be used to specify the assembler that the code
must be generated in 16-bit. Note that students will later be asked to
switch to a 32-bit mode referred to as the \name{Protected Mode}. As such,
another directive will have to be used, hence mixing both 16- and 32-bit
code.

%
% resources
%

\section{Resources}

Students will find additional resources related to the \name{k0} stage
on the \name{Wiki} at \location{http://wiki.opaak.org}.

%
% test
%

\section{Test}

Students will need to test the functions they implement. In order to keep
the test environment from the exercise files, it is advise to rely on
the \name{CPP - C Pre-Processor} for including the exercises' functions
within the test file.

The following example illustrates a test file which includes the first
exercise's file:

\begin{verbatim}
.globl _start

_start:
  mov 0x42, %al
  call print_char

loop:
  jmp loop

#include "ex1.S"
\end{verbatim}

This way, the test code is always kept apart from the exercises so that
the risk for students to submit their tests is limited.

%
% submission
%

\section{Submission}

Every student is expected to submit a snapshot containing the following
structure. Note that the evaluation system is automatic. Therefore, any
snapshot which does not conform with this structure will obviously be assigned
the grade of zero:

\begin{verbatim}
  ex1/
    ex1.S
  ex2/
    ex2.S
  ex3/
    ex3.S
  ex4/
    ex4.S
  ex5/
    ex5.S
  ex6/
    ex6.S
\end{verbatim}

The submitted snapshot should therefore contain nothing more than these
six directories and files. Please be careful with the extension of the source
files; the 'S' is uppercase!

The \name{k0} stage is composed of several exercises. For ever exercise,
the student is asked to write functions in assembly language taking arguments
in some registers and providing output in others. Note that these functions
should take care not to modify any register unless explicitly stated.

Every exercise is attached to a specific file, \code{ex2.S} for instance.
Such a file must therefore contain the labels of the functions that the student
is supposed to provide.

In addition, every file should end with the bootsector signature, making the
functions of the file useable within a boot loader. This signature, composed
of the bytes \code{0xaa55} must be located at the $511$th and $512$th byte
of the first boot sector. Thus, the boot sector must be filled up until the
$511$th byte is reached, at which position the signature must be put.

The example below illustrates such an exercise file, containing two
labels and the bootsector signature:

\begin{verbatim}
  .code16

  requested_function_1:
    mov $0x42, %ah
    mov %bl, %bl

  local_label:
    int $0x10
    xor %dh, %dh
    jmp local_label
    ret

signature:
  .org 510, 0
  .short 0xaa55
\end{verbatim}

Some exercises will require students to output text on the screen. Bear in
mind that exercises are evaluated automatically. Therefore, color is not
taken into account. Besides, students should never try to clear the screen
of the text printed by the \name{BIOS} or the emulator. By doing so, the
evaluation system would catch the sequence and consider it as text being
printed by your functions.

Finally, as explained previously, students will need to test their functions
in order to make sure they behave properly. Students must remember to remove
these tests from the submitted snapshot. Thus, your source files should never
include a label \code{\_start} or any label starting with \code{\_\_}. Note
however that the bootsector signatures must remain.

Note that this stage differs slightly from the following ones regarding
the submission. Indeed, in the following stages, which take place within the
microkernel, the test client automatically packages the student's current
implementation and sends it to the server.

For this stage however, the student is asked to package its source files
manually, according to the hierarchical structure given above. Then,
the freshly created snapshot must be placed in the client directory
under the name \code{snapshot.tar.bz2} \ie{} at the precise location:
\location{test/client/snapshot.tar.bz2}.

Finally, the test client can be used to submit the stage \name{k0}. The
client, noticing the presence of the snapshot, will use it instead of
generating a snapshot from the current kaneton implementation.

The lines below illustrates this process.

\begin{verbatim}
  $k0> ls
  ex1/ ex2/ ex3/ ex4/ ex5/ ex6/
  $k0> tar cjvf k0.tar.bz2 *
  [...]
  $k0> mv k0.tar.bz2 ~/kaneton/test/client/snapshot.tar.bz2
  $k0> cd ~/kaneton/test/client/
  $~/kaneton/test/client> ls
  client.py Makefile README snapshot.tar.bz2
  $~/kaneton/test/client> make submit-k0
  [+] configuration:
  [+]   server:                 https://test.opaak.org:8421
  [+]   capability:             /home/chiche/kaneton/environment/profile/user/florent.chichery/florent.chichery.cap
  [+]   platform:               ibm-pc
  [+]   architecture:           ia32/educational

  [+] loading the snapshot '/home/chiche/kaneton/test/client/snapshot.tar.bz2'
  [+] requesting the server
  [+] the snapshot has been submitted successfully
  $~/kaneton/test/client>
\end{verbatim}

%
% evaluation
%

\section{Evaluation}

The student's code will be evaluated by testing every function automatically.

The evaluation process will therefore go as follows. First, the student
exercise file will be linked with an evaluation file containing the entry
point. This main function will then trigger some function calls and verify
that these functions behave according to the specifications define in this
document.

Let us recall that the whole binary---composed of the student exercise file
and the evaluation function---must fit in a bootsector \ie{} $512$ bytes.

Therefore, for every exercise, the size of the evaluation code is given
as an indication. As such, students should make sure the size of their code
does not exceed the remaining capacity of a bootsector.

In order to make sure a student's exercise does not exceed this capacity,
the following directive can be used:

\begin{verbatim}
  .space 42,0

  #include "ex1.S"
\end{verbatim}

Note that the value $42$ will have to be replaced by the size of the
evaluation code given in each exercise.

If the assembler refuses to generate a binary given a test file containing
this directive, this would mean that the exercise code takes too much space,
in which case the student will have to optimise it. The error the assembler
returns in such scenarios is similar to the one below:

\begin{verbatim}
  ex1.S: Assembler messages:
  ex1.S:12: Error: attempt to move .org backwards
\end{verbatim}

%
% exercises
%

\section{Exercises}

This section discusses the various exercises contained in this stage.

% console
\subsection{Console}

Through this exercise the student will learn how to use \name{BIOS} services
in order to print strings to the console.

\begin{center}
  \begin{tabular}{|p{5cm}|p{5cm}|l}
    \cline{1-2}

    \centering{\textbf{File}} &
    \centering{\textbf{Space}} &
    \\

    \cline{1-2}

    \centering{\location{ex1/ex1.S}} &
    \centering{$175$ bytes} &
    \\

    \cline{1-2}
  \end{tabular}
\end{center}

This exercise is divided into several steps through the implementation
of the following functions.

\command{print\_char}
{
  This function prints the character whose \name{ASCII} code is given in
  \argument{\%al}.

  \-

  Note that the character is displayed at current cursor's position.
}

\command{cursor\_set}
{
  This function set the cursor's position at row \argument{\%dh} and
  column \argument{\%dl}.
}

\command{print\_string}
{
  Finally, this function prints the string referred to by \argument{\%si} at
  the position given by row \argument{\%dh} and column \argument{\%dl}.
}

% registers
\subsection{Registers}

In this exercise, students are invited to implement several functions
well-known from \name{POSIX} programmers. These functions will finally be
used to provide a function used to dump the state of the $16$-bit registers.

\begin{center}
  \begin{tabular}{|p{5cm}|p{5cm}|l}
    \cline{1-2}

    \centering{\textbf{File}} &
    \centering{\textbf{Space}} &
    \\

    \cline{1-2}

    \centering{\location{ex2/ex2.S}} &
    \centering{$155$ bytes} &
    \\

    \cline{1-2}
  \end{tabular}
\end{center}

\command{malloc}
{
  This function allocates \argument{\%ax} bytes of memory and returns the
  address in \argument{\%di}.

  \-

  First, the base address of the head should be statically declare. Then,
  whenever the function is called, the next available address is used and
  incremented.
}

\command{itoa\_hex}
{
  This function converts the integer in \argument{\%ax} into a string
  using the hexadecimal representation, the address of the string being
  returned in \argument{\%si}

  \-

  Note that the hexadecimal format must comply with the regular expression
  \code{0x[0-9a-f]\{4\}}. For example, the number $42$ should be converted
  to the string \code{'0x002a'}.

  \-

  This function could for instance be used by students to check the address
  returned by the \code{malloc()} function.
}

\command{dump\_registers}
{
  This function dumps the values of the registers \name{ax}, \name{bx},
  \name{cx}, \name{dx}, \name{si} and \name{di}.

  \-

  This function must start displaying at the current cursor's position.
  Besides, the evaluation system expects the function to come back to the
  first line should the end of the screen be reached.

  \-

  In essence, the output should be exactly identical to the one below:
}
  \begin{verbatim}
    ax = 0x1234
    bx = 0x0000
    cx = 0xabcd
    dx = 0x00ff
    si = 0x1000
    di = 0x0ff8
  \end{verbatim}

% keyboard
\subsection{Keyboard}

In this exercise, the aim is to implement a basic keyboard driver by
relying on the \name{BIOS} services.

\begin{center}
  \begin{tabular}{|p{5cm}|p{5cm}|l}
    \cline{1-2}

    \centering{\textbf{File}} &
    \centering{\textbf{Space}} &
    \\

    \cline{1-2}

    \centering{\location{ex3/ex3.S}} &
    \centering{$155$ bytes} &
    \\

    \cline{1-2}
  \end{tabular}
\end{center}

\command{get\_key}
{
  This function returns in \argument{\%al} the \name{ASCII} code of the
  key which has been pressed.
}

\command{getln}
{
  This function reads and prints every character received from the keyboard.
  When \name{ENTER} is entered however, a \name{newline} is printed and
  the function returns.

  \-

  Note that none of the modifiers such as \name{SHIFT}, \name{ALT},
  \name{CTRL}, \name{CAPS LOCK} \etc{} will be tested. Likewise, the arrows,
  \name{BACKSPACE} \etc{} will also be ignored.

  \-

  Finally, the end of line does not have to be handled as it will not
  be tested.
}

% floppy
\subsection{Floppy}

This interrupts introduced other \name{BIOS} interrupts enabling the $16-bit$
programmer to control the floopy drive. The objective of this exercise is
to read the second sector from the floppy drive in order to checker
whether it is a valid bootsector.

Note that the second sector is considered because the first has already
been loaded into memory. This technique is used by boot loaders, such as
\name{GRUB}, to provide \name{chain loading}.

\begin{center}
  \begin{tabular}{|p{5cm}|p{5cm}|l}
    \cline{1-2}

    \centering{\textbf{File}} &
    \centering{\textbf{Space}} &
    \\

    \cline{1-2}

    \centering{\location{ex4/ex4.S}} &
    \centering{$155$ bytes} &
    \\

    \cline{1-2}
  \end{tabular}
\end{center}

\command{floppy\_init}
{
  This function initializes the floppy drive.
}

\command{is\_bootsector}
{
  This function calls the initialization of the floppy drive, loads the
  second sector into memory and checks that it is a valid bootsector.

  \-

  Noteworthy is that sectors are numbered starting with 1, not 0.

  \-

  Once the sector checked, the function prints a string, along with the
  read boot signature. Note that the function must prints at the current
  cursor's position.

  \-

  The following illustrates both cases, respectively a valid and invalid
  bootsector.
}
\begin{verbatim}
  magic found: 0xaa55
  wrong magic: 0x0042
\end{verbatim}

% mode
\subsection{Mode}

Until know, the code students have been developing were written in $16$-bit,
hence evolving in \name{Real Mode}. This mode is a relique of the past. As
most kernels are written in $32$-bit, the boot loader must prepare the
environment so as to fit the needs of the kernel.

Through this exercise, the students are going to switch from the \name{Real
Mode} to the \name{Protected Mode}. This process requires the developer to
set up some hardware data structures indicating the microprocessor how it
must behave. The data structure of interest if the \name{GDT - Global
Descriptor Table}.

\begin{center}
  \begin{tabular}{|p{5cm}|p{5cm}|l}
    \cline{1-2}

    \centering{\textbf{File}} &
    \centering{\textbf{Space}} &
    \\

    \cline{1-2}

    \centering{\location{ex5/ex5.S}} &
    \centering{$225$ bytes} &
    \\

    \cline{1-2}
  \end{tabular}
\end{center}

Students should read the necessary material, especially the \name{Intel}
manual \name{Volume 2A} which contains a chapter on the \name{Protected Mode}.

The following describes the steps to follow for performing a switch to the
\name{Protected Mode}:

\begin{enumerate}
  \item
    Create and fill the \name{GDT} with the necessary entries;
  \item
    Disable the interrupts through the \code{cli} instruction which masks
    the hardware maskable interrupts;
  \item
    Set the \name{PE} flag in the control register \name{CR0};
  \item
    Immediately perform a long jump in order to change the \name{CS - Code
    Segment} selector, referencing the appropriate \name{GDT} entry;
  \item
    Update the other segment selectors \name{DS}, \name{ES}, \name{FS},
    \name{GS} and \name{SS}.
\end{enumerate}

Do not forget that the code executed after switching from the \name{Real Mode}
to the \name{Protected Mode} must be assembled in 32-bit.

In addition, \name{BIOS} interrupts cannot be used in \name{Protected Mode}.
Therefore, the basic functionalities must be re-implemented in $32$-bit and
without the support of the \name{BIOS} services.

\command{pmode\_switch}
{
  This function switches from the \name{Real Mode} to the \name{Protected
  Mode}.

  \-

  Note that when calling this function, the \name{CPU} will push the return
  address as a $16$-bit value, since operating in \name{Real Mode}. However,
  when returning, the \name{CPU} will pop a $32$-bit value since now operating
  in \name{Protected Mode}.

  \-

  The student should not worry about this detail when implementing this
  function. Indeed, it is the responsability of the caller to push another
  $16$-bit zero on the stack before calling so that when pop-ed in $32$-bit
  both values coincide.
}

\command{print\_string32}
{
  This function prints the string whose address is given in \argument{\%esi}.

  \-

  The location is specified through the row by \argument{\%ecx} and the
  column by \argument{\%edx}. Note that the row and column are starting at
  index 0. As such, the upper left corner has coordinates $(0,0)$.

  \-

  Finally, the student should ignore the end of line as this will not be
  tested. Besides, colors can be changed but will not be considered by
  the evaluation system. Moreover, the special characters such as \code{'\r},
  \code{'\t}, \code{'\n} \etc{} will not be tested either.
}

% elf
\subsection{ELF}

The finally exercise of the \name{k0} stage consists in writing the core
functionality of any boot loader: loading the kernel binary from a device
into memory and executing it.

Since most kernels are written in high-level languages such as \name{C} or
\name{C++}, such binary are actually packaged according to a flexible format,
the most common one of \name{UNIX} systems being the \name{ELF - Executable
and Linking Format}.

\begin{center}
  \begin{tabular}{|p{5cm}|p{5cm}|l}
    \cline{1-2}

    \centering{\textbf{File}} &
    \centering{\textbf{Space}} &
    \\

    \cline{1-2}

    \centering{\location{ex6/ex6.S}} &
    \centering{$145$ bytes} &
    \\

    \cline{1-2}
  \end{tabular}
\end{center}

Note that an \name{ELF} example binary is provided on the \name{Wiki}. As
mentioned earlier, boot loaders are usually located on the first sector
of the boot image being on a floppy or hard drive. Then follows, on one or
more sectors the kernel, the role of the boot loader being to load the kernel
sectors and execute it.

The following command shows how to create a bootable image given a boot loader
binary and an \name{ELF} kernel:

\begin{verbatim}
  $> cat bootsector kernel.elf /dev/zero | dd bs=512 count=2880 of=disk.img
\end{verbatim}

The \name{ELF} is a complex format composed of headers, segments and tables
referencing them. The kernel binary will therefore have to be parsed in order
to extract the code and its entry point on which the boot loader will have
to jump to transfer the execution flow to the kernel.

An overview of the \name{ELF} is given on the \name{Wiki} though the student
may want to read a more detailed documentation.

\command{kernel\_preload}
{
  This function loads into memory the number of sectors given into
  \argument{\%al}, from the drive specified in \argument{\%dl}, starting from
  the second sector.

  \-

  In addition, the function must switch to the \name{Protected Mode} and
  return into \argument{\%eax} the address of the memory area where sectors
  have been loaded \ie{} the location of the \name{ELF} in memory.

  \-

  Noteworthy is that this function must succeed even if the kernel image
  spans over multiple sectors.
}

\command{elf\_load}
{
  This function executes the \name{ELF} located at the address given in
  \argument{\%esi}, assuming that the \name{kernel\_preload()} function has
  been called prior to this function.
}
