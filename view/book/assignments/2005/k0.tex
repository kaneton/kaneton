%%
%% licence       kaneton licence
%%
%% project       kaneton
%%
%% file          /home/mycure/kaneton/view/papers/assignments/k0.tex
%%
%% created       matthieu bucchianeri   [tue feb  7 11:49:23 2006]
%% updated       julien quintard   [sat jun 17 00:05:22 2006]
%%

%
% k0
%

\chapter{k0}

The \textbf{k0} project consists in the development of the bootstrap.

This project is very specific and will not be reused in the future
projects.

Indeed, in the next steps, we will use a multibootloader like \textit{grub}
or \textit{lilo} as the bootstrap.

The main goal of the bootstrap is to launch the bootloader in a better
execution environment, with a better memory addressing model etc..

Note that this project is recommended to better understand the
microprocessor's architecture very first memory model.

Nevertheless, this project may get needless on some architectures.

\newpage

%
% informations
%

\section{Informations}

\begin{tabular}{p{7cm}l}
Duration: & One week \\
File name: & \textit{[group]}-k0.tar.gz \\
In charge: & Julien Quintard \\
Newsgroup: & kaneton-students@googlegroups.com \\
Languages: & Assembly and C \\
Students per group: & Three \\
Target directories: & \textit{kaneton/bootstrap/arch/[architecture]/} \\
\end{tabular}

%
% assignments
%

\section{Assignments}

The only  goal of this project is to develop a bootstrap in assembly language,
loading an ELF binary object from the floppy drive into main memory and
finally jumping onto the binary entry point.

Generally, and on many architectures, the bootstrap also installs
a more evolved memory addressing model.

The project's diffculty resides in the development of an
application first evolving in a basic addressing model, then in
a more evolved one.

Another difficulty is to deal with microprocessor's crashes since
nothing is provided to handle microprocessor's crashes yet.

%
% ia32
%

\section{Intel Architecture 32-bit}

First of all, the students will certainly want to use the BIOS interrupts
to perform the most complex tasks like loading the ELF binary object
from the floppy drive into main memory.

The only alternative to this is to develop your own floppy device driver.

Then the students will have to install and activate the flat protected mode.
Indeed, on the Intel architectures, the microprocessor first runs in
the real addressing mode.

At this point, the microprocessor is evolving into its 32-bit addressing
memory model.

Finally, the binary object just needs to be launched.
