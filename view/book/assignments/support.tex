%
% ---------- header -----------------------------------------------------------
%
% project       kaneton
%
% license       kaneton
%
% file          /home/mycure/kaneton/view/book/assignments/support.tex
%
% created       julien quintard   [fri nov 28 04:29:40 2008]
% updated       julien quintard   [tue feb  8 17:17:57 2011]
%

%
% ---------- support ----------------------------------------------------------
%

\chapter{Support}
\label{chapter:support}

This chapter discusses the means for students to get help.

\newpage

%
% ---------- text -------------------------------------------------------------
%

Programming a complete micro-kernel can be very hard for your nerves, and
it is sometimes just as easy to ask someone for advise, opinion, tips \etc{}

The \name{kaneton} people therefore decided to provide students a number of
tools for students to seek information.

%
% mailing-list
%

\section{Mailing-List}

The mailing-list can be used by students for both communicating with
other students and \name{kaneton} people.

Note that rules must be followed when it comes to using this communication
medium. Indeed, people should be careful regarding the topics addressed
on this mailing-list especially when it comes to giving too much details
related to a specific implementation problem.

Since other students are also available on this mailing-list, you should not
hesitate to ask for help as they probably ran into the same problem at the
time they were working on this part. If not, the kaneton teachers will always
try to answer your questions as best as they can.

The mailing-list can be accessed at the following email address:

\begin{center}
  \location{kaneton@googlegroups.com}
\end{center}

Subscribe by sending an email to:

\begin{center}
  \location{kaneton+subscribe@googlegroups.com}
\end{center}

%
% wiki
%

\section{Wiki}

The \name{Wiki} enables students to access information useful to the
\name{kaneton} educational project. Note that students should consider it
at their information space and are obviously welcome to suggest content
which should be added or even modified.

The \name{kaneton} \name{Wiki} can be accessed at the following address:

\begin{center}
  \location{http://kaneton.opaak.org/wiki}
\end{center}
