%
% ---------- header -----------------------------------------------------------
%
% project       kaneton
%
% license       kaneton
%
% file          /home/mycure/kane...book/assignments/future/introduction.tex
%
% created       julien quintard   [fri may 23 21:47:38 2008]
% updated       julien quintard   [thu jan 29 12:11:51 2009]
%

%
% ---------- introduction -----------------------------------------------------
%

\chapter{Introduction}
\label{chapter:introduction}

This chapter briefly introduces the purpose of this documentation
and the assignments in general

\newpage

%
% ---------- text -------------------------------------------------------------
%

The \name{kaneton} educational project enables students to develop their own
micro-kernel as a way of understanding operating systems internals.

As anyone can imagine, such a project takes a huge amount of time and
motivation. While the motivation will anyway play an important role in the
success of students' project, the time spent can be greatly reduced if
students focus on implementing some specific parts rather than developing a
complete micro-kernel from scratch.

Indeed, as we will see later in this document, the current \name{kaneton}
educational project comes with a student \term{snapshot} which contains
a complete development environment as well as the source code skeleton of
the kernel.

Some people would probably prefer working on their own micro-kernel design
and implementation, starting from scratch. All we can wish to such people
is enough motivation to keep working on their project long enough to be
satisfied, luck and hard work.

Either way, going through the \name{kaneton} micro-kernel documentation
should not be a waste of time. Especially, people interested in developing
their own project from scratch could take a look at the \name{kaneton} design
in case they like it enough to implement it their way.

The remaining of this document is organised as follows. \reference{Chapter
\ref{chapter:requirements}} lists what students willing to undertake the
project should know beforehand. \reference{Chapter \ref{chapter:support}}
details the multiple ways for students to get help. \reference{Chapter
\ref{chapter:k0}} presents the first project stage. Then, \reference{Chapter
\ref{chapter:snapshot}} breifly presents the student snapshot.
\reference{Chapter \ref{chapter:k1}}, \reference{Chapter \ref{chapter:k2}},
\reference{Chapter \ref{chapter:k3}} and \reference{Chapter
\ref{chapter:k4}} details the assignments of the different stages. Finally,
\reference{Chapter \ref{chapter:further}} discusses what students could do
after having undertaken such a project.
