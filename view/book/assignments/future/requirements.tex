%
% ---------- header -----------------------------------------------------------
%
% project       kaneton
%
% license       kaneton
%
% file          /home/mycure/kane...book/assignments/future/requirements.tex
%
% created       julien quintard   [fri may 23 22:13:42 2008]
% updated       julien quintard   [sat may 24 14:22:00 2008]
%

%
% ---------- requirements -----------------------------------------------------
%

\chapter{Requirements}
\label{chapter:requirements}

This chapter discusses the requirements for students to undertake the
\name{kaneton} educational project.

\newpage

%
% ---------- text -------------------------------------------------------------
%

The requirements can be divided into two categories: \term{theoretical} and
\term{practical}. Any student lacking one of the following points should
first make sure he/she has a good knowledge of it before starting the
project.

%
% theory
%

\section{Theory}

The following knowledge is assumed from every student willing to
undertake the project.

   * Operating systems, kernels and especially micro-kernel architecture;
   * Microprocessor architecture.


Technical Knowledge

Since the goal of the project remains to develop a micro-kernel, every
student should be familiar with the programming languages and
development environments the project uses.

   * UNIX environment or your operating system environment --- as
long as it is supported by the kaneton development environment;
   * C programming language;
   * Assembly programming language.
