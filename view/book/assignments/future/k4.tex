%
% ---------- header -----------------------------------------------------------
%
% project       kaneton
%
% license       kaneton
%
% file          /home/mycure/kaneton/view/book/assignments/future/k4.tex
%
% created       julien quintard   [fri nov 28 05:25:37 2008]
% updated       julien quintard   [wed dec  3 15:25:35 2008]
%

%
% ---------- k4 ---------------------------------------------------------------
%

\chapter{k4}
\label{chapter:k4}

\name{k4} gives students the possibility to experiment through the development
of a file system.

\newpage

%
% ---------- text -------------------------------------------------------------
%

%
% objectives
%

\section{Objectives}

Through this project, students are going to explore the multiple ways of
providing the file system abstraction. Students will have to study the pros
and cons of every technique in order to choose the right algorithms to use in
their implementation.

Therefore, in this project, students are free to design and implement whatever
they wish as long as it provides the users the ability to store files.

%
% requirements
%

\section{Requirements}

This stage requires a working storage device driver such as an \name{ATA -
Advanced Technology Attachment} driver.

The file system will rely on this driver for storing blocks of data.

%
% snapshot
%

\section{Snapshot}

For this project, nothing is provided to the students but the freedom to
innovate.

%
% assignments
%

\section{Assignments}

% design

\subsection*{Design}

Students are asked to study the different data structures and algorithms
that could be used for providing the well-known file abstraction so that
a file system design is sketched.

% implementation

\subsection*{Implementation}

Students are then asked to implement the file system they designed.

In order to illustrate the functionalities of the file system, students
should also provide user-level applications.
