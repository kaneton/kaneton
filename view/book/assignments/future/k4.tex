%
% ---------- header -----------------------------------------------------------
%
% project       kaneton
%
% license       kaneton
%
% file          /home/mycure/kaneton/view/book/assignments/future/k4.tex
%
% created       julien quintard   [fri nov 28 05:25:37 2008]
% updated       julien quintard   [wed dec  3 14:33:19 2008]
%

%
% ---------- k4 ---------------------------------------------------------------
%

\chapter{k4}
\label{chapter:k4}

\name{k4} consists for students to experiment through the development of
a file system.

\newpage

%
% ---------- text -------------------------------------------------------------
%

%
% objectives
%

\section{Objectives}

This project will lead the students to explore the multiple ways of providing
the file system abstraction. Students will have to study the pros and cons of
every technique in order to choose the right algorithms to use in their
implementation.

%
% requirements
%

\section{Requirements}

This stage requires a working storage device driver such as an \name{ATA -
Advanced Technology Attachment} driver.

The file system will rely on this driver for storing blocks of data.

%
% snapshot
%

\section{Snapshot}

For this project, nothing is provided to the students but the freedom to
innovate.

%
% assignments
%

\section{Assignments}

% design

\subsection*{Design}

Students are asked to study the different data structures and algorithms
that could be used for providing the well-known file abstraction in
order to sketch a file system design.

Note that students' file system is not meant to be compliant to any standard.
As such, points could be granted to students showing interest in designing
a modern interface as long as students can argue for such a choice.

% implementation

\subsection*{Implementation}

Students are then asked to implement the file system they designed.

In order to illustrate the functionalities of the file system, students
should also provide user-level applications.
