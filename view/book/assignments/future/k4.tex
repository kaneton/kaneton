%
% ---------- header -----------------------------------------------------------
%
% project       kaneton
%
% license       kaneton
%
% file          /home/mycure/kaneton/view/book/assignments/future/k4.tex
%
% created       julien quintard   [fri nov 28 05:25:37 2008]
% updated       julien quintard   [fri nov 28 16:20:52 2008]
%

%
% ---------- k4 ---------------------------------------------------------------
%

\chapter{k4}
\label{chapter:k4}

\name{k4} consists for students to experiment through the development of
a file system.

\newpage

%
% ---------- text -------------------------------------------------------------
%

%
% objectives
%

\section{Objectives}

This project will lead the students to explore the multiple ways of providing
the file system abstraction but also to study the pros and cons of every
technique in order to choose the right algorithms to implement within the
dedicated time.

%
% prerequisites
%

\section{Prerequisites}

This stage requires a working storage device driver such as an \name{ATA -
Advanced Technology Attachment} driver.

The file system will rely on this driver for storing blocks of data.

%
% snapshot
%

\section{Snapshot}

For this project, nothing is provided to the students but the freedom to
innovate.

%
% assignments
%

\section{Assignments}

% study

\subsection{Study}

First, students are asked to study the different data structures and algorithms
that could be used.

Once studied, the student should decide what techniques to use in their
file system implementation.

% implementation

\subsection{Implementation}

implementation

%
% evaluation
%

\section{Evaluation}

viva: design(why), interface, test applications.

XXX: note that the interface is completely free, as such points will be given
  to students that think about such problems that POSIX limitations etc.
  take advantage of this freedom
XXX: plus, distributed file systems could also be implemented
