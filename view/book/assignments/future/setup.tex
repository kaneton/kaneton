%
% ---------- header -----------------------------------------------------------
%
% project       kaneton
%
% license       kaneton
%
% file          /home/mycure/kaneton/view/book/assignments/future/setup.tex
%
% created       julien quintard   [fri may 23 21:47:38 2008]
% updated       julien quintard   [sat nov 29 16:28:55 2008]
%

%
% ---------- setup ------------------------------------------------------------
%

\chapter{Set-up}
\label{chapter:setup}

This chapter contains information regarding how to, very simply, set up the
\name{kaneton} development environment and build your first microkernel image
to finally run it wherever you want: emulator, real computer \etc{}

\newpage

%
% ---------- text -------------------------------------------------------------
%

%
% requirements
%

\section{Requirements}

Note that the kaneton development environment relies on two well-known tools:

\begin{itemize}
  \item
    \term{GNU Make};
  \item
    \term{Python}.
\end{itemize}

Without all of those tools, a student would not be able to develop the
kaneton educational project.

%
% profile
%

\section{Profile}

This step consists for the developer to specify the kaneton development
environment user-specific configuration values.

As a user, say, \name{foo}, the following directory should be created:

\begin{center}
  \location{environment/profile/user/foo/}
\end{center}

This directory must, at least, a file named \location{foo.conf} specifying
the user's personal configuration variables. The following gives an example of
what this file should at least contain:

\begin{verbatim}
  _MAKE_                          =       make

  _MBL_SCRIPT_                    =       ${_MBL_DIR_}/grub/grub.py

  _BOOT_MODE_                     =       image
  _BOOT_DEVICE_                   =       floppy

  _IMAGE_                         =       ${_SOURCE_DIR_}/kaneton.img
\end{verbatim}

Change the value of \code{\_MAKE\_} if the binary name for \name{GNU Make}
is not \code{make} but \code{gmake} for instance.

The stack protections should also be disabled if a recent version of \name{GCC}
is used:

\begin{verbatim}
  _CC_FLAGS_                     +=       -fno-stack-protector
\end{verbatim}

This configuration tells the kaneton environment system to build kernel images
for floppy devices based on the \name{GRUB} multi-bootloader. The kaneton
micro-kernel image will be generated at the repository root directory.

This file is not mandatory but without it, the user would not be able to
build a complete bootable image, though she could still build the micro-kernel
binary file.

%
% environment variables
%

\section{Environment Variables}

kaneton benefits from a complete and extremely sophisticated development
environment.

This environment allows, amongst others, the developer to choose which
platform and architecture the micro-kernel must be built for in a completely
cooperative way.

The development environment needs to know which user profile, which
architecture and platform to use when compiling the micro-kernel, but also
which compiler to use in order to produce the right assembly language.

Therefore, every developer is to provide the following \name{UNIX} environment
variables:

\begin{itemize}
  \item
    \code{\$\{KANETON\_USER\}} contains the name of the user profile;
  \item
    \code{\$\{KANETON\_HOST\}} contains the host profile name which is
    composed of a couple: operating system and microprocessor;
  \item
    \code{\$\{KANETON\_PLATEFORM\}} holds the name of the target platform;
  \item
    \code{\$\{KANETON\_ARCHITECTURE\}} holds the name of the target
    architecture;
  \item
    \code{\$\{KANETON\_PYTHON\}} contains the path of the python binary. This
    variable is used once in order to generate the current user's development
    environment;
\end{itemize}

The very common case consists in compiling kaneton on for an \name{IBM PC}
with an \name{IA-32 microprocessor}. In such cases, the following values
could be used:

\begin{verbatim}
  export KANETON_USER="foo"
  export KANETON_HOST="linux/ia32"
  export KANETON_PLATFORM="ibm-pc"
  export KANETON_ARCHITECTURE="ia32/educational"
  export KANETON_PYTHON="/usr/bin/python"
\end{verbatim}

%
% initialising
%

\section{Initialising}

Once the \name{UNIX} environment variables are set, the development environment
can be initialised through the command below:

\begin{verbatim}
  $> make initialize
\end{verbatim}

This command will display the user's current configuration values and ask
her to confirm.

%
% compiling
%

\section{Compiling}

Every script, \name{Make} file \etc{} can now behave properly by relying on
the development environment.

The next step consists in compiling the kaneton micro-kernel. This is done
through the following command:

\begin{verbatim}
  $> make kaneton
\end{verbatim}

If everything goes fine, the micro-kernel binary file should have been
generated at the following location:

\begin{center}
  \location{kaneton/kaneton}
\end{center}

The following command can be used for building both the kaneton micro-kernel
along with everything necessary including the bootloader as well as libraries,
modules and so forth.

\begin{verbatim}
  $> make
\end{verbatim}

%
% building
%

\section{Building}

This step consists in initializing the boot device with the multi-bootloader
specified in your configuration file --- probably \name{GRUB} --- through the
command below:

\begin{verbatim}
  $> make build
\end{verbatim}

Once again, the script asks the user to confirm the values the kaneton
development environment has detected. Many of these are irrelevant has
it depends on your boot mode.

For instance, for an \name{image} boot mode, the only other relevant variable
is the image path.

Note that this step only needs to be performed once. Once this is done, the
kaneton micro-kernel can be installed as many times as necessary. Obviously,
if the user changes some of the boot-related environment variables, the image
should be rebuilt.

%
% installing
%

\section{Installing}

The final step consists in copying the kaneton micro-kernel binary as well as
its dependencies: modules, configuration files \etc{} onto the boot device
or image, depending on your configuration.

\begin{verbatim}
  $> make install
\end{verbatim}

If everything went fine, the device should be ready for usage.

%
% launching
%

\section{Launching}

You can test your freshly built kaneton micro-kernel via an emulator
for instance.

The command below illustrates such a test with the \name{QEMU} emulator:

\begin{verbatim}
  $> qemu -fda kaneton.img
\end{verbatim}

%
% testing
%

\section{Testing}

XXX[get a certificate or account and they launch with make test or something]
