%
% ---------- header -----------------------------------------------------------
%
% project       kaneton
%
% license       kaneton
%
% file          /home/mycure/kane...view/book/assignments/future/further.tex
%
% created       julien quintard   [fri may 23 21:47:38 2008]
% updated       julien quintard   [fri nov 28 14:05:04 2008]
%

%
% ---------- further ----------------------------------------------------------
%

\chapter{Going Further}
\label{chapter:further}

This chapter details what students who succeeded in implementing the
\name{kaneton} educational project could do next.

\newpage

%
% ---------- text -------------------------------------------------------------
%

XXX

   You can simply stop as you think that you have learned what you
wanted and that is fine for you.

   Another possibility could be to continue your kaneton
implementation by providing higher level functionalities like a system
call interface so that servers can practically send messages. You
might then be interested in developing servers i.e drivers, services
and applications on top of your freshly developed micro-kernel to see
how it really works.

   In such a case, you should probably take a look at the kaneton
book and more precisely the core book which details the managers
designed and implemented in the kaneton research project.

   Finally, you could also join the kaneton research team and
contribute to an exciting project by thinking differently from people
who work on Linux, BSD and other UNIX-like projects. Indeed, kaneton
does not intend to be a UNIX-like but rather tries to innovate by
considering systems in different ways. Come and bring your ideas let's
talk about your expectations.
