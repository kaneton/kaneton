%
% ---------- header -----------------------------------------------------------
%
% project       kaneton
%
% license       kaneton
%
% file          /home/mycure/kaneton/view/book/assignments/k2.tex
%
% created       julien quintard   [fri nov 28 05:25:37 2008]
% updated       julien quintard   [mon feb  7 20:38:36 2011]
%

%
% ---------- k2 ---------------------------------------------------------------
%

\chapter{k2}
\label{chapter:k2}

The \name{k2} stage consists for students to provide the microkernel a
complete set of memory management functionalities.

Modern kernels rely on virtual memory for ensuring isolation between the
various entities, thus creating what is commonly referred to as
\name{address space}s.

Virtual memory offers many advantages compared to the ancient segmentation
model present on \name{Intel} architectures. Besides, though implemented
differently, most architectures now provide a paging system, hence easing
the task consisting in porting a kernel on several microprocessor
architectures.

\newpage

%
% ---------- text -------------------------------------------------------------
%

%
% requirements
%

\section{Requirements}

Every student should read the chapter titled \name{Paging} from the
\name{IA-32 Architectures Software Developer's Manual, Volume 3A: System
Programming Guide, Part 1}; this chapter contains everything necessary for
implementing this stage.

In addition, students are encouraged to read the chapter \name{Protection}
which contains material regarding the \name{Intel} ring-based protection
mechanism.

%
% assignments
%

\section{Assignments}

Although kaneton already embeds several managers related to memory management,
the microkernel is devoid of an implementation on the \name{IA32} educational
architecture.

The assignments for this stage consists for the student to implement the
missing functionalities especially related to the \name{Intel} microprocessor
architecture.

% core
\subsection{Core}

The core is composed of four managers involved in the memory management.

The \name{segment} manager provides functions for managing the physical memory.
On most platforms, including the \name{ibm-pc}, the physical memory is
represented by the \name{RAM - Random Access Memory}, though not directly
accessible. Thus, whenever the kernel or one of its servers needs memory to
store information, the segment manager is requested.

The \name{region} manager provides the virtual memory capability enabling
the kernel and its servers to access segments directly.

The \name{as} manager provides the possibility to reserve an address space,
itself containing a set of segments as well as a set of regions mapping some of
the segments.

Finally, the \name{map} manager provides high-level functions for easily
reserving, resizing and releasing memory without dealing with segments and
regions or any identifier.

% as
\subsubsection*{as}

\begin{itemize}
  \item
    \location{kaneton/core/as/as.c}
  \item
    \location{kaneton/core/include/as.h}
\end{itemize}

Every address space is linked to its task and contains a set of segments
and a set of regions. In other words, an address space object references all
the physical memory owned by the address space along with the areas of
virtual memory mapping the segments, making these segments accessible directly.

Although the as manager's core is complete, students should definitely look at
it.

% region
\subsubsection*{region}

\begin{itemize}
  \item
    \location{kaneton/core/region/region.c}
  \item
    \location{kaneton/core/include/region.h}
\end{itemize}

The region manager provides a collection of functions for managing regions
\ie{} areas of virtual memory.

As for the as manager, this manager is already complete but students
should browse its source code to fully understand its purpose.

% glue
\subsection{Glue}

The segment manager's glue implements the segment manager's machine through
the \name{Intel} segmentation model while the region manager's glue relies on
the paging mechanism for providing virtual memory.

The particularity of the \name{ia32} architecture compared to the kaneton
design is that \name{Intel}'s paging system assigns permissions to virtual
pages while kaneton assigns permissions to segments \ie{} physical memory.
This difference requires the kaneton machine to adapt by setting pages'
permissions according to the permissions associated with the segment to map.

% as
\subsubsection*{as}

\begin{itemize}
  \item
    \location{kaneton/machine/glue/ibm-pc.ia32/educational/as.c}
  \item
    \location{kaneton/machine/glue/ibm-pc.ia32/educational/include/as.h}
\end{itemize}

The as manager's glue is already complete though the student may want to extend
it.

A particular attention should be paid to the \code{glue\_as\_reserve()}
function which initializes the address space depending on its nature,
either the kernel address space, or one of the server's.

% region
\subsubsection*{region}

\begin{itemize}
  \item
    \location{kaneton/machine/glue/ibm-pc.ia32/educational/region.c}
  \item
    \location{kaneton/machine/glue/ibm-pc.ia32/educational/include/region.h}
\end{itemize}

The region manager's glue must perform the necessary operations in order to
satisfy the core request according to the underlying machine.

For instance, given a \code{region\_reserve()} request, the glue should
probably rely on the architecture in order to map the segment by updating
the page directory and tables accordingly.

The student must therefore complete the region manager's glue by implementing
the necessary functions in order to provide the virtual memory mechanism.

% architecture
\subsection{Architecture}

The \name{ia32/educational} implementation provides skeleton functions for
the virtual memory management. Students are however welcome to change
everything as long as the name and prototype of the functions
\code{architecture\_paging\_pdbr()}, \code{architecture\_paging\_map()} and
\code{architecture\_paging\_unmap()} remain as these functions are directly
used by the tests.

% pd
\subsubsection*{pd}

\begin{itemize}
  \item
    \location{kaneton/machine/architecture/ia32/educational/pd.c}
  \item
    \location{kaneton/machine/architecture/ia32/educational/include/pd.h}
\end{itemize}

Unlike the previous stage, students will notice that the \location{pd.c} file
contains empty functions. Note however that students are welcome to modify
the names/prototypes and re-implement everything their way though this may
require updating other files.

This file should contain functions for manipulating \name{PD - Page
Directories} including building a page directory inside a given memory
location, inserting a page directory entry, deleting a page directory
entry, mapping and unmapping a page directory \etc{}

Note that the \code{architecture\_pd\_map()} does not take the virtual address
where the page directory should be mapped. Thus, it is the function's
responsability to find an available address. Students should take care
to avoid infinite loops by choosing the right functions to use for that
purpose.

% pt
\subsubsection*{pt}

\begin{itemize}
  \item
    \location{kaneton/machine/architecture/ia32/educational/pt.c}
  \item
    \location{kaneton/machine/architecture/ia32/educational/include/pt.h}
\end{itemize}

The file \location{pt.c} is analoguous to \location{pd.c} but instead operates
on \name{PT - Page Table}s.

% paging
\subsubsection*{paging}

\begin{itemize}
  \item
    \location{kaneton/machine/architecture/ia32/educational/paging.c}
  \item
    \location{kaneton/machine/architecture/ia32/educational/include/paging.h}
\end{itemize}

The \location{paging.c} file should contain functions for setting up the
paging mechanism or making a given page directory the system's current one.

Besides, the student will have to complete the two most important paging
functions: \code{architecture\_paging\_map()} takes a segment to map and
updates the necessary page directory and tables accordingly while
\code{architecture\_paging\_unmap()} does the opposite. Note that both
functions should rely on the functions provided in \location{pd.c} and
\location{pt.c}.

% environment
\subsubsection*{environment}

\begin{itemize}
  \item
    \location{kaneton/machine/architecture/ia32/educational/environment.c}
  \item
    \location{kaneton/machine/architecture/ia32/educational/include/environment.h}
\end{itemize}

Finally, the \location{environment.c} file should be updated. Indeed,
three \code{FIXME} can be found in the source code, all related to memory
management.

Note that the page directory set up by the boot loader can be retrieved
in \code{\_init->machine.pd}. Indeed, it is interesting to discover that the
kernel, when launched, already evolves in a virtual environment.

The student can actually notice that the boot loader's page directory is
imported in \code{architecture\_environment\_kernel()} and is later
cleaned from all the needless page tables.

%
% example
%

\section{Example}

The following example illustrates the as, segment and region managers by
setting up a segment being shared between the kernel address space and
another one.

\begin{verbatim}
extern m_kernel         _kernel;

t_error                 example(void)
{
  i_task                task;
  i_as                  as;
  i_segment             segment;
  i_region              region;
  o_region*             o;
  t_uint32*             data;

  /*
   * kernel
   */

  if (segment_reserve(_kernel.as,
                      ___kaneton$pagesz,
                      PERMISSION_READ,
                      SEGMENT_OPTION_NONE,
                      &segment) != ERROR_OK)
    CORE_ESCAPE("unable to reserve a segment");

  if (region_reserve(_kernel.as,
                     segment,
                     0x0,
                     REGION_OPTION_NONE,
                     0x0,
                     ___kaneton$pagesz,
                     &region) != ERROR_OK)
    CORE_ESCAPE("unable to reserve a region");

  if (region_get(_kernel.as, region, &o) != ERROR_OK)
    CORE_ESCAPE("unable to retrieve the region object");

  data = (t_uint32*)o->address;

  *data = 0x42;

  /*
   * server
   */

  if (task_reserve(TASK_CLASS_GUEST,
                   TASK_BEHAVIOUR_INTERACTIVE,
                   TASK_PRIORITY_INTERACTIVE,
                   &task) != ERROR_OK)
    CORE_ESCAPE("unable to reserve a guest task");

  if (as_reserve(task, &as) != ERROR_OK)
    CORE_ESCAPE("unable to reserve an address space");

  if (region_reserve(as,
                     segment,
                     0x0,
                     REGION_OPTION_NONE,
                     0x0,
                     ___kaneton$pagesz,
                     &region) != ERROR_OK)
    CORE_ESCAPE("unable to reserve a region");

  CORE_LEAVE();
}
\end{verbatim}

%
% advices
%

\section{Advices}

This section contains advices that students are welcome to consider:

\begin{itemize}
  \item
    Students should set a page fault handler displaying the instruction causing
    the fault along with the reason and the address of the page being accessed.

    \-

    Such a simple handler would provide students precious information for
    debugging this stage in which page faults are common.
  \item
    It is strongly advised for students to implement functions dumping the
    state of a given page directory or table.

    \-

    Once again, such functions would make the debugging process considerably
    simpler. Indeed, via these functions students could make sure the mappings
    are correct but also that some flags are not responsible for an
    unexpected behaviour.
  \item
    Finally, remember that any change to the current page directory/table
    hierarchy must be followed by cache invalidations in order to maintain
    consistency.
\end{itemize}
