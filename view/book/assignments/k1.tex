%
% ---------- header -----------------------------------------------------------
%
% project       kaneton
%
% license       kaneton
%
% file          /home/mycure/kaneton/view/book/assignments/k1.tex
%
% created       julien quintard   [tue feb  1 21:40:19 2011]
% updated       julien quintard   [mon feb  7 11:16:50 2011]
%

%
% ---------- k1 ---------------------------------------------------------------
%

\chapter{k1}
\label{chapter:k1}

\name{k1} is the first stage taking place within the \name{kaneton}
microkernel. This stage focuses on the event processing mechanism, one of
the most fundamental kernel component. This mechanism provides the kernel
the capacity to react to internal or external stimuli and to perform
computations based on their nature.

The event processing mechanism enables the kernel to catch errors, known
as \name{exceptions} on \name{Intel} architectures, such as division by
zero, page faults \etc{} but also to communicate with external devices.
Indeed, although the easiest way to communicate with devices consists in
probing \name{I/O} ports until the device's state changes to something
expected, \name{interrupt}s provide a mechanism consisting for devices
to generate asynchronously an event which will, as its name suggests, suspends
the kernel's activity for the interrupt to be handled.

Such events are extremely important as they are also used to simulate
multitasking \ie{} the ability to execute multiple threads of execution
in parallel. Note that the mechanism underlying this concept will be
discussed in \name{k3}.

\newpage

%
% ---------- text -------------------------------------------------------------
%

%
% requirements
%

\section{Requirements}

Every student should read the \name{IA-32 Architectures Software
Developer's Manual, Volume 3A: System Programming Guide, Part 1} which
is available on the \name{Wiki}. More precisely students should focus
on the chapter \name{Interrupt and Exception Handling} which contains
all the necessary material to complete this stage.

Finally, if not already the case, students should also read the chapters
\name{Protected-Mode Memory Management} and \name{Protection} as interrupt
handling makes use of the protected mode capabilities.

%
% assignments
%

\section{Assignments}

The \name{k1} stage, focusing on the event processing mechanism, takes place
entirely in the \name{event} manager. However, since every processing is
highly machine-dependent, related code can also be found in the \name{glue},
component whose main task is to redirect calls to the functionalities provided
by the \name{platform} and \name{architecture}.

Although, as students will soon notice, the distributed snapshot includes a
complete core event manager, the glue's is completely empty while the
architecture does not provide any functionality related to the event
processing mechanism.

Therefore, the \name{k1} assignments consist for the student to equip the
kaneton educational implementation with the necessary functionalities making
the event manager fully operational.

Though the stage tries to give the most freedom to students regarding their
implementation, skeleton source files have been provided, hence acting
as entry point for students to start experimenting.

% core
\subsection{Core}

Although the event manager is obviously the most important manager when
it comes to the event processing mechanism, the student should consider
the timer manager as well.

Indeed, the timer manager relies on the underlying hardware timer event
for providing the possibility to trigger functions after some time. The
student may be interested in this manager as being a perfect application
of the event manager.

% event
\subsubsection*{event}

\begin{itemize}
  \item
    \location{kaneton/core/event/event.c}
  \item
    \location{kaneton/core/include/event.h}
\end{itemize}

The core event manager provides the high level interface enabling the kernel
and its servers to plug an event handler with a specific event number \ie{}
event identifier. Note that event identifiers are machine-dependent; for
instance the event \code{\#32} on \name{Intel} architectures represents the
timer. The core event manager thus does not understand the meaning of the event
identifiers and it is therefore the responsability of the machine, through
the glue component, to perform the necessary operations according to the
underlying platform/architecture.

Although the core manager is complete, the student will probably find it
useful to browse its source code to better understand its purpose.

% glue
\subsection{Glue}

As mentioned earlier, the timer manager relies on the hardware timer event.
The student will notice in
\location{kaneton/machine/glue/ibm-pc.ia32/educational/timer.c} that this
event is actually reserved and linked with a handler located within the
core timer manager.

% event
\subsubsection*{event}

\begin{itemize}
  \item
    \location{kaneton/machine/glue/ibm-pc.ia32/educational/event.c}
  \item
    \location{kaneton/machine/glue/ibm-pc.ia32/educational/include/event.h}
\end{itemize}

The glue event manager makes the binding between a core request and the
underlying platform and architecture functionalities.

The student will notice that the glue is empty when it comes to the
event manager. Indeed, it is the student's responsability to complete the
glue in order to equip some of the event manager's functionalities with a
machine-specific behaviour.

% architecture
\subsection{Architecture}

The architecture must be completed in order to provide functionalities for
controlling the microprocessor's capability to handle the interrupts and
exceptions.

For that purpose, skeleton source files are provided for the student to
implement functions regarding the management of the \name{IDT - Interrupt
Descriptor Table} and the set up of the interrupt handlers.

% idt
\subsubsection*{idt}

\begin{itemize}
  \item
    \location{kaneton/machine/architecture/ia32/educational/idt.c}
  \item
    \location{kaneton/machine/architecture/ia32/educational/include/idt.h}
\end{itemize}

The file \location{idt.c} must be completed by students in order to
implement functions related to the \name{IDT} management such as building
an \name{IDT} from a given address, inserting an entry in the \name{IDT}
according to some arguments, deleting an existing \name{IDT} entry \etc{}

Note that in order to create \name{IDT} entries, students should have a look
at the function \code{architecture\_gdt\_selector()} which enables the
caller to generate a segment selector according to several parameters.

% handler
\subsubsection*{handler}

\begin{itemize}
  \item
    \location{kaneton/machine/architecture/ia32/educational/handler.c}
  \item
    \location{kaneton/machine/architecture/ia32/educational/include/handler.h}
\end{itemize}

\name{Intel} interrupts can be classified according to their purpose into
one of the following categories: exceptions, \name{IRQ - Interrupt Request}s,
\name{IPI - Inter-Processor Interrupt}s and system calls \aka{}
\name{syscall}s. Noteworthy is that \name{IPI}s can be ignored since the
educational implementation only supports mono-processor architectures.

The objective is for students to implement the low-level interrupt handlers
which are triggered whenever the associated interrupt is received. Such a
handler, sometimes referred to as \name{lower half}, performs two basic
operations. First, the environment is secure for the execution of the
interrupted thread to be resumed properly, once the interrupt treated.
Second, the high-level event handler function registered through
\code{event\_reserve()} is triggered; this function is sometimes referred
to as the \name{higher half}.

The file \location{handler.c} should thus contain the interrupt handlers
associated with all the interrupts the kernel wishes to handle. Besides,
this file should make use of the functions provided in \location{idt.c}
for setting up the system's \name{IDT}.

% architecture
\subsubsection*{architecture}

\begin{itemize}
  \item
    \location{kaneton/machine/architecture/ia32/educational/include/architecture.h}
\end{itemize}

The student should also take care to record, whenever appropriate, the
error code associated with the interrupt in \code{\_architecture.error}.

A special attention must be given to the \code{\_architecture} structure.
Indeed, because this structure is being used by the testing system, its
organisation should never be modified.

% platform
\subsection{Platform}

% pic
\subsubsection*{PIC}

\begin{itemize}
  \item
    \location{kaneton/machine/platform/ibm-pc/pic.c}
  \item
    \location{kaneton/machine/platform/ibm-pc/include/pic.h}
\end{itemize}

The platform plays a minor role in the event processing mechanism. However,
students should have a look at the \name{PIC - Programmable Interrupt
Controller} which may prove useful in this stage.

%
% example
%

\section{Example}

This section presents an example which students can use to better understand
the use of the event manager. This example illustrates a breakpoint exception
being generating manually through the \code{int} assembly instruction.
The exception is then caught by the kernel which notices that an event handler
has been registered and thus triggers it.

\begin{verbatim}
void                    exception_bp(t_id                       id,
                                     t_vaddr                    data)
{
  printf("[event %qd] breakpoint exception caught\n",
         id);
}

void                    example(void)
{
  if (event_reserve(ARCHITECTURE_IDT_EXCEPTION_BP,
                    EVENT_TYPE_FUNCTION,
                    EVENT_ROUTINE(exception_bp),
                    EVENT_DATA(NULL)) != ERROR_OK)
    {
      printf("[event_reserve] error");
      return;
    }

  asm volatile("int $3");

  if (event_release(ARCHITECTURE_IDT_EXCEPTION_BP) != ERROR_OK)
    {
      printf("[event_release] error");
      return;
    }
}
\end{verbatim}

%
% advices
%

\section{Advices}

This section contains advices that students are welcome to consider:

\begin{itemize}
  \item
    Students should write macro-functions for setting the fields within
    the \name{IDT} entries, making it easier to transform a single address
    into scattered fragments placed at different position within an entry.

    \-

    In addition, the opposite should also be set up in order to easily
    retrieve a field from an \name{IDT} entry.
  \item
    The interrupt handlers located in \location{handler.c} should be
    composed of two parts.

    \-

    The first one, composing the entry point of the \name{ISR - Interrupt
    Service Routine}, should be written in assembly in order to prevent
    the compiler from generating additional instructions.

    \-

    The second one, called by the first one and written in \name{C}, triggers
    the high-level event handlers registered for this event.
  \item
    The assembly part of the interrupt handlers are actually all very similar,
    differing slightly depending on the nature of the interrupt: exception,
    \name{IRQ}, \name{IPI} \etc{} and depending on the presence of an
    error code.

    \-

    Therefore, instead of writing all these assembly functions manually,
    one could really on a macro-function for generating the assembly
    code automatically.
  \item
    In order to ease the debugging process, students should get used to
    writing functions for dumping the state of the processor's structure,
    such as the \name{IDT}.
  \item
    An interrupt handler, noticing that no high-level handler has been
    registered for this event identifier, should display a message on the
    console warning that this interrupt was unexpected.

    \-

    This way, the developer will easily notice exceptions occuring as well
    as unhandled IRQs.
\end{itemize}
