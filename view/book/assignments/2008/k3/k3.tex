%%
%% licence       kaneton licence
%%
%% project       kaneton
%%
%% file          /home/buckman/kaneton/view/books/assignments-k2/k2.tex
%%
%% created       matthieu bucchianeri   [tue feb  7 11:49:56 2006]
%% updated       matthieu bucchianeri   [wed mar 14 23:46:36 2007]
%%

%
% k3
%

\chapter{K3: tasks \& scheduling}

%
% informations
%

\begin{tabular}{p{7cm}l}
Duration: & 3 weeks \\
Directory name: & kaneton/ \\
In charge: & Julian Pidancet \& Elie Bleton\\
Newsgroup: & epita.cours.kaneton \\
Languages: & C \\
Students per group: & 2 (\textit{pour le meilleur et pour le pire}) \\
\end{tabular}

\section{Abstract}

K3 project consists in giving kaneton the ability to execute
tasks. Threads are implemented in kernel-land (1:1 model). The
scheduling model is time-sharing with fixed priorities.

The concerned managers are:

\begin{enumerate}
  \item
    {\bf The task manager}\\
    This manager works on the PCB (Process Control Block) of every
    task. This PCB is very simple and made of a few properties, a
    parent identifier, a set of , an address space identifier
    and a set of threads.
  \item
    {\bf The thread manager}\\
    The thread object is composed of the parent task identifier, stack
    information and scheduling properties like priority. The thread
    manager manages both this part of the thread object plus the
    machine-dependent object made of the thread's execution context.
  \item
    {\bf The sched manager}\\ This manager implements the
    scheduler. The scheduler is called periodically (on timer for
    example) to allocate the microprocessor's time.
\end{enumerate}

Your work for this stage of kaneton is to provide a working
time-sharing execution environment both for kernel-land threads and
user-land threads.

\textbf{Important}: do no forget the FIXME in the machine-dependent
part of the as manager, near the end of the file.

Remember that you can add code even if there is no FIXME.


%
% task manager
%

\newpage

\section{Introduction to the \textbf{task} manager}

\begin{itemize}
  \item {\bf Overview}\\

    The \textbf{task manager} provides a complete interface for the
    task objects manipulation.

    A \textbf{task object} \textit{o\_task} describes a complete
    execution entity. Nevertheless, a task object is never scheduled
    since a task is not an active entity (threads are).

    A task is composed of an address space (\textit{i\_as}) and threads
    (set of \textit{i\_thread}).

  \item {\bf Assignments}\\

    No assignments. No modification of the task manager is
    required. The following interface is given as information.

    \textbf{Warning:} remember to ``unpatch'' the line is
    \emph{task\_current} (modified for K1 and K2).

  \item {\bf The \emph{task} interface}\\
    XXX FIXME

  \item {\bf Interface}\\

\function{task\_show}{(i\_task \argument{id})}
	 {
	   This function displays information on the task \argument{id}.
	 }

\function{task\_dump}{(void)}
	 {
	   This function displays information on every task.
	 }

\function{task\_clone}{(i\_task \argument{old},
                        i\_task* \argument{new})}
	 {
	   This function clones a task.

	   This function must take care of cloning everything necessary
	   including the address space and the threads.
	 }

\function{task\_reserve}{(t\_class \argument{class},
                          t\_behav \argument{behav},
                          t\_prior \argument{prior},
                          i\_task* \argument{id})}
	 {
	   This function reserves a task object with the given
	   properties: \argument{class}, \argument{behav} and
	   \argument{prior}.

	   Note that once reserved, the task is marked as stopped.

	   Task classes include:

	   \begin{itemize}
	     \item
	       \emph{TASK\_CLASS\_CORE}: the tasks executes with
	       supervisor privileges.
	     \item
	       \emph{TASK\_CLASS\_PROGRAM}: the tasks executes as a
	       classical program with user privileges.
	     \item
	       Other classes must not be supported and will not be
	       tested.
	   \end{itemize}

	   Task behavior is used to force a priority interval. Values
	   are (from higher priorities towards lower):

	   \begin{enumerate}
	     \item \emph{TASK\_BEHAV\_CORE}
	     \item \emph{TASK\_BEHAV\_REALTIME}
	     \item \emph{TASK\_BEHAV\_INTERACTIVE}
	     \item \emph{TASK\_BEHAV\_TIMESHARING}
	     \item \emph{TASK\_BEHAV\_BACKGROUND}
	   \end{enumerate}

	   The default priority of a task is
	   \emph{TASK\_PRIOR\_$<<$behav$>>$}.
	 }

\newpage

\function{task\_release}{(i\_task \argument{id})}
	 {
	   This function releases the task object \argument{id}.
	 }

\function{task\_priority}{(i\_task \argument{id},
                           t\_prior \argument{prior})}
	 {
	   This function updates the task's priority to \argument{prior}.

	   The priority must be in the interval corresponding to the
	   task behavior, between \emph{TASK\_LPRIOR\_$<<$behav$>>$} and
	   \emph{TASK\_HPRIOR\_$<<$behav$>>$}.
	 }

\function{task\_state}{(i\_task \argument{id},
                        t\_state \argument{sched})}
	 {
	   This function starts or stop a task (this means all its
	   threads).

	   Values for \argument{sched} are:

	   \begin{itemize}
	     \item
	       \emph{SCHED\_STATE\_RUN}: run the task.
	     \item
	       \emph{SCHED\_STATE\_STOP}: stop the task.
	   \end{itemize}
	 }

\function{task\_wait}{(i\_task \argument{id},
                       t\_opts \argument{opts},
                       t\_wait* \argument{wait})}
	 {
	   This function waits for state change in one or more tasks
	   depending on the options \argument{opts}.

	   \notice{This feature is not yet implemented.}
	 }

\function{task\_get}{(i\_task \argument{id},
                      o\_task** \argument{o})}
	 {
	   This function returns in \argument{o} the task object corresponding
	   to \argument{id}.
	 }

\function{task\_init}{(void)}
	 {
	   This function initializes the task manager.
	 }

\function{task\_clean}{(void)}
	 {
	   This function cleans the task manager.
	 }

  \item {\bf {Files}}\\

    \begin{tabular}{| l | l |}
      \hline
      machine-independent & {\em kaneton/core/task/task.c}\\
      &  {\em kaneton/include/core/task.h}\\\hline
      machine-dependent & {\em kaneton/core/arch/ibm-pc.ia32-virtual/task.c}\\
      & {\em kaneton/include/arch/ibm-pc.ia32-virtual/core/task.h}\\\hline
      libarch & {\em libs/libia32/task/*.c}\\
      & {\em libs/libia32/include/task/*.h}\\\hline
    \end{tabular}
\end{itemize}


%
% thread manager
%

\newpage

\section{\textbf{thread} manager}
\subsection*{Overview}

The \textbf{thread manager} manages the real active execution
context called threads.

A \textbf{thread object} is an active entity which describes the
current state of an execution context including the
\textit{program counter}, the \textit{stack pointer} etc.

Indeed, a thread is composed of a current \textit{program counter}
which keeps the next instruction address to execute and the
\textit{stack pointer} which keeps the stack address. With these
two characteristics, a thread can be described.

Needless to say, some additional architecture-dependent properties
are required to fully describe a thread context.

\subsection*{Types overview}
XXX FIXME


\subsection*{Assignments}

You have to write the IA-32 part of the thread manager.

This part of the manager manages execution context of the threads,
which is by definition of context very specific between
architectures. First of all, you will have to create a structure
in the machine-dependent part of the \emph{o\_thread} object that
represents the execution context of the thread.

Next, your only work is to maintain this structure.

\textbf{Hint:} Values of segment selector registers depend on the
task's class. See \emph{ia32\_segment\_init} for details.

\textbf{Hint:} Flags must be set correctly in a new
thread. Interrupt Flag (IF) must be enabled in order to process
interrupts correctly.

\subsection*{Interface}

\function{thread\_show}{(i\_thread \argument{id})}
	 {
	   This function displays information about the thread \argument{id}.
	 }

\function{thread\_dump}{(void)}
	 {
	   This function displays information about all the threads.
	 }

\function{thread\_give}{(i\_task \argument{task},
                         i\_thread \argument{id})}
	 {
	   This function gives the thread object \argument{id} to the
	   task \argument{task}.
	 }

\function{thread\_clone}{(i\_task \argument{task},
                          i\_thread \argument{old},
                          i\_thread* \argument{new})}
	 {
	   This function clones the thread \argument{old} into a new one
	   \argument{new}.

	   This new thread, having the exact same properties as \argument{old},
	   will be held by the task \argument{task}.
	 }

\function{thread\_reserve}{(i\_task \argument{task},
                            t\_prior \argument{prior},
                            i\_thread* \argument{id})}
	 {
	   This function reserves a thread for the task \argument{task}
	   given the default thread priority \argument{prior}.

	   Note that once reserved, the thread is marked as stopped.

	   The priority must be in the interval \emph{THREAD\_LPRIOR},
	   \emph{THREAD\_HPRIOR}.
	 }

\function{thread\_release}{(i\_thread \argument{id})}
	 {
	   This function releases the thread object \argument{id}.
	 }

\function{thread\_priority}{(i\_thread \argument{id},
                             t\_prior \argument{prior})}
	 {
	   This function updates the current thread priority
	   to \argument{prior}.

	   The priority must be in the interval \emph{THREAD\_LPRIOR},
	   \emph{THREAD\_HPRIOR}.
	 }

\function{thread\_state}{(i\_thread \argument{id},
                          t\_state \argument{sched})}
	 {
	   This function stops or starts a thread.

	   Values for \argument{sched} are:

	   \begin{itemize}
	     \item
	       \emph{SCHED\_STATE\_RUN}: run the task.
	     \item
	       \emph{SCHED\_STATE\_STOP}: stop the task.
	   \end{itemize}
	 }

\function{thread\_wait}{(i\_thread \argument{id},
                         t\_opts \argument{opts},
                         t\_wait* \argument{wait})}
	 {
	   This function acts like the function \emph{task\_wait}().

	   This function waits for the thread's state to change depending on
	   the options \argument{opts}.

	   \notice{This feature is not yet implemented. Do not care
	     about it}
	 }

\function{thread\_get}{(i\_thread \argument{id},
                        o\_thread** \argument{o})}
	 {
	   This function returns in \argument{o} the thread object
	   corresponding to \argument{id}.
	 }

\function{thread\_flush}{(i\_task \argument{task})}
	 {
	   This function removes every thread that belongs to the
	   task object \argument{task}.
	 }

\function{thread\_stack}{(i\_thread \argument{id},
                          t\_stack \argument{stack})}
	 {
	   This function allocates a stack for the thread \argument{id}.

	   The \argument{stack} argument is a struct containing the
	   size of the stack and a field called \emph{base}. If the
	   base is set to a value different of 0, then no space will
	   be allocated and the base of the stack will be set to the
	   address \argument{base}.
	 }

\function{thread\_load}{(i\_thread \argument{id},
                         t\_thread\_context \argument{context})}
	 {
	   This function loads a new execution context in the thread
	   object \argument{id}.

	   A thread execution context \textit{t\_thread\_context}
	   only contains the \textit{program counter} and the
	   \textit{stack pointer}.
	 }

\function{thread\_store}{(i\_thread \argument{id},
                          t\_thread\_context* \argument{context})}
	 {
	   This function stores in \argument{context} the current
	   thread execution context of the thread object \argument{id}.
	 }

\function{thread\_initialize}{(void)}
	 {
	   This function initializes the thread manager.
	 }

\function{thread\_clean}{(void)}
	 {
	   This function cleans the thread manager.
	 }

\newpage

\subsection*{Files}

\begin{tabular}{| l | l |}
  \hline
  machine-independent & {\em kaneton/core/thread/thread.c}\\
  &  {\em kaneton/include/core/thread.h}\\\hline
  machine-dependent & {\em kaneton/core/arch/ibm-pc.ia32-virtual/thread.c}\\
  & {\em kaneton/include/arch/ibm-pc.ia32-virtual/core/thread.h}\\\hline
  libarch & {\em libs/libia32/task/*.c}\\
  &  {\em libs/libia32/include/task/*.h}\\\hline
\end{tabular}


%
% sched manager
%

\newpage

\section{\textbf{sched} manager}
\subsection*{Overview}

The scheduler's role is to allocate cpu's resources to the
active threads depending on their priorities.

Be careful, the scheduler schedules the threads which are the active
execution entities and not the tasks.

\subsection*{Assignments}

For K3, you have to implement the whole sched manager, consisting
in a simple round-robin algorithm with fixed priorities and the
machine-dependent part doing the context switching.

The architecture-dependent part of the scheduler is also
responsible of calling the \emph{sched\_switch} periodically. This
is typically done with a timer.

\textbf{Hint:} the special case of the kernel main thread must be
considered. The scheduler must create a kernel thread
(corresponding to the execution context in the kernel). Think
twice about it, there is a very simple implementation.

\textbf{Note about priorities:} the task priority will not be
used. Only priorities of thread are used for scheduling. There is
no real policiy required for priorities. Our tests will check that
on a given amount of time, higher-priority threads are executed
more often than lower-priority one.

\subsection*{Interface}

\function{sched\_dump}{(void)}
{
  This (\textbf{optional}) function displays the scheduler state.
}

\function{sched\_quantum}{(t\_quantum \argument{quantum})}
{
  This function sets the scheduler quantum to \argument{quantum}.

  The quantum corresponds to the period the scheduler is called.
}

\function{sched\_yield}{(i\_cpu \argument{cpuid})}
{
  This function permits the current thread to relinquish
  the processor voluntarily.

  Do not care about the argument \argument{cpuid}.
}

\function{sched\_add}{(i\_thread \argument{thread})}
{
  This function adds a runnable thread to the scheduler. Even
  if the added thread has the highest priority, do not yield
  the current thread.
}

\function{sched\_remove}{(i\_thread \argument{thread})}
{
  This function removes a thread from the
  scheduler. \textbf{Be careful:} the thread to remove can be
  the currently executing thread.
}

\function{sched\_update}{(i\_thread \argument{thread})}
{
  This function asks the scheduler to update the thread
  \argument{thread} in its internal data structures since
  for example the thread's priority just changed.
}

\function{sched\_current}{(i\_thread* \argument{thread})}
{
  This function returns in \argument{thread} the identifier
  of the thread currently executed.
}

\function{sched\_switch}{(void)}
{
  This function just elects and schedules a new thread.

  \textbf{Note}: the machine-dependent code of this function
  takes an additional parameter.

  \function{ia32\_sched\_switch}{(i\_thread \argument{elected})}
  {
    The argument \argument{elected} is the new
    thread to run.
  }
}

\function{sched\_init}{(void)}
{
  This function initializes the scheduler.
}

\function{sched\_clean}{(void)}
{
  This function cleans the scheduler.
}

\subsection*{Files}

\begin{tabular}{| l | l |}
  \hline
  machine-independent & {\em kaneton/core/sched/sched.c}\\
  &  {\em kaneton/include/core/sched.h}\\\hline
  machine-dependent & {\em kaneton/core/arch/ibm-pc.ia32-virtual/sched.c}\\
  & {\em kaneton/include/arch/ibm-pc.ia32-virtual/core/sched.h}\\\hline
  libarch & {\em libs/libia32/task/*.c}\\
  &  {\em libs/libia32/include/task/*.h}\\\hline
\end{tabular}

%
% advanced topics
%

\newpage

\section{Bonuses}

kaneton microkernel is first of all a pedagogical project which do not
aims at being optimized. That is why, when nothing is specified, you
always will implement the simplest algorithms.\\
\\
Nevertheless, we will always encourage students who want to write
additional bonuses, as far as they respect the following rules:

\begin{enumerate}
  \item Bonuses will be evaluated only if a basic implementation is
  actually working.
  \item Bonuses must be either picked from the following list, or
  accepted by the kaneton team.\\
\end{enumerate}

Bonuses ideas:
\begin{itemize}
\item
  Implement a better scheduler. Feel free to implement any other
  algorithm. You can try the multi-level feedback queue explained
  during the lesson.

\item
  Add dynamic priorities.

\item
  Implement support for Floating Point Unit (FPU) contexts. The FPU
  context must be saved only when required.
\end{itemize}

%
% appendix
%

\newpage

\section{Appendix}

\textbf{Example of task creation}

\begin{verbatim}
i_task           tsk;
i_as             as;
i_thread         thr;
o_thread*        o;
t_thread_context ctx;
t_stack          stack;

task_reserve(TASK_CLASS_PROGRAM,
             TASK_BEHAV_INTERACTIVE,
             TASK_PRIOR_INTERACTIVE,
             &tsk);

as_reserve(tsk, &as);

thread_reserve(tsk, THREAD_PRIOR, &thr);

stack.base = 0;
stack.size = THREAD_MIN_STACKSZ;
thread_stack(thr, stack);

thread_get(thr, &o);

ctx.sp = o->stack + o->stacksz - 16;
ctx.pc = (t_vaddr)entry_point;

thread_load(thr, ctx);

task_state(tsk, SCHEDULER_STATE_RUN);

...

void             entry_point()
{
  printf("This is a thread.\n");

  /*                         */
  /*   I MUST NOT RETURN !   */
  /*                         */
  while (1)
    ;
}
\end{verbatim}
