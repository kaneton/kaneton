%
% ---------- header -----------------------------------------------------------
%
% project       kaneton
%
% license       kaneton
%
% file          /home/mycure/kaneton/view/book/assignments/k3.tex
%
% created       julien quintard   [fri nov 28 05:25:37 2008]
% updated       julien quintard   [fri feb 11 09:30:43 2011]
%

%
% ---------- k3 ---------------------------------------------------------------
%

\chapter{k3}
\label{chapter:k3}

\name{k3} consists in providing the kaneton microkernel with a multitasking
environment.

kaneton relies on the concept of \name{task} as the execution entity. Every
task is actually composed of an address space and one or more \name{thread}s.
These threads are the actual active entity being managed by the
\name{scheduler} which decides which one should be executed next. By switching
between threads at a fast pace, the kernel gives the user the illusion to
execute programs in parallel.

This stage will allow students to learn the following concepts:

\begin{itemize}
  \item
    \term{Scheduler Algorithms}

    \-

    Throughout this stage's lectures, students will learn how scheduler
    algorithms can vary depending on the system's target, from general-purpose
    to specialized algorithms such as for real-time systems for instance.

    \-

    In addition, students will have to implement their own algorithm, hence
    providing the kaneton microkernel with a mechanism for controlling the
    threads' execution.
  \item
    \term{Hardware Context Switch}

    \-

    Students will also have to implement the low-level context switch
    mechanism which relies on interrupts. Through this implementation students
    will manipulate \name{TSS - Task State Segment}s and the \name{KIS - Kernel
    Interrupt Stack} among others.
\end{itemize}

\newpage

%
% ---------- text -------------------------------------------------------------
%

%
% requirements
%

\section{Requirements}

Every student will find the necessary material in the chapter \name{Task
Management} of the \name{IA-32 Architectures Software Developer's Manual,
Volume 3A: System Programming Guide, Part 1}.

In addition, students should not hesistate to dive into the chapters
\name{Interrupt and Exception Handling} and \name{Protection} which contain
material directly related to this stage.

%
% assignments
%

\section{Assignments}

The objective of this stage is for students to \textit{(i)} complete the
scheduler manager's core by implementing their own algorithm and \textit{(ii)}
implement the hardware \name{Intel} context switch.

% core
\subsection{Core}

The core contains four managers related to the execution. The \name{task}
manager provides functions for managing the global execution entities \ie{}
tasks. The \name{thread} manager handle the active entities referred to as
threads. The \name{cpu} manager controls the multiple processors present
on the machine. Finally, the \name{scheduler} manager maintains the multiple
schedulers, one for each \name{CPU}, each of which is responsible for the
execution of a colletion of threads.

% scheduler
\subsubsection*{scheduler}

\begin{itemize}
  \item
    \location{kaneton/core/scheduler/scheduler.c}
  \item
    \location{kaneton/core/include/scheduler.h}
\end{itemize}

The student will find in \location{scheduler.h} the structure of the
scheduler object \code{o\_scheduler}. Let us recall that the scheduler
manager is responsible for as many schedulers as the machine supports
processors. Students are welcome to modify the structure of the scheduler
object along with anything else as long as the scheduler manager's
interface remains intact.

Note that although students are welcome to implement their own scheduler
algorithm, the test system obviously expects the algorithm to be
preemptive.

The following describe in detail every one of the functions which has
been left to the student to implement.

\function{t\_error}{scheduler\_quantum}{\type{t\_quantum} \argument{quantum}}
         {
           This function modifies the scheduler manager's quantum \ie{}
           the smallest unit of execution time; in other words threads are
           always executed for a duration multiple of the scheduler's quantum.

           \-

           Note that in order to provide the best precision possible, the
           argument \argument{quantum} should be a multiple of the timer
           delay.

           \-

           This function's objectives are twofold. First the scheduler
           manager's quantum must be updated. Second, the timeslice---\ie{}
           the duration a thread is allowed to be executed---of the threads
           registered in the scheduler, being active or not, should be
           re-computed since the quantum has changed.
         }

\function{t\_error}{scheduler\_yield}{}
         {
           This function is one of the most important in the scheduler
           manager. Whenever this function is called by the running thread,
           its execution must be stopped immediately hence inevitably
           leading to an election followed by a context switch.

           \-

           Students are advised to thoroughly study the way the whole
           context switch mechanism works on the \name{ia32/educational}
           machine before implementing this functionality.

           \-

           Noteworthy is that this function should also ensure that another
           thread is scheduled \ie{} that the thread relinquishing the
           execution will not end up being re-elected right away. Indeed,
           the student can notice in the task and thread managers that this
           function is sometimes used to remove a thread definitely from
           the scheduler, for instance after its death. If the thread happens
           to be re-elected, its execution will continue while the kernel
           assumes it had been removed. Note however that the thread yielding
           its execution could very much be re-elected soon after, with a
           single thread being elected in between.

           \-

           The student must therefore find a way to relinquish the execution
           right away while ensuring that the next elected thread is different.
         }

\function{t\_error}{scheduler\_elect}{}
         {
           This function is at the heart of the scheduler manager. The role
           of the \code{scheduler\_elect()} function is to, as its name
           suggests, elect the next thread to execute.

           \-

           This function is called on a regular basis, more precisely
           every quantum milliseconds. Thus, the currently being executed
           thread may not have exceeded its timeslice---\ie{} allocated time
           of execution---in which case the thread could be re-elected.
           However, depending on the student's algorithm, a thread with a
           higher priority may need to be scheduled in which case the current
           thread could be put on hold for some time.

           \-

           Noteworthy is that the current thread whose state is no longer
           \code{THREAD\_START} or whose task's state is no longer
           \code{TASK\_START} should be removed from the scheduler's queues.
           For more information on this specific case, please refer to
           \code{scheduler\_remove()}.

           \-

           In order to ensure that the \name{CPU} is always executing
           something, this function must make sure to always elect a thread.
           How to execute code when there is no thread left to execute
           is for students to figure out.

           \-

           Finally, should the scheduler's state change to
           \code{SCHEDULER\_STOP}, this function must elect the special
           kernel thread \ie{} \code{\_kernel.thread} in order to come back
           to the very first thread which started the scheduler. Note that
           this is extremely important as required by the test system to
           operate. Should this functionality be missing, all the tests
           would probably time out.
         }

\function{t\_error}{scheduler\_add}{\type{i\_thread} \argument{id}}
         {
           This function adds the thread identified by \argument{id} to
           the scheduler for execution.
         }

\function{t\_error}{scheduler\_remove}{\type{i\_thread} \argument{id}}
         {
           This function removes the thread \argument{id} from the scheduler.

           \-

           Note that a special case must be made if the thread to remove
           is currently being executed; the thread's execution should be
           relinquished immediately. Note that this function is always
           called whenever the thread's state or its task's is being changed
           from \code{START} to something else. For more information, please
           refer to \code{thread\_stop()}, \code{thread\_block()},
           \code{thread\_exit()} or their task manager's counterparts.
         }

\function{t\_error}{scheduler\_update}{\type{i\_thread} \argument{id}}
         {
           This function is called whenever the priority of the thread
           or of its task's has changed. This function aims at updating
           the scheduler priority of the given thread \argument{id}.

           \-

           Note that, depending on the student's algorithm, updating the
           thread's scheduler priority may require the thread to be moved
           to another queue, for instance. In addition, the thread's
           timeslice may also need to be re-computed.
         }

Students should thoroughly test the scheduler implementation, making
sure that threads are properly elected before moving to the hardware
context switch.

%
% glue
%

\subsection{Glue}

Although, once again, the code of the \name{cpu}, \name{task}, and
\name{thread} managers' glue is already provided, students are advised
to take a look at them, especially the thread manager's glue which makes
use of the \code{context\_setup()}, \code{context\_build()} and
\code{context\_destroy()} functions which will be discussed later.

% scheduler
\subsubsection*{scheduler}

\begin{itemize}
  \item
    \location{kaneton/machine/glue/ibm-pc.ia32/educational/scheduler.c}
  \item
    \location{kaneton/machine/glue/ibm-pc.ia32/educational/include/scheduler.h}
\end{itemize}

The scheduler manager's glue already contains a number of functions. Among
the most useful ones, \code{glue\_scheduler\_start()} which reserves the
timer used for triggering the \code{glue\_scheduler\_switch()} handler on
a regular basis. Note that the \argument{cpu} argument is ignored because
the machine knows that multiprocessor architectures are not supported.
Therefore, a single timer is reserved whatever the value of \argument{cpu}.
On the other hand, \code{glue\_scheduler\_stop()} relinquishes the execution
in order for an election to be triggered. Thus, \code{scheduler\_elect()}
will notice that the scheduler's state has changed to \code{SCHEDULER\_STOP}
and the execution will be handed to the kernel thread, as expected.

The \code{glue\_scheduler\_switch()} is at the heart of this glue. This
handler is triggered after quantum milliseconds and takes care to elect
a new thread and call the \code{architecture\_context\_switch()} function
in order to prepare the hardware context switch. Indeed, as discussed
next, the \code{architecture\_context\_switch()} function does not perform
the hardware context switch \textit{per se}.

Noteworthy is that these functions assume that the currently running or
being elected thread is referenced through the \name{CPU}'s scheduler at
\code{scheduler->thread}.

Students will probably need to add functions to this glue in order to
complete the scheduler's functionalities on this machine.

%
% architecture
%

\subsection{Architecture}

The \name{ia32/educational} architecture relies on the interrupts in order
to provide a context switch mechanism. This concept is based on the fact
that whenever a thread is interrupted, its context is saved on a stack
so that, when returning from the interrupt through the \code{iret} assembly
instruction, the \name{CPU} retrieves the context from the stack, hence
resuming the thread's execution. The context switch mechanism basically
consists in \textit{``replacing''} the context to be resumed in order for
the processor to retrieve the context of another thread, hence switching
the execution from the interrupted thread to another.

Note that the particularity of the \name{ia32/educational} architecture lies
in the fact that a special stack, known as the \name{KIS - Kernel Interrupt
Stack}, is used for handling interrupts. Indeed, while in the stage \name{k1}
students were free to handle the interrupts their way, it is mandatory in
\name{k3} to comply with this design.

Therefore, whenever an interrupt is triggered, the assembly part of the
interrupt handler must save the necessary information regarding the interrupted
thread. Then, the address space of the kernel must be loaded so as to
set up the special interrupt-specific stack known as the \name{KIS}.
Thus, the high-level interrupt handler is called and operates in the
kernel address space, on a specific stack. Finally, once the interrupt
handled, the kernel does the opposite actions by coming back from
the \name{KIS} to the stack on which the interrupted thread's context
has been saved, of another thread to resume. Then, the thread's context
is restored so as for the \name{CPU} to resume its execution.

% context
\subsubsection*{context}

\begin{itemize}
  \item
    \location{kaneton/machine/architecture/ia32/educational/context.c}
  \item
    \location{kaneton/machine/architecture/ia32/educational/include/context.h}
\end{itemize}

The \name{ia32} context management is composed of several functions which
are briefly described next.

The \code{architecture\_context\_build()} function takes the identifier of a
thread, and sets up its initial context depending on its privilege. Indeed,
let us recall that while the context of an interrupted thread evolving in
\name{ring0} is stored on its current stack, this is not the case for threads
with lower privileges. Indeed, such threads possess an additional
stack---known as the \name{pile} in kaneton and the \name{kernel stack} in
many other kernels---on which their context is stored when interrupted.
The function \code{architecture\_context\_destroy()} does the opposite by
releasing the resources related to the thread's low-level context.

In addition, the functions \code{architecture\_context\_set()} and
\code{architecture\_context\_get()} manipulate the given thread's context.
Note that the context which these functions assumed to find on the thread's
stack or pile is given by the structure \code{as\_context}.

Finally, the \code{architecture\_context\_setup()} function initializes the
context switch mechanism by reserving the \name{KIS - Kernel Interrupt Stack}
\ie{} the stack used specifically to handle interrupts. Moreover, this
function allocates and initializes the system's \name{TSS - Task State
Segment}. To better understand why a \name{TSS} is used, students are advised
to read the chapters given in the \name{Requirements} section.

The objective of this stage is for students to implement the
\code{architecture\_context\_switch()} function along with the macro-functions
\code{ARCHITECTURE\_CONTEXT\_SAVE()} and
\code{ARCHITECTURE\_CONTEXT\_RESTORE()}. Note however that students are
welcome to modify the whole structure of the context managenent system or
even re-implement everything they need.

Since the context switch mechanism relies on interrupts, students will
have to modify their interrupt handling system in order to use both
\code{ARCHITECTURE\_CONTEXT\_SAVE()} and
\code{ARCHITECTURE\_CONTEXT\_RESTORE()}. Indeed, until now the only thread
being executed, hence interrupted, was the kernel thread which evolves in
\name{ring0}. The save/restore mechanism will therefore need to be adapted
in order to support non-\name{ring0} threads.

The \code{architecture\_context\_switch()} function, which is called by
\code{glue\_scheduler\_switch()}, should prepare the context switch by
recording, in a structure such as \code{\_architecture}, the information
required for \code{ARCHITECTURE\_CONTEXT\_RESTORE()} to restore the context
of the thread to resume. Then, when returning from the interrupt handler,
the \code{ARCHITECTURE\_CONTEXT\_RESTORE()} macro-function will finally be
called whose role will be to set up the information related to the thread
to resume so that the \name{CPU} restore its registers.

% architecture
\subsubsection*{architecture}

\begin{itemize}
  \item
    \location{kaneton/machine/architecture/ia32/educational/architecture.c}
  \item
    \location{kaneton/machine/architecture/ia32/educational/include/architecture.h}
\end{itemize}

The \code{\_architecture} variable whose structure is given in
\location{architecture.h} can act as a bridge between the
\code{architecture\_context\_switch()} and the
\code{ARCHITECTURE\_CONTEXT\_RESTORE()} macro-function.

The student will notice that this structure contains important information
such as the kernel \name{PDBR - Page Directory Base Register}, the location
of the \name{KIS - Kernel Interrupt Stack} among others.

%
% example
%

\section{Example}

The following example illustrates how the scheduler manager can be used
to execute threads. The reader may be surprised to see \code{event\_enable()}
immediately followed by \code{event\_disable()}. However, one must understand
that once the scheduler has been started and the events enabled, the
thread will start being scheduled until one of those threads calls
\code{scheduler\_stop()}. Indeed, remember that the \code{scheduler\_elect()}
function, noticing the change of state to \code{SCHEDULER\_STATE\_STOP}, will
elect the kernel thread for execution which happens to be the very first
thread, the one which enabled the events. Therefore, resuming its execution,
the kernel thread continues and disables the events right away.

\begin{verbatim}
  extern m_kernel       _kernel;

  void                  example_thread(void)
  {
    i_cpu               cpu;

    if (cpu_current(&cpu) != STATUS_OK)
      {
        printf("unable to retrieve the current CPU\n");
        return;
      }

    if (scheduler_stop(cpu) != STATUS_OK)
      {
        printf("unable to stop the scheduler\n");
        return;
      }

    printf("unreachable");

    while (1)
      ;
  }

  t_status               example(void)
  {
    i_thread            thread;
    i_cpu               cpu;

    if (thread_reserve(_kernel.task,
                       THREAD_PRIORITY,
                       THREAD_STACK_ADDRESS_NONE,
                       THREAD_STACK_SIZE_LOW,
                       (t_vaddr)example_thread,
                       &thread) != STATUS_OK)
      CORE_ESCAPE("unable to reserve a thread");

    if (thread_start(thread) != STATUS_OK)
      CORE_ESCAPE("unable to start the thread");

    if (cpu_current(&cpu) != STATUS_OK)
      CORE_ESCAPE("unable to retrieve the current CPU");

    if (scheduler_start(cpu) != STATUS_OK)
      CORE_ESCAPE("unable to start the scheduler");

    if (event_enable() != STATUS_OK)
      CORE_ESCAPE("unable to enable the events");

    if (event_disable() != STATUS_OK)
      CORE_ESCAPE("unable to disable the events");

    CORE_LEAVE();
  }
\end{verbatim}

%
% advices
%

\section{Advices}

This section contains advices that students are welcome to consider:

\begin{itemize}
  \item
    Students should take care to update the thread's context location
    \ie{} \code{thread->machine.context} whenever entering in the interrupt
    handler in order for the kernel to access the interrupted thread's context
    through the \code{architecture\_context\_set()} and
    \code{architecture\_context\_get()}.
  \item
    The \name{TSS - Task State Segment} should be updated appropriately
    according to the nature of the thread to be resumed.
  \item
    Students should remember to update the registers whenever switching
    from an environment to another such as from the thread's to the kernel's
    or the other way around.
  \item
    For the purpose of the \code{scheduler\_yield()} and
    \code{scheduler\_elect()} which must ensure that a thread is
    elected, even though there may be no thread left, students could
    very well introduce a special thread.
\end{itemize}
