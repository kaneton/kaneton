%
% ---------- header -----------------------------------------------------------
%
% project       kaneton
%
% license       kaneton
%
% file          /home/mycure/kaneton/view/book/assignments/snapshot.tex
%
% created       julien quintard   [fri may 23 21:47:38 2008]
% updated       julien quintard   [mon apr 11 19:45:41 2011]
%

%
% ---------- snapshot ---------------------------------------------------------
%

\chapter{Snapshot}
\label{chapter:snapshot}

This chapter introduces the snapshot which provides students everything
necessary for starting the development of the \name{kaneton} microkernel.

\newpage

%
% ---------- text -------------------------------------------------------------
%

Although \name{k0} is a standalone project, the following stages, namely,
\name{k1}, \name{k2} and \name{k3} take place within the microkernel snapshot.
In addition, students require the educational snapshot in order to submit
their implementation. Thus, the student should download the educational
snapshot from the kaneton website \location{http://kaneton.opaak.org}. Once
downloaded, students should read the dedicated article on the \name{Opaak}
\name{Wiki} and follow the steps in order to configure their kaneton
environment.

Note that in order for both the testing and submission processes to work
properly, students must comply with the following rules:

\begin{itemize}
  \item
    The test and submission procedures rely on a mechanism which essentially
    consists in archiving the whole kaneton development directoy, cleaning it
    from the object files and versoning directories such as \location{.svn/},
    \location{.git/} \etc{} before compressing it and sending it to the server.

    \-

    Once on the server, the snapshot is installed on a small disk image and
    passed to a specific virtual machine whose purpose is to compile the
    kaneton microkernel, generating a bootable image. Thus, the snapshot
    provided to the server should be small enough, around $400$KB, in order
    to fit on the disk image.

    \-

    Students should therefore make sure to avoid placing needless data in
    the development directory such as architecture manuals, copies of
    the whole source code etc. Students can check the snapshot size of their
    current development environment through the following command:

    \begin{verbatim}
      $> cd kaneton/export/
      $kaneton/export> make export-test:group
      [...]
      $kaneton/export> ls -l output/
      total 404K
      -rw------- 1 mycure mycure 398K Mar  7 13:40 test:group.tar.bz2
      $kaneton/export> 
    \end{verbatim}

    In this example, the snapshot has a size of $398$KB. Note that every
    snapshot exceeding the size of $5$MB will for sure generate an error
    once tested or evaluted.
  \item
    The \name{test} user profile located in
    \location{environment/profile/user/} should never be removed as
    it is used by the testing and evaluation procedures.
\end{itemize}

Finally, note that a test capability must first be acquired in order to be
able to use the test and evaluation system. The following assumes that
students possess such a capability. If not, he or he should contact the
kaneton contributors by email at:

\begin{center}
  \location{admin@opaak.org}
\end{center}

%
% test
%

\section{Test}

The kaneton test system enables students to test their implementation against
a set of test suites. However, the number of test suites students are allowed
to request is limited. As such, the kaneton test system should be seen more as
a validation system. Students are therefore invited to implement their own
test suite in order to thorougly test their kaneton implementation.

The following command shows how to retrieve information regarding the
student test profile.

\begin{verbatim}
  $> cd test/client/
  $test/client> make information
  [+] configuration:
  [+]   server:                 https://test.opaak.org:8421
  [+]   capability:             /home/chiche/kaneton/environment/profile/user/chiche/chiche.cap
  [+]   platform:               ibm-pc
  [+]   architecture:           ia32/educational

  [+] information:
  [+]   profile:
  [+]     attributes:
  [+]     identifier:           chiche
  [+]     type:                 student
  [+]     members:
  [+]       name:               Florent Chichery
  [+]       email:              chiche@epita.fr
  [+]   suites:
  [+]     k3:                   This test suite contains tests related to the execution.
  [+]     k2:                   This test suite focuses on the memory management.
  [+]     k1:                   This test suite focuses on the event processing.
  [+]   stages:
  [+]     k3:                   This stage evaluates the kaneton's execution functionalities.
  [+]     k2:                   This stage evaluates the kaneton's memory management.
  [+]     k1:                   This stage evaluates the kaneton's event procesing capabilities.
  [+]   environments:
  [+]     xen:                  The 'xen' environment is used to thoroughly test a kaneton implementation in a Xen hardware-assisted virtual machine.
  [+]     qemu:                 The 'qemu' environment is used to test a kaneton implementation through the QEMU processor emulator.
  [+]   database:
  [+]     reports:
  [+]       xen:
  [+]         ibm-pc.ia32/educational:
  [+]           k3:
  [+]                           20110206:113544
  [+]                           20110206:114823
  [+]           k2:
  [+]                           20110205:222312
  [+]           k1:
  [+]                           20110205:211847
  [+]                           20110205:213502
  [+]                           20110205:214408
  [+]     quotas:
  [+]       xen:
  [+]         ibm-pc.ia32/educational:
  [+]           k3:             3
  [+]           k2:             3
  [+]           k1:             3
  [+]   deliveries:
  [+]     k1:                   None
  [+]     k2:                   None
  [+]     k3:                   None
  $test/client> 
\end{verbatim}

Finally, the following command illustrates a student triggering a test
suite on the \name{Xen} environment.

\begin{verbatim}
  $test/client> make test-xen::k1
  [+] configuration:
  [+]   server:                 https://test.opaak.org:8421
  [+]   capability:             /home/chiche/kaneton/environment/profile/user/chiche/chiche.cap
  [+]   platform:               ibm-pc
  [+]   architecture:           ia32/educational

  [+] generating the kaneton snapshot
  [+] loading the kaneton snapshot
  [+] requesting the server
  [+] the snapshot has been scheduled for testing under the identifier: 20110208:175055
  $test/client> 
\end{verbatim}

Once the testing complete, the student will receive an email containing
a detailed report.

%
% submission
%

\section{Submission}

In order to submit an implementation for evaluation, the student simply
has to issue the \name{submit} command, as shown below.

\begin{verbatim}
  $test/client> make submit-k2
  [+] configuration:
  [+]   server:                 https://test.opaak.org:8421
  [+]   capability:             /home/chiche/kaneton/environment/profile/user/chiche/chiche.cap
  [+]   platform:               ibm-pc
  [+]   architecture:           ia32/educational

  [+] generating the kaneton snapshot
  [+] loading the kaneton snapshot
  [+] requesting the server
  [+] the snapshot has been submitted successfully
  $test/client> 
\end{verbatim}

Note that any submitted snapshot must comply with the rules related to the
testing procedure since the evaluation process is essentially the same.

%
% cheating
%

\section{Cheating}

Students should already be familiar with the fact that cheating is
strictly prohibited. As such, should the implementation of a function,
an algorithm or even a test look too similar to another student's, the
teachers will be forced to take measures ranging from a $-42$ grade for
the current stage to the definitive exclusion of the student from the
project.
