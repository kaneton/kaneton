%
% ---------- header -----------------------------------------------------------
%
% project       kaneton
%
% license       kaneton
%
% file          /home/mycure/kaneton/view/book/assignments/snapshot.tex
%
% created       julien quintard   [fri may 23 21:47:38 2008]
% updated       julien quintard   [thu feb 10 21:18:46 2011]
%

%
% ---------- snapshot ---------------------------------------------------------
%

\chapter{Snapshot}
\label{chapter:snapshot}

This chapter introduces the snapshot which provides students everything
necessary for starting the development of the \name{kaneton} microkernel.

\newpage

%
% ---------- text -------------------------------------------------------------
%

Although \name{k0} is a standalone project, the following stages, namely,
\name{k1}, \name{k2} and \name{k3} take place within the microkernel snapshot.
In addition, students require the educational snapshot in order to submit
their implementation. Thus, the student should download the educational
snapshot from the kaneton website \location{http://kaneton.opaak.org}. Once
downloaded, students should read the dedicated article on the \name{Opaak}
\name{Wiki} and follow the steps in order to configure their kaneton
environment.

%
% test
%

\section{Test}

The kaneton test system enables students to test their implementation against
a set of test suites. However, the number of test suites students are allowed
to request is limited. As such, the kaneton test system should be seen more as
a validation system. Students are therefore invited to implement their own
test suite in order to thorougly test their kaneton implementation.

A test capability must first be acquired in order to use the test system. The
following assumes that students possess such a capability. If not, he or
she should contact the kaneton contributors by email at:

\begin{center}
  \location{admin@opaak.org}
\end{center}

The following command shows how to retrieve information regarding the
student test profile.

\begin{verbatim}
  $> cd test/client/
  $test/client> make information
  [+] configuration:
  [+]   server:                 https://test.opaak.org:8421
  [+]   capability:             /home/chiche/kaneton/environment/profile/user/chiche/chiche.cap
  [+]   platform:               ibm-pc
  [+]   architecture:           ia32/educational

  [+] information:
  [+]   profile:
  [+]     attributes:
  [+]     identifier:           chiche
  [+]     type:                 student
  [+]     members:
  [+]       name:               Florent Chichery
  [+]       email:              chiche@epita.fr
  [+]   suites:
  [+]     k3:                   This test suite contains tests related to the execution.
  [+]     k2:                   This test suite focuses on the memory management.
  [+]     k1:                   This test suite focuses on the event processing.
  [+]   stages:
  [+]     k3:                   This stage evaluates the kaneton's execution functionalities.
  [+]     k2:                   This stage evaluates the kaneton's memory management.
  [+]     k1:                   This stage evaluates the kaneton's event procesing capabilities.
  [+]   environments:
  [+]     xen:                  The 'xen' environment is used to thoroughly test a kaneton implementation in a Xen hardware-assisted virtual machine.
  [+]     qemu:                 The 'qemu' environment is used to test a kaneton implementation through the QEMU processor emulator.
  [+]   database:
  [+]     reports:
  [+]       xen:
  [+]         ibm-pc.ia32/educational:
  [+]           k3:
  [+]                           20110206:113544
  [+]                           20110206:114823
  [+]           k2:
  [+]                           20110205:222312
  [+]           k1:
  [+]                           20110205:211847
  [+]                           20110205:213502
  [+]                           20110205:214408
  [+]     quotas:
  [+]       xen:
  [+]         ibm-pc.ia32/educational:
  [+]           k3:             3
  [+]           k2:             3
  [+]           k1:             3
  [+]   deliveries:
  [+]     k1:                   None
  [+]     k2:                   None
  [+]     k3:                   None
  $test/client> 
\end{verbatim}

Finally, the following command illustrates a student triggering a test
suite on the \name{Xen} environment.

\begin{verbatim}
  $test/client> make test-xen::k1
  [+] configuration:
  [+]   server:                 https://test.opaak.org:8421
  [+]   capability:             /home/chiche/kaneton/environment/profile/user/chiche/chiche.cap
  [+]   platform:               ibm-pc
  [+]   architecture:           ia32/educational

  [+] generating the kaneton snapshot
  [+] loading the kaneton snapshot
  [+] requesting the server
  [+] the snapshot has been scheduled for testing under the identifier: 20110208:175055
  $test/client> 
\end{verbatim}

Once the testing complete, the student will receive an email containing
a detailed report.

%
% submission
%

\section{Submission}

In order to submit an implementation for evaluation, the student simply
has to issue the \name{submit} command, as shown below.

\begin{verbatim}
  $test/client> make submit-k2
  [+] configuration:
  [+]   server:                 https://test.opaak.org:8421
  [+]   capability:             /home/chiche/kaneton/environment/profile/user/chiche/chiche.cap
  [+]   platform:               ibm-pc
  [+]   architecture:           ia32/educational

  [+] generating the kaneton snapshot
  [+] loading the kaneton snapshot
  [+] requesting the server
  [+] the snapshot has been submitted successfully
  $test/client> 
\end{verbatim}
