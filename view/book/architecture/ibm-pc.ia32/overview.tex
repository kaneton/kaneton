%
% ---------- header -----------------------------------------------------------
%
% project       kaneton
%
% license       kaneton
%
% file          /home/buckman/cry...view/book/architecture/ia32/overview.tex
%
% created       matthieu bucchianeri   [sat sep  2 11:42:35 2006]
% updated       matthieu bucchianeri   [tue oct  2 16:13:04 2007]
%

%
% overview
%

\chapter{Overview}

The IA-32 architecture, often called x86, x86-32 or i386, is one of
the most successful CISC microprocessor architecture developed by
Intel and mainly used by Intel and AMD though the 80x86, Pentium, K6
and K7 processors.

Processors of x86 architecture support backward compatibility with
older 16-bits microprocessors, offer multiple privileged or
unprivileged execution mode, provide full MMU implementation of
segmentation and paging, permit customizable interrupts catching and
are shipped with an internal floating point unit.

Through the years and the different microarchitectures and
innovations, the Pentium series from Intel has become the most
suitable for home use, workstation and servers.

In contrast with other microprocessors like MIPS or SPARC, only a few
platform's architectures are based on IA-32 processors. The most known
is the IBM-PC, derived from IBM AT, XT and PS/2. An IBM-PC platform is
composed of one x86 CPU, two interrupt controllers, two direct memory
access controllers, one CMOS memory chip and one programmable timer.

The educational version of kaneton for IA-32 on IBM-PC takes advantage
of the CPU's running modes, the MMU, the FPU, the interrupt controller
and the timer. The goal of this port is to be as simple as
possible. For this reason, only the basic mecanisms of memory
management, context switching and interrupt management are
implemented.

Later, a more complete version using all IA-32 features will be
release under the name ``IA-32 Optimised''.
