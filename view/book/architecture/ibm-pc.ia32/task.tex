%
% ---------- header -----------------------------------------------------------
%
% project       kaneton
%
% license       kaneton
%
% file          /home/buckman/kaneton/view/book/architecture/ia32/task.tex
%
% created       matthieu bucchianeri   [sat sep  2 11:41:15 2006]
% updated       matthieu bucchianeri   [thu aug 16 17:49:33 2007]
%

%
% task management
%

\chapter{Task management}

This chapter describes all the IA-32 specific mecanisms used in the
thread, task and scheduler managers.

These mecanisms mainly includes the manipulation of contexts. The
context is the whole state of the microprocessor. The machine
dependent operations consists in creating, maintaining and changing
the contexts.

\newpage

%
% overview of execution contexts on ia-32
%

\section{Overview of execution contexts on IA-32}

%
% reserving a task
%

\section{Reserving a task}

%
% setting up the execution context
%

\section{Setting up the execution context}

%
% preparing a new thread
%

\section{Preparing a new thread}

%
% scheduling and context switch
%

\section{Scheduling and context switch}

The thread's scheduling involves many mecanisms and operations on
contexts. The following sections describes the manipulations of
contexts and the context switching as implemented in kaneton for
IA-32.

%
% context saving and restoring
%

\subsection{Context saving and restoring}

Saving the context is done only on interrupt or exception. Restoring
the context is done only when resuming from such event.

The code of the low-level interrupt and exception handlers must be
generated using one of these four macro functions:

\begin{itemize}
\item \textit{EXCEPTION\_PREHANDLER\_CODE}, for exceptions with error
code appended by the microprocessor ;
\item \textit{EXCEPTION\_PREHANDLER\_NOCODE}, for exceptions without
any error code ;
\item \textit{INTERRUPT\_PREHANDLER}, for interrupts ;
\item \textit{IRQ\_PREHANDLER}, for interrupts with PIC aknowlegment.
\end{itemize}

The macro functions generates the context management code, which
skeleton is:

\begin{itemize}
\item Saving the context, \textit{SAVE\_CONTEXT()} ;
\item Handling ``from-Ring 0'' switch ;
\item Calling the higher-level handler ;
\item Handling ``to-Ring 0'' switch-back ;
\item Restoring the context, \textit{RESTORE\_CONTEXT()} ;
\item Returning from interrupt or exception.
\end{itemize}

The ``from-Ring 0'' and ``to-Ring 0'' steps are explained in the
section \textbf{Ring 0 to Ring 0 switch}.

The \textit{SAVE\_CONTEXT} step makes successive \textit{push} of all
general purpose registers onto the stack, plus the CR3 register. Then,
it changes the value of the page-directory base register with kernel
page-directory base (stored by the kernel in the
\textit{interrupt\_pdbr} variable). To finish, it pushes the segment
selectors onto the stack and set the kernel segments (variable
\textit{interrupt\_ds}).

\textit{RESTORE\_CONTEXT} does the opposite: it pops every registers
pushed on the spack.

The code of the generated handlers is then placed into the
\textit{.handler} ELF section, which will be mapped in every address
space.

We'll talk about the FPU and SSE contexts in the next section.

%
% context switching
%

\subsection{Context switching}

The context of a thread is stored onto the kernel stack of the thread
when an exception or an interrupt occurs. When resuming, the context
at the top of the stack is used to resume execution.

Accessing the context for reading is quite simple: the
\textit{context} variable points the place where the context has been
stored. Editing the context of the running thread is also done
\textit{via} this variable.

Therefore, the implementation of \textit{sched\_switch} dependent code
consists only in few steps:

\begin{enumerate}
\item Saving back the context of the currently executing thread ;
\item Copying the context of the newly executed thread on the stack ;
\item Copying the permission bitmap of the task into the active TSS
(see \textbf{I/O management} for details about the permission bitmap).
\end{enumerate}

The code is incrediblely short: the \textit{context\_copy} function is
used for both steps 1 and 2. A simple \textit{memcpy} is used for step
3.

The context switching of FPU and extended registers (SSE and SSE2) is
implemented using IA-32 facilities: when the TS bit of CR0 is set, an
exception is thrown each time a floating or extension instruction is
executed.

Thus, switching from one thread to another sets the TS bit to 1. If
the thread executes no special instruction, the FPU \& extended context
stay unchanged. If the thread executes such instruction, an exception
is thrown. The exception handler (\textit{ia32\_sched\_switch\_mmx}):

\begin{enumerate}
\item Save the FPU \& extended register set into the \textit{mmx\_context}
thread ;
\item Restore the FPU \& extended context of the currently executing thread ;
\item Clear the TS flag ;
\item Set \textit{mmx\_context} to the current thread id.
\end{enumerate}

\textit{mmx\_context} is a scheduler variable that contains the
identifier of the last thread that executed a FPU or extension
instruction.

Saving and restoring the FPU \& extended context is done using
dedicated instructions: \textit{fsave} and \textit{frstor} when no SSE
extensions are present, \textit{fxsave} and \textit{fxrstor} when SSE
is supported.

%
% ring 0 to ring 0 switch
%

\subsection{Ring 0 to Ring 0 switch}

On interrupts and exceptions, the microprocessor enters in supervisor
state (also known as ring 0) and jumps to the appropriated event
handler.

On IA-32 architecture, these operations are preceded by a stack
switch: the ESP and SS registers ared loaded with the values extracted
from the TSS (ESP0 and SS0).

But this stack switch only occurs when the microprocessor is not
running on supervisor mode (when it is running in ring 1, 2 or
3). This leads to a special case when an interrupt or exception
happend in ring 0: this stack switching (needed for the whole context
switching) must be done by hand.

The \textit{FORCE\_RING0\_SWITCH} macro function is used to perform
the stack switching when coming from ring 0. The destination stack is
pointed by \textit{interrupt\_stack} (filled by the kernel). As this
process uses a few registers, it must be executed after the context
switching, and then it must copy the saved context to the destination
stack (using \textit{memcpy}).

When resuming from an interrupt or exception, the
\textit{FORCE\_RING0\_BACK} macro function performs a stack switch
when the resuming thread runs in ring 0. As described previously it
copies the content of the stack to the destination stack, which is
determined by the value of ESP in the new context.

%
% address space switching
%

\subsection{Address space switching}

The address space switching was implicitely described in previous
sections: as the only thing to do in order to switch the current
address space to another one is changing the page-directory base
register's value, this is done when saving and restoring the
registers.

The CR3 register is saved and restored the same way as EAX or
EBX\ldots{} No additional operation is required, since loading the CR3
register involves flushing the translation caches.
