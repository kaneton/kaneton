%
% ---------- header -----------------------------------------------------------
%
% project       kaneton
%
% license       kaneton
%
% file          /home/enguerrand/...ew/book/architecture/MIPS64/platform.tex
%
% created       enguerrand raymond   [mon oct 13 11:31:56 2008]
% updated          [fri apr 10 00:51:37 2009]
%

%
% platform
%

\chapter{MIPS64 platforms}

This chapter describes the different platform for which kaneton has been developed. These platforms are described following you need to understand the current document.


\subsection{qemu-mips platform}

\subsubsection{Description}

In 2008 We did not have MIPS machine enough in EPITA laboratories. We have decided to develop the first MIPS64 kaneton
port on MIPS64 emulator only. The student must already develop on ia32 architecture and they must execute their kernel
on real machine. So, it is not necessary to have a real MIPS 64 bits machine.

The qemu-mips platform is the simplest emulated platform to develop kernel for MIPS 64 bits architecture. So, we have decided to develop firstly on this platform. It is composed of the following devices :

\begin{itemize}
  \item
    \textbf{A big range of MIPS64 microprocessor}
  \item
    \textbf{IBM PC style serial port}
  \item
    \textbf{IBM PC style IDE disk}
  \item
    \textbf{NE2000 network card}
\end{itemize}
 
\subsubsection{Emulation launching}

Qemu-mips does not have any firmware (on MIPS platform the bios is called a firmware).

We use the qemu-mips platform with the R4000 microprocessor. The command line to launch him it the following :

\begin{verbatim}
  qemu-system-mips64el [-L path ot bios] -fda kaneton.img -M mips -cpu R4000
\end{verbatim}

There is no mips firmware (or bios for qemu) today but we will launch the kernel as it. It will load at the address \code{0xFFFFFFFFBFC00000}. To do that, the kernel must be copied to \code{mips\_bios.bin} and the \code{L} option must indicate the path to this file. We will develop a firmware later.
The bootloader will be executed like a firmware.

\subsubsection{Emulation platform warning}

The emulators do not respect perfectly the real architecture. Moreover, the qemu-mips platform has not
real equivalent and so platform reference. But, the most important is the microprocessor that can have bug behaviour
and false quality, for example the status register is always well set after reset contrarily to the documentation that specifies the content of this register
is undefined except for four bits.
