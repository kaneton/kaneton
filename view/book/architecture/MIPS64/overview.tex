%
% ---------- header -----------------------------------------------------------
%
% project       kaneton
%
% license       kaneton
%
% file          /home/enguerrand/...ew/book/architecture/MIPS64/overview.tex
%
% created       enguerrand raymond   [sun oct  5 16:54:35 2008]
% updated       enguerrand raymond   [sun nov  9 10:21:50 2008]
%

%
% overview
%

\chapter{Overview}

The MIPS64 kaneton implementation is born two purchase three main goals.
The first one is the kaneton kernel architecture portability system test,
the second is the development for 64 bits architecture and the third is pedagogical.
In fact, the EPITA students must follow MIPS architecture lessons, so it is interesting 
for them to develop on MIPS architecture concurently to the kaneton microkernel development
project.

The MIPS 64 bits kaneton port gathers several microprocessors. The first of them is the MIPS R4000. In the following chapters, subsection will describe the MIPS R4000 development specification.

This document is composed of X parts, each part corresponds to one or several source files. The memory management for example is developed in Region dependant code and MIPS64 library. For each chapter, the source files concerned will be given.
