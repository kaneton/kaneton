%
% ---------- header -----------------------------------------------------------
%
% project       kaneton
%
% license       kaneton
%
% file          /home/buckman/kan...view/book/architecture/ia32/syscalls.tex
%
% created       matthieu bucchianeri   [tue aug 14 20:10:52 2007]
% updated       matthieu bucchianeri   [thu aug 16 19:33:03 2007]
%

%
% system calls
%

\chapter{System calls}

This chapter describes the implementation details of the low-level
system calls used in the message manager.

\newpage

%
% software interrupts and register parameter-passing
%

\section{Software interrupts and register parameter-passing}

The IA-32 microprocessors offers two facilities to implement system
calls:

\begin{itemize}
  \item
    The software interrupts with the \textit{INT} instruction;
  \item
    The \textit{SYSENTER} and \textit{SYSEXIT} instructions.
\end{itemize}

We use the first technique as it is widely used and supported on any
IA-32 microprocessor.

The fastest way to pass parameters to a system call is to use the
registers. We can pass 32 bits per register. As the message primitives
take many arguments, it is difficult to put all of them in the
registers. For this reason, we have to economize one register by
associating one interrupt per system call.

%
% kernel-side code
%

\subsection{Kernel-side code}

The function \textit{ia32\_syscalls\_init()} calls
\textit{event\_reserve()} for each system call as defined in the table
in the \textit{System calls map} section.

In every handler, we use the interrupted context pointer to get the
register values and to set the output values.

\begin{verbatim}
ia32_syshandler_register()
{
  type_1    arg1;
  type_2    arg2;
  t_status   res;

  arg1 = ia32_context->eax;
  arg2 = ia32_context->ebx;

  res = function(arg1, arg2);

  ia32_context->eax = res;
}
\end{verbatim}

%
% user-side code
%

\subsection{User-side code}

The user-side system call code put the arguments in registers, call
the corresponding software interrupt and get back the result(s) from
registers.

\begin{verbatim}
syscall(arg1, arg2)
{
  eax <- arg1
  ebx <- arg2

  int 56

  res <- eax

  return (res);
}
\end{verbatim}

%
% system calls map
%

\section{System calls map}

As explained just above, one software interrupt slot is allocated for
each system call. Here is the table of these indexes, the associated
system call and the register mapping of the parameters.

\begin{center}
\begin{tabular}{| c | c | c |}
\hline
Soft-int & Syscall           & Registers \\
\hline
56       & message\_register & \textit{eax} = type; \textit{ebx} = size \\
\hline
57       & message\_send     & \textit{eax,ebx,ecx,edx} = node; \textit{esi} = type; \textit{edi} = data; \textit{ebp} = size \\
\hline
58       & message\_transmit & \textit{eax,ebx,ecx,edx} = node; \textit{esi} = type; \textit{edi} = data; \textit{ebp} = size \\
\hline
59       & message\_throw    & XXX \\
\hline
60       & message\_receive  & \textit{eax} = type; \textit{ebx} = data; \textbf{output}: \textit{ebx} = size; \textit{ecx,edx,esi,edi} = node \\
\hline
61       & message\_grab     & XXX \\
\hline
62       & message\_poll     & \textit{eax} = type; \textit{ebx} = data; \textbf{output}: \textit{ebx} = size; \textit{ecx,edx,esi,edi} = node \\
\hline
63       & message\_state    & XXX \\
\hline
64       & message\_wait     & XXX \\
\hline
\end{tabular}
\end{center}

The register \textit{eax} is always used as the result output register
(of type \textit{t\_error}).

%
% upcalls
%

\section{Upcalls}

XXX