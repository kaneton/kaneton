%
% ---------- header -----------------------------------------------------------
%
% project       kaneton
%
% license       kaneton
%
% file          /home/buckman/cry...neton/view/book/architecture/ia32/io.tex
%
% created       matthieu bucchianeri   [sat sep  2 11:39:52 2006]
% updated       matthieu bucchianeri   [mon oct  1 20:04:37 2007]
%

%
% io management
%

\chapter{I/O management}

This part provides information about input and output control under
kaneton. These kind of operations should be performed only by driver
class tasks, but kaneton also offers the possibility to classical
non-privileged programs to communicate with hardware via I/O (though
capability checking).

\newpage

%
% driver class tasks
%

\section{Driver class tasks}

Performing I/O on IA-32 is done through special instructions
\textit{IN} and \textit{OUT} or through memory-mapped I/O.

With the I/O specialized instruction set, two check are performed to
ensure the process has the right to execute such operation:

\begin{itemize}
\item
  The \textit{IOPL} field of the \textit{eflags} register is compared
  with the current privilege level. If the latter is more or equaly
  privileged than the \textit{IOPL}, the operation is granted;
\item
  Otherwise, the microprocessor check the I/O Permission Bitmap. This
  bitmap associates one bit to one port and is located in the current
  TSS. When the n\textit{th} bit of the bitmap is set, I/O operations
  to the n\textit{th} port are granted.
\end{itemize}

For drivers, the \textit{ia32\_init\_context} function sets the
\textit{IOPL} to 1, which is the running privilege level for driver
class tasks. Thus, drivers can access all I/O ports in the whole
system with very good performances (no need to check the IO Bitmap in
memory).

In other cases, \textit{IOPL} is set to 0 (no privileges for I/O).

%
% allowing or denying ios to userland
%

\section{Allowing or denying I/Os to userland}

For userland processes (of class service or program), accessing I/O
ports requires the capability to \textit{io\_grant} the given port.

The machine-dependent code of this operation calls the
\textit{ia32\_set\_io\_bitmap} function which sets or clears the
needed bit in the IO Permission Bitmap of the task. This is possible
because this bitmap is part of the \textit{o\_task} machine-dependent
struct.

But the context switching is slown down because the I/O Permission
Bitmap is 8192 bytes large and needs to be copied into the current TSS
on task switch.
