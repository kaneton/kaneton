%
% ---------- header -----------------------------------------------------------
%
% project       kaneton
%
% license       kaneton
%
% file          /home/buckman/cry...on/view/book/architecture/ia32/timer.tex
%
% created       matthieu bucchianeri   [tue aug 14 20:09:50 2007]
% updated       matthieu bucchianeri   [tue oct  2 16:14:31 2007]
%

%
% timer management
%

\chapter{Timer management}

This chapter describes the architecture-dependent code driving the
kaneton timer manager.

\newpage

%
% fixed frequency interrupts
%

\section{Fixed frequency interrupts}

The kaneton timer manager requires the function
\textit{timer\_handler()} to be called every
\textit{TIMER\_MS\_PER\_TICK} milliseconds.

There are three components able to throw an interrupt at fixed rate:

\begin{itemize}
  \item
    The microprocessor itself, through the local APIC;
  \item
    The Programmable Interval Timer (PIT);
  \item
    The Real Time Clock (RTC).
\end{itemize}

The first solution is the most accurate but needs to be calibrated,
which is a more complex process. The last solution is also very
accurate but the available frequencies need floating point computing
to be millisecond-accurate.

To keep the code as simple as possible, we will use the PIT, which is
a very easily programmable component.

The PIT is connected to a fixed frequency oscillator of 1193180
Hz. The component act as a counter/decounter. We use the mode 2: a
value is loaded in the counter register and then decremented on each
clock tick. Once the counter register reaches zero, an interrupt is
signaled and the counter is reset to its initially loaded value.

The interrupt pin is connected to the IRQ line 0.

The very short function \textit{ibmpc\_timer\_init()} setup the PIT
channel 0 at mode 2 with an initial counter register computed as
follow:

$$
latch = \frac{IBMPC\_CLOCK\_TICK\_RATE}{\frac{1000}{TIMER\_MS\_PER\_TICK}}
$$

Where \textit{IBMPC\_CLOCK\_TICK\_RATE} is the fixed oscillator
frequency (1193180 Hz).

To finish, the glue code of \textit{timer\_initialize()} hooks the IRQ
0 to the function \textit{timer\_handler()}.
