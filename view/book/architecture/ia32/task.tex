%
% ---------- header -----------------------------------------------------------
%
% project       kaneton
%
% license       kaneton
%
% file          /home/buckman/kaneton/view/book/architecture/ia32/task.tex
%
% created       matthieu bucchianeri   [sat sep  2 11:41:15 2006]
% updated       matthieu bucchianeri   [sat dec 22 23:11:16 2007]
%

%
% task management
%

\chapter{Task management}

This chapter describes all the IA-32 specific mecanisms used in the
thread, task and scheduler managers.

These mecanisms mainly includes the manipulation of contexts. The
context is the whole state of the microprocessor. The machine
dependent operations consists in creating, maintaining and changing
the contexts.

\newpage

%
% reserving a task
%

\section{Reserving a task} % XXX

%
% preparing a new thread
%

\section{Preparing a new thread} % XXX

%
% scheduling and context switch
%

\section{Thread switching}

The thread's scheduling involves many mecanisms and operations on
contexts. The following sections describes the manipulations of
contexts and the context switching as implemented in kaneton for
IA-32.

%
% context saving and restoring
%

\subsection{Context saving and restoring}

Saving the context is done only on interrupt or exception. Restoring
the context is done only when resuming from such event.

The code of the low-level interrupt and exception handlers must be
generated using one of these four macro functions:

\begin{itemize}
\item \textit{IA32\_EXCEPTION\_PREHANDLER\_CODE}, for exceptions with error
code appended by the microprocessor ;
\item \textit{IA32\_EXCEPTION\_PREHANDLER\_NOCODE}, for exceptions without
any error code ;
\item \textit{IA32\_INTERRUPT\_PREHANDLER}, for interrupts ;
\item \textit{IA32\_IRQ\_PREHANDLER}, for interrupts with PIC aknowlegment ;
\item \textit{IA32\_SYSCALL\_PREHANDLER}, for system calls.
\end{itemize}

The macro functions generates the context management code, which
skeleton is:

\begin{itemize}
\item Switching to the thread's kernel stack (hardware performed) ;
\item Saving the context, \textit{IA32\_SAVE\_CONTEXT()} ;
\item Switching to the kernel interrupt stack ;
\item Calling the higher-level handler ;
\item Switching to the resuming thread's kernel stack ;
\item Restoring the context, \textit{IA32\_RESTORE\_CONTEXT()} ;
\item Returning from interrupt or exception.
\end{itemize}

The \textit{IA32\_SAVE\_CONTEXT} step makes a \textit{pusha} to save
all the general purpose registers and then pushes the data segment
register. Then, it changes the value of the page-directory base
register with kernel page-directory base (stored by the kernel in the
\textit{ia32\_interrupt\_pdbr} variable). Next it sets the kernel
segments (variable \textit{ia32\_interrupt\_ds}). To finish, it saves
the current stack pointer to \textit{ia32\_local\_jump\_stack} before
jumping to the interrupt dedicated stack
(\textit{ia32\_local\_interrupt\_stack}).

\textit{RESTORE\_CONTEXT} does the opposite: it switches back to the
stack \textit{ia32\_local\_jump\_stack}, it pops every registers
pushed on the stack and then switches address space to the one stored
in \textit{ia32\_local\_jump\_pdbr}.

The code of the generated handlers is then placed into the
\textit{.handler} ELF section, which will be mapped in every address
space.

We'll talk about the FPU and SSE contexts in the section
\textbf{Context switching}.

%
% accessing a thread's context
%

\subsection{Accessing a thread's context}

As the execution context of one thread is stored on its kernel stack,
it is impossible to access it directly since this kernel stack is only
mapped into the thread's address space.

Accessing a thread's context requires calling the \textit{as\_read}
and \textit{as\_write} operations.

The function \textit{ia32\_get\_context} retrieve the address of a
thread's kernel stack and uses \textit{as\_read} to read the stored
context. The function \textit{ia32\_set\_context} does the opposite,
except you can specify a mask of registers to update in the thread's
context. This requires to read the context and then to rewrite only
the masked registers, which is slower but easier for the programer.

%
% context switching
%

\subsection{Context switching}

The context of a thread is stored onto the kernel stack of the thread
when an exception or an interrupt occurs. When resuming, the context
at the top of the stack is used to resume execution.

Therefore, the implementation of \textit{sched\_switch} dependent code
consists only in few steps:

\begin{enumerate}
\item Setting the \textit{ia32\_local\_jump\_pdbr} variable to the
PDBR of the resuming thread ;
\item Setting the \textit{ia32\_local\_jump\_stack} variable to the
address of the kernel stack of the resuming thread ;
\item Setting up the TSS to switch stack to the thread's kernel stack
on interrupt ;
\item Copying the permission bitmap of the task into the active TSS
(see \textbf{I/O management} for details about the permission bitmap).
\end{enumerate}

The context switching of FPU and extended registers (SSE and SSE2) is
implemented using IA-32 facilities: when the TS bit of CR0 is set, an
exception is thrown each time a floating or extension instruction is
executed.

Thus, switching from one thread to another sets the TS bit to 1. If
the thread executes no special instruction, the FPU \& extended context
stay unchanged. If the thread executes such instruction, an exception
is thrown. The exception handler (\textit{ia32\_extended\_context\_switch}):

\begin{enumerate}
\item Save the FPU \& extended register set into the \textit{mmx\_context}
thread ;
\item Restore the FPU \& extended context of the currently executing thread ;
\item Clear the TS flag ;
\item Set \textit{mmx\_context} to the current thread id.
\end{enumerate}

\textit{mmx\_context} is a scheduler variable that contains the
identifier of the last thread that executed a FPU or extension
instruction.

Saving and restoring the FPU \& extended context is done using
dedicated instructions: \textit{fsave} and \textit{frstor} when no SSE
extensions are present, \textit{fxsave} and \textit{fxrstor} when SSE
is supported.

%
% address space switching
%

\subsection{Address space switching}

No additional operation is required, since loading the CR3 register
involves flushing the translation caches.

%
% stack switching switching
%

\subsection{Stack switching}

As explained above, the code executed by the kernel on
interrupt/exception runs on a global kernel stack.

XXX creation + TSS
