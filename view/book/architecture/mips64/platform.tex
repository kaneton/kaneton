%
% ---------- header -----------------------------------------------------------
%
% project       kaneton
%
% license       kaneton
%
% file          /home/enguerrand/...ew/book/architecture/mips64/platform.tex
%
% created       enguerrand raymond   [mon oct 13 11:31:56 2008]
% updated       enguerrand raymond   [wed may 13 17:07:47 2009]
%

%
% platform
%

\chapter{MIPS64 platforms}

This chapter describes the different platform for which kaneton has been developed. These platforms are described following you need to understand the current document.


\subsection{Qemu-mips platform}

\subsubsection{Description}

In 2008 We did not have MIPS machine enough in EPITA laboratories. We have decided to develop the first MIPS64 kaneton
port on MIPS64 emulator only. The student must already develop on ia32 architecture and they must execute their kernel
on real machine. So, it is not necessary to have a real MIPS 64 bits machine.

The qemu-mips platform is the simplest emulated platform to develop kernel for MIPS 64 bits architecture. So, we have decided to develop firstly on this platform. It is composed of the following devices :

\begin{itemize}
  \item
    \textbf{A big range of MIPS64 microprocessor}
  \item
    \textbf{IBM PC style serial port}
  \item
    \textbf{IBM PC style IDE disk}
  \item
    \textbf{NE2000 network card}
\end{itemize}
 
\subsubsection{Emulation launching}

Qemu-mips does not have any firmware (on MIPS platform the bios is called a firmware).

We use the qemu-mips platform with the R4000 microprocessor. The command line to launch him it the following :

\begin{verbatim}
  qemu-system-mips64el [-L path to bios] -fda kaneton.img -M mips -cpu R4000
\end{verbatim}

\subsubsection{Emulation platform warning}

The emulators do not respect perfectly the real architecture. Moreover, the qemu-mips platform has not
real equivalent and so platform reference. But, the most important is the microprocessor that can have bug behaviour
and false quality, for example the status register is always well set after reset contrarily to the documentation that specifies the content of this register is undefined except for four bits.

The qemu-mips plaftorm does not have any firmware. We have begun to develop a simple one to startup the platform. The firmware has no driver to load the kaneton image, but an alternative method described below has been developed.

\subsection{Qemu-mips firmware}

\subsubsection{MIPS64 initialization prerequisites}

At starting, a MIPS64 microprocessor begins to execute instruction at the following virtual addresses :

\begin{itemize}
  \item
    \textbf{0xBFC00000 in 32 bits mode}
  \item
    \textbf{0xFFFFFFFF BFC00000 in 64 bits mode}
\end{itemize}

These virtual addresses take part of an unmapped segment, the address translation is automatic. See the chapter about memory management. These addresses correspond to firmware memory. The memory must content the firmware code, this region is read only.

\subsubsection{Operations}

The firmware is a file loaded by qemu. The file is directly loaded to starting address. For development time reason, the kaneton MBL script concatenates the firmware code, the kaneton bootloader, the kaneton kernel and these modules. Also, the file is loaded in memory with all file needed.
