%
% ---------- header -----------------------------------------------------------
%
% project       kaneton
%
% license       kaneton
%
% file          /home/mycure/kane...view/book/architecture/mips64/mips64.tex
%
% created       enguerrand raymond   [wed oct 01 19:07:46 2008]
% updated       julien quintard   [mon jan 31 14:28:33 2011]
%

%
% ---------- setup ------------------------------------------------------------
%

%
% path
%

\def\path{../../..}

%
% template
%

%%
%% licence       kaneton licence
%%
%% project       kaneton
%%
%% file          /home/mycure/kaneton/view/templates/book.tex
%%
%% created       julien quintard   [wed mar  1 23:45:22 2006]
%% updated       julien quintard   [thu may  4 12:36:54 2006]
%%

%
% class
%

\documentclass[10pt,a4wide]{book}

%
% packages
%

\usepackage[english]{babel}
\usepackage[T1]{fontenc}
\usepackage{a4wide}
\usepackage{fancyheadings}
\usepackage{multicol}
\usepackage{indentfirst}
\usepackage{graphicx}
\usepackage{color}
\usepackage{xcolor}
\usepackage{verbatim}

\usepackage{aeguill}

\usepackage[Lenny]{../../../tools/latex/fncychap}

\pagestyle{fancy}

\setlength{\footrulewidth}{0.3pt}
\setlength{\parindent}{0.3cm}
\setlength{\parskip}{2ex plus 0.5ex minus 0.2ex}

%
% logos
%

\newcommand{\logos}
  {
    \begin{center}
      \includegraphics[scale=0.8]{../../logos/kaneton.pdf}
    \end{center}
  }

%
% colors
%

\definecolor{functioncolor}{rgb}{0.40,0.00,0.00}
\definecolor{commandcolor}{rgb}{0.00,0.00,0.40}
\definecolor{verbatimcolor}{rgb}{0.00,0.40,0.00}
\definecolor{noticecolor}{rgb}{0.87,0.84,0.02}

%
% function
%

\newcommand\function[3]{
  \begin{tabular}{p{0.2cm}p{13.8cm}}
  & {\color{functioncolor}\textbf{#1}}#2
  \end{tabular}

  \begin{tabular}{p{1cm}p{13cm}}
  & #3
  \end{tabular}}

%
% align
%

\newcommand\align[1]{
  \\ & \hspace{#1}}

%
% argument
%

\newcommand\argument[1]{\textit{#1}}

%
% command
%

\newcommand\command[2]{
  \begin{tabular}{p{0.2cm}p{13.8cm}}
  & {\color{commandcolor}\textbf{#1}}
  \end{tabular}

  \begin{tabular}{p{1cm}p{13cm}}
  & #2
  \end{tabular}}

%
% notice
%

\newcommand\notice[1]{
  {\color{noticecolor}\textbf{Notice}}

  \begin{tabular}{p{0.2cm}p{13.8cm}}
  & #1
  \end{tabular}}

%
% example
%

\newcommand\example[1]{
  \textit{Example:}

  \begin{tabular}{p{0.2cm}p{13.8cm}}
  & \textit{#1}
  \end{tabular}}

%
% warning XXX
%

%
% verbatim stuff
%

\makeatletter

\renewcommand{\verbatim@font}
  {\ttfamily\footnotesize\color{verbatimcolor}\selectfont}

\def\verbatim@processline{\hskip15ex\the\verbatim@line\par}

\makeatother

%
% header
%

\rhead{}
\rfoot{\scriptsize{The kaneton microkernel project}}

\date{\scriptsize{\today}}


%
% header
%

\lhead{\scriptsize{The kaneton microkernel :: mips64}}

%
% title
%

\title{The kaneton microkernel :: mips64 \\
       \version
       \logo}

%
% document
%

\begin{document}

%
% title
%

\maketitle

%
% --------- text --------------------------------------------------------------
%

This document describes the kaneton microkernel reference project over
MIPS64 architecture on qemu platform (for educational purpose).

\-

This document should be used by every student willing implement the
kaneton microkernel as people looking for more details on the kaneton
microkernel implementation on MIPS64 architecture. Several microprocessor 
kaneton implementations will be described inside this document.

\-

All the kaneton documents are available on
the official website
  \footnote{\location{http://kaneton.opaak.org}}.

\-

\textbf{Prerequisites}: this document details the kaneton implementation
for MIPS64 architecture on Qemu platform (for educational purpose). A
knowledge of this architecture's basics is strongly recommended as
this document is not intended to be a MIPS64 lesson (see
\textit{Bibliography} for references).

%
% table of contents
%

\tableofcontents

%
% identation
%

\indentation{}

%
% chapters
%

%%
%% licence       kaneton licence
%%
%% project       kaneton
%%
%% file          /home/mycure/kaneton/view/papers/kaneton/overview.tex
%%
%% created       matthieu bucchianeri   [mon jan 30 17:09:45 2006]
%% updated       julien quintard   [thu mar  2 13:12:22 2006]
%%

%
% overview
%

\chapter{Overview}

XXX ce chapitre va vous aider a reconnaitre les fonctionnalites principale
XXX d'un kernel dans kaneton.

The kaneton microkernel is only the core of an operating system.
Main tasks like hardware drivers or user services are implemented as
\textbf{servers}. So the microkernel only has a few functionalities to
provide:

\begin{itemize}
  \item
    Memory management.
  \item
    Process management.
  \item
    Communication.
  \item
    Events.
\end{itemize}

In this chapter we will describe briefly these tasks and all the
associated managers.

%
% memory management
%

\section{Memory Management}

Handling the memory -- from virtual address space to physical
addressing -- is done by three major managers, the \textbf{as},
\textbf{segment} and \textbf{region} managers.

%
% as
%

\subsection{as}

The address space manager just manages the different address spaces
used by the kaneton tasks.

In kaneton, we call an \textbf{as - address space} a list of memory
locations referenced by a task. Each task has its own address space.

%
% segment
%

\subsection{segment}

The segment manager just manages the segments reserved by
the different kaneton entities including the kernel, the drivers etc..

In kaneton terms a \textbf{segment} is a contiguous area of reserved
physical memory.

%
% region
%

\subsection{region}

The region manager keeps track of regions used to map segments for
each address space reserved on the system.

In kaneton, a \textbf{region} is contiguous area of virtual memory
mapping a segment's part.

%
% process management
%

\section{Process Management}

XXX

%
% communication
%

\section{Communication}

XXX

%
% events
%

\section{Events}

XXX

%
% ---------- header -----------------------------------------------------------
%
% project       kaneton
%
% license       kaneton
%
% file          /home/enguerrand/...ew/book/architecture/MIPS64/platform.tex
%
% created       enguerrand raymond   [mon oct 13 11:31:56 2008]
% updated          [mon nov 24 17:24:25 2008]
%

%
% platform
%

\chapter{MIPS64 platforms}

This chapter describes the different platform for which kaneton has been developed. These platforms are described following you need to understand the current document.


\subsection{qemu-mips platform}

\subsubsection{Description}

In 2008 We did not have MIPS machine enough in EPITA laboratories. We have decided to develop the first MIPS64 kaneton
port on MIPS64 emulator only. The student must already develop on ia32 architecture and they must execute their kernel
on real machine. So, it is not necessary to have a real MIPS 64 bits machine.

The qemu-mips platform is the simplest emulated platform to develop kernel for MIPS 64 bits architecture. So, we have decided to develop firstly on this platform. It is composed of the following devices :

\begin{itemize}
  \item
    \textbf{A big range of MIPS64 microprocessor}
  \item
    \textbf{IBM PC style serial port}
  \item
    \textbf{IBM PC style IDE disk}
  \item
    \textbf{NE2000 network card}
\end{itemize}
 
\subsubsection{Emulation launching}

Qemu-mips does not have any firmware (on MIPS platform the bios is called a firmware).

We use the qemu-mips platform with the R4000 microprocessor. The command line to launch him it the following :

\begin{verbatim}
  qemu-system-mips64el [-L path ot bios] -fda kaneton.img -M mips -cpu R4000
\end{verbatim}

There is no mips firmware (or bios for qemu) today but we will launch the kernel as it. It will load at the address \code{0xFFFFFFFFBFC00000}. To do that, the kernel must be copied to \code{mips\_bios.bin} and the \code{L} option must indicate the path to this file. We will develop a firmware later.

\subsubsection{MIPS R4000 microprocessor}

The first 64 bits MIPS microprocessor was the R4000, it released in 1991 but MIPS could not produce it themselves.
In 1992 SGI bought the MIPS company that becam MIPS technologies. Thus the microprocessor could be produce for SGI platforms. The MIPS R4000 has been designed in three configurations for low-cost system (R4000PC), high performance uniprocessor system (R4000SC) and for high multiprocessor platform (R4000MC).

Whe have chosen this processor because it is the simplest MIPS 64 bits microprocessor and it has all functionalities that we need for the pedagogical purposes.

%%
%% copyright quintard julien
%% 
%% kaneton
%% 
%% development-environment.tex
%% 
%% path          /home/mycure/kaneton
%% 
%% made by mycure
%%         quintard julien   [quinta_j@epita.fr]
%% 
%% started on    Tue Jul  5 12:23:08 2005   mycure
%% last update   Sun Oct 23 02:55:45 2005   mycure
%%

%
% class
%

\documentclass[8pt]{beamer}

%
% packages
%

\usepackage{pgf,pgfarrows,pgfnodes,pgfautomata,pgfheaps,pgfshade}
\usepackage{colortbl}
\usepackage{times}
\usepackage{amsmath,amssymb}
\usepackage{graphics}
\usepackage{graphicx}
\usepackage{color}
\usepackage{xcolor}
\usepackage[english]{babel}
\usepackage{enumerate}
\usepackage[latin1]{inputenc}

%
% style
%

\usepackage{beamerthemesplit}
\setbeamercovered{dynamic}

%
% verbatim font
%

\definecolor{verbatimcolor}{rgb}{0,0.4,0}

\makeatletter
\renewcommand{\verbatim@font}
  {\ttfamily\footnotesize\color{verbatimcolor}\selectfont}
\makeatother

%
% new line
%

\newcommand{\nl}[0]{\vspace{0.4cm}}

%
% title
%

\title{Development Environment}

%
% authors
%

\author
{
  Julien~Quintard\inst{1} \\
  {\tiny julien.quintard@gmail.com}
}

\institute
{
  \inst{1} kaneton distributed operating system project
}

%
% date
%

\date{\today}

%
% logos
%

\pgfdeclareimage[interpolate=true,width=34pt,height=18pt]
                {epita}{../../logos/epita}
\pgfdeclareimage[interpolate=true,width=49pt,height=18pt]
                {upmc}{../../logos/upmc}
\pgfdeclareimage[interpolate=true,width=25pt,height=18pt]
                {lse}{../../logos/lse}

%
% table of contents at the beginning of each section
%

\AtBeginSection[]
{
  \begin{frame}<beamer>
   \frametitle{Outline}
    \tableofcontents[current]
  \end{frame}
}

%
% table of contents at the beginning of each subsection
%

\AtBeginSubsection[]
{
  \begin{frame}<beamer>
   \frametitle{Outline}
    \tableofcontents[current,currentsubsection]
  \end{frame}
}

%
% document
%

\begin{document}

%
% title frame
%

\begin{frame}
  \titlepage

  \begin{center}
    \pgfuseimage{epita} \hspace{0.1cm} \pgfuseimage{upmc} \hspace{0.1cm}
    \pgfuseimage{lse} \hspace{0.1cm}
  \end{center}
\end{frame}

%
% outline frame
%

\begin{frame}
  \frametitle{Outline}
  \tableofcontents
\end{frame}

%
% overview
%

\section{Overview}

% 1)

\begin{frame}
  \frametitle{Introduction}

  From the previous years, a development environment was introduced.

  \nl

  The questions are:

  \begin{enumerate}[<+->]
    \item
      Why?
    \item
      What are the advantages and disadvantages of such a
      development environment?
    \item
      How did the other promotions do?
  \end{enumerate}
\end{frame}

% 2)

\begin{frame}
  \frametitle{Explanations}

  Over the years, the kaneton project evolved, starting with a very
  simple introduction to low-level programming, to microkernel
  development and finally to a distributed operating system project.

  \nl

  Going always further implies many modifications in the project
  including:

  \begin{itemize}[<+->]
    \item
      The courses given which now go from the Intel processor to
      the distributed operating system concepts
    \item
      The assignments which always evolve to study advanced topics
    \item
      The context because we now have to provide parts of the microkernel
      to avoid students a development from scratch
    \item
      .. and so the requirements
  \end{itemize}
\end{frame}

% 3)

\begin{frame}
  \frametitle{The Courses}

  The kaneton project now comes with four courses:

  \begin{enumerate}
    \item
      The design of the kaneton distributed operating system including
      the microkernel
    \item
      The Intel processor
    \item
      The kernel concepts
    \item
      The distributed operating system concepts
  \end{enumerate}
\end{frame}

% 4)

\begin{frame}
  \frametitle{The Assignments}

  During the year 2005, the students develop a poor microkernel
  from scratch with few functionalities, a driver and finally a baby
  file system.

  \nl

  We cannot ask the students of the year 2006 to develop the same project
  but to go further to study advanced topics like distributed algorithms.

  \nl

  So, we cannot ask the students to develop every parts of the microkernel
  because this takes much time and implies to not study advanced
  topics.
\end{frame}

% 5)

\begin{frame}
  \frametitle{The Context}

  Providing students parts of the microkernel is not enough.

  \nl

  Indeed, we decided to provide a complete development environment
  including:

  \begin{itemize}
    \item
      Makefiles
    \item
      Shell scripts
    \item
      Papers
    \item
      Tools
    \item
      .. everything you need to start microkernel development
  \end{itemize}
\end{frame}

% 6)

\begin{frame}
  \frametitle{Why?}

  The remaining question is:

  \nl

  \textbf{Why providing such a development environment and not letting us
    develop one ourself?}

  \nl

  The answers simply are:

  \begin{itemize}
    \item
      Developing such a development environment takes much time and
      need experience
    \item
      This development environment include very powerful features:
      multiusers cooperation, different operating systems etc..
    \item
      Finally, students will not be able to create such a complicated
      development tree so it is provided to not waste time.
  \end{itemize}
\end{frame}

% 7)

\begin{frame}
  \frametitle{The Requirements}

  The students starting the kaneton project should think that they
  will learn many many things during the year.

  \nl

  This year, we are trying to lead students to a distributed operating
  system.

  \nl

  This implies more concepts, algorithms and techniques to learn.

  \nl

  To do this we introduced more courses but the students will have
  to work hard to be able to success.
\end{frame}

% 8)

\begin{frame}[containsverbatim]
  \frametitle{Tree}

  \begin{center}

  \begin{verbatim}
    /
      conf/
      core/
      doc/
      drivers/
      env/
      export/
      libs/
      papers/
      programs/
      services/
      tools/
  \end{verbatim}

  \end{center}
\end{frame}

%
% conf
%

\section{conf}

% 1)

\begin{frame}
  \frametitle{Overview}

  The \textbf{conf} directory contains user variables used to parameterise:

  \begin{itemize}
    \item
      the development environment: makefiles, scripts etc..
    \item
      the kernel
  \end{itemize}

  \nl

  This configuration system is very interesting coupled with versionning
  system.

  \nl

  Indeed, you can develop using special compilation flags, specific kernel
  configuration without conflicts with other developers.
\end{frame}

% 2)

\begin{frame}[containsverbatim]
  \frametitle{Tree}

  \begin{verbatim}
    conf/
      mycure/
        conf.c
        conf.h
        kaneton.conf
        modules.conf
        mycure.conf
      pwipwi/
      chiche/
  \end{verbatim}

  This configuration system uses the shell variable \$USER to find
  the main configuration file: \textbf{conf/\$USER/\$USER.conf}.
\end{frame}

% 3)

\begin{frame}
  \frametitle{conf.c}

  This file is not used yet.
\end{frame}

% 4)

\begin{frame}
  \frametitle{conf.h}

  This file contains macros to configure the kernel:

  \begin{itemize}
    \item
      \textbf{CONF\_TITLE}
    \item
      \textbf{CONF\_VERSION}
    \item
      \textbf{CONF\_DEBUG}
    \item
      etc..
  \end{itemize}

  \nl

  This file is included by the kernel code.
\end{frame}

% 5)

\begin{frame}
  \frametitle{kaneton.conf}

  This configuration file is used to pass arguments at the runtime to the
  servers.

  \nl

  This file is also used to configure kernel and servers input variables.
\end{frame}

% 6)

\begin{frame}
  \frametitle{modules.conf}

  This file contains the list of the modules to be loaded by the
  multi-bootloader.

  \nl

  These modules will be passed to the kernel at the boot time.

  \nl

  Be careful, a module here is not a module in the Linux or BSD terms.

  \nl

  A module is simply a file to load.
\end{frame}

% 7)

\begin{frame}
  \frametitle{\$USER.conf}

  Finally the main configuration file contains the configuration
  variables for the development environment.

  \nl

  This file uses the syntax of the make files.

  \nl

  Every variable defined in this file will be used by the makefiles
  and the scripts.
\end{frame}

%
% env
%

\section{env}

% 1)

\begin{frame}
  \frametitle{Overview}

  The \textbf{env} directory contains the different development environments.

  \nl

  This directory is the heart of the kaneton development system.

  \nl

  Indeed, a user can develop the kaneton project on a Mac machine using
  cross compilation for Intel processors ('cause PowerPC processor)
  while another one is using a FreeBSD operating system on an Intel processor.

  \nl

  So, the development environment has to deal with these different operating
  systems and architectures just for the development.
\end{frame}

% 2)

\begin{frame}
  \frametitle{Our System}

  To do this, we decided to introduce an environment system.

  \nl

  Every time a user gets the kaneton development tarball, he first has to
  create his development environment given a couple operating system and
  architecture which leads to an environment.

  \nl

  Once the environment is installed, the user can develop, compile the kernel
  etc.. without problems because everything (makefiles, scripts etc..) use
  the binaries, variables etc.. for his environment.

  \nl

  The environment is specified in the user configuration file.
\end{frame}

% 3)

\begin{frame}[containsverbatim]
  \frametitle{Tree}

  \begin{verbatim}
    env/
      clean.sh
      init.sh
      unix/
        clean.sh
        init.sh
        kaneton.mk
      macos-powerpc.ia32/
  \end{verbatim}

  \nl

  Here the \textbf{unix} is considered as the generic unix
  environment but everyone can add a specific linux, FreeBSD, Solaris etc..
  environment.
\end{frame}

% 4)

\begin{frame}
  \frametitle{init.sh}

  The \textbf{init.sh} shell script is used to install the development
  environment.

  \nl

  This script first gets the configuration variables from the user
  configuration file, then calls the specific \textbf{init.sh} script
  of the given environment.

  \nl

  Finally the script installs some links and initialises the makefile
  dependencies.

  \nl

  The \textit{[environment]}/init.sh shell script is used to install
  specific stuff.
\end{frame}

% 5)

\begin{frame}
  \frametitle{clean.sh}

  The \textbf{clean.sh} shell script just cleans the environment.

  \nl

  This shell script also call the environment specific clean.sh script.
\end{frame}

% 6)

\begin{frame}
  \frametitle{kaneton.mk}

  The \textbf{kaneton.mk} makefile dependency is the heart of the
  kaneton compilation system.

  \nl

  Indeed, every makefile is composed of calls to special routines
  which are implemented by the makefile dependency depending on the
  environment: operating system plus architecture source and destination.

  \nl

  Moreover the \textbf{kaneton.mk} makefile dependency includes the
  user configuration file so each makefile of the system is able to
  use user defined variables.

  \nl

  The kaneton compilation system uses a very special gmake feature:
  the makefile \textbf{call} function.
\end{frame}

% 7)

\begin{frame}[containsverbatim]
  \frametitle{Use}

  \begin{verbatim}
    $ make init
    [+] installing environment

    [+] your current configuration:
    [+]   environment:              unix
    [+]   architecture:             ia32
    [+]   multi-bootloader:         grub

    [...]

    $ make clean
    [+] cleaning environment

    [...]

    $ 
  \end{verbatim}
\end{frame}

%
% tools
%

\section{tools}

% 1)

\begin{frame}
  \frametitle{Overview}

  The \textbf{tools} directory contains programs, scripts, special
  files used by the kaneton project.

  \nl

  For example a script to initialise and install modules on a grub
  bootloader boot device is provided in the subdirectory
  \textit{scripts/multi-bootloaders/grub/}.

  \nl

  The \textbf{tools} directory also contains the ld scripts used
  to correctly compile the bootstrap, the bootloader, the kernel, the
  drivers, the services and the programs.
\end{frame}

% 2)

\begin{frame}[containsverbatim]
  \frametitle{Tree}

  \begin{verbatim}
    tools/
      scripts/
        ld/
          arch/
            ia32/
              bootstrap.lds
              bootloader.lds
              kaneton.lds
              driver.lds
              service.lds
              user.lds
        multi-bootloaders/
          grub/
          lilo/
        prototypes/
          mkp.py
  \end{verbatim}
\end{frame}

% 3)

\begin{frame}[containsverbatim]
  \frametitle{Use}

  \begin{verbatim}
    $ make build
    [+] initialising boot system

    [+] boot system initialised successfully
    $ make install
    [+] initialising boot system

    [+] /tmp/menu.lst
    [+] core/bootloader/bootloader
    [+] core/kaneton/kaneton
    [+] conf/mycure/kaneton.conf
    [+] drivers/cons/cons
    [+] services/dsh/dsh

    [+] boot system initialised successfully
    $ 
  \end{verbatim}
\end{frame}

% 4)

\begin{frame}[containsverbatim]
  \frametitle{Prototypes}

  The compilation system permits to generate the prototypes in a very easy
  and elegant way.

  \begin{verbatim}
    $ make proto
    [PROTOTYPES]            libdata.h
    [PROTOTYPES]            libstring.h
    [PROTOTYPES]            libsys.h
    [PROTOTYPES]            bootloader.h
    [PROTOTYPES]            ia32.h
    [PROTOTYPES]            kaneton.h
    [PROTOTYPES]            as.h
    [PROTOTYPES]            conf.h
    [PROTOTYPES]            serial.h

    [...]

    $ 
  \end{verbatim}
\end{frame}

% 5)

\begin{frame}[containsverbatim]
  \frametitle{Explanations}

  This system is based on tags in the header files which specify
  from which files to extract prototypes.

  \nl

  The tags are of the form:

  \begin{verbatim}
    /*
     * ---------- prototypes -------------------------------------------------
     *
     *      ../../kaneton/set/set.c
     *      ../../kaneton/set/set_array.c
     *      ../../kaneton/set/set_ll.c
     *      ../../kaneton/set/set_bpt.c
     */
  \end{verbatim}
\end{frame}

% 5)

\begin{frame}[containsverbatim]
  \frametitle{Dependencies}

  The compilation system uses full dependencies between files.

  \nl

  To regenerate the dependencies, for example when adding a
  \textit{\#include} c-preprocessor directive in a source file:

  \begin{verbatim}
    $ make dep
    [REMOVE]                .makefile.mk
    [DEPENDENCIES]          dump.c
    [DEPENDENCIES]          alloc.c
    [DEPENDENCIES]          sum2.c

    [...]

    $ 
  \end{verbatim}
\end{frame}

%
% libs
%

\section{libs}

% 1)

\begin{frame}
  \frametitle{Overview}

  The \textbf{libs} directory contains the libraries used by the kaneton
  project like:

  \begin{itemize}
    \item
      libc
    \item
      crt
    \item
      libposix
    \item
      etc..
  \end{itemize}
\end{frame}

%
% core
%

\section{core}

% 1)

\begin{frame}
  \frametitle{Overview}

  The \textbf{core} directory contains the source code for the microkernel
  including the bootstrap, the bootloader and the kernel itsef.

  \nl

  Each part contains an \textbf{arch} directory used for architecture
  dependent soure code.
\end{frame}

% 2)

\begin{frame}[containsverbatim]
  \frametitle{Tree}

  \begin{verbatim}
    core/
      bootstrap/
        arch/
          ia32/ <---;
          machdep --+
      bootloader/
        arch/
      kaneton/
        arch/
        as/
        conf/
        debug/
        id/
        segment/
        set/
        stats/
  \end{verbatim}
\end{frame}

%
% drivers
%

\section{drivers}

% 1)

\begin{frame}
  \frametitle{Overview}

  The \textbf{drivers} directory contains the drivers of the kaneton
  microkernel.

  \nl

  A driver, in the kaneton terms, is a microkernel server which is allowed
  to communicate with hardware devices.
\end{frame}

% 2)

\begin{frame}[containsverbatim]
  \frametitle{Tree}

  \begin{verbatim}
    drivers/
      cons/
        Makefile
        cons.c
      dma/
      kbd/
      ide/
  \end{verbatim}
\end{frame}

%
% services
%

\section{services}

% 1)

\begin{frame}
  \frametitle{Overview}

  The \textbf{services} directory contains the services of the kaneton
  microkernel.

  \nl

  A service, in the kaneton terms, in simply a server which does not
  communicate with the hardware.
\end{frame}

% 2)

\begin{frame}[containsverbatim]
  \frametitle{Tree}

  \begin{verbatim}
    services/
      dsh/
      mod/
        Makefile
        mod.c
        modfs.c
  \end{verbatim}
\end{frame}

%
% programs
%

\section{programs}

% 1)

\begin{frame}
  \frametitle{Overview}

  The \textbf{programs} directory contains the sources of common
  programs.

  \nl

  A program in the kaneton terms is just a non-privilegied
  process.
\end{frame}

% 2)

\begin{frame}[containsverbatim]
  \frametitle{Tree}

  \begin{verbatim}
    programs/
      ls/
      wc/
      cat/
      mount/
      umount/
      gcc/
      emacs/
  \end{verbatim}
\end{frame}

%
% export
%

\section{export}

% 1)

\begin{frame}
  \frametitle{Overview}

  The \textbf{export} directory is used to create kaneton distribution.

  \nl

  This feature is especially used by the maintainers of the kaneton
  project which create very special kaneton distribution for
  the students.
\end{frame}

% 2)

\begin{frame}[containsverbatim]
  \frametitle{Use}

  The only way to export kaneton is to do like this:

  \begin{verbatim}
    $ make export
    [!] usage: exporter.sh [stage]

    available stages: k0 k1 k2 k3 k4 k5 k6 k7 k8 k9 kaneton dist
    $ make export-k3
  \end{verbatim}

  \begin{itemize}
    \item
      \textbf{k[0-9]}: create a special kaneton version for the k[0-9]
      subproject
    \item
      \textbf{kaneton}: create an entire kaneton version for the lastest
      subproject
    \item
      \textbf{dist}: create an entire backup of the kaneton development
      project
  \end{itemize}
\end{frame}

%
% papers
%

\section{papers}

% 1)

\begin{frame}
  \frametitle{Overview}

  The \textbf{papers} directory contains the papers and lectures
  in relation with the kaneton project.

  \nl

  We prefered set the papers directly into the tarball so every student
  can easily read them.
\end{frame}

% 2)

\begin{frame}[containsverbatim]
  \frametitle{Tree}

  \begin{verbatim}
    papers/
      assignments/
      design/
      kaneton/
      seminar/
      lectures/
        kernels/
        inline-assembly/
        c-preprocessor/
        distributed-operating-systems/
        arch-ia32/
  \end{verbatim}
\end{frame}

% 3)

\begin{frame}[containsverbatim]
  \frametitle{Use}

  \begin{verbatim}
    $ make view
    [+] papers:

    [+]   assignments
    [+]   design
    [+]   arch-ia32
    [+]   c-preprocessor
    [+]   distributed-operating-systems
    [+]   inline-assembly
    [+]   kernels
    [+]   development-environment

    [!] usage: viewer.sh [paper]
    $ make view-design
  \end{verbatim}
\end{frame}

%
% doc
%

\section{doc}

% 1)

\begin{frame}
  \frametitle{Overview}

  The \textbf{doc} directory contains every document useful for
  the development of the kaneton project.

  \nl

  This directory will theorically contain documents on the different
  architectures, documents on some hardware devices like ide, usb etc..
\end{frame}

\end{document}

%
% ---------- header -----------------------------------------------------------
%
% project       kaneton
%
% license       kaneton
%
% file          /home/enguerrand/.../book/architecture/mips64/bootloader.tex
%
% created       enguerrand raymond   [sun oct  5 16:54:35 2008]
% updated       enguerrand raymond   [wed may 13 21:09:14 2009]
%

%
% booloader
%

\chapter{Bootloader}

This chapter describes the bootloading phase of kaneton on MIPS 64 systems. Each processor bootstrap are described. For the moment, the kaneton bootloader and the kernel will be loaded at this address. Later a more real firmware will be developed to load from floppy disk the kernel and these modules

\subsection{Bootloader operations}

Schema


The microprocessor initalization corresponds to a reset or a soft reset exception, see chapter 5 in \cite{MIPS64:MIPS64V3} for a complete CPU state description after these exceptions occur.

%%
%% licence       kaneton licence
%%
%% project       kaneton
%%
%% file          /home/buckman/kaneton/view/books/arch-ia32-virtual/memory.tex
%%
%% created       matthieu bucchianeri   [sat sep  2 11:40:35 2006]
%% updated       matthieu bucchianeri   [sat nov  4 18:38:29 2006]
%%

%
% memory management
%

\chapter{Memory management}

\newpage

%
% overview of the ia-32 mmu
%

\section{Overview of the IA-32 MMU}

%
% privilege-check capable segmentation model
%

\section{Privilege-check capable segmentation model}

%
% raw accesses to kaneton segments
%

\section{Raw accesses to kaneton segments}

Direct accesses to memory segments (in kaneton terms) is not possible
on IA-32, as it needs to bypass the MMU, which is impossible.

The \textit{segment\_read}, \textit{segment\_write} and
\textit{segment\_copy} dependent code is implemented using the region
manager. We create a temporary mapping using \textit{region\_reserve},
then we copy the data to a kernel buffer or directly from one segment
to another, and to finish we call \textit{region\_release}.

As a result, the three operations described above are very slow on
IA-32 architecture, and their use is not recommended.

%
% creating address spaces
%

\section{Creating address spaces}

Creating an address space is not just allocating a few structures and
filling it. It also involves preparing the address space for later
operations.

The case of the kernel address space must be handled separately, as it
is more complex and critical.

%
% creating the kernel address space
%

\subsection{Creating the kernel address space}

%
% creating a classical address space
%

\subsection{Creating a classical address space}

When creating a new address space, we must install a few basic
structures into this one:

\begin{enumerate}
\item A page-directory is built. This is done reserving a one-page
segment and the calling \textit{pd\_build} ;
\item The global memory areas must be injected and mapped: the
\textit{.handler} section, the TSS, both GDT and IDT, and to finish, a
ring 0 stack (see the \textbf{Task management} chapter for more
information about these) ;
\item The page-directory base register is given to all threads, so
when switching from one thread to another, the address space is
automatically switched.
\end{enumerate}

The need of mapping some common regions into each address space comes
from the absense of Task State Segment, an high-level context
switching mecanism implemented on IA-32. Instead, the context
switching is done by hand. Structures like the interrupt vector and a
kernel stack are needed when doing this context switch. They must be
mapped permanently, because the address space is switched to the
kernel address space only at some advanced stages of the switching
algorithm. But these structures are mapped with the ``Supervisor''
flag enabled, so even the owner of the address space cannot access it
in normal running conditions. Please refer to the \textbf{Task
management} chapter for further information.

The third point is done getting the task associated with the new
address space and browsing its thread set. For each thread, we fill
the register CR3 (the page-directory base register) with the
page-directory base, computed with \textit{pd\_get\_cr3}. See
\textbf{Address space switching} section in \textbf{Task management}
chapter.

%
% operations on regions
%

\section{Operations on regions}

The region manager defines three operations on regions:

\begin{itemize}
\item The reservation ;
\item The releasing ;
\item The resizing.
\end{itemize}

We must add the permission changing operation, coming from the segment
manager (see the corresponding paragraph for more information).

Other operations (splitting and coalescing) don't affect architecture
specific structure, and have no implementation issues on IA-32.

%
% reserving a region
%

\subsection{Reserving a region}

Reserving a region is the operation that create in a given address
space a mapping between one or more physical pages and the same amount
of virtual pages.

The only action performed to reserve a region is to fill the
page-tables tree, creating one or more page-tables and adding a few
entries into these.

As region is a high-level frontend, it only performs a mapping of
contiguous areas. The algorithm for contiguous mapping is quite simple:

\begin{enumerate}
\item We compute the \textit{pde\_start} and \textit{pde\_end}
values. These values give the interval of page-directory entries we
will need to loop thought. We use a macro named \textit{PDE\_ENTRY},
that returns for a given address the index of the page-directory entry
that leads to the good page-table ;
\item Identically, we compute \textit{pte\_start} and
\textit{pte\_end}, indicating the index of the first page-table entry
to add into the first page-table, and the last page-table entry into
the last page-table ;
\item Now, we will need to loop throught the page-directory. So, first
of all, we must map it into the current address space (the kernel
address space). This is done using the
\textit{ia32\_region\_map\_chunk} function. This particular function
uses the \textbf{mirroring technique}, explained below this algorithm ;
\item We loop throught \textit{pde\_start} and \textit{pde\_end}
page-directory entries ;
  \begin{enumerate}
  \item If the page-table we need to modify does not exist, we create
  it. This is done by reserving a segment and adding the corresponding
  page-directory entry ;
  \item Next, as we mapped the page-directory, we also need to map the
  page-table into the kernel address space in order to change it. This
  is done the same way as previously ;
  \item Now, we are able to add some page-table entries. We loop
  throught the good interval, which is determined as follow:
    \begin{itemize}
    \item If the current page-table is the first-one, then we start at
    \textit{pte\_start}, otherwise, we start at 0 ;
    \item If the current page-table is the last-one, then we stop our
    loop at \textit{pte\_end}, otherwise, we use the maximum value
    (1024), to fill all the page-table entries ;
    \end{itemize}
    \item We fill these entries. Each one maps exactly one page ;
    \item Before moving on to the next page-table, we take care of
    unmapping the previous one ;
  \end{enumerate}
\item Once the whole process is complete, we unmap the page-directory.
\end{enumerate}

The mirroring technique permits to access the current page-directory
and page-tables hierarchy in the kernel address space. To access a
page-table, this one must be mapped. And to map it, we must access
another page-table to create a mapping. As you can see, there is a
cyclic problem here.

The solution is to consider the page-directory as a page-table. One
entry in the page-directory points to the page-directory itself. This
way, accessing a page-table is done by reading or writing to a precise
address computed as follow:

$page\_table = ENTRY\_ADDR(PD\_MIRROR, page\_table\_index)$

The \textit{PD\_MIRROR} constant correspond to the index of the entry
in the page-directory that points to the page-directory. To access the
page-directory itself, the formula is the following:

$page\_directory = ENTRY\_ADDR(PD\_MIRROR, PD\_MIRROR)$

With this technique, we are able to modify the page-directory and the
page-tables without mapping them each time. Both
\textit{ia32\_region\_map\_chunk} and
\textit{ia32\_region\_unmap\_chunk} rely on the mirroring technique.

%
% releasing a region
%

\subsection{Releasing a region}

Releasing a region is exactly the same processing as for reserving
one, but instead of filling the page-table entries, we remove it.

\begin{enumerate}
\item We compute \textit{pde\_start}, \textit{pde\_end}, \textit{pte\_start}
and \textit{pte\_end} the same way ;
\item We map the page-directory into the current address space ;
\item We loop throught \textit{pde\_start} and \textit{pde\_end}
page-directory entries ;
  \begin{enumerate}
  \item We map the current page-table into the kernel address space in order
    to change it ;
  \item Now, we are able to add some page-table entries. We loop
  throught the good interval, which is determined as follow:
    \begin{itemize}
    \item If the current page-table is the first-one, then we start at
    \textit{pte\_start}, otherwise, we start at 0 ;
    \item If the current page-table is the last-one, then we stop our
    loop at \textit{pte\_end}, otherwise, we use the maximum value
    (1024), to fill all the page-table entries ;
    \end{itemize}
    \item We remove the page-table entries ;
    \item Before moving on to the next page-table, we take care of
    unmapping the current one ;
    \item If we unmapped all the entries present in the page-table, we
    remove it and release the associated segment ;
  \end{enumerate}
\item Once the whole process is complete, we unmap the page-directory.
\end{enumerate}

%
% changing permissions of a region
%

\subsection{Changing permissions of a region}

On IA-32, we manage permissions with regions and not with
segments. The \textit{segment\_perms} machine dependent code is very
similar to the two previous one.

The algorithm is quite the same, except that we modify each page-table
entry only changing the permission flags.

%
% resizing of a region
%

\subsection{Resizing of a region}

The current way of resizing a region is not optimized and rely on
other operations of the region manager for simplicity reasons.

When shrinking a region, the independent code changes the
corresponding \textit{o\_region} object size field. The work done by
the dependent code on IA-32 is to remove the invalidated page-table
entries. The current implementation inject a temporary ``virtual''
region corresponding to the memory chunk to unmap, and the directly
calls \textit{ia32\_region\_release}.

When a region needs to be enlarged, the process is very similar: we
create a region directly after the region we need to enlarge. We use
\textit{ia32\_region\_reserve} to fill the page-tables. Then, we
coalesce the two region into one unique.

%
% translation cache flushing
%

\subsection{Translation cache flushing}

The translation caches (also known as Translation Lookaside Buffers,
TLB), are the low latency memories used to make the translation of
virtual addresses into physical addresses.

In fact, the whole page-tables tree is not browsed each time the
microprocessor accesses a word in memory. Doing so will leads the
translation mecanism to be very slow (as each memory access need two
other accesses, one to read the page-directory and another to read a
page-table entry). Each time such translation is done, the
corresponding page-table entry is stored into the TLB. So, next accesses
to the same page will not need to go thought the paging tree.

But what appends when we add or remove a page translation while the TLB
already have a translation for the same address? Let's imagine the
following situation: we map the page A to the page frame B. The
microprocessor accesses a word in A. The TLB is filled with the
translation A $\rightarrow$ B. Next, we change the mapping so A
$\rightarrow$ C. Now, the CPU accesses a word in A. The translation
will be fetched from the TLB, still having A $\rightarrow$ B.

So, when such mapping change occurs, we need to invalidate one or more
TLB entries. Additionnaly, when switching from one address space to
another, we must flush the entire TLB.

On IA-32, flushing the whole TLB is done automatically when switching
address space (changing the value in the page-directory base
register). As all the functions modifying the virtual addresses
mapping are executed in kernel-land, we must flush some TLB entries
only when modifying mapping into the kernel address space. When
changing some mappings into a task address space, it is not necessary
to flush the TLB, as switching back to the task will invalidate the
whole caches.

%%%
%% licence       kaneton licence
%%
%% project       kaneton
%%
%% file          /home/buckman/kaneton/view/books/ia32-virtual/event.tex
%%
%% created       matthieu bucchianeri   [sat sep  2 11:40:54 2006]
%% updated       matthieu bucchianeri   [sun sep  3 12:24:43 2006]
%%

%
% event management
%

\chapter{Event management}

%
% overview of exceptions, interrupts and hardware interrupts
%

\section{Overview of exceptions, interrupts and hardware interrupts}


%%
% ---------- header -----------------------------------------------------------
%
% project       kaneton
%
% license       kaneton
%
% file          /home/buckman/cry...on/view/book/architecture/ia32/timer.tex
%
% created       matthieu bucchianeri   [tue aug 14 20:09:50 2007]
% updated       matthieu bucchianeri   [tue oct  2 16:14:31 2007]
%

%
% timer management
%

\chapter{Timer management}

This chapter describes the architecture-dependent code driving the
kaneton timer manager.

\newpage

%
% fixed frequency interrupts
%

\section{Fixed frequency interrupts}

The kaneton timer manager requires the function
\textit{timer\_handler()} to be called every
\textit{TIMER\_MS\_PER\_TICK} milliseconds.

There are three components able to throw an interrupt at fixed rate:

\begin{itemize}
  \item
    The microprocessor itself, through the local APIC;
  \item
    The Programmable Interval Timer (PIT);
  \item
    The Real Time Clock (RTC).
\end{itemize}

The first solution is the most accurate but needs to be calibrated,
which is a more complex process. The last solution is also very
accurate but the available frequencies need floating point computing
to be millisecond-accurate.

To keep the code as simple as possible, we will use the PIT, which is
a very easily programmable component.

The PIT is connected to a fixed frequency oscillator of 1193180
Hz. The component act as a counter/decounter. We use the mode 2: a
value is loaded in the counter register and then decremented on each
clock tick. Once the counter register reaches zero, an interrupt is
signaled and the counter is reset to its initially loaded value.

The interrupt pin is connected to the IRQ line 0.

The very short function \textit{ibmpc\_timer\_init()} setup the PIT
channel 0 at mode 2 with an initial counter register computed as
follow:

$$
latch = \frac{IBMPC\_CLOCK\_TICK\_RATE}{\frac{1000}{TIMER\_MS\_PER\_TICK}}
$$

Where \textit{IBMPC\_CLOCK\_TICK\_RATE} is the fixed oscillator
frequency (1193180 Hz).

To finish, the glue code of \textit{timer\_initialize()} hooks the IRQ
0 to the function \textit{timer\_handler()}.

%%%
%% licence       kaneton licence
%%
%% project       kaneton
%%
%% file          /home/buckman/kaneton/view/books/ia32-virtual/task.tex
%%
%% created       matthieu bucchianeri   [sat sep  2 11:41:15 2006]
%% updated       matthieu bucchianeri   [sun sep  3 12:19:13 2006]
%%

%
% task management
%

\chapter{Task management}

%
% overview of execution contexts on ia-32
%

\section{Overview of execution contexts on IA-32}

%
% reserving a task
%

\section{Reserving a task}

%
% setting up the execution context
%

\section{Setting up the execution context}

%
% preparing a new thread
%

\section{Preparing a new thread}

%
% scheduling and context switch
%

\section{Scheduling and context switch}

%%%
%% licence       kaneton licence
%%
%% project       kaneton
%%
%% file          /home/buckman/kaneton/view/books/ia32-virtual/io.tex
%%
%% created       matthieu bucchianeri   [sat sep  2 11:39:52 2006]
%% updated       matthieu bucchianeri   [sun sep  3 12:21:36 2006]
%%

%
% io management
%

\chapter{I/O management}

%
% overview of io on ia-32
%

\section{Overview of I/Os on IA-32}

%
% allowing or denying ios
%

\section{Allowing or denying I/Os}

%%
% ---------- header -----------------------------------------------------------
%
% project       kaneton
%
% license       kaneton
%
% file          /home/buckman/kan...view/book/architecture/ia32/syscalls.tex
%
% created       matthieu bucchianeri   [tue aug 14 20:10:52 2007]
% updated       matthieu bucchianeri   [thu aug 16 19:33:03 2007]
%

%
% system calls
%

\chapter{System calls}

This chapter describes the implementation details of the low-level
system calls used in the message manager.

\newpage

%
% software interrupts and register parameter-passing
%

\section{Software interrupts and register parameter-passing}

The IA-32 microprocessors offers two facilities to implement system
calls:

\begin{itemize}
  \item
    The software interrupts with the \textit{INT} instruction;
  \item
    The \textit{SYSENTER} and \textit{SYSEXIT} instructions.
\end{itemize}

We use the first technique as it is widely used and supported on any
IA-32 microprocessor.

The fastest way to pass parameters to a system call is to use the
registers. We can pass 32 bits per register. As the message primitives
take many arguments, it is difficult to put all of them in the
registers. For this reason, we have to economize one register by
associating one interrupt per system call.

%
% kernel-side code
%

\subsection{Kernel-side code}

The function \textit{ia32\_syscalls\_init()} calls
\textit{event\_reserve()} for each system call as defined in the table
in the \textit{System calls map} section.

In every handler, we use the interrupted context pointer to get the
register values and to set the output values.

\begin{verbatim}
ia32_syshandler_register()
{
  type_1    arg1;
  type_2    arg2;
  t_status   res;

  arg1 = ia32_context->eax;
  arg2 = ia32_context->ebx;

  res = function(arg1, arg2);

  ia32_context->eax = res;
}
\end{verbatim}

%
% user-side code
%

\subsection{User-side code}

The user-side system call code put the arguments in registers, call
the corresponding software interrupt and get back the result(s) from
registers.

\begin{verbatim}
syscall(arg1, arg2)
{
  eax <- arg1
  ebx <- arg2

  int 56

  res <- eax

  return (res);
}
\end{verbatim}

%
% system calls map
%

\section{System calls map}

As explained just above, one software interrupt slot is allocated for
each system call. Here is the table of these indexes, the associated
system call and the register mapping of the parameters.

\begin{center}
\begin{tabular}{| c | c | c |}
\hline
Soft-int & Syscall           & Registers \\
\hline
56       & message\_register & \textit{eax} = type; \textit{ebx} = size \\
\hline
57       & message\_send     & \textit{eax,ebx,ecx,edx} = node; \textit{esi} = type; \textit{edi} = data; \textit{ebp} = size \\
\hline
58       & message\_transmit & \textit{eax,ebx,ecx,edx} = node; \textit{esi} = type; \textit{edi} = data; \textit{ebp} = size \\
\hline
59       & message\_throw    & XXX \\
\hline
60       & message\_receive  & \textit{eax} = type; \textit{ebx} = data; \textbf{output}: \textit{ebx} = size; \textit{ecx,edx,esi,edi} = node \\
\hline
61       & message\_grab     & XXX \\
\hline
62       & message\_poll     & \textit{eax} = type; \textit{ebx} = data; \textbf{output}: \textit{ebx} = size; \textit{ecx,edx,esi,edi} = node \\
\hline
63       & message\_state    & XXX \\
\hline
64       & message\_wait     & XXX \\
\hline
\end{tabular}
\end{center}

The register \textit{eax} is always used as the result output register
(of type \textit{t\_error}).

%
% upcalls
%

\section{Upcalls}

XXX
%%
% ---------- header -----------------------------------------------------------
%
% project       kaneton
%
% license       kaneton
%
% file          /home/mycure/kaneton/view/book/kaneton/references.tex
%
% created       julien quintard   [tue jun 19 15:36:51 2007]
% updated       julien quintard   [sun dec 16 17:33:44 2007]
%

%
% ---------- references -------------------------------------------------------
%

\chapter{References}
\label{chapter:references}

This chapter contains a listing of the kaneton documents that readers might
be interested in as well.

\newpage

%
% ---------- text -------------------------------------------------------------
%

This document aimed at presenting the kaneton microkernel and its organisation.

The books listed below go further by describing more precisely a specific
component of the kaneton microkernel:

\begin{enumerate}
  \item
    \textbf{The kaneton microkernel :: core}

    \-

    That document focuses on the design and implementation of the kaneton
    microkernel core through the different managers;
  \item
    \textbf{The kaneton microkernel :: \textit{architecture}}

    \-

    Theses documents tackle the implementation of kaneton on the given
    \textit{architecture}. Additionally, these documents contain information
    on platforms coupled with this architecture.

    \-

    Thes documents can be found in two forms, either public or private. Indeed,
    these documents contain information students must not have access to.

    \-

    The architectures currently documented are listed below:

    \begin{itemize}
      \item
	\textbf{ia32}

	\-

	\textit{Platforms}: \textbf{ibm-pc}
    \end{itemize}
\end{enumerate}

%%
%% licence       kaneton licence
%%
%% project       kaneton
%%
%% file          /home/mycure/kaneton/view/papers/kaneton/bibliography.tex
%%
%% created       julien quintard   [mon may  8 18:35:35 2006]
%% updated       julien quintard   [mon may  8 20:38:56 2006]
%%

%
% bibliograpy
%

\chapter{Bibliography}

This chapter contains the bibliography.

%
% text
%

\begin{thebibliography}{0}
  \bibitem{AST-SCO}
    \textbf{Structured Computer Organization};
    by
    \textit{Andrew S. Tanenbaum}
  \bibitem{AST-CN}
    \textbf{Computer Networks};
    by
    \textit{Andrew S. Tanenbaum}
  \bibitem{AST-OSDI}
    \textbf{Operating Systems: Design and Implementation};
    by
    \textit{Andrew S. Tanenbaum, Albert S Woodhull}
  \bibitem{AST-MOS}
    \textbf{Modern Operating Systems};
    by
    \textit{Andrew S. Tanenbaum}
  \bibitem{AST-DOS}
    \textbf{Distributed Operating Systems};
    by
    \textit{Andrew S. Tanenbaum}
  \bibitem{AST-DSPP}
    \textbf{Distributed Systems: Principles and Paradigms};
    by
    \textit{Andrew S. Tanenbaum, Maarten van Steen}
  \bibitem{NAL-DA}
    \textbf{Distributed Algorithms};
    by
    \textit{Nancy A. Lynch}
\end{thebibliography}


\end{document}
