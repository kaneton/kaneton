%
% ---------- header -----------------------------------------------------------
%
% project       kaneton
%
% license       kaneton
%
% file          /home/enguerrand/...ew/book/architecture/mips64/overview.tex
%
% created       enguerrand raymond   [sun oct  5 16:54:35 2008]
% updated       enguerrand raymond   [wed may 13 16:51:52 2009]
%

%
% overview
%

\chapter{Overview}

The first 64 bits MIPS microprocessor was the R4000, it released in 1991 but MIPS could not produce it themselves. In 1992 SGI bought the MIPS company that becam MIPS technologies. Thus the microprocessor could be produce for SGI platforms. Later in 1999, the spceficiations MIPS64 has been re
leased and the kaneton implementation is based on current documentation about it. The main microprocessors following the specification are the R5000 and the R5000f (for floating point support) and the R20000 and the R20000f. Qemu emaultes both of them.

\subsection{Development purposes}

The MIPS64 kaneton implementation is born to purchase three main goals.
The first one is the kaneton kernel architecture portability system test,
the second is the development for 64 bits architecture and the third is pedagogical.
In fact, the EPITA students must follow MIPS architecture lessons, so it is interesting 
for them to develop on MIPS architecture during the kaneton microkernel development
project.

\subsection{Development scope}

The kaneton development on MIPS64 does not follow a specefic microprocessor, it respects the MIPS64 specification provided by MIPS technology company.
To port kaneton on MIPS64, we have needed a machine, we could use real machine, SGI stations for example, but emulators does not emulate them correctly and real test would be difficult for several students in two EPITA specialities. We have prefered the mips qemu machine. This platform does not emulate real machine but is sufficient for kernel development.

\subsection{How to read this book}

This document is composed of X parts, each part corresponds to one or several source files. The memory management for example is developed in Region dependant code and MIPS64 library. For each chapter, the source files concerned will be given.
For each of them, if a kernel part is different according to a specific MIPS64 CPUs or platforms a specific part will describe this particularyties.
