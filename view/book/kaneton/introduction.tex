%
% ---------- header -----------------------------------------------------------
%
% project       kaneton
%
% license       kaneton
%
% file          /home/mycure/kaneton/view/book/kaneton/introduction.tex
%
% created       julien quintard   [mon jun 18 12:41:38 2007]
% updated       julien quintard   [sat dec 15 15:41:42 2007]
%

%
% ---------- introduction -----------------------------------------------------
%

\chapter{Introduction}

This chapter introduces the present document by drawing a list of the subjects
discussed in this book.

\newpage

%
% ---------- text -------------------------------------------------------------
%

The kaneton microkernel research project design and implementation are
described through a number of documents. The first one in the list is
the present book called \textit{The kaneton microkernel}. This book introduces
the kaneton microkernel project but also presents the kaneton portability
system for instance. Then, a document named \textit{The kaneton
microkernel :: core} specifically describes the kaneton \textit{core}.
Finally, there exists a document for every couple platform/architecture on
which the kaneton microkernel has been ported.

The present document therefore introduces the notions the reader needs to
understand for the other kaneton documents. In addition, this document contains
a chapter which draws a list of the further kaneton readings.

The remaining of this document is organised as follows. \textit{Chapter
\ref{chapter:background}} introduces the fundamental concepts the reader
should knows before reading this document.  \textit{Chapter
\ref{chapter:terminology}} introduces the kaneton terminology. Then,
\textit{Chapter \ref{chapter:portability}} describes in depth the kaneton
portability system. \textit{Chapter \ref{chapter:references}} enumerates and
briefly describes what are the further readings. Finally, \textit{Chapter
\ref{chapter:licenses}} contains information about the licenses related
to the kaneton microkernel project.
