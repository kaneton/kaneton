%
% ---------- header -----------------------------------------------------------
%
% project       kaneton
%
% license       kaneton
%
% file          /home/mycure/kaneton/view/book/kaneton/introduction.tex
%
% created       julien quintard   [mon jun 18 12:41:38 2007]
% updated       julien quintard   [wed dec 19 20:26:09 2007]
%

%
% ---------- introduction -----------------------------------------------------
%

\chapter{Introduction}

This chapter introduces the present document by drawing a list of the subjects
discussed in this book.

\newpage

%
% ---------- text -------------------------------------------------------------
%

The kaneton microkernel research project design and implementation are
described through a number of documents. The first one in the chain is
the present book called \textit{The kaneton microkernel}. This book introduces
the kaneton microkernel project through its goals, design, terminology and so
forth.

The books listed below go further by describing more precisely a specific
component of the kaneton microkernel:

\begin{enumerate}
  \item
    \textbf{The kaneton microkernel :: design}

    \-

    This paper overviews the kaneton microkernel project by describing its
    concepts and principles;
  \item
    \textbf{The kaneton microkernel :: core}

    \-

    That document focuses on the design and implementation of the kaneton
    microkernel core through the different managers;
  \item
    \textbf{The kaneton microkernel :: }\textit{[architecture]}

    \-

    Theses documents tackle the implementation of kaneton on the given
    \textit{[architecture]}. Additionally, these documents contain information
    on platforms supporting this architecture.

    \-

    These documents can be found in two forms, either public or private.
    Indeed, these documents contain information students must not have
    access to. Therefore, the private version of these documents is not
    available and contains more details on the implementation of the given
    machine.

    \-

    The architectures, with their platforms, currently documented are
    listed below:

    \begin{itemize}
      \item
	\textbf{ia32}: \textit{ibm-pc}
    \end{itemize}
\end{enumerate}
