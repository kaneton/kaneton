%%
%% licence       kaneton licence
%%
%% project       kaneton
%%
%% file          /home/buckman/kaneton/view/books/kaneton/history.tex
%%
%% created       julien quintard   [sat mar  4 14:15:39 2006]
%% updated       matthieu bucchianeri   [sun may  6 13:29:37 2007]
%%

%
% history
%

\chapter{History}

In this chapter we will detail the kaneton history from the first
year with low-level programming introduction to the last kaneton
reference implementation.

\newpage

%
% text
%

\textit{Since this chapter is intended to retrace the kaneton history,
  no insurance can be made on emails validity.}

As said in the previous chapter, the kaneton microkernel was introduced
by two students
Julien Quintard
  \footnote{julien.quintard@gmail.com} and
Jean-Pascal Billaud
  \footnote{billau\_j@epita.fr}
during the year 2004.

The kaneton microkernel is currently studied during the
EPITA
  \footnote{http://www.epita.fr}
System, Network and Security specialization curriculum
  \footnote{http://srs.epita.fr}.

During the kaneton history, the project evolved and lectures were added
to the curriculum to make the whole kaneton project more interesting and
understandable by the students.

%
% year 2004
%

\section{Year 2004}

The first year, a low-level programming introduction course was proposed
for the engineering school's first year students.

Then, about fourteen hours courses were made to introduce microprocessor's
external architecture and low-level programming.

The students had to develop small, poor and messy device drivers for the
console and keyboard peripherals. Moreover, a tiny shell was developed
by students so they were able to enter a command to launch a special
kernel action.

The course was a bit chaotic but this first shot was a success.

Therefore, the System, Network and Security specialization students
themselves asked the two students a more evolved project in their
curriculum so they were able to learn about operating systems internals.

%
% year 2005
%

\section{Year 2005}

Then, for the next year, the two students prepared a complete microkernel
design with two complete courses on kernel design and Intel Architecture
programming.

Moreover, other students joined the project:
C\'edric Aubouy
  \footnote{aubouy\_c@epita.fr},
Fabien Le-Mentec
  \footnote{le-men\_f@epitech.net} and
Renaud Lienhart
  \footnote{lienha\_r@epita.fr}.

The project was composed of six steps, from the bootstrap, passing by
the kernel internals including memory management, task management etc..
to the servers with an IDE device driver and finally a FAT file system.

Once again, the whole project was a success but the kaneton people
noticed that the students took much time doing boring work like
filling in header files, dealing with versionning problems, writing
makefiles and shell scripts etc..

Moreover, the courses were too messy and the students had difficulties
to make the relation between kaneton design and microprocessor's
architecture implementation.

Finally, kaneton people started to implement a kaneton microkernel
reference in C language.

%
% year 2006
%

\section{Year 2006}

People joined the project:
Matthieu Bucchianeri
  \footnote{bucchi\_m@epita.fr} and
Renaud Voltz
  \footnote{voltz\_r@epita.fr}.

These students actively contributed to the kaneton implementation.

This year, kaneton people decided to introduce a development environment,
based on the kaneton implementation reference, including everything
necessary to set up a collaborative kernel development.

While, previously, the students had to write the entire microkernel
and servers from scratch, this year, students only had to write precise
parts of the microkernel.

These major modifications in the kaneton project were made to
allow students to explore advanced fields like distributed systems
aspects and/or microkernel's security.

Finally, courses were added to the EPITA System, Network and Security
specialization curriculum including microprocessor's architectures,
kernels history etc..

The kaneton microkernel implementation, in 2006, counted
\footnote{Estimations were realized with the UNIX program \textit{sloccount}.}
about \textit{7,000} lines for the core and about \textit{2,000} lines for the
microprocessor's architecture implementation.

Nevertheless, the microkernel is not complete yet, so a full implementation
of about \textit{15,000} lines can be expected which seems correct
considering the kaneton microkernel advanced design.

%
% year 2007
%

\section{Year 2007}

People joined the project:
Pierre Duteil
  \footnote{duteil\_p@epita.fr} and
Julian Pidancet
  \footnote{pidanc\_j@epita.fr}.

As the kaneton reference implementation was much more advanced as in
2006, the students were given more code and then focused only on
interesting and system-related parts.

Students totally implemented the physical and virtual memory
management, the event and timers, the thread manager, the scheduler
(with a round-robin algorithm) and the IPC. The implementation was on
IA-32 architecture.

The course is now given at Realtime \& Embedded Systems specialization
in addition of System, Network and Security specialization, for a
total of about 50 students.

The project is evaluated using a testsuite of about a hundred
tests. Many students completed a working microkernel, able to run tasks
and to implement drivers and services.

The reference implementation has grown\footnote{Estimations were
realized with the UNIX program \textit{sloccount}.} to \textit{9,000}
lines of code for the core and \textit{5,500} lines for the
microprocessor's architecture implementation on Intel's IA-32.

The kaneton reference was able to start modules (as standalone
binaries) in userland and to communicate between them.
