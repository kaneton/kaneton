%
% ---------- header -----------------------------------------------------------
%
% project       kaneton
%
% license       kaneton
%
% file          /home/mycure/kaneton/view/book/kaneton/implementation.tex
%
% created       julien quintard   [mon dec 17 12:05:06 2007]
% updated       julien quintard   [thu may 14 16:34:18 2009]
%

%
% ---------- implementation ---------------------------------------------------
%

\chapter{Implementation}
\label{chapter:implementation}

This chapter specifies some general rules enabling the reader to explore
the kaneton implementation without any difficulty.

\newpage

%
% ---------- text -------------------------------------------------------------
%

%
% interface
%

\section{Interface}

As explained earlier, the kaneton microkernel is subdivided into several
managers. Each manager provides an interface to manipulate the kaneton object
it is in charge of or something the manager provides an abstraction for.

The naming scheme used for these provided functions is normalised and
detailed below.

% manager

\subsection*{Manager}

Every manager must provide functions for initialisation and cleanup.

Therefore, the \code{initialize()} function initialises a manager given
parameters though, most of the time, the manager does not take any parameter.

The \code{clean()} function, cleans the manager from anything so that it
can return to a stable and clean state.

% information

\subsection*{Information}

The following functions are provided for displaying information on either
the manager, its objets or both.

The function \code{show()} displays information on a given identified object.

On the other hand, the \code{dump()} function displays information on the
manager, including a dump of every object held by the manager.

% object

\subsection*{Object}

The function \code{reserve()} allocates an object given some properties
whilst the \code{release()} function releases it.

The function \code{clone()} creates a copy of the given object. Cloning an
object implies cloning every object this object holds or depends on.

The \code{object()} function is used for retrieving a kaneton object given its
identifier. Note that this function is private to the manager though it is
sometimes used by other intrusive managers.

% attributes

\subsection*{Attributes}

The following functions enables the caller to access an object attribute.

The function \code{give()} gives the ownership to another entity whilst
the function \code{flush()} releases every object previously reserved.

In addition, the kaneton managers generally provide functions for modifying
a property of a given identified object. Such functions typically take the
name of the property to modify.

Finally, every manager provides an \code{get()} function which
returns the state of an object's property while \code{set()} enables the
caller to update a property.

%
% organisation
%

\section{Organisation}

The \location{kaneton/} directory is organised as follows:

\begin{verbatim}
  core/
  include/
  library/
  machine/
  modules/
\end{verbatim}

The \location{library/} directory contains the microkernel-specific C library
whilst the \location{include/} directory contains links to the include
directories of every component.

The \location{core/} directory obviously contains everything related to the
core including the source code and header files. This directory contains
an \location{include/} directory with the core header files and a directory
for every manager.

The \location{machine/} directory contains three subdirectories, one for each
machine-related component:

\begin{verbatim}
  architecture/
  glue/
  platform/
\end{verbatim}

The \location{glue/} directory contains subdirectories for every couple
platform/architecture supported. Each of these directories contains the
source code files, an \location{include/} directory for the header files and
a \location{layout/} directory which contains information used by the linker
in order to place the different elements at precise memory locations.

The \location{platform/} and \location{architecture/} directories are organised
the same way with subdirectories for every supported platforms and
architectures, respectively.

Finally the \location{modules/} directory contains conditionally-linked
static modules \ie{} pieces of code that add a functionality to the
microkernel without being intrusive.
