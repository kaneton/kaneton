%
% ---------- header -----------------------------------------------------------
%
% project       kaneton
%
% license       kaneton
%
% file          /home/mycure/kaneton/view/book/kaneton/portability.tex
%
% created       julien quintard   [tue jun 19 15:44:42 2007]
% updated       julien quintard   [mon may 19 23:09:54 2008]
%

%
% ---------- portability ------------------------------------------------------
%

\chapter{Portability}
\label{chapter:portability}

In this chapter we will describe the kaneton portability system.

\newpage

%
% ---------- text -------------------------------------------------------------
%

%
% background
%

\section{Background}

The kaneton portability system is very different from what other kernels
rely on to ensure good portability of the kernel on many architectures.

kaneton people believe that portability systems can be classified into one
of the two following categories:

\begin{enumerate}
  \item
    The first type of portability system consists in letting the
    machine-dependent source code defining for example the whole memory manager
    for its precise architecture.

    \-

    If the kernel is ported on another architecture, then the whole memory
    manager must be rewritten.

    \-

    This kind of system enables kernel developers to fully optimize every
    data structure to better fit every architecture features and needs. On
    the other hand, if a bug is detected in a generic part of the memory
    manager, then every machine-dependent implementation must be corrected.

    \-

    The problem of such a portability system is the code redundancy in each
    architecture implementation of the memory manager.
  \item
    The second type of portability system consists in the definition of
    a machine interface. This machine interface defines common machine-specific
    basic operations like flushing the \name{Translation Lookaside Buffer},
    installing a new virtual address space \etc{}

    \-

    Indeed, these basic operations are present on about every modern
    architectures. Nevertheless, some architectures --- currently existing or
    not ---- may possibly not fit this interface.
\end{enumerate}

%
% kaneton
%

\section{kaneton}

The two previous portability concepts are actually used in some kernels but
kaneton people were not satisfied by those.

Therefore, they decided to design a specific kaneton portability system
perfectly fitting in the kaneton requirements based on the kaneton microkernel
design. Indeed, the kaneton microkernel is composed of objects on which
the kernel applies modifications according to system calls.

Since the objects form the kernel state, the machine-dependent code should
act on these objects. Designers decided that the machine-dependent code
should be called every time a kaneton object is either created, modified or
destroyed.

The kaneton portability system is therefore based on an interface. However,
this interface, rather than being well established on well-known architectures
functionalities, follows the implicit interface dictated by the kaneton
managers.

In other words, rather than calling the machine-dependent code to create
a new page directory, the as manager will call the machine-dependent code
each time an address space object is created. Therefore, the
machine-dependent code might be able to set up an appropriate page directory
each time an address space object is reserved.

Having said that the objects represented the kernel state, the
machine-dependent code must be able to include machine-dependent data into
these objects. For instance, a thread object, from the core point of view,
does not contain the registers necessary for storing the state of the current
execution context. The machine-dependent code must therefore include some
fields in the thread object structure.

This portability system is experimental and needs more experiments to be
validated.

Theoritically, this portability system seems to be far better than the others
as there is no code redundancy and there are no constraints for the
machine-dependent code.

%
% interface
%

\section{Interface}

The kaneton portability system relies on c-prepreccesor macros and
macro-functions as it enables the machine-dependent source code to
\textit{link} itself with the core without anything more.

Therefore, everything is transparent as the core functions use a macro-function
and do not know what is called.

Note that these macro-functions are generally handled by the glue component
as it is the most suitable for knowing what to do when dealing with
portability depending on the underlying platform and architecture.

\function{machine\_include}{\argument{manager}}
	 {
	   This macro-function can be used by the core to include
	   machine-dependent code for the precise manager \argument{manager}.

	   \-

	   More precisely, this enables the machine-dependent code to declare
	   machine-dependent variables in the core.
	 }

\function{machine\_data}{\argument{object}}
	 {
	   This macro-function can be used by the machine-dependent code
	   to add extra fields into the kaneton object type \argument{object}.
	 }

\function{machine\_call}{\argument{manager},
                         \argument{function},
                         \argument{args...}}
	 {
	   This macro-function can be used by the core to call the
	   machine-dependent function \argument{function} of the manager
	   \argument{manager}.
	 }

%
% machine
%

\section{Machine}

As said before, the glue component is responsible for setting the links between
the core and the platform and architecture components through the portability
macro-functions.

Below is presented a common example of how to use these macros in the
thread object.

\begin{verbatim}
  typedef struct
  {
    t_prior                       prior;

    machine_data(o_thread);
  }                               o_thread;
\end{verbatim}

In this example, the thread object structure is only composed of a priority
attribute which is not sufficient. The architecture probably needs to
adds fields for storing the registers state i.e the context.

In order to do that, the glue component has to define the
\code{machine\_data()} macro-function:

\begin{verbatim}
  #define         machine_data(_object_)                                  \
    machine_data_##_object_()
\end{verbatim}

This way, every call to \code{machine\_data()} will be redirected to another
specific macro-function for the given object type \ie{} \code{o\_thread}.

Then, for each object type, a macro-function must be defined. The following
illustrates the case of the \code{o\_thread} object type.

\begin{verbatim}
  #define         machine_data_o_thread()                                 \
    struct                                                                \
    {                                                                     \
      t_vaddr             interrupt_stack;                                \
                                                                          \
      union                                                               \
      {                                                                   \
        t_x87_state       x87;                                            \
        t_sse_state       sse;                                            \
      }                   u;                                              \
    }                     machdep;
\end{verbatim}

On that architecture, the developers decided to store the thread context
on a specific stack at the \code{interrupt\_stack} address.

Therefore, once the initial call to the \code{machine\_data()} macro-function
is processed, the \code{o\_thread} structure will look like:

\begin{verbatim}
  typedef struct
  {
    t_prior                       prior;

    struct
    {
      t_vaddr             interrupt_stack;

      union
      {
        t_x87_state       x87;
        t_sse_state       sse;
      }                   u;
    }                     machdep;
  }                               o_thread;
\end{verbatim}

Finally, the core will behave normally, not knowing that additional fields
have been added by the machine-related components whilst the architecture
component for instance will be able to access its machine-specific fields
without any difficulty.
