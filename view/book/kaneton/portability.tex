%
% ---------- header -----------------------------------------------------------
%
% project       kaneton
%
% license       kaneton
%
% file          /home/mycure/kaneton/view/book/kaneton/portability.tex
%
% created       julien quintard   [tue jun 19 15:44:42 2007]
% updated       julien quintard   [tue jun 19 15:45:01 2007]
%

%
% ---------- portability ------------------------------------------------------
%

\chapter{Portability}
\label{chapter:portability}

In this chapter we will describe the kaneton portability system designed
to perfectly fit the kaneton requirements.

\newpage

%
% ---------- text -------------------------------------------------------------
%

kaneton people wanted to design a powerful microkernel easily portable
on every microprocessor's architecture.

Therefore, a portability system had to be defined to fit the whole kaneton
design and implementation.

The different portability systems can be classified into two categories.

The first type of portability system consists in letting the
machine-dependent source code defining for example the whole memory manager
for its precise architecture so it can optimize every data structure to
better fit this architecture.

The problem of such a portability system is the code redundancy in each
architecture implementation of the memory manager.

This kind of portability system seemed quite ugly to kaneton people
even if many kernels use this one to optimize the kernel implementation
on every architecture. For this reason, this solution was not adopted
for the kaneton microkernel.

The second type of portability system consists in the definition of
an architecture interface. This architecture interface defines common
architectures basic operations like flushing the Translation Lookaside Buffer,
installing a new virtual memory, etc. Indeed, these basic operations
are present on about every modern architectures. Nevertheless, some
architectures - currently existing or not - may possibly not fit this
interface.

This portability system is better than the previous one but still is too
limitated. Once again, for this reason, the kaneton people decided not
to adopt this system.

Therefore, kaneton people decided to design a specific kaneton portability
system perfectly fitting in the kaneton requirements based on the kaneton
microkernel design.

The kaneton portability system was designed based on the kaneton microkernel
design. Indeed, the kaneton microkernel is composed of objects in
kaneton terms.

The kaneton objects describe the core's internal state. For this reason,
the architecture-dependent code should only perform actions based on
the modification of the kaneton objects' states.

Therefore, the architecture-dependent code should be called every time
a kaneton object is either created, modified or destroyed. Indeed, these
three cases perfectly describe the core's internel state changes.

The kaneton portability system is so equivalent to the one which uses
an architecture interface. Nevertheless, the logic used was reversed.
Indeed, rather than building an interface based on the architectures, the
kaneton portability system uses an interface based on the kaneton managers
interfaces.

Moreover, the kaneton portability system goes further with the possibility
for the architecture-dependent code to transparently include data in the
kaneton objects.

In other words, rather than calling the machine-dependent code to create
a new page directory, the as manager will call the machine-dependent code
each time an address space object is created. Therefore, the
machine-dependent code could build an appropriate page directory each time
an address space object is reserved.

This portability system is experimental but seems correct.

Theoritically, this portability system is perfect since there is no
code redundancy and there are no constraints for the machine-dependent
code.

%
% interface
%

\subsubsection{Interface}

The whole kaneton portability system was implemented using the
c-preprocessor facilities.

Therefore, everything is transparent from the core point of view.

\function{machdep\_include}{(\argument{manager})}
	 {
	   This macro-function can be used by the core to include
	   machine-dependent code for the precise manager \argument{manager}.
	 }

\function{machdep\_data}{(\argument{object})}
	 {
	   This macro-function can be used by the machine-dependent code
	   to add data in the kaneton object \argument{object}.
	 }

\function{machdep\_call}{(\argument{manager},
                          \argument{function},
                          \argument{args...})}
	 {
	   This macro-function can be used by the core to call the
	   machine-dependent function \argument{function} of the manager
	   \argument{manager}.
	 }

To better undestand how the kaneton portability system works, the reader
should take a look at the file
\textit{kaneton/include/arch/[architecture]/machdep.h}
and an example of manager using the portability system like the address
space manager
\textit{kaneton/include/arch/[architecture]/core/as.h}.
