
-- guidelines?
The project was primarily designed by two students in computer science,
\textit{Julien Quintard} and \textit{Jean-Pascal Billaud}.

These two students previously actively contributed to the development
of a nanokernel-based operating system project in a French research laboratory.
This system was not powerful enough from the design point of view.

Therefore, the two students started the design of a new microkernel
by their own, called \textbf{kaneton}, for an educational purpose.

The design was based on five fundamental guidelines.

First, the kaneton project is built to become a pedagogical project.
Therefore, kaneton must be as understandable as it could be so that everyone
interested by kernel internals can go through the design and implementation,
and actually understand how it works. This understandable property can be
achieved through a very clear and coherent design. Moreover, the implementation
should be written using modern tools and techniques to make the code
as generic as possible and easily readable.

The microkernel was particularly designed to be portable on many architectures.
The designers tried to develop a portability system powerful enough to port
kaneton on any, existing or not, architectures.

The third point is about maintainability. Although, microkernel based operating
systems are based on a modular design, kaneton designers also wanted the
microkernel itself to be modular and maintainable.

Fourth, the kaneton microkernel must be designed to fit distributed operating
systems requirements. Indeed, the kaneton microkernel was developed in order
to design and implement a distributed operating system named \textbf{kayou}.
This point led to many specific choices in the kaneton microkernel design.

Fifth, kaneton people wanted like to break some well-known kinds of computer
science rules. Indeed, for instance, many computer scientists consider that
the source code plays the role of the project documentation. Also, for many
low-level programmers, the kernel boot source code and more generally the
kernel source code cannot be understandable, clear and coherent as it is
related to low-level parts: microprocessor, devices etc.

kaneton people paid particular attention to the microkernel source code to be
easily understandable, maintainable and extendable. Moreover, kaneton
people tried to write documentation for every part of the project.

Notice that building an educational microkernel project is nothing innovative.
Indeed few other projects already exist; the most popular being \textit{MINIX}
from \textit{Vrije Universiteit}, \textit{NachOS} from \textit{Berkeley
University} or \textit{PintOS} from \textit{Stanford University}.

kaneton people tried to design and implement a modern microkernel since, the
\textit{MINIX} microkernel for example, do not use modern development tools.
Moreover, the kaneton source code is heavily commented and use modern
languages techniques while trying to stay easily understandable.

XXX d'abord un design propre et apres on optimise a la limite
--
