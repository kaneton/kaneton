%
% ---------- header -----------------------------------------------------------
%
% project       kaneton
%
% license       kaneton
%
% file          /home/mycure/kaneton/view/book/kaneton/boot.tex
%
% created       julien quintard   [mon dec 17 21:36:28 2007]
% updated       julien quintard   [sun feb 10 18:28:52 2008]
%

%
% ---------- boot -------------------------------------------------------------
%

\chapter{Boot}
\label{chapter:label}

This chapter contains the kaneton boot specifications. Every bootloader
willing to run a kaneton microkernel instance needs to comply to these
specifications.

\newpage

%
% ---------- text -------------------------------------------------------------
%

The kaneton microkernel is not directly launched when the computer is
turned on. Indeed, a \textbf{bootloader} first set up an execution environment
so that the kernel can be launched properly. The bootloader takes some
\textbf{inputs} which represents additional files: configuration files,
execution files etc. For example, the first input the bootloader uses is
the kaneton microkernel binary file which is loaded and launched by the
bootloader itself. In addition, the second input must be the \textit{mod}
service which is launched by the kernel.

The kaneton microkernel is launched with an \textbf{init} structure
as argument. This structure is described next.

\begin{verbatim}
  typedef struct
  {
    t_paddr                       mem;
    t_psize                       memsz;

    t_paddr                       kcode;
    t_psize                       kcodesz;

    t_paddr                       mcode;
    t_psize                       mcodesz;
    t_vaddr                       mlocation;
    t_vaddr                       mentry;

    t_paddr                       init;
    t_psize                       initsz;

    t_inputs*                     inputs;
    t_psize                       inputssz;

    t_uint32                      nsegments;
    s_segment*                    segments;
    t_psize                       segmentssz;

    t_uint32                      nregions;
    s_region*                     regions;
    t_psize                       regionssz;

    t_uint32                      ncpus;
    s_cpu*                        cpus;
    t_psize                       cpussz;
    i_cpu                         bsp;

    t_paddr                       kstack;
    t_psize                       kstacksz;

    t_paddr                       alloc;
    t_psize                       allocsz;

    machine_data(init);
  }                               t_init;
\end{verbatim}

This structure informs the kernel about the memory layout i.e the location
of the different elements in memory as these locations vary according to
the machine.

Note that size fields must be aligned on \texttt{PAGESZ}. Indeed, the
core memory managers behave at the byte level. It is the machine responsibility
to call the core with properly aligned sizes.

%
% core
%

\section{Core}

% memory

\subsection*{Memory}

The \texttt{mem} and \texttt{memsz} fileds specify the offset and the size of
the underlying hardware's RAM.

The \texttt{mem} attribute is very likely to be set to zero but could
vary on specific platforms.

% kernel code

\subsection*{Kernel Code}

The two fields \texttt{kcode} and \texttt{kcodesz} specify the physical memory
location and size of the kernel code.

% mod service

\subsection*{\textit{mod} Service}

The \textit{mod} service is the very first server launched by the kaneton
microkernel. This server is responsible for creating and starting the
other servers.

Fields \texttt{mcode} and \texttt{mcodesz} specify the location of the physical
memory area containing the \textit{mod} service code.

\texttt{mlocation} contains the virtual memory address the code area must
be mapped whilst \texttt{mentry} contains the virtual address of the code's
entry point.

These fields were introduced so that the parsing of the \textit{mod} binary
is performed by the bootloader. This way, the kernel does not have take
care of handling multiples executable file formats like \textit{ELF},
\textit{COFF} etc.

Indeed, the kernel receives the \textit{init} structure, creates a new task,
maps the \textit{mod} service code and points the task's thread to the
entry point.

% init structure

\subsection*{\textit{init} Structure}

The \texttt{init} and \texttt{initsz} fields contain the location and size
of the \textit{init} structure itself.

% inputs

\subsection*{Inputs}

Inputs are additional files passed to the \textit{mod} service.

Theses files are gathered together in a single memory area specified through
the \texttt{input} and \texttt{inputsz} fields.

This area first contains metadata with the \texttt{t\_inputs} structure:

\begin{verbatim}
  typedef struct
  {
    t_uint32                      ninputs;
  }                               t_inputs;
\end{verbatim}

The \texttt{ninputs} field of the metadata obviously indicates the number
of inputs. These inputs follow the metadata as explained next.

Inputs are actually organised in an array of elements composed of the input
metadata and the input contents. The input metadata is described by the
\texttt{t\_init} structure:

\begin{verbatim}
  typedef struct
  {
    char*                         name;
    t_psize                       size;
  }                               t_input;
\end{verbatim}

Everything related to inputs is packed in a single location so that passing
these information to the \textit{mod} service is as simple as passing the
address and size of this memory area i.e \texttt{inputs} and \texttt{inputssz}.

% segments

\subsection*{Segments}

Segments passed by the bootloader to the kernel indicate the zones of
physical memory which are already used. Thus, the kaneton microkernel can
initialise its memory managers, especially the segment manager, according
to those zones.

The \texttt{nsegments} attribute indicates the number of segments in the
array located in the memory area specified by \texttt{segments} and
\texttt{segmentssz}.

Elements of the array of segment are of the following type:

\begin{verbatim}
  typedef struct
  {
    t_paddr                       address;
    t_psize                       size;
    t_perms                       perms;
  }                               s_segment;
\end{verbatim}

% regions

\subsection*{Regions}

Regions provided through the \textit{init} structure indicate the kernel
which memory locations are already mapped.

The kernel can use these information for initialising its memory managers,
in this case the region manager, so that data structures are coherent.

As for the segments, the \texttt{nregions} regions are gathered in an
array located at \texttt{regions} of size \texttt{regionssz}.

Every element of the region array are of the following type:

\begin{verbatim}
  typedef struct
  {
    t_uint32                      segment;

    t_vaddr                       address;
    t_paddr                       offset;
    t_vsize                       size;
    t_opts                        opts;
  }                               s_region;
\end{verbatim}

Note that the \texttt{segment} field of this last structure correspond to
an index in the array of segments.

% processors

\subsection*{Processors}

The \texttt{ncpus} field indicates the number of cpu elements in the array
located at \texttt{cpus} of size \texttt{cpussz}. Each element is of the
following form:

\begin{verbatim}
  typedef struct
  {
    i_cpu                         cpuid;
  }                               s_cpu;
\end{verbatim}

Additionally, the \texttt{bsp} field indicates the identifier of the boot
processor.

% kernel stack

\subsection*{Kernel Stack}

The kernel stack is specified through the \texttt{kstack} and
\texttt{kstacksz}.

% allocator's pre-reserved memory

\subsection*{Allocator's Pre-Reserved Memory}

The memory area specified by \texttt{alloc} and \texttt{allocsz} is used
by the kaneton microkernel for performing allocations in order to set up
the memory managers.

Indeed, when the kaneton microkernel starts, the memory managers are
not initialised and hence cannot provide memory management functionalities.
Traditional kernels tend to rely on a specifically designed physical memory
manager for this purpose. This design leads to an ugly implementation.

kaneton people wanted to avoid that and decided to rely on a pre-reserved
memory area provided by the bootloader.

%
% machine
%

\section{Machine}

The reader would have probably noticed the use of the \texttt{machine\_data()}
macro-function in the \textit{init} structure.

Indeed, the \textit{init} structure can include machine-dependent information
that will be later used by the kaneton machine components.

For instance, the \textit{IA-32} kaneton bootloader includes, through
the \texttt{machine\_data()} macro-function, the following information in
the \textit{init} structure:

\begin{verbatim}
  #define         machine_data_init()                                     \
    struct                                                                \
    {                                                                     \
      t_ia32_gdt                  gdt;                                    \
      t_ia32_directory            pd;                                     \
    }
\end{verbatim}
