%
% ---------- header -----------------------------------------------------------
%
% project       kaneton
%
% license       kaneton
%
% file          /home/mycure/kaneton/view/book/kaneton/boot.tex
%
% created       julien quintard   [mon dec 17 21:36:28 2007]
% updated       julien quintard   [mon dec 17 23:52:00 2007]
%

%
% ---------- boot -------------------------------------------------------------
%

\chapter{Boot}
\label{chapter:label}

This chapter contains the kaneton boot specifications. Every bootloader
willing to run a kaneton microkernel instance needs to XXX

\newpage

%
% ---------- text -------------------------------------------------------------
%

XXX
The kaneton microkernel is not directly launched when the computer is
turned on. Indeed, a \textbf{bootloader} first set up an execution environment
so that the kernel can be launched properly. The \textit{bootloader} takes
some \textbf{inputs} which represents additional files: configuration
files, execution files etc. For example, the first \textit{input} the
\textit{bootloader} uses is the kaneton microkernel binary file which is
loaded and launched by the \textit{bootloader} itself.

The kaneton microkernel manages tasks which are classified into \textit{four}
categories according to the privileges they get on the system. These different
classes of task are described next.

A \textbf{program} is the lowest priviliged task of the system. The common
user programs are the well known UNIX{\copyright} binaries like
\textit{/bin/ls}, \textit{/bin/sh} etc. A \textbf{service} is a microkernel
server which provides a logical service. For example, a service could be the
\textit{Virtual File System} which dispatches the calls to the file systems
servers. A \textbf{driver} is a service which performs hardware communication.
For example, a \textit{Wireless driver} is a \textit{driver} in the kaneton
terms. Finally, a \textbf{kernel} is a task which is a kind of super-driver
in which it has full rights on the whole underlying hardware.

For more information on the design on the kaneton microkernel, please
refer to the paper \textit{The kaneton microkernel project}.
XXX

XXX[expliquer t\_init dand book:kaneton et ensute l'implem dans arch:XXX]
