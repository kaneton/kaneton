%
% ---------- header -----------------------------------------------------------
%
% project       kaneton
%
% license       kaneton
%
% file          /home/mycure/kaneton/view/book/kaneton/kaneton.tex
%
% created       julien quintard   [mon may 14 19:56:45 2007]
% updated       julien quintard   [sun dec 16 17:23:52 2007]
%

%
% ---------- setup ------------------------------------------------------------
%

%
% path
%

\def\path{../..}

%
% template
%

%%
%% licence       kaneton licence
%%
%% project       kaneton
%%
%% file          /home/mycure/kaneton/view/templates/book.tex
%%
%% created       julien quintard   [wed mar  1 23:45:22 2006]
%% updated       julien quintard   [thu may  4 12:36:54 2006]
%%

%
% class
%

\documentclass[10pt,a4wide]{book}

%
% packages
%

\usepackage[english]{babel}
\usepackage[T1]{fontenc}
\usepackage{a4wide}
\usepackage{fancyheadings}
\usepackage{multicol}
\usepackage{indentfirst}
\usepackage{graphicx}
\usepackage{color}
\usepackage{xcolor}
\usepackage{verbatim}

\usepackage{aeguill}

\usepackage[Lenny]{../../../tools/latex/fncychap}

\pagestyle{fancy}

\setlength{\footrulewidth}{0.3pt}
\setlength{\parindent}{0.3cm}
\setlength{\parskip}{2ex plus 0.5ex minus 0.2ex}

%
% logos
%

\newcommand{\logos}
  {
    \begin{center}
      \includegraphics[scale=0.8]{../../logos/kaneton.pdf}
    \end{center}
  }

%
% colors
%

\definecolor{functioncolor}{rgb}{0.40,0.00,0.00}
\definecolor{commandcolor}{rgb}{0.00,0.00,0.40}
\definecolor{verbatimcolor}{rgb}{0.00,0.40,0.00}
\definecolor{noticecolor}{rgb}{0.87,0.84,0.02}

%
% function
%

\newcommand\function[3]{
  \begin{tabular}{p{0.2cm}p{13.8cm}}
  & {\color{functioncolor}\textbf{#1}}#2
  \end{tabular}

  \begin{tabular}{p{1cm}p{13cm}}
  & #3
  \end{tabular}}

%
% align
%

\newcommand\align[1]{
  \\ & \hspace{#1}}

%
% argument
%

\newcommand\argument[1]{\textit{#1}}

%
% command
%

\newcommand\command[2]{
  \begin{tabular}{p{0.2cm}p{13.8cm}}
  & {\color{commandcolor}\textbf{#1}}
  \end{tabular}

  \begin{tabular}{p{1cm}p{13cm}}
  & #2
  \end{tabular}}

%
% notice
%

\newcommand\notice[1]{
  {\color{noticecolor}\textbf{Notice}}

  \begin{tabular}{p{0.2cm}p{13.8cm}}
  & #1
  \end{tabular}}

%
% example
%

\newcommand\example[1]{
  \textit{Example:}

  \begin{tabular}{p{0.2cm}p{13.8cm}}
  & \textit{#1}
  \end{tabular}}

%
% warning XXX
%

%
% verbatim stuff
%

\makeatletter

\renewcommand{\verbatim@font}
  {\ttfamily\footnotesize\color{verbatimcolor}\selectfont}

\def\verbatim@processline{\hskip15ex\the\verbatim@line\par}

\makeatother

%
% header
%

\rhead{}
\rfoot{\scriptsize{The kaneton microkernel project}}

\date{\scriptsize{\today}}


%
% header
%

\lhead{\scriptsize{The kaneton microkernel}}
\rhead{}

%
% title
%

\title{The kaneton microkernel
       \logos}

%
% authors
%

\author{\small{Julien Quintard}}

%
% document
%

\begin{document}

%
% title
%

\maketitle

%
% --------- text --------------------------------------------------------------
%

This document describes the kaneton microkernel research project design
and implementation.

\-

This document should be used by every student willing implement the
kaneton educational microkernel as well as by people looking for more
details on the kaneton microkernel design and implementation.

\-

All the kaneton documents are available on
the official website
  \footnote{http://www.kaneton.org}.

%
% toc
%

\toc

%
% chapters
%

%
% ---------- header -----------------------------------------------------------
%
% project       kaneton
%
% license       kaneton
%
% file          /home/mycure/kane...book/assignments/future/introduction.tex
%
% created       julien quintard   [fri may 23 21:47:38 2008]
% updated       julien quintard   [fri oct 24 17:05:56 2008]
%

%
% ---------- introduction -----------------------------------------------------
%

\chapter{Introduction}
\label{chapter:introduction}

This chapter briefly introduces the purpose of this documentation
and the assignments in general

\newpage

%
% ---------- text -------------------------------------------------------------
%

The \name{kaneton} educational project enables students to develop their own
micro-kernel as a way of understanding operating systems internals.

As anyone can imagine, such a project takes a huge amount of time and
motivation. While the motivation will anyway play an important role in the
success of students' project, the time spent can be greatly reduced if
students focuse on implementing some specific parts rather than developing a
complete micro-kernel from scratch.

Indeed, as we will see later in this document, the current \name{kaneton}
educational project comes with a student \term{snapshot} which contains
a complete development environment as well as the source code skeleton of
the kernel.

As enthusiastic computer scientists, \name{kaneton} authors, maintainers and
teachers can understand than some people prefer working on their own
micro-kernel design and implementation, from scratch. All we can wish to
such people is enough motivation to keep working on their project long
enough to be satisfied, luck and hard work.

Either way, going through the \name{kaneton} micro-kernel documentation
should be a waste of time. Especially, people interested in developing their
own project from scratch could take a look at the \name{kaneton} design
in case they like it enough to implement it their way.

The remaining of this document is organised as follows. \reference{Chapter
\ref{chapter:requirements}} lists what students willing to undertake the
project should know beforehand. \reference{Chapter \ref{chapter:support}}
details the multiple ways for students to get help. \reference{Chapter
\ref{chapter:k0}} presents the first project stage. Then \reference{Chapter
\ref{chapter:snapshot}} presents the student snapshot while \reference{Chapter
\ref{chapter:setup}} introduces the development environment and its set-up.
Then, \reference{Chapter \ref{chapter:k1}}, \reference{Chapter
\ref{chapter:k2}}, \reference{Chapter \ref{chapter:k3}} and \reference{Chapter
\ref{chapter:k4}} details the assignments of the different stages. Finally,
\reference{Chapter \ref{chapter:further}} discusses what students could do
after having undertaken such a project.

%
% ---------- header -----------------------------------------------------------
%
% project       kaneton
%
% license       kaneton
%
% file          /home/mycure/kaneton/view/book/kaneton/background.tex
%
% created       julien quintard   [tue jun 19 11:48:36 2007]
% updated       julien quintard   [mon may 19 23:09:01 2008]
%

%
% ---------- background -------------------------------------------------------
%

\chapter{Background}
\label{chapter:background}

This chapter introduces some notions about operating systems and kernels.
However, this chapter assumes the reader already has a substential knowledge
about operating systems internals.

\newpage

%
% ---------- text -------------------------------------------------------------
%

%
% distributed operating system
%

\section{Distributed Operating System}

To understand the kaneton microkernel design and implementation, the reader
must first understand what are the distributed operating systems requirements
and more generally what are distributed systems inherent problems.

A distributed operating system can be defined through different ways and
the reader should probably find different definitions in the literature. In
this document, we assume that the main goal of a distributed operating
system is to connect users and resources in a transparent, open, and scalable
way.

For instance, when a user launches a program, the distributed operating system
decides on which machine of the network to run it. More generally, for
any action to perform, the distributed operating system tries to find the
most suitable place for executing it depending on resources availability.

While the processes generally communicate with each other on the same
machine, in a distributed operating system, they also communicate with the
other processes of other machines as illustrated in \reference{Figure
\ref{figure:distributed-operating-system}}.

\begin{figure}[h]
  \begin{center}
    \includegraphics[scale=0.5]{\path/figures/distributed-operating-system.pdf}
    \caption{Communications in a distributed operating system.}
    \label{figure:distributed-operating-system}
  \end{center}
\end{figure}

In \reference{Figure \ref{figure:distributed-operating-system}}, the machine
named \name{E} is running three processes. One is just running some
computations performing no communication, another is receiving a message
from the latter which is also sending two messages to processes over the
network.

These communications can be simple point-to-point communication for example
between a client and a web server while other can be distributed operating
system specific messages intended to make decisions on the location of a file
on the distributed file system for instance.

We saw that machines on a distributed operating system are running processes
which communicate with any other processes of any other node. Every machine
needs an operating system to organize the computer resources and to make
them communicate.

Microkernel-based operating systems are the best suitable systems
for building a distributed operating system as these systems are modular
and reliable by nature.

%
% microkernel
%

\section{Microkernel}

A microkernel is a kernel type designed to be modular hence more reliable than
monolithic kernels. Instead of building a large binary object containing the
kernel itself, the drivers, the file systems \etc{} microkernels try to be as
small as possible, providing only the fundamental services for managing
memory, execution contexts, communication and input/output. Besides, the other
classical operating system services are provided by userland processes with
special privileges.

This very specific design is very interesting for security, maintainability
and extensibility. Moreover, this type of design perfectly fits
distributed operating systems requirements where special processes
communicate with other node's processes to organize the whole distributed
system resources.

\reference{Figure \ref{figure:microkernel}} illustrates a microkernel view
with different layers.

\begin{figure}[h]
  \begin{center}
    \includegraphics[scale=0.4]{\path/figures/microkernel-behaviour.pdf}
    \caption{Communications in a microkernel hierarchy.}
    \label{figure:microkernel}
  \end{center}
\end{figure}

The whole microkernel design is based on the client-server communication
model where clients ask servers to perform a specific task. For instance,
to print some text to the screen, a user program has to ask the \name{tty}
server to do it for him because the user program itself does not have any
right to perform such an action.

To avoid complications and deadlocks the microkernel follows a fundamental
rule which restrict communication from clients only to more or equal
privileged tasks. On the \reference{Figure \ref{figure:microkernel}}, the
\name{vfs} server can ask the \name{ata} server and the \name{kernel} to
peform a task but the kernel cannot ask a less privileged server to do
something for him.

%
% ---------- header -----------------------------------------------------------
%
% project       kaneton
%
% license       kaneton
%
% file          /home/mycure/kaneton/view/book/kaneton/terminology.tex
%
% created       julien quintard   [mon jun 18 15:45:43 2007]
% updated       julien quintard   [tue jun 19 15:32:03 2007]
%

%
% ---------- terminology ------------------------------------------------------
%

\chapter{Terminology}
\label{chapter:terminology}

In this chapter the kaneton terminology is detailed. This terminology is
quite different from other kernel and operating system projects on many points.

\newpage

%
% ---------- text -------------------------------------------------------------
%

The kaneton terminology was introduced for making communication easier.
Indeed, it is sometimes difficult to communicate about a technical
field like kernels design and development with terms as specific as
``architecture-independent soure code'', ``module of the \textit{mod}
service'', ``contiguous area of free virtual memory'' etc.

The reader should notice that these terms are complex and a bit confusing
when used together in the same sentence. Therefore, kaneton people decided to
introduce a well defined terminology to make things clear and more
understandable.

First, note that the kaneton project is actually composed of two projects:
the \textit{kaneton microkernel educational project} which provides everything
necessary to students willing to learn about kernels internals; and the
\textit{kaneton microkernel research project} which focuses on designing and
implementing a powerful, reliable, flexible microkernel. Obviously these
two projects are highly related as the kaneton educational project relies on
the implementation of the kaneton research project.

This section tries to detail the terminology inherent to these projects by
classifying the terms according to their context. Indeed, the remaining of
this chapter is divided into two categories: \textit{Design} and
\textit{Implementation}. The \textit{Design} section contains the kaneton
general terminology while the \textit{Implementation} section contains some
naming rules the kaneton microkernel research project implementation follows.

%
% design
%

\section{Design}

Although microkernel-based operating systems are, by nature, modular, kaneton
people wanted the microkernel itself to be modular, subdivided into logical
parts. This subdivision was introduced to make the whole microkernel clearer
and more understandable.

The kaneton microkernel is thus divided into \textbf{managers}. These
managers are generally responsible of a \textbf{kaneton object} type but there
exist managers which manage something else or just create an abstraction over
other kaneton managers. A kaneton object represents a logical and fundamental
kernel entity. These objects are described later in this section.

\textit{Figure XXX} illustrates the decomposition of the microkernel into
multiples managers.

XXX [figure]

Note that this decomposition has no direct relation with \textit{Object
Programming}. Indeed, even if kaneton designers tried to reduce the
dependencies between the managers, some managers remain intrusive as they
access the data structures of one or more other managers.

A kaneton object represents a kernel entity. Every object is identified by
a \textbf{kaneton identifier} and protected by a \textbf{kaneton capability}
over the operating and distributed system.

Below are listed the most important objects:

\begin{itemize}
  \item
    A \textbf{segment} represents a continuous area of physical memory.
  \item
    A \textbf{region} represents a continuous area of virtual memory which
    maps a part of or a whole \textit{segment}.
  \item
    An \textbf{as} or \textbf{address space} represents a set of physical
    memory areas which can potentially be accessed through a set of virtual
    memory addresses.

    \-

    An \textit{as} is composed of a set of \textit{segments} and a set of
    \textit{regions}.
  \item
    A \textbf{task} is represents a complete execution context.

    \-

    However, a \textit{task} is not an active entity as it is not the
    one scheduled. Indeed, a \textit{task} is actually composed of an
    \textit{address space} and one or more \textit{threads}.
  \item
    A \textbf{thread} represents the active execution context in a
    \textit{task}.
  \item
    An \textbf{event} describes an external event including hardware
    events like interrupts as well as software events also known as
    \textit{system calls} or \textit{syscalls}.
\end{itemize}

The reader should notice that there is a hierarchical relation between
these objects as it is illustrated by \textit{Figure XXX}.

XXX [figure]

There is another kaneton object which is internally widely used: the
\textbf{set} object. A \textit{set} object is an abstract data structure
managed by a dedicated manager, the \textit{set manager}. The \textit{set
manager} is used by the other kaneton managers for storing data without taking
care of how it is technically done. The set concept was introduced to make the
microkernel code as clear as a pseudo code. For more information on the
\textit{set manager}, please refer to the book \textit{The kaneton
microkernel :: core}.

The kaneton microkernel was designed to be ported on many different - existing
or not - architectures. Therefore, the microkernel is divided into two
major components: the \textbf{core} and the \textbf{machine}. The \textit{core}
designs the kaneton source code which is independent from the underlying
computer. For more information on the \textit{core} component, please refer
to the appopriate book \textit{The kaneton microkernel :: core}. On the
contrary, the \textit{machine} component contains the source code related to
the underlying specific hardware.

Since a microprocessor architecture can be used on different mother boards
with various chipsets, the \textit{machine} component is also divided into
a \textbf{platform} which represents the board package; and a
\textbf{architecture} which represents the microprocessor.

Note that the behaviour of the \textit{machine} component can change depending
on the \textit{platform}/\textit{architecture} coupling. Therefore, the
\textbf{glue} was introduced to deal with the multiple
\textit{platform}/\textit{architecture} combinations.

\textit{Figure XXX} illustrates this decomposition. For more information on
the portability system, please refer to the \textit{Chapter
\ref{chapter:portability}}.

XXX [figure]

The kaneton microkernel provides powerful features for distributing computing
including security through capabilities but also communication through a
complete message passing system. Additionally, the microkernel already
integrates the notion of communicating node. As such, the microkernel knows
that the \textbf{distributed system} is composed of
  \textbf{machines}\footnote{The term \textit{machine} is used to represents
                             both the architecture-dependent microkernel's
                             source code and a computer in the distributed
                             system.},
each machine running a kaneton microkernel-based operating system. Moreover,
the communicating \textit{tasks} are named \textbf{nodes} in the distributed
system context.

\textit{Figure XXX} illustrates a distributed system composed of \textit{five}
machines and multiple \textit{nodes} which communicate with each other.

XXX [figure]

The kaneton microkernel is not directly launched when the computer is
turned on. Indeed, a \textbf{bootloader} first set up an execution environment
so that the kernel can be launched properly. The \textit{bootloader} takes
some \textbf{inputs} which represents additional files: configuration
files, execution files etc. For example, the first \textit{input} the
\textit{bootloader} uses is the kaneton microkernel binary file which is
loaded and launched by the \textit{bootloader} itself.

The kaneton microkernel manages tasks which are classified into \textit{four}
categories according to the privileges they get on the system. These different
classes of task are described next.

A \textbf{program} is the lowest priviliged task of the system. The common
user programs are the well known UNIX{\copyright} binaries like
\textit{/bin/ls}, \textit{/bin/sh} etc. A \textbf{service} is a microkernel
server which provides a logical service. For example, a service could be the
\textit{Virtual File System} which dispatches the calls to the file systems
servers. A \textbf{driver} is a service which performs hardware communication.
For example, a \textit{Wireless driver} is a \textit{driver} in the kaneton
terms. Finally, a \textbf{kernel} is a task which is a kind of super-driver
in which it has full rights on the whole underlying hardware.

For more information on the design on the kaneton microkernel, please
refer to the paper \textit{The kaneton microkernel project}.

%
% implementation
%

\section{Implementation}

This section focuses on the terminology used in the \textit{kaneton microkernel
research project} implementation. Obviously, this terminology is also used
in the educational project's context.

As explained in the previous section, the kaneton microkernel is subdivided
into multiples managers.

Each manager provides an interface to manipulate the kaneton object or
something the manager is in charge. The naming scheme used for these provided
functions is detailed below.

Recall that the interface provided by the set of kaneton managers is not
absolutely identical to the system call interface. Indeed, some functions
are private to the microkernel. Moreover, some functions of a manager's
interface must only be called by the manager itself and should not be called
by the other managers.

The couple of functions \textbf{initialize}() and \textbf{clean}() initialize
and clean the manager, respectively.

The function \textbf{show}() displays information on a given identified object
while the function \textbf{dump}() displays information on every objects
hold by the manager.

The function \textbf{reserve}() reserves an object given some properties. On
the contrary, the function \textbf{release}() releases it.

The function \textbf{clone}() clones an object. It is important to understand
that cloning an object does not just mean generating an identical object.
Indeed, cloning an object implies cloning every object this object holds or
depends on.

The function \textbf{give}() gives an object to another entity. The function
\textbf{flush}() cleans every object previously reserved.

The function \textbf{get}() is used to retrieve a kaneton object given its
identifier. Note that this function is private to the manager.

In addition, the kaneton managers generally provide functions for modifying
a property of a given identified object. Moreover, every manager provides
a function \textbf{attribute}() which can be used to retrieve the state
of an object's property.

% %
% ---------- header -----------------------------------------------------------
%
% project       kaneton
%
% license       kaneton
%
% file          /home/mycure/kaneton/view/book/kaneton/boot.tex
%
% created       julien quintard   [mon dec 17 21:36:28 2007]
% updated       julien quintard   [wed dec 19 21:33:14 2007]
%

%
% ---------- boot -------------------------------------------------------------
%

\chapter{Boot}
\label{chapter:label}

This chapter contains the kaneton boot specifications. Every bootloader
willing to run a kaneton microkernel instance needs to comply to these
specifications.

\newpage

%
% ---------- text -------------------------------------------------------------
%

The kaneton microkernel is not directly launched when the computer is
turned on. Indeed, a \textbf{bootloader} first set up an execution environment
so that the kernel can be launched properly. The bootloader takes some
\textbf{inputs} which represents additional files: configuration files,
execution files etc. For example, the first input the bootloader uses is
the kaneton microkernel binary file which is loaded and launched by the
bootloader itself. In addition, the second input must be the \textit{mod}
service which is launched by the kernel.

The kaneton microkernel is launched with an \textbf{init} structure
as argument. This structure is described next.

\begin{verbatim}
  typedef struct
  {
    t_paddr                       mem;
    t_psize                       memsz;

    t_paddr                       kcode;
    t_psize                       kcodesz;

    t_paddr                       mcode;
    t_psize                       mcodesz;
    t_vaddr                       mlocation;
    t_vaddr                       mentry;

    t_paddr                       init;
    t_psize                       initsz;

    t_inputs*                     inputs;
    t_psize                       inputssz;

    t_uint32                      nsegments;
    s_segment*                    segments;
    t_psize                       segmentssz;

    t_uint32                      nregions;
    s_region*                     regions;
    t_psize                       regionssz;

    t_uint32                      ncpus;
    s_cpu*                        cpus;
    t_psize                       cpussz;
    i_cpu                         bsp;

    t_paddr                       kstack;
    t_psize                       kstacksz;

    t_paddr                       alloc;
    t_psize                       allocsz;

    machine_data(init);
  }                               t_init;
\end{verbatim}

This structure informs the kernel about the memory layout i.e the location
of the different elements in memory as these locations vary according to
the machine.

Note that size fields must be aligned on \texttt{PAGESZ}. Indeed, the
core memory managers behave at the byte level. It is the machine responsibility
to call the core with properly aligned sizes.

%
% core
%

\section{Core}

% memory

\subsection*{Memory}

The \texttt{mem} and \texttt{memsz} fileds specify the offset and the size of
the underlying hardware's RAM.

The \texttt{mem} attribute is very likely to be set to zero but could
vary on specific platforms.

% kernel code

\subsection*{Kernel Code}

The two fields \texttt{kcode} and \texttt{kcodesz} specify the physical memory
location and size of the kernel code.

% mod service

\subsection*{\textit{mod} Service}

The \textit{mod} service is the very first server launched by the kaneton
microkernel. This server is responsible for creating and starting the
other servers.

Fields \texttt{mcode} and \texttt{mcodesz} specify the location of the physical
memory area containing the \textit{mod} service code.

\texttt{mlocation} contains the virtual memory address the code area must
be mapped whilst \texttt{mentry} contains the virtual address of the code's
entry point.

These fields were introduced so that the parsing of the \textit{mod} binary
is performed by the bootloader. This way, the kernel does not have take
care of handling multiples executable file formats like \textit{ELF},
\textit{COFF} etc.

Indeed, the kernel receives the \textit{init} structure, creates a new task,
maps the \textit{mod} service code and points the task's thread to the
entry point.

% init structure

\subsection*{\textit{init} Structure}

The \texttt{init} and \textit{initsz} fields contain the location and size
of the \textit{init} structure itself.

% inputs

\subsection*{Inputs}

Inputs are additional files passed to the \textit{mod} service.

Theses files are gathered together in a single memory area specified through
the \texttt{input} and \texttt{inputsz} fields.

This area first contains metadata with the \texttt{t\_inputs} structure:

\begin{verbatim}
  typedef struct
  {
    t_uint32                      ninputs;
  }                               t_inputs;
\end{verbatim}

The \texttt{ninputs} field of the metadata obviously indicates the number
of inputs. These inputs follow the metadata as explained next.

Inputs are actually organised in an array of elements composed of the input
metadata and the input contents. The input metadata is described by the
\texttt{t\_init} structure:

\begin{verbatim}
  typedef struct
  {
    char*                         name;
    t_psize                       size;
  }                               t_input;
\end{verbatim}

Everything related to inputs is packed in a single location so that passing
these information to the \textit{mod} service is as simple as passing the
address and size of this memory area i.e \texttt{inputs} and \texttt{inputssz}.

% segments

\subsection*{Segments}

Segments passed by the bootloader to the kernel indicate the zones of
physical memory which are already used. Thus, the kaneton microkernel can
initialise its memory managers, especially the segment manager, according
to those zones.

The \texttt{nsegments} attribute indicates the number of segments in the
array located in the memory area specified by \texttt{segments} and
\texttt{segmentssz}.

Elements of the array of segment are of the following type:

\begin{verbatim}
  typedef struct
  {
    t_paddr                       address;
    t_psize                       size;
    t_perms                       perms;
  }                               s_segment;
\end{verbatim}

% regions

\subsection*{Regions}

Regions provided through the \textit{init} structure indicate the kernel
which memory locations are already mapped.

The kernel can use these information for initialising its memory managers,
in this case the region manager, so that data structures are coherent.

As for the segments, the \texttt{nregions} regions are gathered in an
array located at \texttt{regions} of size \texttt{regionssz}.

Every element of the region array are of the following type:

\begin{verbatim}
  typedef struct
  {
    t_uint32                      segment;

    t_vaddr                       address;
    t_paddr                       offset;
    t_vsize                       size;
    t_opts                        opts;
  }                               s_region;
\end{verbatim}

Note that the \texttt{segment} field of this last structure correspond to
an index in the array of segments.

% processors

\subsection*{Processors}

The \texttt{ncpus} field indicates the number of cpu elements in the array
located at \texttt{cpus} of size \texttt{cpussz}. Each element is of the
following form:

\begin{verbatim}
  typedef struct
  {
    i_cpu                         cpuid;
  }                               s_cpu;
\end{verbatim}

Additionally, the \texttt{bsp} field indicates the identifier of the boot
processor.

% kernel stack

\subsection*{Kernel Stack}

The kernel stack is specified through the \texttt{kstack} and
\texttt{kstacksz}.

% allocator's pre-reserved memory

\subsection*{Allocator's Pre-Reserved Memory}

The memory area specified by \texttt{alloc} and \texttt{allocsz} is used
by the kaneton microkernel for performing allocations in order to set up
the memory managers.

Indeed, when the kaneton microkernel starts, the memory managers are
not initialised and hence cannot provide memory management functionalities.
Traditional kernels tend to rely on a specifically designed physical memory
manager for this purpose. This design leads to an ugly implementation.

kaneton people wanted to avoid that and decided to rely on a pre-reserved
memory area provided by the bootloader.

%
% machine
%

\section{Machine}

The reader would have probably noticed the use of the \texttt{machine\_data()}
macro-function in the \textit{init} structure.

Indeed, the \textit{init} structure can include machine-dependent information
that will be later used by the kaneton machine components.

For instance, the \textit{IA-32} kaneton bootloader includes, through
the \texttt{machine\_data()} macro-function, the following information in
the \textit{init} structure:

\begin{verbatim}
  #define         machine_data_init()                                     \
    struct                                                                \
    {                                                                     \
      t_ia32_gdt                  gdt;                                    \
      t_ia32_directory            pd;                                     \
    }
\end{verbatim}

%
% ---------- header -----------------------------------------------------------
%
% project       kaneton
%
% license       kaneton
%
% file          /home/mycure/kane...ture/kernels/portability/portability.tex
%
% created       julien quintard   [fri oct 24 17:31:58 2008]
% updated       julien quintard   [fri may 21 15:08:57 2010]
%

%
% ---------- setup ------------------------------------------------------------
%

%
% path
%

\def\path{../../..}

%
% template
%

%%
%% copyright     (c) julien quintard
%%
%% project       kaneton
%%
%% file          /home/mycure/kaneton/view/templates/lecture.tex
%%
%% created       julien quintard   [sat nov 19 17:13:03 2005]
%% updated       julien quintard   [fri dec  2 22:36:34 2005]
%%

%
% class
%

\documentclass[8pt]{beamer}

%
% packages
%

\usepackage{pgf,pgfarrows,pgfnodes,pgfautomata,pgfheaps,pgfshade}
\usepackage{colortbl}
\usepackage{times}
\usepackage{amsmath,amssymb}
\usepackage{graphics}
\usepackage{graphicx}
\usepackage{color}
\usepackage{xcolor}
\usepackage[english]{babel}
\usepackage{enumerate}
\usepackage[latin1]{inputenc}

%
% style
%

\usepackage{beamerthemesplit}
\setbeamercovered{dynamic}

%
% verbatim font
%

\definecolor{verbatimcolor}{rgb}{0,0.4,0}

\makeatletter
\renewcommand{\verbatim@font}
  {\ttfamily\footnotesize\color{verbatimcolor}\selectfont}
\makeatother

%
% new line
%

\newcommand{\nl}[0]{\vspace{0.4cm}}

%
% date
%

\date{\today}

%
% logos
%

\pgfdeclareimage[interpolate=true,width=34pt,height=18pt]
                {epita}{../../logos/epita}
\pgfdeclareimage[interpolate=true,width=49pt,height=18pt]
                {upmc}{../../logos/upmc}
\pgfdeclareimage[interpolate=true,width=25pt,height=18pt]
                {lse}{../../logos/lse}

\newcommand{\logos}
  {
    \pgfuseimage{epita}
  }

%
% institute
%

\institute
{
  \inst{1} kaneton microkernel project
}

%
% table of contents at the beginning of each section
%

\AtBeginSection[]
{
  \begin{frame}<beamer>
   \frametitle{Outline}
    \tableofcontents[current]
  \end{frame}
}

%
% table of contents at the beginning of each subsection
%

\AtBeginSubsection[]
{
  \begin{frame}<beamer>
   \frametitle{Outline}
    \tableofcontents[current,currentsubsection]
  \end{frame}
}


%
% title
%

\title{Portability}

%
% document
%

\begin{document}

%
% title frame
%

\begin{frame}
  \titlepage
\end{frame}

%
% outline frame
%

\begin{frame}
  \frametitle{Outline}

  \tableofcontents
\end{frame}

%
% ---------- text -------------------------------------------------------------
%

%
% introduction
%

\section{Introduction}

\begin{frame}
  \frametitle{Introduction}

  Nowadays, a lot of CPU types exist (IA-32, IA-32\_64, IA-64, ARM, ARM, MIPS, PowerPC, SH, m68k, Blackfin, ...) and a lot of machines uses these CPUs, in different ways.

  \-

  The modern operating systems are trying to cope with that, and they tend to support all these platforms, so that a user can use the same system on all the different machines he uses.

  \-

  Portability requires some design in the kernel, to avoid rewriting everything from scratch and maintaining a separate branch for each supported architecture.

  \-

  This course describes what are the differences between machines, and how to design a kernel so it can be ported on several platforms.

\end{frame}

\section{Architecture}
\subsection{Machine architecture vs CPU architecture}

\begin{frame}
  \frametitle{Machine architecture vs CPU architecture}

  From a kernel point of view, there are two kind of architectures on a machine :

  \begin{itemize}
  \item The CPU Architecture
  \item The Machine/Platform Architecture
  \end{itemize}

  \-

  The following slides will explain what makes a CPU architecture specific, what makes a Machine architecture specific, and how a kernel can be designed so it can be as easily as possible ported on several machines.  

\end{frame}

\subsection{Microprocessor architecture}

\begin{frame}
  \frametitle{Microprocessor architecture}

  The architecture of a microprocessor defines :

  \begin{itemize}
  \item Its intructions set
  \item Its registers
  \item Its operational modes/behaviour
  \end{itemize}

  \-

  It's the interface between the software and the CPU itself, so it's basically what will be documented by the manufacturer in the datasheet of the CPU.

\end{frame}

\begin{frame}
  \frametitle{Microprocessor architecture}
  
  Several microprocessors models can share the same architecture.

  \-

  IA32 (commonly called x86) processors are manufactured by Intel, AMD, Via, \ldots

  \-

  ARM designs CPU architectures, manufacturers can use these specifications to make their own CPU.

\end{frame}

\begin{frame}
  \frametitle{Microprocessor architecture extensions}

  Some manufacturers are expanding one CPU architecture to provide more features, but keeping the main CPU architecture as a base.

  \-

  A code that was made to run on the base architecture will work on all the derivatives architectures, but not the opposite.

  \-

  This approach has been used quite a lot, for example, with IA-32 architecture, and the additional instruction sets (MMX, SSE, SSE2, SSE3 from Intel, 3dNow from AMD)

  \-

  A portable kernel would use these features if they are available, but must provide a software alternative in the other case. That way, the kernel can benefit from the performance gain provided by these instructions set, but doesn't depend on their presence to work.

\end{frame}

\begin{frame}
  \frametitle{CPU Architecture classes}

  Some categories have been made to distinguish families of CPU architectures :

  \begin{itemize}
  \item RISC Architecture: Reduced Instruction Set Computer - This family of CPU architectures focus on providing a really basic and simple set of instructions, and they try to optimize each instruction as much as possible, so that even something quite complex, that will require several instructions, could be faster than on a CPU where one single instruction would have been used. Such an architecture contains generally a lot of registers, so that programs can work as much as possible on data stored in registers instead of doing several memory accesses.

  \item CISC Architecture: Complex Instruction Set Computer - This family of CPU architectures focus more on providing a consequent set of instructions, so that some usual operations, that could be done using several elementary operations, are available through a single instruction. For that reason, the instruction set of a CISC CPU contains quite a lot of operations. This kind of CPU actually quite often contains a RISC CPU and a Microcode that describes how to do each instruction.

  \end{itemize}

\end{frame}

\begin{frame}
  \frametitle{Microprocessor internal and external architecture}

  All this was about external architecture.

  \-

  There is another kind of Microprocessor architecture : the internal architecture.

  \-

  This is how the CPU is actually achieving the implementation of the interface required by the external architecture.

  \-

  This is the CPU manufacturers core business, it's what makes the difference between two CPUs that have the same external architecture, like Intel vs AMD, or Via on IA-32 processors, since it's what makes the difference in terms of performance, power consumption, \ldots

\end{frame}

\subsection{Machine architecture}

\begin{frame}
  \frametitle{Machine architecture}
  
  A microprocessor can't work alone. It has interfaces to communicate with other components : memory buses, peripherial buses, interrupts lines, \ldots

  \-

  This provides a way to connect the memory required by the microprocessor, and some peripherials that can be useful in a machine, such as time-sources/clocks, user interfaces, data storage, network interfaces, \ldots

  \-

  The firmware provided in a machine is also something specific to the machine architecture since it provides services to interface with the hardware to the operating system.

  \-

  All these things are external components that are connected to the CPU in a specific way. A machine architecture describes the set of components that are used, and how they are connected together.

\end{frame}

\begin{frame}
  \frametitle{Machine architecture}

  The IBM-PC architecture is the most known architecture that uses an IA-32 CPU, but there are actually some other. SGI, for instance, made a machine based on an IA-32 CPU, but that was different from the IBM-PC architecture (SGI 320) since it didn't have a BIOS, but a firmware based on their own ARCS firmware they used on their other MIPS machines.

  \-

  More recently, Apple released their own architecture based on IA-32, where they also don't use the BIOS, but instead provide a firmware called EFI.

  \-

  Some other CPU architectures are used in several different machine architectures. For example, the MIPS R5000 CPU was used in the SGI Indy architecture, and in the Sony Playstation 2 architecture. These two machines use the same CPU architecture but they are totally different. An OS written for the SGI Indy won't work on the Sony Playstation 2.

  \-

  Almost every single Windows mobile based smartphone is a different Machine architecture, although they all share the ARM CPU architecture.

\end{frame}

\section{Kernel splitting}
\subsection{Independant part}

\begin{frame}
  \frametitle{Independant part}

  The independant part of a kernel consists in all the code that will work on every machine.

  \-

  It basically consists of all the code handling the kernel's concepts. For example, a task is a high level concept that we can find in almost all kernels. A portable kernel will be able to manage tasks on all the architectures it supports. Some of the task management done by the kernel consists in the manipulation of structures describing the task in a generic way. All this code does not rely on machines specific features, the same code will run and work on all machines. This is an independant code.

  \-

  Of course, creating a task will certainly require some specific actions as well. The goal of a portable kernel will be to isolate as much as possible those parts of the code, and to put as much things as possible in the independant part, to avoid redundancy in the dependant part.

  \-

  No ASM code can be found in the independant part since ASM is by definition dependant of the CPU architecture.

\end{frame}

\subsection{Dependant part}

\begin{frame}
  \frametitle{Dependant part}

  The dependant part of a kernel contains all the code that is aimed to support a specific feature of a CPU, a machine, or a specific hardware.

  \-

  Writing entries in the MMU cache is something specific to the CPU. It's generally done using some assembly instructions that are different from one CPU architecture to another. That's why it's a code that depends on the CPU architecture.

  \-

  Setting up the platform to trigger an interrupt on a regular basis is something specific to the Machine architecture. In general, a machine contains a dedicated hardware, connected to a CPU interrupt line. That's why configuring it is done by some code that depends on the Machine architecture.

  \-

  These examples are not always true : if a CPU embeds its own clock source that can trigger an interrupt internally, then the code to configure it is not Machine architecture dependant, but it is CPU architecture dependant.

\end{frame}

\subsection{Board support package (BSP)}

\begin{frame}
  \frametitle{Dependant part}

  Some kernels, especially the kernels for embedded operating systems, introduced the Board Support Package (BSP) concept.
  
  \-

  For example, Windows CE, Microsoft's kernel for embedded systems, is working on 3 CPU architectures : IA-32, ARM9, SuperH, but it runs on a lot of devices with a lot of various hardware devices. Each device has its own machine architecture, and its own set of peripherials. For that reason, each platform requires some specific code to make the kernel to work on it.

\end{frame}


\section{Kaneton portability}

\begin{frame}
  \frametitle{Kaneton Portability}

  The Kaneton microkernel is designed to be portable. For that reason, the code is splitted in several sections :

  \begin{itemize}
  \item Core
  \item Machine
  \begin{itemize}
  \item Architecture
  \item Platform
  \item Glue
  \end{itemize}
  \end{itemize}

\end{frame}

\subsection{Core}

\begin{frame}
  \frametitle{Kaneton Core}

  In the Kaneton design, the machine independant code is called Core.

  \-

  This section of code contains all the kernel code which is not specific to a machine. It contains the high level functions and code that are used for all the kernel internal operations, such as :
  \begin{itemize}
  \item Allocate physical memory
  \item Allocate and map virtual memory
  \item Create a new task
  \item Send a message between two tasks
  \item \ldots
  \end{itemize}

  Most of these operations don't require to do any CPU specific operations, but some of them do. For example, mapping virtual memory to physical memory requires to configure the MMU through the caches, this is something different for each CPU. For that reason, every function in the Core module of Kaneton will make a call to a potential machine dependant code.

  \-
  
  This is achieved through a macro called machine\_call.

\end{frame}

\subsection{Machine}

\begin{frame}
  \frametitle{Kaneton Architecture}
  
  This section of the Kaneton code contains all the code specific to the CPU architecture.

  \-

  This is mainly the code that will require to use specific assembly calls (inline assembly) or the code that works on data structures imposed by the CPU.

\end{frame}

\begin{frame}
  \frametitle{Kaneton Platform}
  
  This section of the Kaneton code contains all the code specific to the machine architecture.

  \-

  This is the code that will handle some peripherials attached to the CPU in a specific architecture, and that are used by the kernel itself.

\end{frame}

\begin{frame}
  \frametitle{Kaneton Glue}

  In the Kaneton design, the independant code (Core) calls the dependant code.

  \-

  To make it possible, the dependant code must have a generic interface, so that the Core code doesn't contain specifically any call to a function specific to an architecture.

  \-

  In Kaneton, the independant code and the dependant code functions share exactly the same prototypes, so that the Core code can call the Machine code in a generic way.

  \-

  For that reason, a wrapper was required, to implement the correct interface, that would then call the Architecture and the Machine code accordingly. This is the role of Glue.

\end{frame}

\section{Conclusion}

\begin{frame}
  \frametitle{Conclusion}

  In order to make a portable kernel, one must well distinguish what code can be reused for every machine, and what code is specific to the machine he is working on.

  \-

  This problem has now been addressed in most of the modern operating systems kernels (Linux, BSD, Windows CE, \ldots) with more or less style, but it's generally much better when a kernel has been designed to be portable from the beginning.

  \-

  This portability is very important nowadays, since the hardware is evolving fast. Making a new kernel without thinking about portability is a bad strategy since it makes it dependant on the durability of the architecture it is being made for.

  
\end{frame}


%
% bibliography
%

\begin{frame}[allowframebreaks]
  \frametitle{Bibliography}

  \bibliographystyle{amsplain}
  \bibliography{\path/bibliography/bibliography}
\end{frame}

\end{document}

%
% ---------- header -----------------------------------------------------------
%
% project       kaneton
%
% license       kaneton
%
% file          /home/mycure/kaneton/view/book/kaneton/references.tex
%
% created       julien quintard   [tue jun 19 15:36:51 2007]
% updated       julien quintard   [sun dec 16 17:33:44 2007]
%

%
% ---------- references -------------------------------------------------------
%

\chapter{References}
\label{chapter:references}

This chapter contains a listing of the kaneton documents that readers might
be interested in as well.

\newpage

%
% ---------- text -------------------------------------------------------------
%

This document aimed at presenting the kaneton microkernel and its organisation.

The books listed below go further by describing more precisely a specific
component of the kaneton microkernel:

\begin{enumerate}
  \item
    \textbf{The kaneton microkernel :: core}

    \-

    That document focuses on the design and implementation of the kaneton
    microkernel core through the different managers;
  \item
    \textbf{The kaneton microkernel :: \textit{architecture}}

    \-

    Theses documents tackle the implementation of kaneton on the given
    \textit{architecture}. Additionally, these documents contain information
    on platforms coupled with this architecture.

    \-

    Thes documents can be found in two forms, either public or private. Indeed,
    these documents contain information students must not have access to.

    \-

    The architectures currently documented are listed below:

    \begin{itemize}
      \item
	\textbf{ia32}

	\-

	\textit{Platforms}: \textbf{ibm-pc}
    \end{itemize}
\end{enumerate}

\input{licenses.tex}

\end{document}

--

--

bouger les figures dans view/figures
XXX dans book::kaneton, archi supportees, core, glue, machdep system, syscall
    interface, dans book::core, un fichier par manager.

--

expliquer t_init dand book:kaneton et ensute l'implem dans arch:XXX
