%
% ---------- header -----------------------------------------------------------
%
% project       kaneton
%
% license       kaneton
%
% file          /home/mycure/kaneton/view/book/kaneton/kaneton.tex
%
% created       julien quintard   [mon may 14 19:56:45 2007]
% updated       julien quintard   [fri jun  1 15:02:48 2007]
%

%
% path
%

\def\path{../..}

%
% template
%

%%
%% licence       kaneton licence
%%
%% project       kaneton
%%
%% file          /home/mycure/kaneton/view/templates/book.tex
%%
%% created       julien quintard   [wed mar  1 23:45:22 2006]
%% updated       julien quintard   [thu may  4 12:36:54 2006]
%%

%
% class
%

\documentclass[10pt,a4wide]{book}

%
% packages
%

\usepackage[english]{babel}
\usepackage[T1]{fontenc}
\usepackage{a4wide}
\usepackage{fancyheadings}
\usepackage{multicol}
\usepackage{indentfirst}
\usepackage{graphicx}
\usepackage{color}
\usepackage{xcolor}
\usepackage{verbatim}

\usepackage{aeguill}

\usepackage[Lenny]{../../../tools/latex/fncychap}

\pagestyle{fancy}

\setlength{\footrulewidth}{0.3pt}
\setlength{\parindent}{0.3cm}
\setlength{\parskip}{2ex plus 0.5ex minus 0.2ex}

%
% logos
%

\newcommand{\logos}
  {
    \begin{center}
      \includegraphics[scale=0.8]{../../logos/kaneton.pdf}
    \end{center}
  }

%
% colors
%

\definecolor{functioncolor}{rgb}{0.40,0.00,0.00}
\definecolor{commandcolor}{rgb}{0.00,0.00,0.40}
\definecolor{verbatimcolor}{rgb}{0.00,0.40,0.00}
\definecolor{noticecolor}{rgb}{0.87,0.84,0.02}

%
% function
%

\newcommand\function[3]{
  \begin{tabular}{p{0.2cm}p{13.8cm}}
  & {\color{functioncolor}\textbf{#1}}#2
  \end{tabular}

  \begin{tabular}{p{1cm}p{13cm}}
  & #3
  \end{tabular}}

%
% align
%

\newcommand\align[1]{
  \\ & \hspace{#1}}

%
% argument
%

\newcommand\argument[1]{\textit{#1}}

%
% command
%

\newcommand\command[2]{
  \begin{tabular}{p{0.2cm}p{13.8cm}}
  & {\color{commandcolor}\textbf{#1}}
  \end{tabular}

  \begin{tabular}{p{1cm}p{13cm}}
  & #2
  \end{tabular}}

%
% notice
%

\newcommand\notice[1]{
  {\color{noticecolor}\textbf{Notice}}

  \begin{tabular}{p{0.2cm}p{13.8cm}}
  & #1
  \end{tabular}}

%
% example
%

\newcommand\example[1]{
  \textit{Example:}

  \begin{tabular}{p{0.2cm}p{13.8cm}}
  & \textit{#1}
  \end{tabular}}

%
% warning XXX
%

%
% verbatim stuff
%

\makeatletter

\renewcommand{\verbatim@font}
  {\ttfamily\footnotesize\color{verbatimcolor}\selectfont}

\def\verbatim@processline{\hskip15ex\the\verbatim@line\par}

\makeatother

%
% header
%

\rhead{}
\rfoot{\scriptsize{The kaneton microkernel project}}

\date{\scriptsize{\today}}


%
% header
%

\lhead{\scriptsize{The kaneton microkernel}}
\rhead{}

%
% title
%

\title{The kaneton microkernel
       \logos}

%
% authors
%

\author{\small{Julien Quintard}}

%
% document
%

\begin{document}

%
% title
%

\maketitle

%
% --------- text --------------------------------------------------------------
%

This document describes the kaneton microkernel research project.

This document should be used by every student willing implement the
kaneton microkernel as well as by people looking for more details on the
kaneton microkernel design and implementation.

All the kaneton documents are available on
the official website
  \footnote{http://www.kaneton.org}.

%
% toc
%

\tableofcontents

%
% chapters
%

%
% ---------- header -----------------------------------------------------------
%
% project       kaneton
%
% license       kaneton
%
% file          /home/mycure/kaneton/view/book/kaneton/goals.tex
%
% created       julien quintard   [fri jun  1 13:58:12 2007]
% updated       julien quintard   [mon may 19 23:09:48 2008]
%

%
% ---------- goals ------------------------------------------------------------
%

\chapter{Goals}
\label{chapter:goals}

In this chapter we will briefly introduce the kaneton microkernel
through the kaneton microkernel goals.

\newpage

%
% ---------- text -------------------------------------------------------------
%

The project was primarily designed by two students in computer science,
\name{Julien Quintard} and \name{Jean-Pascal Billaud}.

These two students previously actively contributed to the development
of a nanokernel-based operating system project in a French research laboratory.
This system was not powerful enough, especially from the design point of view.

Therefore, the two students started the design of a new microkernel
by their own, called \term{kaneton}, for educational purposes.

The design was based on five fundamental guidelines.

\begin{enumerate}
  \item
    \textbf{Educational}

    \-

    The kaneton project is built to become an educational project. The design
    as well as the implementation must therefore be as understandable as
    possible so that everyone interested in kernel internals can go through the
    documents and source code and actually understand how it works.

    \-

    This \textit{understandable} property can be achieved through a very clear
    and coherent design. Moreover, the implementation should be written using
    modern tools and techniques to make the code as generic as possible and
    easily readable.
  \item
    \textbf{Portability}

    \-

    The microkernel was particularly designed to be portable. The designers
    tried to develop a portability system powerful enough to port kaneton on
    any, existing or not, architectures.
  \item
    \textbf{Maintanability}

    \-

    Although microkernel-based operating systems rely on a modular design,
    kaneton designers also wanted the microkernel itself to be modular and
    maintainable.
  \item
    \textbf{Distributed Computing}

    \-

    The kaneton microkernel must be designed to fit distributed operating
    systems requirements. Indeed, the kaneton microkernel was developed in
    order to design and implement a distributed operating system named
    \term{kayou}.

    \-

    This point led to many specific choices in the kaneton microkernel design.
  \item
    \textbf{Demystification}

    \-

    kaneton people wanted to break some well-known kind of computer
    science rules. Indeed, for instance, many computer scientists consider
    the source code as the actual documentation. Also, for many low-level
    programmers, the kernel boot source code and more generally the
    kernel source code itself cannot be understandable, clear and coherent as
    it is related to low-level programming: microprocessor, devices \etc{}

    kaneton people paid particular attention to the microkernel source code to
    be easily understandable, maintainable and extendable. Moreover, kaneton
    people tried to write documentation for every part of the project.
\end{enumerate}

Notice that building an educational microkernel project is nothing innovative.
Indeed few other projects already exist; the most popular being \name{MINIX}
from \name{Vrije Universiteit}, \name{NachOS} from \name{Berkeley University}
or \name{PintOS} from \name{Stanford University}.

kaneton people tried to design and implement a modern microkernel since, the
original \name{MINIX} microkernel for example, do not use modern development
tools. Moreover, the kaneton source code is heavily commented and use modern
languages techniques while trying to stay easily understandable.

The educational characteristic of kaneton does not constraint it from being
optimised afterwards. kaneton people believe that implementing optimised
algorithms in the first place does not lead to maintainable implementations.

Finally, note that the kaneton project is actually composed of two projects:
the \name{kaneton microkernel \term{educational} project} which provides
everything necessary to students willing to learn about kernels internals;
and the \name{kaneton microkernel \term{research} project} which focuses on
designing and implementing a powerful, reliable, flexible microkernel.
Obviously these two projects are highly related as the kaneton educational
project relies on the implementation of the kaneton research project.

%
% ---------- header -----------------------------------------------------------
%
% project       kaneton
%
% license       kaneton
%
% file          /home/mycure/kaneton/view/book/development/history.tex
%
% created       julien quintard   [fri jun  1 14:33:16 2007]
% updated       julien quintard   [mon jun 18 12:38:23 2007]
%

%
% ---------- history ----------------------------------------------------------
%

\chapter{History}

In this chapter we detail the kaneton history from the first
year with low-level programming introduction to the last kaneton
microkernel implementation.

\newpage

%
% ---------- text -------------------------------------------------------------
%

During the kaneton history, the project evolved and courses were added
to the curriculum to make the whole kaneton project more interesting and
understandable by the students. Moreover, the educational project, which
was already targetting the \textit{EPITA}'s \textit{System, Network and
Security} major, was also used in other contexts, victim of its success
and of the very hard work achieved by kaneton people over the years.

%
% 2004
%

\section{2004}

The first year, a low-level programming introduction course named \textbf{k}
was proposed for the \textit{EPITA} Engineering School's first year students.

About fourteen hours courses were taught introducing the \textit{Intel 32-bit}
microprocessor's external architecture and low-level programming.

The students had to develop small, poor and messy device drivers for the
console and keyboard peripherals. Moreover, a tiny command interpreter was
developed by students so that a kernel action could be triggered by entering
a command.

The course was a bit chaotic but this first shot was a success.

Therefore, the students majoring in \textit{System, Network and Security}
asked the two students \textit{Julien Quintard} and \textit{Jean-Pascal
Billaud} a complete kernel project for their curriculum so that they can
learn more about operating systems internals.

%
% 2005
%

\section{2005}

The two, still students, \textit{Julien Quintard} and \textit{Jean-Pascal
Billaud} then prepared a complete microkernel design the students will have
to implement. This was the premises of the \textbf{kaneton microkernel
educational project}. Additionally, two complete courses on kernel design and
\textit{Intel 32-bit Architecture} programming were prepared.

The project was composed of six steps, from the bootstrap, passing by
the kernel internals including memory management, task management etc.
to the servers with an \textit{IDE} device driver and finally a \textit{FAT}
file system.

Notice that the majority of the students did not success in implementing a
complete scheduling system allowing the creation of user-land task. Indeed,
the best groups achieved in providing the management of kernel-land tasks only.
Therefore, the \textit{IDE} driver, \textit{FAT} file system etc. were
running in the kernel.

Inspite of this, once again, the whole project was a success. However,
kaneton people noticed that the students took much time doing boring work
like filling in header files, dealing with versionning problems, writing
\textit{Make} files and \textit{Shell} scripts etc.

Moreover, the courses were too messy and the students had difficulties
to make the relation between the kaneton design and the microprocessor's
architecture implementation.

As a result, kaneton people decided to start implementing a kaneton microkernel
reference by their own in the \textit{C} language. This implementation will
then be used to compare the behaviour of students' implementation with
the reference. Moreover, this implementation led to the creation of a
new project: the \textbf{kaneton microkernel research project}.

%
% 2006
%

\section{2006}

While \textit{Jean-Pascal Billaud} leaved the project, people joined it
starting with \textit{EPITA} last year students \textit{C\'edric Aubouy},
\textit{Renaud Lienhart} but also \textit{Fabien Le-Mentec} from
\textit{EPITECH} who knew these people from the \textit{EPITA Computer Systems
Laboratory} where they were all working together a year before.
\textit{C\'edric Aubouy} and \textit{Renaud Lienhard} were in charge of the
kernel and \textit{Intel 32-bit Architecture} courses, respectively.
\textit{Julien Quintard} was still in charge of the kaneton educational
project given to the students.

More over two \textit{EPITA} first year students joined the \textit{EPITA
Computer Systems Laboratory}, \textit{Matthieu Bucchianeri} and \textit{Renaud
Voltz}. Indeed, from this date, the \textit{EPITA Computer Systems Laboratory}
was a strong partner of the kaneton microkernel project. These two students
were hired for contributing to the development of the kaneton research project.
Moreover, these students were supposed to teach and supervise the kaneton
educational project the following year.

\textit{Matthieu Bucchianeri} and \textit{Renaud Voltz} did an amazing work
on the kaneton research project implementation. Indeed, most of the
code related to the \textit{Intel 32-bit Architecture} comes from them. In
addition, the test suite as well as many tests were written by them. Thanks
to their work.

This year, kaneton people decided to introduce a development environment,
based on the kaneton research reference implementation, including everything
necessary to set up a collaborative kernel development.

While, previously, the students had to write the entire microkernel and
servers from scratch, this year, students only had to write precise parts
of the microkernel including some set implementations, memory management,
task scheduling etc.

Few mistakes were made especially about the choice of parts the students
had to implement. Indeed, asking the students to implement set implementations
like linked-list, array etc. was a very bad idea. This year, the project
was not completed and students stopped the project before the messaging
system implementation.

A course was also added to the \textit{EPITA} \textit{System, Network
and Security} major's curriculum about microprocessors' internals. This
course was introduced and taught by \textit{Julien Quintard}.

In conclusion, the kaneton educational project was not a real success this year
and needed some modifications. For instance, the course about the \textit{Intel
32-bit Architecture} was too specific and hard to understand but also hard
to teach. Instead, kaneton people decided to introduce a more general course
about kernel principles for the next year.

The kaneton research project implementation, in 2006,
  counted\footnote{Estimations realised with the software \textit{sloccount}.}
about \textit{7,000} lines for the \textit{core} and about \textit{2,000}
lines for the \textit{Intel 32-bit Architecture} implementation.

%
% 2007
%

\section{2007}

People affiliated with the \textit{EPITA Computer Systems Laboratory} joined
the project: \textit{Pierre Duteil} and \textit{Julian Pidancet}. Moreover,
students who implemented the kaneton educational project the previous year
decided to join the project: \textit{Enguerrand Raymond} and \textit{Mathieu
S\'elari\`es}, mainly working on the \textit{MIPS Architecture} portage among
other contributions.

This year, \textit{Matthieu Bucchianeri} and \textit{Renaud Voltz} were in
charge of the educational project by giving the kaneton courses as well
as supervising students' educational implementations.

As the kaneton research implementation was much more advanced as in 2006,
the students were given more code and then focused only on interesting and
system-related parts.

Students totally implemented the physical and virtual memory management, the
event and timers, the thread manager, the scheduler and the messaging system.
As for the previous years, the implementation was based on the \textit{Intel
32-bit Architecture}.

This year, the whole kaneton educational project was also given to students
from the \textit{Realtime \& Embedded Systems} specialization, for a total
of about $50$ students. The project was evaluated using a test suite,
developed for the kaneton research project, of about a hundred tests.

This year, the kaneton educational project was an amazing success as many
students completed a working microkernel, able to run tasks and to implement
some servers running on top of the kaneton microkernel.

The kaneton research implementation has grown to \textit{9,000} lines of source
code for the \textit{core} and \textit{5,500} lines for the microprocessor's
architecture implementation on \textit{Intel 32-bit}. The kaneton research
implementation was able to start modules - as standalone binaries - in
user-land as well as to make them communicate through the kaneton messaging
system.

This year, \textit{Pierre Duteil} leaved the project.

%%
%% licence       kaneton licence
%%
%% project       kaneton
%%
%% file          /home/mycure/kaneton/view/papers/kaneton/overview.tex
%%
%% created       matthieu bucchianeri   [mon jan 30 17:09:45 2006]
%% updated       julien quintard   [thu mar  2 13:12:22 2006]
%%

%
% overview
%

\chapter{Overview}

XXX ce chapitre va vous aider a reconnaitre les fonctionnalites principale
XXX d'un kernel dans kaneton.

The kaneton microkernel is only the core of an operating system.
Main tasks like hardware drivers or user services are implemented as
\textbf{servers}. So the microkernel only has a few functionalities to
provide:

\begin{itemize}
  \item
    Memory management.
  \item
    Process management.
  \item
    Communication.
  \item
    Events.
\end{itemize}

In this chapter we will describe briefly these tasks and all the
associated managers.

%
% memory management
%

\section{Memory Management}

Handling the memory -- from virtual address space to physical
addressing -- is done by three major managers, the \textbf{as},
\textbf{segment} and \textbf{region} managers.

%
% as
%

\subsection{as}

The address space manager just manages the different address spaces
used by the kaneton tasks.

In kaneton, we call an \textbf{as - address space} a list of memory
locations referenced by a task. Each task has its own address space.

%
% segment
%

\subsection{segment}

The segment manager just manages the segments reserved by
the different kaneton entities including the kernel, the drivers etc..

In kaneton terms a \textbf{segment} is a contiguous area of reserved
physical memory.

%
% region
%

\subsection{region}

The region manager keeps track of regions used to map segments for
each address space reserved on the system.

In kaneton, a \textbf{region} is contiguous area of virtual memory
mapping a segment's part.

%
% process management
%

\section{Process Management}

XXX

%
% communication
%

\section{Communication}

XXX

%
% events
%

\section{Events}

XXX

%
% ---------- header -----------------------------------------------------------
%
% project       kaneton
%
% license       kaneton
%
% file          /home/mycure/kane...ture/kernels/portability/portability.tex
%
% created       julien quintard   [fri oct 24 17:31:58 2008]
% updated       julien quintard   [fri may 21 15:08:57 2010]
%

%
% ---------- setup ------------------------------------------------------------
%

%
% path
%

\def\path{../../..}

%
% template
%

%%
%% copyright     (c) julien quintard
%%
%% project       kaneton
%%
%% file          /home/mycure/kaneton/view/templates/lecture.tex
%%
%% created       julien quintard   [sat nov 19 17:13:03 2005]
%% updated       julien quintard   [fri dec  2 22:36:34 2005]
%%

%
% class
%

\documentclass[8pt]{beamer}

%
% packages
%

\usepackage{pgf,pgfarrows,pgfnodes,pgfautomata,pgfheaps,pgfshade}
\usepackage{colortbl}
\usepackage{times}
\usepackage{amsmath,amssymb}
\usepackage{graphics}
\usepackage{graphicx}
\usepackage{color}
\usepackage{xcolor}
\usepackage[english]{babel}
\usepackage{enumerate}
\usepackage[latin1]{inputenc}

%
% style
%

\usepackage{beamerthemesplit}
\setbeamercovered{dynamic}

%
% verbatim font
%

\definecolor{verbatimcolor}{rgb}{0,0.4,0}

\makeatletter
\renewcommand{\verbatim@font}
  {\ttfamily\footnotesize\color{verbatimcolor}\selectfont}
\makeatother

%
% new line
%

\newcommand{\nl}[0]{\vspace{0.4cm}}

%
% date
%

\date{\today}

%
% logos
%

\pgfdeclareimage[interpolate=true,width=34pt,height=18pt]
                {epita}{../../logos/epita}
\pgfdeclareimage[interpolate=true,width=49pt,height=18pt]
                {upmc}{../../logos/upmc}
\pgfdeclareimage[interpolate=true,width=25pt,height=18pt]
                {lse}{../../logos/lse}

\newcommand{\logos}
  {
    \pgfuseimage{epita}
  }

%
% institute
%

\institute
{
  \inst{1} kaneton microkernel project
}

%
% table of contents at the beginning of each section
%

\AtBeginSection[]
{
  \begin{frame}<beamer>
   \frametitle{Outline}
    \tableofcontents[current]
  \end{frame}
}

%
% table of contents at the beginning of each subsection
%

\AtBeginSubsection[]
{
  \begin{frame}<beamer>
   \frametitle{Outline}
    \tableofcontents[current,currentsubsection]
  \end{frame}
}


%
% title
%

\title{Portability}

%
% document
%

\begin{document}

%
% title frame
%

\begin{frame}
  \titlepage
\end{frame}

%
% outline frame
%

\begin{frame}
  \frametitle{Outline}

  \tableofcontents
\end{frame}

%
% ---------- text -------------------------------------------------------------
%

%
% introduction
%

\section{Introduction}

\begin{frame}
  \frametitle{Introduction}

  Nowadays, a lot of CPU types exist (IA-32, IA-32\_64, IA-64, ARM, ARM, MIPS, PowerPC, SH, m68k, Blackfin, ...) and a lot of machines uses these CPUs, in different ways.

  \-

  The modern operating systems are trying to cope with that, and they tend to support all these platforms, so that a user can use the same system on all the different machines he uses.

  \-

  Portability requires some design in the kernel, to avoid rewriting everything from scratch and maintaining a separate branch for each supported architecture.

  \-

  This course describes what are the differences between machines, and how to design a kernel so it can be ported on several platforms.

\end{frame}

\section{Architecture}
\subsection{Machine architecture vs CPU architecture}

\begin{frame}
  \frametitle{Machine architecture vs CPU architecture}

  From a kernel point of view, there are two kind of architectures on a machine :

  \begin{itemize}
  \item The CPU Architecture
  \item The Machine/Platform Architecture
  \end{itemize}

  \-

  The following slides will explain what makes a CPU architecture specific, what makes a Machine architecture specific, and how a kernel can be designed so it can be as easily as possible ported on several machines.  

\end{frame}

\subsection{Microprocessor architecture}

\begin{frame}
  \frametitle{Microprocessor architecture}

  The architecture of a microprocessor defines :

  \begin{itemize}
  \item Its intructions set
  \item Its registers
  \item Its operational modes/behaviour
  \end{itemize}

  \-

  It's the interface between the software and the CPU itself, so it's basically what will be documented by the manufacturer in the datasheet of the CPU.

\end{frame}

\begin{frame}
  \frametitle{Microprocessor architecture}
  
  Several microprocessors models can share the same architecture.

  \-

  IA32 (commonly called x86) processors are manufactured by Intel, AMD, Via, \ldots

  \-

  ARM designs CPU architectures, manufacturers can use these specifications to make their own CPU.

\end{frame}

\begin{frame}
  \frametitle{Microprocessor architecture extensions}

  Some manufacturers are expanding one CPU architecture to provide more features, but keeping the main CPU architecture as a base.

  \-

  A code that was made to run on the base architecture will work on all the derivatives architectures, but not the opposite.

  \-

  This approach has been used quite a lot, for example, with IA-32 architecture, and the additional instruction sets (MMX, SSE, SSE2, SSE3 from Intel, 3dNow from AMD)

  \-

  A portable kernel would use these features if they are available, but must provide a software alternative in the other case. That way, the kernel can benefit from the performance gain provided by these instructions set, but doesn't depend on their presence to work.

\end{frame}

\begin{frame}
  \frametitle{CPU Architecture classes}

  Some categories have been made to distinguish families of CPU architectures :

  \begin{itemize}
  \item RISC Architecture: Reduced Instruction Set Computer - This family of CPU architectures focus on providing a really basic and simple set of instructions, and they try to optimize each instruction as much as possible, so that even something quite complex, that will require several instructions, could be faster than on a CPU where one single instruction would have been used. Such an architecture contains generally a lot of registers, so that programs can work as much as possible on data stored in registers instead of doing several memory accesses.

  \item CISC Architecture: Complex Instruction Set Computer - This family of CPU architectures focus more on providing a consequent set of instructions, so that some usual operations, that could be done using several elementary operations, are available through a single instruction. For that reason, the instruction set of a CISC CPU contains quite a lot of operations. This kind of CPU actually quite often contains a RISC CPU and a Microcode that describes how to do each instruction.

  \end{itemize}

\end{frame}

\begin{frame}
  \frametitle{Microprocessor internal and external architecture}

  All this was about external architecture.

  \-

  There is another kind of Microprocessor architecture : the internal architecture.

  \-

  This is how the CPU is actually achieving the implementation of the interface required by the external architecture.

  \-

  This is the CPU manufacturers core business, it's what makes the difference between two CPUs that have the same external architecture, like Intel vs AMD, or Via on IA-32 processors, since it's what makes the difference in terms of performance, power consumption, \ldots

\end{frame}

\subsection{Machine architecture}

\begin{frame}
  \frametitle{Machine architecture}
  
  A microprocessor can't work alone. It has interfaces to communicate with other components : memory buses, peripherial buses, interrupts lines, \ldots

  \-

  This provides a way to connect the memory required by the microprocessor, and some peripherials that can be useful in a machine, such as time-sources/clocks, user interfaces, data storage, network interfaces, \ldots

  \-

  The firmware provided in a machine is also something specific to the machine architecture since it provides services to interface with the hardware to the operating system.

  \-

  All these things are external components that are connected to the CPU in a specific way. A machine architecture describes the set of components that are used, and how they are connected together.

\end{frame}

\begin{frame}
  \frametitle{Machine architecture}

  The IBM-PC architecture is the most known architecture that uses an IA-32 CPU, but there are actually some other. SGI, for instance, made a machine based on an IA-32 CPU, but that was different from the IBM-PC architecture (SGI 320) since it didn't have a BIOS, but a firmware based on their own ARCS firmware they used on their other MIPS machines.

  \-

  More recently, Apple released their own architecture based on IA-32, where they also don't use the BIOS, but instead provide a firmware called EFI.

  \-

  Some other CPU architectures are used in several different machine architectures. For example, the MIPS R5000 CPU was used in the SGI Indy architecture, and in the Sony Playstation 2 architecture. These two machines use the same CPU architecture but they are totally different. An OS written for the SGI Indy won't work on the Sony Playstation 2.

  \-

  Almost every single Windows mobile based smartphone is a different Machine architecture, although they all share the ARM CPU architecture.

\end{frame}

\section{Kernel splitting}
\subsection{Independant part}

\begin{frame}
  \frametitle{Independant part}

  The independant part of a kernel consists in all the code that will work on every machine.

  \-

  It basically consists of all the code handling the kernel's concepts. For example, a task is a high level concept that we can find in almost all kernels. A portable kernel will be able to manage tasks on all the architectures it supports. Some of the task management done by the kernel consists in the manipulation of structures describing the task in a generic way. All this code does not rely on machines specific features, the same code will run and work on all machines. This is an independant code.

  \-

  Of course, creating a task will certainly require some specific actions as well. The goal of a portable kernel will be to isolate as much as possible those parts of the code, and to put as much things as possible in the independant part, to avoid redundancy in the dependant part.

  \-

  No ASM code can be found in the independant part since ASM is by definition dependant of the CPU architecture.

\end{frame}

\subsection{Dependant part}

\begin{frame}
  \frametitle{Dependant part}

  The dependant part of a kernel contains all the code that is aimed to support a specific feature of a CPU, a machine, or a specific hardware.

  \-

  Writing entries in the MMU cache is something specific to the CPU. It's generally done using some assembly instructions that are different from one CPU architecture to another. That's why it's a code that depends on the CPU architecture.

  \-

  Setting up the platform to trigger an interrupt on a regular basis is something specific to the Machine architecture. In general, a machine contains a dedicated hardware, connected to a CPU interrupt line. That's why configuring it is done by some code that depends on the Machine architecture.

  \-

  These examples are not always true : if a CPU embeds its own clock source that can trigger an interrupt internally, then the code to configure it is not Machine architecture dependant, but it is CPU architecture dependant.

\end{frame}

\subsection{Board support package (BSP)}

\begin{frame}
  \frametitle{Dependant part}

  Some kernels, especially the kernels for embedded operating systems, introduced the Board Support Package (BSP) concept.
  
  \-

  For example, Windows CE, Microsoft's kernel for embedded systems, is working on 3 CPU architectures : IA-32, ARM9, SuperH, but it runs on a lot of devices with a lot of various hardware devices. Each device has its own machine architecture, and its own set of peripherials. For that reason, each platform requires some specific code to make the kernel to work on it.

\end{frame}


\section{Kaneton portability}

\begin{frame}
  \frametitle{Kaneton Portability}

  The Kaneton microkernel is designed to be portable. For that reason, the code is splitted in several sections :

  \begin{itemize}
  \item Core
  \item Machine
  \begin{itemize}
  \item Architecture
  \item Platform
  \item Glue
  \end{itemize}
  \end{itemize}

\end{frame}

\subsection{Core}

\begin{frame}
  \frametitle{Kaneton Core}

  In the Kaneton design, the machine independant code is called Core.

  \-

  This section of code contains all the kernel code which is not specific to a machine. It contains the high level functions and code that are used for all the kernel internal operations, such as :
  \begin{itemize}
  \item Allocate physical memory
  \item Allocate and map virtual memory
  \item Create a new task
  \item Send a message between two tasks
  \item \ldots
  \end{itemize}

  Most of these operations don't require to do any CPU specific operations, but some of them do. For example, mapping virtual memory to physical memory requires to configure the MMU through the caches, this is something different for each CPU. For that reason, every function in the Core module of Kaneton will make a call to a potential machine dependant code.

  \-
  
  This is achieved through a macro called machine\_call.

\end{frame}

\subsection{Machine}

\begin{frame}
  \frametitle{Kaneton Architecture}
  
  This section of the Kaneton code contains all the code specific to the CPU architecture.

  \-

  This is mainly the code that will require to use specific assembly calls (inline assembly) or the code that works on data structures imposed by the CPU.

\end{frame}

\begin{frame}
  \frametitle{Kaneton Platform}
  
  This section of the Kaneton code contains all the code specific to the machine architecture.

  \-

  This is the code that will handle some peripherials attached to the CPU in a specific architecture, and that are used by the kernel itself.

\end{frame}

\begin{frame}
  \frametitle{Kaneton Glue}

  In the Kaneton design, the independant code (Core) calls the dependant code.

  \-

  To make it possible, the dependant code must have a generic interface, so that the Core code doesn't contain specifically any call to a function specific to an architecture.

  \-

  In Kaneton, the independant code and the dependant code functions share exactly the same prototypes, so that the Core code can call the Machine code in a generic way.

  \-

  For that reason, a wrapper was required, to implement the correct interface, that would then call the Architecture and the Machine code accordingly. This is the role of Glue.

\end{frame}

\section{Conclusion}

\begin{frame}
  \frametitle{Conclusion}

  In order to make a portable kernel, one must well distinguish what code can be reused for every machine, and what code is specific to the machine he is working on.

  \-

  This problem has now been addressed in most of the modern operating systems kernels (Linux, BSD, Windows CE, \ldots) with more or less style, but it's generally much better when a kernel has been designed to be portable from the beginning.

  \-

  This portability is very important nowadays, since the hardware is evolving fast. Making a new kernel without thinking about portability is a bad strategy since it makes it dependant on the durability of the architecture it is being made for.

  
\end{frame}


%
% bibliography
%

\begin{frame}[allowframebreaks]
  \frametitle{Bibliography}

  \bibliographystyle{amsplain}
  \bibliography{\path/bibliography/bibliography}
\end{frame}

\end{document}

% core
%%
%% licence       kaneton licence
%%
%% project       kaneton
%%
%% file          /home/mycure/kaneton/view/papers/kaneton/architectures.tex
%%
%% created       julien quintard   [sun apr 23 17:08:41 2006]
%% updated       julien quintard   [sun apr 23 17:08:41 2006]
%%

%
% architectures
%

\chapter{Architectures}

This chapter will overview the different architectures supported by
the kaneton microkernel reference.

\newpage

%
% text
%

The kaneton microkernel references currently supports the following
architectures.

Notice that, sometimes, the kaneton microkernel can support multiples
\textit{architectures} of a single microprocessor's architecture.
These multiples \textit{architectures} generally exploit different
facilities of the same microprocessor's architecture.

There is a paper specific to every supported architecture. These papers
are available on the kaneton official website
  \footnote{http://www.kaneton.org}.

%
% intel architecture 32-bit
%

\section{Intel Architecture 32-bit}

The kaneton microkernel reference was first developed on the Intel Architecture
32-bit since this architecture is very popular and cheap.

The following architecture implementations are supported by the kaneton
microkernel.

%
% ia32-virtual
%

\subsection{ia32-virtual}

This architecture uses a flat segmentation model with paging enabled
allowing multiple virtual address space and so virtual address space
protections.

Nevertheless, this architecture implementation is kept as simple as possible
since it is used as the basis of the operating system courses.

One of the particularities of this architecture is that the core has its
own address space. Then, each time a system call occurs, an address space
switch is performed to work in the core's address space.

This particularity leads to bad performances since each time an address
space switch occurs, the microprocessor's caches are flushed.

%
% ia32-optimized
%

\subsection{ia32-optimized}

This architecture also relies on the Intel Architecture 32-bit but uses
every optimisations provided by the microprocessor's architecture.

\notice{This architecture is not implemented yet.}

\input{licenses}
%%
%% licence       kaneton licence
%%
%% project       kaneton
%%
%% file          /home/mycure/kaneton/view/papers/kaneton/bibliography.tex
%%
%% created       julien quintard   [mon may  8 18:35:35 2006]
%% updated       julien quintard   [mon may  8 20:38:56 2006]
%%

%
% bibliograpy
%

\chapter{Bibliography}

This chapter contains the bibliography.

%
% text
%

\begin{thebibliography}{0}
  \bibitem{AST-SCO}
    \textbf{Structured Computer Organization};
    by
    \textit{Andrew S. Tanenbaum}
  \bibitem{AST-CN}
    \textbf{Computer Networks};
    by
    \textit{Andrew S. Tanenbaum}
  \bibitem{AST-OSDI}
    \textbf{Operating Systems: Design and Implementation};
    by
    \textit{Andrew S. Tanenbaum, Albert S Woodhull}
  \bibitem{AST-MOS}
    \textbf{Modern Operating Systems};
    by
    \textit{Andrew S. Tanenbaum}
  \bibitem{AST-DOS}
    \textbf{Distributed Operating Systems};
    by
    \textit{Andrew S. Tanenbaum}
  \bibitem{AST-DSPP}
    \textbf{Distributed Systems: Principles and Paradigms};
    by
    \textit{Andrew S. Tanenbaum, Maarten van Steen}
  \bibitem{NAL-DA}
    \textbf{Distributed Algorithms};
    by
    \textit{Nancy A. Lynch}
\end{thebibliography}


\end{document}

--

overview: kernel = arch + generic :: data-struct + proc

--

bouger les figures dans view/figures
XXX dans book::kaneton, archi supportees, core, glue, machdep system, syscall
    interface, dans book::core, un fichier par manager.
