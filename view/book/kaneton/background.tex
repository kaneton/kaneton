%
% ---------- header -----------------------------------------------------------
%
% project       kaneton
%
% license       kaneton
%
% file          /home/mycure/kaneton/view/book/kaneton/background.tex
%
% created       julien quintard   [tue jun 19 11:48:36 2007]
% updated       julien quintard   [sun dec 16 17:09:39 2007]
%

%
% ---------- background -------------------------------------------------------
%

\chapter{Background}
\label{chapter:background}

This chapter introduces some notions about operating systems and kernels.
However, this chapter assumes the reader already has a substential knowledge
about operating systems internals.

\newpage

%
% ---------- text -------------------------------------------------------------
%

%
% operating system
%

\section{Operating System}

XXX[classic -> NOS -> distributed OS]

%
% kernel
%

\section{Kernel}

XXX[mono, micro, hybrid + nano, exo]

%
% kaneton
%

\section{kaneton}

The project was primarily designed by two students in computer science,
\textit{Julien Quintard} and \textit{Jean-Pascal Billaud}.

These two students previously actively contributed to the development
of a nanokernel-based operating system project in a French research laboratory.
This system was not powerful enough from the design point of view.

Therefore, the two students started the design of a new microkernel
by their own, called \textbf{kaneton}, for educational purposes.

The design was based on five fundamental guidelines.

\begin{enumerate}
  \item
    \textbf{Educational}

    \-

    The kaneton project is built to become an educational project. The design
    as well as the implementation must therefore be as understandable as
    possible that everyone interested in kernel internals can go through the
    documents and source code and actually understand how it works.

    \-

    This \textit{understandable} property can be achieved through a very clear
    and coherent design. Moreover, the implementation should be written using
    modern tools and techniques to make the code as generic as possible and
    easily readable.
  \item
    \textbf{Portability}

    \-

    The microkernel was particularly designed to be portable on many
    architectures. The designers tried to develop a portability system
    powerful enough to port kaneton on any, existing or not, architectures.
  \item
    \textbf{Maintanability}

    \-

    Although, microkernel-based operating systems rely on a modular design,
    kaneton designers also wanted the microkernel itself to be modular and
    maintainable.
  \item
    \textbf{Distributed Computing}

    \-

    The kaneton microkernel must be designed to fit distributed operating
    systems requirements. Indeed, the kaneton microkernel was developed in
    order to design and implement a distributed operating system named
    \textbf{kayou}.

    \-

    This point led to many specific choices in the kaneton microkernel design.
  \item
    \textbf{Demystification}

    \-

    kaneton people wanted to break some well-known kinds of computer
    science rules. Indeed, for instance, many computer scientists consider that
    the source code plays the role of the project documentation. Also, for many
    low-level programmers, the kernel boot source code and more generally the
    kernel source code itself cannot be understandable, clear and coherent as
    it is related to low-level programming: microprocessor, devices etc.

    kaneton people paid particular attention to the microkernel source code to
    be easily understandable, maintainable and extendable. Moreover, kaneton
    people tried to write documentation for every part of the project.
\end{enumerate}

Notice that building an educational microkernel project is nothing innovative.
Indeed few other projects already exist; the most popular being \textit{MINIX}
from \textit{Vrije Universiteit}, \textit{NachOS} from \textit{Berkeley
University} or \textit{PintOS} from \textit{Stanford University}.

kaneton people tried to design and implement a modern microkernel since, the
\textit{MINIX} microkernel for example, do not use modern development tools.
Moreover, the kaneton source code is heavily commented and use modern
languages techniques while trying to stay easily understandable.

The educational characteristic of kaneton does not contraint it to be
optimised afterwards. kaneton people believe that implementing optimised
algorithms does not lead to maintainable implementations.

Finally, note that the kaneton project is actually composed of two projects:
the \textit{kaneton microkernel educational project} which provides everything
necessary to students willing to learn about kernels internals; and the
\textit{kaneton microkernel research project} which focuses on designing and
implementing a powerful, reliable, flexible microkernel. Obviously these
two projects are highly related as the kaneton educational project relies on
the implementation of the kaneton research project.
