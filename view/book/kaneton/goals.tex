%
% ---------- header -----------------------------------------------------------
%
% project       kaneton
%
% license       kaneton
%
% file          /home/mycure/kaneton/view/book/kaneton/goals.tex
%
% created       julien quintard   [fri jun  1 13:58:12 2007]
% updated       julien quintard   [fri jun  1 14:08:23 2007]
%

%
% ---------- goals ------------------------------------------------------------
%

\chapter{Goals}

In this chapter we will briefly introduce the kaneton microkernel
through the kaneton microkernel goals.

\newpage

%
% ---------- text -------------------------------------------------------------
%

The project was primarily designed by two students in computer science,
Julien Quintard and Jean-Pascal Billaud.

These two students previously actively contributed to the development
of a laboratory's microkernel based operating system project. This system
was not powerful enough from the design point of view.

Therefore, the two students started the design of a new microkernel
by their own, called \textbf{kaneton}, for an educational purpose.

The design was based on five fundamental guidelines.

First, the kaneton project was built to become a pedagogical project.
Therefore, kaneton must be understandable be everyone willing learn
operating systems internals.

A pedagogical project needs to be understandable by every student. So
the kaneton design must be as clear as possible and by the way the
kaneton microkernel implementation source code should also use
special techniques to be easily comprehensible and generic.

Second, the microkernel was particularly designed to be portable on many
architectures. The designers tried to develop a portability system powerful
enough to port kaneton on any, existing or not, architectures.

The third point is the maintainability. In fact, microkernel based
operating systems use a modular design, but kaneton designers
also wanted the microkernel itself to be modular and maintainable.

Fourth, the kaneton microkernel had to be designed to fit distributed
operating systems requirements. Indeed, the kaneton microkernel was
developed in order to design and implement a distributed operating
system called \textbf{kayou}.

This point led to many specific choices in the kaneton microkernel design.

Fifth, kaneton people would like to break some well-known kinds of
computer science rules. Indeed, for example, many computer scientists
consider that a project documentation is its source code itself.
Also, for many low-level programmers, the kernel boot source code and
more generally the kernel source code cannot be understandable by everybody
because too messy.

kaneton people paid particular attention at the microkernel source code to be
easily understandable, maintainable and extendable. Moreover, kaneton
people tried to write documentation for every part of the project.

Notice that building a pedagogical microkernel project is nothing innovative.
Indeed few other projects already exist; the most popular being \textit{MINIX}
from \textit{Vrije Universiteit}, \textit{NachOS} from \textit{Berkeley
University} or \textit{PintOS} from \textit{Stanford University}.

kaneton people tried to design and implement a more modern microkernel
since, the \textit{MINIX} microkernel for example, do not use modern
development tools. Moreover, the kaneton source code is heavily commented
and use modern languages techniques.

Finally, the kaneton microkernel was first designed to fit in the
\textit{kayou} requirements. The kaneton project itself is just a step
leading to the \textit{kayou} distributed operating system which will
be used as a educational material for distributed courses.
