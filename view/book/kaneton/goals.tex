%
% ---------- header -----------------------------------------------------------
%
% project       kaneton
%
% license       kaneton
%
% file          /home/mycure/kaneton/view/book/kaneton/goals.tex
%
% created       julien quintard   [fri jun  1 13:58:12 2007]
% updated       julien quintard   [mon may 19 23:09:48 2008]
%

%
% ---------- goals ------------------------------------------------------------
%

\chapter{Goals}
\label{chapter:goals}

In this chapter we will briefly introduce the kaneton microkernel
through the kaneton microkernel goals.

\newpage

%
% ---------- text -------------------------------------------------------------
%

The project was primarily designed by two students in computer science,
\name{Julien Quintard} and \name{Jean-Pascal Billaud}.

These two students previously actively contributed to the development
of a nanokernel-based operating system project in a French research laboratory.
This system was not powerful enough, especially from the design point of view.

Therefore, the two students started the design of a new microkernel
by their own, called \term{kaneton}, for educational purposes.

The design was based on five fundamental guidelines.

\begin{enumerate}
  \item
    \textbf{Educational}

    \-

    The kaneton project is built to become an educational project. The design
    as well as the implementation must therefore be as understandable as
    possible so that everyone interested in kernel internals can go through the
    documents and source code and actually understand how it works.

    \-

    This \textit{understandable} property can be achieved through a very clear
    and coherent design. Moreover, the implementation should be written using
    modern tools and techniques to make the code as generic as possible and
    easily readable.
  \item
    \textbf{Portability}

    \-

    The microkernel was particularly designed to be portable. The designers
    tried to develop a portability system powerful enough to port kaneton on
    any, existing or not, architectures.
  \item
    \textbf{Maintanability}

    \-

    Although microkernel-based operating systems rely on a modular design,
    kaneton designers also wanted the microkernel itself to be modular and
    maintainable.
  \item
    \textbf{Distributed Computing}

    \-

    The kaneton microkernel must be designed to fit distributed operating
    systems requirements. Indeed, the kaneton microkernel was developed in
    order to design and implement a distributed operating system named
    \term{kayou}.

    \-

    This point led to many specific choices in the kaneton microkernel design.
  \item
    \textbf{Demystification}

    \-

    kaneton people wanted to break some well-known kind of computer
    science rules. Indeed, for instance, many computer scientists consider
    the source code as the actual documentation. Also, for many low-level
    programmers, the kernel boot source code and more generally the
    kernel source code itself cannot be understandable, clear and coherent as
    it is related to low-level programming: microprocessor, devices \etc{}

    kaneton people paid particular attention to the microkernel source code to
    be easily understandable, maintainable and extendable. Moreover, kaneton
    people tried to write documentation for every part of the project.
\end{enumerate}

Notice that building an educational microkernel project is nothing innovative.
Indeed few other projects already exist; the most popular being \name{MINIX}
from \name{Vrije Universiteit}, \name{NachOS} from \name{Berkeley University}
or \name{PintOS} from \name{Stanford University}.

kaneton people tried to design and implement a modern microkernel since, the
original \name{MINIX} microkernel for example, do not use modern development
tools. Moreover, the kaneton source code is heavily commented and use modern
languages techniques while trying to stay easily understandable.

The educational characteristic of kaneton does not constraint it from being
optimised afterwards. kaneton people believe that implementing optimised
algorithms in the first place does not lead to maintainable implementations.

Finally, note that the kaneton project is actually composed of two projects:
the \name{kaneton microkernel \term{educational} project} which provides
everything necessary to students willing to learn about kernels internals;
and the \name{kaneton microkernel \term{research} project} which focuses on
designing and implementing a powerful, reliable, flexible microkernel.
Obviously these two projects are highly related as the kaneton educational
project relies on the implementation of the kaneton research project.
