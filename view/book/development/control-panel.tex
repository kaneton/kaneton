%
% ---------- header -----------------------------------------------------------
%
% project       kaneton
%
% license       kaneton
%
% file          /home/mycure/kaneton/view/book/development/control-panel.tex
%
% created       julien quintard   [sun may 20 15:22:52 2007]
% updated       julien quintard   [sun jun  3 19:03:38 2007]
%

%
% ---------- control panel ----------------------------------------------------
%

\subsection{Control Panel}
\label{section:control panel}

The kaneton environment allows the developer to trigger every action from
the \textit{Make} file of the project's root directory.

These actions are listed below:

\command{initialize}
        {
	  This action initializes the kaneton development environment by
	  invoking the \textit{init.py} script of the \textit{environment/}
	  directory.

	  \-

	  \example{\$ make initialize}
	}

\command{clean}
	{
	  This action cleans the kaneton development environment.

	  \-

	  \example{\$ make clean}
	}

\command{kaneton}
	{
	  This action builds the kaneton microkernel.

	  \-

	  \example{\$ make kaneton}
	}

\command{clear}
	{
	  This action removes every compiled files.

	  \-

	  \example{\$ make clear}
	}

\command{purge}
	{
	  This action cleans directories from unecessary files.

	  \-

	  \example{\$ make purge}
	}

\command{headers}
	{
	  This action generates \textit{Make} files' \textit{C} header
	  files dependencies.

	  \-

	  \example{\$ make headers}
	}

\command{prototypes}
	{
	  This action generates C prototypes.

	  \-

	  \example{\$ make prototypes}
	}

\command{prototypes}
	{
	  This action generates C prototypes.

	  \-

	  \example{\$ make prototypes}
	}

\command{test}
	{
	  This action runs the test suite in order to validate the kaneton
	  microkernel behaviour.

	  \-

	  \example{\$ make test}
	}

\command{cheat}
	{
	  This action launches the cheat tests on students kaneton
	  implementations.

	  \-

	  \example{\$ make cheat}

	  \-

	  \example{\$ make cheat-EPITA::2006.k3}
	}

\command{build}
	{
	  This action builds the boot device.

	  \-

	  \example{\$ make build}
	}

\command{install}
	{
	  This action installs the kaneton microkernel with its dependencies:
	  configuration files, bootloader etc. on the boot device.

	  \-

	  \example{\$ make install}
	}

\command{export}
	{
	  This action builds a kaneton distribution package.

	  \-

	  \example{\$ make export}

	  \-

	  \example{\$ make export-k3,5}

	  \-

	  \example{\$ make export-back}
	}

\command{view}
	{
	  This action builds and displays a kaneton document.

	  \-

	  \example{\$ make view}

	  \-

	  \example{\$ make view-develop}

	  \-

	  \example{\$ make view-book::kaneton}
	}

\command{record}
	{
	  This action records a real-time session.

	  \-

	  \example{\$ make record}

	  \-

	  \example{\$ make record-basic::test.ts}
	}

\command{play}
	{
	  This action plays a recorded session.

	  \-

	  \example{\$ make play}

	  \-

	  \example{\$ make play-session::prototypes.ts}

	  \-

	  \example{\$ make play-prototy}
	}

\command{info}
	{
	  This action displays general information about kaneton.

	  \-

	  \example{\$ make info}
	}
