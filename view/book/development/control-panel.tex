%
% ---------- header -----------------------------------------------------------
%
% project       kaneton
%
% license       kaneton
%
% file          /home/mycure/kaneton/view/book/development/control-panel.tex
%
% created       julien quintard   [sun may 20 15:22:52 2007]
% updated       julien quintard   [mon sep 15 18:07:03 2008]
%

%
% ---------- control panel ----------------------------------------------------
%

\subsection{Control Panel}
\label{section:control panel}

The kaneton environment allows the developer to trigger every action from
the \name{Make} file of the project's root directory.

These actions are listed below:

\command{make initialize}
        {
	  This action initializes the kaneton development environment by
	  invoking the \location{init.py} script of the \location{environment/}
	  directory.
	}

\command{make clean}
	{
	  This action cleans the kaneton development environment.
	}

\command{make kaneton}
	{
	  This action builds the kaneton microkernel.
	}

\command{make main}
	{
	  This action triggers the default rule which aims at compiling every
	  piece the final system needs to be set up on a bootable device.

	  \-

	  \example{\$ make main}

	  \example{\$ make}
	}

\command{make clear}
	{
	  This action removes every compiled files.
	}

\command{make headers}
	{
	  This action generates \name{Make} files' \name{C} header
	  files dependencies.
	}

\command{make prototypes}
	{
	  This action generates C prototypes.
	}

\command{make test}
	{
	  This action runs the test suite in order to validate the kaneton
	  microkernel behaviour.
	}

\command{make cheat}
	{
	  This action launches the cheat tests on students kaneton
	  implementations.

	  \-

	  \example{\$ make cheat}

	  \-

	  \example{\$ make cheat-EPITA::2006::k3}
	}

\command{make build}
	{
	  This action builds the boot device.
	}

\command{make install}
	{
	  This action installs the kaneton microkernel with its dependencies:
	  configuration files, bootloader \etc{} on the boot device.
	}

\command{make export}
	{
	  This action builds a kaneton distribution package.

	  \-

	  \example{\$ make export}

	  \-

	  \example{\$ make export-k3,5}

	  \-

	  \example{\$ make export-back}
	}

\command{make view}
	{
	  This action builds and displays a kaneton document.

	  \-

	  \example{\$ make view}

	  \-

	  \example{\$ make view-devel}

	  \-

	  \example{\$ make view-book::kaneton}
	}

\command{make record}
	{
	  This action records a real-time session.

	  \-

	  \example{\$ make record}

	  \-

	  \example{\$ make record-basic::test.ts}
	}

\command{make play}
	{
	  This action plays a recorded session.

	  \-

	  \example{\$ make play}

	  \-

	  \example{\$ make play-basic::prototypes.ts}

	  \-

	  \example{\$ make play-prototy}
	}

\command{make info}
	{
	  This action displays general information about kaneton.
	}
