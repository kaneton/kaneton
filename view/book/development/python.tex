%
% ---------- header -----------------------------------------------------------
%
% project       kaneton
%
% license       kaneton
%
% file          /home/mycure/kaneton/view/book/development/python.tex
%
% created       julien quintard   [sat jun  2 13:06:42 2007]
% updated       julien quintard   [thu aug 16 23:10:17 2007]
%

%
% ---------- python -----------------------------------------------------------
%

\section{Python}
\label{section:python}

The \textit{Python} language is the other language for which the kaneton
development environment provides an interface for performing task in
a portable way.

Historically, kaneton scripts were written in \textit{Shell} but scripts
are pretty hard to write, especially when dealing with file modification.
Even if \textit{sed}, \textit{awk} etc. tools are very powerful, this
is not a good solution for many reasons. Therefore, the scripts were
re-written in \textit{Python} whilst a \textit{Python} development environment
interface was written.

%
% naming
%

\subsection{Naming}

File names but also function names must be expressed in lower-case letters
and a dash must be used for composite names.

Comments must also be written in lower-case letters.

Global variable names, which are often used in \textit{Python}, must be
prefixed by \textit{g\_}.

As usual, names must be expressed in a single word as long as this is possible.
Moreover, the shortcut prefixes and suffixes recommended for the \textit{C}
language also stand for \textit{Python}. For more information, take a look
at the \textit{Section \ref{section:c}}.

% environment

\subsection{Environment}

Every \textit{Python} file must rely on the development environment as it
provides a way for performing operations in a very simple and portable way.

Therefore, each \textit{Python} must start by including the configured
\textit{Python} environment module located in the \textit{environment/}
directory under the name: \textit{env.py}.

Also, since operations are provided by the development environment, note
that \textit{Python} scripts must not perform operations relying on the
\textit{sys} and/or \textit{os} modules. Others modules like \textit{re},
\textit{yaml} etc. are allowed. Moreover, the \textit{sys} module is
sometimes used - badly. If you do not know whether the usage of a module
is allowed, please ask your supervisor or directly the kaneton community
through the mailing-list for instance.

%
% layout
%

\subsection{Layout}

The \textit{Python} files must be divided into sections. Below are listed
the most common sections developers should use.

\begin{itemize}
  \item
    \textbf{header}: this section contains the well-known informations about
    the file: dates, author, license etc.
  \item
    \textbf{information}: this section describes what the \textit{Pyhton}
    do or provide if it is a package.
  \item
    \textbf{imports}: this section contains the module imports.

    \begin{verbatim}
      #
      # ---------- imports ----------------------------------------------------
      #

      import env

      import sys
      import re
    \end{verbatim}
  \item
    \textbf{globals}: this section contains the global variable declarations.

    \begin{verbatim}
      #
      # ---------- globals ----------------------------------------------------
      #

      g_directories = (`book',
                       `curriculum',
                       `exam',
                       `feedback',
                       `internship',
                       `lecture',
                       `paper')
      g_store = []
      g_document = None
      g_path = None
    \end{verbatim}
  \item
    \textbf{functions}: this section contains the actual \textit{Python}
    functions.
  \item
    \textbf{entry point}: this section contains the real entry point.

    This section was introduced in order to localise the script's entry
    point more easily. Note that this section is generally located at
    the end of the script.

    \begin{verbatim}
      #
      # ---------- entry point ------------------------------------------------
      #

      if __name__ == `__main__':
        main()
    \end{verbatim}
\end{itemize}

%
% style
%

\subsection{Style}

Every function must be preceded by a comment. This comment plays the role
of visual separator but it also describes what the function does.

This comment first contains the name of the function with two parentheses.
Then follows an empty comment line and the function description.

\begin{verbatim}
  #
  # usage()
  #
  # this function prints the documents names.
  #
  def                     usage():
    location = None
    store = None

    env.display(env.HEADER_ERROR, `usage: view.py [document]',
                env.OPTION_NONE)

    env.display(env.HEADER_NONE, `', env.OPTION_NONE)

    env.display(env.HEADER_ERROR, `documents:', env.OPTION_NONE)

    for store in g_store:
      for location in store:
        env.display(env.HEADER_ERROR, `  ' + location, env.OPTION_NONE)
\end{verbatim}

The identation used in \textit{Python} script must be fixed at \textit{2}
whitespaces. This is a common kaneton rule which allows files to be
readable while respecting the limitation about the \textit{80} characters
in width.

Finally, note, that in \textit{Python}, variables do not need to be declared
before being actually used. However, kaneton people believe this lead
to unreadable code as well as to common hard-to-find bugs. For this reason,
every variable must be declared at the beginning of the function. Finally,
these variable should also be initialized.
