%
% ---------- header -----------------------------------------------------------
%
% project       kaneton
%
% license       kaneton
%
% file          /home/mycure/kaneton/view/book/development/cheat.tex
%
% created       julien quintard   [thu may 24 11:57:37 2007]
% updated       julien quintard   [mon sep 15 13:17:04 2008]
%

%
% ---------- cheat ------------------------------------------------------------
%

\subsection{Cheat}
\label{section:cheat}

The \name{cheat} tool checks whether students cheated by using pieces of
code from other students' implementation of the current and/or previous years.

The \location{history/} directory is composed of directories organizing the
kaneton students implementations over the years and for every school and
university the education project has been used for. Then, each subdirectory
represents a year and contains subdirectories for each students group of this
year.

Each student group directory contains a \location{sources/} subdirectory
containing the snapshots of the different kaneton stages: \name{k0},
\name{k1}, \name{k2} and so on.

The \name{cheat} tool takes a school, a year and a stage as arguments. Its
first task is to generate the fingerprints of the other kaneton
implementations.

Once the fingerprints have been generated, they are gathered into a database
file. The tool then performs the verification process by comparing snapshots
against each others.

Note that in order to prevent the tool from detecting matches in the source
code that has been provided by the teachers, the tool first removes the parts
common to both the students snapshots and the base snapshot. Additionally,
to reduce the amount of work to be done, everything contained in the
\code{\_CHEAT\_FILTER\_} environment variable is removed from the students'
snapshots. This variable is likely to contain directories such as
\location{environment/}, \location{license/}, \location{tool/} \etc{}

Finally, the tool generates a \name{HTML} page summarising the matches
found between the students. The matches are classified according to the number
of tokens found.

Note that teachers are asked to add student snapshots to the repository
in a careful way, taking care that snapshots do not contain any object file,
revision control directories such as \location{.svn/} \etc{} and that once
extracted, the snapshot produced a single \location{kaneton/} directory.
Moreover, if the student snapshot does not follow the base organisation,
false-positive matches will emerge.

The \name{cheat} tool is based on another tool whose name cannot be revealed
here. For more information, please contact your supervisor.
