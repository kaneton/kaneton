%
% ---------- header -----------------------------------------------------------
%
% project       kaneton
%
% license       kaneton
%
% file          /home/mycure/kaneton/view/book/development/cheat.tex
%
% created       julien quintard   [thu may 24 11:57:37 2007]
% updated       julien quintard   [fri jun  1 01:03:21 2007]
%

%
% ---------- cheat ------------------------------------------------------------
%

\subsection{Cheat}
\label{section:cheat}

The \textit{cheat} tool checks whether students cheated by using pieces of
code from kaneton projects of the previous years.

The \textit{history/} directory is composed of directories organizing the
kaneton students implementations over the years and for every school and
university the education project was used. Then each subdirectory represents
a year and contains subdirectories for each students group of this year.

Each student group directory contains a \textit{sources/} subdirectory
containing the tarballs of the different kaneton stages: \textit{k0},
\textit{k1}, \textit{k2} and so on; a \textit{fingerprints/} directory
containing an internal source representation used for detecting cheating,
a \textit{tests/} directory containing a summary of the testing results
for each stage and a \textit{cheats/} directory which contains a list of
commonalities with other kaneton implementations of the same and previous
years.

The \textit{cheat} tool takes a year and a stage as arguments. Its first
task is to generate the fingerprints of the other kaneton implementations
for this stage of the same and previous years.

Once the fingerprints are generated, the tool performs the checks by
comparing each pair of kaneton implementations for this stage.

The \textit{cheat} tool is based on another tool which cannot be revealed
here. For more information, please contact your supervisor.
