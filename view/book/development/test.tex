%
% ---------- header -----------------------------------------------------------
%
% project       kaneton
%
% license       kaneton
%
% file          /home/mycure/kaneton/view/book/development/test.tex
%
% created       julien quintard   [thu may 24 12:18:23 2007]
% updated       julien quintard   [wed dec  8 22:37:27 2010]
%

%
% ---------- test -------------------------------------------------------------
%

\subsection{Test}
\label{section:test}

The \name{test} tool enables students to test their kaneton implementation
against a set of tests that have been designed by the contributors. Below
is briefly described the terminology used by this tool in order to give
the reader an overview of the general scenario involving students, the
administrator and the server running the test system.

\begin{itemize}
  \item
    A \name{certificate} is used to make sure clients can authenticate the
    test server;
  \item
    Every certificate is sealed by a cryptographic \name{key};
  \item
    Each user is provided with a \name{capability} in order to identify
    herself to the server;
  \item
    These capabilities are sealed by a \name{code} which the server uses
    in order to detect illegally forged capabilities;
  \item
    A \name{configuration} specifies the number of tests a user is allowed
    to requests the server;
  \item
    The user's \name{database} is generated based on a configuration and
    maintains the current user's state on the server including the number
    of tests performed so far, the kaneton implementations submitted for
    evaluation \etc{}
  \item
    A \name{snapshot} is a kaneton implementation in its shipping form;
  \item
    The \name{machine} represents the target
    \name{platform}/\name{architecture} couple on which a snapshot is
    supposed to be tested or evaluated for instance;
  \item
    An \name{image} represents a kaneton snapshot compiled in a bootable
    form;
  \item
    A \name{test} is a function included in the kernel which performs a
    specific set of operations and possibly prints information to the console;
  \item
    The tests are often gathered together in a \name{suite} which represents
    the testing unit students are offered to trigger for their kaneton
    implementation;
  \item
    Once a snapshot is received by the server in order to be tested,
    the system compiles it into an image. The server also takes care to
    include the tests in the compilation process so that they can be triggered.
    These pre-compiled tests are referred to as the \name{bundle};
  \item
    The image can then be tested by triggering the tests of the suite. The
    image is therefore booted in an emulated \name{environment}. This
    environment can sometimes be chosen and offers a trade-off between
    simplicity and realism. The most common environments are \name{QEMU}
    and \name{Xen};
  \item
    Depending on the success of the tests, a set of results is generated
    and compiled in a \name{bulletin} file;
  \item
    Finally, the server retrieves this bulletin, adds some meta information
    such as the date of the test, the environment and machine used \etc{}
    and stores everything in a \name{report}. Note that this report is
    also sent back to the user so she can consult it;
  \item
    Students can also decide to submit their kaneton implementation for
    a specific \name{stage} for future evaluation. Note that suites and
    stages are completely different though they often bear the same names:
    \name{k0}, \name{k1}, \name{k2} etc{};
  \item
    The administrator can decide to evaluate the snapshots which have been
    submitted for a stage by invoking a script which will attribute grades
    according to the \textit{point}s associated with every test.
  \item
    Finally, a \name{statement} is produced containing the grades of every
    student for a given stage.
\end{itemize}

The following describes the \name{test} tool according to the user's role
regarding the system: either the administrator who sets up the system or
a student who uses it in order to improve and/or evaluate his implementation.

%
% administrator
%
\subsubsection{Administrator}

The administrator is responsible for setting up the system but also maintaining
it on a daily basis.

% requirements
\subsubsubsection{Requirements}

The \name{test} tool must be installed on a publicly accessible server since
the server script will be waiting for incoming requests. Note that by default,
the clients assume the test server to be accessible at the address:
\location{https://test.opaak.org:8421}.

Besides, since the purpose of the \name{test} tool is to run the students'
kaneton implementation in emulated environments, both \name{QEMU} and
\name{Xen} should be available though one might want to configure the tool
for supporting a single environment, \name{QEMU} for instance.

Note that the test system has been developed with \name{Python 2.6} and
may be out of date by the time the administrator sets it up. In addition,
the system depends on a variety of \name{Python} packages including
\name{argparse}, \name{yaml}, \name{pyopenssl}, \name{hmac}, \name{pickle},
\name{xmlrpc}, \name{subprocess}, \name{serial} among others.

Finally, the administrator should make sure the following applications
are installed since some test scripts need them: \name{dd}, \name{mkfs.ext2},
\name{mount}, \name{umount}, \name{mutt}, \name{sendmail}, \name{qemu}
and \name{mkisofs}.

% set up
\subsubsubsection{Set Up}

The first step for an administrator consists in generating the necessary
files, especially the certificates, code and capabilities required for
securing the test system.

The \location{test/utilities/} directory contains the scripts that perform
such operations. Note that all the generated files are stored in the
\location{test/store/} directory.

First the \name{CA - Certification Authority}'s and server's certificates
must be generated. The first is used to issue certificates while the latter
is used for clients to identify the server with absolute certainty.

\begin{verbatim}
  $> make certificate
  [+] generating the CA and server's key/certificate pair
  [+] CA key/certificate generated
  [+] server key/certificate generated
  [+] CA and server's key/certificate pair generated and stored
  $> 
\end{verbatim}

The next step consists in generating a code for the administrator to
issue capabilities but also for the server to verify that the received
capabilities have not been illegally forged.

\begin{verbatim}
  $> make code
  [+] generating the server's code
  [+] server code successfuly generated and stored
  $> 
\end{verbatim}

With a server code, the students' and contributor's capabilities can be
built, hence granting them the right the contact the server.

The following generates the contributor's capability. This capability is
special in the way that contributors can perform any operation in a completely
contrain-free manner.

\begin{verbatim}
  $> make capability-contributor
  [+] generating the contributor's capability
  [+] contributor's capability generated and stored
  $> 
\end{verbatim}

In contrast, the following command generates a set of capabilities for the
students belonging to the school referred to as \name{``epita::2010''}. Note
that the script requires the \location{history/epita/2010/} to be populated
with the groups and their \location{people} file.

\begin{verbatim}
  $> make capability-school@epita::2010
  [+] generating students' capabilities
  [+] extracting the students from the history 'epita/2010'
  [+] students information retrieved
  [+] generating the students' capabilities:
  [+]   epita::2010::group11
  [+]   epita::2010::group10
  [+]   epita::2010::group13
  [+]   epita::2010::group12
  [+]   epita::2010::group33
  [+]   epita::2010::group32
  [+]   epita::2010::group17
  [+]   epita::2010::group30
  [+]   epita::2010::group19
  [+]   epita::2010::group18
  [+]   epita::2010::group5
  [+]   epita::2010::group4
  [+]   epita::2010::group7
  [+]   epita::2010::group6
  [+]   epita::2010::group1
  [+]   epita::2010::group3
  [+]   epita::2010::group2
  [+]   epita::2010::group15
  [+]   epita::2010::group9
  [+]   epita::2010::group8
  [+]   epita::2010::group14
  [+]   epita::2010::group31
  [+]   epita::2010::group16
  [+]   epita::2010::group24
  [+]   epita::2010::group25
  [+]   epita::2010::group26
  [+]   epita::2010::group27
  [+]   epita::2010::group20
  [+]   epita::2010::group21
  [+]   epita::2010::group22
  [+]   epita::2010::group23
  [+]   epita::2010::group28
  [+]   epita::2010::group29
  [+]   epita::2010::group34
  [+] students' capabilities generated and stored
  $> 
\end{verbatim}

In addition, the administrator could decide to generate or re-generate
a capability for a specific student of a school. The following shows an
example for such an action.

\begin{verbatim}
  $> make capability-student@epita::2010::group8
  [+] generating the student's capability:
  [+]   epita::2010::group8
  [+] student's capability generated and stored
  $> 
\end{verbatim}

The next step consists in the databases generation. A database contains the
state of a user profile including the number of test requests, the quota for
such tests, the submitted snapshots and so forth. The database files are
absolutely fundamental to the server since such databases are updated after
each client's request.

The syntax for generating databases follows the one for capabilities, as
shown next for the contributor.

\begin{verbatim}
  $> make database-contributor
  [+] generating database from contributor's configuration
  [+] contributor's database generated and stored
  $> 
\end{verbatim}

Once the certificates, code, capabilities and databases generated, the
administrator can move on to the deployment process.

% deployment
\subsubsubsection{Deployment}

The deployment basically consists in copying the \location{test/} environment
to the test server though one might want to copy the whole kaneton environment
or the smallest subset of the \location{test/} directory which should, in this
case, include the following absolutely necessary items:

\begin{itemize}
  \item
    The \location{test/environments/} directory which contains the descriptions
    of the supported test environments;
  \item
    The \location{test/images/} directory which contains a script for
    automatically generating a \name{Debian Live} system which is used
    for compiling a kaneton snapshot into a bootable image;
  \item
    The \location{test/packages/} directory which contains the \name{ktp -
    Kaneton Test Package} required by the server-side standalone scripts
    for manipulating files such as databases, capabilities \etc{} but also
    for performing cryptographic operations and send/receive \name{XMLRPC}
    requests;
  \item
    The \location{test/scripts/} directory which contains the fundamental
    scripts for building bootable images, distributing the capabilities to
    the students through emails, evaluating the submitted snapshots and so on;
  \item
    The \location{test/server/} directory which contains the server script
    for handling the clients' requests;
  \item
    The \location{test/stages/} directory which contains the files requirement
    for evaluating the students' snapshots;
  \item
    The \location{test/store/} directory which contains the generated files
    such as the users' databases, the server's code and certificate; and
  \item
    The \location{test/suites/} directory which contains the files describing
    the tests to be including in a given tests suite.
\end{itemize}

Once copied, the administrator only needs to launch the server script located
in the \location{test/server/} directory, as shown below:

\begin{verbatim}
  $> ./server.py
  [meta] serving on 88.191.84.128:8421
\end{verbatim}

Note that a few additional steps may be required depending on the current state
of the kaneton development.

The first of these steps may consist in generating a \name{Debian Live} system
since this is absolutely required for the test system to work. For more
information regarding the generation of such an image, please refer to the
\location{test/images/} directory.

The second step should consist for the administrator in building the kaneton
tests bundle. The bundle represents a pre-compiled set of tests that is
included in the students' snapshot compilation process. The tests are
pre-compiled in order to prevent leaking information since students could
very well dump the content of those tests and force the compilation to fail,
hence retrieving the source code in the compilation process' error log.

In order to generate such a bundle, the administrator must first activate
the \name{test} module, as show next:

\begin{verbatim}
  _MODULES_               +=              test
\end{verbatim}

Then, the administrator must move to the \location{test/tests/} directory
and launch a compilation process through the following command:

\begin{verbatim}
  $> make
\end{verbatim}

Once generated, the test bundle, located in \location{store/bundle/[machine]/}
must be copied to the server, at the same location.

Finally, for more information on the server script, please refer to the
\location{test/server/} directory.

% scripts
\subsubsubsection{Scripts}

Although the deployment process is pretty straightforward, the administrator
is required to manage the test system through several scripts.

First, the \name{distribute} script must be used by the administrator to send
the capabilities to the respective owners so that the students can use
the test system. Note that this script relies on the \name{Mutt} mailing
system for sending the emails containing the attached capabilities.

\begin{verbatim}
  $> ./distribute.py
  recipients:
    contributor
  $>
\end{verbatim}

While the \name{construct} script enables the administrator to build a
bootable image from a kaneton snapshot, the \name{stress} script takes
a bootable image and triggers the tests belonging to the given test
suite. Note that both scripts are directly used by the server script for
building and testing the received kaneton snapshots.

\begin{verbatim}
  $> ./construct.py --snapshot kaneton.tar.bz2                          \
                    --image kaneton.img                                 \
                    --environment xen                                   \
  the kaneton image has been constructed in 'kaneton.img'
  $> ./stress.py --image kaneton.img                                    \
                 --suite k2                                             \
                 --environment xen                                      \
                 --verbose
  segment
    permissions/01 :: true
  id
    simple :: true
    clone :: true
    multiple :: true
  $> 
\end{verbatim}

Note that the administrator could also test a kaneton image manually,
especially through the following command:

\begin{verbatim}
  $> qemu -fda kaneton.img -curses
\end{verbatim}

Besides, note that an administrator willing to include a new test in the
system would probably want to test it locally first since testing through
the server takes some time. In order to test locally, the administrator
must first activate the bundle module in its user profile
\location{environment/profile/user/\${KANETON\_USER}/\${KANETON\_USER}.conf}:

\begin{verbatim}
  _MODULES_               +=              bundle
\end{verbatim}

Then, the administrator must trigger the test by calling the test function
manually in its kaneton implementation. For instance, in order to trigger
the \name{kaneton/core/task/guest} test, the administrator could add the
following line after \code{kernel\_initialize()} and before running the
test system in \location{kaneton/core/core.c}:

\begin{verbatim}
  [...]

  module_call(console, message,
              '+', "starting the kernel\n");

  assert(kernel_initialize() == STATUS_OK);

  /* XXX[temporary] */
  test_core_task_guest();

  module_call(test, run);

  [...]
\end{verbatim}

Once the kaneton image rebuilt, the administrator can boot it locally
through \name{QEMU} and get the output, hence check that the test went
as excepted:

\begin{verbatim}
  $> qemu -fda environment/profile/user/${KANETON_USER}/${KANETON_USER}.img
\end{verbatim}

Back to the server side, the \name{evaluate} script can be used by the
administrator in order to assign grades to the snapshots submitted by the
students. The script generates a statement containing the results of this
evaluation process.

\begin{verbatim}
  $> ./evaluate.py --stage k2                                           \
                   --pattern "^epita::2010::.*$"
  the statement has been saved in '../store/statement/20101102-223645.db'
  $> 
\end{verbatim}

Finally, the \name{dump} script takes any \name{YAML}-based file and
displays its inner structure in a hierarchical manner.

\begin{verbatim}
  $> ./dump.py --path ../store/statement/20101102-223848.db
  meta:
    reference:              20.0
    stage:                  k2
  data:
    epita::2010::group7:
      date:                 2010/11/02 20:46:44
      grade:                16.0
      snapshot:             20101102-204644
      members:
        email:              admin@opaak.org
        name:               admin
      configurations:
        Xen:
          report:           20101102-224213
          notch:            4
          score:            4
        QEMU:
          report:           20101102-223848
          notch:            4
          score:            0
\end{verbatim}

%
% student
%
\subsubsection{Student}

The student has the possibility to request actions from the test server
through the client script located in \location{test/client/}.

% requirements
\subsubsubsection{Requirements}

Although the client script is integrated in the kaneton environment, it also
makes use of the \name{ktp}. Therefore, as for the server, the client depends
on a variety of \name{Python} packages including \name{yaml}, \name{pyopenssl},
\name{hmac}, \name{pickle}, \name{xmlrpc}, \name{subprocess} among others.

% use
\subsubsubsection{Use}

The client script enables the user to request one of the five operations
described below.

\begin{verbatim}
  $> make
  [!] usage: client.py [command]

  [!] commands:
  [!]   information
  [!]   submit-[stage]
  [!]   test-[environment]::[suite]
  [!]   list
  [!]   display-[identifier]
  [!]   retest-[identifier]
  $>
\end{verbatim}

The \name{information} operation requests the server to return information
on the current state of the user's profile. The information returned range
from the number of tests performed, the quota for every test suite to the
available stages or the snapshots having been previously submitted.

\begin{verbatim}
  $> make information
  [+] configuration:
  [+]   server:                 https://test.opaak.org:8421
  [+]   capability:             /data/mycure/repositories/kaneton/environment/profile/user/julien.quintard/julien.quintard.cap
  [+]   platform:               ibm-pc
  [+]   architecture:           ia32/educational

  [+] information:
  [+]   profile:
  [+]     identifier:           contributor
  [+]     community:            contributors
  [+]     members:
  [+]       name:               admin
  [+]       email:              admin@opaak.org
  [+]   suites:
  [+]                           k1
  [+]                           k3
  [+]                           k2
  [+]                           kaneton
  [+]   stages:
  [+]                           k1
  [+]                           k2
  [+]                           k3
  [+]   environments:
  [+]                           qemu
  [+]                           xen
  [+]   database:
  [+]     reports:
  [+]       xen:
  [+]         ibm-pc.ia32/educational:
  [+]           k3:
  [+]           k2:
  [+]           k1:
  [+]       qemu:
  [+]         ibm-pc.ia32/educational:
  [+]           k3:
  [+]           k2:
  [+]           k1:
  [+]     settings:
  [+]       xen:
  [+]         ibm-pc.ia32/educational:
  [+]           k3:
  [+]             requests:     0
  [+]             quota:        -1
  [+]           k2:
  [+]             requests:     0
  [+]             quota:        -1
  [+]           k1:
  [+]             requests:     0
  [+]             quota:        -1
  [+]       qemu:
  [+]         ibm-pc.ia32/educational:
  [+]           k3:
  [+]             requests:     0
  [+]             quota:        -1
  [+]           k2:
  [+]             requests:     0
  [+]             quota:        -1
  [+]           k1:
  [+]             requests:     0
  [+]             quota:        -1
  $> 
\end{verbatim}

The \name{test} command enables the user to trigger a test suite for the
current kaneton implementation on the specified environment such as \name{QEMU}
or \textit{Xen} for instance.

The server then returns the resulted report which the client stores locally
in \location{test/store/report/}. In addition, the client displays a quick
summary of the report in order for the user to know whether things went
as expected.

\begin{verbatim}
  $> make test-xen::k2
  [+] configuration:
  [+]   server:                 https://test.opaak.org:8421
  [+]   capability:             /data/mycure/repositories/kaneton/environment/profile/user/julien.quintard/julien.quintard.cap
  [+]   platform:               ibm-pc
  [+]   architecture:           ia32/educational

  [+] report(20101103:140601):
  [+]   segment                                                           [1/1]
  [+]   id                                                                [3/3]
  $> 
\end{verbatim}

The \name{list} command enables the user to display the identifiers of the
reports in the local store.

\begin{verbatim}
  $> make list
  [+] reports:
  [+]   20101103:140601:
  [+]     xen :: ibm-pc :: ia32/educational :: k2 :: 2010/11/03 14:06:01
\end{verbatim}

The \name{display} command gives the user the possibility to dump a locally
stored report in a very detailed way.

\begin{verbatim}
  $> make display-20101103:140601
  [+] report:
  [+]   meta:
  [+]     platform:               ibm-pc
  [+]     date:                   2010/11/03 14:06:01
  [+]     architecture:           ia32/educational
  [+]     duration:               63.499
  [+]     suite:                  k2
  [+]     identifier:             20101103:140601
  [+]     environments:
  [+]       stress:               xen
  [+]       construct:            xen
  [+]   data:
  [+]     segment:                                                        [1/1]
  [+]       permissions/01:
  [+]         status: True
  [+]         description: This test creates a task and address space before reserving a segment and changing its permissions.
  [+]         duration: 0.010
  [+]         output: 
  [+]     id:                                                             [3/3]
  [+]       simple:
  [+]         status: True
  [+]         description: This test reserves a single identifier.
  [+]         duration: 0.004
  [+]         output: 
  [+]       clone:
  [+]         status: True
  [+]         description: This test reserves, clones and releases identifiers.
  [+]         duration: 0.005
  [+]         output: 
  [+]       multiple:
  [+]         status: True
  [+]         description: This test reserves thousands of identifiers, checking that no collisions occured.
  [+]         duration: 0.040
  [+]         output: 
  $> 
\end{verbatim}

The \name{submit} command sends the user's snapshot to the server so as to
be evaluated for the given stage.

\begin{verbatim}
  $> make submit-k1
  [+] configuration:
  [+]   server:                 https://test.opaak.org:8421
  [+]   capability:             /data/mycure/repositories/kaneton/environment/profile/user/julien.quintard/julien.quintard.cap
  [+]   platform:               ibm-pc
  [+]   architecture:           ia32/educational

  [+] the snapshot has been submitted successfully
  $> 
\end{verbatim}

Finally, the \name{retest} command provides contributors the possibility to
re-launch the test suite of the given identified test. This command is
especially useful to re-test a snapshot should an unexpected error occur on
the test server.

Indeed since test requests are limited for students, it would be unfair for the
student to be forced to sacrifice a test slot because something went wrong
on the server-side. By requesting a contributor, the student's snapshot can
be re-tested. Once the test complete, an email is sent to the student along
with the attached report.

\begin{verbatim}
  $> make retest-20101103:140601
  [+] configuration:
  [+]   server:                 https://test.opaak.org:8421
  [+]   capability:             /data/mycure/repositories/kaneton/environment/profile/user/julien.quintard/julien.quintard.cap
  [+]   platform:               ibm-pc
  [+]   architecture:           ia32/educational

  [+] the snapshot has been re-tested successfully
  $> 
\end{verbatim}

%
% robot
%
\subsubsection{Robot}

The \name{robot} test tool enables contributors to test the kaneton research
implementation on a regular basis; hence control the status of the development.

The robot basically retrieves the kaneton implementation by checking out the
\name{Subversion} repository. Then, several test suites are triggered through
the test client. Once the reports have been received, a message is built
summarizing the results. This message is then sent to the kaneton contributors
mailing-list.

The deployment of the \name{robot} is quite straigthforward. First, the
\location{test/robot/} directory must be copied to the server. Note that
the \name{robot.py} script depends upon the \name{ktp} package which must
therefore be copied as well.

Then, the \name{SSH} configuration file \name{config} must be placed in
the \location{\${HOME}/.ssh/} directory. Besides, this file should be edited in
order to properly reference the \name{SSH} keys since the default configuration
assumes the kaneton test directory to be located at \location{/kaneton/}.

Finally, the \name{robot.cron} crontab file must be setup through the
following command in order to trigger the robot every night:

\begin{verbatim}
  $> crontab robot.cron
\end{verbatim}

Once again, the administrator should make sure to edit this file should
the robot files not be located in the default location \ie{}
\location{/kaneton/}.
