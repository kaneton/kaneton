%
% ---------- header -----------------------------------------------------------
%
% project       kaneton
%
% license       kaneton
%
% file          /home/mycure/kaneton/view/book/development/rules.tex
%
% created       julien quintard   [mon may 28 19:44:49 2007]
% updated       julien quintard   [thu may 31 07:05:15 2007]
%

%
% ---------- rules ------------------------------------------------------------
%

\chapter{Rules}
\label{chapter:rules}

This section describes the rules which apply to any language or tool. These
rules can therefore be considered as the most important ones.

% header

\subsubsection{Header}

All files must start with a file header. This file header specifies the
project name, the license of this file, the file name, the author and
date of the file creation and finally the author and date of the last
edition. This header must comply to the following template, depending
on the sequence of characters used for comments. The example below illustrates
the template for \textit{C} files.

\begin{verbatim}
  /*
   * ---------- header --------------------------------------------------------
   *
   * project       <project>
   *
   * license       <license>
   *
   * file          <file location>
   *
   * created       <first author>   [<creation date>]
   * updated       <last author>   [<last update date>]
   */
\end{verbatim}

Then, an example for \textit{TeX} files:

\begin{verbatim}
  %
  % ---------- header ---------------------------------------------------------
  %
  % project       kaneton
  %
  % license       kaneton
  %
  % file          /home/mycure/kaneton/view/book/development/general.tex
  %
  % created       julien quintard   [mon may 28 19:44:49 2007]
  % updated       julien quintard   [mon may 28 19:48:07 2007]
  %
\end{verbatim}

Note that the kaneton project provides an \textit{Emacs} file which contains
everything necessary to build such headers. This file is located in the
\textit{tool/emacs/} directory.

Obviously, the project and license fields must be filled, in the kaneton
project context, with \textit{kaneton} and \textit{kaneton}, respectively. The
author field must contain the author's full name - firsname and lastname -
in lower case letters. Note that auto-generated values must comply to
the general kaneton rules especially they must be in lower case letters and
must not exceed \textit{80} characters in width.

% markings

\subsubsection{Markings}

Any developer must put the sequence \textit{XXX} everywhere a piece of code
is considered as unfinished. This way, any unfixed piece of code can be
easily retrieved via a very simple command line or script.

% naming

\subsubsection{Naming}

When using a language which does not support namespaces, the developer should
prefix every entity by the package, module etc. name it actually belongs to.

As long as it is possible, entities must be named with a \textit{unique}
word, excluding the namespace prefix.

Names must obviously be expressed in English, without any spelling mistakes.

% layout

\subsubsection{Layout}

Files are composed of sections in order to make the organisation clearer.
Each section starts with a specific header and then contains code, text etc.
related to the section.

A section header is basically a commented separator.

Any file must include a \textit{header} section as explained above. Moreover,
some sections are mandatory depending of the type of file. For instance,
even configuration files, description files etc. must provide a
\textit{information} section.

XXX exemple configuration file: header + information

Files related to any language should provide a section for the core
content like \textit{rules} for \textit{Make} files or \textit{functions}
for many other languages.

For more information on the mandatory sections, please refer to
the sections about the language your are interested in and/or look at
examples in the kaneton repository.

Files must not exceed \textit{80} characters in width, including
the trailing newline character. Moreover, the \textit{DOS} \textit{CR+LF}
line terminator must not be used. Finally, there must not be any whitespace
at the end of a line.

XXX si quelque chose n'est pas conforme il faut le dire
XXX si une regle manque dans ce document il faut l'ajoter, avec accord de
  son superviseur
