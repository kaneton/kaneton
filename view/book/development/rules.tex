%
% ---------- header -----------------------------------------------------------
%
% project       kaneton
%
% license       kaneton
%
% file          /home/mycure/kaneton/view/book/development/rules.tex
%
% created       julien quintard   [mon may 28 19:44:49 2007]
% updated       julien quintard   [thu jun 14 17:12:54 2007]
%

%
% ---------- rules ------------------------------------------------------------
%

\chapter{Rules}
\label{chapter:rules}

This chapter describes the rules which apply to the whole kaneton microkernel
project. These rules can therefore be considered as the most important ones.

\newpage

%
% ---------- text -------------------------------------------------------------
%

% kaneton

\subsubsection{kaneton}

The most important thing when contributing or talking about a project is
to know how to write its name.

The authors of the project decided kaneton must be written in lower case:
\textbf{kaneton}. The same goes for the kaneton project names which contain
a \textit{k} letter in lower-case: \textbf{k1}, \textbf{k2}, \textbf{k3} and
so on.

Also remember that the kaneton project is composed of two distincts
sub-projects named, the \textit{kaneton educational project} and the
\textit{kaneton research project}.

% header

\subsubsection{Header}

Files must start with a file header. This file header specifies the
project name, the license of this file, the file name, the author and
date of the file creation and finally the author and date of the last
edition. This header must comply to the following template, depending
on the sequence of characters used for comments. The example below illustrates
the template for \textit{C} files.

\begin{verbatim}
  /*
   * ---------- header --------------------------------------------------------
   *
   * project       <project>
   *
   * license       <license>
   *
   * file          <file location>
   *
   * created       <first author>   [<creation date>]
   * updated       <last author>   [<last update date>]
   */
\end{verbatim}

Then, an example for \textit{TeX} files:

\begin{verbatim}
  %
  % ---------- header ---------------------------------------------------------
  %
  % project       kaneton
  %
  % license       kaneton
  %
  % file          /home/mycure/kaneton/view/book/development/community.tex
  %
  % created       julien quintard   [mon may 28 19:44:49 2007]
  % updated       julien quintard   [mon may 28 19:48:07 2007]
  %
\end{verbatim}

Note that the kaneton project provides an \textit{Emacs} file which contains
everything necessary to build such headers. This file is located in the
\textit{tool/emacs/} directory.

Obviously, the project and license fields must be filled, in the kaneton
project context, with \textit{kaneton} and \textit{kaneton}, respectively. The
author field must contain the author's full name - firsname and lastname -
in lower-case letters. Note that auto-generated values must comply to
the general kaneton rules especially they must be in lower-case letters and
must not exceed \textit{80} characters in width.

% markings

\subsubsection{Markings}

Any developer must put the sequence \textit{XXX} everywhere a piece of code
is considered as unfinished. This way, any unfixed piece of code can be
easily retrieved via a very simple command line or script.

% naming

\subsubsection{Naming}

When using a language which does not support namespaces, the developer should
prefix every entity by the package, module etc. name it actually belongs to.

As long as it is possible, entities must be named with a \textit{unique}
word, excluding the namespace prefix.

Names must obviously be expressed in English, lower-case letters and
without any spelling mistake.

Composite names should be separated by a dash \textit{-} when the language
allows it, including file names. Otherwise, the underscore character
\textit{\_} must be used.

% layout

\subsubsection{Layout}

Files are composed of sections in order to make the organisation clearer.
Each section starts with a specific header and then contains code, text etc.
related to the section.

A section header is basically a commented separator.

Any file must include a \textit{header} section as explained above. Moreover,
some sections are mandatory depending of the type of file. For instance,
even configuration files, description files, frame file, \textit{Python}
files etc. must provide an \textit{information} section.

Below is an example of a \textit{Python} script which illustrates the
use of sections:

\begin{verbatim}
#
# ---------- header -----------------------------------------------------------
#
# project       kaneton
#
# license       kaneton
#
# file          /home/mycure/kaneton/foo/foo.py
#
# created       julien quintard   [sun may 13 11:04:52 2007]
# updated       julien quintard   [mon may 28 12:42:57 2007]
#

#
# ---------- information ------------------------------------------------------
#
# this script is used to illustrates the use of sections in kaneton
# files.
#

#
# ---------- imports ----------------------------------------------------------
#

import env

import sys
import re

#
# ---------- globals ----------------------------------------------------------
#

g_string = `kaneton'

#
# ---------- functions --------------------------------------------------------
#

#
# main()
#
# this function does the main work: displaying a string.
#
def			main():
  env.display(env.HEADER_NONE, g_string, env.OPTION_NONE)



#
# ---------- entry point ------------------------------------------------------
#

if __name__ == "__main__":
  main()

\end{verbatim}

Files related to any language should provide a section for the core
content like \textit{rules} for \textit{Make} files or \textit{functions}
for many other languages.

For more information on the mandatory sections, please refer to
the sections about the language your are interested in and/or look at
examples in the kaneton repository.

However, note that a section which does not contain any code must not
appear in the file.

Files must not exceed \textit{80} characters in width, including
the trailing newline character. Moreover, the \textit{DOS} \textit{CR+LF}
line terminator must not be used. Finally, there must not be any whitespace
at the end of a line.

An indentation a two spaces must be used. Moreover, the \textit{Emacs}
default indentation must always be taken as a referencial.

% rules

\subsubsection{Rules}

Any contributor which notices a misuage of these rules must inform the
kaneton community and especially its supervisor so that the mistake can be
corrected.

Moreover if something is missing in this book, any contributor is welcomed
to inform the community about it so that a rule is added or modified.

Contributors are asked to read the rules enumerated in this book but also
to look at the kaneton repository as it contains many examples of applications
of these rules.
