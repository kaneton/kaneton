%
% ---------- header -----------------------------------------------------------
%
% project       kaneton
%
% license       kaneton
%
% file          /home/mycure/kaneton/view/book/development/make.tex
%
% created       julien quintard   [sat jun  2 12:34:28 2007]
% updated       julien quintard   [thu jun 14 21:48:18 2007]
%

%
% ---------- make -------------------------------------------------------------
%

\section{Make}
\label{section:make}

The coding style related to \textit{Make} files is very simple.

% naming

\subsection{Naming}

First, \textit{Make} files must be named \textbf{Make} file, with the first
letter in upper case letter.

Rules must be named with a single word in lower-case letters while variables
must be named in upper-case letters.

% environment

\subsection{Environment}

Every \textit{Make} file must rely on the development environment as it
provides a way for performing operations in a very simple and portable way.

Therefore, each \textit{Make} file must start by including the configured
\textit{Make} environment file located in the \textit{environment/} directory
under the name: \textit{env.mk}.

Note that the configured environment \textit{Make} file provides a hidden
rule named \textit{\_}. This rule name is therefore reserved. This rule is
triggered every time a \textit{Make} file is called without specifying
a rule name. This rule aims at verifying the configured environment is
up-to-date. If not, the rule regenerates it. Finally, this rule calls the
\textit{main} rule to perform the main work of the \textit{Make} file.

Therefore, every \textit{Make} file has to provide a \textit{main} rule.

% layout

\subsection{Layout}

\textit{Make} files must be organised as sections.

\begin{itemize}
  \item
    \textbf{header}: this section contains the file header which provides
    information on the file edition: creation, last update etc.
  \item
    \textbf{dependencies}: this section contains the \textit{include}
    directives.

    \begin{verbatim}
      #
      # ---------- dependencies -----------------------------------------------
      #

      include                 ../../environment/env.mk

      [...]

      #
      # ---------- dependencies -----------------------------------------------
      #

      -include                ./$(_DEPENDENCY_MK_)
    \end{verbatim}
  \item
    \textbf{directives}: this section contains \textit{Make} directives
    like \textit{.PHONY}, \textit{.SUFFIXES} etc.

    \begin{verbatim}
      #
      # ---------- directives -------------------------------------------------
      #

      .PHONY:         clear prototypes headers dependencies
    \end{verbatim}
  \item
    \textbf{variables}: this section contains the variable definitions.

    \begin{verbatim}
      #
      # ---------- variables --------------------------------------------------
      #

      SUBDIRS                 :=              arch                      \
                                              kernel                    \
                                              as                        \
                                              conf                      \
                                              id                        \
                                              region                    \
                                              segment                   \
                                              debug                     \
                                              map                       

      CORE_INCLUDE            :=              $(_CORE_INCLUDE_DIR_)/core.h

      CORE_C                  :=              core.c

      CORE_O                  :=              $(CORE_C:.c=.o)
    \end{verbatim}
  \item
    \textbf{rules}: this section contains the \textit{Make} rules.

    \begin{verbatim}
      #
      # ---------- rules ------------------------------------------------------
      #

      $(LIBSTRING_LO):        $(LIBSTRING_O)
              $(call env_remove,$(LIBSTRING_LO),)

              $(call env_archive,$(LIBSTRING_LO),$(LIBSTRING_O),)

      clear:
              $(call env_remove,$(LIBSTRING_O),)

              $(call env_remove,$(LIBSTRING_LO),)

              $(call env_purge,)

      prototypes:
              $(call env_prototypes,$(LIBSTRING_INCLUDE),)

      headers:
              $(call env_remove,$(_DEPENDENCY_MK_),)

              $(call env_headers,$(LIBSTRING_C),)
    \end{verbatim}
\end{itemize}

% style

\subsection{Style}

Recall that the kaneton development environment provides a \textit{Make}
interface allowing \textit{Make} files to perform operations in a portable
way. Thus, \textit{Make} files must not contain any operations specific
to any operating system.

As such, every operation should be a call to one of the \textit{Make} functions
provided by the kaneton development environment interface.

Any contributor writing a \textit{Make} file should take care of properly
cleaning the directory from temporary files. Indeed, every \textit{Make} file
provides a \textit{clear} rule which is intended to remove these temporary
files. Therefore, every developer should take care of that and verify every
file was cleaned before committing.

The kaneton project provides a tool for generating \textit{C} prototypes
through a \textit{Make} rule called \textit{prototypes}. Therefore, the
\textit{Make} file must carefully use this rule to trigger the generation
of prototypes in the right header files. For more information about the
prototypes generation, please refer to \textit{Section
\ref{section:prototypes}}.

One of the other important rules is called \textit{headers} and generated
a file containing all the header file dependencies related to a set of
\textit{C} source files. This generated file must be included in the
\textit{Make} so that it can use it. A \textit{Make} file inclusion is
present at the end of many \textit{Make} files for this purpose:

\begin{verbatim}
  #
  # ---------- dependencies ---------------------------------------------------
  #

  -include                ./$(_DEPENDENCY_MK_)
\end{verbatim}

Additionally, \textit{Make} files must resolve dependencies before performing
any other task. Therefore, the rule name \textit{dependencies} is reserved
for this purpose by launching every \textit{Make} file in charge of one of
the dependency.

Finally, recall that a \textit{main} rule must be provided by every
\textit{Make} file since this rule is called everytime the \textit{Make}
file is launch without rule name argument.
