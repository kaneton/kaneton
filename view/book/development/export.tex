%
% ---------- header -----------------------------------------------------------
%
% project       kaneton
%
% license       kaneton
%
% file          /home/mycure/kaneton/view/book/development/export.tex
%
% created       julien quintard   [wed may 23 18:58:41 2007]
% updated       julien quintard   [sun may 27 21:59:12 2007]
%

%
% ---------- export -----------------------------------------------------------
%

\subsection{Export}

The \textit{export} tool was introduced for making the process of releasing
easier. The tool takes an argument specifying the type of target release.

Recall the kaneton microkernel project is used as a material for operating
system courses. The source code of the microkernel is distributed to
the students with some parts missing. Then, students have to re-write
these pieces of code in order to prove their well-understanding of the
kernel internals. Additionnaly, the kaneton project is also a research project
in operating systems design.

As a result, the \textit{export} tool sometimes has to build a release
with pieces of code removed, sometimes not. Below are listed the different
type of release:

\begin{itemize}
  \item
    \textbf{backup}: this release type is basically a bare backup of
    the kaneton microkernel project source code.
  \item
    \textbf{dist}: the distribution release is performed by removing
    the repository-specific stuff.
  \item
    \textbf{k}$\gamma$\textbf{,}$\tau$: this type of release is intended
    to students. Therefore, repository-specific stuff is removed. Also
    teaching materials such as courses, testing scripts, cheating scripts
    etc.. - specified in the kaneton development environment variable
    \textit{\_HIDDEN\_} - are removed.

    \-

    Finally, the pieces of code comprised in the range $[\gamma,\tau]$
    are removed from the release. These pieces of code are marked using the
    \textit{export} syntax described next.

    \-

    The stages $\gamma$ and $\tau$ represent kaneton sub-project ranks:

    \begin{itemize}
      \item
	\textbf{0}: boot stuff: boostrap, bootloader etc..;
      \item
	\textbf{1}: memory management;
      \item
	\textbf{2}: event management: interrupts, I/O etc..;
      \item
	\textbf{3}: task management, scheduling;
      \item
	\textbf{4}: communication management.
    \end{itemize}
\end{itemize}

Finally, the generated release is named based on the \textit{\_EXPORT\_}
kaneton environment variable followed by the date and type of the release:
\textit{backup}, \textit{dist} or \textit{k}$\gamma$\textit{,}$\tau$.

% syntax

\subsubsubsection{Syntax}

As explained previously, pieces of code are removed in order to build
\textit{stage} releases.

The kaneton source code is marked so that the \textit{export} tool knows
what piece of code to remove and for what stage. Indeed, every piece of
educational code is marked by a tag indicating the stage it is related to.

The syntax used is illustrated below:

\begin{verbatim}
  /*                                                                [cut] k1 */

  /*
   * this function clones a segment.
   *
   * steps:
   *
   * 1) get the original segment object.
   * 2) reserve a new segment of same size with same permissions.
   * 3) copy the data from the old segment.
   * 4) call machine-dependent code.
   */

  t_error                 segment_clone(i_as                      asid,
                                        i_segment                 old,
                                        i_segment*                new)
  {
    o_segment*            from;
    t_perms               perms;

    SEGMENT_ENTER(segment);

    /*
     * 1)
     */

    if (segment_get(old, &from) != ERROR_NONE)
      SEGMENT_LEAVE(segment, ERROR_UNKNOWN);

    [...]

    /*
     * 4)
     */

    if (machdep_call(segment, segment_clone, asid, old, new) != ERROR_NONE)
      SEGMENT_LEAVE(segment, ERROR_UNKNOWN);

    SEGMENT_LEAVE(segment, ERROR_NONE);
  }

  /*                                                               [cut] /k1 */
\end{verbatim}

In this example, the kaneton teachers decided \textit{segment\_clone()}
was a functionality the students should implement.

The markings at the top \verb|[cut] k1| and bottom \verb|[cut] /k1| of this
example indicate the \textit{export} tool the location of the piece of code
to remove for the stage \textit{k1}.

Note that the marked areas must not overlap, the \textit{export} tool's
behaviour being undetermined in such cases.

Let us suppose a teacher $T_{1}$ wants to use kaneton leading the students to
the development of the memory management functionality, only. On the other
hand, another teacher, $T_{2}$, wants to use the whole kaneton project starting
with the bootloader implementation to the task management. Additionally,
this teacher chooses to hide the communication management pieces of code,
to avoid cheating between students of different universities for instance.

In the first case, since the students have to implement the kaneton managers
around the memory management, $T_{1}$ has to provide the students everything
the memory management stuff relies on, including some fundamental managers,
the bootloader etc.. Also, the teacher does not need to provide the source
code of the upper level managers. As a result, a \textit{k1,1} release will
remove the pieces of codes with every marking $k_{\alpha}$ for
$1 \le \alpha \le 1$.

The teacher $T_{2}$ needs something different since the students are going
to implement every major piece of the kaneton source code. Since this teacher
wants their students to implement all the steps, starting with \textit{k0}
to \textit{k4}, a \textit{k0,4} release will not contain the pieces of source
code marked $k_{\alpha}$ for $0 \le \alpha \le 4$.
