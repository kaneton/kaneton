%
% ---------- header -----------------------------------------------------------
%
% project       kaneton
%
% license       kaneton
%
% file          /home/mycure/kaneton/view/book/development/mailing-list.tex
%
% created       julien quintard   [thu may 24 19:55:18 2007]
% updated       julien quintard   [thu may 24 20:43:15 2007]
%

%
% ---------- mailing-list -----------------------------------------------------
%

\subsection{Mailing-List}

Because people do not want to use several communication tools: email,
newsgroup, forum etc.. and because everybody has an email address, the
kaneton people communicate through a mailing-list.

This mailing-list is in fact a \textit{Google} group. Indeed, the kaneton
project relies on three distinct communication groups:

\begin{itemize}
  \item
    \textbf{kaneton} which is used to make announcements about new releases,
    patchs etc..

    \-

    This group is not used yet.
  \item
    \textbf{kaneton-developers} is dedicated to the communication between the
    people involved in the development of the kaneton microkernel reference.

    \-

    This group is therefore private.
  \item
    \textbf{kaneton-students} can be used in an absolutely free-way for
    students for communicating about their kaneton implementation.

    \-

    Anybody can join this group.
\end{itemize}

Needless to say, community behavioural rules enumerated in a previous section
must be followed when communicating on the kaneton mailing-list. Every
contributor is welcomed to give its point of view, to ask questions etc.. but
this must be done with politness, respect and humility.

Everyone communicating through the mailing-list must read the
\textit{Netiquette} which describes the rules inherent to the communication
on the Internet. Especially, people should take care of writing messages in
respect of the \textit{80} columns; and should always cut off useless parts
of a previous message when responding.

It is likely a real-time communication tool will be very useful in the
future, as an \textit{IRC} channel, for instance. However, communicating on
these extra media will not be mandatory until kaneton people decide it is.

kaneton people are asked to use the mailing-list communication medium
in a perfect way as it is the unique intra-communication channel. In
addition then, contributors must read their emails on a regular-basis
as some people rely on the decision of others.

The \textit{kaneton-students} mailing-list must be used carefully. As
an example, people should never paste pieces of source code or ask
questions implying an answer with the solution. Even if it is a free
group, people abusing of this communication channel could easily be banned.

People must always respond in the appropriate discussion. If, in a discussion,
a different subject is discussed, then, one of the contributor must create
a new discussion in order facilitate the communication.

Finally, the discussion subjects must be tagged like the following examples:

\begin{verbatim}
  [ia32/optimised] mapping issues

  [segment] segment_clone() :: bug?

  [research] new paper about OS design
\end{verbatim}

There is no list of official tags. The users are simply asked to make
their discussion subjects as clear as possible to simplify the task
consisting in looking for old topics in the archives. Indeed, remember
that newcomers should - if they respect the rules - look at the archives
to avoid discussing an old subject on the mailing-list.
