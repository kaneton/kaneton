%
% ---------- header -----------------------------------------------------------
%
% project       kaneton
%
% license       kaneton
%
% file          /home/mycure/kaneton/view/book/development/mailing-list.tex
%
% created       julien quintard   [thu may 24 19:55:18 2007]
% updated       julien quintard   [fri nov 28 05:21:53 2008]
%

%
% ---------- mailing-list -----------------------------------------------------
%

\subsection{Mailing-List}
\label{section:mailing-list}

Because people do not want to use several communication tools: email,
newsgroup, forum \etc{} and because everybody has an email address,
kaneton people communicate through a mailing-list.

The kaneton project relies on two distinct communication groups:

\begin{itemize}
  \item
    \location{contributors@kaneton.opaak.org} is dedicated to the communication
    between the people involved in both the development and teaching of the
    kaneton microkernel project.

    \-

    This group is therefore private.
  \item
    \location{students@kaneton.opaak.org} can be used in an absolutely
    free-way by students for communicating about their kaneton educational
    implementation.

    \-

    Anybody can join this group.
\end{itemize}

Needless to say, community rules discussed in \reference{Chapter
\ref{chapter:community}} must be followed when communicating on the kaneton
mailing-lists. Every contributor is welcome to give her point of view, to
ask questions \etc{} but this must be done with politness, respect and
humility.

Everyone communicating through the mailing-list must read the
\name{Netiquette} which describes the rules inherent to the communication
on the Internet. Especially, people should take care of writing messages in
respect of the $80$ columns; and should always cut off useless parts
in previous messages when responding.

It is likely that a real-time communication tool will be very useful in the
future, an \name{IRC} channel for instance. However, communicating on
these extra media will not be mandatory unless kaneton people decide so.

kaneton people are asked to use the mailing-list communication medium
in a perfect way as it is the unique intra-communication channel. In
addition then, contributors must read their emails on a regular-basis
as some people rely on the decision of others.

The \name{students} mailing-list must be used carefully as well. As
an example, people should never paste pieces of source code or ask
questions implying an answer with the solution. Even if it is a free
group, people abusing of this communication channel could be easily banned.

People must always respond in the appropriate discussion. If, in a discussion,
a different subject is discussed, then, one of the contributor must create
a new discussion in order facilitate the communication.

Finally, the discussion subjects must be tagged like the following examples:

\begin{verbatim}
  [ia32/optimised] mapping issues

  [segment] segment_clone() :: bug?

  [research] new paper about OS design
\end{verbatim}

There is no list of official tags. The users are simply asked to make
their discussion subjects as clear as possible.
