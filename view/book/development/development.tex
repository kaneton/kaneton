%
% ---------- header -----------------------------------------------------------
%
% project       kaneton
%
% license       kaneton
%
% file          /home/mycure/kaneton/view/book/development/development.tex
%
% created       julien quintard   [mon may 14 19:56:45 2007]
% updated       julien quintard   [wed may 16 21:42:55 2007]
%

%
% path
%

\newcommand{\path}{../..}

%
% template
%

%%
%% licence       kaneton licence
%%
%% project       kaneton
%%
%% file          /home/mycure/kaneton/view/templates/book.tex
%%
%% created       julien quintard   [wed mar  1 23:45:22 2006]
%% updated       julien quintard   [thu may  4 12:36:54 2006]
%%

%
% class
%

\documentclass[10pt,a4wide]{book}

%
% packages
%

\usepackage[english]{babel}
\usepackage[T1]{fontenc}
\usepackage{a4wide}
\usepackage{fancyheadings}
\usepackage{multicol}
\usepackage{indentfirst}
\usepackage{graphicx}
\usepackage{color}
\usepackage{xcolor}
\usepackage{verbatim}

\usepackage{aeguill}

\usepackage[Lenny]{../../../tools/latex/fncychap}

\pagestyle{fancy}

\setlength{\footrulewidth}{0.3pt}
\setlength{\parindent}{0.3cm}
\setlength{\parskip}{2ex plus 0.5ex minus 0.2ex}

%
% logos
%

\newcommand{\logos}
  {
    \begin{center}
      \includegraphics[scale=0.8]{../../logos/kaneton.pdf}
    \end{center}
  }

%
% colors
%

\definecolor{functioncolor}{rgb}{0.40,0.00,0.00}
\definecolor{commandcolor}{rgb}{0.00,0.00,0.40}
\definecolor{verbatimcolor}{rgb}{0.00,0.40,0.00}
\definecolor{noticecolor}{rgb}{0.87,0.84,0.02}

%
% function
%

\newcommand\function[3]{
  \begin{tabular}{p{0.2cm}p{13.8cm}}
  & {\color{functioncolor}\textbf{#1}}#2
  \end{tabular}

  \begin{tabular}{p{1cm}p{13cm}}
  & #3
  \end{tabular}}

%
% align
%

\newcommand\align[1]{
  \\ & \hspace{#1}}

%
% argument
%

\newcommand\argument[1]{\textit{#1}}

%
% command
%

\newcommand\command[2]{
  \begin{tabular}{p{0.2cm}p{13.8cm}}
  & {\color{commandcolor}\textbf{#1}}
  \end{tabular}

  \begin{tabular}{p{1cm}p{13cm}}
  & #2
  \end{tabular}}

%
% notice
%

\newcommand\notice[1]{
  {\color{noticecolor}\textbf{Notice}}

  \begin{tabular}{p{0.2cm}p{13.8cm}}
  & #1
  \end{tabular}}

%
% example
%

\newcommand\example[1]{
  \textit{Example:}

  \begin{tabular}{p{0.2cm}p{13.8cm}}
  & \textit{#1}
  \end{tabular}}

%
% warning XXX
%

%
% verbatim stuff
%

\makeatletter

\renewcommand{\verbatim@font}
  {\ttfamily\footnotesize\color{verbatimcolor}\selectfont}

\def\verbatim@processline{\hskip15ex\the\verbatim@line\par}

\makeatother

%
% header
%

\rhead{}
\rfoot{\scriptsize{The kaneton microkernel project}}

\date{\scriptsize{\today}}


%
% header
%

\lhead{\scriptsize{The kaneton microkernel project reference}}
\rhead{}

%
% title
%

\title{The kaneton microkernel reference $_{beta}$
       \logos}

%
% authors
%

\author{\small{Julien Quintard}}

%
% document
%

\begin{document}

%
% title
%

\maketitle

%
% --------- text --------------------------------------------------------------
%

This document describes how people can contribute to the kaneton microkernel
project and the rules they have to follow.

This document must be read by everyone involved in the kaneton microkernel
project.

All the kaneton documents are available on
the official website
  \footnote{http://www.kaneton.org}.

%
% toc
%

\tableofcontents

%
% chapters
%

%
% ---------- header -----------------------------------------------------------
%
% project       kaneton
%
% license       kaneton
%
% file          /home/mycure/kane...book/assignments/future/introduction.tex
%
% created       julien quintard   [fri may 23 21:47:38 2008]
% updated       julien quintard   [fri oct 24 17:05:56 2008]
%

%
% ---------- introduction -----------------------------------------------------
%

\chapter{Introduction}
\label{chapter:introduction}

This chapter briefly introduces the purpose of this documentation
and the assignments in general

\newpage

%
% ---------- text -------------------------------------------------------------
%

The \name{kaneton} educational project enables students to develop their own
micro-kernel as a way of understanding operating systems internals.

As anyone can imagine, such a project takes a huge amount of time and
motivation. While the motivation will anyway play an important role in the
success of students' project, the time spent can be greatly reduced if
students focuse on implementing some specific parts rather than developing a
complete micro-kernel from scratch.

Indeed, as we will see later in this document, the current \name{kaneton}
educational project comes with a student \term{snapshot} which contains
a complete development environment as well as the source code skeleton of
the kernel.

As enthusiastic computer scientists, \name{kaneton} authors, maintainers and
teachers can understand than some people prefer working on their own
micro-kernel design and implementation, from scratch. All we can wish to
such people is enough motivation to keep working on their project long
enough to be satisfied, luck and hard work.

Either way, going through the \name{kaneton} micro-kernel documentation
should be a waste of time. Especially, people interested in developing their
own project from scratch could take a look at the \name{kaneton} design
in case they like it enough to implement it their way.

The remaining of this document is organised as follows. \reference{Chapter
\ref{chapter:requirements}} lists what students willing to undertake the
project should know beforehand. \reference{Chapter \ref{chapter:support}}
details the multiple ways for students to get help. \reference{Chapter
\ref{chapter:k0}} presents the first project stage. Then \reference{Chapter
\ref{chapter:snapshot}} presents the student snapshot while \reference{Chapter
\ref{chapter:setup}} introduces the development environment and its set-up.
Then, \reference{Chapter \ref{chapter:k1}}, \reference{Chapter
\ref{chapter:k2}}, \reference{Chapter \ref{chapter:k3}} and \reference{Chapter
\ref{chapter:k4}} details the assignments of the different stages. Finally,
\reference{Chapter \ref{chapter:further}} discusses what students could do
after having undertaken such a project.

%%
% ---------- header -----------------------------------------------------------
%
% project       kaneton
%
% license       kaneton
%
% file          /home/mycure/kaneton/view/book/development/community.tex
%
% created       julien quintard   [sun may 20 18:08:17 2007]
% updated       julien quintard   [fri aug  1 15:51:12 2008]
%

%
% ---------- community --------------------------------------------------------
%

\chapter{Community}
\label{chapter:community}

This chapter discusses what is a community and how contributors must integrate
the kaneton community.

\newpage

%
% ---------- text -------------------------------------------------------------
%

kaneton can obviously be considered as an open source community although the
produced soure code is actually not open source.

Driving an open source community is complicated since people have different
personal goals at working on a free project. Some people contribute for
the knowledge, other for building the next generation system, other to
provide free open source softwares, other to become famous \etc{}

kaneton is a community driven microkernel that acts with the best interest of
the students at heart. Rules and regulations that keep the project moving
forward are fundamental even if the size of the kaneton community is
relatively small, for now.

Indeed, the main objective of the kaneton project remains to be as
understandable as possible in order to lead students to implement parts of
it very quickly.

The remaining of this chapter draws a list of rules contributors must agree
to respect.

% objective

\subsubsection{Objective}

kaneton aims at providing a powerful, understandable and maintainable
microkernel. This objective must be kept in mind of every contributor
since many design and implementation were/are/will be made according to
this precise objective.

Note that the kaneton microkernel does not intend to be a desktop operating
system nor an as optimised as the Linux operating system. Every contributor
should be well-aware of that in order to avoid behaviours stating that a
feature is fundamental or useless for performance concerns, for instance.

This rule does not prohibit people to suggest ideas but instead regulates
behaviours of people who wants to change major design and/or implementation
choices for bad reasons.

% behaviour

\subsubsection{Behaviour}

Open source projects does not mean constraint-free projects. The kaneton
people, whilst being relatively young, try to act for the project's good
by behaving remarkably in the kaneton community.

Therefore, contributors are asked to do the same by avoiding some bad/young
behaviours.

\begin{enumerate}
  \item
    Follow the rules. People who do not respect these rules could be banned
    from the kaneton project.
  \item
    Avoid the \textit{cowboy} behaviour consisting for a contributor to
    implement a feature without discussing about its usefulness with the
    community first. Another effect of this behaviour can be to distract
    the contributors from its major focuses.
  \item
    Always act and think in the project interest rather than your personal
    interest.
  \item
    Respect the other kaneton people, especially the ones who have worked
    on this project for a long time and who made this whole project possible.
    When people disagree, they are asked to do it respectfully.
  \item
    Take your responsability when you realise that you did something wrong:
    insults, mistakes in an implemented feature \etc{}
  \item
    \textit{``The Perfect is the Enemy of the Good!''}: even nice
    contributors can unintentionally do bad things by being perfectionists
    and/or too much into the project and/or obsessed with process.
  \item
    ... \textit{Politness}, \textit{Respect}, \textit{Trust} and
    \textit{Humility} are the key qualities that make a good contributor in
    any community.
\end{enumerate}

% communication

\subsubsection{Communication}

The communication mainly takes two forms in the kaneton microkernel project:
the \name{mailing-list} for internal communication and the \name{kaneton
website} for external communication. The \name{Developers Intranet} is another
source of communication as well as the commit logs \etc{}

The rules related to these tools are described in \reference{Chapter
\ref{chapter:tools}} and will therefore not be discussed here.

Every contributor must take the time to communicate in the mailing-list as well
as through the public website. Indeed, kaneton people must, frequently,
briefly describe what they are working on in order to inform the other
contributors who are not aware of everyone's current work. Note that these
kind of messages are very different from messages generated by repository
commits. Indeed, while these commit messages indicate a modification, they
do not describe the whole work behind them.

Additionally, contributors can communicate informing the kaneton community of
their unavailability for the next two months, for instance. This behaviour
allows people to be aware that some tasks will not be done because the
contributors in charge of it cannot work at this moment.

Althoug people are highly welcomed to communicate, some rules apply to
avoid further problems.

First, any new contributor should obviously read the kaneton documents before
asking anything which has already been discussed and decided, unless the
developer knows exactly what he/she is talking about. Indeed, asking too many
questions about the source code is a form of disrespect to the other
contributors. Moreover, many things can be found out just looking at the
kaneton documents and/or source code.

Although people are asked to communicate, people are also asked to act
respectully. Contributors should not respond to every message in every
discussion, this is a ridiculous behaviour. Instead, every developer should
carefully read the discussion, think about its response and then write a clear
message stating his point of view, ideas \etc{}

Depending on the contributor status, reading the mailing-list frequently is
absolutely fundamental as some people rely on other contributors' decisions,
advices \etc{}

Finally, the mailing-list must be considered as the official internal
communication medium. If contributors previously had a private conversation,
in real-life or on \name{IRC} for instance, the discussion must be reported
on the mailing-list so that everyone can take these new ideas into account.

% work

\subsubsection{Work}

Working on the kaneton microkernel project does not imply low-level programming
all the time. Indeed, the kaneton project is composed of two parts: the
kaneton microkernel research project and the kaneton educational project.

Although the microkernel research project requires highly skilled programmers,
it also needs documentations and some tools for performing important tasks as
diverse as generating the prototypes, testing the microkernel behaviour,
generating the documentation \etc{}

The educational project essentially needs documentation, lecture materials
and tools for managing the project: testing the students' implementation,
checking if some students cheated and many others.

This means that kaneton people must contribute to every type of task
that need to be done. Also, contributors are asked to well document
any work they have done including source code comments but also through
kaneton official documents which are then made available on the website.

% supervisor

\subsubsection{Supervisor}

A supervisor is attached to each new contributor for a certain, not fixed,
period of time. The role of this supervisor is to advise, correct and
encourage the newcomer so that it integrates well the kaneton community.

The supervisor will be someone having a well understanding of the new
contributor's project. A high level of communication is expected between
the supervisor and the contributor. A direct phone communication is highly
recommanded through \name{Skype} for instance.

% trust

\subsubsection{Trust}

A new contributor joining the kaneton project must acquire the trust of
the community. Therefore, the contributor first does not get any access
to restricted tools and must submit patches to its supervisor who review
them, taking care of advising the newcomer of its mistakes.

At the end of the test period, the community decides whether the contributor
is helpful to the project or not. Then, the contributor is either granted
of full access to kaneton tools or eliminated from the project.

%%
% ---------- header -----------------------------------------------------------
%
% project       kaneton
%
% license       kaneton
%
% file          /home/mycure/kaneton/view/book/development/source-tree.tex
%
% created       julien quintard   [thu may 17 22:41:36 2007]
% updated       julien quintard   [thu may 31 08:34:23 2007]
%

%
% ---------- source tree ------------------------------------------------------
%

\chapter{Source Tree}
\label{chapter:source tree}

In this chapter we will briefly describe the kaneton microkernel project
source tree.

\newpage

%
% ---------- text -------------------------------------------------------------
%

The kaneton microkernel reference source tree looks like the following
listing:

\begin{verbatim}
cheat/
configure/
environment/
export/
history/
kaneton/
library/
license/
test/
tool/
transcript/
view/
\end{verbatim}

%
% cheat/
%

\subsection*{cheat/}

Since the kaneton microkernel is implemented by students, the kaneton
people need to check whether students are cheating by re-using parts of
previous years projects or other kernel source codes available on the
\textit{Internet}.

To avoid cheating, kaneton people developed a software checking for
commonalities between different source codes.

This directory contains scripts that performs these verifications. However,
the students work over the years are not stored in this directory but in
the \textit{history/} directory instead.

%
% configure/
%

\subsection*{configure/}

This directory contains everything necessary for configuring its own
kaneton microkernel development environment through the compiling process
to the boot system.

Any new contributor should first look at this directory. However, note that
this directory mainly contains tools targeting final-users rather than
kaneton contributors. Indeed, for instance, the \textit{configure} utility
aims at providing a user-friendly way for configuration but does not take
advantage of the power of the kaneton development environment.

Contributors should then learn about how the development environment works
while final-users should use the \textit{configure} tool.

%
% environment/
%

\subsection*{environment/}

This directory contains everything necessary to the kaneton development
environment.

The kaneton development environment allows different developers to
interact on the development of the same microkernel in a pretty easy way.

The development environment aims at providing developers to possibility to
work in a collaborative manner without interfering with each other. These
developers are likely to run different operating systems on different
microprocessors. In addition, the kaneton microkernel can be targeted for
different microprocessor architectures. The development environment was
introduced to cope with these combinations by providing profiles, each
profile describing the behaviour of a component: underlying operating system,
target architecture, user-specific stuff etc.

As a result, each developer can use a different operating system and
microprocessor architecture with its own specific compiling flags, kaneton
parameters etc. without modifying another developer's configuration.

The development environment is detailed in \textit{Section
\ref{section:environment}}.

%
% export/
%

\subsection*{export/}

The \textit{export/} directory contains scripts used to generate a kaneton
tarball in order to be distributed to the students at the beginning of the
kaneton educational project.

Indeed, these scripts rearrange the kaneton hierarchy hidding some important
directories the students do not need to be aware of. Moreover some source
code parts are removed since the students have to rewrite these pieces
of code as assignments.

These scripts are also used for making backups and distribution tarbalss of
the kaneton microkernel.

%
% history/
%

\subsection*{history/}

The \textit{history/} directory contains the students work over the years
in the universities and schools the kaneton project was used as an operating
system course's implementation material.

The tools of the \textit{cheat/} directory use these students works for
performing cheating verifications.

%
% kaneton/
%

\subsection*{kaneton/}

This directory is the most important of the project since it contains
the whole microkernel source code.

The directory is composed of three important subdirectories: \textit{core/},
\textit{platform/} and \textit{architecture/}. These subdirectories are
described next.

% core/

\subsubsection*{core/}

This directory contains the kaneton core source code.

The directory is divided as shown below:

\begin{verbatim}
as/
region/
sched/
segment/
set/
task/
thread/
[...]
\end{verbatim}

Each directory represents a kaneton core manager. For more information on
the kaneton core, please refer to the appropriate document:
\textit{The kaneton microkernel :: core}

% platform/

\subsubsection*{platform/}

This directory contains everything in relation with what the kaneton
microkernel project calls a \textit{platform}. The platform represents the
board supporting the devices: microprocessor, memory, peripherals etc.

This directory obviously contains subdirectories for each platform
supported by the kaneton microkernel.

% architecture/

\subsubsection*{architecture/}

The \textit{architecture/} directory contains the source-code related to
the microprocessor architectures supported by the kaneton microkernel.

This directory is composed of subdirectories, each one representing a
supported architecture: \textit{ia32}, \textit{mips64} etc. Note that each
architecture can be specialised. For instance, the \textit{ia32/optimised}
architecture represents an optimised implementation of the \textit{Intel IA-32}
microprocessor architecture.

%
% library/
%

\subsection*{library/}

This directory contains the libraries used by the kaneton microkernel itself,
the kaneton microkernel servers or maybe both. This directory especially
contains the standard \textit{kaneton C library}.

%
% license/
%

\subsection*{license/}

This directory contains the licenses used for any program or document
in relation with the kaneton microkernel project. Indeed, the kaneton
microkernel is under the \textit{kaneton license} which is described in
depth in the documents contained in this directory. Note that these licenses
are also available in \textit{Chapter \ref{chapter:licenses}}.

Each student has to read and agree with the kaneton license before
implementing or even using the kaneton microkernel project..

Indeed, every user of the kaneton-related stuff is considered as having
implicitly accepted the kaneton license.

%
% test/
%

\subsection*{test/}

Since the kaneton microkernel is used as a material for operating system
courses, the kaneton microkernel reference, which is the basis of students
work, must be extremely reliable.

The kaneton project therefore contains a set of tools in order to validate
the kaneton reference implementation behaviour. These tools are also used
for evaluating the correctness of the students implementation.

The \textit{test/} directory contains the set of kaneton scripts and tests
for validating a kaneton microkernel implementation.

%
% tool/
%

\subsection*{tool/}

This directory contains additional scripts and configuration files used by
the kaneton development environment or the kaneton developers.

As examples, this directory contains scripts for generating prototypes,
building a boot device etc.

%
% transcript/
%

\subsection*{transcript/}

This directory contains real-time recorded sessions. These sessions can be
replayed in order to present a feature of the development environment or
of the kaneton microkernel.

%
% view/
%

\subsection*{view/}

This directory contains all the kaneton documents including kaneton
administrative documents, examinations, lectures materials, kaneton papers
and books etc.

Additionally, scripts are provided in order to very easily build and
display these documents.
%%%
%% copyright quintard julien
%% 
%% kaneton
%% 
%% development-environment.tex
%% 
%% path          /home/mycure/kaneton
%% 
%% made by mycure
%%         quintard julien   [quinta_j@epita.fr]
%% 
%% started on    Tue Jul  5 12:23:08 2005   mycure
%% last update   Sun Oct 23 02:55:45 2005   mycure
%%

%
% class
%

\documentclass[8pt]{beamer}

%
% packages
%

\usepackage{pgf,pgfarrows,pgfnodes,pgfautomata,pgfheaps,pgfshade}
\usepackage{colortbl}
\usepackage{times}
\usepackage{amsmath,amssymb}
\usepackage{graphics}
\usepackage{graphicx}
\usepackage{color}
\usepackage{xcolor}
\usepackage[english]{babel}
\usepackage{enumerate}
\usepackage[latin1]{inputenc}

%
% style
%

\usepackage{beamerthemesplit}
\setbeamercovered{dynamic}

%
% verbatim font
%

\definecolor{verbatimcolor}{rgb}{0,0.4,0}

\makeatletter
\renewcommand{\verbatim@font}
  {\ttfamily\footnotesize\color{verbatimcolor}\selectfont}
\makeatother

%
% new line
%

\newcommand{\nl}[0]{\vspace{0.4cm}}

%
% title
%

\title{Development Environment}

%
% authors
%

\author
{
  Julien~Quintard\inst{1} \\
  {\tiny julien.quintard@gmail.com}
}

\institute
{
  \inst{1} kaneton distributed operating system project
}

%
% date
%

\date{\today}

%
% logos
%

\pgfdeclareimage[interpolate=true,width=34pt,height=18pt]
                {epita}{../../logos/epita}
\pgfdeclareimage[interpolate=true,width=49pt,height=18pt]
                {upmc}{../../logos/upmc}
\pgfdeclareimage[interpolate=true,width=25pt,height=18pt]
                {lse}{../../logos/lse}

%
% table of contents at the beginning of each section
%

\AtBeginSection[]
{
  \begin{frame}<beamer>
   \frametitle{Outline}
    \tableofcontents[current]
  \end{frame}
}

%
% table of contents at the beginning of each subsection
%

\AtBeginSubsection[]
{
  \begin{frame}<beamer>
   \frametitle{Outline}
    \tableofcontents[current,currentsubsection]
  \end{frame}
}

%
% document
%

\begin{document}

%
% title frame
%

\begin{frame}
  \titlepage

  \begin{center}
    \pgfuseimage{epita} \hspace{0.1cm} \pgfuseimage{upmc} \hspace{0.1cm}
    \pgfuseimage{lse} \hspace{0.1cm}
  \end{center}
\end{frame}

%
% outline frame
%

\begin{frame}
  \frametitle{Outline}
  \tableofcontents
\end{frame}

%
% overview
%

\section{Overview}

% 1)

\begin{frame}
  \frametitle{Introduction}

  From the previous years, a development environment was introduced.

  \nl

  The questions are:

  \begin{enumerate}[<+->]
    \item
      Why?
    \item
      What are the advantages and disadvantages of such a
      development environment?
    \item
      How did the other promotions do?
  \end{enumerate}
\end{frame}

% 2)

\begin{frame}
  \frametitle{Explanations}

  Over the years, the kaneton project evolved, starting with a very
  simple introduction to low-level programming, to microkernel
  development and finally to a distributed operating system project.

  \nl

  Going always further implies many modifications in the project
  including:

  \begin{itemize}[<+->]
    \item
      The courses given which now go from the Intel processor to
      the distributed operating system concepts
    \item
      The assignments which always evolve to study advanced topics
    \item
      The context because we now have to provide parts of the microkernel
      to avoid students a development from scratch
    \item
      .. and so the requirements
  \end{itemize}
\end{frame}

% 3)

\begin{frame}
  \frametitle{The Courses}

  The kaneton project now comes with four courses:

  \begin{enumerate}
    \item
      The design of the kaneton distributed operating system including
      the microkernel
    \item
      The Intel processor
    \item
      The kernel concepts
    \item
      The distributed operating system concepts
  \end{enumerate}
\end{frame}

% 4)

\begin{frame}
  \frametitle{The Assignments}

  During the year 2005, the students develop a poor microkernel
  from scratch with few functionalities, a driver and finally a baby
  file system.

  \nl

  We cannot ask the students of the year 2006 to develop the same project
  but to go further to study advanced topics like distributed algorithms.

  \nl

  So, we cannot ask the students to develop every parts of the microkernel
  because this takes much time and implies to not study advanced
  topics.
\end{frame}

% 5)

\begin{frame}
  \frametitle{The Context}

  Providing students parts of the microkernel is not enough.

  \nl

  Indeed, we decided to provide a complete development environment
  including:

  \begin{itemize}
    \item
      Makefiles
    \item
      Shell scripts
    \item
      Papers
    \item
      Tools
    \item
      .. everything you need to start microkernel development
  \end{itemize}
\end{frame}

% 6)

\begin{frame}
  \frametitle{Why?}

  The remaining question is:

  \nl

  \textbf{Why providing such a development environment and not letting us
    develop one ourself?}

  \nl

  The answers simply are:

  \begin{itemize}
    \item
      Developing such a development environment takes much time and
      need experience
    \item
      This development environment include very powerful features:
      multiusers cooperation, different operating systems etc..
    \item
      Finally, students will not be able to create such a complicated
      development tree so it is provided to not waste time.
  \end{itemize}
\end{frame}

% 7)

\begin{frame}
  \frametitle{The Requirements}

  The students starting the kaneton project should think that they
  will learn many many things during the year.

  \nl

  This year, we are trying to lead students to a distributed operating
  system.

  \nl

  This implies more concepts, algorithms and techniques to learn.

  \nl

  To do this we introduced more courses but the students will have
  to work hard to be able to success.
\end{frame}

% 8)

\begin{frame}[containsverbatim]
  \frametitle{Tree}

  \begin{center}

  \begin{verbatim}
    /
      conf/
      core/
      doc/
      drivers/
      env/
      export/
      libs/
      papers/
      programs/
      services/
      tools/
  \end{verbatim}

  \end{center}
\end{frame}

%
% conf
%

\section{conf}

% 1)

\begin{frame}
  \frametitle{Overview}

  The \textbf{conf} directory contains user variables used to parameterise:

  \begin{itemize}
    \item
      the development environment: makefiles, scripts etc..
    \item
      the kernel
  \end{itemize}

  \nl

  This configuration system is very interesting coupled with versionning
  system.

  \nl

  Indeed, you can develop using special compilation flags, specific kernel
  configuration without conflicts with other developers.
\end{frame}

% 2)

\begin{frame}[containsverbatim]
  \frametitle{Tree}

  \begin{verbatim}
    conf/
      mycure/
        conf.c
        conf.h
        kaneton.conf
        modules.conf
        mycure.conf
      pwipwi/
      chiche/
  \end{verbatim}

  This configuration system uses the shell variable \$USER to find
  the main configuration file: \textbf{conf/\$USER/\$USER.conf}.
\end{frame}

% 3)

\begin{frame}
  \frametitle{conf.c}

  This file is not used yet.
\end{frame}

% 4)

\begin{frame}
  \frametitle{conf.h}

  This file contains macros to configure the kernel:

  \begin{itemize}
    \item
      \textbf{CONF\_TITLE}
    \item
      \textbf{CONF\_VERSION}
    \item
      \textbf{CONF\_DEBUG}
    \item
      etc..
  \end{itemize}

  \nl

  This file is included by the kernel code.
\end{frame}

% 5)

\begin{frame}
  \frametitle{kaneton.conf}

  This configuration file is used to pass arguments at the runtime to the
  servers.

  \nl

  This file is also used to configure kernel and servers input variables.
\end{frame}

% 6)

\begin{frame}
  \frametitle{modules.conf}

  This file contains the list of the modules to be loaded by the
  multi-bootloader.

  \nl

  These modules will be passed to the kernel at the boot time.

  \nl

  Be careful, a module here is not a module in the Linux or BSD terms.

  \nl

  A module is simply a file to load.
\end{frame}

% 7)

\begin{frame}
  \frametitle{\$USER.conf}

  Finally the main configuration file contains the configuration
  variables for the development environment.

  \nl

  This file uses the syntax of the make files.

  \nl

  Every variable defined in this file will be used by the makefiles
  and the scripts.
\end{frame}

%
% env
%

\section{env}

% 1)

\begin{frame}
  \frametitle{Overview}

  The \textbf{env} directory contains the different development environments.

  \nl

  This directory is the heart of the kaneton development system.

  \nl

  Indeed, a user can develop the kaneton project on a Mac machine using
  cross compilation for Intel processors ('cause PowerPC processor)
  while another one is using a FreeBSD operating system on an Intel processor.

  \nl

  So, the development environment has to deal with these different operating
  systems and architectures just for the development.
\end{frame}

% 2)

\begin{frame}
  \frametitle{Our System}

  To do this, we decided to introduce an environment system.

  \nl

  Every time a user gets the kaneton development tarball, he first has to
  create his development environment given a couple operating system and
  architecture which leads to an environment.

  \nl

  Once the environment is installed, the user can develop, compile the kernel
  etc.. without problems because everything (makefiles, scripts etc..) use
  the binaries, variables etc.. for his environment.

  \nl

  The environment is specified in the user configuration file.
\end{frame}

% 3)

\begin{frame}[containsverbatim]
  \frametitle{Tree}

  \begin{verbatim}
    env/
      clean.sh
      init.sh
      unix/
        clean.sh
        init.sh
        kaneton.mk
      macos-powerpc.ia32/
  \end{verbatim}

  \nl

  Here the \textbf{unix} is considered as the generic unix
  environment but everyone can add a specific linux, FreeBSD, Solaris etc..
  environment.
\end{frame}

% 4)

\begin{frame}
  \frametitle{init.sh}

  The \textbf{init.sh} shell script is used to install the development
  environment.

  \nl

  This script first gets the configuration variables from the user
  configuration file, then calls the specific \textbf{init.sh} script
  of the given environment.

  \nl

  Finally the script installs some links and initialises the makefile
  dependencies.

  \nl

  The \textit{[environment]}/init.sh shell script is used to install
  specific stuff.
\end{frame}

% 5)

\begin{frame}
  \frametitle{clean.sh}

  The \textbf{clean.sh} shell script just cleans the environment.

  \nl

  This shell script also call the environment specific clean.sh script.
\end{frame}

% 6)

\begin{frame}
  \frametitle{kaneton.mk}

  The \textbf{kaneton.mk} makefile dependency is the heart of the
  kaneton compilation system.

  \nl

  Indeed, every makefile is composed of calls to special routines
  which are implemented by the makefile dependency depending on the
  environment: operating system plus architecture source and destination.

  \nl

  Moreover the \textbf{kaneton.mk} makefile dependency includes the
  user configuration file so each makefile of the system is able to
  use user defined variables.

  \nl

  The kaneton compilation system uses a very special gmake feature:
  the makefile \textbf{call} function.
\end{frame}

% 7)

\begin{frame}[containsverbatim]
  \frametitle{Use}

  \begin{verbatim}
    $ make init
    [+] installing environment

    [+] your current configuration:
    [+]   environment:              unix
    [+]   architecture:             ia32
    [+]   multi-bootloader:         grub

    [...]

    $ make clean
    [+] cleaning environment

    [...]

    $ 
  \end{verbatim}
\end{frame}

%
% tools
%

\section{tools}

% 1)

\begin{frame}
  \frametitle{Overview}

  The \textbf{tools} directory contains programs, scripts, special
  files used by the kaneton project.

  \nl

  For example a script to initialise and install modules on a grub
  bootloader boot device is provided in the subdirectory
  \textit{scripts/multi-bootloaders/grub/}.

  \nl

  The \textbf{tools} directory also contains the ld scripts used
  to correctly compile the bootstrap, the bootloader, the kernel, the
  drivers, the services and the programs.
\end{frame}

% 2)

\begin{frame}[containsverbatim]
  \frametitle{Tree}

  \begin{verbatim}
    tools/
      scripts/
        ld/
          arch/
            ia32/
              bootstrap.lds
              bootloader.lds
              kaneton.lds
              driver.lds
              service.lds
              user.lds
        multi-bootloaders/
          grub/
          lilo/
        prototypes/
          mkp.py
  \end{verbatim}
\end{frame}

% 3)

\begin{frame}[containsverbatim]
  \frametitle{Use}

  \begin{verbatim}
    $ make build
    [+] initialising boot system

    [+] boot system initialised successfully
    $ make install
    [+] initialising boot system

    [+] /tmp/menu.lst
    [+] core/bootloader/bootloader
    [+] core/kaneton/kaneton
    [+] conf/mycure/kaneton.conf
    [+] drivers/cons/cons
    [+] services/dsh/dsh

    [+] boot system initialised successfully
    $ 
  \end{verbatim}
\end{frame}

% 4)

\begin{frame}[containsverbatim]
  \frametitle{Prototypes}

  The compilation system permits to generate the prototypes in a very easy
  and elegant way.

  \begin{verbatim}
    $ make proto
    [PROTOTYPES]            libdata.h
    [PROTOTYPES]            libstring.h
    [PROTOTYPES]            libsys.h
    [PROTOTYPES]            bootloader.h
    [PROTOTYPES]            ia32.h
    [PROTOTYPES]            kaneton.h
    [PROTOTYPES]            as.h
    [PROTOTYPES]            conf.h
    [PROTOTYPES]            serial.h

    [...]

    $ 
  \end{verbatim}
\end{frame}

% 5)

\begin{frame}[containsverbatim]
  \frametitle{Explanations}

  This system is based on tags in the header files which specify
  from which files to extract prototypes.

  \nl

  The tags are of the form:

  \begin{verbatim}
    /*
     * ---------- prototypes -------------------------------------------------
     *
     *      ../../kaneton/set/set.c
     *      ../../kaneton/set/set_array.c
     *      ../../kaneton/set/set_ll.c
     *      ../../kaneton/set/set_bpt.c
     */
  \end{verbatim}
\end{frame}

% 5)

\begin{frame}[containsverbatim]
  \frametitle{Dependencies}

  The compilation system uses full dependencies between files.

  \nl

  To regenerate the dependencies, for example when adding a
  \textit{\#include} c-preprocessor directive in a source file:

  \begin{verbatim}
    $ make dep
    [REMOVE]                .makefile.mk
    [DEPENDENCIES]          dump.c
    [DEPENDENCIES]          alloc.c
    [DEPENDENCIES]          sum2.c

    [...]

    $ 
  \end{verbatim}
\end{frame}

%
% libs
%

\section{libs}

% 1)

\begin{frame}
  \frametitle{Overview}

  The \textbf{libs} directory contains the libraries used by the kaneton
  project like:

  \begin{itemize}
    \item
      libc
    \item
      crt
    \item
      libposix
    \item
      etc..
  \end{itemize}
\end{frame}

%
% core
%

\section{core}

% 1)

\begin{frame}
  \frametitle{Overview}

  The \textbf{core} directory contains the source code for the microkernel
  including the bootstrap, the bootloader and the kernel itsef.

  \nl

  Each part contains an \textbf{arch} directory used for architecture
  dependent soure code.
\end{frame}

% 2)

\begin{frame}[containsverbatim]
  \frametitle{Tree}

  \begin{verbatim}
    core/
      bootstrap/
        arch/
          ia32/ <---;
          machdep --+
      bootloader/
        arch/
      kaneton/
        arch/
        as/
        conf/
        debug/
        id/
        segment/
        set/
        stats/
  \end{verbatim}
\end{frame}

%
% drivers
%

\section{drivers}

% 1)

\begin{frame}
  \frametitle{Overview}

  The \textbf{drivers} directory contains the drivers of the kaneton
  microkernel.

  \nl

  A driver, in the kaneton terms, is a microkernel server which is allowed
  to communicate with hardware devices.
\end{frame}

% 2)

\begin{frame}[containsverbatim]
  \frametitle{Tree}

  \begin{verbatim}
    drivers/
      cons/
        Makefile
        cons.c
      dma/
      kbd/
      ide/
  \end{verbatim}
\end{frame}

%
% services
%

\section{services}

% 1)

\begin{frame}
  \frametitle{Overview}

  The \textbf{services} directory contains the services of the kaneton
  microkernel.

  \nl

  A service, in the kaneton terms, in simply a server which does not
  communicate with the hardware.
\end{frame}

% 2)

\begin{frame}[containsverbatim]
  \frametitle{Tree}

  \begin{verbatim}
    services/
      dsh/
      mod/
        Makefile
        mod.c
        modfs.c
  \end{verbatim}
\end{frame}

%
% programs
%

\section{programs}

% 1)

\begin{frame}
  \frametitle{Overview}

  The \textbf{programs} directory contains the sources of common
  programs.

  \nl

  A program in the kaneton terms is just a non-privilegied
  process.
\end{frame}

% 2)

\begin{frame}[containsverbatim]
  \frametitle{Tree}

  \begin{verbatim}
    programs/
      ls/
      wc/
      cat/
      mount/
      umount/
      gcc/
      emacs/
  \end{verbatim}
\end{frame}

%
% export
%

\section{export}

% 1)

\begin{frame}
  \frametitle{Overview}

  The \textbf{export} directory is used to create kaneton distribution.

  \nl

  This feature is especially used by the maintainers of the kaneton
  project which create very special kaneton distribution for
  the students.
\end{frame}

% 2)

\begin{frame}[containsverbatim]
  \frametitle{Use}

  The only way to export kaneton is to do like this:

  \begin{verbatim}
    $ make export
    [!] usage: exporter.sh [stage]

    available stages: k0 k1 k2 k3 k4 k5 k6 k7 k8 k9 kaneton dist
    $ make export-k3
  \end{verbatim}

  \begin{itemize}
    \item
      \textbf{k[0-9]}: create a special kaneton version for the k[0-9]
      subproject
    \item
      \textbf{kaneton}: create an entire kaneton version for the lastest
      subproject
    \item
      \textbf{dist}: create an entire backup of the kaneton development
      project
  \end{itemize}
\end{frame}

%
% papers
%

\section{papers}

% 1)

\begin{frame}
  \frametitle{Overview}

  The \textbf{papers} directory contains the papers and lectures
  in relation with the kaneton project.

  \nl

  We prefered set the papers directly into the tarball so every student
  can easily read them.
\end{frame}

% 2)

\begin{frame}[containsverbatim]
  \frametitle{Tree}

  \begin{verbatim}
    papers/
      assignments/
      design/
      kaneton/
      seminar/
      lectures/
        kernels/
        inline-assembly/
        c-preprocessor/
        distributed-operating-systems/
        arch-ia32/
  \end{verbatim}
\end{frame}

% 3)

\begin{frame}[containsverbatim]
  \frametitle{Use}

  \begin{verbatim}
    $ make view
    [+] papers:

    [+]   assignments
    [+]   design
    [+]   arch-ia32
    [+]   c-preprocessor
    [+]   distributed-operating-systems
    [+]   inline-assembly
    [+]   kernels
    [+]   development-environment

    [!] usage: viewer.sh [paper]
    $ make view-design
  \end{verbatim}
\end{frame}

%
% doc
%

\section{doc}

% 1)

\begin{frame}
  \frametitle{Overview}

  The \textbf{doc} directory contains every document useful for
  the development of the kaneton project.

  \nl

  This directory will theorically contain documents on the different
  architectures, documents on some hardware devices like ide, usb etc..
\end{frame}

\end{document}

%\include{coding}
%\include{licenses}
%%%
%% licence       kaneton licence
%%
%% project       kaneton
%%
%% file          /home/mycure/kaneton/view/papers/kaneton/bibliography.tex
%%
%% created       julien quintard   [mon may  8 18:35:35 2006]
%% updated       julien quintard   [mon may  8 20:38:56 2006]
%%

%
% bibliograpy
%

\chapter{Bibliography}

This chapter contains the bibliography.

%
% text
%

\begin{thebibliography}{0}
  \bibitem{AST-SCO}
    \textbf{Structured Computer Organization};
    by
    \textit{Andrew S. Tanenbaum}
  \bibitem{AST-CN}
    \textbf{Computer Networks};
    by
    \textit{Andrew S. Tanenbaum}
  \bibitem{AST-OSDI}
    \textbf{Operating Systems: Design and Implementation};
    by
    \textit{Andrew S. Tanenbaum, Albert S Woodhull}
  \bibitem{AST-MOS}
    \textbf{Modern Operating Systems};
    by
    \textit{Andrew S. Tanenbaum}
  \bibitem{AST-DOS}
    \textbf{Distributed Operating Systems};
    by
    \textit{Andrew S. Tanenbaum}
  \bibitem{AST-DSPP}
    \textbf{Distributed Systems: Principles and Paradigms};
    by
    \textit{Andrew S. Tanenbaum, Maarten van Steen}
  \bibitem{NAL-DA}
    \textbf{Distributed Algorithms};
    by
    \textit{Nancy A. Lynch}
\end{thebibliography}


\end{document}

%
% ---------- intro ------------------------------------------------------------
%

kaneton est un projet pedagogique -> code clair, documente, cours etc..

un point tres important est la clarete du code. cela implique une rigeur
exceptionnel de la part des developpeurs.

le projet est compose du code du kernel, d'un certain nombre de scripts
et makefiles pour faciliter bcp d'operations et de documentations.

ce document decrit donc en profondeurs les regles qu'un contributeur doit
absolument suivre mais revient egalement sur la philisophie d'un travail
communautaire qui implique un comportement exemplaire de la part de
chaque developpeurs.

%
% ---------- community --------------------------------------------------------
%

regles communautaire.

-- comportement

mieux plutot que meilleur

savoir participer en faisant des trucs moins cool que d'autres car ca
doit etre fait

...

le projet dispose de certains outils pour effectuer le minimum vital:
outils de comm interne, dev communautaire, outils de comm ext, gestion
de taches.

%
% ---------- tools ------------------------------------------------------------
%

regles concernant les outils.

-- communication

parceque c'est chiant d'avoir plusieurs outils de comm mail, news, forum
et que tout le monde dispose d'un mail, on utilise une ml.

comportement tres important sur une ml: eviter les trolls, repondre
en suvant la netiquette, participer en donnant son avis etc..

pour la communication temps-reel, un serveur irc pourrait etre mis en
place mais ce n'est pas le cas actuellemeent. si vous en ressentez le
desir, proposez l'idee sur la ml. dans tous les cas, cela ne sera pas
obligatoire.

de maniere generale, des outils libres, pre-fournis et simple d'utilisation
seront privlegies. ainsi google fournit un certains nombre d'outils que
kaneton utilise: gmail opur les mails si vous le desirez, googlegroup pour
la ml, project hosting pour le projet ..

-- repository

le repository est l'outil le plus important du projet puisque tout y
est stocke, le code bien sur mais egalement les documentations, les
projets des etudiants au fil des annees ...

le repository actuel contient toute l'histoire du projet kaneton depuis
le premier jour jusqu'a aujourdhui. nous utilisons subversion car a l'epoque
c'etait de loin le meilleur outil, son concurrent CVS etant deja trop
limite.

durant les premieres annees aucune normalisation de l'utilisation du
repository ne fut instauree. maintenant, les commits doivent suivre une
procedure praticuliere qui implique un descriptif normalise.

de plus l'organisation du repository est tres tres important et chaque ajout
doit etre fait intelligemment afin d'eviter que quelqu'un soit oblige de
repasser derriere pour reparer ces erreurs.

-- wiki

le wiki contient trois parties: une partie publique et deux wikis: un pour
les etudiants et un pour les developpeurs.

le wiki kaneton.org doit etre utilise en priorite du wiki associe avec
le systeme de tickets etc..

participer au wiki est tres important puisqu'il permet de mettre en avant
l'evolution du projet.

neanmoins suivant le nombre de developpeurs impliques il est normal que
certains soient moins actifs que d'autres mais neanmoins participer a
la documentation et donc au wiki est aussi important que le code lui-meme.

-- tickets

le systeme de tickets/bugs est egalement tres important. chaque ticket se
voit affecte une priorite et il est important de comprendre que pour le
bien global du projet, un developpeur ne peut se contenter de faire ce
qui lui plait, il se doit de contribuer egalement a la resolution de problemes.

encore une fois les tickets doivent suivre une norme.

%
% ---------- languages --------------------------------------------------------
%

pour chaque language explique les regles: header, commentaire etc..

-- c

-- make

-- python

-- latex

figures in .fig

-- asm

-- shell
