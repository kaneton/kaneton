%
% ---------- header -----------------------------------------------------------
%
% project       kaneton
%
% license       kaneton
%
% file          /home/mycure/kaneton/view/book/development/development.tex
%
% created       julien quintard   [mon may 14 19:56:45 2007]
% updated       julien quintard   [sun may 20 18:08:13 2007]
%

%
% path
%

\def\path{../..}

%
% template
%

%%
%% licence       kaneton licence
%%
%% project       kaneton
%%
%% file          /home/mycure/kaneton/view/templates/book.tex
%%
%% created       julien quintard   [wed mar  1 23:45:22 2006]
%% updated       julien quintard   [thu may  4 12:36:54 2006]
%%

%
% class
%

\documentclass[10pt,a4wide]{book}

%
% packages
%

\usepackage[english]{babel}
\usepackage[T1]{fontenc}
\usepackage{a4wide}
\usepackage{fancyheadings}
\usepackage{multicol}
\usepackage{indentfirst}
\usepackage{graphicx}
\usepackage{color}
\usepackage{xcolor}
\usepackage{verbatim}

\usepackage{aeguill}

\usepackage[Lenny]{../../../tools/latex/fncychap}

\pagestyle{fancy}

\setlength{\footrulewidth}{0.3pt}
\setlength{\parindent}{0.3cm}
\setlength{\parskip}{2ex plus 0.5ex minus 0.2ex}

%
% logos
%

\newcommand{\logos}
  {
    \begin{center}
      \includegraphics[scale=0.8]{../../logos/kaneton.pdf}
    \end{center}
  }

%
% colors
%

\definecolor{functioncolor}{rgb}{0.40,0.00,0.00}
\definecolor{commandcolor}{rgb}{0.00,0.00,0.40}
\definecolor{verbatimcolor}{rgb}{0.00,0.40,0.00}
\definecolor{noticecolor}{rgb}{0.87,0.84,0.02}

%
% function
%

\newcommand\function[3]{
  \begin{tabular}{p{0.2cm}p{13.8cm}}
  & {\color{functioncolor}\textbf{#1}}#2
  \end{tabular}

  \begin{tabular}{p{1cm}p{13cm}}
  & #3
  \end{tabular}}

%
% align
%

\newcommand\align[1]{
  \\ & \hspace{#1}}

%
% argument
%

\newcommand\argument[1]{\textit{#1}}

%
% command
%

\newcommand\command[2]{
  \begin{tabular}{p{0.2cm}p{13.8cm}}
  & {\color{commandcolor}\textbf{#1}}
  \end{tabular}

  \begin{tabular}{p{1cm}p{13cm}}
  & #2
  \end{tabular}}

%
% notice
%

\newcommand\notice[1]{
  {\color{noticecolor}\textbf{Notice}}

  \begin{tabular}{p{0.2cm}p{13.8cm}}
  & #1
  \end{tabular}}

%
% example
%

\newcommand\example[1]{
  \textit{Example:}

  \begin{tabular}{p{0.2cm}p{13.8cm}}
  & \textit{#1}
  \end{tabular}}

%
% warning XXX
%

%
% verbatim stuff
%

\makeatletter

\renewcommand{\verbatim@font}
  {\ttfamily\footnotesize\color{verbatimcolor}\selectfont}

\def\verbatim@processline{\hskip15ex\the\verbatim@line\par}

\makeatother

%
% header
%

\rhead{}
\rfoot{\scriptsize{The kaneton microkernel project}}

\date{\scriptsize{\today}}


%
% header
%

\lhead{\scriptsize{The kaneton microkernel :: development book}}
\rhead{}

%
% title
%

\title{The kaneton microkernel :: development book
       \logos}

%
% authors
%

\author{\small{Julien Quintard}}

%
% document
%

\begin{document}

%
% title
%

\maketitle

%
% --------- text --------------------------------------------------------------
%

This document describes how people can contribute to the kaneton microkernel
project and the rules they have to follow.

This document must be read by everyone involved in the kaneton microkernel
project.

All the kaneton documents are available on
the official website
  \footnote{http://www.kaneton.org}.

%
% toc
%

\tableofcontents

%
% chapters
%

%
% ---------- header -----------------------------------------------------------
%
% project       kaneton
%
% license       kaneton
%
% file          /home/mycure/kane...book/assignments/future/introduction.tex
%
% created       julien quintard   [fri may 23 21:47:38 2008]
% updated       julien quintard   [fri oct 24 17:05:56 2008]
%

%
% ---------- introduction -----------------------------------------------------
%

\chapter{Introduction}
\label{chapter:introduction}

This chapter briefly introduces the purpose of this documentation
and the assignments in general

\newpage

%
% ---------- text -------------------------------------------------------------
%

The \name{kaneton} educational project enables students to develop their own
micro-kernel as a way of understanding operating systems internals.

As anyone can imagine, such a project takes a huge amount of time and
motivation. While the motivation will anyway play an important role in the
success of students' project, the time spent can be greatly reduced if
students focuse on implementing some specific parts rather than developing a
complete micro-kernel from scratch.

Indeed, as we will see later in this document, the current \name{kaneton}
educational project comes with a student \term{snapshot} which contains
a complete development environment as well as the source code skeleton of
the kernel.

As enthusiastic computer scientists, \name{kaneton} authors, maintainers and
teachers can understand than some people prefer working on their own
micro-kernel design and implementation, from scratch. All we can wish to
such people is enough motivation to keep working on their project long
enough to be satisfied, luck and hard work.

Either way, going through the \name{kaneton} micro-kernel documentation
should be a waste of time. Especially, people interested in developing their
own project from scratch could take a look at the \name{kaneton} design
in case they like it enough to implement it their way.

The remaining of this document is organised as follows. \reference{Chapter
\ref{chapter:requirements}} lists what students willing to undertake the
project should know beforehand. \reference{Chapter \ref{chapter:support}}
details the multiple ways for students to get help. \reference{Chapter
\ref{chapter:k0}} presents the first project stage. Then \reference{Chapter
\ref{chapter:snapshot}} presents the student snapshot while \reference{Chapter
\ref{chapter:setup}} introduces the development environment and its set-up.
Then, \reference{Chapter \ref{chapter:k1}}, \reference{Chapter
\ref{chapter:k2}}, \reference{Chapter \ref{chapter:k3}} and \reference{Chapter
\ref{chapter:k4}} details the assignments of the different stages. Finally,
\reference{Chapter \ref{chapter:further}} discusses what students could do
after having undertaken such a project.

%
% ---------- header -----------------------------------------------------------
%
% project       kaneton
%
% license       kaneton
%
% file          /home/mycure/kaneton/view/book/development/source-tree.tex
%
% created       julien quintard   [thu may 17 22:41:36 2007]
% updated       julien quintard   [thu may 31 08:34:23 2007]
%

%
% ---------- source tree ------------------------------------------------------
%

\chapter{Source Tree}
\label{chapter:source tree}

In this chapter we will briefly describe the kaneton microkernel project
source tree.

\newpage

%
% ---------- text -------------------------------------------------------------
%

The kaneton microkernel reference source tree looks like the following
listing:

\begin{verbatim}
cheat/
configure/
environment/
export/
history/
kaneton/
library/
license/
test/
tool/
transcript/
view/
\end{verbatim}

%
% cheat/
%

\subsection*{cheat/}

Since the kaneton microkernel is implemented by students, the kaneton
people need to check whether students are cheating by re-using parts of
previous years projects or other kernel source codes available on the
\textit{Internet}.

To avoid cheating, kaneton people developed a software checking for
commonalities between different source codes.

This directory contains scripts that performs these verifications. However,
the students work over the years are not stored in this directory but in
the \textit{history/} directory instead.

%
% configure/
%

\subsection*{configure/}

This directory contains everything necessary for configuring its own
kaneton microkernel development environment through the compiling process
to the boot system.

Any new contributor should first look at this directory. However, note that
this directory mainly contains tools targeting final-users rather than
kaneton contributors. Indeed, for instance, the \textit{configure} utility
aims at providing a user-friendly way for configuration but does not take
advantage of the power of the kaneton development environment.

Contributors should then learn about how the development environment works
while final-users should use the \textit{configure} tool.

%
% environment/
%

\subsection*{environment/}

This directory contains everything necessary to the kaneton development
environment.

The kaneton development environment allows different developers to
interact on the development of the same microkernel in a pretty easy way.

The development environment aims at providing developers to possibility to
work in a collaborative manner without interfering with each other. These
developers are likely to run different operating systems on different
microprocessors. In addition, the kaneton microkernel can be targeted for
different microprocessor architectures. The development environment was
introduced to cope with these combinations by providing profiles, each
profile describing the behaviour of a component: underlying operating system,
target architecture, user-specific stuff etc.

As a result, each developer can use a different operating system and
microprocessor architecture with its own specific compiling flags, kaneton
parameters etc. without modifying another developer's configuration.

The development environment is detailed in \textit{Section
\ref{section:environment}}.

%
% export/
%

\subsection*{export/}

The \textit{export/} directory contains scripts used to generate a kaneton
tarball in order to be distributed to the students at the beginning of the
kaneton educational project.

Indeed, these scripts rearrange the kaneton hierarchy hidding some important
directories the students do not need to be aware of. Moreover some source
code parts are removed since the students have to rewrite these pieces
of code as assignments.

These scripts are also used for making backups and distribution tarbalss of
the kaneton microkernel.

%
% history/
%

\subsection*{history/}

The \textit{history/} directory contains the students work over the years
in the universities and schools the kaneton project was used as an operating
system course's implementation material.

The tools of the \textit{cheat/} directory use these students works for
performing cheating verifications.

%
% kaneton/
%

\subsection*{kaneton/}

This directory is the most important of the project since it contains
the whole microkernel source code.

The directory is composed of three important subdirectories: \textit{core/},
\textit{platform/} and \textit{architecture/}. These subdirectories are
described next.

% core/

\subsubsection*{core/}

This directory contains the kaneton core source code.

The directory is divided as shown below:

\begin{verbatim}
as/
region/
sched/
segment/
set/
task/
thread/
[...]
\end{verbatim}

Each directory represents a kaneton core manager. For more information on
the kaneton core, please refer to the appropriate document:
\textit{The kaneton microkernel :: core}

% platform/

\subsubsection*{platform/}

This directory contains everything in relation with what the kaneton
microkernel project calls a \textit{platform}. The platform represents the
board supporting the devices: microprocessor, memory, peripherals etc.

This directory obviously contains subdirectories for each platform
supported by the kaneton microkernel.

% architecture/

\subsubsection*{architecture/}

The \textit{architecture/} directory contains the source-code related to
the microprocessor architectures supported by the kaneton microkernel.

This directory is composed of subdirectories, each one representing a
supported architecture: \textit{ia32}, \textit{mips64} etc. Note that each
architecture can be specialised. For instance, the \textit{ia32/optimised}
architecture represents an optimised implementation of the \textit{Intel IA-32}
microprocessor architecture.

%
% library/
%

\subsection*{library/}

This directory contains the libraries used by the kaneton microkernel itself,
the kaneton microkernel servers or maybe both. This directory especially
contains the standard \textit{kaneton C library}.

%
% license/
%

\subsection*{license/}

This directory contains the licenses used for any program or document
in relation with the kaneton microkernel project. Indeed, the kaneton
microkernel is under the \textit{kaneton license} which is described in
depth in the documents contained in this directory. Note that these licenses
are also available in \textit{Chapter \ref{chapter:licenses}}.

Each student has to read and agree with the kaneton license before
implementing or even using the kaneton microkernel project..

Indeed, every user of the kaneton-related stuff is considered as having
implicitly accepted the kaneton license.

%
% test/
%

\subsection*{test/}

Since the kaneton microkernel is used as a material for operating system
courses, the kaneton microkernel reference, which is the basis of students
work, must be extremely reliable.

The kaneton project therefore contains a set of tools in order to validate
the kaneton reference implementation behaviour. These tools are also used
for evaluating the correctness of the students implementation.

The \textit{test/} directory contains the set of kaneton scripts and tests
for validating a kaneton microkernel implementation.

%
% tool/
%

\subsection*{tool/}

This directory contains additional scripts and configuration files used by
the kaneton development environment or the kaneton developers.

As examples, this directory contains scripts for generating prototypes,
building a boot device etc.

%
% transcript/
%

\subsection*{transcript/}

This directory contains real-time recorded sessions. These sessions can be
replayed in order to present a feature of the development environment or
of the kaneton microkernel.

%
% view/
%

\subsection*{view/}

This directory contains all the kaneton documents including kaneton
administrative documents, examinations, lectures materials, kaneton papers
and books etc.

Additionally, scripts are provided in order to very easily build and
display these documents.
%
% ---------- header -----------------------------------------------------------
%
% project       kaneton
%
% license       kaneton
%
% file          /home/mycure/kaneton/view/book/development/community.tex
%
% created       julien quintard   [sun may 20 18:08:17 2007]
% updated       julien quintard   [fri aug  1 15:51:12 2008]
%

%
% ---------- community --------------------------------------------------------
%

\chapter{Community}
\label{chapter:community}

This chapter discusses what is a community and how contributors must integrate
the kaneton community.

\newpage

%
% ---------- text -------------------------------------------------------------
%

kaneton can obviously be considered as an open source community although the
produced soure code is actually not open source.

Driving an open source community is complicated since people have different
personal goals at working on a free project. Some people contribute for
the knowledge, other for building the next generation system, other to
provide free open source softwares, other to become famous \etc{}

kaneton is a community driven microkernel that acts with the best interest of
the students at heart. Rules and regulations that keep the project moving
forward are fundamental even if the size of the kaneton community is
relatively small, for now.

Indeed, the main objective of the kaneton project remains to be as
understandable as possible in order to lead students to implement parts of
it very quickly.

The remaining of this chapter draws a list of rules contributors must agree
to respect.

% objective

\subsubsection{Objective}

kaneton aims at providing a powerful, understandable and maintainable
microkernel. This objective must be kept in mind of every contributor
since many design and implementation were/are/will be made according to
this precise objective.

Note that the kaneton microkernel does not intend to be a desktop operating
system nor an as optimised as the Linux operating system. Every contributor
should be well-aware of that in order to avoid behaviours stating that a
feature is fundamental or useless for performance concerns, for instance.

This rule does not prohibit people to suggest ideas but instead regulates
behaviours of people who wants to change major design and/or implementation
choices for bad reasons.

% behaviour

\subsubsection{Behaviour}

Open source projects does not mean constraint-free projects. The kaneton
people, whilst being relatively young, try to act for the project's good
by behaving remarkably in the kaneton community.

Therefore, contributors are asked to do the same by avoiding some bad/young
behaviours.

\begin{enumerate}
  \item
    Follow the rules. People who do not respect these rules could be banned
    from the kaneton project.
  \item
    Avoid the \textit{cowboy} behaviour consisting for a contributor to
    implement a feature without discussing about its usefulness with the
    community first. Another effect of this behaviour can be to distract
    the contributors from its major focuses.
  \item
    Always act and think in the project interest rather than your personal
    interest.
  \item
    Respect the other kaneton people, especially the ones who have worked
    on this project for a long time and who made this whole project possible.
    When people disagree, they are asked to do it respectfully.
  \item
    Take your responsability when you realise that you did something wrong:
    insults, mistakes in an implemented feature \etc{}
  \item
    \textit{``The Perfect is the Enemy of the Good!''}: even nice
    contributors can unintentionally do bad things by being perfectionists
    and/or too much into the project and/or obsessed with process.
  \item
    ... \textit{Politness}, \textit{Respect}, \textit{Trust} and
    \textit{Humility} are the key qualities that make a good contributor in
    any community.
\end{enumerate}

% communication

\subsubsection{Communication}

The communication mainly takes two forms in the kaneton microkernel project:
the \name{mailing-list} for internal communication and the \name{kaneton
website} for external communication. The \name{Developers Intranet} is another
source of communication as well as the commit logs \etc{}

The rules related to these tools are described in \reference{Chapter
\ref{chapter:tools}} and will therefore not be discussed here.

Every contributor must take the time to communicate in the mailing-list as well
as through the public website. Indeed, kaneton people must, frequently,
briefly describe what they are working on in order to inform the other
contributors who are not aware of everyone's current work. Note that these
kind of messages are very different from messages generated by repository
commits. Indeed, while these commit messages indicate a modification, they
do not describe the whole work behind them.

Additionally, contributors can communicate informing the kaneton community of
their unavailability for the next two months, for instance. This behaviour
allows people to be aware that some tasks will not be done because the
contributors in charge of it cannot work at this moment.

Althoug people are highly welcomed to communicate, some rules apply to
avoid further problems.

First, any new contributor should obviously read the kaneton documents before
asking anything which has already been discussed and decided, unless the
developer knows exactly what he/she is talking about. Indeed, asking too many
questions about the source code is a form of disrespect to the other
contributors. Moreover, many things can be found out just looking at the
kaneton documents and/or source code.

Although people are asked to communicate, people are also asked to act
respectully. Contributors should not respond to every message in every
discussion, this is a ridiculous behaviour. Instead, every developer should
carefully read the discussion, think about its response and then write a clear
message stating his point of view, ideas \etc{}

Depending on the contributor status, reading the mailing-list frequently is
absolutely fundamental as some people rely on other contributors' decisions,
advices \etc{}

Finally, the mailing-list must be considered as the official internal
communication medium. If contributors previously had a private conversation,
in real-life or on \name{IRC} for instance, the discussion must be reported
on the mailing-list so that everyone can take these new ideas into account.

% work

\subsubsection{Work}

Working on the kaneton microkernel project does not imply low-level programming
all the time. Indeed, the kaneton project is composed of two parts: the
kaneton microkernel research project and the kaneton educational project.

Although the microkernel research project requires highly skilled programmers,
it also needs documentations and some tools for performing important tasks as
diverse as generating the prototypes, testing the microkernel behaviour,
generating the documentation \etc{}

The educational project essentially needs documentation, lecture materials
and tools for managing the project: testing the students' implementation,
checking if some students cheated and many others.

This means that kaneton people must contribute to every type of task
that need to be done. Also, contributors are asked to well document
any work they have done including source code comments but also through
kaneton official documents which are then made available on the website.

% supervisor

\subsubsection{Supervisor}

A supervisor is attached to each new contributor for a certain, not fixed,
period of time. The role of this supervisor is to advise, correct and
encourage the newcomer so that it integrates well the kaneton community.

The supervisor will be someone having a well understanding of the new
contributor's project. A high level of communication is expected between
the supervisor and the contributor. A direct phone communication is highly
recommanded through \name{Skype} for instance.

% trust

\subsubsection{Trust}

A new contributor joining the kaneton project must acquire the trust of
the community. Therefore, the contributor first does not get any access
to restricted tools and must submit patches to its supervisor who review
them, taking care of advising the newcomer of its mistakes.

At the end of the test period, the community decides whether the contributor
is helpful to the project or not. Then, the contributor is either granted
of full access to kaneton tools or eliminated from the project.

%
% ---------- header -----------------------------------------------------------
%
% project       kaneton
%
% license       kaneton
%
% file          /home/mycure/kaneton/view/book/development/tools.tex
%
% created       julien quintard   [sun may 20 14:48:11 2007]
% updated       julien quintard   [thu may 31 06:46:06 2007]
%

%
% ---------- tools ------------------------------------------------------------
%

\chapter{Tools}
\label{chapter:tools}

This chapter describes every tool kaneton contributors use on a daily-basis.

\newpage

%
% ---------- text -------------------------------------------------------------
%

%
% internal
%

\section{Internal}

The kaneton project contains several tools which makes the developer's life
easier. This section describes these tools in order for the contributor to
use it but also to improve them.

% environment

%
% ---------- header -----------------------------------------------------------
%
% project       kaneton
%
% license       kaneton
%
% file          /home/mycure/kaneton/view/book/development/environment.tex
%
% created       julien quintard   [sun may 20 14:49:26 2007]
% updated       julien quintard   [sun may 20 18:07:59 2007]
%

%
% ---------- environment ------------------------------------------------------
%

\subsection{Environment}

Over the years, the kaneton microkernel evolved, starting with a very simple
introduction to low-level programming and finally to a complete microkernel
development.

kaneton people wanted to lead students to a complete microkernel development
to finally introduce distributed computing. This would not have been possible
if students had to build an entire development environment because developing
such an environment is a whole project by itself.

As a result, kaneton people decided to provide students a complete development
environment. The kaneton development environment is composed of make files,
python scripts and configuration files. This development environment can be
considered as one of the major kaneton tools since contributors use it
everytime.

The kaneton development environment aims at providing an easy and portable
way for managing the kaneton microkernel project from a development point
of view. Therefore, the kaneton environment provides everything necessary
for compiling, assembling, etc.. These tasks highly rely on the underlying
running operating system as well as on the kaneton microkernel's target
microprocessor. Moreover, the user could need to redefine some behaviours
depending on its personal operating system configuration to use a specific
C compiler for instance.

The kaneton development environment provides a layered organisation of
profiles, each profile defining variables and functions used by the final
environment engine. The goal of the layered model is to allow layers to
override the definitions of lower layers.

%
% profiles
%

\subsubsection{Profiles}

A configuration is composed of profiles including a \textit{host} profile which
describes the behaviour of the underlying operating system, a \textit{kaneton}
profile which parameters the kaneton core and a \textit{user} profile which
permits the user to redefine lower layers' declarations.

These profiles eventually hold sub-profiles which actually define variables
and functions. These actual profiles are accessed according to user-defined
shell variables.

% host

\subsubsubsection{Host}

The \textit{host} profile essentially describes how to perform basic tasks:
compile, assemble, change the current directory, display a message etc.. These
tasks rely on the operating system currently running as well as on the target
processor which kaneton will be built for. For these reasons, there are
several host sub-profiles.

Let us suppose a developer is running a \textit{Linux} operating system and
that kaneton will be built for running on a \textit{PowerPC} microprocessor. In
such a case, the C compiler program will be different depending on the
microprocessor \textit{Linux} is running on. Indeed, if Linux is running on
a \textit{PowerPC} microprocessor, then using the default compiler should
produce \textit{PowerPC} object files. This is well-known to be the common
compiling way. On the other hand, if \textit{Linux} is running on a
different microprocessor, then a cross-compiler must be used to produce
binary objects targeting a specific different microprocessor architecture.

To avoid this issue, a \textit{host} sub-profile name is composed of two parts
separated by a slash. The first part is the name of the operating system and
the latter is a pair source/target processors separated by a period. For
example, \textit{linux/ia32.ppc} names a \textit{host} profile running the
\textit{Linux} operating system on a \textit{Intel 32-bit} microprocessor
which aims at building a kaneton microkernel for the \textit{PowerPC}
target architecture. Needless to say that \textit{linux/ia32.ia32} represents
a non cross-compiling environment.

To avoid configuration duplications, it is common to see the configuration
file of a host sub-profile to include files of the parent directory as
shown below:

\begin{verbatim}
  linux/
    linux.desc
    linux.conf
    linux.mk
    linux.py
    ia32.ia32/
      virtual -> .
      optimised -> .
      smp -> .
      ia32.desc
      ia32.conf
      ia32.mk
      ia32.py
    ia32.mips64/
      mips64.desc
      mips64.conf
      mips64.mk
      mips64.py
\end{verbatim}

Note that the files \textit{linux.*} are not directly included by the
development environment engine since \textit{linux} is not a valid host
profile name.

Two host profiles are illustrated here. The first one is named
\textit{linux/ia32.ia32} while the second's name is \textit{linux/ia32.mips64}.

For example, the \textit{linux/ia32.mips64} \textit{host} profile represents a
\textit{Linux} operating system running on a \textit{Intel 32-bit}
microprocessor while kaneton is built for a \textit{MIPS 64-bit} target
architecture. This profile is likely to include the \textit{linux.*} of the
parent directory since there are not much difference between all the
\textit{linux/*.*} \textit{host} profiles. However, such a profile will
certainly redefine the binary paths of the C compiler, linker etc.. in order
to produce \textit{MIPS 64-bit} binary objects.

To conclude, the \textit{host} sub-profile is accessed by the following
construct:

\begin{verbatim}
  profile/host/${KANETON_HOST}/${KANETON_ARCHITECTURE}
\end{verbatim}

With, for instance, the following values:

\begin{verbatim}
  KANETON_HOST = linux/ia32
  KANETON_ARCHTECTURE = ia32/virtual
\end{verbatim}

Note that the possibility to include files in the configuration syntax allows
very similar profiles to share a huge amount of definitions.

% kaneton

\subsubsubsection{kaneton}

The \textit{kaneton} profile is composed of three sub-profiles: \textit{core},
\textit{platform} and \textit{architecture}.

The \textit{core} sub-profile contains variables for parameterizing the
kaneton core. The \textit{platform} and \textit{architecture} sub-profiles
focus on the configuration of the platform- and architecture-dependent code
of the kaneton microkernel.

The user-defined shell variables \textit{\$\{KANETON\_PLATFORM\}} and
\textit{\$\{KANETON\_ARCHITECTURE\}} are used to address the \textit{platform}
and \textit{architecture} sub-profiles, respectively.

% user

\subsubsubsection{User}

Let us suppose that a developer would like the kaneton microkernel to
use a specific memory management entirely based on a \textit{Slab Allocator}
and with all microprocessor optimisations enabled. These user-specific
configurations are actually allowed by the \textit{user} profile.

The user-defined shell variable \textit{\$\{KANETON\_USER\}} defines the name
of the \textit{user} profile. This profile contains user-specific
configurations allowing a contributor to overwrite lower layer defintions
in order to specialise a behaviour.

The kaneton project also provides a tool allowing developers to configure
their development environment. This tool is named \textbf{configure} and is
available from the kaneton project root directory.

%
% requirements
%

\subsubsection{Requirements}

The whole kaneton development environment needs exactly two fundamental tools
to work. The first one is \textit{GNU make}, used to build powerful make files,
and the second one is \textit{Python}, used to write portable scripts. If an
operating system has these two tools, then kaneton can certainly be developed
on it.

As said previously, the user has to specify some shell variables which are
used by the kaneton development environment engine. These variables are
described below:

\begin{itemize}
  \item
    \textbf{\$\{KANETON\_USER\}}: the name of the kaneton developer.

    A \textit{user} profile name must be composed of the first name, a period
    and finally, the last name of the developer.
  \item
    \textbf{\$\{KANETON\_HOST\}}: the name of the host which is composed of
    a couple operating system/microprocessor.
  \item
    \textbf{\$\{KANETON\_PYTHON\}} contains the path of the python binary.

    This path is necessary since the very first scripts which set up the
    configured environment are based on python scripts.
  \item
    \textbf{\$\{KANETON\_PLATFORM\}}: the name of the target platform.
  \item
    \textbf{\$\{KANETON\_ARCHITECTURE\}}: the name of the target architecture.
\end{itemize}

Note that once the configured environment is set up, these variables are
no longer used by the kaneton environment engine. Indeed, instead, the
kaneton environment operations are based on the \textit{host} profile on
which rely the configured environment.

The profiles names must all be lowercase. Below are some examples of what
could contain these variables:

\begin{verbatim}
  KANETON_USER='julien.quintard'

  KANETON_HOST='linux/ppc'
  KANETON_HOST='windows~cygwin/ia32'

  KANETON_PYTHON='/usr/bin/python'

  KANETON_PLATFORM='ibm-pc'
  KANETON_PLATFORM='sgi/o2'
  KANETON_PLATFORM='sgi/octane'

  KANETON_ARCHITECTURE='mips64'
  KANETON_ARCHITECTURE='ia32/virtual'
  KANETON_ARCHITECTURE='ia32/smp'
\end{verbatim}

%
% organisation
%

\subsubsection{Organisation}

The development environment configuration files and scripts are located in
the \textit{environment/} directory. The directory contains the three
following scripts:

\begin{verbatim}
  critical.py
  init.py
  clean.py
\end{verbatim}

The \textit{critical.py} script essentially generates a configured development
environment. The result of this generation are two files called
\textit{env.mk} and \textit{env.py} which contains the configured environment
variables and functions for the \textit{Make} files and \textit{Python}
scripts, respectively. This file is called critical because it does not rely
on the portable development environment as it generates it.

The \textit{init.py} script relies on the file \textit{env.py} previously
generated. This script set up everything necessary for building the
kaneton microkernel based on the configured environment.

Finally, the \textit{clean.py} script cleans everything installed by the
\textit{init.py} script and removes the generated configured environment files.

The generation of the configured environment is done by going through
the configuration files of all the profiles and sub-profiles associated
to the user configuration. In other words, the kaneton environment engine
processes the configuration files according to the layered organisation
described below, starting with the lowest layer thourgh the highest one.

\begin{verbatim}
  profile/
  profile/host
  profile/host/${KANETON_HOST}/${KANETON_ARCHITECTURE}
  profile/kaneton
  profile/kaneton/core
  profile/kaneton/platform
  profile/kaneton/platform/${KANETON_PLATFORM}
  profile/kaneton/architecture
  profile/kaneton/architecture/${KANETON_ARCHITECTURE}
  profile/user
  profile/user/${KANETON_USER}         
\end{verbatim}

\begin{verbatim}
XXX $ XXX
\end{verbatim}

In this layered organisation, a variable defined in, for instance, the
\textit{host} profile could be overwritten anywhere in the upper layers
\textit{kaneton}, \textit{kaneton/architecture/\$\{KANETON\_ARCHITECTURE\}},
\textit{user} etc..

The \textit{host} and \textit{kaneton} profiles are theoretically completed
separated. However, the environment engine does not check for such
unauthorised overridings. Therefore the \textit{core} configuration could
override a variable previously defined in the \textit{host} profile.

Finally, the \textit{user} profile can override any definition adjusting the
environment to his needs.

The environment engine looks for the following types of files in the
kaneton environment profile directories:

\begin{itemize}
  \item
    \textbf{.conf}: the \textit{configuration} files gathered by the
    development environment engine for generating the configured environment
    files.
  \item
    \textbf{.desc}: these \textit{description} files contain descriptions of
    the variables of the current profile or sub-profile. These descriptions
    are used by the \textit{configure} tool.
  \item
    \textbf{.mk}: the \textit{Make} files usually contains the implementation
    of the kaneton \textit{Make} interface.
  \item
    \textbf{.py}: the \textit{Python} files usually contains the
    implementation of the kaneton \textit{Python} interface.
\end{itemize}

The engine supposes that there is no variable or function overriding in
a single profile. more precisely, if there are more than a single
configuration file in a directory, the engine cannot guarantee anything
on the order these files will be processed. As a result, the overridings
could differ depending on the processing order.

The kaneton development environment engine first gathers the
\textit{configuration} files and process them creating an in-memory list of
configuration variables. Then it generates the configured environment files
\textit{env.mk} and \textit{env.py}. Indeed, the engine outputs the
configuration variables in each file and then append the content of the
\textit{Make} files and \textit{Python} files to the configured environment
file \textit{env.mk} and \textit{env.py}, respectively.

Note that the \textit{description} files are not directly used by the
environment engine.

%
% syntaxes
%

\subsubsection{Syntaxes}

% description

\subsubsubsection{Description}

The \textit{description} files describe the environment variables in order
to specify what kind of value a variable can take etc..

Each variable description is contained between braces. A description is
composed of fields, some are mandatory and some are optional.

Examples of description for variables named \textit{\_FOO\_}, \textit{\_BAR\_}
and \textit{\_CHICHE\_} are given next:

\begin{verbatim}
  _FOO_ :: the foo flag
  {
    <on> -D_FOO_FLAG_=1
    <off> -D_FOO_FLAG_=0

    This is a description of the two-state variable _FOO_.
  }

  _BAR_ :: the bar parameter
  {
    <simple> BAR_SIMPLE
    <normal> BAR_NORMAL
    <optimised> BAR_OPTIMISED

    This is another parameter which can take three values: simple,
    normal and optimised.
  }

  _CHICHE_ :: the most powerful optimisation
  {
    This is the magic kaneton optimisation.
  }
\end{verbatim}

Note that environment engine never takes these descriptions into account.
Indeed, this is the r\^ole of the \textit{configure} tool.

In this syntax, variables are classified according to the type of value
they can take: \textit{state}, \textit{set} and \textit{any}.

A \textit{state} variable is either activated or deactivated. If the two
fields \textit{<on>} and \textit{<off>} are present, then, this variable
is considered as a \textit{state} variable.

A \textit{set} variable can take any value of a given list of values. This
is the most common type of variables. In this case, each field detected
is considered as a potential value.

Finally, a \textit{any} variable represents a variable which can take any
value. This case is detected by the absence of value field in the description.

The value fields follow the next pattern:

\begin{verbatim}
  <name> value
\end{verbatim}

The \textit{name} is displayed by the \textit{configure} tool to the final
user while the \textit{value} value is affected to the described variable. This
way, the tool can displayed more human-readable description. For instance,
if the \textit{optimised} option is chosen, then the \textit{BAR\_OPTIMISED}
will be affected to the \textit{\_BAR\_} variable.

The name of the variable follows the pattern:

\begin{verbatim}
  variable :: name
\end{verbatim}

Once again, this construct was introduced to avoid displaying internal
non-user-friendly variable names. The \textit{variable} will not be directly
displayed by the \textit{configure} tool which will use the \textit{name}
string instead.

Finally, any remaining text between the braces is considered as a variable's
description text.

% configuration

\subsubsubsection{Configuration}

The \textit{configuration} files contains the actual variable definitions
through a very simple syntax.

The syntax allows both assignments and completion of variables' value
as show in the next example:

\begin{verbatim}
  FOO = bar
  FOO += baz
  FOO = kaneton
\end{verbatim}

The \textit{FOO} variable first took the initial value \textit{bar}. Then,
the value \textit{baz} was added to the previous \textit{FOO}'s value
leading the the value \textit{bar baz}. Finally, the last assignment
overwrite the previous definitions by setting the value of \textit{FOO}'s
variable to \textit{kaneton}.

The configuration syntax enables the use of variables in values. These
variables can be both environment variable or shell variable. The following
example illustrates it.

\begin{verbatim}
  BAR = ${FOO} is a very powerful microkernel
  SH = the shell currently used is $(SHELL)
\end{verbatim}

The reader certainly notice the \textit{\$\{\}} construct is used to reference
a kaneton environment variable while the \textit{\$()} one references a shell
variable.

Finally, a configuration file can also include another file using the
\textit{include} statement:

\begin{verbatim}
  include ../an/other/file/far/../far/../away
\end{verbatim}

This construct is very useful to centralize the definitions common to
multiple sub-profiles in a single location.

Note that kaneton environment variables start and end with an underscore
for avoiding naming collisions.

% make

\subsubsubsection{Make}

The \textit{Make} files must implement the whole kaneton \textit{Make}
interface which will be described next.

The syntax used in these files is based on the \textit{GNU Make} syntax.

% python

\subsubsubsection{Python}

The \textit{Python} files must implement the whole kaneton \textit{Python}
interface.

The syntax used in these files is based on the \textit{Python} syntax.

%
% interfaces
%

\subsubsection{Interfaces}

% make

\subsubsubsection{Make}

In this section we will detail the make interface that every host profile
must implement. The reader should look closer to the host profiles already
implemented.

Since the \textit{GNU Make} syntax does not provide any name space feature,
every kaneton \textit{Make} function is prefixed by \textit{env\_} in order
to avoid name conflicts.

\function{env\_display}{(\argument{color},
                         \argument{action},
                         \argument{file},
                         \argument{indentation},
                         \argument{options})}
         {
	   This function display a message representing an action performed
	   by the kaneton \textit{Make} interface.

	   \-

	   The option \textit{\$(OPTION\_NO\_NEWLINE)} can be used not to
	   output the trailing newline.
	 }

\function{env\_cd}{(\argument{directory},
                    \argument{options})}
         {
	   This function changes the current working directory.
	 }

\function{env\_contents}{(\argument{file},
                          \argument{options})}
         {
	   This function returns the contents of the file \argument{file}.
	 }

\function{env\_launch}{(\argument{file},
                        \argument{arguments},
                        \argument{options})}
         {
	   This function launches a new program/script/make etc..

	   \-

	   This function must look at the file name in order to determine
	   how to launch it.

	   \-

	   For \textit{Python} files, this function must take care of
	   setting and exporting the \textit{PYTHONPATH} shell environment
	   variable with a value including the
	   \textit{\_PYTHON\_INCLUDE\_DIR\_} kaneton environment variable.
	 }

\function{env\_preprocess}{(\argument{preprocessed file},
                            \argument{c file},
                            \argument{options})}
         {
	   This function launches the C preprocessor the \argument{c file}
	   and generates the \argument{preprocessed file}.
	 }

\function{env\_compile-c}{(\argument{object file},
                           \argument{c file},
                           \argument{options})}
         {
	   This function compile a \argument{c file} generating an
	   \argument{object file}.
	 }

\function{env\_lex-l}{(\argument{c file},
                       \argument{lex file},
                       \argument{options})}
         {
	   This function generates a \argument{c file} from a
	   \argument{lex file}.
	 }

\function{env\_assemble-S}{(\argument{object file},
                            \argument{S file},
                            \argument{options})}
         {
	   This function assemble an \argument{S file}.
	 }

\function{env\_assemble-asm}{(\argument{object file},
                              \argument{asm file},
                              \argument{options})}
         {
	   This function assemble an asm file.

	   \-

	   The option \textit{\$(ENV\_OUTPUT\_OBJECT)} forces the function
	   to generate an object file while the
	   \textit{\$(ENV\_OUTPUT\_BINARY)} option forces the output to be
	   a pure binary file.
	 }

\function{env\_static-library}{(\argument{static library file name},
                                \argument{object files and/or libraries},
                                \argument{options})}
         {
	   This function builds a static library from object files.
	 }

\function{env\_dynamic-library}{(\argument{dynamic library file name},
                                 \argument{object files and/or libraries},
                                 \argument{options})}
         {
	   This function builds a dynamic library from object files and/or
	   libraries.
	 }

\function{env\_executable}{(\argument{executable file name},
                            \argument{object files and/or libraries},
                            \argument{layout file},
                            \argument{options})}
         {
	   This function builds a executable from object files and/or
	   libraries. The \argument{layout file} describes where to
	   place the different data: code, read-only data, stack etc..

	   \-

	   The option \textit{\$(ENV\_OPTION\_NO\_STANDARD)} tells the function
	   not to use the operating system standard stuff: libraries, includes
	   etc..
	 }

\function{env\_archive}{(\argument{archive file name},
                         \argument{object files},
                         \argument{options})}
         {
	   This function builds an archive of object from multiple
	   \argument{object files}.
	 }

\function{env\_remove}{(\argument{files},
                        \argument{options})}
         {
	   This function removes the files in the list.
	 }

\function{env\_purge}{()}
         {
	   This function just cleans the current working directory from
	   unecessary files.
	 }

\function{env\_prototypes}{(\argument{files},
                            \argument{options})}
         {
	   This function generates prototypes in relation with the given
	   \argument{files}.
	 }

\function{env\_dependencies}{(\argument{files},
                              \argument{output},
                              \argument{options})}
         {
	   This function generates dependencies for the \argument{files}
	   by building a \textit{Make} dependency file named \argument{output}.
	 }

\function{env\_version}{(\argument{file})}
         {
	   This function generates a version \argument{file} from the operating
	   system's informations: user, host, date etc..
	 }

\function{env\_link}{(\argument{link},
                      \argument{file},
                      \argument{options})}
         {
	   This function creates a link \argument{link} to the \argument{file}.
	 }

\function{env\_compile-tex}{(\argument{file},
                             \argument{options})}
         {
	   This function compiles the file \argument{file}.tex and
	   will generate a readable document.
	 }

\function{env\_paper}{(\argument{file},
                       \argument{options})}
         {
	   This function builds a \textit{paper} by calling the
	   \textbf{env\_compile-tex()} function.
	 }

\function{env\_lecture}{(\argument{file},
                         \argument{options})}
         {
	   This function builds a \textit{lecture} document by calling the
	   \textbf{env\_compile-tex()} function.
	 }

\function{env\_subject}{(\argument{file},
                         \argument{options})}
         {
	   This function builds a \textit{subject} by calling the
	   \textbf{env\_compile-tex()} function.
	 }

\function{env\_correction}{(\argument{file},
                            \argument{options})}
         {
	   This function builds a \textit{correction} document by calling the
	   \textbf{env\_compile-tex()} function.
	 }

\function{env\_view}{(\argument{file},
                      \argument{options})}
         {
	   This function launches a viewer for the readable document
	   generated by the function \textbf{env\_compile-tex()}.
	 }

% python

\subsubsubsection{Python}

In this section we will detail the kaneton \textit{Python} interface that
every \textit{host} profile must implement.

The \textit{Python} language was designed in a portable way. For this
reason, the major part of the \textit{Python} interface is implemented
by the \textit{host} generic profile.

Note that the \textit{Python} language provides modularity through packages.
Therefore, each \textit{Python} script has to import the \textit{env} package
generated by the development environment engine. Then, environment functions
and variables are accessed through this package.

Below are described the functions implemented by the \textit{env} package.

\function{display}{(\argument{header},
                    \argument{text},
                    \argument{options})}
         {
	   This function outputs some text to the screen depending on the
	   header \textit{HEADER\_NONE}, \textit{HEADER\_OK},
	   \textit{HEADER\_ERROR}, \textit{HEADER\_INTERACTIVE}.
	 }

\function{contents}{(\argument{file},
                     \argument{options})}
         {
	   This function returns the contents of the \argument{file}.
	 }

\function{temporary}{(\argument{options})}
         {
	   This function creates a temporary file system object.

	   \-

	   The options \textit{OPTION\_FILE} and \textit{OPTION\_DIRECTORY}
	   specify which type of object to create.
	 }

\function{cwd}{(\argument{options})}
         {
	   This function returns the path of the current working directory.
	 }

\function{input}{(\argument{options})}
         {
	   This function waits for an input.
	 }

\function{launch}{(\argument{file},
                   \argument{arguments},
                   \argument{options})}
         {
	   This function launches a new program/script/make file etc..

	   \-

	   This function must look at the file name in order to determine
	   how to launch it.

	   \-

	   For \textit{Python} files, this function must take care of
	   setting and exporting the \textit{PYTHONPATH} shell environment
	   variable with a value including the
	   \textit{\_PYTHON\_INCLUDE\_DIR\_} kaneton environment variable.
	 }

\function{copy}{(\argument{source},
                 \argument{destination},
                 \argument{options})}
         {
	   This function copies the file \argument{source} to
	   \argument{destination}.
	 }

\function{link}{(\argument{source},
                 \argument{destination},
                 \argument{options})}
         {
	   This function builds a link between the file \argument{source}
	   and the file \argument{destination}.
	 }

\function{remove}{(\argument{target},
                   \argument{options})}
         {
	   This function removes the \argument{target} which can be either
	   a file or a directory.
	 }

\function{list}{(\argument{directory},
                 \argument{options})}
         {
	   This function lists the file system objects contains in the
	   \argument{directory}.

	   \-

	   The options \textit{OPTION\_FILE} and \textit{OPTION\_DIRECTORY}
	   specify which type of object to list.
	 }

\function{cd}{(\argument{directory},
               \argument{options})}
         {
	   This function changes the current working directory to
	   \argument{directory}.
	 }

\function{search}{(\argument{directory},
                   \argument{pattern},
                   \argument{options})}
         {
	   This function looks for files matching the given \argument{pattern}.

	   \-

	   The options \textit{OPTION\_FILE} and \textit{OPTION\_DIRECTORY}
	   specify which type of object to list while the
	   \textit{OPTION\_RECURSIVE} option tells the function to explore
	   the whole file system sub-tree.
	 }

\function{pack}{(\argument{directory},
                 \argument{file},
                 \argument{options})}
         {
	   This function makes an archive \argument{file} of the
	   directory \argument{directory}.
	 }

\function{unpack}{(\argument{directory},
                   \argument{file},
                   \argument{options})}
         {
	   This function extracts the archive \argument{file} into the
	   directory \argument{directory}, if specified.
	 }

\function{mkdir}{(\argument{directory},
                  \argument{options})}
         {
	   This function builds a new directory named \argument{directory}.
	 }

\function{load}{(\argument{file},
                 \argument{device},
                 \argument{path},
                 \argument{options})}
         {
	   This function copies the \argument{file} on the specificed
	   \argument{device}, more precisly at the location \argument{path}.
	   The device can be virtual: an image.

	   \-

	   The options \textit{OPTION\_DEVICE} and \textit{OPTION\_IMAGE}
	   specify on which type of device the file must be copied.
	 }

\function{stamp}{(\argument{format},
                  \argument{options})}
         {
	   This function returns a date following the given \argument{format}.
	 }

\function{record}{(\argument{log},
                   \argument{time},
                   \argument{options})}
         {
	   This function starts recording a session and outputs
	   the text into the file \argument{log} while the timings
	   are output in the file \argument{time}.
	 }

\function{play}{(\argument{log},
                 \argument{time},
                 \argument{options})}
         {
	   This function plays a previously recorded session where
	   the files \argument{log} and \argument{time} hold the
	   text and timings.
	 }

\function{locate}{(\argument{file},
                   \argument{options})}
         {
	   This function tries to locate the program \argument{file}
	   on the system.
	 }

\function{path}{(\argument{path},
                 \argument{options})}
         {
	   This function returns information on the given \argument{path}.

	   \-

	   The options \textit{OPTION\_FILE} and \textit{OPTION\_DIRECTORY}
	   specify which information the caller is interested in.
	 }


% configure

%
% ---------- header -----------------------------------------------------------
%
% project       kaneton
%
% license       kaneton
%
% file          /home/mycure/kaneton/view/book/development/configure.tex
%
% created       julien quintard   [tue may 22 22:34:37 2007]
% updated       julien quintard   [wed may 30 19:26:54 2007]
%

%
% ---------- configure --------------------------------------------------------
%

\subsection{Configure}

The \textit{configure} tool provides the final user a very user-friendly
software for customizing its development environment.

Recall the development environment is basically composed of three profiles:
\textit{host} which describes the operating system behaviour, \textit{kaneton}
which parameterizes the kaneton microkernel and \textit{user} which contains
some user-specific definitions.

The kaneton development environment is thus used to configure the environment
behaviour as well as the kaneton microkernel itself.

The \textit{environment/} directory, and more precisely the environment
profiles, contain \textit{description} files which actually describe the
environment variables. These files are not used by the development
environment but rather by the \textit{configure} tool.

The \textit{configure/} directory is composed of \textit{frame} files
which contain frame descriptions. A frame can be seen as a menu presented
to the final user. A frame is composed of sub-frame and variable entries.

The \textit{configure} tool works as follow. It starts by processing the
environment development configuration files as the environment engine did
for the generation of the configured environment files. Note that the
\textit{configure} tool also processes the description files. Also, it
focuses on variables and actually ignores the interfaces' functions.

Once this step is done, the tool gets a list of configured and fully described
variables. Then, the \textit{configure} tool displays the first frame and
waits for the user to choose an entry.

The user has the possibility to either move to another menu - if any sub-frame
entry is present - or configure a variable of the list. If the user chooses
to configure a variable, then, the tool displays information based on the
variable's description.

Every modifications of the development environment are private to the actual
user. Therefore, any variable modification adds or modifies an entry in the
related \textit{user} profile.

Note that the \textit{configure} tool is not a environment configuration
files editor. Indeed, this tool targets final users and therefore has to
be as simple as possible.

The basic \textit{configure} behaviour consists in displaying the final
variable's value. If the user enters a new value, no matter whether there is
a relation with its previous value, the tool creates/modifies an entry in the
\textit{user} profile's configuration file overriding any previous definition.

For instance, consider the \textit{\_FOO\_} development environment variable
with the following configuration definition:

In \textit{profile/environment.conf}:

\begin{verbatim}
  _FOO_                         =                       initial
\end{verbatim}

In \textit{profile/core/core.conf}:

\begin{verbatim}
  _FOO_                         +=                      addon
\end{verbatim}

Let us suppose the user enters the following value instead of the current
one: \verb|initial addon|

\begin{verbatim}
  _FOO_                         =                       initial new
\end{verbatim}

Then, the \textit{configure} tool creates a new entry into the \textit{user}
profile configuration file:

\begin{verbatim}
  _FOO_                         =                       initial new
\end{verbatim}

Finally, note that when the \textit{configure} tool is launched, it first
tries to detect whether the user is a newcomer or not. If it is, then the
tool asks the user to create a new \textit{user} profile, step by step. These
actions are performed in the \textit{critical.py} script of the
\textit{configure/} directory.

% requirements

\subsubsubsection{Requirements}

The \textit{configure} tool relies on the \textit{Dialog} software which
is present on many \textit{Unix} systems. Indeed, the \textit{configure}
tool is a user-friendly configuration utility.

% syntax

\subsubsubsection{Syntax}

The syntax of the frame description files is based on \textit{YAML}. Therefore,
the \textit{Python} \textit{PyYAML} module needs to be set up.

As said previously, a frame is composed of sub-frame and variable entries. A
sub-frame entry contains a name and a path to the sub-frame description file
while a variable entry only contains the name of the variable. This variable
name is then used to retrieve the variable description.

The example below illustrates this very simple syntax:

\begin{verbatim}
[XXX]
  - title: Segment Manager
    description: |
      This frame contains configuration about the core
      segment manager

  - frame: optimisations
    path: subsections/optimisations.desc

  - frame: machine dependent
    path: subsections/machine.desc

  - variable: _FOO_

  - variable: _BAR_

  - variable: _CHICHE_
\end{verbatim}


% view

%
% ---------- header -----------------------------------------------------------
%
% project       kaneton
%
% license       kaneton
%
% file          /home/mycure/kaneton/view/book/development/view.tex
%
% created       julien quintard   [wed may 23 00:36:53 2007]
% updated       julien quintard   [mon may  4 19:43:15 2009]
%

%
% ---------- view -------------------------------------------------------------
%

\subsection{View}
\label{section:view}

The \name{view} tool serves as a document database as well as a tool for
building and displaying documents in an easy way.

The kaneton documents are classified, each directory corresponding to a
class of documents. Below are listed the subdirectories of the \location{view/}
directory.

\begin{verbatim}
  bibliography/
  book/
  exam/
  feedback/
  figures/
  internship/
  lecture/
  logo/
  package/
  paper/
  talk/
  template/
\end{verbatim}

The \location{template/} directory contains templates for every class of
document. The \location{bibliography/} and \location{logo/} directories
contain, obviously, the bibliography which is common to all the documents, and
the logos, respectively. The \location{figures/} directory contains figures
common to all the documents while the \location{package/} directory contains
additional {\LaTeX} packages.

The directories \location{curriculum/}, \location{exam/} and
\location{feedback/} contain documents in relation with teaching. The
\location{curriculum/} directory contains documents such as the educational
project year planning \etc{} The \location{feedback/} directory contains
documents which are distributed to the students at the end of the kaneton
project in order to get feedback for improving the project for the next years.
Needless to say the \location{exam/} directory contains everything related to
examinations while the \location{talk/} directory contains conference talks
and various presentations of the education project for instance.

The other directories contain the actual kaneton documentation. The
\name{books} represent the main documents targeting any public:
contributors, teachers, students \etc{} The \name{papers} are lighter
documents intended to present a specific feature, design \etc{} The
\name{lectures} are the courses materials, generally composed of
presentation slides. Finally, the \name{internship} documentation is
composed of documents written by people partially involved in the kaneton
project.

Any document is composed of a \name{Make} file and one or more
{\LaTeX} files. The \name{Make} file always has the same form
with little variations depending on the type of document. For more information
on the rules applying to the \name{Make} and {\LaTeX} files, please
refer to their respective sections: \reference{Section \ref{section:make}}
and \reference{Section \ref{section:python}}.

The \name{view} tool basically starts looking for \location{.tex} files
and builds a list of directories containing documents. Then, it provides
to the user the possibility to build and display a given document. If no
document name is given on the command line, then the tool draws a list
of the available documents.

People contributing to the kaneton documents must take care of following
the rules in relation with the {\LaTeX} language. Moreover, contributors
should look at the existing documents to understand to logic behind all
these rules.

Finally, note that nobody should create a document without discussing it
on the mailing-list first. Especially, be very careful in naming your
documents as people took good care of this directory in order to avoid
it to become messy.

If a document already exists with the same name, then go through the
mailing-list in order to decide whether to keep the current version. If
people decide to keep a document, then, the contributor in charge of writing
the new one should re-organise the documents by creating archives for
each year.


% export

%
% ---------- header -----------------------------------------------------------
%
% project       kaneton
%
% license       kaneton
%
% file          /home/mycure/kaneton/view/book/development/export.tex
%
% created       julien quintard   [wed may 23 18:58:41 2007]
% updated       julien quintard   [thu may 24 20:44:48 2007]
%

%
% ---------- export -----------------------------------------------------------
%

\subsection{Export}

The \textit{export} tool was introduced for making the process of releasing
easier. The tool takes an argument specifying the type of target release.

Recall the kaneton microkernel project is used as a material for operating
system courses. The source code of the microkernel is distributed to
the students with some parts missing. Then, students have to re-write
these pieces of code in order to prove their well-understanding of the
kernel internals. Additionnaly, the kaneton project is also a research project
in operating systems design.

As a result, the \textit{export} tool sometimes has to build a release
with pieces of code removed, sometimes not. Below are listed the different
type of release:

\begin{itemize}
  \item
    \textbf{backup}: this release type is basically a bare backup of
    the kaneton microkernel project source code.
  \item
    \textbf{dist}: the distribution release is performed by removing
    the repository-specific stuff.
  \item
    \textbf{k}$\gamma$\textbf{,}$\epsilon$: this type of release is intended
    to students. Therefore, repository-specific stuff is removed. Also
    teaching materials such as courses, testing scripts, cheating scripts
    etc.. - specified in the kaneton development environment variable
    \textit{\_HIDDEN\_} - are removed.

    \-

    Finally, the pieces of code comprised in the range $[\gamma,\epsilon]$
    are removed from the release. These pieces of code are marked using the
    \textit{export} syntax described next.

    \-

    The stages $\gamma$ and $\epsilon$ represent kaneton sub-project ranks:

    \begin{itemize}
      \item
	\textbf{0}: boot stuff: boostrap, bootloader etc..;
      \item
	\textbf{1}: memory management;
      \item
	\textbf{2}: event management: interrupts, I/O etc..;
      \item
	\textbf{3}: task management, scheduling;
      \item
	\textbf{4}: communication management.
    \end{itemize}
\end{itemize}

Finally, the generated release is named based on the \textit{\_EXPORT\_}
kaneton environment variable followed by the date and type of the release:
\textit{backup}, \textit{dist} or \textit{k}$\gamma$\textit{,}$\epsilon$.

% syntax

\subsubsubsection{Syntax}

As explained previously, pieces of code are removed in order to build
\textit{stage} releases.

The kaneton source code is marked so that the \textit{export} tool knows
what piece of code to remove and for what stage. Indeed, every piece of
educational code is marked by a tag indicating the stage it is related to.

The syntax used is illustrated below:

\begin{verbatim}
  /*                                                                [cut] k1 */

  /*
   * this function clones a segment.
   *
   * steps:
   *
   * 1) get the original segment object.
   * 2) reserve a new segment of same size with same permissions.
   * 3) copy the data from the old segment.
   * 4) call machine-dependent code.
   */

  t_error                 segment_clone(i_as                      asid,
                                        i_segment                 old,
                                        i_segment*                new)
  {
    o_segment*            from;
    t_perms               perms;

    SEGMENT_ENTER(segment);

    /*
     * 1)
     */

    if (segment_get(old, &from) != ERROR_NONE)
      SEGMENT_LEAVE(segment, ERROR_UNKNOWN);

    [...]

    /*
     * 4)
     */

    if (machdep_call(segment, segment_clone, asid, old, new) != ERROR_NONE)
      SEGMENT_LEAVE(segment, ERROR_UNKNOWN);

    SEGMENT_LEAVE(segment, ERROR_NONE);
  }

  /*                                                               [cut] /k1 */
\end{verbatim}

In this example, the kaneton teachers decided \textit{segment\_clone()}
was a functionality the students should implement.

The markings at the top \verb|[cut] k1| and bottom \verb|[cut] /k1| of this
example indicate the \textit{export} tool the location of the piece of code
to remove for the stage \textit{k1}.

Let us suppose a teacher $T_{1}$ wants to use kaneton leading the students to
the development of the memory management functionality, only. On the other
hand, another teacher, $T_{2}$, wants to use the whole kaneton project starting
with the bootloader implementation to the task management. Additionally,
this teacher chooses to hide the communication management pieces of code,
to avoid cheating between students of different universities for instance.

In the first case, since the students have to implement the kaneton managers
around the memory management, $T_{1}$ has to provide the students everything
the memory management stuff relies on, including some fundamental managers,
the bootloader etc.. Also, the teacher does not need to provide the source
code of the upper level managers. As a result, a \textit{k1,1} release will
remove the pieces of codes with every marking $k_{\alpha}$ for
$1 \le \alpha \le 1$.

The teacher $T_{2}$ needs something different since the students are going
to implement every major piece of the kaneton source code. Since this teacher
wants their students to implement all the steps, starting with \textit{k0}
to \textit{k4}, a \textit{k0,4} release will not contain the pieces of source
code marked $k_{\alpha}$ for $0 \le \alpha \le 4$.


% transcript

%
% ---------- header -----------------------------------------------------------
%
% project       kaneton
%
% license       kaneton
%
% file          /home/mycure/kaneton/view/book/development/transcript.tex
%
% created       julien quintard   [thu may 24 05:07:02 2007]
% updated       julien quintard   [fri jun  1 00:58:43 2007]
%

%
% ---------- transcript -------------------------------------------------------
%

\subsection{Transcript}
\label{section:transcript}

The \textit{transcript/} directory is composed of two tools related to
the management of transcripts. The \textit{record} tool captures a
shell session while the \textit{play} tool replays a captured session.

These tool were introduced to allow students to make a dynamic presentation
of their kaneton implementation's features and possibilities. These dynamic
presentations were supposed to replace the oral examinations.

These transcripts are not used by the main contributors of the kaneton
project yet. However, any teacher interested by this tool can use it.

The \textit{transcript/} directory contains subdirectories which classify
the transcripts.

The only transcript class currently in place is named \textit{basic} and
contains transcripts illustrating the use of the kaneton internal tools.


% cheat

%
% ---------- header -----------------------------------------------------------
%
% project       kaneton
%
% license       kaneton
%
% file          /home/mycure/kaneton/view/book/development/cheat.tex
%
% created       julien quintard   [thu may 24 11:57:37 2007]
% updated       julien quintard   [fri jun  1 01:03:21 2007]
%

%
% ---------- cheat ------------------------------------------------------------
%

\subsection{Cheat}
\label{section:cheat}

The \textit{cheat} tool checks whether students cheated by using pieces of
code from kaneton projects of the previous years.

The \textit{history/} directory is composed of directories organizing the
kaneton students implementations over the years and for every school and
university the education project was used. Then each subdirectory represents
a year and contains subdirectories for each students group of this year.

Each student group directory contains a \textit{sources/} subdirectory
containing the tarballs of the different kaneton stages: \textit{k0},
\textit{k1}, \textit{k2} and so on; a \textit{fingerprints/} directory
containing an internal source representation used for detecting cheating,
a \textit{tests/} directory containing a summary of the testing results
for each stage and a \textit{cheats/} directory which contains a list of
commonalities with other kaneton implementations of the same and previous
years.

The \textit{cheat} tool takes a year and a stage as arguments. Its first
task is to generate the fingerprints of the other kaneton implementations
for this stage of the same and previous years.

Once the fingerprints are generated, the tool performs the checks by
comparing each pair of kaneton implementations for this stage.

The \textit{cheat} tool is based on another tool which cannot be revealed
here. For more information, please contact your supervisor.


% test

%
% ---------- header -----------------------------------------------------------
%
% project       kaneton
%
% license       kaneton
%
% file          /home/mycure/kaneton/view/book/development/test.tex
%
% created       julien quintard   [thu may 24 12:18:23 2007]
% updated       julien quintard   [wed dec  8 22:37:27 2010]
%

%
% ---------- test -------------------------------------------------------------
%

\subsection{Test}
\label{section:test}

The \name{test} tool enables students to test their kaneton implementation
against a set of tests that have been designed by the contributors. Below
is briefly described the terminology used by this tool in order to give
the reader an overview of the general scenario involving students, the
administrator and the server running the test system.

\begin{itemize}
  \item
    A \name{certificate} is used to make sure clients can authenticate the
    test server;
  \item
    Every certificate is sealed by a cryptographic \name{key};
  \item
    Each user is provided with a \name{capability} in order to identify
    herself to the server;
  \item
    These capabilities are sealed by a \name{code} which the server uses
    in order to detect illegally forged capabilities;
  \item
    A \name{configuration} specifies the number of tests a user is allowed
    to requests the server;
  \item
    The user's \name{database} is generated based on a configuration and
    maintains the current user's state on the server including the number
    of tests performed so far, the kaneton implementations submitted for
    evaluation \etc{}
  \item
    A \name{snapshot} is a kaneton implementation in its shipping form;
  \item
    The \name{machine} represents the target
    \name{platform}/\name{architecture} couple on which a snapshot is
    supposed to be tested or evaluated for instance;
  \item
    An \name{image} represents a kaneton snapshot compiled in a bootable
    form;
  \item
    A \name{test} is a function included in the kernel which performs a
    specific set of operations and possibly prints information to the console;
  \item
    The tests are often gathered together in a \name{suite} which represents
    the testing unit students are offered to trigger for their kaneton
    implementation;
  \item
    Once a snapshot is received by the server in order to be tested,
    the system compiles it into an image. The server also takes care to
    include the tests in the compilation process so that they can be triggered.
    These pre-compiled tests are referred to as the \name{bundle};
  \item
    The image can then be tested by triggering the tests of the suite. The
    image is therefore booted in an emulated \name{environment}. This
    environment can sometimes be chosen and offers a trade-off between
    simplicity and realism. The most common environments are \name{QEMU}
    and \name{Xen};
  \item
    Depending on the success of the tests, a set of results is generated
    and compiled in a \name{bulletin} file;
  \item
    Finally, the server retrieves this bulletin, adds some meta information
    such as the date of the test, the environment and machine used \etc{}
    and stores everything in a \name{report}. Note that this report is
    also sent back to the user so she can consult it;
  \item
    Students can also decide to submit their kaneton implementation for
    a specific \name{stage} for future evaluation. Note that suites and
    stages are completely different though they often bear the same names:
    \name{k0}, \name{k1}, \name{k2} etc{};
  \item
    The administrator can decide to evaluate the snapshots which have been
    submitted for a stage by invoking a script which will attribute grades
    according to the \textit{point}s associated with every test.
  \item
    Finally, a \name{statement} is produced containing the grades of every
    student for a given stage.
\end{itemize}

The following describes the \name{test} tool according to the user's role
regarding the system: either the administrator who sets up the system or
a student who uses it in order to improve and/or evaluate his implementation.

%
% administrator
%
\subsubsection{Administrator}

The administrator is responsible for setting up the system but also maintaining
it on a daily basis.

% requirements
\subsubsubsection{Requirements}

The \name{test} tool must be installed on a publicly accessible server since
the server script will be waiting for incoming requests. Note that by default,
the clients assume the test server to be accessible at the address:
\location{https://test.opaak.org:8421}.

Besides, since the purpose of the \name{test} tool is to run the students'
kaneton implementation in emulated environments, both \name{QEMU} and
\name{Xen} should be available though one might want to configure the tool
for supporting a single environment, \name{QEMU} for instance.

Note that the test system has been developed with \name{Python 2.6} and
may be out of date by the time the administrator sets it up. In addition,
the system depends on a variety of \name{Python} packages including
\name{argparse}, \name{yaml}, \name{pyopenssl}, \name{hmac}, \name{pickle},
\name{xmlrpc}, \name{subprocess}, \name{serial} among others.

Finally, the administrator should make sure the following applications
are installed since some test scripts need them: \name{dd}, \name{mkfs.ext2},
\name{mount}, \name{umount}, \name{mutt}, \name{sendmail}, \name{qemu}
and \name{mkisofs}.

% set up
\subsubsubsection{Set Up}

The first step for an administrator consists in generating the necessary
files, especially the certificates, code and capabilities required for
securing the test system.

The \location{test/utilities/} directory contains the scripts that perform
such operations. Note that all the generated files are stored in the
\location{test/store/} directory.

First the \name{CA - Certification Authority}'s and server's certificates
must be generated. The first is used to issue certificates while the latter
is used for clients to identify the server with absolute certainty.

\begin{verbatim}
  $> make certificate
  [+] generating the CA and server's key/certificate pair
  [+] CA key/certificate generated
  [+] server key/certificate generated
  [+] CA and server's key/certificate pair generated and stored
  $> 
\end{verbatim}

The next step consists in generating a code for the administrator to
issue capabilities but also for the server to verify that the received
capabilities have not been illegally forged.

\begin{verbatim}
  $> make code
  [+] generating the server's code
  [+] server code successfuly generated and stored
  $> 
\end{verbatim}

With a server code, the students' and contributor's capabilities can be
built, hence granting them the right the contact the server.

The following generates the contributor's capability. This capability is
special in the way that contributors can perform any operation in a completely
contrain-free manner.

\begin{verbatim}
  $> make capability-contributor
  [+] generating the contributor's capability
  [+] contributor's capability generated and stored
  $> 
\end{verbatim}

In contrast, the following command generates a set of capabilities for the
students belonging to the school referred to as \name{``epita::2010''}. Note
that the script requires the \location{history/epita/2010/} to be populated
with the groups and their \location{people} file.

\begin{verbatim}
  $> make capability-school@epita::2010
  [+] generating students' capabilities
  [+] extracting the students from the history 'epita/2010'
  [+] students information retrieved
  [+] generating the students' capabilities:
  [+]   epita::2010::group11
  [+]   epita::2010::group10
  [+]   epita::2010::group13
  [+]   epita::2010::group12
  [+]   epita::2010::group33
  [+]   epita::2010::group32
  [+]   epita::2010::group17
  [+]   epita::2010::group30
  [+]   epita::2010::group19
  [+]   epita::2010::group18
  [+]   epita::2010::group5
  [+]   epita::2010::group4
  [+]   epita::2010::group7
  [+]   epita::2010::group6
  [+]   epita::2010::group1
  [+]   epita::2010::group3
  [+]   epita::2010::group2
  [+]   epita::2010::group15
  [+]   epita::2010::group9
  [+]   epita::2010::group8
  [+]   epita::2010::group14
  [+]   epita::2010::group31
  [+]   epita::2010::group16
  [+]   epita::2010::group24
  [+]   epita::2010::group25
  [+]   epita::2010::group26
  [+]   epita::2010::group27
  [+]   epita::2010::group20
  [+]   epita::2010::group21
  [+]   epita::2010::group22
  [+]   epita::2010::group23
  [+]   epita::2010::group28
  [+]   epita::2010::group29
  [+]   epita::2010::group34
  [+] students' capabilities generated and stored
  $> 
\end{verbatim}

In addition, the administrator could decide to generate or re-generate
a capability for a specific student of a school. The following shows an
example for such an action.

\begin{verbatim}
  $> make capability-student@epita::2010::group8
  [+] generating the student's capability:
  [+]   epita::2010::group8
  [+] student's capability generated and stored
  $> 
\end{verbatim}

The next step consists in the databases generation. A database contains the
state of a user profile including the number of test requests, the quota for
such tests, the submitted snapshots and so forth. The database files are
absolutely fundamental to the server since such databases are updated after
each client's request.

The syntax for generating databases follows the one for capabilities, as
shown next for the contributor.

\begin{verbatim}
  $> make database-contributor
  [+] generating database from contributor's configuration
  [+] contributor's database generated and stored
  $> 
\end{verbatim}

Once the certificates, code, capabilities and databases generated, the
administrator can move on to the deployment process.

% deployment
\subsubsubsection{Deployment}

The deployment basically consists in copying the \location{test/} environment
to the test server though one might want to copy the whole kaneton environment
or the smallest subset of the \location{test/} directory which should, in this
case, include the following absolutely necessary items:

\begin{itemize}
  \item
    The \location{test/environments/} directory which contains the descriptions
    of the supported test environments;
  \item
    The \location{test/images/} directory which contains a script for
    automatically generating a \name{Debian Live} system which is used
    for compiling a kaneton snapshot into a bootable image;
  \item
    The \location{test/packages/} directory which contains the \name{ktp -
    Kaneton Test Package} required by the server-side standalone scripts
    for manipulating files such as databases, capabilities \etc{} but also
    for performing cryptographic operations and send/receive \name{XMLRPC}
    requests;
  \item
    The \location{test/scripts/} directory which contains the fundamental
    scripts for building bootable images, distributing the capabilities to
    the students through emails, evaluating the submitted snapshots and so on;
  \item
    The \location{test/server/} directory which contains the server script
    for handling the clients' requests;
  \item
    The \location{test/stages/} directory which contains the files requirement
    for evaluating the students' snapshots;
  \item
    The \location{test/store/} directory which contains the generated files
    such as the users' databases, the server's code and certificate; and
  \item
    The \location{test/suites/} directory which contains the files describing
    the tests to be including in a given tests suite.
\end{itemize}

Once copied, the administrator only needs to launch the server script located
in the \location{test/server/} directory, as shown below:

\begin{verbatim}
  $> ./server.py
  [meta] serving on 88.191.84.128:8421
\end{verbatim}

Note that a few additional steps may be required depending on the current state
of the kaneton development.

The first of these steps may consist in generating a \name{Debian Live} system
since this is absolutely required for the test system to work. For more
information regarding the generation of such an image, please refer to the
\location{test/images/} directory.

The second step should consist for the administrator in building the kaneton
tests bundle. The bundle represents a pre-compiled set of tests that is
included in the students' snapshot compilation process. The tests are
pre-compiled in order to prevent leaking information since students could
very well dump the content of those tests and force the compilation to fail,
hence retrieving the source code in the compilation process' error log.

In order to generate such a bundle, the administrator must first activate
the \name{test} module, as show next:

\begin{verbatim}
  _MODULES_               +=              test
\end{verbatim}

Then, the administrator must move to the \location{test/tests/} directory
and launch a compilation process through the following command:

\begin{verbatim}
  $> make
\end{verbatim}

Once generated, the test bundle, located in \location{store/bundle/[machine]/}
must be copied to the server, at the same location.

Finally, for more information on the server script, please refer to the
\location{test/server/} directory.

% scripts
\subsubsubsection{Scripts}

Although the deployment process is pretty straightforward, the administrator
is required to manage the test system through several scripts.

First, the \name{distribute} script must be used by the administrator to send
the capabilities to the respective owners so that the students can use
the test system. Note that this script relies on the \name{Mutt} mailing
system for sending the emails containing the attached capabilities.

\begin{verbatim}
  $> ./distribute.py
  recipients:
    contributor
  $>
\end{verbatim}

While the \name{construct} script enables the administrator to build a
bootable image from a kaneton snapshot, the \name{stress} script takes
a bootable image and triggers the tests belonging to the given test
suite. Note that both scripts are directly used by the server script for
building and testing the received kaneton snapshots.

\begin{verbatim}
  $> ./construct.py --snapshot kaneton.tar.bz2                          \
                    --image kaneton.img                                 \
                    --environment xen                                   \
  the kaneton image has been constructed in 'kaneton.img'
  $> ./stress.py --image kaneton.img                                    \
                 --suite k2                                             \
                 --environment xen                                      \
                 --verbose
  segment
    permissions/01 :: true
  id
    simple :: true
    clone :: true
    multiple :: true
  $> 
\end{verbatim}

Note that the administrator could also test a kaneton image manually,
especially through the following command:

\begin{verbatim}
  $> qemu -fda kaneton.img -curses
\end{verbatim}

Besides, note that an administrator willing to include a new test in the
system would probably want to test it locally first since testing through
the server takes some time. In order to test locally, the administrator
must first activate the bundle module in its user profile
\location{environment/profile/user/\${KANETON\_USER}/\${KANETON\_USER}.conf}:

\begin{verbatim}
  _MODULES_               +=              bundle
\end{verbatim}

Then, the administrator must trigger the test by calling the test function
manually in its kaneton implementation. For instance, in order to trigger
the \name{kaneton/core/task/guest} test, the administrator could add the
following line after \code{kernel\_initialize()} and before running the
test system in \location{kaneton/core/core.c}:

\begin{verbatim}
  [...]

  module_call(console, message,
              '+', "starting the kernel\n");

  assert(kernel_initialize() == STATUS_OK);

  /* XXX[temporary] */
  test_core_task_guest();

  module_call(test, run);

  [...]
\end{verbatim}

Once the kaneton image rebuilt, the administrator can boot it locally
through \name{QEMU} and get the output, hence check that the test went
as excepted:

\begin{verbatim}
  $> qemu -fda environment/profile/user/${KANETON_USER}/${KANETON_USER}.img
\end{verbatim}

Back to the server side, the \name{evaluate} script can be used by the
administrator in order to assign grades to the snapshots submitted by the
students. The script generates a statement containing the results of this
evaluation process.

\begin{verbatim}
  $> ./evaluate.py --stage k2                                           \
                   --pattern "^epita::2010::.*$"
  the statement has been saved in '../store/statement/20101102-223645.db'
  $> 
\end{verbatim}

Finally, the \name{dump} script takes any \name{YAML}-based file and
displays its inner structure in a hierarchical manner.

\begin{verbatim}
  $> ./dump.py --path ../store/statement/20101102-223848.db
  meta:
    reference:              20.0
    stage:                  k2
  data:
    epita::2010::group7:
      date:                 2010/11/02 20:46:44
      grade:                16.0
      snapshot:             20101102-204644
      members:
        email:              admin@opaak.org
        name:               admin
      configurations:
        Xen:
          report:           20101102-224213
          notch:            4
          score:            4
        QEMU:
          report:           20101102-223848
          notch:            4
          score:            0
\end{verbatim}

%
% student
%
\subsubsection{Student}

The student has the possibility to request actions from the test server
through the client script located in \location{test/client/}.

% requirements
\subsubsubsection{Requirements}

Although the client script is integrated in the kaneton environment, it also
makes use of the \name{ktp}. Therefore, as for the server, the client depends
on a variety of \name{Python} packages including \name{yaml}, \name{pyopenssl},
\name{hmac}, \name{pickle}, \name{xmlrpc}, \name{subprocess} among others.

% use
\subsubsubsection{Use}

The client script enables the user to request one of the five operations
described below.

\begin{verbatim}
  $> make
  [!] usage: client.py [command]

  [!] commands:
  [!]   information
  [!]   submit-[stage]
  [!]   test-[environment]::[suite]
  [!]   list
  [!]   display-[identifier]
  [!]   retest-[identifier]
  $>
\end{verbatim}

The \name{information} operation requests the server to return information
on the current state of the user's profile. The information returned range
from the number of tests performed, the quota for every test suite to the
available stages or the snapshots having been previously submitted.

\begin{verbatim}
  $> make information
  [+] configuration:
  [+]   server:                 https://test.opaak.org:8421
  [+]   capability:             /data/mycure/repositories/kaneton/environment/profile/user/julien.quintard/julien.quintard.cap
  [+]   platform:               ibm-pc
  [+]   architecture:           ia32/educational

  [+] information:
  [+]   profile:
  [+]     identifier:           contributor
  [+]     community:            contributors
  [+]     members:
  [+]       name:               admin
  [+]       email:              admin@opaak.org
  [+]   suites:
  [+]                           k1
  [+]                           k3
  [+]                           k2
  [+]                           kaneton
  [+]   stages:
  [+]                           k1
  [+]                           k2
  [+]                           k3
  [+]   environments:
  [+]                           qemu
  [+]                           xen
  [+]   database:
  [+]     reports:
  [+]       xen:
  [+]         ibm-pc.ia32/educational:
  [+]           k3:
  [+]           k2:
  [+]           k1:
  [+]       qemu:
  [+]         ibm-pc.ia32/educational:
  [+]           k3:
  [+]           k2:
  [+]           k1:
  [+]     settings:
  [+]       xen:
  [+]         ibm-pc.ia32/educational:
  [+]           k3:
  [+]             requests:     0
  [+]             quota:        -1
  [+]           k2:
  [+]             requests:     0
  [+]             quota:        -1
  [+]           k1:
  [+]             requests:     0
  [+]             quota:        -1
  [+]       qemu:
  [+]         ibm-pc.ia32/educational:
  [+]           k3:
  [+]             requests:     0
  [+]             quota:        -1
  [+]           k2:
  [+]             requests:     0
  [+]             quota:        -1
  [+]           k1:
  [+]             requests:     0
  [+]             quota:        -1
  $> 
\end{verbatim}

The \name{test} command enables the user to trigger a test suite for the
current kaneton implementation on the specified environment such as \name{QEMU}
or \textit{Xen} for instance.

The server then returns the resulted report which the client stores locally
in \location{test/store/report/}. In addition, the client displays a quick
summary of the report in order for the user to know whether things went
as expected.

\begin{verbatim}
  $> make test-xen::k2
  [+] configuration:
  [+]   server:                 https://test.opaak.org:8421
  [+]   capability:             /data/mycure/repositories/kaneton/environment/profile/user/julien.quintard/julien.quintard.cap
  [+]   platform:               ibm-pc
  [+]   architecture:           ia32/educational

  [+] report(20101103:140601):
  [+]   segment                                                           [1/1]
  [+]   id                                                                [3/3]
  $> 
\end{verbatim}

The \name{list} command enables the user to display the identifiers of the
reports in the local store.

\begin{verbatim}
  $> make list
  [+] reports:
  [+]   20101103:140601:
  [+]     xen :: ibm-pc :: ia32/educational :: k2 :: 2010/11/03 14:06:01
\end{verbatim}

The \name{display} command gives the user the possibility to dump a locally
stored report in a very detailed way.

\begin{verbatim}
  $> make display-20101103:140601
  [+] report:
  [+]   meta:
  [+]     platform:               ibm-pc
  [+]     date:                   2010/11/03 14:06:01
  [+]     architecture:           ia32/educational
  [+]     duration:               63.499
  [+]     suite:                  k2
  [+]     identifier:             20101103:140601
  [+]     environments:
  [+]       stress:               xen
  [+]       construct:            xen
  [+]   data:
  [+]     segment:                                                        [1/1]
  [+]       permissions/01:
  [+]         status: True
  [+]         description: This test creates a task and address space before reserving a segment and changing its permissions.
  [+]         duration: 0.010
  [+]         output: 
  [+]     id:                                                             [3/3]
  [+]       simple:
  [+]         status: True
  [+]         description: This test reserves a single identifier.
  [+]         duration: 0.004
  [+]         output: 
  [+]       clone:
  [+]         status: True
  [+]         description: This test reserves, clones and releases identifiers.
  [+]         duration: 0.005
  [+]         output: 
  [+]       multiple:
  [+]         status: True
  [+]         description: This test reserves thousands of identifiers, checking that no collisions occured.
  [+]         duration: 0.040
  [+]         output: 
  $> 
\end{verbatim}

The \name{submit} command sends the user's snapshot to the server so as to
be evaluated for the given stage.

\begin{verbatim}
  $> make submit-k1
  [+] configuration:
  [+]   server:                 https://test.opaak.org:8421
  [+]   capability:             /data/mycure/repositories/kaneton/environment/profile/user/julien.quintard/julien.quintard.cap
  [+]   platform:               ibm-pc
  [+]   architecture:           ia32/educational

  [+] the snapshot has been submitted successfully
  $> 
\end{verbatim}

Finally, the \name{retest} command provides contributors the possibility to
re-launch the test suite of the given identified test. This command is
especially useful to re-test a snapshot should an unexpected error occur on
the test server.

Indeed since test requests are limited for students, it would be unfair for the
student to be forced to sacrifice a test slot because something went wrong
on the server-side. By requesting a contributor, the student's snapshot can
be re-tested. Once the test complete, an email is sent to the student along
with the attached report.

\begin{verbatim}
  $> make retest-20101103:140601
  [+] configuration:
  [+]   server:                 https://test.opaak.org:8421
  [+]   capability:             /data/mycure/repositories/kaneton/environment/profile/user/julien.quintard/julien.quintard.cap
  [+]   platform:               ibm-pc
  [+]   architecture:           ia32/educational

  [+] the snapshot has been re-tested successfully
  $> 
\end{verbatim}

%
% robot
%
\subsubsection{Robot}

The \name{robot} test tool enables contributors to test the kaneton research
implementation on a regular basis; hence control the status of the development.

The robot basically retrieves the kaneton implementation by checking out the
\name{Subversion} repository. Then, several test suites are triggered through
the test client. Once the reports have been received, a message is built
summarizing the results. This message is then sent to the kaneton contributors
mailing-list.

The deployment of the \name{robot} is quite straigthforward. First, the
\location{test/robot/} directory must be copied to the server. Note that
the \name{robot.py} script depends upon the \name{ktp} package which must
therefore be copied as well.

Then, the \name{SSH} configuration file \name{config} must be placed in
the \location{\${HOME}/.ssh/} directory. Besides, this file should be edited in
order to properly reference the \name{SSH} keys since the default configuration
assumes the kaneton test directory to be located at \location{/kaneton/}.

Finally, the \name{robot.cron} crontab file must be setup through the
following command in order to trigger the robot every night:

\begin{verbatim}
  $> crontab robot.cron
\end{verbatim}

Once again, the administrator should make sure to edit this file should
the robot files not be located in the default location \ie{}
\location{/kaneton/}.


% control panel

%
% ---------- header -----------------------------------------------------------
%
% project       kaneton
%
% license       kaneton
%
% file          /home/mycure/kaneton/view/book/development/control-panel.tex
%
% created       julien quintard   [sun may 20 15:22:52 2007]
% updated       julien quintard   [mon may  4 20:01:20 2009]
%

%
% ---------- control panel ----------------------------------------------------
%

\subsection{Control Panel}
\label{section:control panel}

The kaneton environment allows the developer to trigger every action from
the \name{Make} file of the project's root directory.

These actions are listed below:

\command{make initialize}
        {
	  This action initializes the kaneton development environment by
	  invoking the \location{init.py} script of the \location{environment/}
	  directory.
	}

\command{make clean}
	{
	  This action cleans the kaneton development environment.
	}

\command{make main}
	{
	  This action triggers the default rule which aims at compiling every
	  piece the final system needs to be set up on a bootable device.

	  \-

	  \example{\$ make main}

	  \example{\$ make}
	}

\command{make clear}
	{
	  This action removes every compiled files.
	}

\command{make headers}
	{
	  This action generates \name{Make} files' \name{C} header
	  files dependencies.
	}

\command{make prototypes}
	{
	  This action generates C prototypes.
	}

\command{make test}
	{
	  This action runs the test suite in order to validate the kaneton
	  microkernel behaviour.
	}

\command{make cheat}
	{
	  This action launches the cheat tests on students kaneton
	  implementations.

	  \-

	  \example{\$ make cheat}

	  \-

	  \example{\$ make cheat-EPITA::2006::k3}
	}

\command{make build}
	{
	  This action builds the boot device.
	}

\command{make install}
	{
	  This action installs the kaneton microkernel with its dependencies:
	  configuration files, bootloader \etc{} on the boot device.
	}

\command{make export}
	{
	  This action builds a kaneton distribution package.

	  \-

	  \example{\$ make export}

	  \-

	  \example{\$ make export-k3,5}

	  \-

	  \example{\$ make export-back}
	}

\command{make view}
	{
	  This action builds and displays a kaneton document.

	  \-

	  \example{\$ make view}

	  \-

	  \example{\$ make view-devel}

	  \-

	  \example{\$ make view-book::kaneton}
	}

\command{make record}
	{
	  This action records a real-time session.

	  \-

	  \example{\$ make record}

	  \-

	  \example{\$ make record-basic::test.ts}
	}

\command{make play}
	{
	  This action plays a recorded session.

	  \-

	  \example{\$ make play}

	  \-

	  \example{\$ make play-basic::prototypes.ts}

	  \-

	  \example{\$ make play-prototy}
	}

\command{make info}
	{
	  This action displays general information about kaneton.
	}


%
% external
%

\section{External}

The kaneton contributors use several other tools for the communication, the
development etc.. These tools are described in the following sections.

% mailing-list

%
% ---------- header -----------------------------------------------------------
%
% project       kaneton
%
% license       kaneton
%
% file          /home/mycure/kaneton/view/book/development/mailing-list.tex
%
% created       julien quintard   [thu may 24 19:55:18 2007]
% updated       julien quintard   [thu may 24 20:43:15 2007]
%

%
% ---------- mailing-list -----------------------------------------------------
%

\subsection{Mailing-List}

Because people do not want to use several communication tools: email,
newsgroup, forum etc.. and because everybody has an email address, the
kaneton people communicate through a mailing-list.

This mailing-list is in fact a \textit{Google} group. Indeed, the kaneton
project relies on three distinct communication groups:

\begin{itemize}
  \item
    \textbf{kaneton} which is used to make announcements about new releases,
    patchs etc..

    \-

    This group is not used yet.
  \item
    \textbf{kaneton-developers} is dedicated to the communication between the
    people involved in the development of the kaneton microkernel reference.

    \-

    This group is therefore private.
  \item
    \textbf{kaneton-students} can be used in an absolutely free-way for
    students for communicating about their kaneton implementation.

    \-

    Anybody can join this group.
\end{itemize}

Needless to say, community behavioural rules enumerated in a previous section
must be followed when communicating on the kaneton mailing-list. Every
contributor is welcomed to give its point of view, to ask questions etc.. but
this must be done with politness, respect and humility.

Everyone communicating through the mailing-list must read the
\textit{Netiquette} which describes the rules inherent to the communication
on the Internet. Especially, people should take care of writing messages in
respect of the \textit{80} columns; and should always cut off useless parts
of a previous message when responding.

It is likely a real-time communication tool will be very useful in the
future, as an \textit{IRC} channel, for instance. However, communicating on
these extra media will not be mandatory until kaneton people decide it is.

kaneton people are asked to use the mailing-list communication medium
in a perfect way as it is the unique intra-communication channel. In
addition then, contributors must read their emails on a regular-basis
as some people rely on the decision of others.

The \textit{kaneton-students} mailing-list must be used carefully. As
an example, people should never paste pieces of source code or ask
questions implying an answer with the solution. Even if it is a free
group, people abusing of this communication channel could easily be banned.

People must always respond in the appropriate discussion. If, in a discussion,
a different subject is discussed, then, one of the contributor must create
a new discussion in order facilitate the communication.

Finally, the discussion subjects must be tagged like the following examples:

\begin{verbatim}
  [ia32/optimised] mapping issues

  [segment] segment_clone() :: bug?

  [research] new paper about OS design
\end{verbatim}

There is no list of official tags. The users are simply asked to make
their discussion subjects as clear as possible to simplify the task
consisting in looking for old topics in the archives. Indeed, remember
that newcomers should - if they respect the rules - look at the archives
to avoid discussing an old subject on the mailing-list.


% repository

%
% ---------- header -----------------------------------------------------------
%
% project       kaneton
%
% license       kaneton
%
% file          /home/mycure/kaneton/view/book/development/repository.tex
%
% created       julien quintard   [thu may 24 20:43:26 2007]
% updated       julien quintard   [wed jun 13 22:32:31 2007]
%

%
% ---------- repository -------------------------------------------------------
%

\subsection{Repository}
\label{section:repository}

The repository contains everything related to the kaneton microkernel
project, in other words, the kaneton source tree described in
\textit{Chapter \ref{chapter:source tree}}. Indeed, the repository contains
the whole history of the kaneton project including the documentation, the
source code but also the students tarballs over the years.

The actual repository is based on the \textit{Subversion} software which
provides far more advanced features than its historical rival \textit{CVS}.

The repository is actually hosted on the \textit{kaneton.org} server which
also contains the web server and everything else related to the management
of the kaneton microkernel project.

The repository is accessed in a secure way through a \textit{SSH} channel.
Indeed, the kaneton \textit{Subversion} repository can be accessed at the
following address:

\begin{verbatim}
  svn+ssh://subversion@repositories.kaneton.org/kaneton
\end{verbatim}

Note that the security is achieved by the use of \textit{SSH} keys. Therefore,
any new contributor should get in touch with an administrator of the
kaneton server in order to obtain an access. Also note that, a test period
could be set up for a new contributor to get the trust of the kaneton
community. For more information, please refer to \textit{Chapter
\ref{chapter:community}}.

A contributor willing to create a \textit{SSH} key shoud simply use this
\textit{Unix} command:

\begin{verbatim}
  $> ssh-keygen -t dsa
\end{verbatim}

For more information about how to use the repository, please refer to the
official \textit{Subversion} documentation. The same goes for the
\textit{SSH} tools suite.

The example below illustrates the checkout of the kaneton repository.

\begin{verbatim}
  $> svn checkout svn+ssh://subversion@repositories.kaneton.org/kaneton
\end{verbatim}

The contributors getting access to the kaneton repositories must behave
properly according to the obvious cooperative development rules. As an
example, a kaneton developer must not perform any commit before making sure
the kaneton microkernel compiles and passes all the tests.

The repository organisation is crucial. Therefore, nothing should be
added, removed or renamed without the permission of the developers in charge
of the repository.

Finally, any commit must come with a log describing the modifications
implied by the commit. These logs must conform to the following syntax.

\begin{verbatim}
  [kaneton/core/segment/]
    o the bug about the permissions was corrected in segment_clone().
    o an algorithm based on a b-tree was added.

  [environment/profile/user/julien.quintard/]
    o some personal configurations were modified.
\end{verbatim}

Following this syntax is very important as an email is sent to the
\textit{kaneton-developers} mailing-list every time a commit is performed.
Therefore, the contributors reading the mailing-list are aware of every
modification in the kaneton source code. This feature can also be used
to review the modifications done by a new contributor in order to help
him doing things in a better way.

Note that there must not be any file with the executable flag permission
enabled. Moreover, scripts files must not contain any \textit{shebang}.
Indeed, the kaneton development environment knows which interpreter to
use for every type of file. It is therefore a non-sense to introduce a
hard-coded path to an interpreter.

Tarball file names must be extended with \textit{.tar} while \textit{bzip2}
compressed tarballs must be extended with \textit{.tar.bz2}.


% wiki

%
% ---------- header -----------------------------------------------------------
%
% project       kaneton
%
% license       kaneton
%
% file          /home/mycure/kaneton/view/book/development/wiki.tex
%
% created       julien quintard   [thu may 24 23:06:02 2007]
% updated       julien quintard   [fri aug  1 15:52:28 2008]
%

%
% ---------- wiki -------------------------------------------------------------
%

\subsection{Wiki}
\label{section:wiki}

A Wiki is used both for external and internal communication. The software
used is called \name{TWiki} and provides a pretty simple syntax with
many additional plugins to customize the website. This solution was used
for historical reasons but also because the \name{TWiki} rendering can
be customized through templates in order to get a final visual close to
classical websites. Thus, the kaneton website looks like an ordinary
website but powered by a Wiki engine.

The Wiki is hosted at \location{http://kaneton.opaak.org} and contains four
webs: an extranet and three intranets. The main web, accessed through the
address above is the external website. This website contains news, papers,
documentation and general information on the kaneton project. The three other
webs are used more as intranets or wikis more than as websites. Two of these
webs are private to the kaneton developers and the kaneton teachers,
respectively. The latter web is intended to the students and contains
documents, links \etc{} about low-level programming, kernel development \etc{}
as well as information about courses related to the kaneton project.

Note that the Wiki reserved for the kaneton developers must be used
instead of the Wiki eventually provided by the project management tool.

Everybody involved in the kaneton project must contribute to the kaneton
website as well as to the kaneton intranets. Indeed, the external communication
is fundamental, even in an open source project and the kaneton website is
the only public communication medium.

New contributors are then asked to register onto the kaneton \name{TWiki}
at \location{http://kaneton.opaak.org}. Once done, the contributor should
inform the person in charge of the kaneton website so that  the contributor's
account is activated. As a result, the contributor will be able to modify
pages of the website and intranets.


% project management

%
% ---------- header -----------------------------------------------------------
%
% project       kaneton
%
% license       kaneton
%
% file          /home/mycure/kane.../book/development/project-management.tex
%
% created       julien quintard   [fri may 25 19:26:17 2007]
% updated       julien quintard   [thu may 31 06:25:56 2007]
%

%
% ---------- project management -----------------------------------------------
%

\subsection{Project Management}
\label{section:project management}

XXX

\begin{comment}
le systeme de tickets/bugs est egalement tres important. chaque ticket se
voit affecte une priorite et il est important de comprendre que pour le
bien global du projet, un developpeur ne peut se contenter de faire ce
qui lui plait, il se doit de contribuer egalement a la resolution de problemes.

encore une fois les tickets doivent suivre une norme.
\end{comment}



%%%
%% licence       kaneton licence
%%
%% project       kaneton
%%
%% file          /home/mycure/kaneton/view/papers/kaneton/kaneton.tex
%%
%% created       julien quintard   [thu dec  8 00:26:00 2005]
%% updated       julien quintard   [thu mar  2 13:59:43 2006]
%%

%
% template
%

%%
%% licence       kaneton licence
%%
%% project       kaneton
%%
%% file          /home/mycure/kaneton/view/templates/book.tex
%%
%% created       julien quintard   [wed mar  1 23:45:22 2006]
%% updated       julien quintard   [thu may  4 12:36:54 2006]
%%

%
% class
%

\documentclass[10pt,a4wide]{book}

%
% packages
%

\usepackage[english]{babel}
\usepackage[T1]{fontenc}
\usepackage{a4wide}
\usepackage{fancyheadings}
\usepackage{multicol}
\usepackage{indentfirst}
\usepackage{graphicx}
\usepackage{color}
\usepackage{xcolor}
\usepackage{verbatim}

\usepackage{aeguill}

\usepackage[Lenny]{../../../tools/latex/fncychap}

\pagestyle{fancy}

\setlength{\footrulewidth}{0.3pt}
\setlength{\parindent}{0.3cm}
\setlength{\parskip}{2ex plus 0.5ex minus 0.2ex}

%
% logos
%

\newcommand{\logos}
  {
    \begin{center}
      \includegraphics[scale=0.8]{../../logos/kaneton.pdf}
    \end{center}
  }

%
% colors
%

\definecolor{functioncolor}{rgb}{0.40,0.00,0.00}
\definecolor{commandcolor}{rgb}{0.00,0.00,0.40}
\definecolor{verbatimcolor}{rgb}{0.00,0.40,0.00}
\definecolor{noticecolor}{rgb}{0.87,0.84,0.02}

%
% function
%

\newcommand\function[3]{
  \begin{tabular}{p{0.2cm}p{13.8cm}}
  & {\color{functioncolor}\textbf{#1}}#2
  \end{tabular}

  \begin{tabular}{p{1cm}p{13cm}}
  & #3
  \end{tabular}}

%
% align
%

\newcommand\align[1]{
  \\ & \hspace{#1}}

%
% argument
%

\newcommand\argument[1]{\textit{#1}}

%
% command
%

\newcommand\command[2]{
  \begin{tabular}{p{0.2cm}p{13.8cm}}
  & {\color{commandcolor}\textbf{#1}}
  \end{tabular}

  \begin{tabular}{p{1cm}p{13cm}}
  & #2
  \end{tabular}}

%
% notice
%

\newcommand\notice[1]{
  {\color{noticecolor}\textbf{Notice}}

  \begin{tabular}{p{0.2cm}p{13.8cm}}
  & #1
  \end{tabular}}

%
% example
%

\newcommand\example[1]{
  \textit{Example:}

  \begin{tabular}{p{0.2cm}p{13.8cm}}
  & \textit{#1}
  \end{tabular}}

%
% warning XXX
%

%
% verbatim stuff
%

\makeatletter

\renewcommand{\verbatim@font}
  {\ttfamily\footnotesize\color{verbatimcolor}\selectfont}

\def\verbatim@processline{\hskip15ex\the\verbatim@line\par}

\makeatother

%
% header
%

\rhead{}
\rfoot{\scriptsize{The kaneton microkernel project}}

\date{\scriptsize{\today}}


%
% header
%

\lhead{\scriptsize{The kaneton microkernel project reference}}
\rhead{}

%
% title
%

\title{The kaneton microkernel reference
       \logos}

%
% authors
%

\author{\small{Julien Quintard},
        \small{Matthieu Bucchianeri},
        \small{Renaud Voltz}}

%
% document
%

\begin{document}

%
% title
%

\maketitle

%
% --------- text --------------------------------------------------------------
%

%
% authors
%

This document describes the kaneton microkernel reference project.

This document should be used by every student willing implement the
kaneton microkernel.

All the kaneton documents are available on the official website:
\textbf{http://www.kaneton.org}.

This document is under the \textbf{kaneton license}.

This document will reference the kaneton people by the word \textit{we}.

\newpage

The kaneton project was introduced at EPITA by two computer science
students:

\begin{itemize}
  \item
    Julien Quintard \footnote{quinta\_j@epita.fr}
  \item
    Jean-Pascal Billaud \footnote{billau\_j@epita.fr}
\end{itemize}

This document especially describes the kaneton reference microkernel
developed by:

\begin{itemize}
  \item
    Julien Quintard
  \item
    Matthieu Bucchianeri \footnote{bucchi\_m@epita.fr}
  \item
    Renaud Voltz \footnote{voltz\_r@epita.fr}
\end{itemize}

Nevertheless, many people contributed to this project and we thank them.

%
% toc
%

\tableofcontents

%
% chapters
%

XXX revoir ordre pour mieux introduire kaneton: design.pdf

%
% ---------- header -----------------------------------------------------------
%
% project       kaneton
%
% license       kaneton
%
% file          /home/mycure/kaneton/view/book/kaneton/goals.tex
%
% created       julien quintard   [fri jun  1 13:58:12 2007]
% updated       julien quintard   [mon may 19 23:09:48 2008]
%

%
% ---------- goals ------------------------------------------------------------
%

\chapter{Goals}
\label{chapter:goals}

In this chapter we will briefly introduce the kaneton microkernel
through the kaneton microkernel goals.

\newpage

%
% ---------- text -------------------------------------------------------------
%

The project was primarily designed by two students in computer science,
\name{Julien Quintard} and \name{Jean-Pascal Billaud}.

These two students previously actively contributed to the development
of a nanokernel-based operating system project in a French research laboratory.
This system was not powerful enough, especially from the design point of view.

Therefore, the two students started the design of a new microkernel
by their own, called \term{kaneton}, for educational purposes.

The design was based on five fundamental guidelines.

\begin{enumerate}
  \item
    \textbf{Educational}

    \-

    The kaneton project is built to become an educational project. The design
    as well as the implementation must therefore be as understandable as
    possible so that everyone interested in kernel internals can go through the
    documents and source code and actually understand how it works.

    \-

    This \textit{understandable} property can be achieved through a very clear
    and coherent design. Moreover, the implementation should be written using
    modern tools and techniques to make the code as generic as possible and
    easily readable.
  \item
    \textbf{Portability}

    \-

    The microkernel was particularly designed to be portable. The designers
    tried to develop a portability system powerful enough to port kaneton on
    any, existing or not, architectures.
  \item
    \textbf{Maintanability}

    \-

    Although microkernel-based operating systems rely on a modular design,
    kaneton designers also wanted the microkernel itself to be modular and
    maintainable.
  \item
    \textbf{Distributed Computing}

    \-

    The kaneton microkernel must be designed to fit distributed operating
    systems requirements. Indeed, the kaneton microkernel was developed in
    order to design and implement a distributed operating system named
    \term{kayou}.

    \-

    This point led to many specific choices in the kaneton microkernel design.
  \item
    \textbf{Demystification}

    \-

    kaneton people wanted to break some well-known kind of computer
    science rules. Indeed, for instance, many computer scientists consider
    the source code as the actual documentation. Also, for many low-level
    programmers, the kernel boot source code and more generally the
    kernel source code itself cannot be understandable, clear and coherent as
    it is related to low-level programming: microprocessor, devices \etc{}

    kaneton people paid particular attention to the microkernel source code to
    be easily understandable, maintainable and extendable. Moreover, kaneton
    people tried to write documentation for every part of the project.
\end{enumerate}

Notice that building an educational microkernel project is nothing innovative.
Indeed few other projects already exist; the most popular being \name{MINIX}
from \name{Vrije Universiteit}, \name{NachOS} from \name{Berkeley University}
or \name{PintOS} from \name{Stanford University}.

kaneton people tried to design and implement a modern microkernel since, the
original \name{MINIX} microkernel for example, do not use modern development
tools. Moreover, the kaneton source code is heavily commented and use modern
languages techniques while trying to stay easily understandable.

The educational characteristic of kaneton does not constraint it from being
optimised afterwards. kaneton people believe that implementing optimised
algorithms in the first place does not lead to maintainable implementations.

Finally, note that the kaneton project is actually composed of two projects:
the \name{kaneton microkernel \term{educational} project} which provides
everything necessary to students willing to learn about kernels internals;
and the \name{kaneton microkernel \term{research} project} which focuses on
designing and implementing a powerful, reliable, flexible microkernel.
Obviously these two projects are highly related as the kaneton educational
project relies on the implementation of the kaneton research project.

%%
%% licence       kaneton licence
%%
%% project       kaneton
%%
%% file          /home/mycure/kaneton/view/papers/kaneton/overview.tex
%%
%% created       matthieu bucchianeri   [mon jan 30 17:09:45 2006]
%% updated       julien quintard   [thu mar  2 13:12:22 2006]
%%

%
% overview
%

\chapter{Overview}

XXX ce chapitre va vous aider a reconnaitre les fonctionnalites principale
XXX d'un kernel dans kaneton.

The kaneton microkernel is only the core of an operating system.
Main tasks like hardware drivers or user services are implemented as
\textbf{servers}. So the microkernel only has a few functionalities to
provide:

\begin{itemize}
  \item
    Memory management.
  \item
    Process management.
  \item
    Communication.
  \item
    Events.
\end{itemize}

In this chapter we will describe briefly these tasks and all the
associated managers.

%
% memory management
%

\section{Memory Management}

Handling the memory -- from virtual address space to physical
addressing -- is done by three major managers, the \textbf{as},
\textbf{segment} and \textbf{region} managers.

%
% as
%

\subsection{as}

The address space manager just manages the different address spaces
used by the kaneton tasks.

In kaneton, we call an \textbf{as - address space} a list of memory
locations referenced by a task. Each task has its own address space.

%
% segment
%

\subsection{segment}

The segment manager just manages the segments reserved by
the different kaneton entities including the kernel, the drivers etc..

In kaneton terms a \textbf{segment} is a contiguous area of reserved
physical memory.

%
% region
%

\subsection{region}

The region manager keeps track of regions used to map segments for
each address space reserved on the system.

In kaneton, a \textbf{region} is contiguous area of virtual memory
mapping a segment's part.

%
% process management
%

\section{Process Management}

XXX

%
% communication
%

\section{Communication}

XXX

%
% events
%

\section{Events}

XXX

%%
%% copyright quintard julien
%% 
%% kaneton
%% 
%% development-environment.tex
%% 
%% path          /home/mycure/kaneton
%% 
%% made by mycure
%%         quintard julien   [quinta_j@epita.fr]
%% 
%% started on    Tue Jul  5 12:23:08 2005   mycure
%% last update   Sun Oct 23 02:55:45 2005   mycure
%%

%
% class
%

\documentclass[8pt]{beamer}

%
% packages
%

\usepackage{pgf,pgfarrows,pgfnodes,pgfautomata,pgfheaps,pgfshade}
\usepackage{colortbl}
\usepackage{times}
\usepackage{amsmath,amssymb}
\usepackage{graphics}
\usepackage{graphicx}
\usepackage{color}
\usepackage{xcolor}
\usepackage[english]{babel}
\usepackage{enumerate}
\usepackage[latin1]{inputenc}

%
% style
%

\usepackage{beamerthemesplit}
\setbeamercovered{dynamic}

%
% verbatim font
%

\definecolor{verbatimcolor}{rgb}{0,0.4,0}

\makeatletter
\renewcommand{\verbatim@font}
  {\ttfamily\footnotesize\color{verbatimcolor}\selectfont}
\makeatother

%
% new line
%

\newcommand{\nl}[0]{\vspace{0.4cm}}

%
% title
%

\title{Development Environment}

%
% authors
%

\author
{
  Julien~Quintard\inst{1} \\
  {\tiny julien.quintard@gmail.com}
}

\institute
{
  \inst{1} kaneton distributed operating system project
}

%
% date
%

\date{\today}

%
% logos
%

\pgfdeclareimage[interpolate=true,width=34pt,height=18pt]
                {epita}{../../logos/epita}
\pgfdeclareimage[interpolate=true,width=49pt,height=18pt]
                {upmc}{../../logos/upmc}
\pgfdeclareimage[interpolate=true,width=25pt,height=18pt]
                {lse}{../../logos/lse}

%
% table of contents at the beginning of each section
%

\AtBeginSection[]
{
  \begin{frame}<beamer>
   \frametitle{Outline}
    \tableofcontents[current]
  \end{frame}
}

%
% table of contents at the beginning of each subsection
%

\AtBeginSubsection[]
{
  \begin{frame}<beamer>
   \frametitle{Outline}
    \tableofcontents[current,currentsubsection]
  \end{frame}
}

%
% document
%

\begin{document}

%
% title frame
%

\begin{frame}
  \titlepage

  \begin{center}
    \pgfuseimage{epita} \hspace{0.1cm} \pgfuseimage{upmc} \hspace{0.1cm}
    \pgfuseimage{lse} \hspace{0.1cm}
  \end{center}
\end{frame}

%
% outline frame
%

\begin{frame}
  \frametitle{Outline}
  \tableofcontents
\end{frame}

%
% overview
%

\section{Overview}

% 1)

\begin{frame}
  \frametitle{Introduction}

  From the previous years, a development environment was introduced.

  \nl

  The questions are:

  \begin{enumerate}[<+->]
    \item
      Why?
    \item
      What are the advantages and disadvantages of such a
      development environment?
    \item
      How did the other promotions do?
  \end{enumerate}
\end{frame}

% 2)

\begin{frame}
  \frametitle{Explanations}

  Over the years, the kaneton project evolved, starting with a very
  simple introduction to low-level programming, to microkernel
  development and finally to a distributed operating system project.

  \nl

  Going always further implies many modifications in the project
  including:

  \begin{itemize}[<+->]
    \item
      The courses given which now go from the Intel processor to
      the distributed operating system concepts
    \item
      The assignments which always evolve to study advanced topics
    \item
      The context because we now have to provide parts of the microkernel
      to avoid students a development from scratch
    \item
      .. and so the requirements
  \end{itemize}
\end{frame}

% 3)

\begin{frame}
  \frametitle{The Courses}

  The kaneton project now comes with four courses:

  \begin{enumerate}
    \item
      The design of the kaneton distributed operating system including
      the microkernel
    \item
      The Intel processor
    \item
      The kernel concepts
    \item
      The distributed operating system concepts
  \end{enumerate}
\end{frame}

% 4)

\begin{frame}
  \frametitle{The Assignments}

  During the year 2005, the students develop a poor microkernel
  from scratch with few functionalities, a driver and finally a baby
  file system.

  \nl

  We cannot ask the students of the year 2006 to develop the same project
  but to go further to study advanced topics like distributed algorithms.

  \nl

  So, we cannot ask the students to develop every parts of the microkernel
  because this takes much time and implies to not study advanced
  topics.
\end{frame}

% 5)

\begin{frame}
  \frametitle{The Context}

  Providing students parts of the microkernel is not enough.

  \nl

  Indeed, we decided to provide a complete development environment
  including:

  \begin{itemize}
    \item
      Makefiles
    \item
      Shell scripts
    \item
      Papers
    \item
      Tools
    \item
      .. everything you need to start microkernel development
  \end{itemize}
\end{frame}

% 6)

\begin{frame}
  \frametitle{Why?}

  The remaining question is:

  \nl

  \textbf{Why providing such a development environment and not letting us
    develop one ourself?}

  \nl

  The answers simply are:

  \begin{itemize}
    \item
      Developing such a development environment takes much time and
      need experience
    \item
      This development environment include very powerful features:
      multiusers cooperation, different operating systems etc..
    \item
      Finally, students will not be able to create such a complicated
      development tree so it is provided to not waste time.
  \end{itemize}
\end{frame}

% 7)

\begin{frame}
  \frametitle{The Requirements}

  The students starting the kaneton project should think that they
  will learn many many things during the year.

  \nl

  This year, we are trying to lead students to a distributed operating
  system.

  \nl

  This implies more concepts, algorithms and techniques to learn.

  \nl

  To do this we introduced more courses but the students will have
  to work hard to be able to success.
\end{frame}

% 8)

\begin{frame}[containsverbatim]
  \frametitle{Tree}

  \begin{center}

  \begin{verbatim}
    /
      conf/
      core/
      doc/
      drivers/
      env/
      export/
      libs/
      papers/
      programs/
      services/
      tools/
  \end{verbatim}

  \end{center}
\end{frame}

%
% conf
%

\section{conf}

% 1)

\begin{frame}
  \frametitle{Overview}

  The \textbf{conf} directory contains user variables used to parameterise:

  \begin{itemize}
    \item
      the development environment: makefiles, scripts etc..
    \item
      the kernel
  \end{itemize}

  \nl

  This configuration system is very interesting coupled with versionning
  system.

  \nl

  Indeed, you can develop using special compilation flags, specific kernel
  configuration without conflicts with other developers.
\end{frame}

% 2)

\begin{frame}[containsverbatim]
  \frametitle{Tree}

  \begin{verbatim}
    conf/
      mycure/
        conf.c
        conf.h
        kaneton.conf
        modules.conf
        mycure.conf
      pwipwi/
      chiche/
  \end{verbatim}

  This configuration system uses the shell variable \$USER to find
  the main configuration file: \textbf{conf/\$USER/\$USER.conf}.
\end{frame}

% 3)

\begin{frame}
  \frametitle{conf.c}

  This file is not used yet.
\end{frame}

% 4)

\begin{frame}
  \frametitle{conf.h}

  This file contains macros to configure the kernel:

  \begin{itemize}
    \item
      \textbf{CONF\_TITLE}
    \item
      \textbf{CONF\_VERSION}
    \item
      \textbf{CONF\_DEBUG}
    \item
      etc..
  \end{itemize}

  \nl

  This file is included by the kernel code.
\end{frame}

% 5)

\begin{frame}
  \frametitle{kaneton.conf}

  This configuration file is used to pass arguments at the runtime to the
  servers.

  \nl

  This file is also used to configure kernel and servers input variables.
\end{frame}

% 6)

\begin{frame}
  \frametitle{modules.conf}

  This file contains the list of the modules to be loaded by the
  multi-bootloader.

  \nl

  These modules will be passed to the kernel at the boot time.

  \nl

  Be careful, a module here is not a module in the Linux or BSD terms.

  \nl

  A module is simply a file to load.
\end{frame}

% 7)

\begin{frame}
  \frametitle{\$USER.conf}

  Finally the main configuration file contains the configuration
  variables for the development environment.

  \nl

  This file uses the syntax of the make files.

  \nl

  Every variable defined in this file will be used by the makefiles
  and the scripts.
\end{frame}

%
% env
%

\section{env}

% 1)

\begin{frame}
  \frametitle{Overview}

  The \textbf{env} directory contains the different development environments.

  \nl

  This directory is the heart of the kaneton development system.

  \nl

  Indeed, a user can develop the kaneton project on a Mac machine using
  cross compilation for Intel processors ('cause PowerPC processor)
  while another one is using a FreeBSD operating system on an Intel processor.

  \nl

  So, the development environment has to deal with these different operating
  systems and architectures just for the development.
\end{frame}

% 2)

\begin{frame}
  \frametitle{Our System}

  To do this, we decided to introduce an environment system.

  \nl

  Every time a user gets the kaneton development tarball, he first has to
  create his development environment given a couple operating system and
  architecture which leads to an environment.

  \nl

  Once the environment is installed, the user can develop, compile the kernel
  etc.. without problems because everything (makefiles, scripts etc..) use
  the binaries, variables etc.. for his environment.

  \nl

  The environment is specified in the user configuration file.
\end{frame}

% 3)

\begin{frame}[containsverbatim]
  \frametitle{Tree}

  \begin{verbatim}
    env/
      clean.sh
      init.sh
      unix/
        clean.sh
        init.sh
        kaneton.mk
      macos-powerpc.ia32/
  \end{verbatim}

  \nl

  Here the \textbf{unix} is considered as the generic unix
  environment but everyone can add a specific linux, FreeBSD, Solaris etc..
  environment.
\end{frame}

% 4)

\begin{frame}
  \frametitle{init.sh}

  The \textbf{init.sh} shell script is used to install the development
  environment.

  \nl

  This script first gets the configuration variables from the user
  configuration file, then calls the specific \textbf{init.sh} script
  of the given environment.

  \nl

  Finally the script installs some links and initialises the makefile
  dependencies.

  \nl

  The \textit{[environment]}/init.sh shell script is used to install
  specific stuff.
\end{frame}

% 5)

\begin{frame}
  \frametitle{clean.sh}

  The \textbf{clean.sh} shell script just cleans the environment.

  \nl

  This shell script also call the environment specific clean.sh script.
\end{frame}

% 6)

\begin{frame}
  \frametitle{kaneton.mk}

  The \textbf{kaneton.mk} makefile dependency is the heart of the
  kaneton compilation system.

  \nl

  Indeed, every makefile is composed of calls to special routines
  which are implemented by the makefile dependency depending on the
  environment: operating system plus architecture source and destination.

  \nl

  Moreover the \textbf{kaneton.mk} makefile dependency includes the
  user configuration file so each makefile of the system is able to
  use user defined variables.

  \nl

  The kaneton compilation system uses a very special gmake feature:
  the makefile \textbf{call} function.
\end{frame}

% 7)

\begin{frame}[containsverbatim]
  \frametitle{Use}

  \begin{verbatim}
    $ make init
    [+] installing environment

    [+] your current configuration:
    [+]   environment:              unix
    [+]   architecture:             ia32
    [+]   multi-bootloader:         grub

    [...]

    $ make clean
    [+] cleaning environment

    [...]

    $ 
  \end{verbatim}
\end{frame}

%
% tools
%

\section{tools}

% 1)

\begin{frame}
  \frametitle{Overview}

  The \textbf{tools} directory contains programs, scripts, special
  files used by the kaneton project.

  \nl

  For example a script to initialise and install modules on a grub
  bootloader boot device is provided in the subdirectory
  \textit{scripts/multi-bootloaders/grub/}.

  \nl

  The \textbf{tools} directory also contains the ld scripts used
  to correctly compile the bootstrap, the bootloader, the kernel, the
  drivers, the services and the programs.
\end{frame}

% 2)

\begin{frame}[containsverbatim]
  \frametitle{Tree}

  \begin{verbatim}
    tools/
      scripts/
        ld/
          arch/
            ia32/
              bootstrap.lds
              bootloader.lds
              kaneton.lds
              driver.lds
              service.lds
              user.lds
        multi-bootloaders/
          grub/
          lilo/
        prototypes/
          mkp.py
  \end{verbatim}
\end{frame}

% 3)

\begin{frame}[containsverbatim]
  \frametitle{Use}

  \begin{verbatim}
    $ make build
    [+] initialising boot system

    [+] boot system initialised successfully
    $ make install
    [+] initialising boot system

    [+] /tmp/menu.lst
    [+] core/bootloader/bootloader
    [+] core/kaneton/kaneton
    [+] conf/mycure/kaneton.conf
    [+] drivers/cons/cons
    [+] services/dsh/dsh

    [+] boot system initialised successfully
    $ 
  \end{verbatim}
\end{frame}

% 4)

\begin{frame}[containsverbatim]
  \frametitle{Prototypes}

  The compilation system permits to generate the prototypes in a very easy
  and elegant way.

  \begin{verbatim}
    $ make proto
    [PROTOTYPES]            libdata.h
    [PROTOTYPES]            libstring.h
    [PROTOTYPES]            libsys.h
    [PROTOTYPES]            bootloader.h
    [PROTOTYPES]            ia32.h
    [PROTOTYPES]            kaneton.h
    [PROTOTYPES]            as.h
    [PROTOTYPES]            conf.h
    [PROTOTYPES]            serial.h

    [...]

    $ 
  \end{verbatim}
\end{frame}

% 5)

\begin{frame}[containsverbatim]
  \frametitle{Explanations}

  This system is based on tags in the header files which specify
  from which files to extract prototypes.

  \nl

  The tags are of the form:

  \begin{verbatim}
    /*
     * ---------- prototypes -------------------------------------------------
     *
     *      ../../kaneton/set/set.c
     *      ../../kaneton/set/set_array.c
     *      ../../kaneton/set/set_ll.c
     *      ../../kaneton/set/set_bpt.c
     */
  \end{verbatim}
\end{frame}

% 5)

\begin{frame}[containsverbatim]
  \frametitle{Dependencies}

  The compilation system uses full dependencies between files.

  \nl

  To regenerate the dependencies, for example when adding a
  \textit{\#include} c-preprocessor directive in a source file:

  \begin{verbatim}
    $ make dep
    [REMOVE]                .makefile.mk
    [DEPENDENCIES]          dump.c
    [DEPENDENCIES]          alloc.c
    [DEPENDENCIES]          sum2.c

    [...]

    $ 
  \end{verbatim}
\end{frame}

%
% libs
%

\section{libs}

% 1)

\begin{frame}
  \frametitle{Overview}

  The \textbf{libs} directory contains the libraries used by the kaneton
  project like:

  \begin{itemize}
    \item
      libc
    \item
      crt
    \item
      libposix
    \item
      etc..
  \end{itemize}
\end{frame}

%
% core
%

\section{core}

% 1)

\begin{frame}
  \frametitle{Overview}

  The \textbf{core} directory contains the source code for the microkernel
  including the bootstrap, the bootloader and the kernel itsef.

  \nl

  Each part contains an \textbf{arch} directory used for architecture
  dependent soure code.
\end{frame}

% 2)

\begin{frame}[containsverbatim]
  \frametitle{Tree}

  \begin{verbatim}
    core/
      bootstrap/
        arch/
          ia32/ <---;
          machdep --+
      bootloader/
        arch/
      kaneton/
        arch/
        as/
        conf/
        debug/
        id/
        segment/
        set/
        stats/
  \end{verbatim}
\end{frame}

%
% drivers
%

\section{drivers}

% 1)

\begin{frame}
  \frametitle{Overview}

  The \textbf{drivers} directory contains the drivers of the kaneton
  microkernel.

  \nl

  A driver, in the kaneton terms, is a microkernel server which is allowed
  to communicate with hardware devices.
\end{frame}

% 2)

\begin{frame}[containsverbatim]
  \frametitle{Tree}

  \begin{verbatim}
    drivers/
      cons/
        Makefile
        cons.c
      dma/
      kbd/
      ide/
  \end{verbatim}
\end{frame}

%
% services
%

\section{services}

% 1)

\begin{frame}
  \frametitle{Overview}

  The \textbf{services} directory contains the services of the kaneton
  microkernel.

  \nl

  A service, in the kaneton terms, in simply a server which does not
  communicate with the hardware.
\end{frame}

% 2)

\begin{frame}[containsverbatim]
  \frametitle{Tree}

  \begin{verbatim}
    services/
      dsh/
      mod/
        Makefile
        mod.c
        modfs.c
  \end{verbatim}
\end{frame}

%
% programs
%

\section{programs}

% 1)

\begin{frame}
  \frametitle{Overview}

  The \textbf{programs} directory contains the sources of common
  programs.

  \nl

  A program in the kaneton terms is just a non-privilegied
  process.
\end{frame}

% 2)

\begin{frame}[containsverbatim]
  \frametitle{Tree}

  \begin{verbatim}
    programs/
      ls/
      wc/
      cat/
      mount/
      umount/
      gcc/
      emacs/
  \end{verbatim}
\end{frame}

%
% export
%

\section{export}

% 1)

\begin{frame}
  \frametitle{Overview}

  The \textbf{export} directory is used to create kaneton distribution.

  \nl

  This feature is especially used by the maintainers of the kaneton
  project which create very special kaneton distribution for
  the students.
\end{frame}

% 2)

\begin{frame}[containsverbatim]
  \frametitle{Use}

  The only way to export kaneton is to do like this:

  \begin{verbatim}
    $ make export
    [!] usage: exporter.sh [stage]

    available stages: k0 k1 k2 k3 k4 k5 k6 k7 k8 k9 kaneton dist
    $ make export-k3
  \end{verbatim}

  \begin{itemize}
    \item
      \textbf{k[0-9]}: create a special kaneton version for the k[0-9]
      subproject
    \item
      \textbf{kaneton}: create an entire kaneton version for the lastest
      subproject
    \item
      \textbf{dist}: create an entire backup of the kaneton development
      project
  \end{itemize}
\end{frame}

%
% papers
%

\section{papers}

% 1)

\begin{frame}
  \frametitle{Overview}

  The \textbf{papers} directory contains the papers and lectures
  in relation with the kaneton project.

  \nl

  We prefered set the papers directly into the tarball so every student
  can easily read them.
\end{frame}

% 2)

\begin{frame}[containsverbatim]
  \frametitle{Tree}

  \begin{verbatim}
    papers/
      assignments/
      design/
      kaneton/
      seminar/
      lectures/
        kernels/
        inline-assembly/
        c-preprocessor/
        distributed-operating-systems/
        arch-ia32/
  \end{verbatim}
\end{frame}

% 3)

\begin{frame}[containsverbatim]
  \frametitle{Use}

  \begin{verbatim}
    $ make view
    [+] papers:

    [+]   assignments
    [+]   design
    [+]   arch-ia32
    [+]   c-preprocessor
    [+]   distributed-operating-systems
    [+]   inline-assembly
    [+]   kernels
    [+]   development-environment

    [!] usage: viewer.sh [paper]
    $ make view-design
  \end{verbatim}
\end{frame}

%
% doc
%

\section{doc}

% 1)

\begin{frame}
  \frametitle{Overview}

  The \textbf{doc} directory contains every document useful for
  the development of the kaneton project.

  \nl

  This directory will theorically contain documents on the different
  architectures, documents on some hardware devices like ide, usb etc..
\end{frame}

\end{document}

%
% ---------- header -----------------------------------------------------------
%
% project       kaneton
%
% license       kaneton
%
% file          /home/mycure/kaneton/view/book/development/source-tree.tex
%
% created       julien quintard   [thu may 17 22:41:36 2007]
% updated       julien quintard   [thu may 31 08:34:23 2007]
%

%
% ---------- source tree ------------------------------------------------------
%

\chapter{Source Tree}
\label{chapter:source tree}

In this chapter we will briefly describe the kaneton microkernel project
source tree.

\newpage

%
% ---------- text -------------------------------------------------------------
%

The kaneton microkernel reference source tree looks like the following
listing:

\begin{verbatim}
cheat/
configure/
environment/
export/
history/
kaneton/
library/
license/
test/
tool/
transcript/
view/
\end{verbatim}

%
% cheat/
%

\subsection*{cheat/}

Since the kaneton microkernel is implemented by students, the kaneton
people need to check whether students are cheating by re-using parts of
previous years projects or other kernel source codes available on the
\textit{Internet}.

To avoid cheating, kaneton people developed a software checking for
commonalities between different source codes.

This directory contains scripts that performs these verifications. However,
the students work over the years are not stored in this directory but in
the \textit{history/} directory instead.

%
% configure/
%

\subsection*{configure/}

This directory contains everything necessary for configuring its own
kaneton microkernel development environment through the compiling process
to the boot system.

Any new contributor should first look at this directory. However, note that
this directory mainly contains tools targeting final-users rather than
kaneton contributors. Indeed, for instance, the \textit{configure} utility
aims at providing a user-friendly way for configuration but does not take
advantage of the power of the kaneton development environment.

Contributors should then learn about how the development environment works
while final-users should use the \textit{configure} tool.

%
% environment/
%

\subsection*{environment/}

This directory contains everything necessary to the kaneton development
environment.

The kaneton development environment allows different developers to
interact on the development of the same microkernel in a pretty easy way.

The development environment aims at providing developers to possibility to
work in a collaborative manner without interfering with each other. These
developers are likely to run different operating systems on different
microprocessors. In addition, the kaneton microkernel can be targeted for
different microprocessor architectures. The development environment was
introduced to cope with these combinations by providing profiles, each
profile describing the behaviour of a component: underlying operating system,
target architecture, user-specific stuff etc.

As a result, each developer can use a different operating system and
microprocessor architecture with its own specific compiling flags, kaneton
parameters etc. without modifying another developer's configuration.

The development environment is detailed in \textit{Section
\ref{section:environment}}.

%
% export/
%

\subsection*{export/}

The \textit{export/} directory contains scripts used to generate a kaneton
tarball in order to be distributed to the students at the beginning of the
kaneton educational project.

Indeed, these scripts rearrange the kaneton hierarchy hidding some important
directories the students do not need to be aware of. Moreover some source
code parts are removed since the students have to rewrite these pieces
of code as assignments.

These scripts are also used for making backups and distribution tarbalss of
the kaneton microkernel.

%
% history/
%

\subsection*{history/}

The \textit{history/} directory contains the students work over the years
in the universities and schools the kaneton project was used as an operating
system course's implementation material.

The tools of the \textit{cheat/} directory use these students works for
performing cheating verifications.

%
% kaneton/
%

\subsection*{kaneton/}

This directory is the most important of the project since it contains
the whole microkernel source code.

The directory is composed of three important subdirectories: \textit{core/},
\textit{platform/} and \textit{architecture/}. These subdirectories are
described next.

% core/

\subsubsection*{core/}

This directory contains the kaneton core source code.

The directory is divided as shown below:

\begin{verbatim}
as/
region/
sched/
segment/
set/
task/
thread/
[...]
\end{verbatim}

Each directory represents a kaneton core manager. For more information on
the kaneton core, please refer to the appropriate document:
\textit{The kaneton microkernel :: core}

% platform/

\subsubsection*{platform/}

This directory contains everything in relation with what the kaneton
microkernel project calls a \textit{platform}. The platform represents the
board supporting the devices: microprocessor, memory, peripherals etc.

This directory obviously contains subdirectories for each platform
supported by the kaneton microkernel.

% architecture/

\subsubsection*{architecture/}

The \textit{architecture/} directory contains the source-code related to
the microprocessor architectures supported by the kaneton microkernel.

This directory is composed of subdirectories, each one representing a
supported architecture: \textit{ia32}, \textit{mips64} etc. Note that each
architecture can be specialised. For instance, the \textit{ia32/optimised}
architecture represents an optimised implementation of the \textit{Intel IA-32}
microprocessor architecture.

%
% library/
%

\subsection*{library/}

This directory contains the libraries used by the kaneton microkernel itself,
the kaneton microkernel servers or maybe both. This directory especially
contains the standard \textit{kaneton C library}.

%
% license/
%

\subsection*{license/}

This directory contains the licenses used for any program or document
in relation with the kaneton microkernel project. Indeed, the kaneton
microkernel is under the \textit{kaneton license} which is described in
depth in the documents contained in this directory. Note that these licenses
are also available in \textit{Chapter \ref{chapter:licenses}}.

Each student has to read and agree with the kaneton license before
implementing or even using the kaneton microkernel project..

Indeed, every user of the kaneton-related stuff is considered as having
implicitly accepted the kaneton license.

%
% test/
%

\subsection*{test/}

Since the kaneton microkernel is used as a material for operating system
courses, the kaneton microkernel reference, which is the basis of students
work, must be extremely reliable.

The kaneton project therefore contains a set of tools in order to validate
the kaneton reference implementation behaviour. These tools are also used
for evaluating the correctness of the students implementation.

The \textit{test/} directory contains the set of kaneton scripts and tests
for validating a kaneton microkernel implementation.

%
% tool/
%

\subsection*{tool/}

This directory contains additional scripts and configuration files used by
the kaneton development environment or the kaneton developers.

As examples, this directory contains scripts for generating prototypes,
building a boot device etc.

%
% transcript/
%

\subsection*{transcript/}

This directory contains real-time recorded sessions. These sessions can be
replayed in order to present a feature of the development environment or
of the kaneton microkernel.

%
% view/
%

\subsection*{view/}

This directory contains all the kaneton documents including kaneton
administrative documents, examinations, lectures materials, kaneton papers
and books etc.

Additionally, scripts are provided in order to very easily build and
display these documents.
%%
%% licence       kaneton licence
%%
%% project       kaneton
%%
%% file          /home/mycure/kaneton/view/papers/kaneton/coding-style.tex
%%
%% created       matthieu bucchianeri   [mon jan 30 17:32:57 2006]
%% updated       julien quintard   [thu mar  2 13:57:17 2006]
%%

%
% coding style
%

\chapter{Coding style}

The kaneton project developers try to follow a coding style. This
coding style was introduced to normalize the source code, leading to a
more readable source code.

Nevertheless, you can adapt this coding style to your own but try to
follow the rules.

%
% case
%

\section{Case}

The whole kaneton source code is written using lower case letters.

Moreover, every text including comments etc.. must be written using
lower case letters

%
% headers
%

\section{Headers}

Each file must start with an header formatted as shown below:

\begin{verbatim}
/*
 * licence       kaneton licence
 *
 * project       kaneton
 *
 * file          /home/mycure/kaneton/core/kaneton/as/as.c
 *
 * created       julien quintard   [fri feb 11 02:23:41 2005]
 * updated       matthieu bucchianeri   [mon jan 30 20:30:57 2006]
 */
\end{verbatim}

An emacs configuration file for automatically generating and updating
this header can be found in \textit{tools/emacs}.

Additionally, you need to set two environment variables to generate
a correct kaneton header:

\begin{itemize}
  \item
    \textbf{EC\_LICENCE} must be set to ``kaneton licence''.
  \item
    \textbf{EC\_DEVELOPER} must be set to your first name and last name.
\end{itemize}

Please, do not use nicknames in headers.

%
% naming convention
%

\section{Naming Convenions}

To keep the code as clear as possible, there are several conventions on
types, functions and variables naming.

%
% variables
%

\subsection{Variables}

Here are a few rules you are encouraged to follow:

\begin{itemize}
  \item
    \textbf{sz} suffix for variables representing a size.

    \begin{verbatim}
      #define PAGESZ          4096

      int                     modsz;
    \end{verbatim}
  \item
    \textbf{n} prefix for variables representing a number of objects.

    \begin{verbatim}
      int                     nclusters;
    \end{verbatim}
  \item
    etc..
\end{itemize}

Moreover, the types are used as pre-names:

\begin{verbatim}
t_vaddr                 video_vaddr;
\end{verbatim}

This example is not correct, instead prefer:

\begin{verbatim}
t_vaddr                 video;
\end{verbatim}

%
% functions
%

\subsection{Functions}

Function names must be prefixed by the file name, context name they are
implemented in.

For example, a function part of the address space manager must be prefixed
by \textit{as\_}.

These names must be chosen carefully: they must explicitely define
what the function does without being too long.

%
% types
%

\subsection{Types}

As variables and functions, type names must be expressed in english
with lower case letters.

Here are the prefixes you must use when writing your own types:

\begin{itemize}
  \item
    \textbf{m\_} for managers main structures.
  \item
    \textbf{o\_} for kaneton objects.
  \item
    \textbf{i\_} for interfaces.
  \item
    \textbf{d\_} for architecture-dependent structures.
  \item
    \textbf{s\_} for general purpose structures.
  \item
    \textbf{t\_} for basic and general purpose typedefs.
  \item
    \textbf{c\_} for kaneton capabilities.
  \item
    \textbf{u\_} for kaneton unique identifiers.
\end{itemize}

Notice that \textbf{d\_} can be combined with other prefixes, for
example \textbf{do\_} for a dependent object.

%
% includes
%

\section{Includes}

To keep the code clear and compact, developers only need to include a
minimal number of header files:

\begin{itemize}
  \item
    \textbf{kaneton.h} for the microkernel declarations.
  \item
    \textbf{klibc.h} for the kaneton specific C library.
\end{itemize}

These files are located in the include path, so do not use relative include
path.

\begin{verbatim}
#include <libc.h>
#include <kaneton.h>

int             main(int                argc,
                     char**             argv)
{
  [...]

  return (0);
}
\end{verbatim}

All include files must be protected against multiple inclusions. The
guard macro to use must be named using the directory name, one underscore,
the file name, one underscore and a capital ``H''.

For example, the file \textit{core/include/kaneton/segment.h} will be
guarded as follow:

\begin{verbatim}
#ifndef KANETON_SEGMENT_H
#define KANETON_SEGMENT_H	1

[...]

#endif
\end{verbatim}

In addition, for architecture-dependent files, the guard macro must begin
with the architecture name; for example for the Intel architecture:
\textit{IA32\_KANETON\_SEGMENT\_H}.

%
% types
%

\section{Types}

You may use as soon as possible standard types: \textbf{t\_uint8},
\textbf{t\_sint32}, \textbf{t\_uint64} etc..

This nomenclature is more understandable than
\textbf{unsigned long long int}.

%
% return values
%

\section{Return Values}

Every function must report whether it successed or failed.

In kaneton, functions' return type must be \textbf{t\_error}.

A function will return \textbf{ERROR\_NONE} on success and anything
else on error, for example \textbf{ERROR\_UNKNOWN} to indicate a non-specific
error.

%
% indentation
%

\section{Indentation}

There are several indentation rules in kaneton.

\begin{enumerate}
  \item
    Field names of structures and unions must be aligned with the
    structure or union name.

    \begin{verbatim}
      struct       s_set
      {
        u_set      id;
        t_setsz    size;
        t_type     type;
      };
    \end{verbatim}

    or

    \begin{verbatim}
      typedef struct
      {
        o_id       id;
        u_stats    stats;
        u_set      container;
      }            m_as;
    \end{verbatim}
  \item
    Macros and variables must be aligned as shown below:

    \begin{verbatim}
      #define TASK_PRIOR_CORE     230
      #define TASK_HPRIOR_CORE    250
      #define TASK_LPRIOR_CORE    210

      m_task*                     task;
      u_task                      ktask = ID_UNUSED;
    \end{verbatim}

    This rule also applies for variables declarations in functions.
  \item
    Function prototypes and bodies should look like this:

    \begin{verbatim}
      t_error             stats_function(u_stats          id,
                                         char*            function,
                                         t_stats_func**   f)
      {
        t_sint64          slot = -1;
        t_sint64          i;

        [...]
      }
    \end{verbatim}

    Notice that argument names are aligned between each other,
    and variable names are aligned with function name and between
    each other.

    Try to respect this alignment between functions in a single file:
    function names may be all aligned and argument names also.
\end{enumerate}

%
% comments
%

\section{Comments}

As kaneton is intended to be a pedagogical project with clear and
understandable source code; no need to say that comments take a very
important part of this objective.

Every file must begin with a comment describing what is done in this
code via the \textit{information} section.

Moreover, every function must be preceded by a comment defining its
behavior.

For complex functions and yo prevent direct comments in the source code,
we used \textbf{steps}:

\begin{itemize}
  \item
    Each critical code section in a function is preceded by a step
    number.
  \item
    The function header comment contains steps descriptions.
\end{itemize}

An example is present below:

\begin{verbatim}
/*
 * this function shows the usage of comments and steps.
 *
 * steps:
 *
 * 1) compute the index.
 * 2) make the operation.
 * 3) check the result.
 */

t_error         test_foobar(int      a,
                            int      b,
                            int*     c)
{
  int           index;

  /*
   * 1)
   */

  index = text_make_index(a, b);

  /*
   * 2)
   */

  index = index * a + b;

  /*
   * 3)
   */

  if (index < 0)
    return (ERROR_UNKNOWN);

  *c = index;

  return (ERROR_NONE);
}
\end{verbatim}

%
% sections
%

\section{Sections}

kaneton files are divided in multiple sections.

Section are delimited as shown below:

\begin{verbatim}
/*
 * ---------- includes ------------------------------------------------
 */
\end{verbatim}

Possible sections in a file are:

\begin{itemize}
  \item
    \textbf{header files}: information, dependencies, defines, types,
    prototypes, macros, etc..
  \item
    \textbf{source files}: information, extern, globals, includes,
    functions, etc..
  \item
    \textbf{make files}: dependencies, directives, variables, rules, etc..
\end{itemize}

Moreover, every important file, for example the main file of each
kaneton manager, have to contain a section \textit{information} describing
the whole manager.

In addition, a section named \textit{assignments} is generally necessary
for manager will be filled in by the students. This section briefly describes
the work to be done by the students.

%
% macros
%

\section{Macros}

The kaneton microkernel uses few fundamental macros lited below:

\begin{itemize}
  \item
    \textbf{\_\_\_bootloader} indicates that this source code belongs to
    the bootloader.
  \item
    \textbf{\_\_\_kernel} indicates that this source code belongs to the
    microkernel.
  \item
    \textbf{\_\_\_kaneton} indicates that this kernel is the kaneton
    microkernel.
  \item
    \textbf{\_\_\_wordsz} indicates the word size: 16-bit, 32-bit,
    64-bit etc..
  \item
    \textbf{\_\_\_endian} indicates the endianness.
\end{itemize}

%%
%% licence       kaneton licence
%%
%% project       kaneton
%%
%% file          /home/mycure/kaneton/view/papers/kaneton/core.tex
%%
%% created       matthieu bucchianeri   [mon jan 30 17:33:29 2006]
%% updated       julien quintard   [fri mar 10 01:50:49 2006]
%%

%
% core
%

\chapter{Core}

\newpage

%
% text
%

%
% bootstrap
%

\section{Bootstrap}

XXX

%
% bootloader
%

\section{Bootloader}

XXX

%
% kaneton
%

\section{kaneton}

XXX

%
% id
%

\section{id}

The following rules describes a typical use of id objects in kaneton:

\begin{itemize}
  \item
    The \textbf{init} function of a manager calls \textbf{id\_build}
    to generate a \textbf{o\_id} object.
  \item
    Functions generating new objects will use \textbf{id\_reserve} to
    reserve new identifiers for the created objects.
  \item
    Functions removing objets will call \textbf{id\_destroy} to release
    identifiers of destroyed objets.
  \item
    The \textbf{clean} function of a manager will release the identifier
    generator with \textbf{id\_release}.
\end{itemize}

%
% ---------- header -----------------------------------------------------------
%
% project       kaneton
%
% license       kaneton
%
% file          /home/mycure/kaneton/view/book/development/tools.tex
%
% created       julien quintard   [sun may 20 14:48:11 2007]
% updated       julien quintard   [thu may 31 06:46:06 2007]
%

%
% ---------- tools ------------------------------------------------------------
%

\chapter{Tools}
\label{chapter:tools}

This chapter describes every tool kaneton contributors use on a daily-basis.

\newpage

%
% ---------- text -------------------------------------------------------------
%

%
% internal
%

\section{Internal}

The kaneton project contains several tools which makes the developer's life
easier. This section describes these tools in order for the contributor to
use it but also to improve them.

% environment

%
% ---------- header -----------------------------------------------------------
%
% project       kaneton
%
% license       kaneton
%
% file          /home/mycure/kaneton/view/book/development/environment.tex
%
% created       julien quintard   [sun may 20 14:49:26 2007]
% updated       julien quintard   [sun may 20 18:07:59 2007]
%

%
% ---------- environment ------------------------------------------------------
%

\subsection{Environment}

Over the years, the kaneton microkernel evolved, starting with a very simple
introduction to low-level programming and finally to a complete microkernel
development.

kaneton people wanted to lead students to a complete microkernel development
to finally introduce distributed computing. This would not have been possible
if students had to build an entire development environment because developing
such an environment is a whole project by itself.

As a result, kaneton people decided to provide students a complete development
environment. The kaneton development environment is composed of make files,
python scripts and configuration files. This development environment can be
considered as one of the major kaneton tools since contributors use it
everytime.

The kaneton development environment aims at providing an easy and portable
way for managing the kaneton microkernel project from a development point
of view. Therefore, the kaneton environment provides everything necessary
for compiling, assembling, etc.. These tasks highly rely on the underlying
running operating system as well as on the kaneton microkernel's target
microprocessor. Moreover, the user could need to redefine some behaviours
depending on its personal operating system configuration to use a specific
C compiler for instance.

The kaneton development environment provides a layered organisation of
profiles, each profile defining variables and functions used by the final
environment engine. The goal of the layered model is to allow layers to
override the definitions of lower layers.

%
% profiles
%

\subsubsection{Profiles}

A configuration is composed of profiles including a \textit{host} profile which
describes the behaviour of the underlying operating system, a \textit{kaneton}
profile which parameters the kaneton core and a \textit{user} profile which
permits the user to redefine lower layers' declarations.

These profiles eventually hold sub-profiles which actually define variables
and functions. These actual profiles are accessed according to user-defined
shell variables.

% host

\subsubsubsection{Host}

The \textit{host} profile essentially describes how to perform basic tasks:
compile, assemble, change the current directory, display a message etc.. These
tasks rely on the operating system currently running as well as on the target
processor which kaneton will be built for. For these reasons, there are
several host sub-profiles.

Let us suppose a developer is running a \textit{Linux} operating system and
that kaneton will be built for running on a \textit{PowerPC} microprocessor. In
such a case, the C compiler program will be different depending on the
microprocessor \textit{Linux} is running on. Indeed, if Linux is running on
a \textit{PowerPC} microprocessor, then using the default compiler should
produce \textit{PowerPC} object files. This is well-known to be the common
compiling way. On the other hand, if \textit{Linux} is running on a
different microprocessor, then a cross-compiler must be used to produce
binary objects targeting a specific different microprocessor architecture.

To avoid this issue, a \textit{host} sub-profile name is composed of two parts
separated by a slash. The first part is the name of the operating system and
the latter is a pair source/target processors separated by a period. For
example, \textit{linux/ia32.ppc} names a \textit{host} profile running the
\textit{Linux} operating system on a \textit{Intel 32-bit} microprocessor
which aims at building a kaneton microkernel for the \textit{PowerPC}
target architecture. Needless to say that \textit{linux/ia32.ia32} represents
a non cross-compiling environment.

To avoid configuration duplications, it is common to see the configuration
file of a host sub-profile to include files of the parent directory as
shown below:

\begin{verbatim}
  linux/
    linux.desc
    linux.conf
    linux.mk
    linux.py
    ia32.ia32/
      virtual -> .
      optimised -> .
      smp -> .
      ia32.desc
      ia32.conf
      ia32.mk
      ia32.py
    ia32.mips64/
      mips64.desc
      mips64.conf
      mips64.mk
      mips64.py
\end{verbatim}

Note that the files \textit{linux.*} are not directly included by the
development environment engine since \textit{linux} is not a valid host
profile name.

Two host profiles are illustrated here. The first one is named
\textit{linux/ia32.ia32} while the second's name is \textit{linux/ia32.mips64}.

For example, the \textit{linux/ia32.mips64} \textit{host} profile represents a
\textit{Linux} operating system running on a \textit{Intel 32-bit}
microprocessor while kaneton is built for a \textit{MIPS 64-bit} target
architecture. This profile is likely to include the \textit{linux.*} of the
parent directory since there are not much difference between all the
\textit{linux/*.*} \textit{host} profiles. However, such a profile will
certainly redefine the binary paths of the C compiler, linker etc.. in order
to produce \textit{MIPS 64-bit} binary objects.

To conclude, the \textit{host} sub-profile is accessed by the following
construct:

\begin{verbatim}
  profile/host/${KANETON_HOST}/${KANETON_ARCHITECTURE}
\end{verbatim}

With, for instance, the following values:

\begin{verbatim}
  KANETON_HOST = linux/ia32
  KANETON_ARCHTECTURE = ia32/virtual
\end{verbatim}

Note that the possibility to include files in the configuration syntax allows
very similar profiles to share a huge amount of definitions.

% kaneton

\subsubsubsection{kaneton}

The \textit{kaneton} profile is composed of three sub-profiles: \textit{core},
\textit{platform} and \textit{architecture}.

The \textit{core} sub-profile contains variables for parameterizing the
kaneton core. The \textit{platform} and \textit{architecture} sub-profiles
focus on the configuration of the platform- and architecture-dependent code
of the kaneton microkernel.

The user-defined shell variables \textit{\$\{KANETON\_PLATFORM\}} and
\textit{\$\{KANETON\_ARCHITECTURE\}} are used to address the \textit{platform}
and \textit{architecture} sub-profiles, respectively.

% user

\subsubsubsection{User}

Let us suppose that a developer would like the kaneton microkernel to
use a specific memory management entirely based on a \textit{Slab Allocator}
and with all microprocessor optimisations enabled. These user-specific
configurations are actually allowed by the \textit{user} profile.

The user-defined shell variable \textit{\$\{KANETON\_USER\}} defines the name
of the \textit{user} profile. This profile contains user-specific
configurations allowing a contributor to overwrite lower layer defintions
in order to specialise a behaviour.

The kaneton project also provides a tool allowing developers to configure
their development environment. This tool is named \textbf{configure} and is
available from the kaneton project root directory.

%
% requirements
%

\subsubsection{Requirements}

The whole kaneton development environment needs exactly two fundamental tools
to work. The first one is \textit{GNU make}, used to build powerful make files,
and the second one is \textit{Python}, used to write portable scripts. If an
operating system has these two tools, then kaneton can certainly be developed
on it.

As said previously, the user has to specify some shell variables which are
used by the kaneton development environment engine. These variables are
described below:

\begin{itemize}
  \item
    \textbf{\$\{KANETON\_USER\}}: the name of the kaneton developer.

    A \textit{user} profile name must be composed of the first name, a period
    and finally, the last name of the developer.
  \item
    \textbf{\$\{KANETON\_HOST\}}: the name of the host which is composed of
    a couple operating system/microprocessor.
  \item
    \textbf{\$\{KANETON\_PYTHON\}} contains the path of the python binary.

    This path is necessary since the very first scripts which set up the
    configured environment are based on python scripts.
  \item
    \textbf{\$\{KANETON\_PLATFORM\}}: the name of the target platform.
  \item
    \textbf{\$\{KANETON\_ARCHITECTURE\}}: the name of the target architecture.
\end{itemize}

Note that once the configured environment is set up, these variables are
no longer used by the kaneton environment engine. Indeed, instead, the
kaneton environment operations are based on the \textit{host} profile on
which rely the configured environment.

The profiles names must all be lowercase. Below are some examples of what
could contain these variables:

\begin{verbatim}
  KANETON_USER='julien.quintard'

  KANETON_HOST='linux/ppc'
  KANETON_HOST='windows~cygwin/ia32'

  KANETON_PYTHON='/usr/bin/python'

  KANETON_PLATFORM='ibm-pc'
  KANETON_PLATFORM='sgi/o2'
  KANETON_PLATFORM='sgi/octane'

  KANETON_ARCHITECTURE='mips64'
  KANETON_ARCHITECTURE='ia32/virtual'
  KANETON_ARCHITECTURE='ia32/smp'
\end{verbatim}

%
% organisation
%

\subsubsection{Organisation}

The development environment configuration files and scripts are located in
the \textit{environment/} directory. The directory contains the three
following scripts:

\begin{verbatim}
  critical.py
  init.py
  clean.py
\end{verbatim}

The \textit{critical.py} script essentially generates a configured development
environment. The result of this generation are two files called
\textit{env.mk} and \textit{env.py} which contains the configured environment
variables and functions for the \textit{Make} files and \textit{Python}
scripts, respectively. This file is called critical because it does not rely
on the portable development environment as it generates it.

The \textit{init.py} script relies on the file \textit{env.py} previously
generated. This script set up everything necessary for building the
kaneton microkernel based on the configured environment.

Finally, the \textit{clean.py} script cleans everything installed by the
\textit{init.py} script and removes the generated configured environment files.

The generation of the configured environment is done by going through
the configuration files of all the profiles and sub-profiles associated
to the user configuration. In other words, the kaneton environment engine
processes the configuration files according to the layered organisation
described below, starting with the lowest layer thourgh the highest one.

\begin{verbatim}
  profile/
  profile/host
  profile/host/${KANETON_HOST}/${KANETON_ARCHITECTURE}
  profile/kaneton
  profile/kaneton/core
  profile/kaneton/platform
  profile/kaneton/platform/${KANETON_PLATFORM}
  profile/kaneton/architecture
  profile/kaneton/architecture/${KANETON_ARCHITECTURE}
  profile/user
  profile/user/${KANETON_USER}         
\end{verbatim}

\begin{verbatim}
XXX $ XXX
\end{verbatim}

In this layered organisation, a variable defined in, for instance, the
\textit{host} profile could be overwritten anywhere in the upper layers
\textit{kaneton}, \textit{kaneton/architecture/\$\{KANETON\_ARCHITECTURE\}},
\textit{user} etc..

The \textit{host} and \textit{kaneton} profiles are theoretically completed
separated. However, the environment engine does not check for such
unauthorised overridings. Therefore the \textit{core} configuration could
override a variable previously defined in the \textit{host} profile.

Finally, the \textit{user} profile can override any definition adjusting the
environment to his needs.

The environment engine looks for the following types of files in the
kaneton environment profile directories:

\begin{itemize}
  \item
    \textbf{.conf}: the \textit{configuration} files gathered by the
    development environment engine for generating the configured environment
    files.
  \item
    \textbf{.desc}: these \textit{description} files contain descriptions of
    the variables of the current profile or sub-profile. These descriptions
    are used by the \textit{configure} tool.
  \item
    \textbf{.mk}: the \textit{Make} files usually contains the implementation
    of the kaneton \textit{Make} interface.
  \item
    \textbf{.py}: the \textit{Python} files usually contains the
    implementation of the kaneton \textit{Python} interface.
\end{itemize}

The engine supposes that there is no variable or function overriding in
a single profile. more precisely, if there are more than a single
configuration file in a directory, the engine cannot guarantee anything
on the order these files will be processed. As a result, the overridings
could differ depending on the processing order.

The kaneton development environment engine first gathers the
\textit{configuration} files and process them creating an in-memory list of
configuration variables. Then it generates the configured environment files
\textit{env.mk} and \textit{env.py}. Indeed, the engine outputs the
configuration variables in each file and then append the content of the
\textit{Make} files and \textit{Python} files to the configured environment
file \textit{env.mk} and \textit{env.py}, respectively.

Note that the \textit{description} files are not directly used by the
environment engine.

%
% syntaxes
%

\subsubsection{Syntaxes}

% description

\subsubsubsection{Description}

The \textit{description} files describe the environment variables in order
to specify what kind of value a variable can take etc..

Each variable description is contained between braces. A description is
composed of fields, some are mandatory and some are optional.

Examples of description for variables named \textit{\_FOO\_}, \textit{\_BAR\_}
and \textit{\_CHICHE\_} are given next:

\begin{verbatim}
  _FOO_ :: the foo flag
  {
    <on> -D_FOO_FLAG_=1
    <off> -D_FOO_FLAG_=0

    This is a description of the two-state variable _FOO_.
  }

  _BAR_ :: the bar parameter
  {
    <simple> BAR_SIMPLE
    <normal> BAR_NORMAL
    <optimised> BAR_OPTIMISED

    This is another parameter which can take three values: simple,
    normal and optimised.
  }

  _CHICHE_ :: the most powerful optimisation
  {
    This is the magic kaneton optimisation.
  }
\end{verbatim}

Note that environment engine never takes these descriptions into account.
Indeed, this is the r\^ole of the \textit{configure} tool.

In this syntax, variables are classified according to the type of value
they can take: \textit{state}, \textit{set} and \textit{any}.

A \textit{state} variable is either activated or deactivated. If the two
fields \textit{<on>} and \textit{<off>} are present, then, this variable
is considered as a \textit{state} variable.

A \textit{set} variable can take any value of a given list of values. This
is the most common type of variables. In this case, each field detected
is considered as a potential value.

Finally, a \textit{any} variable represents a variable which can take any
value. This case is detected by the absence of value field in the description.

The value fields follow the next pattern:

\begin{verbatim}
  <name> value
\end{verbatim}

The \textit{name} is displayed by the \textit{configure} tool to the final
user while the \textit{value} value is affected to the described variable. This
way, the tool can displayed more human-readable description. For instance,
if the \textit{optimised} option is chosen, then the \textit{BAR\_OPTIMISED}
will be affected to the \textit{\_BAR\_} variable.

The name of the variable follows the pattern:

\begin{verbatim}
  variable :: name
\end{verbatim}

Once again, this construct was introduced to avoid displaying internal
non-user-friendly variable names. The \textit{variable} will not be directly
displayed by the \textit{configure} tool which will use the \textit{name}
string instead.

Finally, any remaining text between the braces is considered as a variable's
description text.

% configuration

\subsubsubsection{Configuration}

The \textit{configuration} files contains the actual variable definitions
through a very simple syntax.

The syntax allows both assignments and completion of variables' value
as show in the next example:

\begin{verbatim}
  FOO = bar
  FOO += baz
  FOO = kaneton
\end{verbatim}

The \textit{FOO} variable first took the initial value \textit{bar}. Then,
the value \textit{baz} was added to the previous \textit{FOO}'s value
leading the the value \textit{bar baz}. Finally, the last assignment
overwrite the previous definitions by setting the value of \textit{FOO}'s
variable to \textit{kaneton}.

The configuration syntax enables the use of variables in values. These
variables can be both environment variable or shell variable. The following
example illustrates it.

\begin{verbatim}
  BAR = ${FOO} is a very powerful microkernel
  SH = the shell currently used is $(SHELL)
\end{verbatim}

The reader certainly notice the \textit{\$\{\}} construct is used to reference
a kaneton environment variable while the \textit{\$()} one references a shell
variable.

Finally, a configuration file can also include another file using the
\textit{include} statement:

\begin{verbatim}
  include ../an/other/file/far/../far/../away
\end{verbatim}

This construct is very useful to centralize the definitions common to
multiple sub-profiles in a single location.

Note that kaneton environment variables start and end with an underscore
for avoiding naming collisions.

% make

\subsubsubsection{Make}

The \textit{Make} files must implement the whole kaneton \textit{Make}
interface which will be described next.

The syntax used in these files is based on the \textit{GNU Make} syntax.

% python

\subsubsubsection{Python}

The \textit{Python} files must implement the whole kaneton \textit{Python}
interface.

The syntax used in these files is based on the \textit{Python} syntax.

%
% interfaces
%

\subsubsection{Interfaces}

% make

\subsubsubsection{Make}

In this section we will detail the make interface that every host profile
must implement. The reader should look closer to the host profiles already
implemented.

Since the \textit{GNU Make} syntax does not provide any name space feature,
every kaneton \textit{Make} function is prefixed by \textit{env\_} in order
to avoid name conflicts.

\function{env\_display}{(\argument{color},
                         \argument{action},
                         \argument{file},
                         \argument{indentation},
                         \argument{options})}
         {
	   This function display a message representing an action performed
	   by the kaneton \textit{Make} interface.

	   \-

	   The option \textit{\$(OPTION\_NO\_NEWLINE)} can be used not to
	   output the trailing newline.
	 }

\function{env\_cd}{(\argument{directory},
                    \argument{options})}
         {
	   This function changes the current working directory.
	 }

\function{env\_contents}{(\argument{file},
                          \argument{options})}
         {
	   This function returns the contents of the file \argument{file}.
	 }

\function{env\_launch}{(\argument{file},
                        \argument{arguments},
                        \argument{options})}
         {
	   This function launches a new program/script/make etc..

	   \-

	   This function must look at the file name in order to determine
	   how to launch it.

	   \-

	   For \textit{Python} files, this function must take care of
	   setting and exporting the \textit{PYTHONPATH} shell environment
	   variable with a value including the
	   \textit{\_PYTHON\_INCLUDE\_DIR\_} kaneton environment variable.
	 }

\function{env\_preprocess}{(\argument{preprocessed file},
                            \argument{c file},
                            \argument{options})}
         {
	   This function launches the C preprocessor the \argument{c file}
	   and generates the \argument{preprocessed file}.
	 }

\function{env\_compile-c}{(\argument{object file},
                           \argument{c file},
                           \argument{options})}
         {
	   This function compile a \argument{c file} generating an
	   \argument{object file}.
	 }

\function{env\_lex-l}{(\argument{c file},
                       \argument{lex file},
                       \argument{options})}
         {
	   This function generates a \argument{c file} from a
	   \argument{lex file}.
	 }

\function{env\_assemble-S}{(\argument{object file},
                            \argument{S file},
                            \argument{options})}
         {
	   This function assemble an \argument{S file}.
	 }

\function{env\_assemble-asm}{(\argument{object file},
                              \argument{asm file},
                              \argument{options})}
         {
	   This function assemble an asm file.

	   \-

	   The option \textit{\$(ENV\_OUTPUT\_OBJECT)} forces the function
	   to generate an object file while the
	   \textit{\$(ENV\_OUTPUT\_BINARY)} option forces the output to be
	   a pure binary file.
	 }

\function{env\_static-library}{(\argument{static library file name},
                                \argument{object files and/or libraries},
                                \argument{options})}
         {
	   This function builds a static library from object files.
	 }

\function{env\_dynamic-library}{(\argument{dynamic library file name},
                                 \argument{object files and/or libraries},
                                 \argument{options})}
         {
	   This function builds a dynamic library from object files and/or
	   libraries.
	 }

\function{env\_executable}{(\argument{executable file name},
                            \argument{object files and/or libraries},
                            \argument{layout file},
                            \argument{options})}
         {
	   This function builds a executable from object files and/or
	   libraries. The \argument{layout file} describes where to
	   place the different data: code, read-only data, stack etc..

	   \-

	   The option \textit{\$(ENV\_OPTION\_NO\_STANDARD)} tells the function
	   not to use the operating system standard stuff: libraries, includes
	   etc..
	 }

\function{env\_archive}{(\argument{archive file name},
                         \argument{object files},
                         \argument{options})}
         {
	   This function builds an archive of object from multiple
	   \argument{object files}.
	 }

\function{env\_remove}{(\argument{files},
                        \argument{options})}
         {
	   This function removes the files in the list.
	 }

\function{env\_purge}{()}
         {
	   This function just cleans the current working directory from
	   unecessary files.
	 }

\function{env\_prototypes}{(\argument{files},
                            \argument{options})}
         {
	   This function generates prototypes in relation with the given
	   \argument{files}.
	 }

\function{env\_dependencies}{(\argument{files},
                              \argument{output},
                              \argument{options})}
         {
	   This function generates dependencies for the \argument{files}
	   by building a \textit{Make} dependency file named \argument{output}.
	 }

\function{env\_version}{(\argument{file})}
         {
	   This function generates a version \argument{file} from the operating
	   system's informations: user, host, date etc..
	 }

\function{env\_link}{(\argument{link},
                      \argument{file},
                      \argument{options})}
         {
	   This function creates a link \argument{link} to the \argument{file}.
	 }

\function{env\_compile-tex}{(\argument{file},
                             \argument{options})}
         {
	   This function compiles the file \argument{file}.tex and
	   will generate a readable document.
	 }

\function{env\_paper}{(\argument{file},
                       \argument{options})}
         {
	   This function builds a \textit{paper} by calling the
	   \textbf{env\_compile-tex()} function.
	 }

\function{env\_lecture}{(\argument{file},
                         \argument{options})}
         {
	   This function builds a \textit{lecture} document by calling the
	   \textbf{env\_compile-tex()} function.
	 }

\function{env\_subject}{(\argument{file},
                         \argument{options})}
         {
	   This function builds a \textit{subject} by calling the
	   \textbf{env\_compile-tex()} function.
	 }

\function{env\_correction}{(\argument{file},
                            \argument{options})}
         {
	   This function builds a \textit{correction} document by calling the
	   \textbf{env\_compile-tex()} function.
	 }

\function{env\_view}{(\argument{file},
                      \argument{options})}
         {
	   This function launches a viewer for the readable document
	   generated by the function \textbf{env\_compile-tex()}.
	 }

% python

\subsubsubsection{Python}

In this section we will detail the kaneton \textit{Python} interface that
every \textit{host} profile must implement.

The \textit{Python} language was designed in a portable way. For this
reason, the major part of the \textit{Python} interface is implemented
by the \textit{host} generic profile.

Note that the \textit{Python} language provides modularity through packages.
Therefore, each \textit{Python} script has to import the \textit{env} package
generated by the development environment engine. Then, environment functions
and variables are accessed through this package.

Below are described the functions implemented by the \textit{env} package.

\function{display}{(\argument{header},
                    \argument{text},
                    \argument{options})}
         {
	   This function outputs some text to the screen depending on the
	   header \textit{HEADER\_NONE}, \textit{HEADER\_OK},
	   \textit{HEADER\_ERROR}, \textit{HEADER\_INTERACTIVE}.
	 }

\function{contents}{(\argument{file},
                     \argument{options})}
         {
	   This function returns the contents of the \argument{file}.
	 }

\function{temporary}{(\argument{options})}
         {
	   This function creates a temporary file system object.

	   \-

	   The options \textit{OPTION\_FILE} and \textit{OPTION\_DIRECTORY}
	   specify which type of object to create.
	 }

\function{cwd}{(\argument{options})}
         {
	   This function returns the path of the current working directory.
	 }

\function{input}{(\argument{options})}
         {
	   This function waits for an input.
	 }

\function{launch}{(\argument{file},
                   \argument{arguments},
                   \argument{options})}
         {
	   This function launches a new program/script/make file etc..

	   \-

	   This function must look at the file name in order to determine
	   how to launch it.

	   \-

	   For \textit{Python} files, this function must take care of
	   setting and exporting the \textit{PYTHONPATH} shell environment
	   variable with a value including the
	   \textit{\_PYTHON\_INCLUDE\_DIR\_} kaneton environment variable.
	 }

\function{copy}{(\argument{source},
                 \argument{destination},
                 \argument{options})}
         {
	   This function copies the file \argument{source} to
	   \argument{destination}.
	 }

\function{link}{(\argument{source},
                 \argument{destination},
                 \argument{options})}
         {
	   This function builds a link between the file \argument{source}
	   and the file \argument{destination}.
	 }

\function{remove}{(\argument{target},
                   \argument{options})}
         {
	   This function removes the \argument{target} which can be either
	   a file or a directory.
	 }

\function{list}{(\argument{directory},
                 \argument{options})}
         {
	   This function lists the file system objects contains in the
	   \argument{directory}.

	   \-

	   The options \textit{OPTION\_FILE} and \textit{OPTION\_DIRECTORY}
	   specify which type of object to list.
	 }

\function{cd}{(\argument{directory},
               \argument{options})}
         {
	   This function changes the current working directory to
	   \argument{directory}.
	 }

\function{search}{(\argument{directory},
                   \argument{pattern},
                   \argument{options})}
         {
	   This function looks for files matching the given \argument{pattern}.

	   \-

	   The options \textit{OPTION\_FILE} and \textit{OPTION\_DIRECTORY}
	   specify which type of object to list while the
	   \textit{OPTION\_RECURSIVE} option tells the function to explore
	   the whole file system sub-tree.
	 }

\function{pack}{(\argument{directory},
                 \argument{file},
                 \argument{options})}
         {
	   This function makes an archive \argument{file} of the
	   directory \argument{directory}.
	 }

\function{unpack}{(\argument{directory},
                   \argument{file},
                   \argument{options})}
         {
	   This function extracts the archive \argument{file} into the
	   directory \argument{directory}, if specified.
	 }

\function{mkdir}{(\argument{directory},
                  \argument{options})}
         {
	   This function builds a new directory named \argument{directory}.
	 }

\function{load}{(\argument{file},
                 \argument{device},
                 \argument{path},
                 \argument{options})}
         {
	   This function copies the \argument{file} on the specificed
	   \argument{device}, more precisly at the location \argument{path}.
	   The device can be virtual: an image.

	   \-

	   The options \textit{OPTION\_DEVICE} and \textit{OPTION\_IMAGE}
	   specify on which type of device the file must be copied.
	 }

\function{stamp}{(\argument{format},
                  \argument{options})}
         {
	   This function returns a date following the given \argument{format}.
	 }

\function{record}{(\argument{log},
                   \argument{time},
                   \argument{options})}
         {
	   This function starts recording a session and outputs
	   the text into the file \argument{log} while the timings
	   are output in the file \argument{time}.
	 }

\function{play}{(\argument{log},
                 \argument{time},
                 \argument{options})}
         {
	   This function plays a previously recorded session where
	   the files \argument{log} and \argument{time} hold the
	   text and timings.
	 }

\function{locate}{(\argument{file},
                   \argument{options})}
         {
	   This function tries to locate the program \argument{file}
	   on the system.
	 }

\function{path}{(\argument{path},
                 \argument{options})}
         {
	   This function returns information on the given \argument{path}.

	   \-

	   The options \textit{OPTION\_FILE} and \textit{OPTION\_DIRECTORY}
	   specify which information the caller is interested in.
	 }


% configure

%
% ---------- header -----------------------------------------------------------
%
% project       kaneton
%
% license       kaneton
%
% file          /home/mycure/kaneton/view/book/development/configure.tex
%
% created       julien quintard   [tue may 22 22:34:37 2007]
% updated       julien quintard   [wed may 30 19:26:54 2007]
%

%
% ---------- configure --------------------------------------------------------
%

\subsection{Configure}

The \textit{configure} tool provides the final user a very user-friendly
software for customizing its development environment.

Recall the development environment is basically composed of three profiles:
\textit{host} which describes the operating system behaviour, \textit{kaneton}
which parameterizes the kaneton microkernel and \textit{user} which contains
some user-specific definitions.

The kaneton development environment is thus used to configure the environment
behaviour as well as the kaneton microkernel itself.

The \textit{environment/} directory, and more precisely the environment
profiles, contain \textit{description} files which actually describe the
environment variables. These files are not used by the development
environment but rather by the \textit{configure} tool.

The \textit{configure/} directory is composed of \textit{frame} files
which contain frame descriptions. A frame can be seen as a menu presented
to the final user. A frame is composed of sub-frame and variable entries.

The \textit{configure} tool works as follow. It starts by processing the
environment development configuration files as the environment engine did
for the generation of the configured environment files. Note that the
\textit{configure} tool also processes the description files. Also, it
focuses on variables and actually ignores the interfaces' functions.

Once this step is done, the tool gets a list of configured and fully described
variables. Then, the \textit{configure} tool displays the first frame and
waits for the user to choose an entry.

The user has the possibility to either move to another menu - if any sub-frame
entry is present - or configure a variable of the list. If the user chooses
to configure a variable, then, the tool displays information based on the
variable's description.

Every modifications of the development environment are private to the actual
user. Therefore, any variable modification adds or modifies an entry in the
related \textit{user} profile.

Note that the \textit{configure} tool is not a environment configuration
files editor. Indeed, this tool targets final users and therefore has to
be as simple as possible.

The basic \textit{configure} behaviour consists in displaying the final
variable's value. If the user enters a new value, no matter whether there is
a relation with its previous value, the tool creates/modifies an entry in the
\textit{user} profile's configuration file overriding any previous definition.

For instance, consider the \textit{\_FOO\_} development environment variable
with the following configuration definition:

In \textit{profile/environment.conf}:

\begin{verbatim}
  _FOO_                         =                       initial
\end{verbatim}

In \textit{profile/core/core.conf}:

\begin{verbatim}
  _FOO_                         +=                      addon
\end{verbatim}

Let us suppose the user enters the following value instead of the current
one: \verb|initial addon|

\begin{verbatim}
  _FOO_                         =                       initial new
\end{verbatim}

Then, the \textit{configure} tool creates a new entry into the \textit{user}
profile configuration file:

\begin{verbatim}
  _FOO_                         =                       initial new
\end{verbatim}

Finally, note that when the \textit{configure} tool is launched, it first
tries to detect whether the user is a newcomer or not. If it is, then the
tool asks the user to create a new \textit{user} profile, step by step. These
actions are performed in the \textit{critical.py} script of the
\textit{configure/} directory.

% requirements

\subsubsubsection{Requirements}

The \textit{configure} tool relies on the \textit{Dialog} software which
is present on many \textit{Unix} systems. Indeed, the \textit{configure}
tool is a user-friendly configuration utility.

% syntax

\subsubsubsection{Syntax}

The syntax of the frame description files is based on \textit{YAML}. Therefore,
the \textit{Python} \textit{PyYAML} module needs to be set up.

As said previously, a frame is composed of sub-frame and variable entries. A
sub-frame entry contains a name and a path to the sub-frame description file
while a variable entry only contains the name of the variable. This variable
name is then used to retrieve the variable description.

The example below illustrates this very simple syntax:

\begin{verbatim}
[XXX]
  - title: Segment Manager
    description: |
      This frame contains configuration about the core
      segment manager

  - frame: optimisations
    path: subsections/optimisations.desc

  - frame: machine dependent
    path: subsections/machine.desc

  - variable: _FOO_

  - variable: _BAR_

  - variable: _CHICHE_
\end{verbatim}


% view

%
% ---------- header -----------------------------------------------------------
%
% project       kaneton
%
% license       kaneton
%
% file          /home/mycure/kaneton/view/book/development/view.tex
%
% created       julien quintard   [wed may 23 00:36:53 2007]
% updated       julien quintard   [mon may  4 19:43:15 2009]
%

%
% ---------- view -------------------------------------------------------------
%

\subsection{View}
\label{section:view}

The \name{view} tool serves as a document database as well as a tool for
building and displaying documents in an easy way.

The kaneton documents are classified, each directory corresponding to a
class of documents. Below are listed the subdirectories of the \location{view/}
directory.

\begin{verbatim}
  bibliography/
  book/
  exam/
  feedback/
  figures/
  internship/
  lecture/
  logo/
  package/
  paper/
  talk/
  template/
\end{verbatim}

The \location{template/} directory contains templates for every class of
document. The \location{bibliography/} and \location{logo/} directories
contain, obviously, the bibliography which is common to all the documents, and
the logos, respectively. The \location{figures/} directory contains figures
common to all the documents while the \location{package/} directory contains
additional {\LaTeX} packages.

The directories \location{curriculum/}, \location{exam/} and
\location{feedback/} contain documents in relation with teaching. The
\location{curriculum/} directory contains documents such as the educational
project year planning \etc{} The \location{feedback/} directory contains
documents which are distributed to the students at the end of the kaneton
project in order to get feedback for improving the project for the next years.
Needless to say the \location{exam/} directory contains everything related to
examinations while the \location{talk/} directory contains conference talks
and various presentations of the education project for instance.

The other directories contain the actual kaneton documentation. The
\name{books} represent the main documents targeting any public:
contributors, teachers, students \etc{} The \name{papers} are lighter
documents intended to present a specific feature, design \etc{} The
\name{lectures} are the courses materials, generally composed of
presentation slides. Finally, the \name{internship} documentation is
composed of documents written by people partially involved in the kaneton
project.

Any document is composed of a \name{Make} file and one or more
{\LaTeX} files. The \name{Make} file always has the same form
with little variations depending on the type of document. For more information
on the rules applying to the \name{Make} and {\LaTeX} files, please
refer to their respective sections: \reference{Section \ref{section:make}}
and \reference{Section \ref{section:python}}.

The \name{view} tool basically starts looking for \location{.tex} files
and builds a list of directories containing documents. Then, it provides
to the user the possibility to build and display a given document. If no
document name is given on the command line, then the tool draws a list
of the available documents.

People contributing to the kaneton documents must take care of following
the rules in relation with the {\LaTeX} language. Moreover, contributors
should look at the existing documents to understand to logic behind all
these rules.

Finally, note that nobody should create a document without discussing it
on the mailing-list first. Especially, be very careful in naming your
documents as people took good care of this directory in order to avoid
it to become messy.

If a document already exists with the same name, then go through the
mailing-list in order to decide whether to keep the current version. If
people decide to keep a document, then, the contributor in charge of writing
the new one should re-organise the documents by creating archives for
each year.


% export

%
% ---------- header -----------------------------------------------------------
%
% project       kaneton
%
% license       kaneton
%
% file          /home/mycure/kaneton/view/book/development/export.tex
%
% created       julien quintard   [wed may 23 18:58:41 2007]
% updated       julien quintard   [thu may 24 20:44:48 2007]
%

%
% ---------- export -----------------------------------------------------------
%

\subsection{Export}

The \textit{export} tool was introduced for making the process of releasing
easier. The tool takes an argument specifying the type of target release.

Recall the kaneton microkernel project is used as a material for operating
system courses. The source code of the microkernel is distributed to
the students with some parts missing. Then, students have to re-write
these pieces of code in order to prove their well-understanding of the
kernel internals. Additionnaly, the kaneton project is also a research project
in operating systems design.

As a result, the \textit{export} tool sometimes has to build a release
with pieces of code removed, sometimes not. Below are listed the different
type of release:

\begin{itemize}
  \item
    \textbf{backup}: this release type is basically a bare backup of
    the kaneton microkernel project source code.
  \item
    \textbf{dist}: the distribution release is performed by removing
    the repository-specific stuff.
  \item
    \textbf{k}$\gamma$\textbf{,}$\epsilon$: this type of release is intended
    to students. Therefore, repository-specific stuff is removed. Also
    teaching materials such as courses, testing scripts, cheating scripts
    etc.. - specified in the kaneton development environment variable
    \textit{\_HIDDEN\_} - are removed.

    \-

    Finally, the pieces of code comprised in the range $[\gamma,\epsilon]$
    are removed from the release. These pieces of code are marked using the
    \textit{export} syntax described next.

    \-

    The stages $\gamma$ and $\epsilon$ represent kaneton sub-project ranks:

    \begin{itemize}
      \item
	\textbf{0}: boot stuff: boostrap, bootloader etc..;
      \item
	\textbf{1}: memory management;
      \item
	\textbf{2}: event management: interrupts, I/O etc..;
      \item
	\textbf{3}: task management, scheduling;
      \item
	\textbf{4}: communication management.
    \end{itemize}
\end{itemize}

Finally, the generated release is named based on the \textit{\_EXPORT\_}
kaneton environment variable followed by the date and type of the release:
\textit{backup}, \textit{dist} or \textit{k}$\gamma$\textit{,}$\epsilon$.

% syntax

\subsubsubsection{Syntax}

As explained previously, pieces of code are removed in order to build
\textit{stage} releases.

The kaneton source code is marked so that the \textit{export} tool knows
what piece of code to remove and for what stage. Indeed, every piece of
educational code is marked by a tag indicating the stage it is related to.

The syntax used is illustrated below:

\begin{verbatim}
  /*                                                                [cut] k1 */

  /*
   * this function clones a segment.
   *
   * steps:
   *
   * 1) get the original segment object.
   * 2) reserve a new segment of same size with same permissions.
   * 3) copy the data from the old segment.
   * 4) call machine-dependent code.
   */

  t_error                 segment_clone(i_as                      asid,
                                        i_segment                 old,
                                        i_segment*                new)
  {
    o_segment*            from;
    t_perms               perms;

    SEGMENT_ENTER(segment);

    /*
     * 1)
     */

    if (segment_get(old, &from) != ERROR_NONE)
      SEGMENT_LEAVE(segment, ERROR_UNKNOWN);

    [...]

    /*
     * 4)
     */

    if (machdep_call(segment, segment_clone, asid, old, new) != ERROR_NONE)
      SEGMENT_LEAVE(segment, ERROR_UNKNOWN);

    SEGMENT_LEAVE(segment, ERROR_NONE);
  }

  /*                                                               [cut] /k1 */
\end{verbatim}

In this example, the kaneton teachers decided \textit{segment\_clone()}
was a functionality the students should implement.

The markings at the top \verb|[cut] k1| and bottom \verb|[cut] /k1| of this
example indicate the \textit{export} tool the location of the piece of code
to remove for the stage \textit{k1}.

Let us suppose a teacher $T_{1}$ wants to use kaneton leading the students to
the development of the memory management functionality, only. On the other
hand, another teacher, $T_{2}$, wants to use the whole kaneton project starting
with the bootloader implementation to the task management. Additionally,
this teacher chooses to hide the communication management pieces of code,
to avoid cheating between students of different universities for instance.

In the first case, since the students have to implement the kaneton managers
around the memory management, $T_{1}$ has to provide the students everything
the memory management stuff relies on, including some fundamental managers,
the bootloader etc.. Also, the teacher does not need to provide the source
code of the upper level managers. As a result, a \textit{k1,1} release will
remove the pieces of codes with every marking $k_{\alpha}$ for
$1 \le \alpha \le 1$.

The teacher $T_{2}$ needs something different since the students are going
to implement every major piece of the kaneton source code. Since this teacher
wants their students to implement all the steps, starting with \textit{k0}
to \textit{k4}, a \textit{k0,4} release will not contain the pieces of source
code marked $k_{\alpha}$ for $0 \le \alpha \le 4$.


% transcript

%
% ---------- header -----------------------------------------------------------
%
% project       kaneton
%
% license       kaneton
%
% file          /home/mycure/kaneton/view/book/development/transcript.tex
%
% created       julien quintard   [thu may 24 05:07:02 2007]
% updated       julien quintard   [fri jun  1 00:58:43 2007]
%

%
% ---------- transcript -------------------------------------------------------
%

\subsection{Transcript}
\label{section:transcript}

The \textit{transcript/} directory is composed of two tools related to
the management of transcripts. The \textit{record} tool captures a
shell session while the \textit{play} tool replays a captured session.

These tool were introduced to allow students to make a dynamic presentation
of their kaneton implementation's features and possibilities. These dynamic
presentations were supposed to replace the oral examinations.

These transcripts are not used by the main contributors of the kaneton
project yet. However, any teacher interested by this tool can use it.

The \textit{transcript/} directory contains subdirectories which classify
the transcripts.

The only transcript class currently in place is named \textit{basic} and
contains transcripts illustrating the use of the kaneton internal tools.


% cheat

%
% ---------- header -----------------------------------------------------------
%
% project       kaneton
%
% license       kaneton
%
% file          /home/mycure/kaneton/view/book/development/cheat.tex
%
% created       julien quintard   [thu may 24 11:57:37 2007]
% updated       julien quintard   [fri jun  1 01:03:21 2007]
%

%
% ---------- cheat ------------------------------------------------------------
%

\subsection{Cheat}
\label{section:cheat}

The \textit{cheat} tool checks whether students cheated by using pieces of
code from kaneton projects of the previous years.

The \textit{history/} directory is composed of directories organizing the
kaneton students implementations over the years and for every school and
university the education project was used. Then each subdirectory represents
a year and contains subdirectories for each students group of this year.

Each student group directory contains a \textit{sources/} subdirectory
containing the tarballs of the different kaneton stages: \textit{k0},
\textit{k1}, \textit{k2} and so on; a \textit{fingerprints/} directory
containing an internal source representation used for detecting cheating,
a \textit{tests/} directory containing a summary of the testing results
for each stage and a \textit{cheats/} directory which contains a list of
commonalities with other kaneton implementations of the same and previous
years.

The \textit{cheat} tool takes a year and a stage as arguments. Its first
task is to generate the fingerprints of the other kaneton implementations
for this stage of the same and previous years.

Once the fingerprints are generated, the tool performs the checks by
comparing each pair of kaneton implementations for this stage.

The \textit{cheat} tool is based on another tool which cannot be revealed
here. For more information, please contact your supervisor.


% test

%
% ---------- header -----------------------------------------------------------
%
% project       kaneton
%
% license       kaneton
%
% file          /home/mycure/kaneton/view/book/development/test.tex
%
% created       julien quintard   [thu may 24 12:18:23 2007]
% updated       julien quintard   [wed dec  8 22:37:27 2010]
%

%
% ---------- test -------------------------------------------------------------
%

\subsection{Test}
\label{section:test}

The \name{test} tool enables students to test their kaneton implementation
against a set of tests that have been designed by the contributors. Below
is briefly described the terminology used by this tool in order to give
the reader an overview of the general scenario involving students, the
administrator and the server running the test system.

\begin{itemize}
  \item
    A \name{certificate} is used to make sure clients can authenticate the
    test server;
  \item
    Every certificate is sealed by a cryptographic \name{key};
  \item
    Each user is provided with a \name{capability} in order to identify
    herself to the server;
  \item
    These capabilities are sealed by a \name{code} which the server uses
    in order to detect illegally forged capabilities;
  \item
    A \name{configuration} specifies the number of tests a user is allowed
    to requests the server;
  \item
    The user's \name{database} is generated based on a configuration and
    maintains the current user's state on the server including the number
    of tests performed so far, the kaneton implementations submitted for
    evaluation \etc{}
  \item
    A \name{snapshot} is a kaneton implementation in its shipping form;
  \item
    The \name{machine} represents the target
    \name{platform}/\name{architecture} couple on which a snapshot is
    supposed to be tested or evaluated for instance;
  \item
    An \name{image} represents a kaneton snapshot compiled in a bootable
    form;
  \item
    A \name{test} is a function included in the kernel which performs a
    specific set of operations and possibly prints information to the console;
  \item
    The tests are often gathered together in a \name{suite} which represents
    the testing unit students are offered to trigger for their kaneton
    implementation;
  \item
    Once a snapshot is received by the server in order to be tested,
    the system compiles it into an image. The server also takes care to
    include the tests in the compilation process so that they can be triggered.
    These pre-compiled tests are referred to as the \name{bundle};
  \item
    The image can then be tested by triggering the tests of the suite. The
    image is therefore booted in an emulated \name{environment}. This
    environment can sometimes be chosen and offers a trade-off between
    simplicity and realism. The most common environments are \name{QEMU}
    and \name{Xen};
  \item
    Depending on the success of the tests, a set of results is generated
    and compiled in a \name{bulletin} file;
  \item
    Finally, the server retrieves this bulletin, adds some meta information
    such as the date of the test, the environment and machine used \etc{}
    and stores everything in a \name{report}. Note that this report is
    also sent back to the user so she can consult it;
  \item
    Students can also decide to submit their kaneton implementation for
    a specific \name{stage} for future evaluation. Note that suites and
    stages are completely different though they often bear the same names:
    \name{k0}, \name{k1}, \name{k2} etc{};
  \item
    The administrator can decide to evaluate the snapshots which have been
    submitted for a stage by invoking a script which will attribute grades
    according to the \textit{point}s associated with every test.
  \item
    Finally, a \name{statement} is produced containing the grades of every
    student for a given stage.
\end{itemize}

The following describes the \name{test} tool according to the user's role
regarding the system: either the administrator who sets up the system or
a student who uses it in order to improve and/or evaluate his implementation.

%
% administrator
%
\subsubsection{Administrator}

The administrator is responsible for setting up the system but also maintaining
it on a daily basis.

% requirements
\subsubsubsection{Requirements}

The \name{test} tool must be installed on a publicly accessible server since
the server script will be waiting for incoming requests. Note that by default,
the clients assume the test server to be accessible at the address:
\location{https://test.opaak.org:8421}.

Besides, since the purpose of the \name{test} tool is to run the students'
kaneton implementation in emulated environments, both \name{QEMU} and
\name{Xen} should be available though one might want to configure the tool
for supporting a single environment, \name{QEMU} for instance.

Note that the test system has been developed with \name{Python 2.6} and
may be out of date by the time the administrator sets it up. In addition,
the system depends on a variety of \name{Python} packages including
\name{argparse}, \name{yaml}, \name{pyopenssl}, \name{hmac}, \name{pickle},
\name{xmlrpc}, \name{subprocess}, \name{serial} among others.

Finally, the administrator should make sure the following applications
are installed since some test scripts need them: \name{dd}, \name{mkfs.ext2},
\name{mount}, \name{umount}, \name{mutt}, \name{sendmail}, \name{qemu}
and \name{mkisofs}.

% set up
\subsubsubsection{Set Up}

The first step for an administrator consists in generating the necessary
files, especially the certificates, code and capabilities required for
securing the test system.

The \location{test/utilities/} directory contains the scripts that perform
such operations. Note that all the generated files are stored in the
\location{test/store/} directory.

First the \name{CA - Certification Authority}'s and server's certificates
must be generated. The first is used to issue certificates while the latter
is used for clients to identify the server with absolute certainty.

\begin{verbatim}
  $> make certificate
  [+] generating the CA and server's key/certificate pair
  [+] CA key/certificate generated
  [+] server key/certificate generated
  [+] CA and server's key/certificate pair generated and stored
  $> 
\end{verbatim}

The next step consists in generating a code for the administrator to
issue capabilities but also for the server to verify that the received
capabilities have not been illegally forged.

\begin{verbatim}
  $> make code
  [+] generating the server's code
  [+] server code successfuly generated and stored
  $> 
\end{verbatim}

With a server code, the students' and contributor's capabilities can be
built, hence granting them the right the contact the server.

The following generates the contributor's capability. This capability is
special in the way that contributors can perform any operation in a completely
contrain-free manner.

\begin{verbatim}
  $> make capability-contributor
  [+] generating the contributor's capability
  [+] contributor's capability generated and stored
  $> 
\end{verbatim}

In contrast, the following command generates a set of capabilities for the
students belonging to the school referred to as \name{``epita::2010''}. Note
that the script requires the \location{history/epita/2010/} to be populated
with the groups and their \location{people} file.

\begin{verbatim}
  $> make capability-school@epita::2010
  [+] generating students' capabilities
  [+] extracting the students from the history 'epita/2010'
  [+] students information retrieved
  [+] generating the students' capabilities:
  [+]   epita::2010::group11
  [+]   epita::2010::group10
  [+]   epita::2010::group13
  [+]   epita::2010::group12
  [+]   epita::2010::group33
  [+]   epita::2010::group32
  [+]   epita::2010::group17
  [+]   epita::2010::group30
  [+]   epita::2010::group19
  [+]   epita::2010::group18
  [+]   epita::2010::group5
  [+]   epita::2010::group4
  [+]   epita::2010::group7
  [+]   epita::2010::group6
  [+]   epita::2010::group1
  [+]   epita::2010::group3
  [+]   epita::2010::group2
  [+]   epita::2010::group15
  [+]   epita::2010::group9
  [+]   epita::2010::group8
  [+]   epita::2010::group14
  [+]   epita::2010::group31
  [+]   epita::2010::group16
  [+]   epita::2010::group24
  [+]   epita::2010::group25
  [+]   epita::2010::group26
  [+]   epita::2010::group27
  [+]   epita::2010::group20
  [+]   epita::2010::group21
  [+]   epita::2010::group22
  [+]   epita::2010::group23
  [+]   epita::2010::group28
  [+]   epita::2010::group29
  [+]   epita::2010::group34
  [+] students' capabilities generated and stored
  $> 
\end{verbatim}

In addition, the administrator could decide to generate or re-generate
a capability for a specific student of a school. The following shows an
example for such an action.

\begin{verbatim}
  $> make capability-student@epita::2010::group8
  [+] generating the student's capability:
  [+]   epita::2010::group8
  [+] student's capability generated and stored
  $> 
\end{verbatim}

The next step consists in the databases generation. A database contains the
state of a user profile including the number of test requests, the quota for
such tests, the submitted snapshots and so forth. The database files are
absolutely fundamental to the server since such databases are updated after
each client's request.

The syntax for generating databases follows the one for capabilities, as
shown next for the contributor.

\begin{verbatim}
  $> make database-contributor
  [+] generating database from contributor's configuration
  [+] contributor's database generated and stored
  $> 
\end{verbatim}

Once the certificates, code, capabilities and databases generated, the
administrator can move on to the deployment process.

% deployment
\subsubsubsection{Deployment}

The deployment basically consists in copying the \location{test/} environment
to the test server though one might want to copy the whole kaneton environment
or the smallest subset of the \location{test/} directory which should, in this
case, include the following absolutely necessary items:

\begin{itemize}
  \item
    The \location{test/environments/} directory which contains the descriptions
    of the supported test environments;
  \item
    The \location{test/images/} directory which contains a script for
    automatically generating a \name{Debian Live} system which is used
    for compiling a kaneton snapshot into a bootable image;
  \item
    The \location{test/packages/} directory which contains the \name{ktp -
    Kaneton Test Package} required by the server-side standalone scripts
    for manipulating files such as databases, capabilities \etc{} but also
    for performing cryptographic operations and send/receive \name{XMLRPC}
    requests;
  \item
    The \location{test/scripts/} directory which contains the fundamental
    scripts for building bootable images, distributing the capabilities to
    the students through emails, evaluating the submitted snapshots and so on;
  \item
    The \location{test/server/} directory which contains the server script
    for handling the clients' requests;
  \item
    The \location{test/stages/} directory which contains the files requirement
    for evaluating the students' snapshots;
  \item
    The \location{test/store/} directory which contains the generated files
    such as the users' databases, the server's code and certificate; and
  \item
    The \location{test/suites/} directory which contains the files describing
    the tests to be including in a given tests suite.
\end{itemize}

Once copied, the administrator only needs to launch the server script located
in the \location{test/server/} directory, as shown below:

\begin{verbatim}
  $> ./server.py
  [meta] serving on 88.191.84.128:8421
\end{verbatim}

Note that a few additional steps may be required depending on the current state
of the kaneton development.

The first of these steps may consist in generating a \name{Debian Live} system
since this is absolutely required for the test system to work. For more
information regarding the generation of such an image, please refer to the
\location{test/images/} directory.

The second step should consist for the administrator in building the kaneton
tests bundle. The bundle represents a pre-compiled set of tests that is
included in the students' snapshot compilation process. The tests are
pre-compiled in order to prevent leaking information since students could
very well dump the content of those tests and force the compilation to fail,
hence retrieving the source code in the compilation process' error log.

In order to generate such a bundle, the administrator must first activate
the \name{test} module, as show next:

\begin{verbatim}
  _MODULES_               +=              test
\end{verbatim}

Then, the administrator must move to the \location{test/tests/} directory
and launch a compilation process through the following command:

\begin{verbatim}
  $> make
\end{verbatim}

Once generated, the test bundle, located in \location{store/bundle/[machine]/}
must be copied to the server, at the same location.

Finally, for more information on the server script, please refer to the
\location{test/server/} directory.

% scripts
\subsubsubsection{Scripts}

Although the deployment process is pretty straightforward, the administrator
is required to manage the test system through several scripts.

First, the \name{distribute} script must be used by the administrator to send
the capabilities to the respective owners so that the students can use
the test system. Note that this script relies on the \name{Mutt} mailing
system for sending the emails containing the attached capabilities.

\begin{verbatim}
  $> ./distribute.py
  recipients:
    contributor
  $>
\end{verbatim}

While the \name{construct} script enables the administrator to build a
bootable image from a kaneton snapshot, the \name{stress} script takes
a bootable image and triggers the tests belonging to the given test
suite. Note that both scripts are directly used by the server script for
building and testing the received kaneton snapshots.

\begin{verbatim}
  $> ./construct.py --snapshot kaneton.tar.bz2                          \
                    --image kaneton.img                                 \
                    --environment xen                                   \
  the kaneton image has been constructed in 'kaneton.img'
  $> ./stress.py --image kaneton.img                                    \
                 --suite k2                                             \
                 --environment xen                                      \
                 --verbose
  segment
    permissions/01 :: true
  id
    simple :: true
    clone :: true
    multiple :: true
  $> 
\end{verbatim}

Note that the administrator could also test a kaneton image manually,
especially through the following command:

\begin{verbatim}
  $> qemu -fda kaneton.img -curses
\end{verbatim}

Besides, note that an administrator willing to include a new test in the
system would probably want to test it locally first since testing through
the server takes some time. In order to test locally, the administrator
must first activate the bundle module in its user profile
\location{environment/profile/user/\${KANETON\_USER}/\${KANETON\_USER}.conf}:

\begin{verbatim}
  _MODULES_               +=              bundle
\end{verbatim}

Then, the administrator must trigger the test by calling the test function
manually in its kaneton implementation. For instance, in order to trigger
the \name{kaneton/core/task/guest} test, the administrator could add the
following line after \code{kernel\_initialize()} and before running the
test system in \location{kaneton/core/core.c}:

\begin{verbatim}
  [...]

  module_call(console, message,
              '+', "starting the kernel\n");

  assert(kernel_initialize() == STATUS_OK);

  /* XXX[temporary] */
  test_core_task_guest();

  module_call(test, run);

  [...]
\end{verbatim}

Once the kaneton image rebuilt, the administrator can boot it locally
through \name{QEMU} and get the output, hence check that the test went
as excepted:

\begin{verbatim}
  $> qemu -fda environment/profile/user/${KANETON_USER}/${KANETON_USER}.img
\end{verbatim}

Back to the server side, the \name{evaluate} script can be used by the
administrator in order to assign grades to the snapshots submitted by the
students. The script generates a statement containing the results of this
evaluation process.

\begin{verbatim}
  $> ./evaluate.py --stage k2                                           \
                   --pattern "^epita::2010::.*$"
  the statement has been saved in '../store/statement/20101102-223645.db'
  $> 
\end{verbatim}

Finally, the \name{dump} script takes any \name{YAML}-based file and
displays its inner structure in a hierarchical manner.

\begin{verbatim}
  $> ./dump.py --path ../store/statement/20101102-223848.db
  meta:
    reference:              20.0
    stage:                  k2
  data:
    epita::2010::group7:
      date:                 2010/11/02 20:46:44
      grade:                16.0
      snapshot:             20101102-204644
      members:
        email:              admin@opaak.org
        name:               admin
      configurations:
        Xen:
          report:           20101102-224213
          notch:            4
          score:            4
        QEMU:
          report:           20101102-223848
          notch:            4
          score:            0
\end{verbatim}

%
% student
%
\subsubsection{Student}

The student has the possibility to request actions from the test server
through the client script located in \location{test/client/}.

% requirements
\subsubsubsection{Requirements}

Although the client script is integrated in the kaneton environment, it also
makes use of the \name{ktp}. Therefore, as for the server, the client depends
on a variety of \name{Python} packages including \name{yaml}, \name{pyopenssl},
\name{hmac}, \name{pickle}, \name{xmlrpc}, \name{subprocess} among others.

% use
\subsubsubsection{Use}

The client script enables the user to request one of the five operations
described below.

\begin{verbatim}
  $> make
  [!] usage: client.py [command]

  [!] commands:
  [!]   information
  [!]   submit-[stage]
  [!]   test-[environment]::[suite]
  [!]   list
  [!]   display-[identifier]
  [!]   retest-[identifier]
  $>
\end{verbatim}

The \name{information} operation requests the server to return information
on the current state of the user's profile. The information returned range
from the number of tests performed, the quota for every test suite to the
available stages or the snapshots having been previously submitted.

\begin{verbatim}
  $> make information
  [+] configuration:
  [+]   server:                 https://test.opaak.org:8421
  [+]   capability:             /data/mycure/repositories/kaneton/environment/profile/user/julien.quintard/julien.quintard.cap
  [+]   platform:               ibm-pc
  [+]   architecture:           ia32/educational

  [+] information:
  [+]   profile:
  [+]     identifier:           contributor
  [+]     community:            contributors
  [+]     members:
  [+]       name:               admin
  [+]       email:              admin@opaak.org
  [+]   suites:
  [+]                           k1
  [+]                           k3
  [+]                           k2
  [+]                           kaneton
  [+]   stages:
  [+]                           k1
  [+]                           k2
  [+]                           k3
  [+]   environments:
  [+]                           qemu
  [+]                           xen
  [+]   database:
  [+]     reports:
  [+]       xen:
  [+]         ibm-pc.ia32/educational:
  [+]           k3:
  [+]           k2:
  [+]           k1:
  [+]       qemu:
  [+]         ibm-pc.ia32/educational:
  [+]           k3:
  [+]           k2:
  [+]           k1:
  [+]     settings:
  [+]       xen:
  [+]         ibm-pc.ia32/educational:
  [+]           k3:
  [+]             requests:     0
  [+]             quota:        -1
  [+]           k2:
  [+]             requests:     0
  [+]             quota:        -1
  [+]           k1:
  [+]             requests:     0
  [+]             quota:        -1
  [+]       qemu:
  [+]         ibm-pc.ia32/educational:
  [+]           k3:
  [+]             requests:     0
  [+]             quota:        -1
  [+]           k2:
  [+]             requests:     0
  [+]             quota:        -1
  [+]           k1:
  [+]             requests:     0
  [+]             quota:        -1
  $> 
\end{verbatim}

The \name{test} command enables the user to trigger a test suite for the
current kaneton implementation on the specified environment such as \name{QEMU}
or \textit{Xen} for instance.

The server then returns the resulted report which the client stores locally
in \location{test/store/report/}. In addition, the client displays a quick
summary of the report in order for the user to know whether things went
as expected.

\begin{verbatim}
  $> make test-xen::k2
  [+] configuration:
  [+]   server:                 https://test.opaak.org:8421
  [+]   capability:             /data/mycure/repositories/kaneton/environment/profile/user/julien.quintard/julien.quintard.cap
  [+]   platform:               ibm-pc
  [+]   architecture:           ia32/educational

  [+] report(20101103:140601):
  [+]   segment                                                           [1/1]
  [+]   id                                                                [3/3]
  $> 
\end{verbatim}

The \name{list} command enables the user to display the identifiers of the
reports in the local store.

\begin{verbatim}
  $> make list
  [+] reports:
  [+]   20101103:140601:
  [+]     xen :: ibm-pc :: ia32/educational :: k2 :: 2010/11/03 14:06:01
\end{verbatim}

The \name{display} command gives the user the possibility to dump a locally
stored report in a very detailed way.

\begin{verbatim}
  $> make display-20101103:140601
  [+] report:
  [+]   meta:
  [+]     platform:               ibm-pc
  [+]     date:                   2010/11/03 14:06:01
  [+]     architecture:           ia32/educational
  [+]     duration:               63.499
  [+]     suite:                  k2
  [+]     identifier:             20101103:140601
  [+]     environments:
  [+]       stress:               xen
  [+]       construct:            xen
  [+]   data:
  [+]     segment:                                                        [1/1]
  [+]       permissions/01:
  [+]         status: True
  [+]         description: This test creates a task and address space before reserving a segment and changing its permissions.
  [+]         duration: 0.010
  [+]         output: 
  [+]     id:                                                             [3/3]
  [+]       simple:
  [+]         status: True
  [+]         description: This test reserves a single identifier.
  [+]         duration: 0.004
  [+]         output: 
  [+]       clone:
  [+]         status: True
  [+]         description: This test reserves, clones and releases identifiers.
  [+]         duration: 0.005
  [+]         output: 
  [+]       multiple:
  [+]         status: True
  [+]         description: This test reserves thousands of identifiers, checking that no collisions occured.
  [+]         duration: 0.040
  [+]         output: 
  $> 
\end{verbatim}

The \name{submit} command sends the user's snapshot to the server so as to
be evaluated for the given stage.

\begin{verbatim}
  $> make submit-k1
  [+] configuration:
  [+]   server:                 https://test.opaak.org:8421
  [+]   capability:             /data/mycure/repositories/kaneton/environment/profile/user/julien.quintard/julien.quintard.cap
  [+]   platform:               ibm-pc
  [+]   architecture:           ia32/educational

  [+] the snapshot has been submitted successfully
  $> 
\end{verbatim}

Finally, the \name{retest} command provides contributors the possibility to
re-launch the test suite of the given identified test. This command is
especially useful to re-test a snapshot should an unexpected error occur on
the test server.

Indeed since test requests are limited for students, it would be unfair for the
student to be forced to sacrifice a test slot because something went wrong
on the server-side. By requesting a contributor, the student's snapshot can
be re-tested. Once the test complete, an email is sent to the student along
with the attached report.

\begin{verbatim}
  $> make retest-20101103:140601
  [+] configuration:
  [+]   server:                 https://test.opaak.org:8421
  [+]   capability:             /data/mycure/repositories/kaneton/environment/profile/user/julien.quintard/julien.quintard.cap
  [+]   platform:               ibm-pc
  [+]   architecture:           ia32/educational

  [+] the snapshot has been re-tested successfully
  $> 
\end{verbatim}

%
% robot
%
\subsubsection{Robot}

The \name{robot} test tool enables contributors to test the kaneton research
implementation on a regular basis; hence control the status of the development.

The robot basically retrieves the kaneton implementation by checking out the
\name{Subversion} repository. Then, several test suites are triggered through
the test client. Once the reports have been received, a message is built
summarizing the results. This message is then sent to the kaneton contributors
mailing-list.

The deployment of the \name{robot} is quite straigthforward. First, the
\location{test/robot/} directory must be copied to the server. Note that
the \name{robot.py} script depends upon the \name{ktp} package which must
therefore be copied as well.

Then, the \name{SSH} configuration file \name{config} must be placed in
the \location{\${HOME}/.ssh/} directory. Besides, this file should be edited in
order to properly reference the \name{SSH} keys since the default configuration
assumes the kaneton test directory to be located at \location{/kaneton/}.

Finally, the \name{robot.cron} crontab file must be setup through the
following command in order to trigger the robot every night:

\begin{verbatim}
  $> crontab robot.cron
\end{verbatim}

Once again, the administrator should make sure to edit this file should
the robot files not be located in the default location \ie{}
\location{/kaneton/}.


% control panel

%
% ---------- header -----------------------------------------------------------
%
% project       kaneton
%
% license       kaneton
%
% file          /home/mycure/kaneton/view/book/development/control-panel.tex
%
% created       julien quintard   [sun may 20 15:22:52 2007]
% updated       julien quintard   [mon may  4 20:01:20 2009]
%

%
% ---------- control panel ----------------------------------------------------
%

\subsection{Control Panel}
\label{section:control panel}

The kaneton environment allows the developer to trigger every action from
the \name{Make} file of the project's root directory.

These actions are listed below:

\command{make initialize}
        {
	  This action initializes the kaneton development environment by
	  invoking the \location{init.py} script of the \location{environment/}
	  directory.
	}

\command{make clean}
	{
	  This action cleans the kaneton development environment.
	}

\command{make main}
	{
	  This action triggers the default rule which aims at compiling every
	  piece the final system needs to be set up on a bootable device.

	  \-

	  \example{\$ make main}

	  \example{\$ make}
	}

\command{make clear}
	{
	  This action removes every compiled files.
	}

\command{make headers}
	{
	  This action generates \name{Make} files' \name{C} header
	  files dependencies.
	}

\command{make prototypes}
	{
	  This action generates C prototypes.
	}

\command{make test}
	{
	  This action runs the test suite in order to validate the kaneton
	  microkernel behaviour.
	}

\command{make cheat}
	{
	  This action launches the cheat tests on students kaneton
	  implementations.

	  \-

	  \example{\$ make cheat}

	  \-

	  \example{\$ make cheat-EPITA::2006::k3}
	}

\command{make build}
	{
	  This action builds the boot device.
	}

\command{make install}
	{
	  This action installs the kaneton microkernel with its dependencies:
	  configuration files, bootloader \etc{} on the boot device.
	}

\command{make export}
	{
	  This action builds a kaneton distribution package.

	  \-

	  \example{\$ make export}

	  \-

	  \example{\$ make export-k3,5}

	  \-

	  \example{\$ make export-back}
	}

\command{make view}
	{
	  This action builds and displays a kaneton document.

	  \-

	  \example{\$ make view}

	  \-

	  \example{\$ make view-devel}

	  \-

	  \example{\$ make view-book::kaneton}
	}

\command{make record}
	{
	  This action records a real-time session.

	  \-

	  \example{\$ make record}

	  \-

	  \example{\$ make record-basic::test.ts}
	}

\command{make play}
	{
	  This action plays a recorded session.

	  \-

	  \example{\$ make play}

	  \-

	  \example{\$ make play-basic::prototypes.ts}

	  \-

	  \example{\$ make play-prototy}
	}

\command{make info}
	{
	  This action displays general information about kaneton.
	}


%
% external
%

\section{External}

The kaneton contributors use several other tools for the communication, the
development etc.. These tools are described in the following sections.

% mailing-list

%
% ---------- header -----------------------------------------------------------
%
% project       kaneton
%
% license       kaneton
%
% file          /home/mycure/kaneton/view/book/development/mailing-list.tex
%
% created       julien quintard   [thu may 24 19:55:18 2007]
% updated       julien quintard   [thu may 24 20:43:15 2007]
%

%
% ---------- mailing-list -----------------------------------------------------
%

\subsection{Mailing-List}

Because people do not want to use several communication tools: email,
newsgroup, forum etc.. and because everybody has an email address, the
kaneton people communicate through a mailing-list.

This mailing-list is in fact a \textit{Google} group. Indeed, the kaneton
project relies on three distinct communication groups:

\begin{itemize}
  \item
    \textbf{kaneton} which is used to make announcements about new releases,
    patchs etc..

    \-

    This group is not used yet.
  \item
    \textbf{kaneton-developers} is dedicated to the communication between the
    people involved in the development of the kaneton microkernel reference.

    \-

    This group is therefore private.
  \item
    \textbf{kaneton-students} can be used in an absolutely free-way for
    students for communicating about their kaneton implementation.

    \-

    Anybody can join this group.
\end{itemize}

Needless to say, community behavioural rules enumerated in a previous section
must be followed when communicating on the kaneton mailing-list. Every
contributor is welcomed to give its point of view, to ask questions etc.. but
this must be done with politness, respect and humility.

Everyone communicating through the mailing-list must read the
\textit{Netiquette} which describes the rules inherent to the communication
on the Internet. Especially, people should take care of writing messages in
respect of the \textit{80} columns; and should always cut off useless parts
of a previous message when responding.

It is likely a real-time communication tool will be very useful in the
future, as an \textit{IRC} channel, for instance. However, communicating on
these extra media will not be mandatory until kaneton people decide it is.

kaneton people are asked to use the mailing-list communication medium
in a perfect way as it is the unique intra-communication channel. In
addition then, contributors must read their emails on a regular-basis
as some people rely on the decision of others.

The \textit{kaneton-students} mailing-list must be used carefully. As
an example, people should never paste pieces of source code or ask
questions implying an answer with the solution. Even if it is a free
group, people abusing of this communication channel could easily be banned.

People must always respond in the appropriate discussion. If, in a discussion,
a different subject is discussed, then, one of the contributor must create
a new discussion in order facilitate the communication.

Finally, the discussion subjects must be tagged like the following examples:

\begin{verbatim}
  [ia32/optimised] mapping issues

  [segment] segment_clone() :: bug?

  [research] new paper about OS design
\end{verbatim}

There is no list of official tags. The users are simply asked to make
their discussion subjects as clear as possible to simplify the task
consisting in looking for old topics in the archives. Indeed, remember
that newcomers should - if they respect the rules - look at the archives
to avoid discussing an old subject on the mailing-list.


% repository

%
% ---------- header -----------------------------------------------------------
%
% project       kaneton
%
% license       kaneton
%
% file          /home/mycure/kaneton/view/book/development/repository.tex
%
% created       julien quintard   [thu may 24 20:43:26 2007]
% updated       julien quintard   [wed jun 13 22:32:31 2007]
%

%
% ---------- repository -------------------------------------------------------
%

\subsection{Repository}
\label{section:repository}

The repository contains everything related to the kaneton microkernel
project, in other words, the kaneton source tree described in
\textit{Chapter \ref{chapter:source tree}}. Indeed, the repository contains
the whole history of the kaneton project including the documentation, the
source code but also the students tarballs over the years.

The actual repository is based on the \textit{Subversion} software which
provides far more advanced features than its historical rival \textit{CVS}.

The repository is actually hosted on the \textit{kaneton.org} server which
also contains the web server and everything else related to the management
of the kaneton microkernel project.

The repository is accessed in a secure way through a \textit{SSH} channel.
Indeed, the kaneton \textit{Subversion} repository can be accessed at the
following address:

\begin{verbatim}
  svn+ssh://subversion@repositories.kaneton.org/kaneton
\end{verbatim}

Note that the security is achieved by the use of \textit{SSH} keys. Therefore,
any new contributor should get in touch with an administrator of the
kaneton server in order to obtain an access. Also note that, a test period
could be set up for a new contributor to get the trust of the kaneton
community. For more information, please refer to \textit{Chapter
\ref{chapter:community}}.

A contributor willing to create a \textit{SSH} key shoud simply use this
\textit{Unix} command:

\begin{verbatim}
  $> ssh-keygen -t dsa
\end{verbatim}

For more information about how to use the repository, please refer to the
official \textit{Subversion} documentation. The same goes for the
\textit{SSH} tools suite.

The example below illustrates the checkout of the kaneton repository.

\begin{verbatim}
  $> svn checkout svn+ssh://subversion@repositories.kaneton.org/kaneton
\end{verbatim}

The contributors getting access to the kaneton repositories must behave
properly according to the obvious cooperative development rules. As an
example, a kaneton developer must not perform any commit before making sure
the kaneton microkernel compiles and passes all the tests.

The repository organisation is crucial. Therefore, nothing should be
added, removed or renamed without the permission of the developers in charge
of the repository.

Finally, any commit must come with a log describing the modifications
implied by the commit. These logs must conform to the following syntax.

\begin{verbatim}
  [kaneton/core/segment/]
    o the bug about the permissions was corrected in segment_clone().
    o an algorithm based on a b-tree was added.

  [environment/profile/user/julien.quintard/]
    o some personal configurations were modified.
\end{verbatim}

Following this syntax is very important as an email is sent to the
\textit{kaneton-developers} mailing-list every time a commit is performed.
Therefore, the contributors reading the mailing-list are aware of every
modification in the kaneton source code. This feature can also be used
to review the modifications done by a new contributor in order to help
him doing things in a better way.

Note that there must not be any file with the executable flag permission
enabled. Moreover, scripts files must not contain any \textit{shebang}.
Indeed, the kaneton development environment knows which interpreter to
use for every type of file. It is therefore a non-sense to introduce a
hard-coded path to an interpreter.

Tarball file names must be extended with \textit{.tar} while \textit{bzip2}
compressed tarballs must be extended with \textit{.tar.bz2}.


% wiki

%
% ---------- header -----------------------------------------------------------
%
% project       kaneton
%
% license       kaneton
%
% file          /home/mycure/kaneton/view/book/development/wiki.tex
%
% created       julien quintard   [thu may 24 23:06:02 2007]
% updated       julien quintard   [fri aug  1 15:52:28 2008]
%

%
% ---------- wiki -------------------------------------------------------------
%

\subsection{Wiki}
\label{section:wiki}

A Wiki is used both for external and internal communication. The software
used is called \name{TWiki} and provides a pretty simple syntax with
many additional plugins to customize the website. This solution was used
for historical reasons but also because the \name{TWiki} rendering can
be customized through templates in order to get a final visual close to
classical websites. Thus, the kaneton website looks like an ordinary
website but powered by a Wiki engine.

The Wiki is hosted at \location{http://kaneton.opaak.org} and contains four
webs: an extranet and three intranets. The main web, accessed through the
address above is the external website. This website contains news, papers,
documentation and general information on the kaneton project. The three other
webs are used more as intranets or wikis more than as websites. Two of these
webs are private to the kaneton developers and the kaneton teachers,
respectively. The latter web is intended to the students and contains
documents, links \etc{} about low-level programming, kernel development \etc{}
as well as information about courses related to the kaneton project.

Note that the Wiki reserved for the kaneton developers must be used
instead of the Wiki eventually provided by the project management tool.

Everybody involved in the kaneton project must contribute to the kaneton
website as well as to the kaneton intranets. Indeed, the external communication
is fundamental, even in an open source project and the kaneton website is
the only public communication medium.

New contributors are then asked to register onto the kaneton \name{TWiki}
at \location{http://kaneton.opaak.org}. Once done, the contributor should
inform the person in charge of the kaneton website so that  the contributor's
account is activated. As a result, the contributor will be able to modify
pages of the website and intranets.


% project management

%
% ---------- header -----------------------------------------------------------
%
% project       kaneton
%
% license       kaneton
%
% file          /home/mycure/kane.../book/development/project-management.tex
%
% created       julien quintard   [fri may 25 19:26:17 2007]
% updated       julien quintard   [thu may 31 06:25:56 2007]
%

%
% ---------- project management -----------------------------------------------
%

\subsection{Project Management}
\label{section:project management}

XXX

\begin{comment}
le systeme de tickets/bugs est egalement tres important. chaque ticket se
voit affecte une priorite et il est important de comprendre que pour le
bien global du projet, un developpeur ne peut se contenter de faire ce
qui lui plait, il se doit de contribuer egalement a la resolution de problemes.

encore une fois les tickets doivent suivre une norme.
\end{comment}


%
% ---------- header -----------------------------------------------------------
%
% project       kaneton
%
% license       kaneton
%
% file          /home/mycure/kaneton/view/internship/check/check.tex
%
% created       julien quintard   [wed may 16 18:06:23 2007]
% updated       julien quintard   [thu may 22 16:30:20 2008]
%

%
% ---------- setup ------------------------------------------------------------
%

%
% path
%

\def\path{../..}

%
% template
%

\input{\path/template/internship.tex}

%
% header
%

\lhead{\scriptsize{The kaneton microkernel :: check}}

%
% title
%

\title{The kaneton microkernel :: check}

%
% authors
%

\author{\small{Solal Jacob}}

%
% document
%

\begin{document}

%
% title
%

\maketitle

%
% ---------- abstract ---------------------------------------------------------
%

\begin{abstract}

\indent The test framework ains at mesuring the reliability of the kaneton
microkernel. It is organized around two parts. On the one hand, tests are
written and executed within kaneton, on the other hand, a Python script runs on
another independant system. This document concerns users who want to build a
new tests set and also contributors who would like to upgrade the test
environment itself.

\end{abstract}

%
% --------- text --------------------------------------------------------------
%

\section{Framework architecture}

\indent As the test environment is supposed to check kaneton's reliability, it
cannot assume that kaneton is safe and therefore it cannot be installed on the
tested system. Thus, the test environment has been designed to check kaneton
from another independant system. Both of kaneton and the independant system can
communicate via the serial port.\\
\\
\indent Once the tests are written (kaneton-side), the Makefile needs to be
modified to turn the kernel into debug mode. This mode enables the tests,
executes them and sends their results through the serial port to the script.
When debug mode is enabled, kaneton first initializes the serial driver (default
settings are COM1 at 56kb/s) and then waits for commands from the serial
connection.\\
\\
The test script is written in Python and depends on a C/Python module whose
role is to manage serial communication. It is based on a dedicated protocol
also used in kaneton. All was done to simplify the script development.\\
\\
\indent Strong conventions have been established to write test scripts and to
manage the kaneton Makefile. They must be known and respected to add new tests.
These conventions are deeply detailed in the next paragraphs.


\section{How to add and to launch tests}

\subsection{How to add tests}
As already said, test writting must obey certain rules. Conventions on
directory and file naming were chosen as follow:
\begin{enumerate}
\item A test dedicated to a specific part of kaneton must be created in the same
directory tree than the piece of kaneton it is supposed to check. This new
directory tree must be placed in kaneton/check/.
\item A test is a directory named 01, 02, \ldots, N and containing the following
files:
\begin{itemize}
\item .c: the test itself which is part of kaneton
\item .res: the expected output for the corresponding .c
\item parse\_res.py: a python function which must be modify to parse the
result.
\end{itemize}
\item A file named list and containing all the names of the tests separated by a
single \textbackslash n must be placed in every level of the check directory
tree.
\\
\end{enumerate}
{\bf Example:}\\
Assuming we want to write two tests checking the printf function
which is located in kaneton/klibc/libstring/, we need to create the following
 directory tree:

\begin{verbatim}
    kaneton/check/
                 libs/
                     klibc/
                          listring/
                                  printf/
                                        01/
                                           01.c
                                           01.res
                                           pares\_res.py
                                        02/
                                           02.c
                                           02.res
                                           pares\_res.py
                                        list		 => 01\n02\n
                                  list			 => printf\n
                          list				 => libstring\n
                     list				 => klibc\n
                 list					 => libs\n

\end{verbatim}


\subsection{How to launch tests}
\begin{itemize}
\item Install the Python module by running ./domodules in kaneton/tools/python/
\item Select the tests to enable by adding their name in the appropriate `list`
text file (see Directory Convention for further details).
\item Configure kaneton to run in debug mode:
\begin{description}
\item environment/users/user.conf: delete the line: {\tt override \_CHECK\_}
\item environment/users/conf.h: add the line:  {\tt \#define SERIAL}
\end{description}
\item Build the kernel
\item During kernel execution, run ./check.py to get the tests results
\end{itemize}


\section{Serial driver and the test framework on kaneton}

The serial driver can be found in kaneton/core/kaneton/debug/serial.c.\\
Initialization is performed by passing the desired com\_port and baud\_rate to
the serial\_init function. Basic values as SERIAL\_8N1 and SERIAL\_FIFO\_8 are
defined for recurrent usages; you will find them in
kaneton/core/include/kaneton/serial.h.\\
Once the driver is setup, the functions serial\_read and serial\_write permit
to transfert data trough the serial port (see serial.h for further
information). Keep in mind that this serial driver was written in pole mode. So
it requires a 1GHz processor to achieve high-speed (56kbp/s) communication
without trouble.

The test framework routines can be found in
kaneton/core/kaneton/debug/debug.c.\\
A call to the debug\_init function performs all initializations needed by the
test framework. It calls serial\_init, printf\_init, allocates sufficient memory
and waits for new messages in a never ending loop. printf\_init permits to
redirect printf output towards the serial port using serial\_put function as a
parameter. debug\_recv reads in input until it receives "command". Then it
waits for the address of the command to execute, executes it and uses
serial\_put to send the command result through the serial port.\\


\section{Python script and C/Python modules}

check.py is the script which analyzes the tests results. It  can be found in
kaneton/check/. The C/Python kserialmodule in kaneton/tools/python/ must be
installed prior to launch ./check.py. Just run ./domodules to do so. This
module permits Python applications to use serial\_init, serial\_recv and
serial\_send functions, and thus to communicate with kaneton. The line\\
\\
\indent{\tt from kserial import *}\\
\\
in the main function of check.py shows how to use the kserial module.\\
\\
\indent The script check.py first initializes the serial communication by a
call to serial\_init("/dev/ttyS0"), what implies that we run check.py on Linux
on COM1. Then the script uses the function ListTest to recursively find the tests
it will have to execute. All the tests found are added to a list. This list is
parsed and test functions are sequentially sent to kaneton for execution. Their
result is compared to their corresponding .res by their associated
parse\_res.py script. Every failure or success is stored in order to compute
and display totals.

\end{document}

%%
%% licence       kaneton licence
%%
%% project       kaneton
%%
%% file          /home/mycure/kaneton/view/papers/kaneton/glossary.tex
%%
%% created       matthieu bucchianeri   [mon jan 30 17:34:32 2006]
%% updated       julien quintard   [thu mar  2 13:07:22 2006]
%%

%
% glossary
%

\chapter{Glossary}

\subsubsection{as}

An address space is an entity representing addressable memory,
physical and virtual, associated to a task. In kaneton, an address space
is composed of a set of segments and a set of regions.


%
% epilogue
%

XXX ?

\end{document}

%%
% ---------- header -----------------------------------------------------------
%
% project       kaneton
%
% license       kaneton
%
% file          /home/mycure/kaneton/view/book/development/tools.tex
%
% created       julien quintard   [sun may 20 14:48:11 2007]
% updated       julien quintard   [thu may 31 06:46:06 2007]
%

%
% ---------- tools ------------------------------------------------------------
%

\chapter{Tools}
\label{chapter:tools}

This chapter describes every tool kaneton contributors use on a daily-basis.

\newpage

%
% ---------- text -------------------------------------------------------------
%

%
% internal
%

\section{Internal}

The kaneton project contains several tools which makes the developer's life
easier. This section describes these tools in order for the contributor to
use it but also to improve them.

% environment

%
% ---------- header -----------------------------------------------------------
%
% project       kaneton
%
% license       kaneton
%
% file          /home/mycure/kaneton/view/book/development/environment.tex
%
% created       julien quintard   [sun may 20 14:49:26 2007]
% updated       julien quintard   [sun may 20 18:07:59 2007]
%

%
% ---------- environment ------------------------------------------------------
%

\subsection{Environment}

Over the years, the kaneton microkernel evolved, starting with a very simple
introduction to low-level programming and finally to a complete microkernel
development.

kaneton people wanted to lead students to a complete microkernel development
to finally introduce distributed computing. This would not have been possible
if students had to build an entire development environment because developing
such an environment is a whole project by itself.

As a result, kaneton people decided to provide students a complete development
environment. The kaneton development environment is composed of make files,
python scripts and configuration files. This development environment can be
considered as one of the major kaneton tools since contributors use it
everytime.

The kaneton development environment aims at providing an easy and portable
way for managing the kaneton microkernel project from a development point
of view. Therefore, the kaneton environment provides everything necessary
for compiling, assembling, etc.. These tasks highly rely on the underlying
running operating system as well as on the kaneton microkernel's target
microprocessor. Moreover, the user could need to redefine some behaviours
depending on its personal operating system configuration to use a specific
C compiler for instance.

The kaneton development environment provides a layered organisation of
profiles, each profile defining variables and functions used by the final
environment engine. The goal of the layered model is to allow layers to
override the definitions of lower layers.

%
% profiles
%

\subsubsection{Profiles}

A configuration is composed of profiles including a \textit{host} profile which
describes the behaviour of the underlying operating system, a \textit{kaneton}
profile which parameters the kaneton core and a \textit{user} profile which
permits the user to redefine lower layers' declarations.

These profiles eventually hold sub-profiles which actually define variables
and functions. These actual profiles are accessed according to user-defined
shell variables.

% host

\subsubsubsection{Host}

The \textit{host} profile essentially describes how to perform basic tasks:
compile, assemble, change the current directory, display a message etc.. These
tasks rely on the operating system currently running as well as on the target
processor which kaneton will be built for. For these reasons, there are
several host sub-profiles.

Let us suppose a developer is running a \textit{Linux} operating system and
that kaneton will be built for running on a \textit{PowerPC} microprocessor. In
such a case, the C compiler program will be different depending on the
microprocessor \textit{Linux} is running on. Indeed, if Linux is running on
a \textit{PowerPC} microprocessor, then using the default compiler should
produce \textit{PowerPC} object files. This is well-known to be the common
compiling way. On the other hand, if \textit{Linux} is running on a
different microprocessor, then a cross-compiler must be used to produce
binary objects targeting a specific different microprocessor architecture.

To avoid this issue, a \textit{host} sub-profile name is composed of two parts
separated by a slash. The first part is the name of the operating system and
the latter is a pair source/target processors separated by a period. For
example, \textit{linux/ia32.ppc} names a \textit{host} profile running the
\textit{Linux} operating system on a \textit{Intel 32-bit} microprocessor
which aims at building a kaneton microkernel for the \textit{PowerPC}
target architecture. Needless to say that \textit{linux/ia32.ia32} represents
a non cross-compiling environment.

To avoid configuration duplications, it is common to see the configuration
file of a host sub-profile to include files of the parent directory as
shown below:

\begin{verbatim}
  linux/
    linux.desc
    linux.conf
    linux.mk
    linux.py
    ia32.ia32/
      virtual -> .
      optimised -> .
      smp -> .
      ia32.desc
      ia32.conf
      ia32.mk
      ia32.py
    ia32.mips64/
      mips64.desc
      mips64.conf
      mips64.mk
      mips64.py
\end{verbatim}

Note that the files \textit{linux.*} are not directly included by the
development environment engine since \textit{linux} is not a valid host
profile name.

Two host profiles are illustrated here. The first one is named
\textit{linux/ia32.ia32} while the second's name is \textit{linux/ia32.mips64}.

For example, the \textit{linux/ia32.mips64} \textit{host} profile represents a
\textit{Linux} operating system running on a \textit{Intel 32-bit}
microprocessor while kaneton is built for a \textit{MIPS 64-bit} target
architecture. This profile is likely to include the \textit{linux.*} of the
parent directory since there are not much difference between all the
\textit{linux/*.*} \textit{host} profiles. However, such a profile will
certainly redefine the binary paths of the C compiler, linker etc.. in order
to produce \textit{MIPS 64-bit} binary objects.

To conclude, the \textit{host} sub-profile is accessed by the following
construct:

\begin{verbatim}
  profile/host/${KANETON_HOST}/${KANETON_ARCHITECTURE}
\end{verbatim}

With, for instance, the following values:

\begin{verbatim}
  KANETON_HOST = linux/ia32
  KANETON_ARCHTECTURE = ia32/virtual
\end{verbatim}

Note that the possibility to include files in the configuration syntax allows
very similar profiles to share a huge amount of definitions.

% kaneton

\subsubsubsection{kaneton}

The \textit{kaneton} profile is composed of three sub-profiles: \textit{core},
\textit{platform} and \textit{architecture}.

The \textit{core} sub-profile contains variables for parameterizing the
kaneton core. The \textit{platform} and \textit{architecture} sub-profiles
focus on the configuration of the platform- and architecture-dependent code
of the kaneton microkernel.

The user-defined shell variables \textit{\$\{KANETON\_PLATFORM\}} and
\textit{\$\{KANETON\_ARCHITECTURE\}} are used to address the \textit{platform}
and \textit{architecture} sub-profiles, respectively.

% user

\subsubsubsection{User}

Let us suppose that a developer would like the kaneton microkernel to
use a specific memory management entirely based on a \textit{Slab Allocator}
and with all microprocessor optimisations enabled. These user-specific
configurations are actually allowed by the \textit{user} profile.

The user-defined shell variable \textit{\$\{KANETON\_USER\}} defines the name
of the \textit{user} profile. This profile contains user-specific
configurations allowing a contributor to overwrite lower layer defintions
in order to specialise a behaviour.

The kaneton project also provides a tool allowing developers to configure
their development environment. This tool is named \textbf{configure} and is
available from the kaneton project root directory.

%
% requirements
%

\subsubsection{Requirements}

The whole kaneton development environment needs exactly two fundamental tools
to work. The first one is \textit{GNU make}, used to build powerful make files,
and the second one is \textit{Python}, used to write portable scripts. If an
operating system has these two tools, then kaneton can certainly be developed
on it.

As said previously, the user has to specify some shell variables which are
used by the kaneton development environment engine. These variables are
described below:

\begin{itemize}
  \item
    \textbf{\$\{KANETON\_USER\}}: the name of the kaneton developer.

    A \textit{user} profile name must be composed of the first name, a period
    and finally, the last name of the developer.
  \item
    \textbf{\$\{KANETON\_HOST\}}: the name of the host which is composed of
    a couple operating system/microprocessor.
  \item
    \textbf{\$\{KANETON\_PYTHON\}} contains the path of the python binary.

    This path is necessary since the very first scripts which set up the
    configured environment are based on python scripts.
  \item
    \textbf{\$\{KANETON\_PLATFORM\}}: the name of the target platform.
  \item
    \textbf{\$\{KANETON\_ARCHITECTURE\}}: the name of the target architecture.
\end{itemize}

Note that once the configured environment is set up, these variables are
no longer used by the kaneton environment engine. Indeed, instead, the
kaneton environment operations are based on the \textit{host} profile on
which rely the configured environment.

The profiles names must all be lowercase. Below are some examples of what
could contain these variables:

\begin{verbatim}
  KANETON_USER='julien.quintard'

  KANETON_HOST='linux/ppc'
  KANETON_HOST='windows~cygwin/ia32'

  KANETON_PYTHON='/usr/bin/python'

  KANETON_PLATFORM='ibm-pc'
  KANETON_PLATFORM='sgi/o2'
  KANETON_PLATFORM='sgi/octane'

  KANETON_ARCHITECTURE='mips64'
  KANETON_ARCHITECTURE='ia32/virtual'
  KANETON_ARCHITECTURE='ia32/smp'
\end{verbatim}

%
% organisation
%

\subsubsection{Organisation}

The development environment configuration files and scripts are located in
the \textit{environment/} directory. The directory contains the three
following scripts:

\begin{verbatim}
  critical.py
  init.py
  clean.py
\end{verbatim}

The \textit{critical.py} script essentially generates a configured development
environment. The result of this generation are two files called
\textit{env.mk} and \textit{env.py} which contains the configured environment
variables and functions for the \textit{Make} files and \textit{Python}
scripts, respectively. This file is called critical because it does not rely
on the portable development environment as it generates it.

The \textit{init.py} script relies on the file \textit{env.py} previously
generated. This script set up everything necessary for building the
kaneton microkernel based on the configured environment.

Finally, the \textit{clean.py} script cleans everything installed by the
\textit{init.py} script and removes the generated configured environment files.

The generation of the configured environment is done by going through
the configuration files of all the profiles and sub-profiles associated
to the user configuration. In other words, the kaneton environment engine
processes the configuration files according to the layered organisation
described below, starting with the lowest layer thourgh the highest one.

\begin{verbatim}
  profile/
  profile/host
  profile/host/${KANETON_HOST}/${KANETON_ARCHITECTURE}
  profile/kaneton
  profile/kaneton/core
  profile/kaneton/platform
  profile/kaneton/platform/${KANETON_PLATFORM}
  profile/kaneton/architecture
  profile/kaneton/architecture/${KANETON_ARCHITECTURE}
  profile/user
  profile/user/${KANETON_USER}         
\end{verbatim}

\begin{verbatim}
XXX $ XXX
\end{verbatim}

In this layered organisation, a variable defined in, for instance, the
\textit{host} profile could be overwritten anywhere in the upper layers
\textit{kaneton}, \textit{kaneton/architecture/\$\{KANETON\_ARCHITECTURE\}},
\textit{user} etc..

The \textit{host} and \textit{kaneton} profiles are theoretically completed
separated. However, the environment engine does not check for such
unauthorised overridings. Therefore the \textit{core} configuration could
override a variable previously defined in the \textit{host} profile.

Finally, the \textit{user} profile can override any definition adjusting the
environment to his needs.

The environment engine looks for the following types of files in the
kaneton environment profile directories:

\begin{itemize}
  \item
    \textbf{.conf}: the \textit{configuration} files gathered by the
    development environment engine for generating the configured environment
    files.
  \item
    \textbf{.desc}: these \textit{description} files contain descriptions of
    the variables of the current profile or sub-profile. These descriptions
    are used by the \textit{configure} tool.
  \item
    \textbf{.mk}: the \textit{Make} files usually contains the implementation
    of the kaneton \textit{Make} interface.
  \item
    \textbf{.py}: the \textit{Python} files usually contains the
    implementation of the kaneton \textit{Python} interface.
\end{itemize}

The engine supposes that there is no variable or function overriding in
a single profile. more precisely, if there are more than a single
configuration file in a directory, the engine cannot guarantee anything
on the order these files will be processed. As a result, the overridings
could differ depending on the processing order.

The kaneton development environment engine first gathers the
\textit{configuration} files and process them creating an in-memory list of
configuration variables. Then it generates the configured environment files
\textit{env.mk} and \textit{env.py}. Indeed, the engine outputs the
configuration variables in each file and then append the content of the
\textit{Make} files and \textit{Python} files to the configured environment
file \textit{env.mk} and \textit{env.py}, respectively.

Note that the \textit{description} files are not directly used by the
environment engine.

%
% syntaxes
%

\subsubsection{Syntaxes}

% description

\subsubsubsection{Description}

The \textit{description} files describe the environment variables in order
to specify what kind of value a variable can take etc..

Each variable description is contained between braces. A description is
composed of fields, some are mandatory and some are optional.

Examples of description for variables named \textit{\_FOO\_}, \textit{\_BAR\_}
and \textit{\_CHICHE\_} are given next:

\begin{verbatim}
  _FOO_ :: the foo flag
  {
    <on> -D_FOO_FLAG_=1
    <off> -D_FOO_FLAG_=0

    This is a description of the two-state variable _FOO_.
  }

  _BAR_ :: the bar parameter
  {
    <simple> BAR_SIMPLE
    <normal> BAR_NORMAL
    <optimised> BAR_OPTIMISED

    This is another parameter which can take three values: simple,
    normal and optimised.
  }

  _CHICHE_ :: the most powerful optimisation
  {
    This is the magic kaneton optimisation.
  }
\end{verbatim}

Note that environment engine never takes these descriptions into account.
Indeed, this is the r\^ole of the \textit{configure} tool.

In this syntax, variables are classified according to the type of value
they can take: \textit{state}, \textit{set} and \textit{any}.

A \textit{state} variable is either activated or deactivated. If the two
fields \textit{<on>} and \textit{<off>} are present, then, this variable
is considered as a \textit{state} variable.

A \textit{set} variable can take any value of a given list of values. This
is the most common type of variables. In this case, each field detected
is considered as a potential value.

Finally, a \textit{any} variable represents a variable which can take any
value. This case is detected by the absence of value field in the description.

The value fields follow the next pattern:

\begin{verbatim}
  <name> value
\end{verbatim}

The \textit{name} is displayed by the \textit{configure} tool to the final
user while the \textit{value} value is affected to the described variable. This
way, the tool can displayed more human-readable description. For instance,
if the \textit{optimised} option is chosen, then the \textit{BAR\_OPTIMISED}
will be affected to the \textit{\_BAR\_} variable.

The name of the variable follows the pattern:

\begin{verbatim}
  variable :: name
\end{verbatim}

Once again, this construct was introduced to avoid displaying internal
non-user-friendly variable names. The \textit{variable} will not be directly
displayed by the \textit{configure} tool which will use the \textit{name}
string instead.

Finally, any remaining text between the braces is considered as a variable's
description text.

% configuration

\subsubsubsection{Configuration}

The \textit{configuration} files contains the actual variable definitions
through a very simple syntax.

The syntax allows both assignments and completion of variables' value
as show in the next example:

\begin{verbatim}
  FOO = bar
  FOO += baz
  FOO = kaneton
\end{verbatim}

The \textit{FOO} variable first took the initial value \textit{bar}. Then,
the value \textit{baz} was added to the previous \textit{FOO}'s value
leading the the value \textit{bar baz}. Finally, the last assignment
overwrite the previous definitions by setting the value of \textit{FOO}'s
variable to \textit{kaneton}.

The configuration syntax enables the use of variables in values. These
variables can be both environment variable or shell variable. The following
example illustrates it.

\begin{verbatim}
  BAR = ${FOO} is a very powerful microkernel
  SH = the shell currently used is $(SHELL)
\end{verbatim}

The reader certainly notice the \textit{\$\{\}} construct is used to reference
a kaneton environment variable while the \textit{\$()} one references a shell
variable.

Finally, a configuration file can also include another file using the
\textit{include} statement:

\begin{verbatim}
  include ../an/other/file/far/../far/../away
\end{verbatim}

This construct is very useful to centralize the definitions common to
multiple sub-profiles in a single location.

Note that kaneton environment variables start and end with an underscore
for avoiding naming collisions.

% make

\subsubsubsection{Make}

The \textit{Make} files must implement the whole kaneton \textit{Make}
interface which will be described next.

The syntax used in these files is based on the \textit{GNU Make} syntax.

% python

\subsubsubsection{Python}

The \textit{Python} files must implement the whole kaneton \textit{Python}
interface.

The syntax used in these files is based on the \textit{Python} syntax.

%
% interfaces
%

\subsubsection{Interfaces}

% make

\subsubsubsection{Make}

In this section we will detail the make interface that every host profile
must implement. The reader should look closer to the host profiles already
implemented.

Since the \textit{GNU Make} syntax does not provide any name space feature,
every kaneton \textit{Make} function is prefixed by \textit{env\_} in order
to avoid name conflicts.

\function{env\_display}{(\argument{color},
                         \argument{action},
                         \argument{file},
                         \argument{indentation},
                         \argument{options})}
         {
	   This function display a message representing an action performed
	   by the kaneton \textit{Make} interface.

	   \-

	   The option \textit{\$(OPTION\_NO\_NEWLINE)} can be used not to
	   output the trailing newline.
	 }

\function{env\_cd}{(\argument{directory},
                    \argument{options})}
         {
	   This function changes the current working directory.
	 }

\function{env\_contents}{(\argument{file},
                          \argument{options})}
         {
	   This function returns the contents of the file \argument{file}.
	 }

\function{env\_launch}{(\argument{file},
                        \argument{arguments},
                        \argument{options})}
         {
	   This function launches a new program/script/make etc..

	   \-

	   This function must look at the file name in order to determine
	   how to launch it.

	   \-

	   For \textit{Python} files, this function must take care of
	   setting and exporting the \textit{PYTHONPATH} shell environment
	   variable with a value including the
	   \textit{\_PYTHON\_INCLUDE\_DIR\_} kaneton environment variable.
	 }

\function{env\_preprocess}{(\argument{preprocessed file},
                            \argument{c file},
                            \argument{options})}
         {
	   This function launches the C preprocessor the \argument{c file}
	   and generates the \argument{preprocessed file}.
	 }

\function{env\_compile-c}{(\argument{object file},
                           \argument{c file},
                           \argument{options})}
         {
	   This function compile a \argument{c file} generating an
	   \argument{object file}.
	 }

\function{env\_lex-l}{(\argument{c file},
                       \argument{lex file},
                       \argument{options})}
         {
	   This function generates a \argument{c file} from a
	   \argument{lex file}.
	 }

\function{env\_assemble-S}{(\argument{object file},
                            \argument{S file},
                            \argument{options})}
         {
	   This function assemble an \argument{S file}.
	 }

\function{env\_assemble-asm}{(\argument{object file},
                              \argument{asm file},
                              \argument{options})}
         {
	   This function assemble an asm file.

	   \-

	   The option \textit{\$(ENV\_OUTPUT\_OBJECT)} forces the function
	   to generate an object file while the
	   \textit{\$(ENV\_OUTPUT\_BINARY)} option forces the output to be
	   a pure binary file.
	 }

\function{env\_static-library}{(\argument{static library file name},
                                \argument{object files and/or libraries},
                                \argument{options})}
         {
	   This function builds a static library from object files.
	 }

\function{env\_dynamic-library}{(\argument{dynamic library file name},
                                 \argument{object files and/or libraries},
                                 \argument{options})}
         {
	   This function builds a dynamic library from object files and/or
	   libraries.
	 }

\function{env\_executable}{(\argument{executable file name},
                            \argument{object files and/or libraries},
                            \argument{layout file},
                            \argument{options})}
         {
	   This function builds a executable from object files and/or
	   libraries. The \argument{layout file} describes where to
	   place the different data: code, read-only data, stack etc..

	   \-

	   The option \textit{\$(ENV\_OPTION\_NO\_STANDARD)} tells the function
	   not to use the operating system standard stuff: libraries, includes
	   etc..
	 }

\function{env\_archive}{(\argument{archive file name},
                         \argument{object files},
                         \argument{options})}
         {
	   This function builds an archive of object from multiple
	   \argument{object files}.
	 }

\function{env\_remove}{(\argument{files},
                        \argument{options})}
         {
	   This function removes the files in the list.
	 }

\function{env\_purge}{()}
         {
	   This function just cleans the current working directory from
	   unecessary files.
	 }

\function{env\_prototypes}{(\argument{files},
                            \argument{options})}
         {
	   This function generates prototypes in relation with the given
	   \argument{files}.
	 }

\function{env\_dependencies}{(\argument{files},
                              \argument{output},
                              \argument{options})}
         {
	   This function generates dependencies for the \argument{files}
	   by building a \textit{Make} dependency file named \argument{output}.
	 }

\function{env\_version}{(\argument{file})}
         {
	   This function generates a version \argument{file} from the operating
	   system's informations: user, host, date etc..
	 }

\function{env\_link}{(\argument{link},
                      \argument{file},
                      \argument{options})}
         {
	   This function creates a link \argument{link} to the \argument{file}.
	 }

\function{env\_compile-tex}{(\argument{file},
                             \argument{options})}
         {
	   This function compiles the file \argument{file}.tex and
	   will generate a readable document.
	 }

\function{env\_paper}{(\argument{file},
                       \argument{options})}
         {
	   This function builds a \textit{paper} by calling the
	   \textbf{env\_compile-tex()} function.
	 }

\function{env\_lecture}{(\argument{file},
                         \argument{options})}
         {
	   This function builds a \textit{lecture} document by calling the
	   \textbf{env\_compile-tex()} function.
	 }

\function{env\_subject}{(\argument{file},
                         \argument{options})}
         {
	   This function builds a \textit{subject} by calling the
	   \textbf{env\_compile-tex()} function.
	 }

\function{env\_correction}{(\argument{file},
                            \argument{options})}
         {
	   This function builds a \textit{correction} document by calling the
	   \textbf{env\_compile-tex()} function.
	 }

\function{env\_view}{(\argument{file},
                      \argument{options})}
         {
	   This function launches a viewer for the readable document
	   generated by the function \textbf{env\_compile-tex()}.
	 }

% python

\subsubsubsection{Python}

In this section we will detail the kaneton \textit{Python} interface that
every \textit{host} profile must implement.

The \textit{Python} language was designed in a portable way. For this
reason, the major part of the \textit{Python} interface is implemented
by the \textit{host} generic profile.

Note that the \textit{Python} language provides modularity through packages.
Therefore, each \textit{Python} script has to import the \textit{env} package
generated by the development environment engine. Then, environment functions
and variables are accessed through this package.

Below are described the functions implemented by the \textit{env} package.

\function{display}{(\argument{header},
                    \argument{text},
                    \argument{options})}
         {
	   This function outputs some text to the screen depending on the
	   header \textit{HEADER\_NONE}, \textit{HEADER\_OK},
	   \textit{HEADER\_ERROR}, \textit{HEADER\_INTERACTIVE}.
	 }

\function{contents}{(\argument{file},
                     \argument{options})}
         {
	   This function returns the contents of the \argument{file}.
	 }

\function{temporary}{(\argument{options})}
         {
	   This function creates a temporary file system object.

	   \-

	   The options \textit{OPTION\_FILE} and \textit{OPTION\_DIRECTORY}
	   specify which type of object to create.
	 }

\function{cwd}{(\argument{options})}
         {
	   This function returns the path of the current working directory.
	 }

\function{input}{(\argument{options})}
         {
	   This function waits for an input.
	 }

\function{launch}{(\argument{file},
                   \argument{arguments},
                   \argument{options})}
         {
	   This function launches a new program/script/make file etc..

	   \-

	   This function must look at the file name in order to determine
	   how to launch it.

	   \-

	   For \textit{Python} files, this function must take care of
	   setting and exporting the \textit{PYTHONPATH} shell environment
	   variable with a value including the
	   \textit{\_PYTHON\_INCLUDE\_DIR\_} kaneton environment variable.
	 }

\function{copy}{(\argument{source},
                 \argument{destination},
                 \argument{options})}
         {
	   This function copies the file \argument{source} to
	   \argument{destination}.
	 }

\function{link}{(\argument{source},
                 \argument{destination},
                 \argument{options})}
         {
	   This function builds a link between the file \argument{source}
	   and the file \argument{destination}.
	 }

\function{remove}{(\argument{target},
                   \argument{options})}
         {
	   This function removes the \argument{target} which can be either
	   a file or a directory.
	 }

\function{list}{(\argument{directory},
                 \argument{options})}
         {
	   This function lists the file system objects contains in the
	   \argument{directory}.

	   \-

	   The options \textit{OPTION\_FILE} and \textit{OPTION\_DIRECTORY}
	   specify which type of object to list.
	 }

\function{cd}{(\argument{directory},
               \argument{options})}
         {
	   This function changes the current working directory to
	   \argument{directory}.
	 }

\function{search}{(\argument{directory},
                   \argument{pattern},
                   \argument{options})}
         {
	   This function looks for files matching the given \argument{pattern}.

	   \-

	   The options \textit{OPTION\_FILE} and \textit{OPTION\_DIRECTORY}
	   specify which type of object to list while the
	   \textit{OPTION\_RECURSIVE} option tells the function to explore
	   the whole file system sub-tree.
	 }

\function{pack}{(\argument{directory},
                 \argument{file},
                 \argument{options})}
         {
	   This function makes an archive \argument{file} of the
	   directory \argument{directory}.
	 }

\function{unpack}{(\argument{directory},
                   \argument{file},
                   \argument{options})}
         {
	   This function extracts the archive \argument{file} into the
	   directory \argument{directory}, if specified.
	 }

\function{mkdir}{(\argument{directory},
                  \argument{options})}
         {
	   This function builds a new directory named \argument{directory}.
	 }

\function{load}{(\argument{file},
                 \argument{device},
                 \argument{path},
                 \argument{options})}
         {
	   This function copies the \argument{file} on the specificed
	   \argument{device}, more precisly at the location \argument{path}.
	   The device can be virtual: an image.

	   \-

	   The options \textit{OPTION\_DEVICE} and \textit{OPTION\_IMAGE}
	   specify on which type of device the file must be copied.
	 }

\function{stamp}{(\argument{format},
                  \argument{options})}
         {
	   This function returns a date following the given \argument{format}.
	 }

\function{record}{(\argument{log},
                   \argument{time},
                   \argument{options})}
         {
	   This function starts recording a session and outputs
	   the text into the file \argument{log} while the timings
	   are output in the file \argument{time}.
	 }

\function{play}{(\argument{log},
                 \argument{time},
                 \argument{options})}
         {
	   This function plays a previously recorded session where
	   the files \argument{log} and \argument{time} hold the
	   text and timings.
	 }

\function{locate}{(\argument{file},
                   \argument{options})}
         {
	   This function tries to locate the program \argument{file}
	   on the system.
	 }

\function{path}{(\argument{path},
                 \argument{options})}
         {
	   This function returns information on the given \argument{path}.

	   \-

	   The options \textit{OPTION\_FILE} and \textit{OPTION\_DIRECTORY}
	   specify which information the caller is interested in.
	 }


% configure

%
% ---------- header -----------------------------------------------------------
%
% project       kaneton
%
% license       kaneton
%
% file          /home/mycure/kaneton/view/book/development/configure.tex
%
% created       julien quintard   [tue may 22 22:34:37 2007]
% updated       julien quintard   [wed may 30 19:26:54 2007]
%

%
% ---------- configure --------------------------------------------------------
%

\subsection{Configure}

The \textit{configure} tool provides the final user a very user-friendly
software for customizing its development environment.

Recall the development environment is basically composed of three profiles:
\textit{host} which describes the operating system behaviour, \textit{kaneton}
which parameterizes the kaneton microkernel and \textit{user} which contains
some user-specific definitions.

The kaneton development environment is thus used to configure the environment
behaviour as well as the kaneton microkernel itself.

The \textit{environment/} directory, and more precisely the environment
profiles, contain \textit{description} files which actually describe the
environment variables. These files are not used by the development
environment but rather by the \textit{configure} tool.

The \textit{configure/} directory is composed of \textit{frame} files
which contain frame descriptions. A frame can be seen as a menu presented
to the final user. A frame is composed of sub-frame and variable entries.

The \textit{configure} tool works as follow. It starts by processing the
environment development configuration files as the environment engine did
for the generation of the configured environment files. Note that the
\textit{configure} tool also processes the description files. Also, it
focuses on variables and actually ignores the interfaces' functions.

Once this step is done, the tool gets a list of configured and fully described
variables. Then, the \textit{configure} tool displays the first frame and
waits for the user to choose an entry.

The user has the possibility to either move to another menu - if any sub-frame
entry is present - or configure a variable of the list. If the user chooses
to configure a variable, then, the tool displays information based on the
variable's description.

Every modifications of the development environment are private to the actual
user. Therefore, any variable modification adds or modifies an entry in the
related \textit{user} profile.

Note that the \textit{configure} tool is not a environment configuration
files editor. Indeed, this tool targets final users and therefore has to
be as simple as possible.

The basic \textit{configure} behaviour consists in displaying the final
variable's value. If the user enters a new value, no matter whether there is
a relation with its previous value, the tool creates/modifies an entry in the
\textit{user} profile's configuration file overriding any previous definition.

For instance, consider the \textit{\_FOO\_} development environment variable
with the following configuration definition:

In \textit{profile/environment.conf}:

\begin{verbatim}
  _FOO_                         =                       initial
\end{verbatim}

In \textit{profile/core/core.conf}:

\begin{verbatim}
  _FOO_                         +=                      addon
\end{verbatim}

Let us suppose the user enters the following value instead of the current
one: \verb|initial addon|

\begin{verbatim}
  _FOO_                         =                       initial new
\end{verbatim}

Then, the \textit{configure} tool creates a new entry into the \textit{user}
profile configuration file:

\begin{verbatim}
  _FOO_                         =                       initial new
\end{verbatim}

Finally, note that when the \textit{configure} tool is launched, it first
tries to detect whether the user is a newcomer or not. If it is, then the
tool asks the user to create a new \textit{user} profile, step by step. These
actions are performed in the \textit{critical.py} script of the
\textit{configure/} directory.

% requirements

\subsubsubsection{Requirements}

The \textit{configure} tool relies on the \textit{Dialog} software which
is present on many \textit{Unix} systems. Indeed, the \textit{configure}
tool is a user-friendly configuration utility.

% syntax

\subsubsubsection{Syntax}

The syntax of the frame description files is based on \textit{YAML}. Therefore,
the \textit{Python} \textit{PyYAML} module needs to be set up.

As said previously, a frame is composed of sub-frame and variable entries. A
sub-frame entry contains a name and a path to the sub-frame description file
while a variable entry only contains the name of the variable. This variable
name is then used to retrieve the variable description.

The example below illustrates this very simple syntax:

\begin{verbatim}
[XXX]
  - title: Segment Manager
    description: |
      This frame contains configuration about the core
      segment manager

  - frame: optimisations
    path: subsections/optimisations.desc

  - frame: machine dependent
    path: subsections/machine.desc

  - variable: _FOO_

  - variable: _BAR_

  - variable: _CHICHE_
\end{verbatim}


% view

%
% ---------- header -----------------------------------------------------------
%
% project       kaneton
%
% license       kaneton
%
% file          /home/mycure/kaneton/view/book/development/view.tex
%
% created       julien quintard   [wed may 23 00:36:53 2007]
% updated       julien quintard   [mon may  4 19:43:15 2009]
%

%
% ---------- view -------------------------------------------------------------
%

\subsection{View}
\label{section:view}

The \name{view} tool serves as a document database as well as a tool for
building and displaying documents in an easy way.

The kaneton documents are classified, each directory corresponding to a
class of documents. Below are listed the subdirectories of the \location{view/}
directory.

\begin{verbatim}
  bibliography/
  book/
  exam/
  feedback/
  figures/
  internship/
  lecture/
  logo/
  package/
  paper/
  talk/
  template/
\end{verbatim}

The \location{template/} directory contains templates for every class of
document. The \location{bibliography/} and \location{logo/} directories
contain, obviously, the bibliography which is common to all the documents, and
the logos, respectively. The \location{figures/} directory contains figures
common to all the documents while the \location{package/} directory contains
additional {\LaTeX} packages.

The directories \location{curriculum/}, \location{exam/} and
\location{feedback/} contain documents in relation with teaching. The
\location{curriculum/} directory contains documents such as the educational
project year planning \etc{} The \location{feedback/} directory contains
documents which are distributed to the students at the end of the kaneton
project in order to get feedback for improving the project for the next years.
Needless to say the \location{exam/} directory contains everything related to
examinations while the \location{talk/} directory contains conference talks
and various presentations of the education project for instance.

The other directories contain the actual kaneton documentation. The
\name{books} represent the main documents targeting any public:
contributors, teachers, students \etc{} The \name{papers} are lighter
documents intended to present a specific feature, design \etc{} The
\name{lectures} are the courses materials, generally composed of
presentation slides. Finally, the \name{internship} documentation is
composed of documents written by people partially involved in the kaneton
project.

Any document is composed of a \name{Make} file and one or more
{\LaTeX} files. The \name{Make} file always has the same form
with little variations depending on the type of document. For more information
on the rules applying to the \name{Make} and {\LaTeX} files, please
refer to their respective sections: \reference{Section \ref{section:make}}
and \reference{Section \ref{section:python}}.

The \name{view} tool basically starts looking for \location{.tex} files
and builds a list of directories containing documents. Then, it provides
to the user the possibility to build and display a given document. If no
document name is given on the command line, then the tool draws a list
of the available documents.

People contributing to the kaneton documents must take care of following
the rules in relation with the {\LaTeX} language. Moreover, contributors
should look at the existing documents to understand to logic behind all
these rules.

Finally, note that nobody should create a document without discussing it
on the mailing-list first. Especially, be very careful in naming your
documents as people took good care of this directory in order to avoid
it to become messy.

If a document already exists with the same name, then go through the
mailing-list in order to decide whether to keep the current version. If
people decide to keep a document, then, the contributor in charge of writing
the new one should re-organise the documents by creating archives for
each year.


% export

%
% ---------- header -----------------------------------------------------------
%
% project       kaneton
%
% license       kaneton
%
% file          /home/mycure/kaneton/view/book/development/export.tex
%
% created       julien quintard   [wed may 23 18:58:41 2007]
% updated       julien quintard   [thu may 24 20:44:48 2007]
%

%
% ---------- export -----------------------------------------------------------
%

\subsection{Export}

The \textit{export} tool was introduced for making the process of releasing
easier. The tool takes an argument specifying the type of target release.

Recall the kaneton microkernel project is used as a material for operating
system courses. The source code of the microkernel is distributed to
the students with some parts missing. Then, students have to re-write
these pieces of code in order to prove their well-understanding of the
kernel internals. Additionnaly, the kaneton project is also a research project
in operating systems design.

As a result, the \textit{export} tool sometimes has to build a release
with pieces of code removed, sometimes not. Below are listed the different
type of release:

\begin{itemize}
  \item
    \textbf{backup}: this release type is basically a bare backup of
    the kaneton microkernel project source code.
  \item
    \textbf{dist}: the distribution release is performed by removing
    the repository-specific stuff.
  \item
    \textbf{k}$\gamma$\textbf{,}$\epsilon$: this type of release is intended
    to students. Therefore, repository-specific stuff is removed. Also
    teaching materials such as courses, testing scripts, cheating scripts
    etc.. - specified in the kaneton development environment variable
    \textit{\_HIDDEN\_} - are removed.

    \-

    Finally, the pieces of code comprised in the range $[\gamma,\epsilon]$
    are removed from the release. These pieces of code are marked using the
    \textit{export} syntax described next.

    \-

    The stages $\gamma$ and $\epsilon$ represent kaneton sub-project ranks:

    \begin{itemize}
      \item
	\textbf{0}: boot stuff: boostrap, bootloader etc..;
      \item
	\textbf{1}: memory management;
      \item
	\textbf{2}: event management: interrupts, I/O etc..;
      \item
	\textbf{3}: task management, scheduling;
      \item
	\textbf{4}: communication management.
    \end{itemize}
\end{itemize}

Finally, the generated release is named based on the \textit{\_EXPORT\_}
kaneton environment variable followed by the date and type of the release:
\textit{backup}, \textit{dist} or \textit{k}$\gamma$\textit{,}$\epsilon$.

% syntax

\subsubsubsection{Syntax}

As explained previously, pieces of code are removed in order to build
\textit{stage} releases.

The kaneton source code is marked so that the \textit{export} tool knows
what piece of code to remove and for what stage. Indeed, every piece of
educational code is marked by a tag indicating the stage it is related to.

The syntax used is illustrated below:

\begin{verbatim}
  /*                                                                [cut] k1 */

  /*
   * this function clones a segment.
   *
   * steps:
   *
   * 1) get the original segment object.
   * 2) reserve a new segment of same size with same permissions.
   * 3) copy the data from the old segment.
   * 4) call machine-dependent code.
   */

  t_error                 segment_clone(i_as                      asid,
                                        i_segment                 old,
                                        i_segment*                new)
  {
    o_segment*            from;
    t_perms               perms;

    SEGMENT_ENTER(segment);

    /*
     * 1)
     */

    if (segment_get(old, &from) != ERROR_NONE)
      SEGMENT_LEAVE(segment, ERROR_UNKNOWN);

    [...]

    /*
     * 4)
     */

    if (machdep_call(segment, segment_clone, asid, old, new) != ERROR_NONE)
      SEGMENT_LEAVE(segment, ERROR_UNKNOWN);

    SEGMENT_LEAVE(segment, ERROR_NONE);
  }

  /*                                                               [cut] /k1 */
\end{verbatim}

In this example, the kaneton teachers decided \textit{segment\_clone()}
was a functionality the students should implement.

The markings at the top \verb|[cut] k1| and bottom \verb|[cut] /k1| of this
example indicate the \textit{export} tool the location of the piece of code
to remove for the stage \textit{k1}.

Let us suppose a teacher $T_{1}$ wants to use kaneton leading the students to
the development of the memory management functionality, only. On the other
hand, another teacher, $T_{2}$, wants to use the whole kaneton project starting
with the bootloader implementation to the task management. Additionally,
this teacher chooses to hide the communication management pieces of code,
to avoid cheating between students of different universities for instance.

In the first case, since the students have to implement the kaneton managers
around the memory management, $T_{1}$ has to provide the students everything
the memory management stuff relies on, including some fundamental managers,
the bootloader etc.. Also, the teacher does not need to provide the source
code of the upper level managers. As a result, a \textit{k1,1} release will
remove the pieces of codes with every marking $k_{\alpha}$ for
$1 \le \alpha \le 1$.

The teacher $T_{2}$ needs something different since the students are going
to implement every major piece of the kaneton source code. Since this teacher
wants their students to implement all the steps, starting with \textit{k0}
to \textit{k4}, a \textit{k0,4} release will not contain the pieces of source
code marked $k_{\alpha}$ for $0 \le \alpha \le 4$.


% transcript

%
% ---------- header -----------------------------------------------------------
%
% project       kaneton
%
% license       kaneton
%
% file          /home/mycure/kaneton/view/book/development/transcript.tex
%
% created       julien quintard   [thu may 24 05:07:02 2007]
% updated       julien quintard   [fri jun  1 00:58:43 2007]
%

%
% ---------- transcript -------------------------------------------------------
%

\subsection{Transcript}
\label{section:transcript}

The \textit{transcript/} directory is composed of two tools related to
the management of transcripts. The \textit{record} tool captures a
shell session while the \textit{play} tool replays a captured session.

These tool were introduced to allow students to make a dynamic presentation
of their kaneton implementation's features and possibilities. These dynamic
presentations were supposed to replace the oral examinations.

These transcripts are not used by the main contributors of the kaneton
project yet. However, any teacher interested by this tool can use it.

The \textit{transcript/} directory contains subdirectories which classify
the transcripts.

The only transcript class currently in place is named \textit{basic} and
contains transcripts illustrating the use of the kaneton internal tools.


% cheat

%
% ---------- header -----------------------------------------------------------
%
% project       kaneton
%
% license       kaneton
%
% file          /home/mycure/kaneton/view/book/development/cheat.tex
%
% created       julien quintard   [thu may 24 11:57:37 2007]
% updated       julien quintard   [fri jun  1 01:03:21 2007]
%

%
% ---------- cheat ------------------------------------------------------------
%

\subsection{Cheat}
\label{section:cheat}

The \textit{cheat} tool checks whether students cheated by using pieces of
code from kaneton projects of the previous years.

The \textit{history/} directory is composed of directories organizing the
kaneton students implementations over the years and for every school and
university the education project was used. Then each subdirectory represents
a year and contains subdirectories for each students group of this year.

Each student group directory contains a \textit{sources/} subdirectory
containing the tarballs of the different kaneton stages: \textit{k0},
\textit{k1}, \textit{k2} and so on; a \textit{fingerprints/} directory
containing an internal source representation used for detecting cheating,
a \textit{tests/} directory containing a summary of the testing results
for each stage and a \textit{cheats/} directory which contains a list of
commonalities with other kaneton implementations of the same and previous
years.

The \textit{cheat} tool takes a year and a stage as arguments. Its first
task is to generate the fingerprints of the other kaneton implementations
for this stage of the same and previous years.

Once the fingerprints are generated, the tool performs the checks by
comparing each pair of kaneton implementations for this stage.

The \textit{cheat} tool is based on another tool which cannot be revealed
here. For more information, please contact your supervisor.


% test

%
% ---------- header -----------------------------------------------------------
%
% project       kaneton
%
% license       kaneton
%
% file          /home/mycure/kaneton/view/book/development/test.tex
%
% created       julien quintard   [thu may 24 12:18:23 2007]
% updated       julien quintard   [wed dec  8 22:37:27 2010]
%

%
% ---------- test -------------------------------------------------------------
%

\subsection{Test}
\label{section:test}

The \name{test} tool enables students to test their kaneton implementation
against a set of tests that have been designed by the contributors. Below
is briefly described the terminology used by this tool in order to give
the reader an overview of the general scenario involving students, the
administrator and the server running the test system.

\begin{itemize}
  \item
    A \name{certificate} is used to make sure clients can authenticate the
    test server;
  \item
    Every certificate is sealed by a cryptographic \name{key};
  \item
    Each user is provided with a \name{capability} in order to identify
    herself to the server;
  \item
    These capabilities are sealed by a \name{code} which the server uses
    in order to detect illegally forged capabilities;
  \item
    A \name{configuration} specifies the number of tests a user is allowed
    to requests the server;
  \item
    The user's \name{database} is generated based on a configuration and
    maintains the current user's state on the server including the number
    of tests performed so far, the kaneton implementations submitted for
    evaluation \etc{}
  \item
    A \name{snapshot} is a kaneton implementation in its shipping form;
  \item
    The \name{machine} represents the target
    \name{platform}/\name{architecture} couple on which a snapshot is
    supposed to be tested or evaluated for instance;
  \item
    An \name{image} represents a kaneton snapshot compiled in a bootable
    form;
  \item
    A \name{test} is a function included in the kernel which performs a
    specific set of operations and possibly prints information to the console;
  \item
    The tests are often gathered together in a \name{suite} which represents
    the testing unit students are offered to trigger for their kaneton
    implementation;
  \item
    Once a snapshot is received by the server in order to be tested,
    the system compiles it into an image. The server also takes care to
    include the tests in the compilation process so that they can be triggered.
    These pre-compiled tests are referred to as the \name{bundle};
  \item
    The image can then be tested by triggering the tests of the suite. The
    image is therefore booted in an emulated \name{environment}. This
    environment can sometimes be chosen and offers a trade-off between
    simplicity and realism. The most common environments are \name{QEMU}
    and \name{Xen};
  \item
    Depending on the success of the tests, a set of results is generated
    and compiled in a \name{bulletin} file;
  \item
    Finally, the server retrieves this bulletin, adds some meta information
    such as the date of the test, the environment and machine used \etc{}
    and stores everything in a \name{report}. Note that this report is
    also sent back to the user so she can consult it;
  \item
    Students can also decide to submit their kaneton implementation for
    a specific \name{stage} for future evaluation. Note that suites and
    stages are completely different though they often bear the same names:
    \name{k0}, \name{k1}, \name{k2} etc{};
  \item
    The administrator can decide to evaluate the snapshots which have been
    submitted for a stage by invoking a script which will attribute grades
    according to the \textit{point}s associated with every test.
  \item
    Finally, a \name{statement} is produced containing the grades of every
    student for a given stage.
\end{itemize}

The following describes the \name{test} tool according to the user's role
regarding the system: either the administrator who sets up the system or
a student who uses it in order to improve and/or evaluate his implementation.

%
% administrator
%
\subsubsection{Administrator}

The administrator is responsible for setting up the system but also maintaining
it on a daily basis.

% requirements
\subsubsubsection{Requirements}

The \name{test} tool must be installed on a publicly accessible server since
the server script will be waiting for incoming requests. Note that by default,
the clients assume the test server to be accessible at the address:
\location{https://test.opaak.org:8421}.

Besides, since the purpose of the \name{test} tool is to run the students'
kaneton implementation in emulated environments, both \name{QEMU} and
\name{Xen} should be available though one might want to configure the tool
for supporting a single environment, \name{QEMU} for instance.

Note that the test system has been developed with \name{Python 2.6} and
may be out of date by the time the administrator sets it up. In addition,
the system depends on a variety of \name{Python} packages including
\name{argparse}, \name{yaml}, \name{pyopenssl}, \name{hmac}, \name{pickle},
\name{xmlrpc}, \name{subprocess}, \name{serial} among others.

Finally, the administrator should make sure the following applications
are installed since some test scripts need them: \name{dd}, \name{mkfs.ext2},
\name{mount}, \name{umount}, \name{mutt}, \name{sendmail}, \name{qemu}
and \name{mkisofs}.

% set up
\subsubsubsection{Set Up}

The first step for an administrator consists in generating the necessary
files, especially the certificates, code and capabilities required for
securing the test system.

The \location{test/utilities/} directory contains the scripts that perform
such operations. Note that all the generated files are stored in the
\location{test/store/} directory.

First the \name{CA - Certification Authority}'s and server's certificates
must be generated. The first is used to issue certificates while the latter
is used for clients to identify the server with absolute certainty.

\begin{verbatim}
  $> make certificate
  [+] generating the CA and server's key/certificate pair
  [+] CA key/certificate generated
  [+] server key/certificate generated
  [+] CA and server's key/certificate pair generated and stored
  $> 
\end{verbatim}

The next step consists in generating a code for the administrator to
issue capabilities but also for the server to verify that the received
capabilities have not been illegally forged.

\begin{verbatim}
  $> make code
  [+] generating the server's code
  [+] server code successfuly generated and stored
  $> 
\end{verbatim}

With a server code, the students' and contributor's capabilities can be
built, hence granting them the right the contact the server.

The following generates the contributor's capability. This capability is
special in the way that contributors can perform any operation in a completely
contrain-free manner.

\begin{verbatim}
  $> make capability-contributor
  [+] generating the contributor's capability
  [+] contributor's capability generated and stored
  $> 
\end{verbatim}

In contrast, the following command generates a set of capabilities for the
students belonging to the school referred to as \name{``epita::2010''}. Note
that the script requires the \location{history/epita/2010/} to be populated
with the groups and their \location{people} file.

\begin{verbatim}
  $> make capability-school@epita::2010
  [+] generating students' capabilities
  [+] extracting the students from the history 'epita/2010'
  [+] students information retrieved
  [+] generating the students' capabilities:
  [+]   epita::2010::group11
  [+]   epita::2010::group10
  [+]   epita::2010::group13
  [+]   epita::2010::group12
  [+]   epita::2010::group33
  [+]   epita::2010::group32
  [+]   epita::2010::group17
  [+]   epita::2010::group30
  [+]   epita::2010::group19
  [+]   epita::2010::group18
  [+]   epita::2010::group5
  [+]   epita::2010::group4
  [+]   epita::2010::group7
  [+]   epita::2010::group6
  [+]   epita::2010::group1
  [+]   epita::2010::group3
  [+]   epita::2010::group2
  [+]   epita::2010::group15
  [+]   epita::2010::group9
  [+]   epita::2010::group8
  [+]   epita::2010::group14
  [+]   epita::2010::group31
  [+]   epita::2010::group16
  [+]   epita::2010::group24
  [+]   epita::2010::group25
  [+]   epita::2010::group26
  [+]   epita::2010::group27
  [+]   epita::2010::group20
  [+]   epita::2010::group21
  [+]   epita::2010::group22
  [+]   epita::2010::group23
  [+]   epita::2010::group28
  [+]   epita::2010::group29
  [+]   epita::2010::group34
  [+] students' capabilities generated and stored
  $> 
\end{verbatim}

In addition, the administrator could decide to generate or re-generate
a capability for a specific student of a school. The following shows an
example for such an action.

\begin{verbatim}
  $> make capability-student@epita::2010::group8
  [+] generating the student's capability:
  [+]   epita::2010::group8
  [+] student's capability generated and stored
  $> 
\end{verbatim}

The next step consists in the databases generation. A database contains the
state of a user profile including the number of test requests, the quota for
such tests, the submitted snapshots and so forth. The database files are
absolutely fundamental to the server since such databases are updated after
each client's request.

The syntax for generating databases follows the one for capabilities, as
shown next for the contributor.

\begin{verbatim}
  $> make database-contributor
  [+] generating database from contributor's configuration
  [+] contributor's database generated and stored
  $> 
\end{verbatim}

Once the certificates, code, capabilities and databases generated, the
administrator can move on to the deployment process.

% deployment
\subsubsubsection{Deployment}

The deployment basically consists in copying the \location{test/} environment
to the test server though one might want to copy the whole kaneton environment
or the smallest subset of the \location{test/} directory which should, in this
case, include the following absolutely necessary items:

\begin{itemize}
  \item
    The \location{test/environments/} directory which contains the descriptions
    of the supported test environments;
  \item
    The \location{test/images/} directory which contains a script for
    automatically generating a \name{Debian Live} system which is used
    for compiling a kaneton snapshot into a bootable image;
  \item
    The \location{test/packages/} directory which contains the \name{ktp -
    Kaneton Test Package} required by the server-side standalone scripts
    for manipulating files such as databases, capabilities \etc{} but also
    for performing cryptographic operations and send/receive \name{XMLRPC}
    requests;
  \item
    The \location{test/scripts/} directory which contains the fundamental
    scripts for building bootable images, distributing the capabilities to
    the students through emails, evaluating the submitted snapshots and so on;
  \item
    The \location{test/server/} directory which contains the server script
    for handling the clients' requests;
  \item
    The \location{test/stages/} directory which contains the files requirement
    for evaluating the students' snapshots;
  \item
    The \location{test/store/} directory which contains the generated files
    such as the users' databases, the server's code and certificate; and
  \item
    The \location{test/suites/} directory which contains the files describing
    the tests to be including in a given tests suite.
\end{itemize}

Once copied, the administrator only needs to launch the server script located
in the \location{test/server/} directory, as shown below:

\begin{verbatim}
  $> ./server.py
  [meta] serving on 88.191.84.128:8421
\end{verbatim}

Note that a few additional steps may be required depending on the current state
of the kaneton development.

The first of these steps may consist in generating a \name{Debian Live} system
since this is absolutely required for the test system to work. For more
information regarding the generation of such an image, please refer to the
\location{test/images/} directory.

The second step should consist for the administrator in building the kaneton
tests bundle. The bundle represents a pre-compiled set of tests that is
included in the students' snapshot compilation process. The tests are
pre-compiled in order to prevent leaking information since students could
very well dump the content of those tests and force the compilation to fail,
hence retrieving the source code in the compilation process' error log.

In order to generate such a bundle, the administrator must first activate
the \name{test} module, as show next:

\begin{verbatim}
  _MODULES_               +=              test
\end{verbatim}

Then, the administrator must move to the \location{test/tests/} directory
and launch a compilation process through the following command:

\begin{verbatim}
  $> make
\end{verbatim}

Once generated, the test bundle, located in \location{store/bundle/[machine]/}
must be copied to the server, at the same location.

Finally, for more information on the server script, please refer to the
\location{test/server/} directory.

% scripts
\subsubsubsection{Scripts}

Although the deployment process is pretty straightforward, the administrator
is required to manage the test system through several scripts.

First, the \name{distribute} script must be used by the administrator to send
the capabilities to the respective owners so that the students can use
the test system. Note that this script relies on the \name{Mutt} mailing
system for sending the emails containing the attached capabilities.

\begin{verbatim}
  $> ./distribute.py
  recipients:
    contributor
  $>
\end{verbatim}

While the \name{construct} script enables the administrator to build a
bootable image from a kaneton snapshot, the \name{stress} script takes
a bootable image and triggers the tests belonging to the given test
suite. Note that both scripts are directly used by the server script for
building and testing the received kaneton snapshots.

\begin{verbatim}
  $> ./construct.py --snapshot kaneton.tar.bz2                          \
                    --image kaneton.img                                 \
                    --environment xen                                   \
  the kaneton image has been constructed in 'kaneton.img'
  $> ./stress.py --image kaneton.img                                    \
                 --suite k2                                             \
                 --environment xen                                      \
                 --verbose
  segment
    permissions/01 :: true
  id
    simple :: true
    clone :: true
    multiple :: true
  $> 
\end{verbatim}

Note that the administrator could also test a kaneton image manually,
especially through the following command:

\begin{verbatim}
  $> qemu -fda kaneton.img -curses
\end{verbatim}

Besides, note that an administrator willing to include a new test in the
system would probably want to test it locally first since testing through
the server takes some time. In order to test locally, the administrator
must first activate the bundle module in its user profile
\location{environment/profile/user/\${KANETON\_USER}/\${KANETON\_USER}.conf}:

\begin{verbatim}
  _MODULES_               +=              bundle
\end{verbatim}

Then, the administrator must trigger the test by calling the test function
manually in its kaneton implementation. For instance, in order to trigger
the \name{kaneton/core/task/guest} test, the administrator could add the
following line after \code{kernel\_initialize()} and before running the
test system in \location{kaneton/core/core.c}:

\begin{verbatim}
  [...]

  module_call(console, message,
              '+', "starting the kernel\n");

  assert(kernel_initialize() == STATUS_OK);

  /* XXX[temporary] */
  test_core_task_guest();

  module_call(test, run);

  [...]
\end{verbatim}

Once the kaneton image rebuilt, the administrator can boot it locally
through \name{QEMU} and get the output, hence check that the test went
as excepted:

\begin{verbatim}
  $> qemu -fda environment/profile/user/${KANETON_USER}/${KANETON_USER}.img
\end{verbatim}

Back to the server side, the \name{evaluate} script can be used by the
administrator in order to assign grades to the snapshots submitted by the
students. The script generates a statement containing the results of this
evaluation process.

\begin{verbatim}
  $> ./evaluate.py --stage k2                                           \
                   --pattern "^epita::2010::.*$"
  the statement has been saved in '../store/statement/20101102-223645.db'
  $> 
\end{verbatim}

Finally, the \name{dump} script takes any \name{YAML}-based file and
displays its inner structure in a hierarchical manner.

\begin{verbatim}
  $> ./dump.py --path ../store/statement/20101102-223848.db
  meta:
    reference:              20.0
    stage:                  k2
  data:
    epita::2010::group7:
      date:                 2010/11/02 20:46:44
      grade:                16.0
      snapshot:             20101102-204644
      members:
        email:              admin@opaak.org
        name:               admin
      configurations:
        Xen:
          report:           20101102-224213
          notch:            4
          score:            4
        QEMU:
          report:           20101102-223848
          notch:            4
          score:            0
\end{verbatim}

%
% student
%
\subsubsection{Student}

The student has the possibility to request actions from the test server
through the client script located in \location{test/client/}.

% requirements
\subsubsubsection{Requirements}

Although the client script is integrated in the kaneton environment, it also
makes use of the \name{ktp}. Therefore, as for the server, the client depends
on a variety of \name{Python} packages including \name{yaml}, \name{pyopenssl},
\name{hmac}, \name{pickle}, \name{xmlrpc}, \name{subprocess} among others.

% use
\subsubsubsection{Use}

The client script enables the user to request one of the five operations
described below.

\begin{verbatim}
  $> make
  [!] usage: client.py [command]

  [!] commands:
  [!]   information
  [!]   submit-[stage]
  [!]   test-[environment]::[suite]
  [!]   list
  [!]   display-[identifier]
  [!]   retest-[identifier]
  $>
\end{verbatim}

The \name{information} operation requests the server to return information
on the current state of the user's profile. The information returned range
from the number of tests performed, the quota for every test suite to the
available stages or the snapshots having been previously submitted.

\begin{verbatim}
  $> make information
  [+] configuration:
  [+]   server:                 https://test.opaak.org:8421
  [+]   capability:             /data/mycure/repositories/kaneton/environment/profile/user/julien.quintard/julien.quintard.cap
  [+]   platform:               ibm-pc
  [+]   architecture:           ia32/educational

  [+] information:
  [+]   profile:
  [+]     identifier:           contributor
  [+]     community:            contributors
  [+]     members:
  [+]       name:               admin
  [+]       email:              admin@opaak.org
  [+]   suites:
  [+]                           k1
  [+]                           k3
  [+]                           k2
  [+]                           kaneton
  [+]   stages:
  [+]                           k1
  [+]                           k2
  [+]                           k3
  [+]   environments:
  [+]                           qemu
  [+]                           xen
  [+]   database:
  [+]     reports:
  [+]       xen:
  [+]         ibm-pc.ia32/educational:
  [+]           k3:
  [+]           k2:
  [+]           k1:
  [+]       qemu:
  [+]         ibm-pc.ia32/educational:
  [+]           k3:
  [+]           k2:
  [+]           k1:
  [+]     settings:
  [+]       xen:
  [+]         ibm-pc.ia32/educational:
  [+]           k3:
  [+]             requests:     0
  [+]             quota:        -1
  [+]           k2:
  [+]             requests:     0
  [+]             quota:        -1
  [+]           k1:
  [+]             requests:     0
  [+]             quota:        -1
  [+]       qemu:
  [+]         ibm-pc.ia32/educational:
  [+]           k3:
  [+]             requests:     0
  [+]             quota:        -1
  [+]           k2:
  [+]             requests:     0
  [+]             quota:        -1
  [+]           k1:
  [+]             requests:     0
  [+]             quota:        -1
  $> 
\end{verbatim}

The \name{test} command enables the user to trigger a test suite for the
current kaneton implementation on the specified environment such as \name{QEMU}
or \textit{Xen} for instance.

The server then returns the resulted report which the client stores locally
in \location{test/store/report/}. In addition, the client displays a quick
summary of the report in order for the user to know whether things went
as expected.

\begin{verbatim}
  $> make test-xen::k2
  [+] configuration:
  [+]   server:                 https://test.opaak.org:8421
  [+]   capability:             /data/mycure/repositories/kaneton/environment/profile/user/julien.quintard/julien.quintard.cap
  [+]   platform:               ibm-pc
  [+]   architecture:           ia32/educational

  [+] report(20101103:140601):
  [+]   segment                                                           [1/1]
  [+]   id                                                                [3/3]
  $> 
\end{verbatim}

The \name{list} command enables the user to display the identifiers of the
reports in the local store.

\begin{verbatim}
  $> make list
  [+] reports:
  [+]   20101103:140601:
  [+]     xen :: ibm-pc :: ia32/educational :: k2 :: 2010/11/03 14:06:01
\end{verbatim}

The \name{display} command gives the user the possibility to dump a locally
stored report in a very detailed way.

\begin{verbatim}
  $> make display-20101103:140601
  [+] report:
  [+]   meta:
  [+]     platform:               ibm-pc
  [+]     date:                   2010/11/03 14:06:01
  [+]     architecture:           ia32/educational
  [+]     duration:               63.499
  [+]     suite:                  k2
  [+]     identifier:             20101103:140601
  [+]     environments:
  [+]       stress:               xen
  [+]       construct:            xen
  [+]   data:
  [+]     segment:                                                        [1/1]
  [+]       permissions/01:
  [+]         status: True
  [+]         description: This test creates a task and address space before reserving a segment and changing its permissions.
  [+]         duration: 0.010
  [+]         output: 
  [+]     id:                                                             [3/3]
  [+]       simple:
  [+]         status: True
  [+]         description: This test reserves a single identifier.
  [+]         duration: 0.004
  [+]         output: 
  [+]       clone:
  [+]         status: True
  [+]         description: This test reserves, clones and releases identifiers.
  [+]         duration: 0.005
  [+]         output: 
  [+]       multiple:
  [+]         status: True
  [+]         description: This test reserves thousands of identifiers, checking that no collisions occured.
  [+]         duration: 0.040
  [+]         output: 
  $> 
\end{verbatim}

The \name{submit} command sends the user's snapshot to the server so as to
be evaluated for the given stage.

\begin{verbatim}
  $> make submit-k1
  [+] configuration:
  [+]   server:                 https://test.opaak.org:8421
  [+]   capability:             /data/mycure/repositories/kaneton/environment/profile/user/julien.quintard/julien.quintard.cap
  [+]   platform:               ibm-pc
  [+]   architecture:           ia32/educational

  [+] the snapshot has been submitted successfully
  $> 
\end{verbatim}

Finally, the \name{retest} command provides contributors the possibility to
re-launch the test suite of the given identified test. This command is
especially useful to re-test a snapshot should an unexpected error occur on
the test server.

Indeed since test requests are limited for students, it would be unfair for the
student to be forced to sacrifice a test slot because something went wrong
on the server-side. By requesting a contributor, the student's snapshot can
be re-tested. Once the test complete, an email is sent to the student along
with the attached report.

\begin{verbatim}
  $> make retest-20101103:140601
  [+] configuration:
  [+]   server:                 https://test.opaak.org:8421
  [+]   capability:             /data/mycure/repositories/kaneton/environment/profile/user/julien.quintard/julien.quintard.cap
  [+]   platform:               ibm-pc
  [+]   architecture:           ia32/educational

  [+] the snapshot has been re-tested successfully
  $> 
\end{verbatim}

%
% robot
%
\subsubsection{Robot}

The \name{robot} test tool enables contributors to test the kaneton research
implementation on a regular basis; hence control the status of the development.

The robot basically retrieves the kaneton implementation by checking out the
\name{Subversion} repository. Then, several test suites are triggered through
the test client. Once the reports have been received, a message is built
summarizing the results. This message is then sent to the kaneton contributors
mailing-list.

The deployment of the \name{robot} is quite straigthforward. First, the
\location{test/robot/} directory must be copied to the server. Note that
the \name{robot.py} script depends upon the \name{ktp} package which must
therefore be copied as well.

Then, the \name{SSH} configuration file \name{config} must be placed in
the \location{\${HOME}/.ssh/} directory. Besides, this file should be edited in
order to properly reference the \name{SSH} keys since the default configuration
assumes the kaneton test directory to be located at \location{/kaneton/}.

Finally, the \name{robot.cron} crontab file must be setup through the
following command in order to trigger the robot every night:

\begin{verbatim}
  $> crontab robot.cron
\end{verbatim}

Once again, the administrator should make sure to edit this file should
the robot files not be located in the default location \ie{}
\location{/kaneton/}.


% control panel

%
% ---------- header -----------------------------------------------------------
%
% project       kaneton
%
% license       kaneton
%
% file          /home/mycure/kaneton/view/book/development/control-panel.tex
%
% created       julien quintard   [sun may 20 15:22:52 2007]
% updated       julien quintard   [mon may  4 20:01:20 2009]
%

%
% ---------- control panel ----------------------------------------------------
%

\subsection{Control Panel}
\label{section:control panel}

The kaneton environment allows the developer to trigger every action from
the \name{Make} file of the project's root directory.

These actions are listed below:

\command{make initialize}
        {
	  This action initializes the kaneton development environment by
	  invoking the \location{init.py} script of the \location{environment/}
	  directory.
	}

\command{make clean}
	{
	  This action cleans the kaneton development environment.
	}

\command{make main}
	{
	  This action triggers the default rule which aims at compiling every
	  piece the final system needs to be set up on a bootable device.

	  \-

	  \example{\$ make main}

	  \example{\$ make}
	}

\command{make clear}
	{
	  This action removes every compiled files.
	}

\command{make headers}
	{
	  This action generates \name{Make} files' \name{C} header
	  files dependencies.
	}

\command{make prototypes}
	{
	  This action generates C prototypes.
	}

\command{make test}
	{
	  This action runs the test suite in order to validate the kaneton
	  microkernel behaviour.
	}

\command{make cheat}
	{
	  This action launches the cheat tests on students kaneton
	  implementations.

	  \-

	  \example{\$ make cheat}

	  \-

	  \example{\$ make cheat-EPITA::2006::k3}
	}

\command{make build}
	{
	  This action builds the boot device.
	}

\command{make install}
	{
	  This action installs the kaneton microkernel with its dependencies:
	  configuration files, bootloader \etc{} on the boot device.
	}

\command{make export}
	{
	  This action builds a kaneton distribution package.

	  \-

	  \example{\$ make export}

	  \-

	  \example{\$ make export-k3,5}

	  \-

	  \example{\$ make export-back}
	}

\command{make view}
	{
	  This action builds and displays a kaneton document.

	  \-

	  \example{\$ make view}

	  \-

	  \example{\$ make view-devel}

	  \-

	  \example{\$ make view-book::kaneton}
	}

\command{make record}
	{
	  This action records a real-time session.

	  \-

	  \example{\$ make record}

	  \-

	  \example{\$ make record-basic::test.ts}
	}

\command{make play}
	{
	  This action plays a recorded session.

	  \-

	  \example{\$ make play}

	  \-

	  \example{\$ make play-basic::prototypes.ts}

	  \-

	  \example{\$ make play-prototy}
	}

\command{make info}
	{
	  This action displays general information about kaneton.
	}


%
% external
%

\section{External}

The kaneton contributors use several other tools for the communication, the
development etc.. These tools are described in the following sections.

% mailing-list

%
% ---------- header -----------------------------------------------------------
%
% project       kaneton
%
% license       kaneton
%
% file          /home/mycure/kaneton/view/book/development/mailing-list.tex
%
% created       julien quintard   [thu may 24 19:55:18 2007]
% updated       julien quintard   [thu may 24 20:43:15 2007]
%

%
% ---------- mailing-list -----------------------------------------------------
%

\subsection{Mailing-List}

Because people do not want to use several communication tools: email,
newsgroup, forum etc.. and because everybody has an email address, the
kaneton people communicate through a mailing-list.

This mailing-list is in fact a \textit{Google} group. Indeed, the kaneton
project relies on three distinct communication groups:

\begin{itemize}
  \item
    \textbf{kaneton} which is used to make announcements about new releases,
    patchs etc..

    \-

    This group is not used yet.
  \item
    \textbf{kaneton-developers} is dedicated to the communication between the
    people involved in the development of the kaneton microkernel reference.

    \-

    This group is therefore private.
  \item
    \textbf{kaneton-students} can be used in an absolutely free-way for
    students for communicating about their kaneton implementation.

    \-

    Anybody can join this group.
\end{itemize}

Needless to say, community behavioural rules enumerated in a previous section
must be followed when communicating on the kaneton mailing-list. Every
contributor is welcomed to give its point of view, to ask questions etc.. but
this must be done with politness, respect and humility.

Everyone communicating through the mailing-list must read the
\textit{Netiquette} which describes the rules inherent to the communication
on the Internet. Especially, people should take care of writing messages in
respect of the \textit{80} columns; and should always cut off useless parts
of a previous message when responding.

It is likely a real-time communication tool will be very useful in the
future, as an \textit{IRC} channel, for instance. However, communicating on
these extra media will not be mandatory until kaneton people decide it is.

kaneton people are asked to use the mailing-list communication medium
in a perfect way as it is the unique intra-communication channel. In
addition then, contributors must read their emails on a regular-basis
as some people rely on the decision of others.

The \textit{kaneton-students} mailing-list must be used carefully. As
an example, people should never paste pieces of source code or ask
questions implying an answer with the solution. Even if it is a free
group, people abusing of this communication channel could easily be banned.

People must always respond in the appropriate discussion. If, in a discussion,
a different subject is discussed, then, one of the contributor must create
a new discussion in order facilitate the communication.

Finally, the discussion subjects must be tagged like the following examples:

\begin{verbatim}
  [ia32/optimised] mapping issues

  [segment] segment_clone() :: bug?

  [research] new paper about OS design
\end{verbatim}

There is no list of official tags. The users are simply asked to make
their discussion subjects as clear as possible to simplify the task
consisting in looking for old topics in the archives. Indeed, remember
that newcomers should - if they respect the rules - look at the archives
to avoid discussing an old subject on the mailing-list.


% repository

%
% ---------- header -----------------------------------------------------------
%
% project       kaneton
%
% license       kaneton
%
% file          /home/mycure/kaneton/view/book/development/repository.tex
%
% created       julien quintard   [thu may 24 20:43:26 2007]
% updated       julien quintard   [wed jun 13 22:32:31 2007]
%

%
% ---------- repository -------------------------------------------------------
%

\subsection{Repository}
\label{section:repository}

The repository contains everything related to the kaneton microkernel
project, in other words, the kaneton source tree described in
\textit{Chapter \ref{chapter:source tree}}. Indeed, the repository contains
the whole history of the kaneton project including the documentation, the
source code but also the students tarballs over the years.

The actual repository is based on the \textit{Subversion} software which
provides far more advanced features than its historical rival \textit{CVS}.

The repository is actually hosted on the \textit{kaneton.org} server which
also contains the web server and everything else related to the management
of the kaneton microkernel project.

The repository is accessed in a secure way through a \textit{SSH} channel.
Indeed, the kaneton \textit{Subversion} repository can be accessed at the
following address:

\begin{verbatim}
  svn+ssh://subversion@repositories.kaneton.org/kaneton
\end{verbatim}

Note that the security is achieved by the use of \textit{SSH} keys. Therefore,
any new contributor should get in touch with an administrator of the
kaneton server in order to obtain an access. Also note that, a test period
could be set up for a new contributor to get the trust of the kaneton
community. For more information, please refer to \textit{Chapter
\ref{chapter:community}}.

A contributor willing to create a \textit{SSH} key shoud simply use this
\textit{Unix} command:

\begin{verbatim}
  $> ssh-keygen -t dsa
\end{verbatim}

For more information about how to use the repository, please refer to the
official \textit{Subversion} documentation. The same goes for the
\textit{SSH} tools suite.

The example below illustrates the checkout of the kaneton repository.

\begin{verbatim}
  $> svn checkout svn+ssh://subversion@repositories.kaneton.org/kaneton
\end{verbatim}

The contributors getting access to the kaneton repositories must behave
properly according to the obvious cooperative development rules. As an
example, a kaneton developer must not perform any commit before making sure
the kaneton microkernel compiles and passes all the tests.

The repository organisation is crucial. Therefore, nothing should be
added, removed or renamed without the permission of the developers in charge
of the repository.

Finally, any commit must come with a log describing the modifications
implied by the commit. These logs must conform to the following syntax.

\begin{verbatim}
  [kaneton/core/segment/]
    o the bug about the permissions was corrected in segment_clone().
    o an algorithm based on a b-tree was added.

  [environment/profile/user/julien.quintard/]
    o some personal configurations were modified.
\end{verbatim}

Following this syntax is very important as an email is sent to the
\textit{kaneton-developers} mailing-list every time a commit is performed.
Therefore, the contributors reading the mailing-list are aware of every
modification in the kaneton source code. This feature can also be used
to review the modifications done by a new contributor in order to help
him doing things in a better way.

Note that there must not be any file with the executable flag permission
enabled. Moreover, scripts files must not contain any \textit{shebang}.
Indeed, the kaneton development environment knows which interpreter to
use for every type of file. It is therefore a non-sense to introduce a
hard-coded path to an interpreter.

Tarball file names must be extended with \textit{.tar} while \textit{bzip2}
compressed tarballs must be extended with \textit{.tar.bz2}.


% wiki

%
% ---------- header -----------------------------------------------------------
%
% project       kaneton
%
% license       kaneton
%
% file          /home/mycure/kaneton/view/book/development/wiki.tex
%
% created       julien quintard   [thu may 24 23:06:02 2007]
% updated       julien quintard   [fri aug  1 15:52:28 2008]
%

%
% ---------- wiki -------------------------------------------------------------
%

\subsection{Wiki}
\label{section:wiki}

A Wiki is used both for external and internal communication. The software
used is called \name{TWiki} and provides a pretty simple syntax with
many additional plugins to customize the website. This solution was used
for historical reasons but also because the \name{TWiki} rendering can
be customized through templates in order to get a final visual close to
classical websites. Thus, the kaneton website looks like an ordinary
website but powered by a Wiki engine.

The Wiki is hosted at \location{http://kaneton.opaak.org} and contains four
webs: an extranet and three intranets. The main web, accessed through the
address above is the external website. This website contains news, papers,
documentation and general information on the kaneton project. The three other
webs are used more as intranets or wikis more than as websites. Two of these
webs are private to the kaneton developers and the kaneton teachers,
respectively. The latter web is intended to the students and contains
documents, links \etc{} about low-level programming, kernel development \etc{}
as well as information about courses related to the kaneton project.

Note that the Wiki reserved for the kaneton developers must be used
instead of the Wiki eventually provided by the project management tool.

Everybody involved in the kaneton project must contribute to the kaneton
website as well as to the kaneton intranets. Indeed, the external communication
is fundamental, even in an open source project and the kaneton website is
the only public communication medium.

New contributors are then asked to register onto the kaneton \name{TWiki}
at \location{http://kaneton.opaak.org}. Once done, the contributor should
inform the person in charge of the kaneton website so that  the contributor's
account is activated. As a result, the contributor will be able to modify
pages of the website and intranets.


% project management

%
% ---------- header -----------------------------------------------------------
%
% project       kaneton
%
% license       kaneton
%
% file          /home/mycure/kane.../book/development/project-management.tex
%
% created       julien quintard   [fri may 25 19:26:17 2007]
% updated       julien quintard   [thu may 31 06:25:56 2007]
%

%
% ---------- project management -----------------------------------------------
%

\subsection{Project Management}
\label{section:project management}

XXX

\begin{comment}
le systeme de tickets/bugs est egalement tres important. chaque ticket se
voit affecte une priorite et il est important de comprendre que pour le
bien global du projet, un developpeur ne peut se contenter de faire ce
qui lui plait, il se doit de contribuer egalement a la resolution de problemes.

encore une fois les tickets doivent suivre une norme.
\end{comment}


%%%
%% copyright quintard julien
%% 
%% kaneton
%% 
%% languages.tex
%% 
%% path          /root/data/research/projects/svn/kaneton/notes/languages
%% 
%% made by mycure
%%         quintard julien   [quinta_j@epita.fr]
%% 
%% started on    Mon Feb 21 16:03:53 2005   mycure
%% last update   Mon Feb 21 16:03:53 2005   mycure
%%

\documentclass[10pt,a4wide]{article}
\usepackage[english]{babel}
\usepackage{a4wide}
\usepackage{graphicx}
\usepackage{graphics}
\usepackage{fancyheadings}
\pagestyle{fancy}

\bibliographystyle{plain}

\lhead{{\scriptsize kaneton project}}
\rhead{languages notes}
\rfoot{\scriptsize EPITA System Lab}

\title{languages}

\author{Julien Quintard - \small{quinta\_j@epita.fr} \\
        Jean-Pascal Billaud - \small{billau\_j@epita.fr} \\ \\
	\small{last updated by} \\
	Julien Quintard - \small{quinta\_j@epita.fr}}

\date{\today}

\begin{document}
\maketitle

\section{Notes}

\begin{enumerate}

\item pourquoi utiliser l'assembleur ?
\item compilateur nasm
\item les macros [ORG 0X0] et [BITS 16-32]
\item registres g\'en\'eraux
\item registres de contr\^oles
\item registres eflags
\item les intructions de stockages mov ...
\item les instructions d'op\'erations sur les bits shl, shr, and, or ...
\item les instructions de saut jmp, call ...
\item les instructions de comaraisons et de branchement conditionnelle cmp, test ...
\item d'autres instructions facilitant bien la vie lodsb
\item les interruptions bios int 13h, int 10h ...
\item fonctionnement de la stack

\end{enumerate}

\end{document}



%\input{licenses}
%%%
%% licence       kaneton licence
%%
%% project       kaneton
%%
%% file          /home/mycure/kaneton/view/papers/kaneton/bibliography.tex
%%
%% created       julien quintard   [mon may  8 18:35:35 2006]
%% updated       julien quintard   [mon may  8 20:38:56 2006]
%%

%
% bibliograpy
%

\chapter{Bibliography}

This chapter contains the bibliography.

%
% text
%

\begin{thebibliography}{0}
  \bibitem{AST-SCO}
    \textbf{Structured Computer Organization};
    by
    \textit{Andrew S. Tanenbaum}
  \bibitem{AST-CN}
    \textbf{Computer Networks};
    by
    \textit{Andrew S. Tanenbaum}
  \bibitem{AST-OSDI}
    \textbf{Operating Systems: Design and Implementation};
    by
    \textit{Andrew S. Tanenbaum, Albert S Woodhull}
  \bibitem{AST-MOS}
    \textbf{Modern Operating Systems};
    by
    \textit{Andrew S. Tanenbaum}
  \bibitem{AST-DOS}
    \textbf{Distributed Operating Systems};
    by
    \textit{Andrew S. Tanenbaum}
  \bibitem{AST-DSPP}
    \textbf{Distributed Systems: Principles and Paradigms};
    by
    \textit{Andrew S. Tanenbaum, Maarten van Steen}
  \bibitem{NAL-DA}
    \textbf{Distributed Algorithms};
    by
    \textit{Nancy A. Lynch}
\end{thebibliography}


\end{document}

%
% ---------- source tree ------------------------------------------------------
%

decrire l'organisation.

%
% ---------- environment ------------------------------------------------------
%

decrire l'environment, pourquoi, comment l'utiliser, le configurer

make things easier

%
% ---------- kaneton ----------------------------------------------------------
%

XXX regles generales de kaneton

$$$ ne peut on pas les mettre dans development??? $$$

aucun fichier dans le repository avec les droits executables: Makefile,
scripts etc.. must be non-executable.

utiliser des mots entiers plutot que des simplifications sauf si vraiment
le mot est long.

preferez - plutot que _ pour separer sauf si le langage ne le permet pas.

si le langage ne permet pas l'utilisation de namespace ou forme du genre,
prefixez tout par le nom du module.

SURTOUT, regardez le code deja produit afin de vous inspirez de ce qui
existe deja et evitez des erreurs, c'est TRES IMPORTANT!!! si ces regles
ne sont pas suivies tout le monde perd du temps et le projet perd en
clarete. de plus a long terme, du temps devra etre perdu a developper des
outils specifiques pour verifier que vous respectez ces regles ce qui serait
vraiment completement stupide.

le nommage d'une fonctionnalite doit autant que possible se faire en un
seul et unique mot, ce mot designant au mieux la fonctionnalite. il peut bien
evidemment etre prefixe par le nom du module.

arguments en C, doivent etre simple, complet clairs car c'est l'interface
du module. o doit etre utilise opur un objet kernel correspondant au module.
tout doit etre nomme en un mot autant que possible et etre coherent par
rapport a ce qui existe deja. de plus le nom d'une variable ne doit
pas recoupe son type: t_wait waitlist c'est stupide.

XXX

%
% ---------- community --------------------------------------------------------
%

regles communautaire.

-- comportement

mieux plutot que meilleur

savoir participer en faisant des trucs moins cool que d'autres car ca
doit etre fait

...

le projet dispose de certains outils pour effectuer le minimum vital:
outils de comm interne, dev communautaire, outils de comm ext, gestion
de taches.

%
% ---------- tools ------------------------------------------------------------
%

regles concernant les outils.

-- communication

parceque c'est chiant d'avoir plusieurs outils de comm mail, news, forum
et que tout le monde dispose d'un mail, on utilise une ml.

comportement tres important sur une ml: eviter les trolls, repondre
en suvant la netiquette, participer en donnant son avis etc..

pour la communication temps-reel, un serveur irc pourrait etre mis en
place mais ce n'est pas le cas actuellemeent. si vous en ressentez le
desir, proposez l'idee sur la ml. dans tous les cas, cela ne sera pas
obligatoire.

de maniere generale, des outils libres, pre-fournis et simple d'utilisation
seront privlegies. ainsi google fournit un certains nombre d'outils que
kaneton utilise: gmail opur les mails si vous le desirez, googlegroup pour
la ml, project hosting pour le projet ..

-- repository

le repository est l'outil le plus important du projet puisque tout y
est stocke, le code bien sur mais egalement les documentations, les
projets des etudiants au fil des annees ...

le repository actuel contient toute l'histoire du projet kaneton depuis
le premier jour jusqu'a aujourdhui. nous utilisons subversion car a l'epoque
c'etait de loin le meilleur outil, son concurrent CVS etant deja trop
limite.

durant les premieres annees aucune normalisation de l'utilisation du
repository ne fut instauree. maintenant, les commits doivent suivre une
procedure praticuliere qui implique un descriptif normalise.

de plus l'organisation du repository est tres tres important et chaque ajout
doit etre fait intelligemment afin d'eviter que quelqu'un soit oblige de
repasser derriere pour reparer ces erreurs.

-- wiki

le wiki contient trois parties: une partie publique et deux wikis: un pour
les etudiants et un pour les developpeurs.

le wiki kaneton.org doit etre utilise en priorite du wiki associe avec
le systeme de tickets etc..

participer au wiki est tres important puisqu'il permet de mettre en avant
l'evolution du projet.

neanmoins suivant le nombre de developpeurs impliques il est normal que
certains soient moins actifs que d'autres mais neanmoins participer a
la documentation et donc au wiki est aussi important que le code lui-meme.

-- tickets

le systeme de tickets/bugs est egalement tres important. chaque ticket se
voit affecte une priorite et il est important de comprendre que pour le
bien global du projet, un developpeur ne peut se contenter de faire ce
qui lui plait, il se doit de contribuer egalement a la resolution de problemes.

encore une fois les tickets doivent suivre une norme.

%
% ---------- development ------------------------------------------------------
%

pour chaque language explique les regles: header, commentaire etc..

minuscule

-- c

-- make

-- python

-- latex

figures in .fig

-- asm

-- shell

XXX
ca ne devrait pas juste expliquer les regles mais aussi comment utiliser
les outils, syntaxes etc.. genre configure/, test/, cheat/ etc..

------------------------------

introduction

source-tree

tools

  internal
  external

rules

  tools
    XXX
  languages
    c
    python
    ...