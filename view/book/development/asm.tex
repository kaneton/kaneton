%
% ---------- header -----------------------------------------------------------
%
% project       kaneton
%
% license       kaneton
%
% file          /home/mycure/kaneton/view/book/development/asm.tex
%
% created       julien quintard   [sat jun  2 11:08:30 2007]
% updated       julien quintard   [thu jun 14 21:56:04 2007]
%

%
% ---------- asm --------------------------------------------------------------
%

\section{Assembly}
\label{section:assembly}

The \textit{Assembly} language is used in kaneton to perform tasks which
cannot be done in \textit{C}. Therefore, the amount of \textit{Assembly}
code is relatively small compared to the \textit{C} one.

%
% inline assembly
%

\subsection{Inline Assembly}

First of all, inline \textit{Assembly} is recommended as it avoids multiple
files and calls to \textit{Assembly} routines.

Inline \textit{Assembly} code must be composed of lines with single
instructions. Lines must be ended by a newline character \textit{$\backslash$n}
and aligned to make the code readable even when nested in the body of a macro
or macro function.

\begin{verbatim}
  #define EXCEPTION_PREHANDLER_CODE(nr)                                 \
    asm(`.globl prehandler_exception' #nr `            \n'              \
        `prehandler_exception' #nr `:                  \n'              \
        SAVE_CONTEXT()                                                  \
        `iret                                          ')
\end{verbatim}

Note that rules implied by the \textit{C} language must be respected when
dealing with inline \textit{Assembly} as these pieces of code are integrated
in \textit{C} functions.

%
% naming
%

\subsection{Naming}

Bunches of coherent routines must be preceded by a comment, starting with
the name of the routine in upper case letters, and followed by a description
of the routines, in lower-case letters.

%
% layout
%

\subsection{Layout}

An \textit{Assembly} file must be composed of sections:

\begin{itemize}
  \item
    \textbf{header}: this section contains the file header which provides
    information on the file edition: creation, last update etc.
  \item
    \textbf{information}: this section describes what the file is intended
    to do or provide.
  \item
    \textbf{routines}: this section contains the \textit{Assembly} routines.
  \item
    \textbf{data}: this section contains the data definitions.
\end{itemize}

Below is an example:

\begin{verbatim}
  ;
  ; ---------- information ----------------------------------------------------
  ;
  ; this is a example, no additional information is required.
  ;

  ;
  ; ---------- routines -------------------------------------------------------
  ;

  ;
  ; PRINT STRING
  ;
  ; these routines print a string.
  ;

  print_string:
          mov ah, 0x0e            ; the function to call with int 0x10
          mov bx, 0x07            ; the console attributes

  print_string_loop:
          lodsb                   ; load a character from esi into al
          cmp al, 0               ; is the string finished?
          je print_string_done    ; if true jump to the end

          int 0x10                ; ask the BIOS to perform the task

          jmp print_string        ; loop for the next character

  print_string_done:
          ret                     ; return 'cause the string is now displayed

  ;
  ; MAIN
  ;
  ; the main routine.
  ;

  main:
          mov [bootdrive], dl     ; save the bootdrive identifier
          xor ax, ax              ; initialise ax
          mov ds, ax              ; initialise ds

          cli
          mov ss, ax              ; initialise ss
          mov sp, 0xffff          ; initialise the stack pointer
          sti

          mov si, newline
          call print_string

          mov si, rmode_message
          call print_string

          call floppy_read        ; read the ELF from the floppy

          call pmode_enable       ; enable the protected mode

                                  ; once the protected mode is enabled
                                  ; the function pmode_main will be launched

  ;
  ; ---------- data -----------------------------------------------------------
  ;

  newline         db      10, 0

  rmode_message   db      '[+] real mode', 13, 10, 0

  bootdrive       db      0
\end{verbatim}
