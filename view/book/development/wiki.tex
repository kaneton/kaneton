%
% ---------- header -----------------------------------------------------------
%
% project       kaneton
%
% license       kaneton
%
% file          /home/mycure/kaneton/view/book/development/wiki.tex
%
% created       julien quintard   [thu may 24 23:06:02 2007]
% updated       julien quintard   [fri aug  1 15:52:28 2008]
%

%
% ---------- wiki -------------------------------------------------------------
%

\subsection{Wiki}
\label{section:wiki}

A Wiki is used both for external and internal communication. The software
used is called \name{TWiki} and provides a pretty simple syntax with
many additional plugins to customize the website. This solution was used
for historical reasons but also because the \name{TWiki} rendering can
be customized through templates in order to get a final visual close to
classical websites. Thus, the kaneton website looks like an ordinary
website but powered by a Wiki engine.

The Wiki is hosted at \location{http://kaneton.opaak.org} and contains four
webs: an extranet and three intranets. The main web, accessed through the
address above is the external website. This website contains news, papers,
documentation and general information on the kaneton project. The three other
webs are used more as intranets or wikis more than as websites. Two of these
webs are private to the kaneton developers and the kaneton teachers,
respectively. The latter web is intended to the students and contains
documents, links \etc{} about low-level programming, kernel development \etc{}
as well as information about courses related to the kaneton project.

Note that the Wiki reserved for the kaneton developers must be used
instead of the Wiki eventually provided by the project management tool.

Everybody involved in the kaneton project must contribute to the kaneton
website as well as to the kaneton intranets. Indeed, the external communication
is fundamental, even in an open source project and the kaneton website is
the only public communication medium.

New contributors are then asked to register onto the kaneton \name{TWiki}
at \location{http://kaneton.opaak.org}. Once done, the contributor should
inform the person in charge of the kaneton website so that  the contributor's
account is activated. As a result, the contributor will be able to modify
pages of the website and intranets.
