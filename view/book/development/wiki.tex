%
% ---------- header -----------------------------------------------------------
%
% project       kaneton
%
% license       kaneton
%
% file          /home/mycure/kaneton/view/book/development/wiki.tex
%
% created       julien quintard   [thu may 24 23:06:02 2007]
% updated       julien quintard   [fri may 25 19:26:04 2007]
%

%
% ---------- wiki -------------------------------------------------------------
%

\subsection{Wiki}

A Wiki is used both for external and internal communication. The software
used is called \textit{TWiki} and provides a pretty simple syntax with
many additional plugins to customize the website. This solution was used
for historical reason but also because the \textit{TWiki} rendering can
be customized through templates in order to get a final visual close to
classical websites. Thus, the kaneton website looks like an ordinary
website but powered by a Wiki engine.

The Wiki is hosted at \textit{http://www.kaneton.org} and contains three
webs: an extranet and two intranets. The main web, accessed through the
address above is the external website. This website contains news, papers,
documentation and general information on the kaneton project. The two other
webs are used more as intranet or wiki more than as website. One of the web
is private to the kaneton developers. This web contains stuff the kaneton
reference implementation developers share. The other web is intented to the
students and contains documents, links etc. about low-level programming, kernel
development etc. as well as information about courses related to the
kaneton project.

Note that the Wiki reserved for the kaneton developers must be used
instead of the eventual Wiki provided by the project management tool.

Everybody involved in the kaneton project must contribute to the kaneton
website as well as to the kaneton intranets. Indeed, the external communication
is fundamental, even in an open source project and the kaneton website is
the only public communication medium.

New contributors are then asked to register onto the kaneton \textit{TWiki}
at \textit{http://www.kaneton.org}. Once done, the contributor should
inform the person in charge of the kaneton website. Thus, the contributor
account will be activated so that he can modify pages of the website as
well as access to the private developers Wiki.
