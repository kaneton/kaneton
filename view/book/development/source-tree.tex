%
% ---------- header -----------------------------------------------------------
%
% project       kaneton
%
% license       kaneton
%
% file          /home/mycure/kaneton/view/book/development/source-tree.tex
%
% created       julien quintard   [thu may 17 22:41:36 2007]
% updated       julien quintard   [sun may 20 18:09:37 2007]
%

%
% ---------- source tree ------------------------------------------------------
%

\chapter{Source Tree}

In this chapter we will briefly describe the kaneton microkernel project
source tree.

\newpage

%
% ---------- text -------------------------------------------------------------
%

The kaneton microkernel reference source tree looks like the following
listing:

\begin{verbatim}
cheat/
environment/
export/
configure/
kaneton/
library/
license/
tool/
test/
transcript/
view/
\end{verbatim}

% cheat/

\subsection*{cheat/}

Since the kaneton microkernel is implemented by students, the kaneton
people need to check whether students are cheating by re-using parts of
previous years projects or other kernel source code available on the
\textit{Internet}.

To avoid cheating, kaneton people developed a software checking for
commonalities between different source codes.

The \textit{cheat/} kaneton subdirectory contains everything necessary
to perform such tasks including the software, the different source codes
found on the internet, the kaneton students' implementation from the
previous years etc..

% environment/

\subsection*{environment/}

This directory contains everything necessary to the kaneton development
environment.

The kaneton development environment allows different developers to
interact on the development of the same microkernel in a pretty easy way.

The development environment aims at providing developers to possibility to
work in a collaborative manner without interfering with each other. These
developers are likely to run different operating systems on different
microprocessors. In addition, the kaneton microkernel can be targeted for
different microprocessor architectures. The development environment was
introduced to cope with these combinaisons by providing profiles, each
profile describing the behaviour of a component: underlying operating system,
target architecture, user-specific stuff etc..

As a result, each developer can use a different operating system and
microprocessor architecture with his own specific compiling flags, kaneton
parameters etc.. without modifying another developer's configuration.

% export/

\subsection*{export/}

The \textit{export/} directory contains scripts used to generate a kaneton
tarball in order to be distributed to the students at the beginning of the
kaneton project.

Indeed, these scripts rearrange the kaneton hierarchy hidding some
important directories the students do not need to be aware of.

Moreover some source code parts are removed since the students then have to
rewrite these piece of code.

% configure/

\subsection*{configure/}

This directory contains everything necessary for configuring its own
kaneton microkernel development environment through the compiling process
to the boot system.

Any new contributor should first look at this directory.

% kaneton/

\subsection*{kaneton/}

This directory is the most important of the project since it contains
the whole microkernel source code.

The directory is composed of three important subdirectories: \textit{core/},
\textit{platform} and \textit{architecture}. These subdirectories are described
next.

% kaneton/core/

\subsection*{kaneton/core/}

This directory contains the kaneton core source code.

The directory is divided as shown below:

\begin{verbatim}
as/
region/
sched/
segment/
set/
task/
thread/
[...]
\end{verbatim}

Each directory represents a kaneton core manager. For more information on
the kaneton core, please refer to the appropriate document:
\textit{The kaneton microkernel :: reference implementation}

% kaneton/platform/

\subsection*{kaneton/platform/}

This directory contains everything in relation with what the kaneton
microkernel project calls a \textbf{platform}. The platform represents the
board supporting the devices: microprocessor, memory, peripherals etc..

This directory obviously contains subdirectories for each platform
supported by the kaneton microkernel.

% kaneton/architecture

\subsection*{kaneton/architecture}

The \textit{architecture} directory contains the source-code related to
the microprocessor architecture.

This directory is composed of subdirectories, each one representing an
architecture: \textit{ia32}, \textit{mips64} etc..

% library/

\subsection*{library/}

This directory contains the libraries used either by the kaneton microkernel
itself or by the kaneton microkernel servers. This directory especially
contrains the standard \textit{kaneton C library}.

% license/

\subsection*{license/}

This directory contains the licenses used for any program or document
in relation with the kaneton microkernel project.

Each student has to read and accept the kaneton license before implementing
or even using the kaneton microkernel project.

% tool/

\subsection*{tool/}

This directory contains additional files used by the kaneton development
environment.

% test/

\subsection*{test/}

Since the kaneton microkernel is used as a material for operating system
courses, the kaneton microkernel reference which is the basis of students'
work is to be extremely reliable.

The kaneton project therefore contains a set of tools in order to validate
the kaneton reference implementation behaviour. These tools are also used
for evaluating the correctness of the students' implementation.

The \textit{test/} directory so contains the set kaneton scripts and tests
for validating a kaneton microkernel implementation.

% transcript/

\subsection*{transcript/}

This directory contains real-time recorded sessions. These sessions can be
replayed showing how to use the kaneton microkernel project and more
specifically the kaneton development environement.

% view/

\subsection*{view/}

This directory contains all the kaneton documents including kaneton
administrative documents, exams documents, lectures materials, kaneton papers
and books etc..

Additionally, scripts are provided in order to very easily build and
display these documents.
