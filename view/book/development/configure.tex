%
% ---------- header -----------------------------------------------------------
%
% project       kaneton
%
% license       kaneton
%
% file          /home/mycure/kaneton/view/book/development/configure.tex
%
% created       julien quintard   [tue may 22 22:34:37 2007]
% updated       julien quintard   [mon may 26 17:45:12 2008]
%

%
% ---------- configure --------------------------------------------------------
%

\subsection{Configure}
\label{section:configure}

The \name{configure} tool provides the final user a very user-friendly
software for customizing its development environment.

Recall the development environment is basically composed of three profiles:
\name{host} which describes the operating system behaviour, \name{kaneton}
which parameterizes the kaneton microkernel and \name{user} which contains
some user-specific definitions.

The kaneton development environment is thus used to configure the environment
behaviour as well as the kaneton microkernel itself.

The \location{environment/} directory, and more precisely the environment
profile directories, contain \name{description} files which actually
describe the environment variables. These files are not used by the development
environment but rather by the \name{configure} tool.

The \location{configure/} directory is composed of \name{frame} files
which contain frame descriptions. A frame can be seen as a menu presented
to the final user. A frame is composed of meta-data but also sub-frame and
variable entries.

The \name{configure} tool works as follow. It starts by processing the
environment development configuration files as the environment engine did
for the generation of the configured environment files. Note that the
\name{configure} tool also processes the description files. Also, it
focuses on variables and actually ignores the \name{Make} and \name{Python}
functions.

Once this step is done, the tool gets a list of configured and fully described
variables. Then, the \name{configure} tool displays the first frame and
waits for the user to choose an entry.

The user has the possibility to either move to another menu - if any sub-frame
entry is present - or configure a variable of the list. If the user chooses
to configure a variable, then, the tool displays information based on the
variable's description.

Every modifications of the development environment are private to the actual
user. Therefore, any variable modification adds or modifies an entry in the
related \name{user} profile's configuration file.

Note that the \name{configure} tool is not an environment configuration
files editor. Indeed, this tool targets final users and therefore tries to
be as simple as possible.

The basic \name{configure} behaviour consists in displaying the final
variable's value. If the user enters a new value, no matter whether there is
a relation with its previous value, the tool creates/modifies an entry in the
\name{user} profile's configuration file overriding any previous definition.

For instance, consider the \code{\_FOO\_} development environment variable
and the following configuration files:

In \location{environment/profile/environment.conf}:

\begin{verbatim}
  _FOO_                         =                       initial
\end{verbatim}

In \location{environment/profile/core/core.conf}:

\begin{verbatim}
  _FOO_                         +=                      addon
\end{verbatim}

Let us suppose the user enters the following value instead of the current
one: \code{initial addon}.

\begin{verbatim}
  _FOO_                         =                       something new
\end{verbatim}

Then, the \name{configure} tool creates a new entry into the \name{user}
profile configuration file:

\begin{verbatim}
  _FOO_                         =                       something new
\end{verbatim}

Finally, note that when the \name{configure} tool is launched, it first
tries to detect whether the user is a newcomer or not. If it is, then the
tool asks the user to create a new \name{user} profile, step by step. These
actions are performed in the \location{critical.py} script of the
\location{configure/} directory.

% requirements

\subsubsubsection{Requirements}

The \name{configure} tool relies on the \name{Dialog} software which
is present on many \name{Unix} systems. Indeed, the \name{configure}
tool is a user-friendly configuration utility.

Since \name{configure} needs to update the \name{user} profile
configuration file, this file must be unique and easy to locate. Therefore,
the \name{configure} tool supposes this configuration file is accessible at:

\begin{verbatim}
  environment/profile/user/${KANETON_USER}/${KANETON_USER}.conf
\end{verbatim}

Finally, the \name{Python} module \name{PyYAML} is required as the
description and frame files use the \name{YAML} syntax, as described next.

% syntax

\subsubsubsection{Syntax}

The syntax of the \name{frame} files \location{.frm} is based on \name{YAML}.

As said previously, a frame is composed of sub-frame and variable entries. A
sub-frame entry contains a name and a path to the sub-frame \name{frame}
file while a variable entry only contains the name of the variable. This
variable name is then used to retrieve the variable description.

In addition, a section containing a title and a description is used to
customize the display presented to the user.

The example below illustrates this very simple syntax:

\begin{verbatim}
  #
  # ---------- information ----------------------------------------------------
  #
  # this is the main menu of segment manager.
  #

  #
  # ---------- general --------------------------------------------------------
  #

  - title: Segment Manager
    description: |
      This section contains configuration about the core segment manager.

  #
  # ---------- frames ---------------------------------------------------------
  #

  - frame: Optimisations
    path: subsections/optimisations.desc

  - frame: Machine-dependent
    path: subsections/machine.desc

  #
  # ---------- variables ------------------------------------------------------
  #

  - variable: _FOO_

  - variable: _BAR_

  - variable: _CHICHE_
\end{verbatim}
