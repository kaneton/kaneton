%
% ---------- header -----------------------------------------------------------
%
% project       kaneton
%
% license       kaneton
%
% file          /home/mycure/kaneton/view/book/development/configure.tex
%
% created       julien quintard   [tue may 22 22:34:37 2007]
% updated       julien quintard   [wed may 23 01:53:17 2007]
%

%
% ---------- configure --------------------------------------------------------
%

\subsection{Configure}

The \textit{configure} tool provides the final user a very user-friendly
software for customizing its development environment.

Recall the development environment is basically composed of three profiles:
\textit{host} which describes the operating system behaviour, \textit{kaneton}
which parameterizes the kaneton microkernel and \textit{user} which contains
some user-specific definitions.

The kaneton development environment is thus used to configure the environment
behaviour as well as the kaneton microkernel itself.

The \textit{environment/} directory, and more precisely the environment
profiles, contain \textit{description} files which actually describe the
environment variables. These files are not used by the development
environment but rather by the \textit{configure} tool.

The \textit{configure/} directory is composed of \textit{description} files as
well but these files contain frame descriptions rather than variable
descriptions. A frame can be seen as a menu presented to the final user. A
frame is composed of sub-frame and variable entries.

The \textit{configure} tool works as follow. It starts by processing the
environment development configuration files as the environment engine did
for the generation of the configured environment files. Note that the
\textit{configure} tool also processes the description files. Also, it
focuses on variables and actually ignores the interfaces' functions.

Once this step is done, the tool gets a list of configured and fully described
variables. Then, the \textit{configure} tool displays the current frame
according to the first frame description file.

The user has the possibility to either move to another menu - if any sub-frame
entry is present - or configure a variable of the list. If the user chooses
to configure a variable, then, the tool displays information based on the
variable's description.

Every modifications of the development environment are private to the actual
user. Therefore, any variable modification adds or modifies an entry in the
related \textit{user} profile.

Note that the \textit{configure} tool is not a environment configuration
files editor. Indeed, this tool targets final users and therefore has to
be as simple as possible.

The basic \textit{configure} behaviour consists in displaying the final
variable's value. If the user enters a new value, no matter whether there is
a relation with its previous value, the tool creates/modifies an entry in the
\textit{user} profile's configuration file overriding any previous definition.

For instance, consider the \textit{\_FOO\_} development environment variable
with the following configuration definition:

In \textit{profile/environment.conf}:

\begin{verbatim}
  _FOO_                         =                       initial
\end{verbatim}

In \textit{profile/core/core.conf}:

\begin{verbatim}
  _FOO_                         +=                      addon
\end{verbatim}

Let us suppose the user enters the following value instead of the current
one: \verb|initial addon|

\begin{verbatim}
  _FOO_                         =                       initial new
\end{verbatim}

Then, the \textit{configure} tool creates a new entry into the \textit{user}
profile configuration file:

\begin{verbatim}
  _FOO_                         =                       initial new
\end{verbatim}

% requirements

\subsubsubsection{Requirements}

The \textit{configure} tool relies on the \textit{Dialog} software which
is present on many \textit{Unix} systems.

However, if \textit{configure} does not detect the \textit{Dialog} software,
it uses simple text input and output, making the tool much less user-friendly.

% syntax

\subsubsubsection{Syntax}

The syntax of the frame description files is based on \textit{YAML}. Therefore,
the \textit{Python} \textit{PyYAML} module needs to be set up.

As said previously, a frame is composed of sub-frame and variable entries. A
sub-frame entry contains a name and a path to the sub-frame description file
while a variable entry only contains the name of the variable. This variable
name is then used to retrieve the variable description.

The example below illustrates this very simple syntax:

\begin{verbatim}
  - frame: optimisations
    path: subsections/optimisations.desc
    description: |
      This subsection focuses on kaneton optimisations.

  - frame: machine dependent
    path: subsections/machine.desc
    description: |
      Feel free to configure the machine dependent stuff.

  - variable: _FOO_

  - variable: _BAR_

  - variable: _CHICHE_
\end{verbatim}
