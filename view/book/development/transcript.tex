%
% ---------- header -----------------------------------------------------------
%
% project       kaneton
%
% license       kaneton
%
% file          /home/mycure/kaneton/view/book/development/transcript.tex
%
% created       julien quintard   [thu may 24 05:07:02 2007]
% updated       julien quintard   [thu may 31 08:33:22 2007]
%

%
% ---------- transcript -------------------------------------------------------
%

\subsection{Transcript}
\label{section:transcript}

The \textit{transcript/} directory is composed of two tools related to
the management of transcripts. The \textit{record} tool captures a
shell session while the \textit{play} tool replays a captured session.

These tool were introduced to allow the students to make a dynamic presentation
of their kaneton implementation's features and possibilities. These dynamic
presentations were supposed to replace the oral examinations.

These transcripts are not used by the founders of the kaneton project at
the moment. However, any teacher interested by this tool can use it.

The \textit{transcripts/} directory contains subdirectories which classify
the transcripts.

The only transcript class currently in place is named \textit{basics} and
contains transcripts illustrating the use of the kaneton internal tools.

XXX
The \textit{record} tool is based on the well-known \textit{Unix}
\textit{script} software. Therefore, the other operating systems must use
a software generating the same format. On the other hand, the \textit{play}
tool is based on the \textit{scriptreplay.pl} script located in the
\textit{tool/} directory.
