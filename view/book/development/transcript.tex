%
% ---------- header -----------------------------------------------------------
%
% project       kaneton
%
% license       kaneton
%
% file          /home/mycure/kaneton/view/book/development/transcript.tex
%
% created       julien quintard   [thu may 24 05:07:02 2007]
% updated       julien quintard   [wed jun 13 22:13:10 2007]
%

%
% ---------- transcript -------------------------------------------------------
%

\subsection{Transcript}
\label{section:transcript}

The \textit{transcript/} directory is composed of two tools related to
the management of transcripts. The \textit{record} tool captures a
shell session while the \textit{play} tool replays a captured session.

These tool were introduced to allow students to make a dynamic presentation
of their kaneton implementation's features and possibilities. These dynamic
presentations were supposed to replace the oral examinations.

These transcripts are not used by the main contributors of the kaneton
project yet. However, any teacher interested by this tool can use it.

The \textit{transcript/} directory contains subdirectories which classify
the transcripts.

The only transcript class currently in place is named \textit{basic} and
contains transcripts illustrating the use of the kaneton internal tools.

Note that the \textit{Unix} \textit{host} profiles rely on the well-known
\textit{Unix} \textit{script} software. Moreover, a tool is provided in
\textit{tool/script/} for replaying \textit{script} captured sessions.
