%
% ---------- header -----------------------------------------------------------
%
% project       kaneton
%
% license       kaneton
%
% file          /home/mycure/kaneton/view/book/development/prototypes.tex
%
% created       julien quintard   [fri jun  1 01:07:42 2007]
% updated       julien quintard   [tue may 20 01:07:59 2008]
%

%
% ---------- prototypes -------------------------------------------------------
%

\subsection{Prototypes}
\label{section:prototypes}

The \location{tool/} directory contains a interesting tool developed by
the kaneton community for generating \name{C} prototypes automatically.

This tool named \name{mkp} must not be used directly as the prototypes
automatic generation can be triggered from the \name{Make} files via the
\name{prototypes} rule. This rule should be called everytime a developer
wants the prototypes to be correct, for instance, before starting a compiling
process.

This tool was introduced to avoid spending time writing prototypes.
Since, this utility generates the prototypes automatically, developers
should no longer write prototypes manually.

This script takes a list of \name{C} header files as arguments. These header
files are supposed to contain a \name{prototypes} section as illustrated
below:

\begin{verbatim}
  /*
   * ---------- prototypes ----------------------------------------------------
   *
   *      ../../core/set/set.c
   *      ../../core/set/set-array.c
   *      ../../core/set/set-ll.c
   *      ../../core/set/set-bpt.c
   *      ../../core/set/set-pipe.c
   *      ../../core/set/set-stack.c
   */

  [...]

  /*
   * eop
   */
\end{verbatim}

The \name{mkp} tool then retrieve the list of the \name{C} source files
listed in the \name{prototypes} section and remove the text between
the \name{prototypes} opening section and the \code{eop} tag.

Then, the script goes through each of the listed source files and tries to
detect \name{C} function declarations. For each function declaration found,
the tool generates a prototype in the corresponding header file.

Below is an example of a header file with such a \name{prototypes}
section and generated prototypes.

\begin{verbatim}
  /*
   * ---------- prototypes ----------------------------------------------------
   *
   *      ../../core/set/set.c
   *      ../../core/set/set-array.c
   *      ../../core/set/set-ll.c
   *      ../../core/set/set-bpt.c
   *      ../../core/set/set-pipe.c
   *      ../../core/set/set-stack.c
   */

  /*
   * ../../core/set/set.c
   */

  t_status                 set_dump(void);

  t_status                 set_size(i_set                          setid,
                                   t_setsz*                       size);

  t_status                 set_new(o_set*                          o);

  t_status                 set_destroy(i_set                       setid);

  t_status                 set_descriptor(i_set                    setid,
                                         o_set**                  o);

  t_status                 set_get(i_set                           setid,
                                  t_id                            id,
                                  void**                          o);

  t_status                 set_init(void);

  t_status                 set_clean(void);


  /*
   * ../../core/set/set-array.c
   */

  t_status                 set_type_array(i_set                    setid);

  t_status                 set_show_array(i_set                    setid);

  t_status                 set_head_array(i_set                    setid,
                                         t_iterator*              iterator);

  [...]

  /*
   * eop
   */
\end{verbatim}
