%
% ---------- header -----------------------------------------------------------
%
% project       kaneton
%
% license       kaneton
%
% file          /home/mycure/kaneton/view/book/development/view.tex
%
% created       julien quintard   [wed may 23 00:36:53 2007]
% updated       julien quintard   [wed jun 13 22:09:19 2007]
%

%
% ---------- view -------------------------------------------------------------
%

\subsection{View}
\label{section:view}

The \textit{view} tool serves as a document database as well as a tool for
building and displaying documents in an easy way.

The kaneton documents are classified, each directory corresponding to a
class of documents. Below are listed the subdirectories of the \textit{view/}
directory.

\begin{verbatim}
  bibliography/
  book/
  curriculum/
  exam/
  feedback/
  figures/
  internship/
  lecture/
  logo/
  package/
  paper/
  template/
\end{verbatim}

The \textit{template/} directory contains templates for every class of
document. The \textit{bibliography/} and \textit{logo/} directories contain,
obviously, the bibliography which is common to all the documents, and the
logos, respectively. The \textit{figures/} directory contains figures
common to all the documents while the \textit{package/} directory contains
additional {\LaTeX} packages.

The directories \textit{curriculum/}, \textit{exam/} and \textit{feedback/}
contain documents in relation with teaching. The \textit{curriculum/}
directory contains official documents like the project year planning,
presentations of the education project etc. The \textit{feedback/}
directory contains documents which are distributed to the students
at the end of the kaneton project in order to get feedback for improving
the project for the next years. Needless to say the \textit{exam/} directory
contains everything related to examinations.

The other directories contain the actual kaneton documentation. The
\textit{books} represent the main documents targeting any public:
contributors, teachers, students etc.. The \textit{papers} are lighter
documents intended to present a specific feature, design etc. The
\textit{lectures} are the courses materials, generally composed of
presentation slides. Finally, the \textit{internship} documentation is
composed of documents written by people partially involved in the kaneton
project.

Any document is composed of a \textit{Make} file and one or more
\textit{\LaTeX} files. The \textit{Make} file always has the same form
with little variations depending on the type of document. For more information
on the rules applying to the \textit{Make} and \textit{\LaTeX} files, please
refer to their respective sections: \textit{Section \ref{section:make}}
and \textit{Section \ref{section:python}}.

The \textit{view} tool basically starts looking for \textit{.tex} files
and builds a list of directories containing documents. Then, it provides
to the user the possibility to build and display a given document. If no
document name is given on the command line, then the tool draws a list
of the available documents.

People contributing to the kaneton documents must take care of following
the rules in relation with the \textit{\LaTeX} language. Moreover, contributors
should look at the existing documents to understand to logic behind all
these rules.

Finally, note that nobody should create a document without discussing it
on the mailing-list first. Especially, be very careful in naming your
documents as people took good care of this directory in order to avoid
it to become messy.

If a document already exists with the same name, then go through the
mailing-list in order to decide whether to keep the current version. If
people decide to keep a document, then, the contributor in charge of writing
the new one should re-organise the documents by creating archives for
each year.
