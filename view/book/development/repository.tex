%
% ---------- header -----------------------------------------------------------
%
% project       kaneton
%
% license       kaneton
%
% file          /home/mycure/kaneton/view/book/development/repository.tex
%
% created       julien quintard   [thu may 24 20:43:26 2007]
% updated       julien quintard   [fri aug  1 15:52:09 2008]
%

%
% ---------- repository -------------------------------------------------------
%

\subsection{Repository}
\label{section:repository}

The repository contains everything related to the kaneton microkernel
project, in other words, the kaneton source tree described in
\reference{Chapter \ref{chapter:source tree}}. Indeed, the repository contains
the whole history of the kaneton project including the documentation, the
source code but also the students tarballs over the years.

The actual repository is based on the \name{Subversion} software which
provides far more advanced features than its historical rival \name{CVS}.

The repository is actually hosted on the \location{kaneton.opaak.org} server
which also contains the web server and everything else related to the
management of the kaneton microkernel project.

The repository is accessed in a secure way through a \name{SSH} channel.
Indeed, the kaneton \name{Subversion} repository can be accessed at the
following address:
\location{svn+ssh://subversion@repositories.opaak.org/kaneton}.

Note that the security is achieved by the use of \name{SSH} keys. Therefore,
any new contributor should get in touch with an administrator of the
kaneton server in order to obtain an access. Also note that, a test period
could be set up for a new contributor to get the trust of the kaneton
community. For more information, please refer to \reference{Chapter
\ref{chapter:community}}.

A contributor willing to create a \name{SSH} key shoud simply use this
\name{Unix} command:

\begin{verbatim}
  $> ssh-keygen -t dsa
\end{verbatim}

For more information about how to use the repository, please refer to the
official \name{Subversion} documentation. The same goes for the
\name{SSH} tools suite.

The example below illustrates the checkout of the kaneton repository.

\begin{verbatim}
  $> svn checkout svn+ssh://subversion@repositories.opaak.org/kaneton
\end{verbatim}

The contributors getting access to the kaneton repositories must behave
properly according to the obvious cooperative development rules. As an
example, a kaneton developer must not perform any commit before making sure
the kaneton microkernel compiles and passes all the tests.

The repository organisation is crucial. Therefore, nothing should be
added, removed or renamed without the permission of the developers in charge
of the repository.

Finally, any commit must come with a log describing the modifications
implied by the commit. These logs must conform to the following syntax.

\begin{verbatim}
  [kaneton/core/segment/]
    o the bug about the permissions was corrected in segment_clone().
    o an algorithm based on a b-tree was added.

  [environment/profile/user/julien.quintard/]
    o some personal configurations were modified.
\end{verbatim}

Following this syntax is very important as an email is sent to the
\name{kaneton-developers} mailing-list every time a commit is performed.
Therefore, the contributors reading the mailing-list are aware of every
modification in the kaneton source code. This feature can also be used
to review the modifications done by a new contributor in order to help
him doing things in a better way.

Note that there must not be any file with the executable flag permission
enabled. Moreover, scripts files must not contain any \textit{shebang}.
Indeed, the kaneton development environment knows which interpreter to
use for every type of file. It is therefore a non-sense to introduce a
hard-coded path to an interpreter.

Tarball file names must be extended with \location{.tar} while \name{bzip2}
compressed tarballs must be extended with \location{.tar.bz2}.
