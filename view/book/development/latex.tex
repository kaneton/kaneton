%
% ---------- header -----------------------------------------------------------
%
% project       kaneton
%
% license       kaneton
%
% file          /home/mycure/kaneton/view/book/development/latex.tex
%
% created       julien quintard   [thu jun 14 09:53:16 2007]
% updated       julien quintard   [thu jun 14 16:09:48 2007]
%

%
% ---------- latex ------------------------------------------------------------
%

\section{\LaTeX}
\label{section:latex}

The kaneton documents are written in the {\LaTeX} language. These documents
are viewable through the \textit{view} tool. For more information about
this tool, please refer to the \textit{Section \ref{section:view}}.

%
% naming
%

\subsection{Naming}

{\LaTeX} files must be expressed in lower-case letters. Moreover,
composite file names must be separated by a dash \textit{-}.

%
% layout
%

\subsection{Layout}

Since {\LaTeX} is not a very readable language, every contributor
is asked to follow the rules described in order to make the documents
internal representation looks like the resulted output.

Indeed, writers are asked not to use {\LaTeX} commands specifying
layout requirements especially about indentation like \verb|\paragraph|,
\verb|\vspace| etc. Since the kaneton project provides templates, every
kaneton document must rely on a template which specifies the paragraph
indentation, paragraph space etc.

Therefore, the writer willing to distinguish two paragraphs in his
document should simply put a single blank line between to two texts
in his {\LaTeX} file as shown below:

\begin{verbatim}
  This is a paragraph which is intended to explain nothing special but
  how to construct paragraphs in a very simple and readable way.

  Then, after a single blank line, this text will be considered as a new
  paragraph.
\end{verbatim}

This rule aims at making the document code as readable as possible. However,
if a vertical space is needed in some place the default paragraph indentation
does not apply, then, the \verb|\-| {\LaTeX} command must be used. This is
especially useful in \textit{lecture} documents.

The example below illustrates such a \textit{lecture} document and the
need of vertical indentation:

\begin{verbatim}
  \begin{frame}
    \frametitle{Description}

    The MIPS processor is a 32-bit little-endian processor.

    \-

    This processor provides \textbf{32 integer registers}, from R0 to R31.

    [...]
\end{verbatim}

Obvisouly, lines must not exceed \textit{80} characters in width.

The templates provided by kaneton are located in the \textit{view/template/}
directory. These include \textit{book}, \textit{paper}, \textit{exam} etc.
templates.

Therefore, every document should start by including the template file.
However, since the template files need to access other files like
{\LaTeX} dependency files, additional packages etc. these templates need to
know where the document directory is located from the root \textit{view/}
directory. Every document must therefore specify the path to the \textit{view/}
directory before including the template file.

Every file should then starts with something similar to the following:

\begin{verbatim}
  %
  % path
  %

  \def\path{../..}

  %
  % template
  %

  %%
%% licence       kaneton licence
%%
%% project       kaneton
%%
%% file          /home/mycure/kaneton/view/templates/book.tex
%%
%% created       julien quintard   [wed mar  1 23:45:22 2006]
%% updated       julien quintard   [thu may  4 12:36:54 2006]
%%

%
% class
%

\documentclass[10pt,a4wide]{book}

%
% packages
%

\usepackage[english]{babel}
\usepackage[T1]{fontenc}
\usepackage{a4wide}
\usepackage{fancyheadings}
\usepackage{multicol}
\usepackage{indentfirst}
\usepackage{graphicx}
\usepackage{color}
\usepackage{xcolor}
\usepackage{verbatim}

\usepackage{aeguill}

\usepackage[Lenny]{../../../tools/latex/fncychap}

\pagestyle{fancy}

\setlength{\footrulewidth}{0.3pt}
\setlength{\parindent}{0.3cm}
\setlength{\parskip}{2ex plus 0.5ex minus 0.2ex}

%
% logos
%

\newcommand{\logos}
  {
    \begin{center}
      \includegraphics[scale=0.8]{../../logos/kaneton.pdf}
    \end{center}
  }

%
% colors
%

\definecolor{functioncolor}{rgb}{0.40,0.00,0.00}
\definecolor{commandcolor}{rgb}{0.00,0.00,0.40}
\definecolor{verbatimcolor}{rgb}{0.00,0.40,0.00}
\definecolor{noticecolor}{rgb}{0.87,0.84,0.02}

%
% function
%

\newcommand\function[3]{
  \begin{tabular}{p{0.2cm}p{13.8cm}}
  & {\color{functioncolor}\textbf{#1}}#2
  \end{tabular}

  \begin{tabular}{p{1cm}p{13cm}}
  & #3
  \end{tabular}}

%
% align
%

\newcommand\align[1]{
  \\ & \hspace{#1}}

%
% argument
%

\newcommand\argument[1]{\textit{#1}}

%
% command
%

\newcommand\command[2]{
  \begin{tabular}{p{0.2cm}p{13.8cm}}
  & {\color{commandcolor}\textbf{#1}}
  \end{tabular}

  \begin{tabular}{p{1cm}p{13cm}}
  & #2
  \end{tabular}}

%
% notice
%

\newcommand\notice[1]{
  {\color{noticecolor}\textbf{Notice}}

  \begin{tabular}{p{0.2cm}p{13.8cm}}
  & #1
  \end{tabular}}

%
% example
%

\newcommand\example[1]{
  \textit{Example:}

  \begin{tabular}{p{0.2cm}p{13.8cm}}
  & \textit{#1}
  \end{tabular}}

%
% warning XXX
%

%
% verbatim stuff
%

\makeatletter

\renewcommand{\verbatim@font}
  {\ttfamily\footnotesize\color{verbatimcolor}\selectfont}

\def\verbatim@processline{\hskip15ex\the\verbatim@line\par}

\makeatother

%
% header
%

\rhead{}
\rfoot{\scriptsize{The kaneton microkernel project}}

\date{\scriptsize{\today}}

\end{verbatim}

As every other type of kaneton file, the {\LaTeX} files are composed of
sections. Below are listed some of possible sections but note that it
highly depends on the organisation of the document: multiple files, etc.
as well as the type of document: presentation slides, paper, book etc.

The best way to make things properly is to look at the existing documents.

\begin{itemize}
  \item
    \textbf{header}: this section contains the file header which provides
    information on the file edition: creation, last update etc.
  \item
    \textbf{setup}: this section contains the {\LaTeX} setup: path to the
    \textit{view/} directory, template including, title definition etc.

    \begin{verbatim}
      %
      % ---------- setup ------------------------------------------------------
      %

      %
      % path
      %

      \def\path{../..}

      %
      % template
      %

      %%
%% licence       kaneton licence
%%
%% project       kaneton
%%
%% file          /home/mycure/kaneton/view/templates/book.tex
%%
%% created       julien quintard   [wed mar  1 23:45:22 2006]
%% updated       julien quintard   [thu may  4 12:36:54 2006]
%%

%
% class
%

\documentclass[10pt,a4wide]{book}

%
% packages
%

\usepackage[english]{babel}
\usepackage[T1]{fontenc}
\usepackage{a4wide}
\usepackage{fancyheadings}
\usepackage{multicol}
\usepackage{indentfirst}
\usepackage{graphicx}
\usepackage{color}
\usepackage{xcolor}
\usepackage{verbatim}

\usepackage{aeguill}

\usepackage[Lenny]{../../../tools/latex/fncychap}

\pagestyle{fancy}

\setlength{\footrulewidth}{0.3pt}
\setlength{\parindent}{0.3cm}
\setlength{\parskip}{2ex plus 0.5ex minus 0.2ex}

%
% logos
%

\newcommand{\logos}
  {
    \begin{center}
      \includegraphics[scale=0.8]{../../logos/kaneton.pdf}
    \end{center}
  }

%
% colors
%

\definecolor{functioncolor}{rgb}{0.40,0.00,0.00}
\definecolor{commandcolor}{rgb}{0.00,0.00,0.40}
\definecolor{verbatimcolor}{rgb}{0.00,0.40,0.00}
\definecolor{noticecolor}{rgb}{0.87,0.84,0.02}

%
% function
%

\newcommand\function[3]{
  \begin{tabular}{p{0.2cm}p{13.8cm}}
  & {\color{functioncolor}\textbf{#1}}#2
  \end{tabular}

  \begin{tabular}{p{1cm}p{13cm}}
  & #3
  \end{tabular}}

%
% align
%

\newcommand\align[1]{
  \\ & \hspace{#1}}

%
% argument
%

\newcommand\argument[1]{\textit{#1}}

%
% command
%

\newcommand\command[2]{
  \begin{tabular}{p{0.2cm}p{13.8cm}}
  & {\color{commandcolor}\textbf{#1}}
  \end{tabular}

  \begin{tabular}{p{1cm}p{13cm}}
  & #2
  \end{tabular}}

%
% notice
%

\newcommand\notice[1]{
  {\color{noticecolor}\textbf{Notice}}

  \begin{tabular}{p{0.2cm}p{13.8cm}}
  & #1
  \end{tabular}}

%
% example
%

\newcommand\example[1]{
  \textit{Example:}

  \begin{tabular}{p{0.2cm}p{13.8cm}}
  & \textit{#1}
  \end{tabular}}

%
% warning XXX
%

%
% verbatim stuff
%

\makeatletter

\renewcommand{\verbatim@font}
  {\ttfamily\footnotesize\color{verbatimcolor}\selectfont}

\def\verbatim@processline{\hskip15ex\the\verbatim@line\par}

\makeatother

%
% header
%

\rhead{}
\rfoot{\scriptsize{The kaneton microkernel project}}

\date{\scriptsize{\today}}


      %
      % header
      %

      \lhead{\scriptsize{The kaneton microkernel :: development book}}
      \rhead{}

      %
      % title
      %

      \title{The kaneton microkernel :: development book
             \logos}
    \end{verbatim}
  \item
    \textbf{text}: this section is used when the text is relatively short
    and likely to fit in a single file. The section then contains the
    whole document's text.

    Instead, a section whose name is based on the current chapter or section
    can be used if the document is much more larger and generally splitted
    into multiple files.

    \begin{verbatim}
      %
      % ---------- latex ------------------------------------------------------
      %

      \section{\LaTeX}

      The kaneton documents are written [...]
    \end{verbatim}
\end{itemize}

%
% style
%

\subsection{Style}

First of all, comments must be written in lower-case letters. Comments
are not intended, in {\LaTeX} files, to describe what the file do but instead
play the role of visual separators.

Note that documents are either \textit{public} or \textit{private}. Indeed,
the \textit{\LaTeX} definition \verb|\mode|, stored in a temporary file
included by kaneton templates, can be used to hide information which must be
kept private to the kaneton developers community like, for instance,
implementation details. For more information, take a look at the template
you are interested in or at the other equivalent kaneton documents.

Recall that the \textit{view/figures/} directory contains figures related
to the kaneton microkernel documents. These figures must always be prefered
to specific figures. Moreover, writers are asked to put their figures in
this directory if the figure is general enough. Moreover, figures must be
in the \textit{Fig} format. Finally, figures must be exported into a
\textit{PDF} format as it is the only format which is accepted both by
pure \textit{\LaTeX} documents as well as \textit{Beamer} documents.

Every time the \verb|\item| element is used, for instance in \textit{itemize},
or \textit{enumerate} environments, the text related to the item must be
placed on a new line, aligned by two characters from the \verb|\item| element.

\begin{verbatim}
  \begin{itemize}[<+->]
    \item
      \textbf{R2}: return value.
    \item
      \textbf{R29}: frame pointer.
    \item
      \textbf{R30}: global variables area pointer.
  \end{itemize}
\end{verbatim}

The remaining of this section is divided into two parts: papers and slides
because depending on the type of document, either pure \textit{\LaTeX}
or the \textit{Beamer} extension is used.

% paper

\subsubsection{Paper}

A long comment separator, composed of three lines, must be used before
every new subsections, while a single line separator must be used for
subsubsections and other less important components.

\begin{verbatim}
  %
  % naming
  %

  \subsection{Naming}

  Names in kaneton must comply to the following rules.

  % general

  \subsubsection{General}
\end{verbatim}

Nevertheless, this rule also depends on the document organisation.

\textit{\LaTeX} references must be used, via the commands \verb|\label| and
\verb|\ref|, making the reading much more easier.

% slides

\subsubsection{Slides}

Unlike papers, presentation documents are based on the \textit{Beamer} package.
These documents are therefore composed of slides or frame in the
\textit{Beamer} terms.

\textit{Lecture} documents are based on such a package through the kaneton
templates, especially the \textit{lecture} template.

Each frame must be preceded by a commented number indicating the slide's
number into the section of subsection.

\begin{verbatim}
  %
  % introduction
  %

  \section{Introduction}

  % 1)

  \begin{frame}
    \frametitle{Description}

    \begin{itemize}[<+->]
      \item
        About \textbf{thirty} hours course.
      \item
        Concluded by an exam.
    \end{itemize}
  \end{frame}

  % 2)

  \begin{frame}
    \frametitle{Contents}

    \begin{itemize}[<+->]
      \item
        External architecture.
      \item
        Pipeline.
      \item
        Compiler optimisations.
      \item
        Memory.
    \end{itemize}
  \end{frame}
\end{verbatim}
