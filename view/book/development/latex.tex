%
% ---------- header -----------------------------------------------------------
%
% project       kaneton
%
% license       kaneton
%
% file          /home/mycure/kaneton/view/book/development/latex.tex
%
% created       julien quintard   [thu jun 14 09:53:16 2007]
% updated       julien quintard   [mon sep 15 13:23:29 2008]
%

%
% ---------- latex ------------------------------------------------------------
%

\section{\LaTeX}
\label{section:latex}

The kaneton documents are written in the {\LaTeX} language. These documents
are viewable through the \name{view} tool. For more information about
this tool, please refer to the \reference{Section \ref{section:view}}.

%
% naming
%

\subsection{Naming}

{\LaTeX} files must be named in lower-case letters. Moreover,
composite file names must be separated by a dash \code{-}.

%
% layout
%

\subsection{Layout}

Since {\LaTeX} is not a very readable language, every contributor
is asked to follow the rules described in order to make the documents
internal representation looks like the resulted output.

Indeed, writers are asked not to use {\LaTeX} commands specifying
layout requirements especially about indentation like \code{\latex{paragraph}},
\code{\latex{vspace}} \etc{} Since the kaneton project provides templates,
every kaneton document must rely on a template which specifies the paragraph
indentation, paragraph space \etc{}

Thus, the writer willing to distinguish two paragraphs in his
document should simply put a single blank line between to two texts
in his {\LaTeX} file as shown below:

\begin{verbatim}
  This is a paragraph which is intended to explain nothing special but
  how to construct paragraphs in a very simple and readable way.

  Then, after a single blank line, this text will be considered as a new
  paragraph.
\end{verbatim}

This rule aims at making the document code as readable as possible. However,
if a vertical space is needed in some place the default paragraph indentation
does not apply, then, the \code{\latex{-}} {\LaTeX} command must be used. This
is especially useful in \name{lecture} documents.

The example below illustrates such a \name{lecture} document and the
need of vertical indentation:

\begin{verbatim}
  \begin{frame}
    \frametitle{Description}

    The MIPS processor is a 32-bit little-endian processor.

    \-

    This processor provides \textbf{32 integer registers}, from R0 to R31.

    [...]
\end{verbatim}

Obvisouly, lines must not exceed $80$ characters in width.

The templates provided by kaneton are located in the \location{view/template/}
directory. Those include \name{book}, \name{paper}, \name{exam} \etc{}

Therefore, every document should start by including the template file.
However, since the template files need to access other files like
{\LaTeX} dependency files, additional packages \etc{} these templates need to
know where the document directory is located from the root \location{view/}
directory. Every document must therefore specify the path to the
\location{view/} directory before including the template file.

Every file should then start with a \name{setup} section similar to the
following:

\begin{verbatim}
  %
  % ---------- setup ----------------------------------------------------------
  %

  %
  % path
  %

  \def\path{../..}

  %
  % template
  %

  %%
%% licence       kaneton licence
%%
%% project       kaneton
%%
%% file          /home/mycure/kaneton/view/templates/book.tex
%%
%% created       julien quintard   [wed mar  1 23:45:22 2006]
%% updated       julien quintard   [thu may  4 12:36:54 2006]
%%

%
% class
%

\documentclass[10pt,a4wide]{book}

%
% packages
%

\usepackage[english]{babel}
\usepackage[T1]{fontenc}
\usepackage{a4wide}
\usepackage{fancyheadings}
\usepackage{multicol}
\usepackage{indentfirst}
\usepackage{graphicx}
\usepackage{color}
\usepackage{xcolor}
\usepackage{verbatim}

\usepackage{aeguill}

\usepackage[Lenny]{../../../tools/latex/fncychap}

\pagestyle{fancy}

\setlength{\footrulewidth}{0.3pt}
\setlength{\parindent}{0.3cm}
\setlength{\parskip}{2ex plus 0.5ex minus 0.2ex}

%
% logos
%

\newcommand{\logos}
  {
    \begin{center}
      \includegraphics[scale=0.8]{../../logos/kaneton.pdf}
    \end{center}
  }

%
% colors
%

\definecolor{functioncolor}{rgb}{0.40,0.00,0.00}
\definecolor{commandcolor}{rgb}{0.00,0.00,0.40}
\definecolor{verbatimcolor}{rgb}{0.00,0.40,0.00}
\definecolor{noticecolor}{rgb}{0.87,0.84,0.02}

%
% function
%

\newcommand\function[3]{
  \begin{tabular}{p{0.2cm}p{13.8cm}}
  & {\color{functioncolor}\textbf{#1}}#2
  \end{tabular}

  \begin{tabular}{p{1cm}p{13cm}}
  & #3
  \end{tabular}}

%
% align
%

\newcommand\align[1]{
  \\ & \hspace{#1}}

%
% argument
%

\newcommand\argument[1]{\textit{#1}}

%
% command
%

\newcommand\command[2]{
  \begin{tabular}{p{0.2cm}p{13.8cm}}
  & {\color{commandcolor}\textbf{#1}}
  \end{tabular}

  \begin{tabular}{p{1cm}p{13cm}}
  & #2
  \end{tabular}}

%
% notice
%

\newcommand\notice[1]{
  {\color{noticecolor}\textbf{Notice}}

  \begin{tabular}{p{0.2cm}p{13.8cm}}
  & #1
  \end{tabular}}

%
% example
%

\newcommand\example[1]{
  \textit{Example:}

  \begin{tabular}{p{0.2cm}p{13.8cm}}
  & \textit{#1}
  \end{tabular}}

%
% warning XXX
%

%
% verbatim stuff
%

\makeatletter

\renewcommand{\verbatim@font}
  {\ttfamily\footnotesize\color{verbatimcolor}\selectfont}

\def\verbatim@processline{\hskip15ex\the\verbatim@line\par}

\makeatother

%
% header
%

\rhead{}
\rfoot{\scriptsize{The kaneton microkernel project}}

\date{\scriptsize{\today}}

\end{verbatim}

As every other type of kaneton file, the {\LaTeX} files are composed of
sections. Below are listed some of possible sections but note that it
highly depends on the organisation of the document: multiple files, \etc{}
as well as the type of document: presentation slides, paper, book \etc{}

The best way to make things properly is to look at the existing documents.

\begin{itemize}
  \item
    \term{header}: this section contains the file header which provides
    information on the file edition: creation, last update \etc{}
  \item
    \term{setup}: this section contains the {\LaTeX} setup: path to the
    \location{view/} directory, template including, title definition \etc{}

    \begin{verbatim}
      %
      % ---------- setup ------------------------------------------------------
      %

      %
      % path
      %

      \def\path{../..}

      %
      % template
      %

      %%
%% licence       kaneton licence
%%
%% project       kaneton
%%
%% file          /home/mycure/kaneton/view/templates/book.tex
%%
%% created       julien quintard   [wed mar  1 23:45:22 2006]
%% updated       julien quintard   [thu may  4 12:36:54 2006]
%%

%
% class
%

\documentclass[10pt,a4wide]{book}

%
% packages
%

\usepackage[english]{babel}
\usepackage[T1]{fontenc}
\usepackage{a4wide}
\usepackage{fancyheadings}
\usepackage{multicol}
\usepackage{indentfirst}
\usepackage{graphicx}
\usepackage{color}
\usepackage{xcolor}
\usepackage{verbatim}

\usepackage{aeguill}

\usepackage[Lenny]{../../../tools/latex/fncychap}

\pagestyle{fancy}

\setlength{\footrulewidth}{0.3pt}
\setlength{\parindent}{0.3cm}
\setlength{\parskip}{2ex plus 0.5ex minus 0.2ex}

%
% logos
%

\newcommand{\logos}
  {
    \begin{center}
      \includegraphics[scale=0.8]{../../logos/kaneton.pdf}
    \end{center}
  }

%
% colors
%

\definecolor{functioncolor}{rgb}{0.40,0.00,0.00}
\definecolor{commandcolor}{rgb}{0.00,0.00,0.40}
\definecolor{verbatimcolor}{rgb}{0.00,0.40,0.00}
\definecolor{noticecolor}{rgb}{0.87,0.84,0.02}

%
% function
%

\newcommand\function[3]{
  \begin{tabular}{p{0.2cm}p{13.8cm}}
  & {\color{functioncolor}\textbf{#1}}#2
  \end{tabular}

  \begin{tabular}{p{1cm}p{13cm}}
  & #3
  \end{tabular}}

%
% align
%

\newcommand\align[1]{
  \\ & \hspace{#1}}

%
% argument
%

\newcommand\argument[1]{\textit{#1}}

%
% command
%

\newcommand\command[2]{
  \begin{tabular}{p{0.2cm}p{13.8cm}}
  & {\color{commandcolor}\textbf{#1}}
  \end{tabular}

  \begin{tabular}{p{1cm}p{13cm}}
  & #2
  \end{tabular}}

%
% notice
%

\newcommand\notice[1]{
  {\color{noticecolor}\textbf{Notice}}

  \begin{tabular}{p{0.2cm}p{13.8cm}}
  & #1
  \end{tabular}}

%
% example
%

\newcommand\example[1]{
  \textit{Example:}

  \begin{tabular}{p{0.2cm}p{13.8cm}}
  & \textit{#1}
  \end{tabular}}

%
% warning XXX
%

%
% verbatim stuff
%

\makeatletter

\renewcommand{\verbatim@font}
  {\ttfamily\footnotesize\color{verbatimcolor}\selectfont}

\def\verbatim@processline{\hskip15ex\the\verbatim@line\par}

\makeatother

%
% header
%

\rhead{}
\rfoot{\scriptsize{The kaneton microkernel project}}

\date{\scriptsize{\today}}


      %
      % header
      %

      \lhead{\scriptsize{The kaneton microkernel :: development}}

      %
      % title
      %

      \title{The kaneton microkernel :: development
             \version
             \logo}
    \end{verbatim}
  \item
    \term{text}: this section is used when the text is relatively short
    and likely to fit in a single file. The section then contains the
    whole document's text.

    Instead, a section whose name is based on the current chapter or section
    can be used if the document is much more larger and generally split
    into multiple files.

    \begin{verbatim}
      %
      % ---------- latex ------------------------------------------------------
      %

      \section{\LaTeX}

      The kaneton documents are written [...]
    \end{verbatim}
\end{itemize}

%
% style
%

\subsection{Style}

First of all, comments must be written in lower-case letters. Comments
are not intended, in {\LaTeX} files, to describe what the file do but instead
play the role of visual separators.

Note that documents are either \name{public} or \name{private}. Indeed,
the {\LaTeX} definition \code{\latex{mode}}, stored in a temporary file
included by kaneton templates, can be used to hide information which must be
kept private to the kaneton developers community like, for instance,
implementation details. For more information, take a look at the template
you are interested in or at the other equivalent kaneton documents.

Recall that the \location{view/figures/} directory contains figures related
to the kaneton microkernel documents. These figures must always be prefered
to specific figures. Moreover, writers are asked to put their figures in
this directory if the figure is general enough. Furthermore, figures must be
in the \name{FIG} format. Finally, figures must be exported into the
\name{PDF} format as it is the only format which is accepted both by
pure {\LaTeX} documents as well as \name{Beamer} documents while rendering
without quality loss.

Remember that every element must be aligned according to its parent. For
instance, in the \name{enumerate} environment, the text related to the item
must be placed on a new line, aligned by two characters from the
\code{\latex{item}} element as shown below.

\begin{verbatim}
  \begin{itemize}[<+->]
    \item
      \textbf{R2}: return value.
    \item
      \textbf{R29}: frame pointer.
    \item
      \textbf{R30}: global variables area pointer.
  \end{itemize}
\end{verbatim}

% templates

\subsubsection{Templates}

The \name{kaneton} development environment provides templates in order to
making writing documents easier. The provided templates can be grouped
into two categories depending on the type of rendering: \term{article}s
are pure {\LaTeX} documents while \term{presentation}s are \name{Beamer}-based
slides.

The templates \name{book}, \name{exam}, \name{feedback}, \name{internship}
and \name{paper} belong to the \name{article} category while the templates
\name{lecture} and \name{talk} are \name{presentation}s.

The remaining of this section is details the style according to the document
category.

% article

\subsubsubsection{Article}

A long comment separator, composed of three lines, must be used before
every new subsections, while a single line separator must be used for
subsubsections and other less important components.

\begin{verbatim}
  %
  % naming
  %

  \subsection{Naming}

  Names in kaneton must comply to the following rules.

  % general

  \subsubsection{General}
\end{verbatim}

Nevertheless, this rule also depends on the document organisation.

% presentation

\subsubsubsection{Presentation}

Unlike \name{article}s, \name{presentation}s are based on the \name{Beamer}
package. These documents are therefore composed of slides --- frames in the
\name{Beamer} terminology.

Each frame must be preceded by a commented number indicating the slide's
number into the section of subsection.

\begin{verbatim}
  %
  % introduction
  %

  \section{Introduction}

  % 1)

  \begin{frame}
    \frametitle{Description}

    \begin{itemize}[<+->]
      \item
        About \textbf{thirty} hours course.
      \item
        Concluded by an exam.
    \end{itemize}
  \end{frame}

  % 2)

  \begin{frame}
    \frametitle{Contents}

    \begin{itemize}[<+->]
      \item
        External architecture.
      \item
        Pipeline.
      \item
        Compiler optimisations.
      \item
        Memory.
    \end{itemize}
  \end{frame}
\end{verbatim}

% _opk_ package

\subsubsection{\name{opk} Package}

Since every template is likely to need providing writers the same set
of functionalities, a {\LaTeX} package has been written. The \term{opk} package
includes a set of commands useful for building function definitions,
referencing {\LaTeX} labels \etc{} This package is located in
\location{view/package/opk/} and is composed of the functions below.

First, needless to define the \code{\latex{author}} directive as the \name{opk}
package does it automatically, unless the writer knows what he is doing.
Additionally, the package redefines the \name{verbatim} environment.

\function{}{\latex{term}}{\argument{text}}
         {
           This function is used for introducing new terms.
         }

\function{}{\latex{name}}{\argument{text}}
         {
           This function is used for referering to already introduced names.
         }

\function{}{\latex{code}}{\argument{text}}
         {
           This function is used for words that represent function names
           or anything related to source code.
         }

\function{}{\latex{reference}}{\argument{section/figure/\etc{}}}
         {
           This command is used for referencing figures, sections \etc{}.

           \-

           Note that this command should encapsulate the whole text such
           as: \code{\latex{reference}\{Figure
           \latex{ref}\{figure:Experiments\}\}}.
         }

\function{}{\latex{location}}{\argument{location}}
         {
           This command is used for describing locations: path, \name{URL}
           \etc{}
         }

\function{}{\latex{function}}{\argument{return type},
                              \argument{function name},
                              \argument{arguments list},
                              \argument{description text}}
         {
           This special command is used for describing function definitions.
         }

\function{}{\latex{type}}{\argument{argument}}
         {
           This function is for describing arguments.

           \-

           This command should only be used in the \code{\latex{function}()}
           command.
         }

\function{}{\latex{command}}{\argument{command line},
                             \argument{description text}}
         {
           This function is equivalent to the \code{\latex{function}()} one
           but targets \name{Shell}, \name{Python} \etc{} commands rather than
           language functions.
         }

\function{}{\latex{subsubsubsection}}{\argument{section name}}
         {
           This weird command is just an easier way of sub-dividing the
           text, once more.
         }

\function{}{\latex{question}}{\argument{text},
                              \argument{answer length}}
         {
           This command is used in \name{feedback}-template document for
           creating questions.
         }

\function{}{\latex{latex}}{\argument{command name}}
         {
           This command ``latexify'' the given command name by putting
           a backslash in front of it.
         }

\function{}{\latex{note}}{\argument{text}}
         {
           This command is used for making little notes, tips, hints \etc{}
         }

\function{}{\latex{example}}{\argument{text}}
         {
           This command is used for creating examples.
         }

\function{}{\latex{ie}}{}
         {
           This command genereates \ie{}.
         }

\function{}{\latex{eg}}{}
         {
           This command genereates \eg{}.
         }

\function{}{\latex{etc}}{}
         {
           This command genereates \etc{}.
         }

\function{}{\latex{aka}}{}
         {
           This command genereates \aka{}.
         }

Plus, layout-oriented commands are also provided.

\function{}{\latex{indentation}}{}
         {
           This command initialises the indentation for the document.
         }

\function{}{\latex{logo}}{}
         {
           This command displays the \name{kaneton} logo.
         }

\function{}{\latex{version}}{}
         {
           This command displays the version type of this document according
           to the \code{\latex{mode}} definition.
         }

In addition, specific {\LaTeX} environments are provided.

\function{}{\latex{details}}{}
         {
           The \name{details} environment enables writers to include additional
           information which will only be viewable to private members \ie{}
           in the private version of the document.
         }

\function{}{\latex{correction}}{}
         {
           The \name{correction} environment plays the same role as the
           \name{details} environment but is used in \name{exam}-template
           documents for providing exercises' answer.
         }
