%
% ---------- header -----------------------------------------------------------
%
% project       kaneton
%
% license       kaneton
%
% file          /home/mycure/kaneton/view/book/development/history.tex
%
% created       julien quintard   [fri jun  1 14:33:16 2007]
% updated       julien quintard   [mon jun 18 12:38:23 2007]
%

%
% ---------- history ----------------------------------------------------------
%

\chapter{History}

In this chapter we detail the kaneton history from the first
year with low-level programming introduction to the last kaneton
microkernel implementation.

\newpage

%
% ---------- text -------------------------------------------------------------
%

During the kaneton history, the project evolved and courses were added
to the curriculum to make the whole kaneton project more interesting and
understandable by the students. Moreover, the educational project, which
was already targetting the \textit{EPITA}'s \textit{System, Network and
Security} major, was also used in other contexts, victim of its success
and of the very hard work achieved by kaneton people over the years.

%
% 2004
%

\section{2004}

The first year, a low-level programming introduction course named \textbf{k}
was proposed for the \textit{EPITA} Engineering School's first year students.

About fourteen hours courses were taught introducing the \textit{Intel 32-bit}
microprocessor's external architecture and low-level programming.

The students had to develop small, poor and messy device drivers for the
console and keyboard peripherals. Moreover, a tiny command interpreter was
developed by students so that a kernel action could be triggered by entering
a command.

The course was a bit chaotic but this first shot was a success.

Therefore, the students majoring in \textit{System, Network and Security}
asked the two students \textit{Julien Quintard} and \textit{Jean-Pascal
Billaud} a complete kernel project for their curriculum so that they can
learn more about operating systems internals.

%
% 2005
%

\section{2005}

The two, still students, \textit{Julien Quintard} and \textit{Jean-Pascal
Billaud} then prepared a complete microkernel design the students will have
to implement. This was the premises of the \textbf{kaneton microkernel
educational project}. Additionally, two complete courses on kernel design and
\textit{Intel 32-bit Architecture} programming were prepared.

The project was composed of six steps, from the bootstrap, passing by
the kernel internals including memory management, task management etc.
to the servers with an \textit{IDE} device driver and finally a \textit{FAT}
file system.

Notice that the majority of the students did not success in implementing a
complete scheduling system allowing the creation of user-land task. Indeed,
the best groups achieved in providing the management of kernel-land tasks only.
Therefore, the \textit{IDE} driver, \textit{FAT} file system etc. were
running in the kernel.

Inspite of this, once again, the whole project was a success. However,
kaneton people noticed that the students took much time doing boring work
like filling in header files, dealing with versionning problems, writing
\textit{Make} files and \textit{Shell} scripts etc.

Moreover, the courses were too messy and the students had difficulties
to make the relation between the kaneton design and the microprocessor's
architecture implementation.

As a result, kaneton people decided to start implementing a kaneton microkernel
reference by their own in the \textit{C} language. This implementation will
then be used to compare the behaviour of students' implementation with
the reference. Moreover, this implementation led to the creation of a
new project: the \textbf{kaneton microkernel research project}.

%
% 2006
%

\section{2006}

While \textit{Jean-Pascal Billaud} leaved the project, people joined it
starting with \textit{EPITA} last year students \textit{C\'edric Aubouy},
\textit{Renaud Lienhart} but also \textit{Fabien Le-Mentec} from
\textit{EPITECH} who knew these people from the \textit{EPITA Computer Systems
Laboratory} where they were all working together a year before.
\textit{C\'edric Aubouy} and \textit{Renaud Lienhard} were in charge of the
kernel and \textit{Intel 32-bit Architecture} courses, respectively.
\textit{Julien Quintard} was still in charge of the kaneton educational
project given to the students.

More over two \textit{EPITA} first year students joined the \textit{EPITA
Computer Systems Laboratory}, \textit{Matthieu Bucchianeri} and \textit{Renaud
Voltz}. Indeed, from this date, the \textit{EPITA Computer Systems Laboratory}
was a strong partner of the kaneton microkernel project. These two students
were hired for contributing to the development of the kaneton research project.
Moreover, these students were supposed to teach and supervise the kaneton
educational project the following year.

\textit{Matthieu Bucchianeri} and \textit{Renaud Voltz} did an amazing work
on the kaneton research project implementation. Indeed, most of the
code related to the \textit{Intel 32-bit Architecture} comes from them. In
addition, the test suite as well as many tests were written by them. Thanks
to their work.

This year, kaneton people decided to introduce a development environment,
based on the kaneton research reference implementation, including everything
necessary to set up a collaborative kernel development.

While, previously, the students had to write the entire microkernel and
servers from scratch, this year, students only had to write precise parts
of the microkernel including some set implementations, memory management,
task scheduling etc.

Few mistakes were made especially about the choice of parts the students
had to implement. Indeed, asking the students to implement set implementations
like linked-list, array etc. was a very bad idea. This year, the project
was not completed and students stopped the project before the messaging
system implementation.

A course was also added to the \textit{EPITA} \textit{System, Network
and Security} major's curriculum about microprocessors' internals. This
course was introduced and taught by \textit{Julien Quintard}.

In conclusion, the kaneton educational project was not a real success this year
and needed some modifications. For instance, the course about the \textit{Intel
32-bit Architecture} was too specific and hard to understand but also hard
to teach. Instead, kaneton people decided to introduce a more general course
about kernel principles for the next year.

The kaneton research project implementation, in 2006,
  counted\footnote{Estimations realised with the software \textit{sloccount}.}
about \textit{7,000} lines for the \textit{core} and about \textit{2,000}
lines for the \textit{Intel 32-bit Architecture} implementation.

%
% 2007
%

\section{2007}

People affiliated with the \textit{EPITA Computer Systems Laboratory} joined
the project: \textit{Pierre Duteil} and \textit{Julian Pidancet}. Moreover,
students who implemented the kaneton educational project the previous year
decided to join the project: \textit{Enguerrand Raymond} and \textit{Mathieu
S\'elari\`es}, mainly working on the \textit{MIPS Architecture} portage among
other contributions.

This year, \textit{Matthieu Bucchianeri} and \textit{Renaud Voltz} were in
charge of the educational project by giving the kaneton courses as well
as supervising students' educational implementations.

As the kaneton research implementation was much more advanced as in 2006,
the students were given more code and then focused only on interesting and
system-related parts.

Students totally implemented the physical and virtual memory management, the
event and timers, the thread manager, the scheduler and the messaging system.
As for the previous years, the implementation was based on the \textit{Intel
32-bit Architecture}.

This year, the whole kaneton educational project was also given to students
from the \textit{Realtime \& Embedded Systems} specialization, for a total
of about $50$ students. The project was evaluated using a test suite,
developed for the kaneton research project, of about a hundred tests.

This year, the kaneton educational project was an amazing success as many
students completed a working microkernel, able to run tasks and to implement
some servers running on top of the kaneton microkernel.

The kaneton research implementation has grown to \textit{9,000} lines of source
code for the \textit{core} and \textit{5,500} lines for the microprocessor's
architecture implementation on \textit{Intel 32-bit}. The kaneton research
implementation was able to start modules - as standalone binaries - in
user-land as well as to make them communicate through the kaneton messaging
system.

This year, \textit{Pierre Duteil} leaved the project.
