%
% ---------- header -----------------------------------------------------------
%
% project       kaneton
%
% license       kaneton
%
% file          /home/mycure/kaneton/view/book/development/introduction.tex
%
% created       julien quintard   [thu may 17 12:46:30 2007]
% updated       julien quintard   [sat oct 20 14:17:20 2007]
%

%
% ---------- introduction -----------------------------------------------------
%

\chapter{Introduction}
\label{chapter:environment}

In this chapter, the kaneton microkernel project is briefly introduced
in order to emphasize some of its characteristics that makes contribution
specific in several ways.

\newpage

%
% ---------- text -------------------------------------------------------------
%

kaneton is a educational purpose microkernel project. This project aims
at providing a very clear, commented and maintainable microkernel source
code in order to allow people interested in operating systems internals
to look at the source code and understand it very quickly.

The kaneton project is basically composed of the source code of the
microkernel itself, scripts to perform complex tasks and various documents
from design papers to lecture materials.

The most important thing to remember is that the whole project is intended
to be understood as well as possibly maintained by everyone. As a result,
contributions must comply with the level of clarity expected by the project.

These rules are discussed in this paper in order to inform every new
contributor of what makes a good contribution.

The remaining of this document is organised as follows.
\textit{Chapter \ref{chapter:source tree}} introduced the kaneton project
organisation through the source code hierarchy. Next, \textit{Chapter
\ref{chapter:community}} describes how a contributor should behave in a
development community. \textit{Chapter \ref{chapter:rules}} introduces the
general rules which apply to any context around the kaneton project. Then,
the tools inherent to the kaneton project are listed in \textit{Chapter
\ref{chapter:tools}} with some guidelines about how to use them properly.
\textit{Chapter \ref{chapter:languages}} explicitly describes languages rules
informing the developer of the coding style to respect. \textit{Chapter
\ref{chapter:people}} draws a list of the people in charge for the different
parts and tools of the project. Finally, \textit{Chapter
\ref{chapter:licenses}} contains information about the licenses related to
the kaneton microkernel project.
