%
% ---------- header -----------------------------------------------------------
%
% project       kaneton
%
% license       kaneton
%
% file          /home/mycure/kaneton/view/book/development/introduction.tex
%
% created       julien quintard   [thu may 17 12:46:30 2007]
% updated       julien quintard   [sun may 20 18:09:08 2007]
%

%
% ---------- introduction -----------------------------------------------------
%

\chapter{Introduction}

In this chapter, the kaneton microkernel project is briefly introduced
in order to emphasize some of its characteristics that makes contribution
specific in several ways.

\newpage

%
% ---------- text -------------------------------------------------------------
%

kaneton is a educational purpose microkernel project. This project aims
at providing a very clear, commented and maintainable microkernel source
code in order to allow people interested in operating systems internals
to look at the source code and understand it very quickly.

The kaneton project is basically composed of the source code of the
microkernel itself, scripts to perform complex tasks and various documents
from design papers to lecture materials.

The most important thing to remember is that the whole project is intended
to be understood by everyone as well as to be maintained. As a result,
contributions must be made in a burdensome way by following rules.

These rules are discussed in this paper in order to inform every new
contributor of what makes a good contribution.

The remaining of this document is organised as follows. The first chapters
introduced the kaneton project organisation through the source code hierarchy
and the development environment. The following chapter describes how should a
contributor behave in a development community. Then, the tools inherent to
the kaneton project are listed with some guidelines about how to use them
properly. Languages rules are then explicitly described informing the
developer of the coding style to respect.
