%
% ---------- header -----------------------------------------------------------
%
% project       kaneton
%
% license       kaneton
%
% file          /home/mycure/kaneton/view/book/development/community.tex
%
% created       julien quintard   [sun may 20 18:08:17 2007]
% updated       julien quintard   [sun may 20 23:06:24 2007]
%

%
% ---------- community --------------------------------------------------------
%

\chapter{Community}

This chapter discusses what is a community and how contributors must integrate
an existing community.

\newpage

%
% ---------- text -------------------------------------------------------------
%

kaneton can obviously considered as an open source community although the
produced soure code is actually not open source, for now..

Driving an open source community is not obvious since people have different
personal goals at working on a free project. Some people contribute for
the knowledge, other for building the next generation system, other to
provide free open source softwares, other to become famous etc..

kaneton is a community driven microkernel that acts with the best interest of
the students at heart. Rules and regulations that keep us all moving forward
are fundamental even if the size of the kaneton community is relatively small
for now.

Indeed, the main objective of the kaneton project remains to be as
understandable as possible in order to lead students to implement parts of
it very quickly.

The remaining of this chapter draws a list of rules contributors must agree
to respect.

% objective

\subsubsection{Objective}

kaneton aims at providing a powerful, understandable and maintainable
microkernel. This objective must be kept in mind of every contributor
since many design and implementation were/are/will be made according to
this precise objective.

Note that the kaneton microkernel does not intend to be a desktop operating
system nor an as optimised as Linux operating system. Every contributor should
be well-aware of that in order to avoid behaviours stating that a feature
is fundamental or useless for performance reasons, for instance.

This rule does not prohibit people to suggest ideas but instead regulates
behaviours of people who wants to change major design and/or implementation
choices for bad reasons.

% behaviour

\subsubsection{Behaviour}

Open source projects does not mean constraint-free project. The kaneton
people, whilst being relatively young, try to act for the project's good
by behaving remarkably in the kaneton community.

Then, contributors are asked to do the same by avoiding some bad/young
behaviours.

\begin{enumerate}
  \item
    Follow the rules. People who do not these rules could be evinced from
    the kaneton project.
  \item
    Avoid the \textit{cowboy} behaviour consisting for a contributor to
    implement some feature without discussing about its usefulness with the
    community first. Another effect of this behaviour can be to distract
    the contributors from the major focuses.
  \item
    Always act and think in the project interest rather than your personal
    interest.
  \item
    Respect the other kaneton people, especially the ones who have worked
    on this project for a long time and that make this whole project possible.
    When people disagree, they are asked to do it respectfully.
  \item
    Take your responsability when you realise that you did something wrong:
    insults, mistake in an implemented feature.
  \item
    \textit{``The Perfect is the Enemy of the Good!''}: even nice
    contributors can unintentionally do bad things by being perfectionists
    and/or to much into the project and/or obsessed with process.
\end{enumerate}

% communication

\subsubsection{Communication}

politness, respect, trust and humility are the four qualities to have.

mailing-list netiquette to respect

if you're new, then first read the kaneton documents and then go through
the archive before asking anything which was already discussed, unless you
know exactly what you are talking about. otherwise it is a matter of disrespect
for the other developers.

do not respond to every message in every discussion, this is ridiculous.
read the thread, think about your response and then write a clear message
stating your point of view. this behaviour avoids hot responses.

depending on your status, please read your emails frequently because people
can rely on your decision, advice etc..

communiquez regulierement opur informez les gens de ce que vous avez
faire ou de ce que vous avez fait. tous les gens ne sont pas au courant
de tout, il faut donc les avertir de votre etat d'avancement. ceci est
different des mails de commit qui recense generalement de petites taches.
les mails d'update concerne une tache plus importante mais peuvent aussi
faire etat du fait que pendant les 2 mois a venir vous n'allez pas pouvoir
travailler.

--
communication tres importante entre les membres mais aussi avec l'exterieur.

le mieux est l'ennemi du bien

participation aux taches

pas de hierarchie mais tout de meme du respect envers les gens qui sont
la depuis plus longtemps.

documenter ce que vous faites.

%
% ---------- XXX --------------------------------------------------------------
%

regles communautaire.

-- comportement

mieux plutot que meilleur

savoir participer en faisant des trucs moins cool que d'autres car ca
doit etre fait

...

le projet dispose de certains outils pour effectuer le minimum vital:
outils de comm interne, dev communautaire, outils de comm ext, gestion
de taches.


XXX

eviter qu'une seule personne ne soit maitre d'un gros bout de code sans
quoi si elle part, le projet tombe a l'eau. il faut donc que tlm s'interesse
aux autres parties du code en demandant si il faut des infos autres autres
developpeurs sur cette partie.

il faut poser des questions mais attention a lire d'abord les archives
et la doc et eviter de poster tout le temps car ca drain l'energy des devs.

chaque discussion face-a-face ou au tel bref pas sur la ml doit etre reportee
sur la ml comme un report de facon que tlm soit au courant et prenne en
compte ces nouveaux elements.
