%
% ---------- header -----------------------------------------------------------
%
% project       kaneton
%
% license       kaneton
%
% file          /home/mycure/kaneton/view/book/development/community.tex
%
% created       julien quintard   [sun may 20 18:08:17 2007]
% updated       julien quintard   [fri may 25 18:13:44 2007]
%

%
% ---------- community --------------------------------------------------------
%

\chapter{Community}

This chapter discusses what is a community and how contributors must integrate
the kaneton community.

\newpage

%
% ---------- text -------------------------------------------------------------
%

kaneton can obviously be considered as an open source community althoughthe
produced soure code is actually not open source, for now..

Driving an open source community is complicated since people have different
personal goals at working on a free project. Some people contribute for
the knowledge, other for building the next generation system, other to
provide free open source softwares, other to become famous etc.

kaneton is a community driven microkernel that acts with the best interest of
the students at heart. Rules and regulations that keep the project moving
forward are fundamental even if the size of the kaneton community is
relatively small for now.

Indeed, the main objective of the kaneton project remains to be as
understandable as possible in order to lead students to implement parts of
it very quickly.

The remaining of this chapter draws a list of rules contributors must agree
to respect.

% objective

\subsubsection{Objective}

kaneton aims at providing a powerful, understandable and maintainable
microkernel. This objective must be kept in mind of every contributor
since many design and implementation were/are/will be made according to
this precise objective.

Note that the kaneton microkernel does not intend to be a desktop operating
system nor an as optimised as Linux operating system. Every contributor should
be well-aware of that in order to avoid behaviours stating that a feature
is fundamental or useless for performance concerns, for instance.

This rule does not prohibit people to suggest ideas but instead regulates
behaviours of people who wants to change major design and/or implementation
choices for bad reasons.

% behaviour

\subsubsection{Behaviour}

Open source projects does not mean constraint-free projects. The kaneton
people, whilst being relatively young, try to act for the project's good
by behaving remarkably in the kaneton community.

Therefore, contributors are asked to do the same by avoiding some bad/young
behaviours.

\begin{enumerate}
  \item
    Follow the rules. People who do not respect these rules could be evinced
    from the kaneton project.
  \item
    Avoid the \textit{cowboy} behaviour consisting for a contributor to
    implement some feature without discussing about its usefulness with the
    community first. Another effect of this behaviour can be to distract
    the contributors from the major focuses.
  \item
    Always act and think in the project interest rather than your personal
    interest.
  \item
    Respect the other kaneton people, especially the ones who have worked
    on this project for a long time and who made this whole project possible.
    When people disagree, they are asked to do it respectfully.
  \item
    Take your responsability when you realise that you did something wrong:
    insults, mistake in an implemented feature etc.
  \item
    \textit{``The Perfect is the Enemy of the Good!''}: even nice
    contributors can unintentionally do bad things by being perfectionists
    and/or to much into the project and/or obsessed with process.
  \item
    ... \textit{Politness}, \textit{Respect}, \textit{Trust} and
    \textit{Humility} are the key qualities that make a good contributor in
    any community.
\end{enumerate}

% communication

\subsubsection{Communication}

The communication mainly takes two forms in the kaneton microkernel project:
the \textit{mailing-list} for internal communication and the \textit{kaneton
public website} for external communication. The \textit{Developers Wiki}
is another source of communication as well as the commit logs etc.

The rules related to these tools are described in their associated section
and will therefore not be discussed here.

Every contributor must take the time to communicate as well in the mailing-list
as through the public website. Indeed, kaneton people must, frequently,
briefly describe what they are working on in order to inform the other
contributors who are not aware of everyone's current work. Note that these
kind of messages are very different from message generated by repository
commits. Indeed, while these commit messages indicate a modification, they
do not describe the whole work behind them.

Additionally, contributors can communicate informing the kaneton community of
their unavailability for the next two months, for instance. This behaviour
allows people to be aware some tasks will ne be done because some contributors
cannot work at this moment.

Althoug people are highly welcome to communicate, some rules apply in order
to avoid further problems.

First, any new contributor should obviously read the kaneton documents and
then go through the archives before asking anything which was already
discussed, unless the developer knows exactly what he is talking about.
Indeed, asking many questions about the source code is a form of disrespect
to the other contribors time. Moreover, many things can be found out just
looking at the kaneton documents and/or source code.

Althoug, people are asked to communicate, people are also asked to act
respectully. Contributors should not respond to every message in every
discussion, this is a ridiculous behaviour. Instead, every developer should
carefully read the discussion, think about his respons and then write a clear
message stating his point of view, ideas etc.

Depending on the contributor status, frequenlty reading the mailing-list is
absolutely fundamental as some people rely on other contributors decisions,
advises etc.

Finally, the mailing-list must be considered as the official communication
medium. If a contributor has a conversation either directly with a colleague
or on \textit{IRC} for instance, the discussion must be reported on the
mailing-list so that everyone takes into account these new ideas etc.

% work

\subsubsection{Work}

Working on the kaneton microkernel project does not imply low-level programming
all the time. Indeed, the kaneton project is composed of two parts: the
kaneton microkernel reference and the educational project.

Although the microkernel reference requires highly skilled programmers,
it also needs some tools for performing important tasks as diverse as
generating the prototypes, testing the core behaviour, generating the
documentation but also writing the documentation etc.

The educational project essentially needs documentation, lecture materials
and tools for managing the project: testing the students' implementation,
checking if some students cheated and many others.

This means that the kaneton people must contribute to every type of tasks
that need to be done. Also, the contributors are asked to well document
any work they have done include source code comments but also in the
kaneton official documents which are then available on the website.
