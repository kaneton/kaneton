%
% ---------- header -----------------------------------------------------------
%
% project       kaneton
%
% license       kaneton
%
% file          /home/mycure/kaneton/view/curriculum/EPITA/epita.tex
%
% created       julien quintard   [sat oct 20 09:31:28 2007]
% updated       julien quintard   [thu may 22 23:14:10 2008]
%

%
% ---------- setup ------------------------------------------------------------
%

%
% path
%

\def\path{../..}

%
% template
%

%%
%% copyright     (c) julien quintard
%%
%% project       kaneton
%%
%% file          /home/mycure/kaneton/view/templates/paper.tex
%%
%% created       julien quintard   [sat nov 19 18:11:23 2005]
%% updated       julien quintard   [tue dec 13 01:15:46 2005]
%%

%
% class
%

\documentclass[10pt,a4wide]{article}

%
% packages
%

\usepackage[english]{babel}
\usepackage{a4wide}
\usepackage{fancyheadings}
\usepackage{multicol}
\usepackage{indentfirst}
\usepackage{graphicx}
\usepackage{color}
\usepackage{xcolor}
\usepackage{verbatim}

\pagestyle{fancy}

\setlength{\footrulewidth}{0.3pt}
\setlength{\parindent}{0.3cm}
\setlength{\parskip}{2ex plus 0.5ex minus 0.2ex}

%
% verbatim font
%

\definecolor{verbatimcolor}{rgb}{0,0.4,0}

\makeatletter
\renewcommand{\verbatim@font}
  {\ttfamily\footnotesize\color{verbatimcolor}\selectfont}
\makeatother

%
% header
%

\rfoot{\scriptsize{The kaneton microkernel project}}

\date{\scriptsize{\today}}


%
% header
%

\lhead{\scriptsize{The opakk educational project :: EPITA}}

%
% title
%

\title{The opakk educational project :: EPITA
       \version}

%
% document
%

\begin{document}

%
% title
%

\maketitle

%
% --------- text --------------------------------------------------------------
%

\begin{multicols}{2}

%
% table of contents
%

\tableofcontents

%
% indentation
%

\indentation{}

%
% ---------- introduction -----------------------------------------------------
%

\section{Introduction}

The \term{opaak} educational project aims at providing material for teaching
and self-teaching of systems concepts ranging from low-level programming,
to kernel internals, to operating systems principles, to distributed systems
paradigms.

\name{opaak} is composed of three projects:

\begin{enumerate}
  \item
    \term{kastor} - \location{http://kastor.opaak.org};
  \item
    \term{kaneton} - \location{http://kaneton.opaak.org}; and
  \item
    \term{kayou} - \location{http://kayou.opaak.org}.
\end{enumerate}

Although these projects remain educational, \name{kaneton} and \textit{kayou}
also benefit from an aside \term{research} project consisting in designing and
implementing a novel system.

Thus, people in charge of the teaching at \name{EPITA} also contribute to
the development of the educational and research projects. Additionally, past
teachers keep contributing to the project either by developing or supervising
the new comers whilst some past students join the project.

%
% kastor
%

\subsection{kastor}

\name{kastor} is an introductory course targeting low-level programming. The
project lasts several weeks and allows students to understand what is the
microprocessor's role in an operating system by developing an emulator for
arcade games such as \name{Pong}, \name{Arkanoid} \etc{}.

Additionally, this course allows students to better unterstand the link
existing between the user-land applications and the operating system's kernel.

%
% kaneton
%

\subsection{kaneton}

\name{kaneton} is probably the most important project of the trilogy. This
project, taught to second-year engineering students, lasts for about five
months. This project focuses on making students fully understand
internals of a micro-kernel-based operating system.

%
% kayou
%

\subsection{kayou}

\name{kayou} is a distributed operating system built over the \name{kaneton}
micro-kernel. The \name{kayou}'s originality resides in its fully distributed
architecture. In such an architecture, all the machines of the network share
resources with each others. Shared resources include memory, processor,
storage, devices and so forth.

\name{kayou} is not taught at \name{EPITA}, yet. However, it would be a very
good material for distributed systems courses.

%
% ---------- history ----------------------------------------------------------
%

\section{History}

During the \name{kaneton} history, the project evolved and courses were added
to the curriculum, making the whole educational project more interesting and
understandable to the students. Moreover, the educational project, which
was already targeting the \name{EPITA}'s \name{System, Network and
Security} major, was also used in other contexts, victim of its success
and of the very hard work achieved by \name{kaneton} people over the years.

%
% 2004
%

\subsection{2004}

The first year, an optional low-level programming introductory course named
\term{k} was proposed to the \name{EPITA} Engineering School's first-year
students.

About fourteen hours courses were taught introducing the \name{Intel 32-bit}
microprocessor's external architecture and low-level programming.

The students had to develop small, poor and messy device drivers for the
console and keyboard peripherals.

The course was a bit chaotic but this first shot was a success. Therefore, the
students majoring in \name{System, Network and Security} asked the two
students \name{Julien Quintard} and \name{Jean-Pascal Billaud} for a
complete kernel project within their curriculum so that they can learn more
about operating systems internals.

%
% 2005
%

\subsection{2005}

The two, still students, \name{Julien Quintard} and \name{Jean-Pascal
Billaud} then started to work on a complete microkernel design the students
will have to implement. This was the premises of the \name{kaneton
microkernel \term{educational} project}. Additionally, two complete courses on
kernels design and \name{Intel 32-bit Architecture} programming were prepared.

The project was composed of six steps, from the bootstrap, through the
kernel internals including memory management, task management \etc{}
to the servers with an \name{IDE} device driver and finally a \name{FAT}
file system.

The large majority of the students did not succeed in implementing a
complete scheduling system allowing the creation of user-land tasks. Indeed,
the best groups achieved in providing the management of kernel-land tasks only.
Therefore, the \name{IDE} driver, \name{FAT} file system \etc{} were
running inside the kernel, leading more to a monolithic kernel model.

Inspite of this, once again, the whole project was a success. However,
\name{kaneton} people noticed that the students took too much time doing
unnecessary time-demanding work like filling header files, dealing with
versionning problems, writing \name{Make} files and \name{Shell} scripts \etc{}

Moreover, the courses were too messy and the students had difficulties
to make the relation between the \name{kaneton} design and the microprocessor's
architecture implementation.

As a result, \name{kaneton} people decided to start implementing a
\name{kaneton} microkernel reference on their own, written in \name{C}. This
implementation will then be used to compare the behaviour of students'
implementation with the reference. Moreover, this implementation led to the
creation of a new project: the \name{kaneton microkernel \term{research}
project}.

%
% 2006
%

\subsection{2006}

While \name{Jean-Pascal Billaud} leaved the project at the end of his
\name{EPITA} curriculum, people joined it starting with \name{EPITA}
last-year students \name{C\'edric Aubouy}, \name{Renaud Lienhart}
but also \name{Fabien Le-Mentec} from \name{EPITECH} who knew all
these people from the \name{EPITA Computer Systems Laboratory}
where they were all working together a year before. \name{C\'edric Aubouy}
and \name{Renaud Lienhard} were in charge of the kernel and
\name{Intel 32-bit Architecture} courses, respectively. \name{Julien Quintard}
was still in charge of the \name{kaneton} educational project taught to the
students.

Moreover two \name{EPITA} first-year students joined the \name{EPITA
Computer Systems Laboratory}, \name{Matthieu Bucchianeri} and \name{Renaud
Voltz}. Indeed, from this date, the \name{EPITA Computer Systems Laboratory}
was a strong partner of the \name{kaneton} microkernel project and therefore
of the \name{opaak} educational project as well. These two students
were hired for contributing to the development of the \name{kaneton} research
project. Moreover, these students were supposed to teach and supervise the
\name{kaneton} educational project a year after.

\name{Matthieu Bucchianeri} and \name{Renaud Voltz} did an amazing job
on the \name{kaneton} research project implementation. Indeed, most of the
code related to the \name{Intel 32-bit Architecture} comes from them. In
addition, the test suite as well as many tests were written by them.

This year, \name{kaneton} people decided to introduce a development
environment, based on the \name{kaneton} research reference implementation,
including everything necessary to set up a collaborative kernel development.

Whilst, previously, students had to write the entire microkernel and
servers from scratch, this year, students only had to write precise parts
of the microkernel including some set implementations, memory management,
task scheduling \etc{}

A few mistakes were made especially about the choice of parts the students
had to implement. Indeed, asking the students to implement set implementations
like linked-list, array \etc{}. was a very big mistake. This year, the project
was not completed and students stopped the project before the messaging
system implementation.

A course was also added to the \name{EPITA} \name{System, Network
and Security} major's curriculum about microprocessors' internals. This
course was introduced and taught by \name{Julien Quintard}.

In conclusion, the \name{kaneton} educational project was not a real success
this year and needed some modifications. For instance, the course about the
\name{Intel 32-bit Architecture} was too specific and hard to understand but
also hard to teach. Instead, \name{kaneton} people decided to replace it with
a more general course about kernel principles.

The \name{kaneton} research project implementation, in $2006$,
  counted\footnote{Estimations realised with the software \name{sloccount}.}
about $7,000$ lines for the \name{core} and about $2,000$ lines for the
\name{Intel 32-bit Architecture} implementation.

%
% 2007
%

\subsection{2007}

People affiliated with the \name{EPITA Computer Systems Laboratory} joined
the project: \name{Pierre Duteil} and \name{Julian Pidancet}. Moreover,
students who implemented the \name{kaneton} educational project the previous
year decided to join the project: \name{Enguerrand Raymond} and \name{Mathieu
S\'elari\`es}, mainly working on the \name{MIPS Architecture} portage amongst
other contributions.

This year, \name{Matthieu Bucchianeri} and \name{Renaud Voltz} were in
charge of the educational project by giving the kernel course as well
as supervising students' educational implementations.

As the \name{kaneton} research implementation was much more advanced as in
$2006$, the students were given more code and then focused only on interesting
and system-related parts.

Students totally implemented the physical and virtual memory management, the
event and timer managers, the thread manager, the scheduler and the messaging
system. As for the previous years, the implementation was based on the
\name{Intel 32-bit Architecture}.

This year, the whole \name{kaneton} educational project was also given to
students from the \name{Realtime \& Embedded Systems} specialisation, for a
total of about $50$ students. The project was evaluated using a test suite,
developed for the \name{kaneton} research project, of about a hundred tests.

This year, the \name{kaneton} educational project was an amazing success as
many students completed a working microkernel, able to run tasks and to
implement servers running on top of the \name{kaneton} microkernel.

The \name{kaneton} research implementation has grown to $9,000$ lines of source
code for the \name{core} and $5,500$ lines for the microprocessor's
architecture implementation on \name{Intel 32-bit}. The \name{kaneton} research
implementation was able to start modules - as standalone binaries - in
user-land as well as to make them communicate through the \name{kaneton}
messaging system.

This year, \name{Pierre Duteil} leaved the project prematurely. Since the
educational project needed two students for teaching \name{kaneton} to the
second-year students, \name{Elie Bleton} joined the project.

This year was not as good as expected for the \name{kaneton} research project.
As a result, \name{kaneton} people decided to re-organise the curriculum of
students working on \name{kaneton} by introducing the \term{kaneton
``piscine''} so that they can learn everything related to \name{kaneton} in
a very short period of time to be then able to work on their project without
any lack of knowledge.

%
% 2008
%
\subsection{2008}

In $2008$, \name{Laurent Lec} and \name{Nicolas Grandemange} joined the team.
While \name{Laurent Lec} focused on porting \name{kaneton} on the
\name{MIPS 64-bit} architecture, \name{Nicolas Grandemange} moved to the
\name{kayou} project by working on a distributed file
system named \name{Infinit}\footnote{\location{http://i.nfin.it}}.

[XXX]

%
% ---------- schedule ---------------------------------------------------------
%

\section{Schedule}

Until $2007$, two students were recruited every year, in January, via the
\name{EPITA Computer Systems Laboratory}. These students were assigned a
project and had to work on it from February to the end of July.

The project assigned was, in the first years, directly related to the
development of the \name{kaneton} research project \ie{} the reference
implementation. Through these projects, the recruited students discovered
and learned the concepts related to \name{kaneton}: micro-kernel design,
\name{IA-32} microprocessor external architecture \etc{}

Unfortunately, this is no longer the case. Indeed, the development of the
\name{kaneton} reference reaches its end, the next steps being related to the
\name{kayou} distributed operating system. Thus, the new recruited students
do not have the opportunity to learn through their projects.

Since students cannot learn about the \name{kaneton} design and \name{IA-32}
programming, they will not be capable of teaching the \name{kaneton} courses
and project to the second-year students of \name{EPITA}.

As a result, these students have to be taught the fundamentals by more
experienced \name{kaneton} people. This reasoning led to the curriculum's
schedule described below.

The timetable below describes the teaching of the \name{opaak}
projects at \name{EPITA}. Assistants are named \term{Poussins} followed by the
year they have been recruited, \name{Poussins'11} or \name{Poussins$_{11}$} for
assistants recruited in $2011$. For convenience and clarity, the whole year
$\gamma$ of the \name{EPITA} curriculum is detailed next. As an example,
If $\gamma$ represents the year $2008$, then, obviously,
\name{Poussins$_{\gamma - 2}$} are assistants recruited in $2006$.

%
% january \gamma
%

\subsection{January $\gamma$}

At the beginning of January, the \name{EPITA Computer Systems Laboratory}
recruitment takes place. At the end of this period, a few students ---
ideally two or three --- have been accepted to work on the
\name{kaneton} and \name{kayou} projects: the \name{Poussins'$\gamma$}.

As explained above, these new comers need to be taught the fundamentals so
that they can work on their projects with no lack of knowledge but also so
that they can teach and supervise the \name{kaneton educational project}
in $\gamma + 1$. A one-week specific and very intensive period called
\name{``piscine''} --- swimming pool in English --- is used to teach
the \name{Poussins'$\gamma$} about everything related to \name{kaneton}.
Since these students have to acquire many notions in a very short period of
time, \name{EPITA} is asked to arrange a special one-week curriculum for
these students.

This \name{``piscine''} will be taugh by more experienced \name{kaneton}
people, especially the \name{Poussins$_{\gamma - 2}$} possibly assisted by the
\name{Poussins$_{\gamma - 1}$}.

At the end of this \name{``piscine''}, \name{Poussins} judged too weak
could be expelled. At the same time, projects are proposed to these students.
The \name{Poussins} choose the project they would like to work on for the next
couple of months. Every project, hence \name{Poussins}, will be assigned a
supervisor. Note that supervisors can be \name{EPITA Alumni} or people
involved in \name{kaneton} and/or \name{kayou} without any relation with
\name{EPITA}.

%
% february \gamma - june \gamma
%

\subsection{February $\gamma$ - June $\gamma$}

\name{Poussins'$\gamma$} work on their assigned project in close relation
with their supervisor.

Depending on the progress status of the \name{Poussins'$\gamma$} projects,
two students will be chosen for teaching and supervising the
\name{kastor} project to the \name{EPITA} first-year students.

If the \name{Poussins'$\gamma$} supervisors consider that no student is
ready to carry out this task, then the courses will be taught by the
\name{Poussins$_{\gamma - 1}$}, one-year more experienced.

%
% july \gamma
%

\subsection{July $\gamma$}

\name{Poussins'$\gamma$} are asked to write and submit a report on the
work achieved through the five previous months. Additionally, every student
will have to give a talk, hence presenting their work to other \name{EPITA}
students, researchers, teachers \etc{}

This talk will be used by the head of \name{EPITA Computer Systems
Laboratory} for assigning the student a mark which will be used by the
\name{EPITA Board of Studies}.

Once evaluated, two \name{Poussins'$\gamma$} are picked for carrying out,
in February, the teaching and supervising of the \name{kaneton educational
project} to the second-year students major in both \name{System, Network and
Security}; and \name{Embedded and Real-Time Systems}.

Note that past reports can be found in the \name{kaneton} repository in the
\location{view/internship/} directory.

%
% august \gamma - january \gamma + 1
%

\subsection{August $\gamma$ - January $\gamma + 1$}

Whilst interns, \name{Poussins'$\gamma$} are due to prepare the teaching
starting in February. This preparation includes the source code, tests,
lectures, subjects, documentation as well as any administrative task: budget,
timetable and so forth.

Besides, a seminar must take place around December so that the \name{EPITA
Computer Systems Laboratory} is presented to the new students. During this
presentation, the new \name{kaneton} project ideas should be highlighted in
order to recruit some new \name{Poussins}.

%
% february \gamma + 1 - june \gamma + 1
%

\subsection{February $\gamma + 1$ - June $\gamma + 1$}

The selected \name{Poussins'$\gamma$} teach and supervise the second-year
students through the \name{kaneton educational project}.

%
% july \gamma + 1 - january \gamma + 2
%

\subsection{July $\gamma + 1$ - January $\gamma + 2$}

The \name{Poussins'$\gamma$}, before leaving \name{EPITA} for their
last-year internship, can take some time to contribute to the \name{kaneton}
and/or \name{kayou} research projects.

%
% february \gamma + 2 - \infinity
%

\subsection{February $\gamma + 2$ - $\infty$}

The \name{Poussins'$\gamma$} are more than welcome to keep contributing
to any project of the \name{opaak} educational project or to one of the
related research projects.

%
% ---------- opaak ------------------------------------------------------------
%

\section{opaak}

%
% kastor
%

\subsection{kastor}

\name{kastor} is a short but complete introduction to low-level programming and
operating system basics.

Students - in groups of two - develop a tiny monolithic kernel
intended to run very simple applications in a mono-task environment.

While dealing with many fundamental aspects of operating systems like
device drivers and system calls, the project is funny and visual. The
programs to run are tiny 2D-arcade games with graphic rendering and
sound.

% project

\subsubsection{Project}

The project is divided into several parts:

\begin{itemize}
  \item
    \term{pm} manager that takes care of memory management;
  \item
    \term{event} and \term{syscall} managers handling external
    events and user-land to kernel-space transitions;
  \item
    \term{cons}, \term{vga} and \term{speaker} managers for
    multimedia operations;
  \item
    \term{kbd} and \term{mouse} managers to provide user inputs;
  \item
    \term{time} manager to measure time intervals;
  \item
    \term{fs} manager to manage the root filesystem.
\end{itemize}

The project lasts two months and is concluded by an evaluation of the
final projects with each groups.

% courses

\subsubsection{Courses}

The project comes with a single course.

% XXX[cette section devrait etre completee avec 1) les points abordes 2)
%     un detail de chaque session. voir MIA et KER ci dessous pour un modele]

\name{Name}: \term{KSTR}

\name{Duration}: $14$ hours divided into $6$ lectures

\begin{itemize}
  \item
    Introduction to memory management, especially the segmentation mechanism;
  \item
    Presentation of the \name{IA-32} calling convention;
  \item
    Event handling, including hardware interrupts;
  \item
    Privilege modes and system calls;
  \item
    Device driver implementation and filesystems;
  \item
    Binary formats and ELF loading.
\end{itemize}

%
% kaneton
%

\subsection{kaneton}

The \name{kaneton} microkernel project aims at leading students to understand
what really is an operating system and more precisely a kernel.

Every group, composed of two students, has to develop parts of the
\name{kaneton} microkernel.

To simplify the project, each group receives a development environment
containing everything necessary to start the project including a
working microkernel base, documentation, assignments \etc{}

% project

\subsubsection{Project}

The whole \name{kaneton} project is subdivided into stages:

\begin{itemize}
  \item
    \term{k0}: this project will be entirely developed during a
    special practical session of about 6 hours and under the teachers' close
    supervisation. This project consists in an introduction to the
    \name{IA-32} external architecture: basic memory management model,
    \name{IA-32} assembly language, development tools \etc{}; (1 day)
  \item
    \term{k1}: this project will lead the students to an higher abstraction
    level. The goal of the project is to develop: the segment manager to be
    able to manage physical memory; and the architecture-dependent part of the
    region manager to be able to control the virtual memory; (2 weeks)
  \item
    \term{k2}: this project introduces the interrupts management and
    the driver development. Indeed, the student will have to set up
    everything necessary to handle interrupts. Then, some drivers will
    be developed including the keyboard and timer drivers as well as a
    demonstration shell; (2 weeks)
  \item
    \term{k3}: the students have to provide everything necessary
    to handle execution contexts. Indeed, the student will have to
    develop the scheduler but also to implement the context switchs from
    the microprocessor point of view; (3 weeks)
  \item
    \term{k4}: this project introduces the communication into the
    microkernel. Indeed, the students will have to provide some
    communication primitives including sychronous and asynchronous
    functions as well as blocking and non-blocking functions. (2 weeks)
\end{itemize}

% courses

\subsubsection{Courses}

The project comes with two fundamental courses:

% _mips architecture_

\name{Name}: \term{MIA}

\name{Duration}: $30$ hours divided into $10$ lectures

This course is intended to give students a complete view of how a RISC
microprocessor works and what choices the designers made to build the
future modern microprocessors.

Below are listed the concepts studied in this course:

\begin{itemize}
  \item
    External architecture: the instructions set, the memory management model
    \etc{};
  \item
    Pipeline: its inherent problems and limitations. Advanced pipelines will
    also be studied;
  \item
    Compiler optimisations: direct assembly source code optimisations:
    rescheduling, software pipelining \etc{};
  \item
    Memory: bus used on some experimental processors and the different cache
    management techniques.
\end{itemize}

The list below details the different lectures for the microprocessors
architecture course:

\begin{enumerate}
  \item
    Course presentation, \name{MIPS} history, introduction to \name{MIPS}
    external architecture: registers and instructions set. Introduction to
    instruction formats and to inherent limitations; (3 hours)
  \item
    End of instruction formats introduction, explanations of architecture
    designers choices: \name{MIX}s and \name{Amdhal} Rule. Some exercises to
    practice. Introduction to \name{MIPS} architecture addressing and to
    \name{MIPS} pipeline; (3 hours)
  \item
    Introduction to \name{MIPS} spirit and to its pipeline. \name{Moore Law}
    and pipeline rules. Introduction to different pipeline representations:
    simplified and detailed. Some exercises to practice these representations;
    (3 hours)
  \item
    We will study in this lecture the next instruction address computation
    problem and the delay slot. Then, study of branch instruction and their
    problems and limitations. Some exercises to practice; (3 hours)
  \item
    Complete study of instruction dependencies. Some exercise to practice;
    (3 hours)
  \item
    Partial exam correction. Introduction to optimisations; (3 hours)
  \item
    Study of different optimisations used by compilers. Study of
    advanced pipelines: superpipeline, superscalar pipelines \etc{}; (3 hours)
  \item
    Introduction to the Pi-Bus. Then complete study of memory including
    cache management; (3 hours)
  \item
    End of memory course. Some exercises to practice; (3 hours)
  \item
    Revisions for the final exam. (3 hours)
\end{enumerate}

% _kernel_

\name{Name}: \term{KER}

\name{Duration}: $26$ hours divided into $9$ lectures

This course is intended to give the students a general overview of
kernel internals and its problematics.

Below are listed the studied notions:

\begin{itemize}
  \item
    Understand how kernel developement evolves and why, receive a
    minimal general knowledge regarding kernel internals and history;
  \item
    Resources --- memory, execution time --- sharing and management with
    increasing constraints and needs;
  \item
    The microprocessor's facilities are detailed to get the students
    to fully understand the mechanisms. Examples are taken from
    many different architectures;
  \item
    The common kernel design issues will be totally understood by the
    students and they will be able to build their own design;
  \item
    The course aims at giving the students a precise idea of how it
    works inside, thus giving them the ability to fully understand
    the \name{kaneton} project.
\end{itemize}

The list below details the different lectures for the kernel course.

\begin{enumerate}
  \item
    Recalls and prerequisites: presentation of the diffent kind of kernels
    and the role in an operating system. Recalls of low-level basics; (3 hours)
  \item
    Bootstrap: details about processor bootup and bootloading; (2 hours)
  \item
    Memory management - part one: fundamental concepts of memory management,
    overview of memory management unit and address translation though paging;
    (3 hours)
  \item
    Memory management - part two: high-level mechanisms like page swapping,
    copy-on-write, \etc{} Presentation of most known allocation algorithms;
    (3 hours)
  \item
    Events and I/O - part one: interrupt mechanism, interrupt vector and
    processor running modes; (3 hours)
  \item
    Events and I/O - part two: hardware interrupts, interrupt controllers,
    port-mapped input/output, memory-mapped input/output; (3 hours)
  \item
    Task and scheduling - part one: recalls on multiprogramming, execution
    context, context switching; (3 hours)
  \item
    Task and scheduling - part two: process control block on known operating
    systems, task life cycle, scheduling automaton, scheduling algorithms;
    (3 hours)
  \item
    Inter-Process Communication: \name{IPC} in a microkernel, message
    passing, classical \name{IPC}, locking. (3 hours)
\end{enumerate}

%
% kayou
%

\subsection{kayou}

\name{kayou} is not taught at \name{EPITA} yet.

%
% ---------- project suggestions ----------------------------------------------
%

\section{Project Suggestions}

This section contains some project suggestions for the new \name{Poussins}.

%
% hardware description
%

\subsection{Hardware Description}

XXX[hardware/layout description, boot, lib parsing, pre-reserved objects,
    mapping objects to tasks etc. so that everything works fine given a
    single hardware description file]

%
% kayou native library
%

\subsection{\name{kayou} Native Library}

XXX[native library in C++ or equivalent, preparing services development
    and move to the very first clean developments]

%
% security
%

\subsection{Security}

Security is a very important topic in operating systems, even more in
distributed systems. Indeed, in such a system, resources are completely
shared amongst the machines of the system. It is therefore crucial to protect
these resources from being accessed by unauthorised entities.

This project consists in designing and implementing a complete security
model composed of some in-kernel primitives as well as at the operating
system level.

For now, the \name{kaneton} microkernel provides the concept of \name{capabilities}
which seems to perfectly fit the distributed systems' requirements. The student
will have to confirm that this concept is suitable for \name{kayou}'s needs. The
student will also have to complete the current implementation in order to move
to the study of the security in the operating system, even its in distributed
form.

A possibly suitable access control scheme could be the \name{RBAC} -
Role-Based Access Control. The student will have, once again, to confirm this
possibility via a preliminary study and then implement the services
necessary to this scheme so that applications can use it.

The student will have to tackle the notion of multi-users at the operating
system level. Indeed, although this notion is integrated at the kernel
level in every \name{UNIX} systems, \name{kaneton} designers decided to handle this
notion outside the microkernel.

The student will then have to study this concept and implement the necessary
services. Besides, the student could also investigate further in order
to develop a security service based on the concept of user. \name{RBAC}
could, once again, turn out to be a good choice.

%
% posix
%

\subsection{POSIX}

\name{kaneton} provides its own interface which has nothing to do with interfaces
of others \name{UNIX} like \name{Linux}, \name{BSD} or others \etc{}.

\name{POSIX} is an interface which was introduced in order to normalise the
\name{C Library} layer. Most applications --- at least in the \name{UNIX}
world --- rely on this software layer.

If we want our operating system to benefit from the --- already existing and
future --- software without having to re-develop everything, we need to
develop a software layer complying to the \name{POSIX} standard. In our case,
the \name{POSIX} functionalities must be translated into either
functionalities provided either by \name{kaneton} or one or more \name{kayou} servers.

The student will have to study very precisely the \name{POSIX} standard
as well as the \name{kaneton} interface in order to develop the emulation library:
\name{libposix}.

In order to test the library, the student will be in charge of porting some
services and applications. Since \name{UNIX} highly relies on the concept
of \name{files}, it is very likely that the student will have to port
or develop a hard-disk driver as well as a basic file system.

The student could take advantage of the different work existing on the
Internet, especially \name{LSE/OS} which was designed and implemented
at the \name{EPITA Computer Systems Laboratory}.

%
% amd64
%

\subsection{AMD-64}

This project consists in porting \name{kaneton} to a new microprocessor architecture,
in this case, \name{AMD-64}.

This architecture is very interesting for two reasons. First, \name{AMD}
architectures have many commonalities with \name{IA-32} architectures. Since,
\name{kaneton} already supports \name{IA-32}, it would be easy to support this new
architecture. Secondly, this architecture runs in 64-bit and its implementation
could reveal design mistakes and/or implementation bugs in the generic
part of the microkernel, commonly called \name{core}.

The student will have to study a complete new architecture by reading the
microprocessor specifications. Besides, the student will have to report
any problem encountered in his project, especially about the microkernel's
design. For example, types handling in the \name{core} is very likely to
be problematic.

The student will have to modify several parts in the kernel but also
make suggestions about design improvements based on his observations.

Besides, the student will have to complete the microkernel implementation
in various ways: functionalities not yet implemented, bugs correction, error
handling improvement, \etc{}

%
% virtualization
%

\subsection{Virtualization}

This project consists in porting \name{kaneton} for a paravirtualizer, probably
\name{Xen}.

This project would be very useful for the \name{kaneton} microkernel project. Indeed,
once ported, the \name{kaneton} microkernel could be run on a real machine for
launching the test suite for instance without all the constraints we are
facing at this time. It would be much more easier to debug in a real, but
controlled, environment.

Additionally, the \name{kayou} distributed operating system needs to be tested in
a real environment as well, meaning with several computers running the \name{kayou}
operating system. It would be wise to use virtualization for running multiple
instances of the \name{kayou} operating system on a single machine rather than
setting up a dozen of real computers.

The student will have to study the different existing virtualization techniques
and especially the paravirtualization systems so that one can be picked
for portage.

\end{multicols}

\end{document}
