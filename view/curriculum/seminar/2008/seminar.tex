%
% ---------- header -----------------------------------------------------------
%
% project       kaneton
%
% license       kaneton
%
% file          /home/buckman/cry...view/curriculum/seminar/2008/seminar.tex
%
% created       matthieu bucchianeri   [fri nov  2 14:14:33 2007]
% updated       matthieu bucchianeri   [fri nov  2 15:27:29 2007]
%

%
% path
%

\def\path{../../..}

%
% template
%

%%
%% copyright     (c) julien quintard
%%
%% project       kaneton
%%
%% file          /home/mycure/kaneton/view/templates/lecture.tex
%%
%% created       julien quintard   [sat nov 19 17:13:03 2005]
%% updated       julien quintard   [fri dec  2 22:36:34 2005]
%%

%
% class
%

\documentclass[8pt]{beamer}

%
% packages
%

\usepackage{pgf,pgfarrows,pgfnodes,pgfautomata,pgfheaps,pgfshade}
\usepackage{colortbl}
\usepackage{times}
\usepackage{amsmath,amssymb}
\usepackage{graphics}
\usepackage{graphicx}
\usepackage{color}
\usepackage{xcolor}
\usepackage[english]{babel}
\usepackage{enumerate}
\usepackage[latin1]{inputenc}

%
% style
%

\usepackage{beamerthemesplit}
\setbeamercovered{dynamic}

%
% verbatim font
%

\definecolor{verbatimcolor}{rgb}{0,0.4,0}

\makeatletter
\renewcommand{\verbatim@font}
  {\ttfamily\footnotesize\color{verbatimcolor}\selectfont}
\makeatother

%
% new line
%

\newcommand{\nl}[0]{\vspace{0.4cm}}

%
% date
%

\date{\today}

%
% logos
%

\pgfdeclareimage[interpolate=true,width=34pt,height=18pt]
                {epita}{../../logos/epita}
\pgfdeclareimage[interpolate=true,width=49pt,height=18pt]
                {upmc}{../../logos/upmc}
\pgfdeclareimage[interpolate=true,width=25pt,height=18pt]
                {lse}{../../logos/lse}

\newcommand{\logos}
  {
    \pgfuseimage{epita}
  }

%
% institute
%

\institute
{
  \inst{1} kaneton microkernel project
}

%
% table of contents at the beginning of each section
%

\AtBeginSection[]
{
  \begin{frame}<beamer>
   \frametitle{Outline}
    \tableofcontents[current]
  \end{frame}
}

%
% table of contents at the beginning of each subsection
%

\AtBeginSubsection[]
{
  \begin{frame}<beamer>
   \frametitle{Outline}
    \tableofcontents[current,currentsubsection]
  \end{frame}
}


%
% title
%

\title{the kaneton microkernel project\\http://www.kaneton.org}

%
% authors
%

\author
{
  Matthieu~Bucchianeri,
  Renaud~Voltz
}

%
% figures
%

%\pgfdeclareimage[interpolate=true,width=200pt,height=140pt]
%                {set}
%		{figures/set}

\pgfdeclareimage[interpolate=true,width=200pt,height=160pt]
                {kaneton}
		{figures/kaneton}

\pgfdeclareimage[interpolate=true,width=200pt,height=100pt]
                {kernel-in-os}
		{figures/kernel-in-os}

\pgfdeclareimage[interpolate=true,width=200pt,height=132pt]
                {k-snapshot}
		{figures/k-snapshot}

\pgfdeclareimage[interpolate=true,width=200pt,height=100pt]
                {servers-in-os}
		{figures/servers-in-os}

%
% document
%

\begin{document}

%
% title frame
%

\begin{frame}
  \titlepage

  \begin{center}
    \logos
  \end{center}
\end{frame}

%
% outline frame
%

\begin{frame}
  \frametitle{Contents}
  \tableofcontents
\end{frame}

%
% overview
%

\section{Overview}

% -)

\begin{frame}
  \frametitle{Project overview}

  kaneton is a microkernel designed and developed for pedagogical
  purposes.

  \begin{center}
    \pgfuseimage{kernel-in-os}
  \end{center}

  However, kaneton is also meant to be the basis of a distributed
  operating system called kayou.

\end{frame}

% -)

\begin{frame}
  \frametitle{Lessons overview}

  kaneton is also a set of lessons in SRS and GISTR specializations:

  \begin{itemize}
  \item
    Kernels, Operating systems and low-level programnming;
  \item
    Microprocessors design and architecture
  \end{itemize}

  \-

  And 5 projects in both specializations:

  \begin{itemize}
  \item
    K0: bootstrap
  \item
    K1: memory management
  \item
    K2: interrupts and event handling
  \item
    K3: multitasking and multithreading
  \item
    K4: inter-process communication
  \end{itemize}

  \-

  The whole course is given to about 60 students every years.

\end{frame}

% -)

\begin{frame}
  \frametitle{Another closely related project: K}

  The ing1 project k is maintained by the kaneton people. This project
  is a quick introduction to low-level programming.

  \begin{center}
    \pgfuseimage{k-snapshot}
  \end{center}

  Students have to develop the kernel of an operating system dedicated
  to arcade games.

  \-

  Since three years, this optional project is getting more and more
  popular, with about 60 students last years.
\end{frame}

% -)

\begin{frame}
  \frametitle{kaneton people}

    \begin{itemize}
    \item
      Julien Quintard
      \begin{itemize}
	\item
	Creator of the project a few years ago with Jean Pascal Billaud
	\item
	Project manager and designer
	\item
	Responsible of kaneton course from 2004 to 2006
        \item
	Lecturer of Microprocessors architecture lessons
      \end{itemize}
    \item
      Renaud Voltz \& Matthieu Bucchianeri
      \begin{itemize}
	\item
	Working at LSE for two years now
	\item
	Core and IA-32 developers
	\item
	Responsibles of kaneton course in 2007
	\item
	Lecturers of Kernels and operating systems lessons in 2007
      \end{itemize}
    \item
      Julian Pidancet \& Elie Bleton
      \begin{itemize}
        \item
	Working at LSE in 2007
	\item
	Responsibles of kaneton course in 2008
      \end{itemize}
    \end{itemize}

  \textbf{Other people currently involved in kaneton}

  \begin{itemize}
    \item
      Mathieu S�lari�s: Assistant project manager
    \item
      Enguerrand Raymond: MIPS Developer
  \end{itemize}

  \textbf{Other people involved in the past}

  \begin{itemize}
    \item
      C�dric Aubouy: Kernel and operating systems lecturer for 2006
    \item
      Renaud Lienhart: IA-32 architecture lecturer for 2006
    \item
      Jean-Pascal Billaud: kaneton course for 2004 and 2005
  \end{itemize}

\end{frame}

%
% the kaneton microkernel
%

\section{The kaneton microkernel}

% -)

\begin{frame}
  \frametitle{void}
\end{frame}

%
% projects for 2008
%

\section{Projects for 2008}

% -)

\begin{frame}
  \frametitle{void}
\end{frame}

%
% your commitments
%

\section{Your commitments}

% -)

\begin{frame}
  \frametitle{void}
\end{frame}

\end{document}
