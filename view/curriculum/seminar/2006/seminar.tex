%%
%% licence       kaneton licence
%%
%% project       kaneton
%%
%% file          /home/buckman/kaneton/view/lectures/seminar_2006/seminar.tex
%%
%% created       julien quintard   [fri dec  2 07:53:59 2005]
%% updated       matthieu bucchianeri   [wed dec 13 18:06:05 2006]
%%

%
% template
%

%
% ---------- header -----------------------------------------------------------
%
% project       kaneton
%
% license       kaneton
%
% file          /home/mycure/kaneton/view/template/lecture.tex
%
% created       julien quintard   [wed may 16 18:17:26 2007]
% updated       julien quintard   [sun may 18 23:23:40 2008]
%

%
% class
%

\documentclass[8pt]{beamer}

%
% packages
%

\usepackage{pgf,pgfarrows,pgfnodes,pgfautomata,pgfheaps,pgfshade}
\usepackage[T1]{fontenc}
\usepackage{colortbl}
\usepackage{times}
\usepackage{amsmath,amssymb}
\usepackage{graphics}
\usepackage{graphicx}
\usepackage{color}
\usepackage{xcolor}
\usepackage[english]{babel}
\usepackage{enumerate}
\usepackage[latin1]{inputenc}
\usepackage{verbatim}
\usepackage{aeguill}

%
% style
%

\usepackage{beamerthemesplit}
\setbeamercovered{dynamic}

%
% verbatim stuff
%

\definecolor{verbatimcolor}{rgb}{0.00,0.40,0.00}

\makeatletter

\renewcommand{\verbatim@font}
  {\ttfamily\footnotesize\selectfont}

\def\verbatim@processline{
  {\color{verbatimcolor}\the\verbatim@line}\par
}

\makeatother

%
% -
%

\renewcommand{\-}{\vspace{0.4cm}}

%
% date
%

\date{\today}

%
% logos
%

\pgfdeclareimage[interpolate=true,width=34pt,height=18pt]
                {epita}{\path/logo/epita}
\pgfdeclareimage[interpolate=true,width=49pt,height=18pt]
                {upmc}{\path/logo/upmc}
\pgfdeclareimage[interpolate=true,width=25pt,height=18pt]
                {lse}{\path/logo/lse}

\newcommand{\logos}
  {
    \pgfuseimage{epita}
  }

%
% institute
%

\institute
{
  \inst{1} kaneton microkernel project
}

%
% table of contents at the beginning of each section
%

\AtBeginSection[]
{
  \begin{frame}<beamer>
   \frametitle{Outline}
    \tableofcontents[current]
  \end{frame}
}

%
% table of contents at the beginning of each subsection
%

\AtBeginSubsection[]
{
  \begin{frame}<beamer>
   \frametitle{Outline}
    \tableofcontents[current,currentsubsection]
  \end{frame}
}


%
% title
%

\title{Pr\'{e}sentation de kaneton --\\http://www.kaneton.org}

%
% authors
%

\author
{
  Renaud~Voltz,
  Matthieu~Bucchianeri,
  Julien~Quintard
}

%
% figures
%

\pgfdeclareimage[interpolate=true,width=200pt,height=140pt]
                {set}
		{figures/set}

\pgfdeclareimage[interpolate=true,width=300pt,height=140pt]
                {manager}
		{figures/manager}

\pgfdeclareimage[interpolate=true,width=130pt,height=160pt]
                {event}
		{figures/event}

% 10 frames for a microkernel example

\pgfdeclareimage[interpolate=true,width=290pt,height=165pt]
                {microkernel_example_01}
		{figures/microkernel_example_01}
\pgfdeclareimage[interpolate=true,width=290pt,height=165pt]
                {microkernel_example_02}
		{figures/microkernel_example_02}
\pgfdeclareimage[interpolate=true,width=290pt,height=165pt]
                {microkernel_example_03}
		{figures/microkernel_example_03}
\pgfdeclareimage[interpolate=true,width=290pt,height=165pt]
                {microkernel_example_04}
		{figures/microkernel_example_04}
\pgfdeclareimage[interpolate=true,width=290pt,height=165pt]
                {microkernel_example_05}
		{figures/microkernel_example_05}
\pgfdeclareimage[interpolate=true,width=290pt,height=165pt]
                {microkernel_example_06}
		{figures/microkernel_example_06}
\pgfdeclareimage[interpolate=true,width=290pt,height=165pt]
                {microkernel_example_07}
		{figures/microkernel_example_07}
\pgfdeclareimage[interpolate=true,width=290pt,height=165pt]
                {microkernel_example_08}
		{figures/microkernel_example_08}
\pgfdeclareimage[interpolate=true,width=290pt,height=165pt]
                {microkernel_example_09}
		{figures/microkernel_example_09}
\pgfdeclareimage[interpolate=true,width=290pt,height=165pt]
                {microkernel_example_10}
		{figures/microkernel_example_10}

% memory management
\pgfdeclareimage[interpolate=true,width=179pt,height=190pt]
                {vmem}
		{figures/vmem}
\pgfdeclareimage[interpolate=true,width=137pt,height=190pt]
                {vmem_overlap}
		{figures/vmem-overlap}
\pgfdeclareimage[interpolate=true,width=137pt,height=190pt]
                {vmem_contiguous}
		{figures/vmem-contiguous}
\pgfdeclareimage[interpolate=true,width=137pt,height=190pt]
                {vmem_sharing}
		{figures/vmem-sharing}

% process scheduling

\pgfdeclareimage[interpolate=true,width=150pt,height=190pt]
                {mfq1}
		{figures/sched1}
\pgfdeclareimage[interpolate=true,width=150pt,height=190pt]
                {mfq2}
		{figures/sched2}
\pgfdeclareimage[interpolate=true,width=150pt,height=190pt]
                {mfq3}
		{figures/sched3}
\pgfdeclareimage[interpolate=true,width=150pt,height=190pt]
                {mfq4}
		{figures/sched4}
\pgfdeclareimage[interpolate=true,width=150pt,height=190pt]
                {mfq5}
		{figures/sched5}
\pgfdeclareimage[interpolate=true,width=150pt,height=190pt]
                {mfq6}
		{figures/sched6}
\pgfdeclareimage[interpolate=true,width=150pt,height=190pt]
                {mfq7}
		{figures/sched7}
% task

\pgfdeclareimage[interpolate=true,width=210pt,height=65pt]
                {task}
		{figures/task}




\pgfdeclareimage[interpolate=true,width=200pt,height=140pt]
                {set}
		{figures/set}

\pgfdeclareimage[interpolate=true,width=300pt,height=140pt]
                {manager}
		{figures/manager}



%
% document
%

\begin{document}

%
% title frame
%

\begin{frame}
  \titlepage

  \begin{center}
    \logos
  \end{center}
\end{frame}

%
% outline frame
%

\begin{frame}
  \frametitle{Contenu}
  \tableofcontents
\end{frame}

%
% overview
%

\section{Pr\'{e}sentation}

% 1)

\begin{frame}
  \frametitle{Les gens qui travaillent sur kaneton}

  \textbf{Design}

  \begin{itemize}
    \item
      Julien Quintard
    \item
      Jean-Pascal Billaud
  \end{itemize}

  \nl

  \textbf{Implementation}

  \begin{itemize}
    \item
      Renaud Voltz
    \item
      Matthieu Bucchianeri
    \item
      Enguerrand Raymond
  \end{itemize}

  \nl

  \textbf{Autres gens ayant contribu\'{e}s significativement}

  \begin{itemize}
    \item
      C\'{e}dric Aubouy
  \end{itemize}
\end{frame}

% 2)

\begin{frame}
  \frametitle{Description}

  kaneton est un micro-noyau \`{a} but p\'{e}dagogique
  d\'{e}velopp\'{e} par le LSE.

  \-

  \begin{itemize}
  \item
    Un design compr\'{e}hensible
  \item
    Un code clair et comment\'{e}
  \item
    Facilement maintenable et portable
  \item
    Destin\'{e} au distribu\'{e}
  \end{itemize}

\end{frame}

% 3)

\begin{frame}
  \frametitle{Les cours (1)}

  kaneton est aussi un projet en sp\'{e}cialisation SRS et GISTR.\\

  Deux cours :

  \begin{itemize}
  \item
    Noyaux et syst\`{e}mes d'exploitation
  \item
    Architecture des microprocesseurs
  \end{itemize}

  \-

  Cinq projets en SRS :

  \begin{itemize}
  \item
    K0 : le bootstrap
  \item
    K1 : la m\'{e}moire
  \item
    K2 : les \'{e}v\'{e}nements
  \item
    K3 : les process et l'ordonnancement
  \item
    K4 : les IPC
  \end{itemize}

  \-

  Deux projets en GISTR :

  \begin{itemize}
  \item
    KG0 : les \'{e}v\'{e}nements et les I/O
  \item
    KG1 : les process et l'ordonnancement
  \end{itemize}

  \-

  Ces cours et projets concernent au total une quarantaine de
  personnes.

\end{frame}


% -)

\begin{frame}
  \frametitle{Les cours (2)}

  Le p\^{o}le kernel du LSE dispense aussi les cours d'introduction
  aux noyaux et programmation syst\`{e}me aux Ing1 (projet connu sous
  le nom ``K'').

  \-

  Environ 60 \'{e}tudiants de la promotion 2008 ont suivi cette option.

\end{frame}

% 4)

\begin{frame}
  \frametitle{La pr\'{e}sentation de ce soir}

  \begin{itemize}
  \item
    Introduit le design de kaneton
  \item
    Pr\'{e}sente ce qui a \'{e}t\'{e} cod\'{e}
  \item
    Liste de ce qu'il reste a faire
  \end{itemize}
\end{frame}

%
% microkernels
%

\section{Les micro-noyaux}

% 5)

\begin{frame}
  \frametitle{Les micro-noyaux}

  \begin{itemize}
  \item
    $\leftrightarrow$ noyaux monolithiques
  \item
    $\leftrightarrow$ exo-noyaux
  \item
    Conception modulaire en services
  \item
    Fonctionnement de ces services en \emph{userland}
  \item
    Communication inter-processus omnipr\'{e}sente (IPC)
  \end{itemize}

\end{frame}

% 6)

\begin{frame}
  \frametitle{Microkernel example}

  \begin{center}
    \pgfuseimage{microkernel_example_01}
  \end{center}
\end{frame}

% 7)

\begin{frame}
  \frametitle{Microkernel example}

  \begin{center}
    \pgfuseimage{microkernel_example_02}
  \end{center}
\end{frame}

% 8)

\begin{frame}
  \frametitle{Microkernel example}

  \begin{center}
    \pgfuseimage{microkernel_example_03}
  \end{center}
\end{frame}

% 9)

\begin{frame}
  \frametitle{Microkernel example}

  \begin{center}
    \pgfuseimage{microkernel_example_04}
  \end{center}
\end{frame}

% 10)

\begin{frame}
  \frametitle{Microkernel example}

  \begin{center}
    \pgfuseimage{microkernel_example_05}
  \end{center}
\end{frame}

% 11)

\begin{frame}
  \frametitle{Microkernel example}

  \begin{center}
    \pgfuseimage{microkernel_example_06}
  \end{center}
\end{frame}

% 12)

\begin{frame}
  \frametitle{Microkernel example}

  \begin{center}
    \pgfuseimage{microkernel_example_07}
  \end{center}
\end{frame}

% 13)

\begin{frame}
  \frametitle{Microkernel example}

  \begin{center}
    \pgfuseimage{microkernel_example_08}
  \end{center}
\end{frame}

% 14)

\begin{frame}
  \frametitle{Microkernel example}

  \begin{center}
    \pgfuseimage{microkernel_example_09}
  \end{center}
\end{frame}

% 15)

\begin{frame}
  \frametitle{Microkernel example}

  \begin{center}
    \pgfuseimage{microkernel_example_10}
  \end{center}
\end{frame}

% 16)

%
% kaneton design
%

\section{Le design de kaneton}

% 17)

\begin{frame}
  \frametitle{Fonctions remplies par kaneton}

  Le micro-noyau kaneton fourni les fonctionnalit\'{e}s suivants :

  \begin{itemize}
  \item
    Gestion de la m\'emoire
  \item
    Gestion des \'ev\'enements et des I/O
  \item
    Gestion des processus
  \item
    Communication inter-processus
  \end{itemize}

\end{frame}

\begin{frame}
  \frametitle{Le \emph{set} manager}

La gestion des structures de donne\'es d'un noyau g\'en\`ere beaucoup de code.

M\'elanger ce code avec celui de l'impl\'ementation des algorithmes eux-m\^emes
conduit \`a plusieurs inconv\'enients :

  \begin{itemize}
    \item un code lourd
    \item un code redondant
    \item un code complexe
  \end{itemize}

  \-

  Finalement, les algorithmes deviennent plus difficiles \`a comprendre et \`a
  programmer.

  \-

  kaneton consid\`ere qu'un algorithme n'a pas besoin de savoir comment est
  impl\'ement\'ee une file, tant qu'il la voit comme une file.

  \-

  kaneton fournit une interface consacr\'ee \`a la gestion des structures de
  donn\'ees du noyau: le {\bf set manager}.
\end{frame}

% -)

\begin{frame}
  \frametitle{Le \emph{set} manager}

  Le set manager fournit une interface tr\`es simple pour cr\'eer, stocker et
  organiser les structures-objet de kaneton. Voici les diff\'erents types de
  sets qu'elle impl\'emente d\'ej\`a :

  \-

  \begin{center}
    \pgfuseimage{set}
  \end{center}

\end{frame}

% -)

\begin{frame}
  \frametitle{Le \emph{set} manager}

  Avantages du set manager:

  \begin{itemize}
    \item
    minimise les risques d'erreur
    \item
    maximise le temps pass\'e dans l'impl\'ementation de l'algorithme
    lui-m\^eme
    \item
    offre la possibilit\'e de tester facilement un m\^eme algorithme avec
    plusieurs mod\`eles de structures de donn\'ees
  \end{itemize}

\end{frame}

% -)

\begin{frame}[containsverbatim]
  \frametitle{Le \emph{set} manager : un exemple}

  Voici un mor\c{c}eau de code de l'allocateur de m\'{e}moire physique :

  \begin{verbatim}
o_segment*      current;
t_iterator      i;

set_foreach(SET_OPT_FORWARD, segment->segments, &i, state)
  {
    o_segment   *next;
    t_iterator  j;

    set_object(segment->segments, i, (void**)&current);

    if (set_next(segment->segments, i, &j) != ERROR_NONE)
      break;

    set_object(segment->segments, j, (void**)&next);

    if ((next->address - (current->address + current->size)) >= size)
      {
        *address = current->address + current->size;

         return (ERROR_NONE);
      }
  }
  \end{verbatim}

\end{frame}

% -)

\begin{frame}
 \frametitle{Les managers}

  \begin{center}
    \pgfuseimage{manager}
  \end{center}

\end{frame}

% -)

\begin{frame}
  \frametitle{Les managers : portabilit\'e}

   3 fa\c{c}ons d'aborder la portabilit\'e des managers:

   \begin{itemize}
   \item
     Impl\'ementer chaque manager pour chaque architecture
   \item
     D\'efinir une interface g\'en\'erale et l'impl\'ementer pour chaque architecture
   \item
     Impl\'ementer un seul manager haut niveau; et permettre l'appel d'une partie sp\'ecifique
     \`a l'architecture si n\'ecessaire
   \end{itemize}

\end{frame}

% -)

\begin{frame}
 \frametitle{Les managers}

  \begin{itemize}
    \item set manager
    \item id manager
    \item as manager
    \item segment manager
    \item region manager
    \item map manager
    \item event manager
    \item timer manager
    \item thread manager
    \item task manager
    \item io manager
    \item schedule manager
    \item message manager
    \item capability manager
    \item interface manager
    \end{itemize}

\end{frame}


% -)

\begin{frame}
  \frametitle{Gestion de la m\'{e}moire}

  kaneton offre des primitives pour g\'{e}rer la m\'{e}moire physique et la
  m\'{e}moire virtuelle \`{a} travers le segment manager, le region manager et
  enfin l'as manager.

\end{frame}

% -)

\begin{frame}
  \frametitle{as, segment et region}

  \begin{center}
    \pgfuseimage{vmem}
  \end{center}
\end{frame}

% -)

\begin{frame}
  \frametitle{Le mapping}

  \begin{center}
    \pgfuseimage{vmem_overlap}
  \end{center}
\end{frame}

% -)

\begin{frame}
  \frametitle{L'allocation non contig\"{u}e}

  \begin{center}
    \pgfuseimage{vmem_contiguous}
  \end{center}
\end{frame}

% -)

\begin{frame}
  \frametitle{Le partage}

  \begin{center}
    \pgfuseimage{vmem_sharing}
  \end{center}
\end{frame}

% -)

\begin{frame}
  \frametitle{Pourquoi une telle gestion de la m\'{e}moire ?}

  \begin{itemize}
  \item
    Proteger les donn\'{e}es entres les processus\ldots
  \item
    \ldots{} tout en permettant le partage de certaines zones
  \item
    Optimiser l'allocation de la m\'{e}moire physique
  \item
    Permettre l'utilisation de programmes contenant des adresses absolues
  \item
    Avoir \`{a} disposition plus de m\'{e}moire qu'il n'y en a (\emph{swapping})
  \end{itemize}

\end{frame}

% -)

\begin{frame}
  \frametitle{Ev\`{e}nements}

  \begin{center}
    \pgfuseimage{event}
  \end{center}

\end{frame}

% -)

\begin{frame}
  \frametitle{I/O}

  Les I/O sont le seul moyen de dialoguer avec les p\'{e}riph\'{e}riques. On
  distingue deux types d'I/O :

  \begin{itemize}
  \item
    Les PIO (\emph{Processor's I/O})
  \item
    Les I/O mapp\'{e}es en m\'{e}moire
  \end{itemize}

  \-

  Un micro-noyau est con\c{c}u pour \^{e}tre extr\^{e}mement stable. Il faut
  donc assurer que n'importe quel programme ne peut dialoguer avec un
  p\'{e}riph\'{e}rique. Imagin\'{e} qu'un programme UNIX classique puisse
  contr\^{o}ller le bus IDE et ainsi envoyer des commandes \`{a} vos disques
  durs\ldots.

  \-

  Le I/O manager permet ainsi d'autoriser ou non certains processus \`{a}
  effectuer des I/O.

\end{frame}

% -)

\begin{frame}[containsverbatim]
  \frametitle{T\^{a}ches et threads}

  Une t\^{a}che est un ensemble \verb|{| \emph{as}, \emph{threads}
  \verb|}|.

  \begin{center}
    \pgfuseimage{task}
  \end{center}

  Une t\^{a}che est juste un conteneur, ce qui s'\'{e}xecute vraiment
  ce sont les threads.

  \-

  Dans un noyau classique, une t\^{a}che contient aussi d'autres
  donn\'{e}es : table de \emph{file-descriptors}, identifiants divers
  (pid, ppid\ldots). Dans un micro-noyau, ces informations sont
  stock\'{e}es dans les services appropri\'{e}s.

\end{frame}

% -)

\begin{frame}[containsverbatim]
  \frametitle{T\^{a}ches et threads}

  Un thread est une image des registres du microprocesseur (appel\'{e}
  contexte) \`{a} un instant donn\'{e}.

  \-

  Les registres du microprocesseur englobent :

  \begin{itemize}
  \item
    Le compteur ordinal
  \item
    Le pointeur de pile
  \item
    Les registres g\'{e}n\'{e}raux
  \item
    Le contexte FPU
  \item
    Les registres sp\'{e}cifiques (exemple: registres SSE sur Pentium III)
  \item
    Les registres privil\'{e}gi\'{e}s
  \end{itemize}

  \-

  Ainsi, un thread correspond \`{a} l'image d'un fil d'\'{e}xecution.

\end{frame}

% -)

\begin{frame}
  \frametitle{La permutation de contextes}

  Permuter un contexte consiste \`{a} :

  \begin{itemize}
  \item
    Sauvergarder le contexte courant
  \item
    Mettre en place un contexte (nouveau ou ancien)
  \end{itemize}

  \-

  Un microprocesseur mono-coeur et mono-thread ne peut \'{e}xecuter
  qu'un seul thread simultan\'{e}ment. Le principe de permutation de
  contextes permet en \'{e}troite collaboration avec l'ordonnanceur de
  cr\'{e}er une illusion de parallelisme.

\end{frame}

% -)

\begin{frame}
  \frametitle{L'ordonnanceur}

  kaneton dispose d'un ordonnanceur permettant l'ex\'{e}cution
  simultan\'{e}e de plusieurs threads.

  Cet ordonnanceur est pr\'{e}emptif, \'{e}quitable et avec gestion de
  priorit\'{e}s. L'algorithme utilis\'{e} est un \emph{multilevel
  feedback queues} (\`{a} l'instar de GNU/Linux).

\end{frame}

% -)

\begin{frame}
  \frametitle{Le multilevel feedback queue}

  \begin{center}
    \pgfuseimage{mfq1}
  \end{center}
\end{frame}

% -)

\begin{frame}
  \frametitle{Le multilevel feedback queue}

  \begin{center}
    \pgfuseimage{mfq2}
  \end{center}
\end{frame}

% -)

\begin{frame}
  \frametitle{Le multilevel feedback queue}

  \begin{center}
    \pgfuseimage{mfq3}
  \end{center}
\end{frame}

% -)

\begin{frame}
  \frametitle{Le multilevel feedback queue}

  \begin{center}
    \pgfuseimage{mfq4}
  \end{center}
\end{frame}

% -)

\begin{frame}
  \frametitle{Le multilevel feedback queue}

  \begin{center}
    \pgfuseimage{mfq5}
  \end{center}
\end{frame}

% -)

\begin{frame}
  \frametitle{Le multilevel feedback queue}

  \begin{center}
    \pgfuseimage{mfq6}
  \end{center}
\end{frame}

% -)

\begin{frame}
  \frametitle{Le multilevel feedback queue}

  \begin{center}
    \pgfuseimage{mfq7}
  \end{center}
\end{frame}

% -)

\begin{frame}
  \frametitle{Le multilevel feedback queue}

  \begin{center}
    \pgfuseimage{mfq1}
  \end{center}
\end{frame}

% -)

\begin{frame}
  \frametitle{La priorisation dynamique}

  Le terme \emph{feedback} a pour origine le principe de
  modification dynamique des priorit\'{e}s.

  \begin{itemize}
  \item
    Les threads \emph{I/O-bound} sont pouss\'{e}es vers le debut de la
    file active sans que leur timeslice ne soit modifi\'{e}. Ceci permet
    de leur offrir une meilleur r\'{e}activit\'{e}.
  \item
    Les threads \emph{CPU-bound} sont pouss\'{e}s vers la fin de la file
    et leur timeslice ne bouge pas non plus. Ainsi, ils s'\'{e}xecutent en
    tâche de fond sans perdre de leur temps d'\'{e}xecution.
  \end{itemize}

\end{frame}

% -)

\begin{frame}
  \frametitle{Le support multiprocesseur}

  Nous pr\'{e}voyons que kaneton puisse tirer parti des diff\'{e}rents
  microprocesseurs dans un syst\`{e}me multiprocesseur.

  \-

  Dans un syst\`{e}me multiprocesseur sym\'{e}trique (SMP),
  l'ordonnanceur disposera d'une file de threads actifs pour chaque
  microprocesseur.

  Le cpu manager permettra d'effectuer du \emph{load balancing} et de
  la migration de t\^{a}ches d'un processeur \`{a} un autre.

  \-

  Le gestionnaire de m\'{e}moire devra lui aussi subir des
  modifications afin d'assurer dans tout le syst\`{e}me la
  coh\'{e}rence des espaces d'adressages.

\end{frame}

% -)

\begin{frame}
  \frametitle{Les IPC}

  Comme dit pr\'{e}cedement, les IPC (\emph{Inter-Process Communication})
  ont un r\^{o}le critique dans un micro-noyau.

  \-

  Les seules IPC g\'{e}r\'{e}e directement par le noyau sont les messages
  adress\'{e}s, c'est-\`{a}-dire l'envoi et la r\'{e}ception de buffers entre deux
  t\^{a}ches donn\'{e}es. Toutes les autres formes d'IPC (semaphores, mutex,
  conditions, barri\`{e}res\ldots) seront construites au dessus des
  messages.

  \-

  Il existe 2 cat\'{e}gories de messages :

  \begin{itemize}
    \item
      Les messages synchrones, qui suivent un sch\'{e}ma de
      ``rendez-vous'' et dont aucune bufferisation c\^{o}t\'{e} noyau n'est
      requise ;
    \item
      Les messages asynchrones, qui peuvent survenir \`{a} tout instant,
      n\'{e}cessitant leur bufferisation dans une file du c\^{o}t\'{e} du noyau.
  \end{itemize}

\end{frame}

% -)

\begin{frame}
  \frametitle{Les IPC dans un syst\`{e}me distribu\'{e}}

  Dans un syst\`{e}me distribu\'{e}, les IPC peuvent avoir lieu entre
  des t\^{a}ches s'\'{e}xecutant sur deux machines diff\'{e}rentes.

  \-

  Afin d'assurer la gestion des permissions lors de telles IPC,
  kaneton utilisera un syst\`{e}me de capabilities. Une capability est une
  structure int\'{e}grant :

  \begin{itemize}
  \item
    L'identifiant d'un objet ;
  \item
    Les permissions associ\'{e}es \`{a} cet objet ;
  \item
    Un champ crypt\'{e} de s\'{e}curit\'{e} rendant impossible la modification de
  la capability par une autre personne que son cr\'{e}ateur.
  \end{itemize}

  \-

  De plus, afin d'assurer la s\'{e}curit\'{e} des donn\'{e}es des messages, ces
  derni\`{e}res seront crypt\'{e}es.

\end{frame}


%
% avancement
%

\section{Avancement}

% -)

\begin{frame}
  \frametitle{Le portage IA-32}

  L'architecture Intel IA-32 correspond \`{a} la famille des
  microprocesseurs Intel du 80386 aux actuels Pentium, ainsi qu'\`{a} tous
  les autres microprocesseurs estamp\'{e}s ``x86'' (AMD, Cyrix\ldots).

  \-

  kaneton est pratiquement fonctionnel sur ces microprocesseurs. Seuls
  les IPC manquent. kaneton sur IA-32 offre :

  \begin{itemize}
  \item
    Un bootloader fonctionnant au dessus de GRUB ;
  \item
    Gestion de la MMU avec segmentation et pagination (espace d'adressage de 4 Go) ;
  \item
    Gestion des exceptions, \emph{soft-interrupt} et IRQ (\emph{via} le PIC 8259) ;
  \item
    Gestion des permissions sur les I/O (\emph{I/O permission bitmap}) ;
  \item
    Gestion des threads et \emph{context switching} ;
  \item
    Service int\'{e}gr\'{e} de \emph{timers} gr\^{a}ce au PIT 8254 ;
  \item
    Services int\'{e}gr\'{e}s de console, \emph{framebuffer} basse r\'{e}solution et port s\'{e}rie.
  \end{itemize}

\end{frame}

% -)

\begin{frame}
  \frametitle{Les tests}

  kaneton \`{a} pour originalit\'{e} d'\^{e}tre 100\% testable. En effet, chacune
  des fonctionnalit\'{e} du noyau peut-\^{e}tre test\'{e}e, notamment pour assurer
  la non-regression.

  \-

  De la m\^{e}me mani\`{e}re, puisque kaneton est un projet \`{a} faire par des
  \'{e}tudiants, ce syst\`{e}me de test nous est aussi utile pour l'\'{e}valuation
  des rendus.

  \-

  Cette moulinette de test permet notamment de tester le noyau en boot
  r\'{e}el : une premi\`{e}re machine sur laquelle est lanc\'{e} le script de la
  moulinette, reli\'{e}e par port s\'{e}rie \`{a} une seconde machine sur laquelle
  tourne un kaneton qu'on teste.

\end{frame}

% -)

\begin{frame}
  \frametitle{Les tests}

  La moulinette kaneton est \'{e}crite en \emph{python} (travail original
  r\'{e}alis\'{e} par Solal Jacob).

  \-

  Une s\'{e}rie de tests, chacun sous forme d'un fichier C portant un nom
  particulier est alors compil\'{e}e et li\'{e}e avec le binaire du noyau. Ces
  tests peuvent appeler absolument toutes les fonctions du noyau et on
  le niveau de privil\`{e}ges maximum.

  \-

  Le fonctionnement global est le suivant :

  \begin{itemize}
  \item
    Le script python envoit l'adresse d'un test (une fonction) sur le
    port s\'{e}rie ;
  \item
    La fonction est alors \'{e}xecut\'{e}e par le noyau, les r\'{e}sultats sont
    renvoy\'{e}s sur le port s\'{e}rie ;
  \item
    Le script compare avec le r\'{e}sultat attendu, il y a aussi
    possibilit\'{e} de scripter afin de parser le r\'{e}sultat ;
  \item
    Une trace est \'{e}mise : fichier HTML style uTTK (moulinette des
    ACUs) et fichier Ruby pour le calcul de la note en fonction des
    tests pass\'{e}s ou non (toujours dans le style ACU).
  \end{itemize}

\end{frame}

% -)

\begin{frame}[containsverbatim]
  \frametitle{Un test}

  \begin{verbatim}
void		check_as_02(void)
{
  [...]

  ASSERT(task_reserve([...]) == ERROR_NONE, "error creating task\n");
  ASSERT(as_reserve([...]) == ERROR_NONE, "error creating as\n");
  ASSERT(segment_reserve([...]) == ERROR_NONE, "error reserving segment\n");
  ASSERT(region_reserve([...]) == ERROR_NONE, "error reserving region\n");

  ASSERT(segment_get(seg, &oseg) == ERROR_NONE, "error getting segment\n");
  ASSERT(oseg->segid == seg, "Bad segid field in segment\n");
  ASSERT(oseg->asid == as, "Bad asid field in segment\n");

  ASSERT(region_get(as, reg, &oreg) == ERROR_NONE, "error getting region\n");
  ASSERT(oreg->regid == reg, "Bad regid field in region\n");
  ASSERT(oreg->segid == seg, "Bad segid field in region\n");

  ASSERT(region_release([...]) == ERROR_NONE, "failed to release region\n");
  ASSERT(segment_release([...]) == ERROR_NONE, "failed to release segment\n");
  ASSERT(as_release([...]) == ERROR_NONE, "failed to release as\n");
  ASSERT(task_release([...]) == ERROR_NONE, "failed to release task\n");

  [...]
}
  \end{verbatim}

\end{frame}

%
% projets
%

\section{Projets}

\begin{frame}
  \frametitle{Les choses \`{a} faire}

  \begin{itemize}
  \item
    Le portage sur architecture MIPS et eventuellement SPARC
  \item
    Version optimis\'{e} du portage IA-32
  \item
    Version utilisant la segmentation uniquement pour la m\'{e}moire virtuelle
  \item
    Le multiprocesseur sym\'{e}trique (SMP)
  \item
    Les pilotes
  \item
    Une couche r\'{e}seau
  \item
    Un Virtual Filesystem Switch (VFS)
  \item
    Une librairie d'\'{e}mulation UNIX
  \item
    Les services distribu\'{e}s : partage de m\'{e}moire, des
    p\'{e}riph\'{e}riques, migration de process\ldots
  \item
    Am\'{e}liorer la moulinette : injection distante de code, acc\`{e}s au
    contexte, exploration de la m\'{e}moire, \emph{rollback}\ldots
  \item
    Am\'{e}liorer les algorithmes du noyau : allocation de
    m\'{e}moire, ordonnancement\ldots
  \end{itemize}

\end{frame}

%
% ce que nous cherchons
%

\section{Ce que nous cherchons}

\begin{frame}

  \textbf{Deux \'{e}tudiants, de l'Ing1 \`{a} l'Ing3}.

  \begin{itemize}
  \item
    Pour maintenir et faire \'{e}voluer le projet
  \item
    Pour donner les cours aux Ing1 (en 2007)
  \item
    Pour donner les cours de sp\'{e}cialisation (en 2008)
  \end{itemize}

  Ces \'{e}tudiants int\'{e}greront le Laboratoire Syst\`{e}me \& S\'{e}curit\'{e}
  d'EPITA/EPITECH (LSE) de Janvier jusqu'\`{a} la fin de leur scolarit\'{e}.

  \-

  kaneton est un projet qui demande beaucoup de temps. Nous cherchons
  deux personnes motiv\'{e}s et ayant un bon niveau g\'{e}n\'{e}ral. Il n'est pas
  envisageable de postuler pour finalement quitter l'\'{e}quipe deux mois
  apr\`{e}s.

\end{frame}

\end{document}
