%
% ---------- header -----------------------------------------------------------
%
% project       kaneton
%
% license       kaneton
%
% file          /home/mycure/kane...curriculum/presentation/presentation.tex
%
% created       julien quintard   [mon may 14 21:02:29 2007]
% updated       julien quintard   [wed feb 13 16:53:32 2008]
%

%
% path
%

\def\path{../..}

%
% template
%

%
% ---------- header -----------------------------------------------------------
%
% project       kaneton
%
% license       kaneton
%
% file          /home/mycure/kaneton/view/template/lecture.tex
%
% created       julien quintard   [wed may 16 18:17:26 2007]
% updated       julien quintard   [sun may 18 23:23:40 2008]
%

%
% class
%

\documentclass[8pt]{beamer}

%
% packages
%

\usepackage{pgf,pgfarrows,pgfnodes,pgfautomata,pgfheaps,pgfshade}
\usepackage[T1]{fontenc}
\usepackage{colortbl}
\usepackage{times}
\usepackage{amsmath,amssymb}
\usepackage{graphics}
\usepackage{graphicx}
\usepackage{color}
\usepackage{xcolor}
\usepackage[english]{babel}
\usepackage{enumerate}
\usepackage[latin1]{inputenc}
\usepackage{verbatim}
\usepackage{aeguill}

%
% style
%

\usepackage{beamerthemesplit}
\setbeamercovered{dynamic}

%
% verbatim stuff
%

\definecolor{verbatimcolor}{rgb}{0.00,0.40,0.00}

\makeatletter

\renewcommand{\verbatim@font}
  {\ttfamily\footnotesize\selectfont}

\def\verbatim@processline{
  {\color{verbatimcolor}\the\verbatim@line}\par
}

\makeatother

%
% -
%

\renewcommand{\-}{\vspace{0.4cm}}

%
% date
%

\date{\today}

%
% logos
%

\pgfdeclareimage[interpolate=true,width=34pt,height=18pt]
                {epita}{\path/logo/epita}
\pgfdeclareimage[interpolate=true,width=49pt,height=18pt]
                {upmc}{\path/logo/upmc}
\pgfdeclareimage[interpolate=true,width=25pt,height=18pt]
                {lse}{\path/logo/lse}

\newcommand{\logos}
  {
    \pgfuseimage{epita}
  }

%
% institute
%

\institute
{
  \inst{1} kaneton microkernel project
}

%
% table of contents at the beginning of each section
%

\AtBeginSection[]
{
  \begin{frame}<beamer>
   \frametitle{Outline}
    \tableofcontents[current]
  \end{frame}
}

%
% table of contents at the beginning of each subsection
%

\AtBeginSubsection[]
{
  \begin{frame}<beamer>
   \frametitle{Outline}
    \tableofcontents[current,currentsubsection]
  \end{frame}
}


%
% title
%

\title{kaneton}

%
% authors
%

\author
{
  Elie~Bleton,
  Julian~Pidancet
}

%
% document
%

\begin{document}

%
% title frame
%

\begin{frame}
  \titlepage

  \begin{center}
    \logos
  \end{center}
\end{frame}

%
% overview
%

\section{Overview}

% 1)

\begin{frame}
  \frametitle{Who are the kaneton people ?}

  \-

  \textbf{Kaneton People}

  \begin{itemize}
    \item
      Julien Quintard
      \begin{itemize}
	\item Creator of the project a few years ago
        \item Microprocessors architecture lessons
      \end{itemize}
    \item
      Elie Bleton \& Julian Pidancet
      \begin{itemize}
	\item Working at LSE since last year
	\item Kernels and operating systems (KOS) lessons
        \item Kaneton ``student project'' advisors
      \end{itemize}
    \item
      Renaud Voltz \& Matthieu Bucchianeri
      \begin{itemize}
	\item Working at LSE since last year
	\item Former KOS teachers \& project advisors
      \end{itemize}
  \end{itemize}

  \textbf{Former Contributors}
  \begin{itemize}
    \item C\'{e}dric Aubouy (Kernel and operating systems lessons for 2007)
    \item Renaud Lienhart (IA-32 architecture for 2007)
    \item Jean-Pascal Billaud (kaneton course for 2006)
  \end{itemize}

\end{frame}

% 2)

\begin{frame}
  \frametitle{Description}

  kaneton is a \textbf{microkernel} with \textbf{pedagogical purpose}.

  \-
  \textbf{The kernel}
  \begin{itemize}
  \item The kaneton \textbf{reference} is developped by the LSE
  \item The kaneton reference is used as a basis for the \textbf{students tarball}
  \item The students tarball contains \textit{a functional build system}, lots of code and structure, with appropriate parts \textit{to implement} or \textit{to complete}.
  \end{itemize}

  \-
\textbf{The lectures}
  \begin{itemize}
    \item Microprocessor's architecture: describing completely the \textbf{design} and \textbf{mecanisms} of a \textbf{modern microprocessor}
    \item Kernels and operating systems: analysing \textbf{main parts} of a kernel such as low-level programming, memory and tasks, and introducing
    \textbf{higher-level concepts} like scheduling
  \end{itemize}

\end{frame}

% 3)

% 4)

\begin{frame}
  \frametitle{Reference implementation}

  \begin{itemize}
  \item kaneton is implemented in \textbf{C and assembly}
  \item kaneton is a microkernel : very few components are available in the core
  \item kaneton is designed with portability in mind : hardware-independant code, architecture-specific and platform-specific code are isolated from each other.
  \end{itemize}

  \-

  \begin{itemize}
  \item the kaneton core is made of several \textbf{managers}, dedicated to one task
  \item Most managers are split between hardware-specific code and generic code.
  \end{itemize}

\end{frame}

% 4)

\begin{frame}
  \frametitle{Evaluation}

  Your final grade for each term is made of two grades:

  \begin{itemize}
  \item The written evaluation (\emph{partiel})
  \item The grades for each stage of the project
  \end{itemize}

  \-

  Each stage is evaluated using a \textbf{testsuite}. You upload the
  tarball, we run the testsuite, you get your trace and grade. It's
  ACU-style.


\end{frame}

%
% details
%

\section{Stage details}

\begin{frame}
  \frametitle{Stages}

  kaneton is divided into \textbf{five stages} :

  \-

  \begin{itemize}
  \item
    K0: bootstrap
  \item
    K1: memory management
  \item
    K2: interrupts \& IRQ handling
  \item
    K3: process management \& scheduling
  \item
    K4: inter-process communication
  \end{itemize}

  \-

  For those who made the ``introduction to kernels'' (K) course in
  Ing1: kaneton is very different -- microkernel design, virtual
  memory, multitasking, IPCs\ldots -- you will learn lot of thing we
  did not talk about last year.

\end{frame}

% 5)

% Commented out in 2008 because the presentation took place
% after K0

% \begin{frame}
%   \frametitle{K0 - bootstrap}

%   \-

%   The bootstrap phase is one of the early stage of the boot
%   sequence. Its goals are:

%   \begin{itemize}
%   \item
%     To put the hardware in a known state
%   \item
%     To load the kernel from a media (usually disk or network )
%   \item
%     To prepare the kernel's environment
%   \item
%     To print funny messages like ``we all love Chiche''
%   \item
%     To jump on the kernel!
%   \end{itemize}

%   \-

%   IA-32 bootstrap is very specific because of historical reasons. So
%   this project will teach you how to deal with a restricted and
%   unfriendly environment. You will enjoy assembly programming,
%   printf-debugging, joys of addressing\ldots

% \end{frame}

% 6)

\begin{frame}
  \frametitle{K1 - memory management}

  This stages will last \textbf{three weeks}. \\

  You will:
  \begin{itemize}
  \item Enter the kaneton world
    \begin{itemize}
    \item Discover the \textit{kaneton student tarball 2008 edition}
    \item Taste the build system
    \item Familiarize yourself with kaneton's directory layout
    \end{itemize}

  \item Write the architecture-independant part of the \textbf{segment} manager
    \begin{itemize}
    \item Use common kaneton primitives and learn kaneton code-practices by doing
    \item Provide an abstraction for physical memory allocation
    \end{itemize}
  \item Write the IA-32 part of the \textbf{region} manager
    \begin{itemize}
    \item Handle \textit{virtual memory} management
    \item Take your shower on pagination handling
    \item See the MMU like you never seen it before.
    \end{itemize}
  \end{itemize}

\end{frame}

% 7)

\begin{frame}
  \frametitle{K2 - interrupt \& IRQ handling}

  This stage is \textbf{two weeks} long. You will have to develop:

  \begin{itemize}
  \item The low-level part of the \textbf{event} manager
    \begin{itemize}
    \item Taste CPU interrupt vector
    \item Program the IRQ controller
    \item Save \& Restore execution contexts
    \end{itemize}
  \item The whole \textbf{timer} manager providing one-shot or periodical timers
  \item A simple keyboard driver. Let your imagination speak and display poetry such as ``w00T s1gL 1s n()()bZ0r''
  \end{itemize}
\end{frame}

% 8)

\begin{frame}
  \frametitle{K3 - process management \& scheduling}

  The development of this part will spread on \textbf{three
  weeks}. Enjoy the menu:

  \begin{itemize}
  \item
    Contexts management and switching
  \item
    Round-Robin scheduling
  \end{itemize}

  \-

  All your code will be based on previous work. Context switching will
  allow multithreading and multitasking while the scheduler will deal
  execution timeslices to every threads.

  \-

  Once this stage is working, your kaneton will be able to run simple
  processes (we made a demonstration to Ing1 of kaneton running an
  MJPEG player without IPC).

\end{frame}

% 9)

\begin{frame}
  \frametitle{K4 - inter-process communication}

  The final part of kaneton is \textbf{two weeks} long and deals with
  Inter-Process Communication (IPC).

  \-

  The code consists in writing the whole \textbf{message} manager,
  with primitives for synchroneous, asynchroneous, blocking and
  non-blocking message sending and receiving.

  \-

  In addition, you will write a few common synchronization objects,
  such as semaphore or RW-locks.

  \-

  Once K4 finished, your kaneton is able to run multithreaded programs
  or drivers in separated address spaces, with variable privilege
  level and providind IPC.

\end{frame}

% 9)

% \begin{frame}
%   \frametitle{Scheduling}

%   Lessons are scheduled, four hours a week.

%   \begin{itemize}
%   \item Lesson 1: Prerequisites, January 16$^{th}$ (3 hours)
%   \item Lesson 2: Bootstrap, January 23$^{rd}$ (3 hours)
%   \item Lesson 3: Memory Management, January 30$^{th}$ and Febuary 6$^{th}$ (3 hours each)
%   \end{itemize}

%   \begin{itemize}
%   \item K1: 3 weeks, from March 3$^{rd}$ to March 23$^{th}$
%   \end{itemize}

% \end{frame}

%
% homework
%

\section{Your part of the job}

% 11)

\begin{frame}
  \frametitle{kaneton student implementation}

  {\Large How to work} \\
  \begin{itemize}
  \item Groups of two
  \item Each stage is graded separately
  \item Stage $N$ \textbf{always relies} on stage $N-1$. Don't get late.
  \end{itemize}

  \-

  {\Large What to do} \\
  \begin{itemize}
  \item You will work on stages 1 to 4
  \item For each stage, you will only have to implement the \textbf{most important} parts : most of the less interesting details are in the tarball.
  \item The assignments define the level of freedom you have while implementing a stage
  \item You will develop for \textbf{IA-32 architecture} (aka x86) on the \textbf{IBM-PC platform}.
  \item Your work will be tested on a \textit{real machine}, and not in an emulator.
  \end{itemize}
\end{frame}

\begin{frame}
  \frametitle{Your part of the job}

  {\Large Come to the lessons.} \\
  \begin{itemize}
  \item The project \textbf{very hardly feasible} without.
  \end{itemize}

  \-

  {\Large Pay attention}

  \begin{itemize}
  \item Visit the kaneton intranet : \textbf{http://www.lse.epita.fr/kaneton}
    \begin{itemize}
    \item Grab tarballs \& assignments regarding \textit{your} kaneton
    \item Check your traces \& logs
    \item Find additional information regarding kaneton
    \end{itemize}
  \item Don't forget the EPITA newsgroup : \textbf{epita.cours.kaneton}
  \end{itemize}

  \-

  {\Large Take it seriously}
  \begin{itemize}
  \item Care for deadlines !
  \item Care for your tarballs
  \item Care for code warnings
  \item Debug your kernel, not qemu.
  \item Impress us !
  \end{itemize}

\end{frame}

\end{document}
