%
% ---------- header -----------------------------------------------------------
%
% project       kaneton
%
% license       kaneton
%
% file          /home/mycure/kane...curriculum/presentation/presentation.tex
%
% created       julien quintard   [mon may 14 21:02:29 2007]
% updated       julien quintard   [mon may 14 21:03:31 2007]
%

%
% path
%

\newcommand{\path}{../..}

%
% template
%

%
% ---------- header -----------------------------------------------------------
%
% project       kaneton
%
% license       kaneton
%
% file          /home/mycure/kaneton/view/template/lecture.tex
%
% created       julien quintard   [wed may 16 18:17:26 2007]
% updated       julien quintard   [sun may 18 23:23:40 2008]
%

%
% class
%

\documentclass[8pt]{beamer}

%
% packages
%

\usepackage{pgf,pgfarrows,pgfnodes,pgfautomata,pgfheaps,pgfshade}
\usepackage[T1]{fontenc}
\usepackage{colortbl}
\usepackage{times}
\usepackage{amsmath,amssymb}
\usepackage{graphics}
\usepackage{graphicx}
\usepackage{color}
\usepackage{xcolor}
\usepackage[english]{babel}
\usepackage{enumerate}
\usepackage[latin1]{inputenc}
\usepackage{verbatim}
\usepackage{aeguill}

%
% style
%

\usepackage{beamerthemesplit}
\setbeamercovered{dynamic}

%
% verbatim stuff
%

\definecolor{verbatimcolor}{rgb}{0.00,0.40,0.00}

\makeatletter

\renewcommand{\verbatim@font}
  {\ttfamily\footnotesize\selectfont}

\def\verbatim@processline{
  {\color{verbatimcolor}\the\verbatim@line}\par
}

\makeatother

%
% -
%

\renewcommand{\-}{\vspace{0.4cm}}

%
% date
%

\date{\today}

%
% logos
%

\pgfdeclareimage[interpolate=true,width=34pt,height=18pt]
                {epita}{\path/logo/epita}
\pgfdeclareimage[interpolate=true,width=49pt,height=18pt]
                {upmc}{\path/logo/upmc}
\pgfdeclareimage[interpolate=true,width=25pt,height=18pt]
                {lse}{\path/logo/lse}

\newcommand{\logos}
  {
    \pgfuseimage{epita}
  }

%
% institute
%

\institute
{
  \inst{1} kaneton microkernel project
}

%
% table of contents at the beginning of each section
%

\AtBeginSection[]
{
  \begin{frame}<beamer>
   \frametitle{Outline}
    \tableofcontents[current]
  \end{frame}
}

%
% table of contents at the beginning of each subsection
%

\AtBeginSubsection[]
{
  \begin{frame}<beamer>
   \frametitle{Outline}
    \tableofcontents[current,currentsubsection]
  \end{frame}
}


%
% title
%

\title{kaneton}

%
% authors
%

\author
{
  Renaud~Voltz,
  Matthieu~Bucchianeri,
  Julien~Quintard
}

%
% document
%

\begin{document}

%
% title frame
%

\begin{frame}
  \titlepage

  \begin{center}
    \logos
  \end{center}
\end{frame}

%
% overview
%

\section{Overview}

% 1)

\begin{frame}
  \frametitle{Who we are}

  \begin{itemize}
    \item
      Julien Quintard
      \begin{itemize}
	\item
	Creator of the project a few years ago
        \item
	Microprocessors architecture lessons
      \end{itemize}
    \item
      Renaud Voltz \& Matthieu Bucchianeri
      \begin{itemize}
	\item
	Working at LSE since last year
	\item
	Kernels and operating systems lessons
      \end{itemize}
  \end{itemize}

  \nl

  \textbf{Other people involved in the past}

  \begin{itemize}
    \item
      C\'{e}dric Aubouy (Kernel and operating systems lessons for 2007)
    \item
      Renaud Lienhart (IA-32 architecture for 2007)
    \item
      Jean-Pascal Billaud (kaneton course for 2006)
  \end{itemize}

\end{frame}

% 2)

\begin{frame}
  \frametitle{Description}

  kaneton is a \textbf{microkernel} with \textbf{pedagogical purpose}.

  \-

  Like the \textbf{Tiger compiler} project developped by the LRDE, a
  \textbf{reference} is developped (by the LSE) and is used to give a
  \textbf{tarball} to students. This tarball contains some scripts,
  makefiles and code including many \textbf{FIXME}.

  \-

  kaneton is also two lectures:

  \begin{itemize}
    \item
    Microprocessor's architecture: describing completely the \textbf{design}
    and \textbf{mecanisms} of a \textbf{modern microprocessor}
    \item
    Kernels and operating systems: analysing \textbf{main parts} of a kernel
    such as low-level programming, memory and tasks, and introducing
    \textbf{higher-level concepts} like scheduling
  \end{itemize}

  \-

  The students (in groups of two) have to \textbf{apply} the concepts
  learned during the lessons by writing \textbf{many significant}
  parts of the kernel.

\end{frame}

% 3)

\begin{frame}
  \frametitle{Stages}

  kaneton is divided into \textbf{five stages} for SRS and
  \textbf{three stages} for GISTR.

  \-

  The SRS projects are:

  \begin{itemize}
  \item
    K0: bootstrap
  \item
    K1: memory management
  \item
    K2: interrupts \& IRQ handling
  \item
    K3: process management \& scheduling
  \item
    K4: inter-process communication
  \end{itemize}

  \-

  On GISTR side:

  \begin{itemize}
  \item
    K0: bootstrap
  \item
    K2: interrupts \& IRQ handling
  \item
    K3: process management \& scheduling
  \end{itemize}

  \-

  For those who made the ``introduction to kernels'' (K) course in
  Ing1: kaneton is very different -- microkernel design, virtual
  memory, multitasking, IPCs\ldots -- you will learn lot of thing we
  did not talk about last year.

\end{frame}

% 4)

\begin{frame}
  \frametitle{Implementation}

  kaneton is implemented in \textbf{C and assembly}.

  \-

  Some code in kaneton is generic and independent of underlying
  hardware (for example, the scheduler), and the rest is highly bound
  to the architecture (interrupts, virtual memory\ldots). The code is
  divided into several \textbf{managers}, each one made of an
  independent part and dependant part.

  \-

  Even if kaneton is portable, we ask you to develop for \textbf{IA-32
  architecture} (aka x86).

  \-

  Your work will be tested on \textbf{real machine}, not simulators.

\end{frame}

% 4)

\begin{frame}
  \frametitle{Evaluation}

  Your final grade for each term is made of two grades:

  \begin{itemize}
  \item
    The written evaluation (\emph{partiel})
  \item
    The grades for each stage of the project
  \end{itemize}

  \-

  Each stage is evaluated using a \textbf{testsuite}. You upload the
  tarball, we run the testsuite, you get your trace and grade. It's
  ACU-style.

\end{frame}

%
% details
%

\section{Details}

% 5)

\begin{frame}
  \frametitle{K0 - bootstrap}

  This project is a practical work session of 6 hours.

  \-

  The bootstrap phase is one of the early stage of the boot
  sequence. Its goals are:

  \begin{itemize}
  \item
    To put the hardware in a known state
  \item
    To load the kernel from a media (usually disk or network )
  \item
    To prepare the kernel's environment
  \item
    To print funny messages like ``we all love Chiche''
  \item
    To jump on the kernel!
  \end{itemize}

  \-

  IA-32 bootstrap is very specific because of historical reasons. So
  this project will teach you how to deal with a restricted and
  unfriendly environment. You will enjoy assembly programming,
  printf-debugging, joys of addressing\ldots

\end{frame}

% 6)

\begin{frame}
  \frametitle{K1 - memory management}

  This project is \textbf{two weeks} long. In kaneton words, you work is:

  \begin{itemize}
  \item
    Write the architecture-independant part of the \textbf{segment} manager
  \item
    Write the IA-32 part of the \textbf{region} manager
  \end{itemize}

  \-

  The first point consists in managing the physical memory. The second
  involves virtual memory management.

  Together, these two parts will provide complete-enough memory
  management primitives for the rest of the project.

  \-

  This stage will be helpful to handle the whole kaneton tarball. The
  code to write is not so hard but requires to be very careful about
  what you are doing (especially when handling virtual memory).

\end{frame}

% 7)

\begin{frame}
  \frametitle{K2 - interrupt \& IRQ handling}

  This stage is \textbf{two weeks} long. You will have to develop:

  \begin{itemize}
  \item
    The low-level part of the \textbf{event} manager
  \item
    The whole \textbf{timer} manager
  \item
    A simple keyboard driver
  \end{itemize}

  \-

  The development of the IA32 part of the event manager will include
  programming the interrupt vector of the CPU, programming the IRQ
  controller, handling context saving \& restoring and event
  dispatching. The timer manager is based on a programmable timer
  component and offers a high-level frontend to easily manage one-shot
  or periodical timers.

  \-

  With the keyboard driver, let your imagination speak and display
  poetry such as ``w00T s1gL 1s n()()bZ0r''.

\end{frame}

% 8)

\begin{frame}
  \frametitle{K3 - process management \& scheduling}

  The development of this part will spread on \textbf{three
  weeks}. Enjoy the menu:

  \begin{itemize}
  \item
    Contexts management and switching
  \item
    Round-Robin scheduling
  \end{itemize}

  \-

  All your code will be based on previous work. Context switching will
  allow multithreading and multitasking while the scheduler will deal
  execution timeslices to every threads.

  \-

  Once this stage is working, your kaneton will be able to run simple
  processes (we made a demonstration to Ing1 of kaneton running an
  MJPEG player without IPC).

\end{frame}

% 9)

\begin{frame}
  \frametitle{K4 - inter-process communication}

  The final part of kaneton is \textbf{two weeks} long and deals with
  Inter-Process Communication (IPC).

  \-

  The code consists in writing the whole \textbf{message} manager,
  with primitives for synchroneous, asynchroneous, blocking and
  non-blocking message sending and receiving.

  \-

  In addition, you will write a few common synchronization objects,
  such as semaphore or RW-locks.

  \-

  Once K4 finished, your kaneton is able to run multithreaded programs
  or drivers in separated address spaces, with variable privilege
  level and providind IPC.

\end{frame}

% 9)

\begin{frame}
  \frametitle{T5 scheduling}

  The three first lessons and the two first projects are scheduled for
  T5:

  \begin{itemize}
  \item
    Lesson 1: Prerequisites, January 16$^{th}$ (3 hours)
  \item
    Lesson 2: Bootstrap, January 23$^{rd}$ (3 hours)
  \item
    Lesson 3: Memory Management, January 30$^{th}$ and Febuary 6$^{th}$ (3 hours each)
  \end{itemize}

  \begin{itemize}
  \item
    K0: 6 hours pratical work, January 24$^{th}$, upload January 26$^{th}$
  \item
    K1: 2 weeks, from Febuary 6$^{th}$ to Febuary 19$^{th}$
  \end{itemize}

\end{frame}

%
% homework
%

\section{Homework}

% 11)

\begin{frame}
  \frametitle{Homework}

  For the moment, there are a few administrative things to do:

  \begin{itemize}
  \item
    Visit the kaneton intranet\\
    http://www.lse.epita.fr:8000/ from the real world\\
    http://kaneton.lse.epita.fr/ from EPITA
  \item
    Make the groups.
  \item
    Subscribe the Googlegroup\\
    http://groups.google.com/group/kaneton-students
  \item
    The EPITA newsgroup (epita.cours.kaneton) is \textbf{obsolete} and
    you won't get answers on it.
  \end{itemize}

  \-

  In addition of these:

  \begin{itemize}
  \item
    Prepare you rack hard drive for the K0 practical work session
    \begin{itemize}
    \item
      Install a correct \& up-to-date Linux distribution (Gentoo, Ubuntu,
      Fedora). Please, forget about RedHat 7.0 or other prehistorical
      distributions.
    \item
      Have a working GCC (> 3.x), GNU make, NASM and QEMU (not Bochs!). We
      won't waste time during the practical work for this kind of
      problems.
    \end{itemize}
  \item
    Come to the lessons. The project \textbf{is not feasible} without.
  \end{itemize}

\end{frame}

\end{document}
