%%
%% copyright quintard julien
%% 
%% kaneton
%% 
%% k3-subject.tex
%% 
%% path          /root/data/research/projects/svn/kaneton/projects/k3
%% 
%% made by mycure
%%         quintard julien   [quinta_j@epita.fr]
%% 
%% started on    Mon Feb 21 16:02:38 2005   mycure
%% last update   Wed Mar 23 17:19:19 2005   mycure
%%

\documentclass[10pt,a4wide]{article}
\usepackage[english]{babel}
\usepackage{a4wide}
\usepackage{graphicx}
\usepackage{graphics}
\usepackage{fancyheadings}
\pagestyle{fancy}

\bibliographystyle{plain}

\lhead{\scriptsize{kaneton project}}
\rhead{k3 subject}
\rfoot{\scriptsize{EPITA System Lab}}

\title{kaneton-3}

\author{Julien Quintard - \small{quinta\_j@epita.fr} \\
        Jean-Pascal Billaud - \small{billau\_j@epita.fr} \\ \\
	\small{last updated by} \\
	Julien Quintard - \small{quinta\_j@epita.fr}}

\date{\today}

\begin{document}
\maketitle

\section{Informations}

\paragraph{}

\begin{tabular}{p{7cm}l}

Date de rendu: & Lundi 14 Mars 2005 \`a 23h42 \\
Dur\'ee du projet: & 3 semaines \\
Nom du fichier de rendu: & k3.tar.gz \\
Responsable du projet: & Julien Quintard - \small{quinta\_j@epita.fr} \\
                       & Jean-Pascal Billaud - \small{billau\_j@epita.fr} \\
Newsgroups d\'edi\'es: & epita.kaneton, epita.adm.sr \\
Langages: & asm, C \\
Architectures: & Intel 32-bit \\
Nombre de personnes par groupes: & 3 \`a 5

\end{tabular}

\section{Introduction}

\paragraph{}

Votre kernel poss\`ede un gestionnaire de m\'emoire physique, un gestionnaire
d'espace d'adressage et certains drivers.

\paragraph{}

Le but de ce projet est de r\'ealiser un ensemble de fonctions pour la
gestion de la m\'emoire virtuelle et bien entendu de finaliser par la
m\^eme occasion le gestionnaire d'espaces d'adressage.

\paragraph{}

En effet au terme de ce projet, une entit\'e pourra d\'ecider de cr\'eer
un espace d'adressage et de r\'eserver un nombre de pages sans probl\`emes.
Le kernel devra donc s'occuper de la r\'eservation des structures de
donn\'ees g\'en\'eriques mais \'egalement propre au processeur, les
initialiser, les maintenir \`a jour, en installer de nouvelles si n\'ecessaire
etc..

\paragraph{}

Pour le moment le kernel dispose de son espace d'adressage virtuel.
N\'eanmoins, mis \`a part l'installation de son adressage, nous n'avons
fourni aucune interface pour ajouter, modifier ou enlever des \'el\'ements
de cet espace d'adressage virtuel.

\paragraph{}

Au terme de cette \'etape, kaneton sera capable de r\'eserver et
de lib\'erer des pages virtuelles pour n'importe quel espace d'adressage,
mais \'egalement d'accomplir des t\^aches plus complexes sur la
m\'emoire virtuelle.

\section{Travail Demand\'e}

\paragraph{}

Le projet consiste \`a d\'evelopper un gestionnaire de m\'emoire virtuelle.

\paragraph{}

Il ne vous sera rien demander d'autre que le fait de fournir une interface
fonctionnelle.

\section{Interfaces}

\paragraph{}

\subsection{Espaces d'adressage}

\paragraph{}

Dans cette partie il vous sera demand\'e de mener le d\'eveloppement du
gestionnaire d'espace d'adressage \`a son terme en d\'eveloppant toute la
partie espace d'adressage virtuels.

\paragraph{}

Un espace d'adressage virtuel est un ensemble d'\'el\'ements d\'ecrivant
la m\'emoire virtuelle en mettant en relation des zones virtuelles avec des
zones physiques ou bien simplement des zones virtuelles n'ayant aucune
correspondance (tout du moins pour l'instant) avec des zones physiques.

\paragraph{}

Dans le cas du processeur Intel, la m\'emoire virtuelle est d\'ecrite
par le syst\`eme de traduction d'adresses nomm\'e ``Pagination''. La
partie ``machine dependent'' du gestionnaire d'espace d'adressage sera
donc compos\'e d'une sous partie nomm\'ee \textbf{vas} s'occupant de la
gestion des structures hardware en relation avec la m\'emoire virtuelle.

\paragraph{}

Votre travail consiste donc essentiellement \`a d\'evelopper la sous-partie
\textbf{vas} et \`a relier cette partie ``machine dependent'' au gestionnaire
d'espace d'adressage ``machine independent''.

\subsection{M\'emoire virtuelle}

\paragraph{}

Dans un premier temps, il vous faudra initialiser les zones virtuelles
de votre kernel.

\paragraph{}

En effet dans le bootloader, vous mappiez nombre de pages physiques
mais vous ne gardiez aucune trace de ces allocations. Pour initialiser
ce syst\`eme de traces d'allocation, prenez exemple sur votre
initialisation des zones physiques, le but \'etant que l'espace
d'adressage du kernel soit coh\'erent et repr\'esente r\'eellement
la m\'emoire virtuelle utilis\'ee par le kernel.

\paragraph{}

Votre gestionnaire de m\'emoire virtuelle doit fournir un certain nombre
de fonctionnalit\'es permettant d'allouer, de lib\'erer des pages
virtuelles mais \'egalement de manipuler des donn\'ees en virtuel.

\paragraph{}

\hspace{1.5cm}int \textbf{vm\_init}(void);

\paragraph{}

Cette fonction initialise le gestionnaire de m\'emoire virtuelle.

\paragraph{}

\hspace{1.5cm}int \textbf{vm\_rsv}(t\_asid \textbf{asid},
                                   t\_vaddr* \textbf{vaddr},
                                   t\_vsize \textbf{npages},
                                   t\_vmflags \textbf{flags});

\paragraph{}

Cette fonction se contente d'allouer un nombre donn\'e de pages virtuelles
dans un espace d'adressage donn\'e: \textbf{asid}. Le param\`etre
\textbf{*vaddr} contient en entr\'ee l'adresse \`a laquelle l'appellant
d\'esire r\'eserver de la m\'emoire virtuelle. En sortie, ce param\`etre
contient l'adresse trouv\'ee pour effectuer cette allocation.

\paragraph{}

Les options sont identiques \`a celles de la fonction \textbf{pm\_rsv}()
comme par exemple: VM\_FLAG\_ANY, VM\_FLAG\_SPECIFIC etc..

\paragraph{}

Cette fonction modifiera \'egalement l'espace d'adressage \textbf{asid}.

\paragraph{}

\hspace{1.5cm}int \textbf{vm\_attr}(t\_asid \textbf{asid},
                                    t\_vaddr \textbf{vaddr},
                                    t\_vsize \textbf{npages},
                                    t\_vattr \textbf{attr});

\paragraph{}

Cette fonction permet la modification des attributs d'une zone de
m\'emoire virtuelle. Les attributs pourraient \^etre compos\'es
des flags suivants: VM\_ATTR\_EXEC, VM\_ATTR\_READ, VM\_ATTR\_WRITE etc..

\paragraph{}

Cette fonction modifiera l'espace d'adressage \textbf{asid} seulement si
vous d\'ecidez de garder une trace plus compl\`ete de la m\'emoire virtuelle
en software.

\paragraph{}

Nous vous le conseillons pour une question de logique.

\paragraph{}

\hspace{1.5cm}int \textbf{vm\_rel}(t\_asid \textbf{asid},
                                   t\_vaddr \textbf{vaddr},
                                   t\_vsize \textbf{npages});

\paragraph{}

Cette fonction lib\`ere de la m\'emoire virtuelle.

\paragraph{}

Cette fonction modifiera l'espace d'adressage \textbf{asid}.

\paragraph{}

\hspace{1.5cm}int \textbf{vm\_map}(t\_asid \textbf{asid},
				   t\_paddr \textbf{paddr},
                                   t\_vaddr \textbf{vaddr},
                                   t\_vsize \textbf{npages});

\paragraph{}

Cette fonction \'etablie une correspondance entre l'adresse physique
\textbf{paddr} et l'adresse virtuelle \textbf{vaddr} et cela pour
un nombre de pages donn\'e.

\paragraph{}

Cette fonction effectuera des modifications sur l'espace d'adressage.
En effet il faudra que l'espace d'adressage ait bien pris en compte que
d\'esormais cette zone physique est mapp\'ee par une - ou plusieurs -
adresses virtuelles.

\paragraph{}

Vos structures doivent permettre le fait qu'une zone physique soit mapp\'ee
par plusieurs pages virtuelles de ce m\^eme espace d'adressage mais \'egalement
qu'une zone soit mapp\'ee par plusieurs processus voire \textbf{n} fois
chacun. Bien entendu cette fonctionnalit\'e ne vous sera possible et demand\'ee
que si vous avez effectu\'e le bonus \textbf{Sharing} du projet pr\'ec\'edent.

\paragraph{}

Si vous avez effectu\'e le bonus \textbf{Object} vous devez savoir le faire
en utilisant des objets et vous serez donc capable d'expliquer comment
faire et ce qu'il faudrait enlever ou ajouter aux gestionnaires exitants.

\paragraph{}

\hspace{1.5cm}int \textbf{vm\_unmap}(t\_asid \textbf{asid},
                                     t\_vaddr \textbf{vaddr},
                                     t\_vsize \textbf{npages});

\paragraph{}

Cette fonction enl\`eve la correspondance pr\'ec\'edemment \'etablie
sur cette zone virtuelle.

\paragraph{}

\hspace{1.5cm}int \textbf{vm\_copy}(t\_asid \textbf{from},
                                    t\_vaddr \textbf{src},
                                    t\_asid \textbf{to},
                                    t\_vaddr \textbf{dst},
                                    t\_size \textbf{nbytes});

\paragraph{}

Cette fonction se charge de copier un nombre d'octets donn\'e entre
deux espaces d'adressage virtuels.

\paragraph{}

Pour utiliser cette fonction il faudra disposer de droits sp\'eciaux
car il est clair que tout le monde ne peut pas se permettre de copier
des donn\'ees un peu n'importe o\`u.

\paragraph{}

\hspace{1.5cm}int \textbf{vm\_flush}(t\_asid \textbf{asid});

\paragraph{}

Cette fonction se charge de nettoyer l'espace d'adressage \textbf{asid}
de tout son espace d'adressage virtuel.

\paragraph{}

\hspace{1.5cm}int \textbf{vm\_clean}(void);

\paragraph{}

Cette fonction r\'einitialise le gestionnaire de m\'emoire virtuelle.

\subsection{Allocation de m\'emoire}

\paragraph{}

Cette section n'est pas fondamentale mais permet de simplifier l'allocation
de pages m\'emoire pour une utilisation imm\'ediate.

\paragraph{}

\hspace{1.5cm}int \textbf{mm\_rsv}(t\_asid \textbf{asid},
                                   t\_vaddr* \textbf{vaddr},
                                   t\_vsize \textbf{npages});

\paragraph{}

La fonction \textbf{mm\_rsv}() se charge d'allouer \textbf{npages} physiques
et virtuelles et de les mapper. Cette fonction effectue donc les trois
t\^aches fondamentales pour allouer de la m\'emoire et la rendre disponible
et utilisable: \textbf{pm\_rsv}(), \textbf{vm\_rsv}() et \textbf{vm\_map}().

\paragraph{}

\hspace{1.5cm}int \textbf{mm\_rel}(t\_asid \textbf{asid},
                                   t\_vaddr \textbf{vaddr},
                                   t\_vsize \textbf{npages});

\paragraph{}

Cette fonction se charge de lib\'erer la m\'emoire physique et la m\'emoire
virtuelle tout en invalidant le mapping.

\section{Types}

\paragraph{}

Voici une explication plus compl\`ete des nouveaux types que
vous devrez utilis\'es.

\paragraph{}

Dans ce projet, les nouveaux types, sont rares et se limitent en r\'ealit\'e
au seul type suivant.

\subsection{t\_vattr}

\paragraph{}

Ce type d\'ecrit les attributs que peut prendre une page virtuelle.

\begin{verbatim}
#define VM_ATTR_EXEC            0x0
#define VM_ATTR_READ            0x1
#define VM_ATTR_WRITE           0x2

typedef t_uint8         t_vattr;
\end{verbatim}

\subsection{t\_vsize}

\paragraph{}

Ce type d\'ecrit un nombre de pages virtuelles.

\begin{verbatim}
typedef t_uint32        t_vsize;
\end{verbatim}

\subsection{t\_size}

\paragraph{}

Ce type d\'ecrit une taille en octets.

\begin{verbatim}
typedef t_uint32        t_size;
\end{verbatim}

\subsection{t\_flags}

\paragraph{}

Repr\'esente les diff\'erents flags. Ce type a d\'ej\`a \'et\'e vu mais nous
rajoutons ici diff\'erentes valeurs qu'utilise le gestionnaire de m\'emoire
virtuelle.

\begin{verbatim}
#define VM_FLAG_ANY             0x00
#define VM_FLAG_SPECIFIC        0x01

/*
 * ...
 */

typedef t_uint32        t_flags;
\end{verbatim}

\section{Bonus}

\paragraph{}

Voici les bonus de ce projet.

\subsection{malloc}

\paragraph{}

Nous vous proposons en bonus le d\'eveloppement de la fonction malloc
pour pouvoir allouer de la m\'emoire avec une granularit\'e accrue.

\paragraph{}

Il vous sera demand\'e de construire votre malloc intelligemment de sorte
que le m\^eme code puisse fonctionner pour un espace d'adressage diff\'erent,
c'est \`a dire pour le kernel comme pour un processus.

\section{Bibliographie}

\paragraph{}

Voici la bibliographie de k3.

\subsection{M\'emoire virtuelle}

\paragraph{}

\begin{itemize}
\item IA-32 Intel Architecture Software Developer's Manual Volume 3:
      System Programming Guide - Chapter 3 Protected-Mode Memory Management.
\item IA-32 Intel Architecture Software Developer's Manual Volume 3:
      System Programming Guide - Chapter 4 Protection.
\end{itemize}

\end{document}
