%%
%% copyright quintard julien
%% 
%% kaneton
%% 
%% kaneton.tex
%% 
%% path          /root/data/research/projects/svn/kaneton/notes/kaneton
%% 
%% made by mycure
%%         quintard julien   [quinta_j@epita.fr]
%% 
%% started on    Mon Feb 21 16:03:46 2005   mycure
%% last update   Mon Feb 21 16:03:46 2005   mycure
%%

\documentclass[10pt,a4wide]{article}
\usepackage[english]{babel}
\usepackage{a4wide}
\usepackage{graphicx}
\usepackage{graphics}
\usepackage{fancyheadings}
\pagestyle{fancy}

\bibliographystyle{plain}

\lhead{{\scriptsize kaneton project}}
\rhead{kaneton notes}
\rfoot{\scriptsize EPITA System Lab}

\title{kaneton}

\author{Julien Quintard - \small{quinta\_j@epita.fr} \\
        Jean-Pascal Billaud - \small{billau\_j@epita.fr} \\ \\
	\small{last updated by} \\
	Julien Quintard - \small{quinta\_j@epita.fr}}

\date{\today}

\begin{document}
\maketitle

\section{Notes}

\begin{enumerate}

\item explication du projet

\item ce que les \'etudiants doivent attendre du projet: acqu\'erir de
      l'autonomie, connaissance d'un processeur, architecture d'un syst\`eme
      d'exploitation etc..

\item ce que nous attendons des \'el\`eves: de l'acharnement, de l'autonomie,
      de l'int\'er\^et etc..

\item exemple de projet

\end{enumerate}

\end{document}

