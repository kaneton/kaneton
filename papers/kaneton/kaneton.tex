%%
%% copyright quintard julien
%% 
%% kaneton
%% 
%% design.tex
%% 
%% path          /home/mycure/kaneton
%% 
%% made by mycure
%%         quintard julien   [quinta_j@epita.fr]
%% 
%% started on    Fri Feb  4 15:10:05 2005   mycure
%% last update   Fri Sep  2 17:39:54 2005   mycure
%%

%%
%% --------- packages ---------------------------------------------------------
%%

\documentclass[10pt,a4wide]{article}
\usepackage[english]{babel}
\usepackage{a4wide}
\usepackage{graphicx}
\usepackage{fancyheadings}
\usepackage{multicol}
\usepackage{indentfirst}
\pagestyle{fancy}

\setlength{\footrulewidth}{0.3pt}
\setlength{\parindent}{0.3cm}
\setlength{\parskip}{2ex plus 0.5ex minus 0.2ex}

%%
%% --------- header -----------------------------------------------------------
%%

\lhead{\scriptsize{The kaneton Distributed Operating System}}
\rhead{\scriptsize{The kaneton Implementation}}
\rfoot{\scriptsize{EPITA Computer System Laboratory}}

\title{The kaneton Distributed Operating System Implementation}

\author{\small{Julien Quintard} \\
        \scriptsize{EPITA Computer System Laboratory, Paris, France}}

\date{\scriptsize{\today}}

\begin{document}

\maketitle

%%
%% --------- abstract ---------------------------------------------------------
%%

\begin{abstract}

\end{abstract}

%
% --------- text --------------------------------------------------------------
%



\section{The Project}

This document describes the kaneton project.

kaneton is a pedagogic distributed operating system kernel developed
by EPITA students during their specialization year.

This project consists in the design and development of parts of a
distributed operating system including the micro-kernel and the servers.

The kaneton operating system, especially the kaneton micro-kernel was
designed in a very specific way to keep the design the simpler as possible
to be understable by the students.

Nevertheless, the kaneton project and especially the kaneton reference
distributed operating system was implementing using modern tools and
modern programming techniques.

The kaneton micro-kernel is divided into managers which perform very
specific tasks and make the design simpler to understand.

We decided to develop a micro-kernel because this type of kernel is very
modular, secure, strong and clear.

All these characteristics make the micro-kernel a perfect development
project for students.

We also decided to develop a distributed operating system, because these
operating systems are modern, very interesting from the design and
implementation point of view. Moreover, the students have to know about
these operating systems.

The goals of the project are to lead the student to understand one or more
processors, an entire micro-kernel design and implementation and finally
the principles and paradigms of a distributed operating system.

The kaneton project was introduced at EPITA by two students
	Julien Quintard
	  \footnote{quinta\_j@epita.fr} and
	Jean-Pascal Billaud
	  \footnote{billau\_j@epita.fr}.

This document especially describes the kaneton reference distributed
operating system developed by
	Julien Quintard,
	Renaud Lienhart
	  \footnote{lienha\_r@epita.fr},
	Fabien Le-Mentec
	  \footnote{le-men\_f@epita.fr},
	Christophe Eymard
	  \footnote{eymard\_c@epita.fr} and
	C\'edric Aubouy
	  \footnote{aubouy\_c@epita.fr}.



\section{The Bootstrap}

The bootstrap is a little code which is launch by the BIOS.

Its role is to launch the bootloader. To do this, it has to read the
bootloader on a device which can be the floppy, the hard drive,
the network etc..

\subsection{Intel Architecture 32-bit}

For the Intel Architecture 32-bit (IA32), we decided to use the
	GRUB MultiBoot boot loader
	  \footnote{http://www.gnu.org/software/grub/}.

Indeed, the GRUB bootloader handles many different boot devices with
many different hardware devices: floppy, hard drive, network etc..

Moreover, the GRUB bootloader provide the possibility to load
modules in main memory. This feature is very intersting for a micro-kernel
because we can now tell GRUB to load every micro-kernel's servers into
main memory.

The EPITA Computer System Laboratory currently develop an Intel Architecture
32-bit bootstrap. For more information, take a look to the Laboratory
website: \textbf{http://www.lse.epita.fr}.

In technical terms, the bootstrap just load and lanch the bootloader
which will be able to initialise the kernel.



\section{The Bootloader}

The bootloader's goal is to initialise the kernel environment.

In kaneton, the kernel, when launched, has to run in a very specific
environment.

First, the virtual memory has to be initialised. Second, a special
information structure called \textbf{init} has to be provided to
the kernel when started.

So, the bootloader has to fill in these requirements to be able
to correctly laucnh the kernel.

\subsection{Intel Architecture 32-bit}

\end{document}
\label{}
\ref{}
\pageref{}

XXX machdep a la fin car depend de la partie generale contrairement a l autre
