%%
%% copyright quintard julien
%% 
%% kaneton
%% 
%% development-tree.tex
%% 
%% path          /data/kaneton/papers/lectures/development-tree
%% 
%% made by mycure
%%         quintard julien   [quinta_j@epita.fr]
%% 
%% started on    Tue Jul  5 12:23:08 2005   mycure
%% last update   Fri Oct 21 09:10:32 2005   mycure
%%

%
% class
%

\documentclass[8pt]{beamer}

%
% packages
%

\usepackage{pgf,pgfarrows,pgfnodes,pgfautomata,pgfheaps,pgfshade}
\usepackage{colortbl}
\usepackage{times}
\usepackage{amsmath,amssymb}
\usepackage{graphics}
\usepackage{graphicx}
\usepackage{color}
\usepackage{xcolor}
\usepackage[english]{babel}
\usepackage{enumerate}
\usepackage[latin1]{inputenc}

%
% style
%

\usepackage{beamerthemesplit}
\setbeamercovered{dynamic}

%
% verbatim font
%

\definecolor{verbatimcolor}{rgb}{0,0.4,0}

\makeatletter
\renewcommand{\verbatim@font}
  {\ttfamily\footnotesize\color{verbatimcolor}\selectfont}
\makeatother

%
% new line
%

\newcommand{\nl}[0]{\vspace{0.4cm}}

%
% title
%

\title{Development Tree}

%
% authors
%

\author
{
  Julien~Quintard\inst{1} \\
  {\tiny julien.quintard@gmail.com}
}

\institute
{
  \inst{1} kaneton distributed operating system project
}

%
% date
%

\date{\today}

%
% logos
%

\pgfdeclareimage[interpolate=true,width=34pt,height=18pt]
                {epita}{../../logos/epita}
\pgfdeclareimage[interpolate=true,width=49pt,height=18pt]
                {upmc}{../../logos/upmc}
\pgfdeclareimage[interpolate=true,width=25pt,height=18pt]
                {lse}{../../logos/lse}

%
% table of contents at the beginning of each section
%

\AtBeginSection[]
{
  \begin{frame}<beamer>
   \frametitle{Outline}
    \tableofcontents[current]
  \end{frame}
}

%
% table of contents at the beginning of each subsection
%

\AtBeginSubsection[]
{
  \begin{frame}<beamer>
   \frametitle{Outline}
    \tableofcontents[current,currentsubsection]
  \end{frame}
}

%
% document
%

\begin{document}

%
% title frame
%

\begin{frame}
  \titlepage

  \begin{center}
    \pgfuseimage{epita} \hspace{0.1cm} \pgfuseimage{upmc} \hspace{0.1cm}
    \pgfuseimage{lse} \hspace{0.1cm}
  \end{center}
\end{frame}

%
% outline frame
%

\begin{frame}
  \frametitle{Outline}
  \tableofcontents
\end{frame}

%
% overview
%

\section{Overview}

% 1)

\begin{frame}
  \frametitle{Introduction}

  From the previous years, a development tree was introduced
  for the kaneton project.

  \nl

  The remaining questions are:

  \begin{enumerate}
    \item
      Why?
    \item
      What are the advantages and disadvantages of such a
      development environment?
    \item
      How did the other promotions do?
  \end{enumerate}
\end{frame}

% 2)

\begin{frame}
  \frametitle{Explanations}

  Over the years, the kaneton project evolved, starting with a very
  simple introduction to low-level programming, to microkernel
  development and finally to a distributed operating system project.

  \nl

  Going always further implies many modifications in the project
  including:

  \begin{itemize}[<+->]
    \item
      The courses given which now go from the Intel processor to
      the distributed operating system concepts
    \item
      The assignments which always evolve to study advanced topics
    \item
      The context because we now have to provide parts of the microkernel
      to avoid students a development from scratch
    \item
      .. and so the requirements
  \end{itemize}
\end{frame}

% 3)

\begin{frame}
  \frametitle{The Courses}

  The kaneton project now comes with four courses:

  \begin{enumerate}
    \item
      the design of the kaneton distributed operating system including
      the microkernel
    \item
      the Intel processor
    \item
      the kernel concepts
    \item
      the distributed operating system concepts
  \end{enumerate}
\end{frame}

% 4)

\begin{frame}
  \frametitle{The Assignments}

  Due to the constant evolution of the project, the students cannot
  do the more work the previous promotions did to see advanced topics
  without modifications in the project.

  \nl

  For example, during the year 2005, the students develop a poor microkernel
  from scratch with few functionalities, a driver and finally a baby
  file system.

  \nl

  We cannot ask the students of the year 2006 to develop the same project
  but to go further to study advanced topics like the reliable broadcast
  protocol.

  \nl

  So, we cannot ask the students to develop every parts of the microkernel
  because this takes much time and implies to not study advanced
  topics.
\end{frame}

% 5)

\begin{frame}
  \frametitle{The Context}

  Providing students parts of the microkernel is not enough.

  \nl

  Indeed, we decided to provide a complete development environment

\end{frame}

%% choices: faire un truc complexe comme ca, dans le temps, multicoders
%% gere les archis, gere les OS etc..

\section{conf}

\section{env}

\section{tools}

\section{libs}

\section{core}

\section{services}

\section{programs}

\section{export}

\section{papers}

\section{doc}

\end{document}
