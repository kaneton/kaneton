%%
%% copyright quintard julien
%% 
%% kaneton
%% 
%% development-environment.tex
%% 
%% path          /data/kaneton/papers/lectures/development-environment
%% 
%% made by mycure
%%         quintard julien   [quinta_j@epita.fr]
%% 
%% started on    Tue Jul  5 12:23:08 2005   mycure
%% last update   Fri Oct 21 12:20:44 2005   mycure
%%

%
% class
%

\documentclass[8pt]{beamer}

%
% packages
%

\usepackage{pgf,pgfarrows,pgfnodes,pgfautomata,pgfheaps,pgfshade}
\usepackage{colortbl}
\usepackage{times}
\usepackage{amsmath,amssymb}
\usepackage{graphics}
\usepackage{graphicx}
\usepackage{color}
\usepackage{xcolor}
\usepackage[english]{babel}
\usepackage{enumerate}
\usepackage[latin1]{inputenc}

%
% style
%

\usepackage{beamerthemesplit}
\setbeamercovered{dynamic}

%
% verbatim font
%

\definecolor{verbatimcolor}{rgb}{0,0.4,0}

\makeatletter
\renewcommand{\verbatim@font}
  {\ttfamily\color{verbatimcolor}\selectfont}
\makeatother

%
% new line
%

\newcommand{\nl}[0]{\vspace{0.4cm}}

%
% title
%

\title{Development Environment}

%
% authors
%

\author
{
  Julien~Quintard\inst{1} \\
  {\tiny julien.quintard@gmail.com}
}

\institute
{
  \inst{1} kaneton distributed operating system project
}

%
% date
%

\date{\today}

%
% logos
%

\pgfdeclareimage[interpolate=true,width=34pt,height=18pt]
                {epita}{../../logos/epita}
\pgfdeclareimage[interpolate=true,width=49pt,height=18pt]
                {upmc}{../../logos/upmc}
\pgfdeclareimage[interpolate=true,width=25pt,height=18pt]
                {lse}{../../logos/lse}

%
% table of contents at the beginning of each section
%

\AtBeginSection[]
{
  \begin{frame}<beamer>
   \frametitle{Outline}
    \tableofcontents[current]
  \end{frame}
}

%
% table of contents at the beginning of each subsection
%

\AtBeginSubsection[]
{
  \begin{frame}<beamer>
   \frametitle{Outline}
    \tableofcontents[current,currentsubsection]
  \end{frame}
}

%
% document
%

\begin{document}

%
% title frame
%

\begin{frame}
  \titlepage

  \begin{center}
    \pgfuseimage{epita} \hspace{0.1cm} \pgfuseimage{upmc} \hspace{0.1cm}
    \pgfuseimage{lse} \hspace{0.1cm}
  \end{center}
\end{frame}

%
% outline frame
%

\begin{frame}
  \frametitle{Outline}
  \tableofcontents
\end{frame}

%
% overview
%

\section{Overview}

% 1)

\begin{frame}
  \frametitle{Introduction}

  From the previous years, a development environment was introduced
  for the kaneton project.

  \nl

  The questions are:

  \begin{enumerate}
    \item
      Why?
    \item
      What are the advantages and disadvantages of such a
      development environment?
    \item
      How did the other promotions do?
  \end{enumerate}
\end{frame}

% 2)

\begin{frame}
  \frametitle{Explanations}

  Over the years, the kaneton project evolved, starting with a very
  simple introduction to low-level programming, to microkernel
  development and finally to a distributed operating system project.

  \nl

  Going always further implies many modifications in the project
  including:

  \begin{itemize}[<+->]
    \item
      The courses given which now go from the Intel processor to
      the distributed operating system concepts
    \item
      The assignments which always evolve to study advanced topics
    \item
      The context because we now have to provide parts of the microkernel
      to avoid students a development from scratch
    \item
      .. and so the requirements
  \end{itemize}
\end{frame}

% 3)

\begin{frame}
  \frametitle{The Courses}

  The kaneton project now comes with four courses:

  \begin{enumerate}
    \item
      The design of the kaneton distributed operating system including
      the microkernel
    \item
      The Intel processor
    \item
      The kernel concepts
    \item
      The distributed operating system concepts
  \end{enumerate}
\end{frame}

% 4)

\begin{frame}
  \frametitle{The Assignments}

  Due to the constant evolution of the project, the students will have
  to lead the operating system further and furter. Nevertheless, the same
  time is allocated for the project January to November.

  \nl

  For example, during the year 2005, the students develop a poor microkernel
  from scratch with few functionalities, a driver and finally a baby
  file system.

  \nl

  We cannot ask the students of the year 2006 to develop the same project
  but to go further to study advanced topics like distributed algorithms.

  \nl

  So, we cannot ask the students to develop every parts of the microkernel
  because this takes much time and implies to not study advanced
  topics.
\end{frame}

% 5)

\begin{frame}
  \frametitle{The Context}

  Providing students parts of the microkernel is not enough.

  \nl

  Indeed, we decided to provide a complete development environment
  including:

  \begin{itemize}
    \item
      Makefiles
    \item
      Shell scripts
    \item
      Papers
    \item
      Tools
    \item
      .. everything you need to start microkernel development
  \end{itemize}
\end{frame}

% 6)

\begin{frame}
  \frametitle{Why?}

  The remaining question is:

  \nl

  \textbf{Why providing such a development environment and not letting us
    develop one ourself?}

  \nl

  The answers simply are:

  \begin{itemize}
    \item
      Developing such a development environment takes much time and
      need experience
    \item
      This development environment include very powerful features:
      multiusers cooperation, different operating systems etc..
    \item
      Finally, students will not be able to create such a complicated
      development tree so it is provided to not waste time.
  \end{itemize}
\end{frame}

% 7)

\begin{frame}
  \frametitle{The Requirements}

  The students starting the kaneton project should think that they
  will learn many many things during the year.

  \nl

  This year, we are trying to lead students to a distributed operating
  system.

  \nl

  This implies more concepts, algorithms and techniques to learn.

  \nl

  To do this we introduced more courses but the students will have
  to work hard to be able to success.
\end{frame}

% 8)

\begin{frame}[containsverbatim]
  \frametitle{Tree}

  \begin{center}

  \begin{verbatim}
    /
      conf/
      core/
      doc/
      env/
      export/
      libs/
      papers/
      programs/
      services/
      tools/
  \end{verbatim}

  \end{center}
\end{frame}

%
% conf
%

\section{conf}

% 1)

\begin{frame}
  \frametitle{Overview}

  The \textbf{conf} directory contains user variables used to parameterise:

  \begin{itemize}
    \item
      the development environment: makefiles, scripts etc..
    \item
      the kernel
  \end{itemize}

  \nl

  This configuration system is very interesting coupled with versionning
  system.

  \nl

  Indeed, you can develop using special compilation flags, specific kernel
  configuration without conflicts with other developers.
\end{frame}

% 2)

\begin{frame}[containsverbatim]
  \frametitle{Tree}

  \begin{verbatim}
    conf/
      mycure/
        conf.c
        conf.h
        kaneton.conf
        modules.conf
        mycure.conf
      ultima/
      debug/
  \end{verbatim}

  This configuration system uses the shell variable \$USER to find
  the main configuration file: \textbf{conf/\$USER/\$USER.conf}.
\end{frame}

% 3)

\begin{frame}
  \frametitle{conf.c}

  This file is not used yet.
\end{frame}

% 4)

\begin{frame}
  \frametitle{conf.h}

  This file contains macros to configure the kernel:

  \begin{itemize}
    \item
      \textbf{CONF\_TITLE}
    \item
      \textbf{CONF\_VERSION}
    \item
      \textbf{CONF\_DEBUG}
    \item
      etc..
  \end{itemize}

  \nl

  This file is included by the kernel code.
\end{frame}

% 5)

\begin{frame}
  \frametitle{kaneton.conf}

  This configuration file is used to pass arguments at the runtime to the
  services.

  \nl

  This file is also used to configure kernel and services input variables.
\end{frame}

% 6)

\begin{frame}
  \frametitle{modules.conf}

  This file contains the list of the modules to be loaded by the
  multibootloader.

  \nl

  These modules will be passed to the kernel at the boot time.

  \nl

  Be careful, a module here is not a module in the Linux or BSD terms.

  \nl

  A module is simply a file to load.
\end{frame}

% 7)

\begin{frame}
  \frametitle{\$USER.conf}

  Finally the main configuration file contains the configuration
  variables for the development environment.

  \nl

  This file uses the syntax of the make files.

  \nl

  Every variable defined in this file will be used by the makefiles
  and the scripts.
\end{frame}

%
% env
%

\section{env}

% 1)

\begin{frame}
  \frametitle{Overview}

  The \textbf{env} directory contains the different development environments.

  \nl

  This directory is the heart of the kaneton development system.

  \nl

  Indeed, a user can develop the kaneton project on a Mac machine using
  cross compiling ('cause PowerPC processor) while another one is using
  a FreeBSD operating system on an Intel processor.

  \nl

  So, the development environment has to deal with these different operating
  systems and architectures just for the development.
\end{frame}

% 2)

\begin{frame}
  \frametitle{Our System}

  To do this we introduced a environment system.

  \nl

  Every time a user gets the kaneton development tarball, he has to first
  create the environment given a couple operating system and architecture
  which lead to an environment.

  \nl

  Once the environment installed, the user can develop, compile the kernel
  etc.. without problems because everything (makefiles, scripts etc..) use
  the binaries, variables etc.. for its environment.

  \nl

  The environment is specified in the user configuration file.
\end{frame}

% 3)

\begin{frame}[containsverbastim]
  \frametitle{Tree}

  \begin{verbatim}
    env/
      clean
      init
      unix-ia32.ia32/
        clean
        init
        kaneton.mk
      macos-powerpc.ia32/
  \end{verbatim}

  \nl

  Here the \textbf{unix-ia32.ia32} is considered as the generic unix
  environment but everyone can add a specific linux, FreeBSD, Solaris etc..
  environment.
\end{frame}

% 4)

\begin{frame}
  \frametitle{init}

  The \textbf{init} shell script is used to install the development
  environment.

  \nl

  This script first gets the configuration variables from the user
  configuration file, then calls the specific \textbf{init} script
  of the given environment.

  \nl

  Finally the script installs some links and initialises the makefile
  dependencies.

  \nl

  The \textbf{\[environment\]/init} shell script is used to install
  specific stuff.
\end{frame}

% 5)

\begin{frame}
  \frametitle{clean}

  The \textbf{clean} shell script just cleans the environment.
\end{frame}

% 6)

\begin{frame}
  \frametitle{kaneton.mk}

\end{frame}

%
% tools
%

\section{tools}

%
% libs
%

\section{libs}

%
% core
%

\section{core}

%
% services
%

\section{services}

%
% programs
%

\section{programs}

%
% export
%

\section{export}

%
% papers
%

\section{papers}

%
% doc
%

\section{doc}

\end{document}
