%%
%% copyright quintard julien
%% 
%% kaneton
%% 
%% inline-assembly.tex
%% 
%% path          /home/mycure/kaneton/papers/lectures/c-preprocessor
%% 
%% made by mycure
%%         quintard julien   [quinta_j@epita.fr]
%% 
%% started on    Tue Jul  5 12:23:08 2005   mycure
%% last update   Thu Sep 29 22:33:03 2005   mycure
%%

%
% class
%

\documentclass[9pt]{beamer}

%
% packages
%

\usepackage{amsmath,amssymb,color}
\usepackage[english]{babel}
\usepackage{enumerate}
\usepackage[latin1]{inputenc}

%
% style
%

\usepackage{beamerthemesplit}

%
% verbatim font
%

% XXX

%\makeatletter
%\renewcommand{\verbatim@font}
%  {\ttfamily\small\selectfont}
%\makeatother

%
% title
%

\title{The C Preprocessor}

%
% authors
%

\author
{
  Julien~Quintard\inst{1} \\
  {\tiny quinta\_j@lse.epita.fr}
}

\institute
{
  \inst{1} EPITA Computer System Laboratory
}

%
% date
%

\date{\today}

%
% table of contents at the beginning of each section
%

\AtBeginSubsection[]
{
  \begin{frame}<beamer>
   \frametitle{Outline}
    \tableofcontents[current,currentsubsection]
  \end{frame}
}

%
% new line
%

\newcommand{\nl}[0]{\vspace{0.4cm}}

%
% document
%

\begin{document}

%
% title frame
%

\begin{frame}
  \titlepage
\end{frame}

%
% outline frame
%

\begin{frame}
  \frametitle{Outline}
  \tableofcontents
\end{frame}

%
% overview
%

\section{Overview}

% 1)

\begin{frame}
  \frametitle{Description}

  The C preprocessor, often known as cpp, is a macro processor that
  is used automatically by the C compiler to transform your program
  before compilation.

  \nl

  It is called a macro processor because it allows you to define macros,
  which are brief abbreviations for longer constructs.

\end{frame}

% 2)

\begin{frame}
  \frametitle{Syntax}

  The C preprocessor's directives are of the form:

  \begin{itemize}
    \item
      the character \textbf{\#} for specifying a preprocessor operation
    \item
      a directive name for the precise operation
  \end{itemize}

\end{frame}

% 3)

\begin{frame}
  \frametitle{General Rules}

  \begin{itemize}
    \item
      we \alert{cannot} define new directives, the directive set is fixed
    \item
      some directives require arguments: \textit{\#define argument(s)}
  \end{itemize}

\end{frame}

%
% directives
%

\section{Directives}

%
% header files
%

\subsection{Header Files}

% 1)

\begin{frame}
  \frametitle{Overview}

  Header files contain c declarations and macro definitions to be used
  by several source files.

  \nl

  You can request the use of a header file using the c preprocessor
  directive \textit{\#include}.

  \nl

  With a header file, the related declarations appear in only one place.

  \nl

  So the changes will only have to be made on a single header file
  instead of on each source file.
\end{frame}

% 2)

\begin{frame}
  \frametitle{Syntax}

  \begin{itemize}
    \item
      \textit{\#include <file>}: in the system header directories

      \begin{itemize}
	\item
	  \textbf{-I} for the directory list
	\item
	  \textbf{-nostdinc} for not searching in the system list
	\item
	  \textit{<...>} cannot contain neither wildcards nor comments
	\item
	  \textit{<...>} cannot contain the \textbf{>} character but
	  can contain the \textbf{<} character.
      \end{itemize}

    \item
      \textit{\#include ``file''}: in the current working directory

      \begin{itemize}
	\item
	  \textit{``...''} cannot contain the \textbf{``} character
	\item
	  \textit{``...''} does not accept the backslash escaped character,
	  so: ``chiche\\n\\tpresident\\n'' is a file containing
	  three backslahes
      \end{itemize}

    \item
      \textit{\#include anythingelse}: looks for a macro named
      \textbf{anythingelse}.
  \end{itemize}
\end{frame}

% 3)

\begin{frame}[containsverbatim]
  \frametitle{Example}

  \begin{verbatim}
    #if defined(__NetBSD__) || defined(__OpenBSD__) ||          \
        defined(__FreeBSD__)
      #define INCLUDE_FILE                ``bsd/include.h''
    #else
      #define INCLUDE_FILE                ``linux/include.h''
    #endif

    #include INCLUDE_FILE
  \end{verbatim}
\end{frame}

%
% one-only included files
%

\subsection{One-only Included Files}

% 1)

\begin{frame}[containsverbatim]
  \frametitle{Overview}

  It often happens that a header file includes another header file resulting in
  that a certain header is included more than once.

  \nl

  This fact leads to errors if the header files define structures, types etc..

  \nl

  The standard way to prevent these errors is:

  \begin{verbatim}
    #ifndef CHICHE_SEEN_WITH_THE_POPE
    #define CHICHE_SEEN_WITH_THE_POPE

    #endif /* CHICHE_SEEN_WITH_THE_POPE */
  \end{verbatim}
\end{frame}

% 2)

\begin{frame}
  \frametitle{Macro Naming}

  Be careful, user defines may not begin with the character \textbf{\_}
  because these are reserved for system defines.

  \nl

  To avoid conflicts, the system defines generally begin with \textbf{\_\_}.

  \nl

  The macros for one-only inclusion are name based on the header file name.

  \nl

  Moreover, and still to avoid conflicts, these macros are suffixed with
  some text. Otherwise, the two file generic/chiche.h and bsd/chiche.h
  will had generated conflicts.

  \nl

  The kaneton project uses a slightly different style:

  \begin{itemize}
    \item
      no underscore before names
    \item
      no suffixes
    \item
      but instead uses the relative path from the include directory
      as the name
  \end{itemize}

\end{frame}

% 3)

\begin{frame}[containsverbatim]
  \frametitle{Example}

  Let's take a look at a kaneton example:

  The file \textit{core/include/kaneton/set.h} where \textit{core/kaneton}
  is the include dir:

  \begin{verbatim}
    #ifndef KANETON_SET_H
      ...
    #endif
  \end{verbatim}

  The file \textit{core/include/arch/ia32/kaneton/set.h} where
  \textit{core/include/arch/machdep} is the include directory:

  \begin{verbatim}
    #ifndef IA32_KANETON_SET_H
      ...
    #endif
  \end{verbatim}
\end{frame}

\subsection{Macros}

\subsection{Conditionals}

\section{Pitfalls}

\section{Diagnostics}

%
% bibliography
%

\begin{thebibliography}{10}

  \bibitem{Test}[Test, 42]
    Test Bande
    \newblock Bande tout dans le kignon
    \newblock Dechire toi le teton sur une rape a fromage

\end{thebibliography}

\end{document}

XXX

\alert

\logo{---kaneton---}

\insertlogo

\textbf<2>{test}
\only<2>{test}

\onslide<1>{bande}
+*
\uncover
\visible
\only

\item<+->
\item<+->
\item<+->
plutot que 1 2 3, ca incremente

\begin{itemize}[<+->]
incremente chaque item

\part{bande tout} permet de decouper les longues pres genre cours

\lecture{bande}{week 1}
...
\lecture{suce}{week 2}

\enumerate[(i)]
\enumerate[ball] [triangle] [circle] [square]
 item

\note[item]<2>{Tell a joke about plants}

---

