%%
%% copyright quintard julien
%% 
%% kaneton
%% 
%% inline-assembly.tex
%% 
%% path          /home/mycure/kaneton/papers/lectures/requirements
%% 
%% made by mycure
%%         quintard julien   [quinta_j@epita.fr]
%% 
%% started on    Tue Jul  5 12:23:08 2005   mycure
%% last update   Sun Oct 23 12:32:52 2005   mycure
%%

%
% class
%

\documentclass[8pt]{beamer}

%
% packages
%

\usepackage{pgf,pgfarrows,pgfnodes,pgfautomata,pgfheaps,pgfshade}
\usepackage{colortbl}
\usepackage{times}
\usepackage{amsmath,amssymb}
\usepackage{graphics}
\usepackage{graphicx}
\usepackage{color}
\usepackage{xcolor}
\usepackage[english]{babel}
\usepackage{enumerate}
\usepackage[latin1]{inputenc}

%
% style
%

\usepackage{beamerthemesplit}
\setbeamercovered{dynamic}

%
% verbatim font
%

\definecolor{verbatimcolor}{rgb}{0,0.4,0}

\makeatletter
\renewcommand{\verbatim@font}
  {\ttfamily\footnotesize\color{verbatimcolor}\selectfont}
\makeatother

%
% new line
%

\newcommand{\nl}[0]{\vspace{0.4cm}}

%
% title
%

\title{C Preprocessor}

%
% authors
%

\author
{
  Julien~Quintard\inst{1} \\
  {\tiny julien.quintard@gmail.com}
}

\institute
{
  \inst{1} kaneton distributed operating system project
}

%
% date
%

\date{\today}

%
% logos
%

\pgfdeclareimage[interpolate=true,width=34pt,height=18pt]
                {epita}{../../logos/epita}
\pgfdeclareimage[interpolate=true,width=49pt,height=18pt]
                {upmc}{../../logos/upmc}
\pgfdeclareimage[interpolate=true,width=25pt,height=18pt]
                {lse}{../../logos/lse}

%
% table of contents at the beginning of each section
%

\AtBeginSection[]
{
  \begin{frame}<beamer>
   \frametitle{Outline}
    \tableofcontents[current]
  \end{frame}
}

%
% table of contents at the beginning of each subsection
%

\AtBeginSubsection[]
{
  \begin{frame}<beamer>
   \frametitle{Outline}
    \tableofcontents[current,currentsubsection]
  \end{frame}
}

%
% document
%

\begin{document}

%
% title frame
%

\begin{frame}
  \titlepage

  \begin{center}
    \pgfuseimage{epita} \hspace{0.1cm} \pgfuseimage{upmc} \hspace{0.1cm}
    \pgfuseimage{lse} \hspace{0.1cm}
  \end{center}
\end{frame}

%
% outline frame
%

\begin{frame}
  \frametitle{Outline}
  \tableofcontents
\end{frame}

%
% overview
%

\section{Overview}

% 1)

\begin{frame}
  \frametitle{Description}

  The C preprocessor, often known as cpp, is a macro processor
  automatically used by the C compiler to transform your program
  before compilation.

  \nl

  It is called a macro processor because it allows you to define macros,
  which are brief abbreviations for longer constructs.
\end{frame}

% 2)

\begin{frame}
  \frametitle{Syntax}

  The C preprocessor's directives are of the form:

  \begin{itemize}[<+->]
    \item
      the character \textbf{\#} for specifying a preprocessor operation
    \item
      a directive name for the precise operation
  \end{itemize}
\end{frame}

% 3)

\begin{frame}
  \frametitle{General Rules}

  \begin{itemize}[<+->]
    \item
      we \alert{cannot} define new directives, the directive set is fixed
    \item
      some directives require arguments: \textit{\#define argument(s)}
  \end{itemize}
\end{frame}

%
% directives
%

\section{Directives}

%
% header files
%

\subsection{Header Files}

% 1)

\begin{frame}
  \frametitle{Overview}

  Header files contain C declarations and macro definitions to be used
  by several source files.

  \nl

  You can request the use of a header file using the C preprocessor
  directive \textit{\#include}.

  \nl

  With a header file, the related declarations appear in only one place,
  each C source file including the header file.

  \nl

  So the changes will only have to be made on a single header file
  instead of on each source file.
\end{frame}

% 2)

\begin{frame}[containsverbatim]
  \frametitle{First Syntax}

  \textbf{\#include $<$file$>$}

  \nl

  This include looks for the file in the system header directories.

  \begin{itemize}
  \item
    \textbf{-I} for the directory list
  \item
    \textbf{-nostdinc} for not searching in the system list
  \item
    \textit{$<$...$>$} cannot contain neither wildcards nor comments
  \item
    \textit{$<$...$>$} cannot contain the \textbf{$>$} character but
    can contain the \textbf{$<$} character.
  \end{itemize}
\end{frame}

% 3)

\begin{frame}[containsverbatim]
  \frametitle{Second Syntax}

  \textbf{\#include ``file''}

  \nl

  This include looks for the file in the current working directory.

  \begin{itemize}
  \item
    \textit{``...''} cannot contain the \textbf{``} character
  \item
    \textit{``...''} does not accept the backslash escaped character,
    so: \verb|''chiche\n\tpresident\n''| is a file containing
    three backslahes
  \end{itemize}
\end{frame}

% 4)

\begin{frame}[containsverbatim]
  \frametitle{Third Syntax}

  \textbf{\#include anythingelse}

  \nl

  This include looks for a macro named \textbf{anythingelse}; then resolve
  it and reinterpret the current argument with the two previous syntaxes.
\end{frame}

% 5)

\begin{frame}[containsverbatim]
  \frametitle{Example}

  \begin{verbatim}
    #if defined(__NetBSD__) || defined(__OpenBSD__) ||          \
        defined(__FreeBSD__)
      #define INCLUDE_FILE                ``bsd/include.h''
    #else
      #define INCLUDE_FILE                ``linux/include.h''
    #endif

    #include INCLUDE_FILE
  \end{verbatim}
\end{frame}

%
% one-only included files
%

\subsection{One-only Included Files}

% 1)

\begin{frame}[containsverbatim]
  \frametitle{Overview}

  It often happens that a header file includes another header file resulting
  in header files included more than once.

  \nl

  This fact leads to errors if the header files define structures, types etc..

  \nl

  The standard way to prevent these errors is:

  \begin{verbatim}
    #ifndef CHICHE_SEEN_WITH_THE_POPE
    #define CHICHE_SEEN_WITH_THE_POPE

    #endif /* CHICHE_SEEN_WITH_THE_POPE */
  \end{verbatim}
\end{frame}

% 2)

\begin{frame}
  \frametitle{Macro Naming}

  Be careful, user's defines may not begin with the character \textbf{\_}
  because these are reserved for system defines.

  \nl

  To avoid conflicts, the system defines generally begin with \textbf{\_\_}.

  \nl

  The macros for one-only inclusion are name based on the header file name.

  \nl

  Moreover, and still to avoid conflicts, these macros are suffixed with
  some text. Otherwise, the two file generic/chiche.h and bsd/chiche.h
  will had generated conflicts.

  \nl

  The kaneton project uses a slightly different style:

  \begin{itemize}[<+->]
    \item
      no underscore before names
    \item
      no suffixes
    \item
      but instead uses the relative path from the include directory
      as the name
  \end{itemize}

\end{frame}

% 3)

\begin{frame}[containsverbatim]
  \frametitle{Example}

  Let's take a look at a kaneton example:

  \nl

  The file \textit{core/include/kaneton/set.h} where \textit{core/include}
  is the include directory:

  \begin{verbatim}
    #ifndef KANETON_SET_H
      ...
    #endif
  \end{verbatim}

  The file \textit{core/include/arch/ia32/kaneton/set.h} where
  \textit{core/include/arch/ia32} is the include directory:

  \begin{verbatim}
    #ifndef IA32_KANETON_SET_H
      ...
    #endif
  \end{verbatim}
\end{frame}

%
% macros
%

\subsection{Macros}

% 1)

\begin{frame}
  \frametitle{Overview}

  A macros is a kind of abbreviation which you define and then use later.

  \nl

  The standard convention for macro names is to use upper case.

  \nl

  Nevertheless, in few cases it is better to use lower case, especially
  when trying to develop a kind of proxy design pattern or interface as
  in the kaneton's set manager which we will study later in this course.

  \nl

  Indeed, in these few cases the designer especially wants macro calls
  to looks like function calls the user not being aware of the implementation.
\end{frame}

% 2)

\begin{frame}[containsverbatim]
  \frametitle{Example}

  \begin{verbatim}
    #define PAGESZ          4096
    #define SETSZ           PAGESZ

    #define open_readonly(filename)                                     \
      open(filename, O_RDONLY)

    int                 main(void)
    {
      int               size = SETSZ;
    }
  \end{verbatim}

  will result in:

  \begin{verbatim}
    int                 main(void)
    {
      int               size = 4096;
      int               fd = open_readonly(``/etc/passwd'');
    }
  \end{verbatim}
\end{frame}

% 3)

\begin{frame}[containsverbatim]
  \frametitle{Function-like Macros}

  Macros with arguments are called: \textbf{function-like macros}.

  \nl

  Be careful, the preprocessor only understand comma for separating arguments,
  so this call:

  \begin{verbatim}
    foo(array[x = y, x + 1])
  \end{verbatim}

  results in a macro named \textbf{foo} which takes two arguments,
  in this case: \textbf{array[x = y} and \textbf{x + 1]}.

  \nl

  Moreover if a function-like macro takes one argument and the user
  wants to pass an empty argument, he has to specify it using a
  whitespace between the parenthesis: foo( )
\end{frame}

% 4)

\begin{frame}
  \frametitle{Function-like Macros Use}

  It is also possible to use function-like macros which take zero argument
  even if there is no advantage over classical macros.

  \nl

  Think about function-like macro declarations which names \alert{must} be
  followed by an open parenthesis without any whitespace.

  \nl

  Indeed if a whitespace appear between the name and the open parenthesis
  the macro will be considered as a simple macro and the open parenthesis
  will be part of the expansion.
\end{frame}

% 5)

\begin{frame}[containsverbatim]
  \frametitle{Example}

  \begin{verbatim}
    #define chiche(x) - 1

    chiche(3)
  \end{verbatim}

  will be expanded to: \textbf{3 - 1}.

  \begin{verbatim}
    #define chiche (x) - 1

    chiche(3)
  \end{verbatim}

  will lead to an invalid use of the macro because this macro takes no
  argument.

  \nl

  The result will be:

  \begin{verbatim}
    (x) -1(3)
  \end{verbatim}
\end{frame}

% 6)

\begin{frame}
  \frametitle{Predefined Macros}

  The predefined macros fall into two categories:

  \begin{enumerate}
    \item
      standard macros
    \item
      system macros
  \end{enumerate}
\end{frame}

% 7)

\begin{frame}[containsverbatim]
  \frametitle{Standard Prefefined Macros}

  Some interesting standard predefined macros which are very useful.

  \begin{itemize}
    \item
      \textbf{\_\_FILE\_\_}: the current file name, \textit{``foo.c''}
    \item
      \textbf{\_\_LINE\_\_}: the current line number, \textit{42}
    \item
      \textbf{\_\_DATE\_\_}: the date the file was preprocessed,
      \textit{``Feb 1 1996''}
    \item
      \textbf{\_\_TIME\_\_}: same thing for the time, \textit{``23:59:01''}
  \end{itemize}

  These macros are very very useful for debugging:

  \begin{verbatim}
    fprintf(stderr, ``[%s %s] %s:%u message here\n'', 
            __DATE__, __TIME__, __FILE__, __LINE__);
  \end{verbatim}
\end{frame}

% 8)

\begin{frame}
  \frametitle{System Predefined Macros}

  These macros are set by the system for example to distinguish the
  running operating system.

  \begin{itemize}[<+->]
    \item
      \textbf{linux}: for the linux operating system
    \item
      \textbf{\_\_OpenBSD\_\_}: for the OpenBSD operating system
  \end{itemize}
\end{frame}

% 9)

\begin{frame}
  \frametitle{Stringification}

  The stringification action means turning a code fragment into
  a string constant.

  \nl

  For example stringifying \textit{chiche(42)} results in
  \textit{``chiche(42)''}.

  \nl

  The preprocessor operator used for stringification is the simple sharp:
  \textbf{\#} which has to be placed before a name.

  \nl

  This preprocessor feature is very useful for debugging.

  \nl

  It is so possible to print useful information like the macro arguments.
\end{frame}

% 10)

\begin{frame}[containsverbatim]
  \frametitle{Example}

  Let's take an example:

  \begin{verbatim}
    #define forward(function, args...)                                  \
      fprintf(stderr, ``calling: %s(%s)\n'', #function, #args)

    int                 main(void)
    {
      forward(foo, arg1, arg2);
    }
  \end{verbatim}

  This example is a bit complex because it uses a feature we have not seen
  yet.

  \begin{verbatim}
    $ ./example
    calling: foo(arg1, arg2)
    $ 
  \end{verbatim}
\end{frame}

% 11)

\begin{frame}
  \frametitle{Concatenation}

  The concatenation means joining two words into one.

  \nl

  When you define a macro, you request concatenation using the special
  operator \textbf{\#\#} in the macro body.

  \nl

  This often takes place with one constraint word and one of the macro
  argument but it is also possible to apply concatenation between two
  macro arguments
\end{frame}

% 12)

\begin{frame}[containsverbatim]
  \frametitle{Example}

  \begin{verbatim}
    #define mystring(s)         #s
    #define myconcat(name1, name2)                                      \
      mystring( name1##name2 )

    int                 main(void)
    {
      printf(``string: %s\n'', myconcat(glen, benton));
    }
  \end{verbatim}

  When launching this simple program:

  \begin{verbatim}
    $ ./example
    glenbenton
    $ 
  \end{verbatim}
\end{frame}

% 13)

\begin{frame}[containsverbatim]
  \frametitle{Problem}

  Now let's take a look to the most popular problem requesting concatenation:

  \begin{verbatim}
    typedef struct
    {
      char              *name;
      void              (*function)(void);
    }                   t_command;

    t_command           commands[] =
    {
      { ``quit'', quit_command },
      { ``help'', help_command },
    };
  \end{verbatim}
\end{frame}

% 14)

\begin{frame}[containsverbatim]
  \frametitle{Solution}

  This is a common example because this construction is a well-known one,
  used for example when reading a command and launching a corresponding
  function.

  \nl

  You could notice that there were many repetitions in the previous
  array declaration.

  \nl

  This array declaration can be simplified using a macro with concatenation:

  \begin{verbatim}
    #define NEW_COMMAND(name)   { #name, name##_command }

    t_command           commands[] =
    {
      NEW_COMMAND(quit),
      NEW_COMMAND(help),
    };
  \end{verbatim}
\end{frame}

% 15)

\begin{frame}
  \frametitle{Undefining Macros}

  This action is used to cancel a definition.

  \nl

  The directive used is \textbf{\#undef} which is followed by the macro
  name to undefined.
\end{frame}

% 16)

\begin{frame}[containsverbatim]
  \frametitle{Example}

  \begin{verbatim}
    #define FOO         4
    x = FOO;
    #undef FOO
    x = FOO
  \end{verbatim}

  expands to:

  \begin{verbatim}
    x = 4;
    x = FOO;
  \end{verbatim}

  This code will not compile because FOO is now an undefined symbol.

  \nl

  The macro undefinition works for simple macros as for function-like macros.
\end{frame}

% 17)

\begin{frame}
  \frametitle{Redefining Macros}

  Redefining a macro means defining a name that is already in use as a macro.

  \nl

  If the macro redefined matchs the previous declaration (this case is possible
  with multiple inclusions) the redefinition is simply ignored.

  \nl

  A macro redefinition is considered identical to its previous if everything is
  exactly identical excepts whitespaces which are ignored lexical symbols.

  \nl

  For the other cases and to avoid c-preprocessor errors, prefer
  undefined the previous declaration with the \textbf{\#undef} directive
  before the macro redefinition.
\end{frame}

%
% conditionals
%

\subsection{Conditionals}

% 1)

\begin{frame}
  \frametitle{Overview}

  A conditional is a directive which permits to ignore some code for the
  compilation.

  \begin{enumerate}
    \item
      this feature is very used dealing with architectures, to enable
      parts of code for a special architecture. This is also true for
      different operating systems.
    \item
      another reason is to permit this source code to be used for
      two different program compilations.
    \item
      the last reason is for excluding whole parts of code using the
      directive \textbf{\#if 0} which is always false.
  \end{enumerate}
\end{frame}

% 2)

\begin{frame}[containsverbatim]
  \frametitle{\textbf{\#if}}

  This directive tests a condition and includes the appropriate source code.

  \nl

  The syntax is:

  \begin{verbatim}
    #if condition
      controlled-text
    #endif
  \end{verbatim}

  The condition can be composed of:

  \begin{enumerate}
    \item
      numbers
    \item
      macro calls
    \item
      characters
    \item
      arithmetic operators: $||$, \&\&
  \end{enumerate}
\end{frame}

% 3)

\begin{frame}[containsverbatim]
  \frametitle{Example}

  \begin{verbatim}
    #if 1
      printf(``always true\n'');
    #endif

    #if (DEBUG & DEBUG_MALLOC)
      malloc_dump();
    #endif

    #if defined(__NetBSD__) || defined(__OpenBSD__) ||          \
        defined(__FreeBSD__)
      printf(``BSD code here\n'');
    #endif
  \end{verbatim}
\end{frame}

% 4)

\begin{frame}[containsverbatim]
  \frametitle{\textbf{\#ifdef}}

  This directive just tests the definition of the macro \textit{foo}.

  \nl

  The syntax is:

  \begin{verbatim}
    #ifdef foo
      controlled-text
    #endif
  \end{verbatim}
\end{frame}

% 5)

\begin{frame}[containsverbatim]
  \frametitle{\textbf{\#else}}

  This directive provides alternative text to be used if the condition
  is false.

  \nl

  The syntax is:

  \begin{verbatim}
    #if condition
      controlled-text
    #else
      controlled-text
    #endif
  \end{verbatim}
\end{frame}

% 6)

\begin{frame}[containsverbatim]
  \frametitle{\textbf{\#elif}}

  This directive provides an alternative including a new conditional part.

  \nl

  The syntax is:

  \begin{verbatim}
    #if condition
      controlled-text
    #elif condition
      controlled-text
    #else
      controlled-text
    #endif
  \end{verbatim}
\end{frame}

% 7)

\begin{frame}[containsverbatim]
  \frametitle{Example}

  \begin{verbatim}
    #if defined(__NetBSD__)
      printf(``NetBSD\n'');
    #elif defined(__FreeBSD__) || defined(__OpenBSD__)
      printf(``other BSD\n'');
    #else
      printf(``Non-BSD\n'');
    #endif
  \end{verbatim}
\end{frame}

% 8)

\begin{frame}[containsverbatim]
  \frametitle{Others}

  Note that the directive \textbf{defined} returns 1 if the macro is
  defined and 0 otherwise.

  \nl

  Moreover:

  \begin{verbatim}
    #if defined(foo)
  \end{verbatim}

  is equivalent to:

  \begin{verbatim}
    #ifdef foo
  \end{verbatim}

  Also note that the \textbf{\#ifndef} directive just tests whether a macro
  is not defined.
\end{frame}

%
% preprocessor controls
%

\subsection{Preprocessor Controls}

% 1)

\begin{frame}
  \frametitle{Overview}

  The preprocessor control directives are used to modify the behaviour
  of the preprocessor in certain cases.
\end{frame}

% 2)

\begin{frame}[containsverbatim]
  \frametitle{\textbf{\#error}}

  The directive \textbf{\#error} causes the preprocessor to report a
  fatal error.

  \nl

  The directive \textbf{\#error} is usually used inside of a conditional.

  \begin{verbatim}
    #ifndef ___kaneton
      #error ``this software can only compile on the kaneton kernel''
    #endif
  \end{verbatim}

  When the error directive is encoutered, the preprocessor exits.
\end{frame}

% 3)

\begin{frame}
  \frametitle{\textbf{\#warning}}

    The directive \textbf{\#warning} is identical to the error one
    without stopping the work if encountered.
\end{frame}

%
% variadic arguments
%

\subsection{Variadic Arguments}

% 1)

\begin{frame}[containsverbatim]
  \frametitle{Overview}

  The variadic arguments are used to not explicitly specify the number
  of arguments the macro needs.

  \nl

  Let's take an example:

  \begin{verbatim}
    #define dprintf(fmt, args...)                                       \
      printf(fmt, args)

    dprintf(``%s %d\n'', mystring, 42);
  \end{verbatim}
\end{frame}

% 2)

\begin{frame}[containsverbatim]
  \frametitle{Problem}

  But let's discuss about the following example:

  \begin{verbatim}
    dprintf(``very bad idea\n'');
  \end{verbatim}

  This last example will be expanded to:

  \begin{verbatim}
    printf(``very bad idea\n'', );
  \end{verbatim}

  Note that the comma is followed by the closing parenthesis. When coming
  the compiling time, gcc will warn an error because this is a syntax
  error.

  \nl

  Indeed, a variadic argument is a list of one or more arguments, but at
  least one.

  \nl

  Nevertheless, some programs need variadic arguments to become an empty
  argument and not leading to a syntax error.
\end{frame}

% 3)

\begin{frame}[containsverbatim]
  \frametitle{Solution}

  There is absolutely no solution with the common C preprocessor.

  \nl

  Nevertheless, the GNU C preprocessor introduced an extension for
  this common problem.

  \nl

  Let's see the solution:

  \begin{verbatim}
    #define dprintf(fmt, args...)                                       \
      printf(fmt, ##args)

    dprintf(``%s %d\n'', mystring, 42);
  \end{verbatim}

  The GNU C preprocessor specify in its documentation that if a variadic
  argument is empty and preceded by a comma; then the GNU C preprocessor will
  automatically remove the comma to make the syntax correct.
\end{frame}

%
% pitfalls
%

\section{Pitfalls}

%
% grouping macro arguments
%

\subsection{Grouping Macro Arguments}

% 1)

\begin{frame}[containsverbatim]
  \frametitle{Overview}

  Grouping arguments means placing parenthesis around macro arguments
  to avoid misunderstandings.

  \nl

  Think about grouping macro arguments to avoid non-wanted operations.

  \nl

  Let's show a very very simple example:

  \begin{verbatim}
    #define bar(x, y)     x + y
  \end{verbatim}
\end{frame}

% 2)

\begin{frame}[containsverbatim]
  \frametitle{Problem}

  Consider the following use:

  \begin{verbatim}
    bar(8 & 4, 5)
  \end{verbatim}

  will be expanded to:

  \begin{verbatim}
    8 & 4 + 5
  \end{verbatim}

  which is equal to:

  \begin{verbatim}
    8 & (4 + 5)
  \end{verbatim}
\end{frame}

% 3)

\begin{frame}[containsverbatim]
  \frametitle{Solution}

  Note that the result is not the one intended.

  \nl

  In fact, we wanted the macro to do:

  \begin{verbatim}
    (8 & 4) + 5
  \end{verbatim}

  For this, we have to group the macro arguments to avoid these kind of
  problems:

  \begin{verbatim}
    #define m(x, y)     ((x) + (y))
  \end{verbatim}

  Another grouping over the entire macro is not a bad idea...
\end{frame}

%
% macro to expression
%

\subsection{Macro to Expression}

% 1)

\begin{frame}[containsverbatim]
  \frametitle{Overview}

  Sometimes it is useful to return a value from a macro, in other words
  to evaluate a macro call.

  \nl

  The user only has to use the parenthesis to convert a compound statement
  into an expression.

  \begin{verbatim}
    #define call(function, argument)                                    \
      (                                                                 \
        {                                                               \
          int           _r_ = -1;                                       \
                                                                        \
          if (function)                                                 \
            _r_ = function(argument);                                   \
                                                                        \
          _r_;                                                          \
        }                                                               \
      )
  \end{verbatim}
\end{frame}

% 2)

\begin{frame}[containsverbatim]
  \frametitle{Explanations}

  The last value being the \textbf{\_r\_} variable, it will also be
  the expression's value which will be evaluated.

  \nl

  We can now use the macro expecting a return value:

  \begin{verbatim}
    if (call(printf, ``chiche\n'') == -1)
      return (-1);
  \end{verbatim}
\end{frame}

%
% swallowing the semicolon
%

\subsection{Swallowing The Semicolon}

% 1)

\begin{frame}[containsverbatim]
  \frametitle{Overview}

  Sometimes, it is useful to write macros which are a sequence of
  instructions like this one:

  \begin{verbatim}
    #define call(function, argument)                                    \
      {                                                                 \
        if (function != NULL)                                           \
          function(argument);                                           \
      }
  \end{verbatim}

  The strict use of this macro is:

  \begin{verbatim}
    call(listen, 42)
  \end{verbatim}

  Note that there is no semicolon at the end of the macro call.
\end{frame}

% 2)

\begin{frame}[containsverbatim]
  \frametitle{Problem}

  Nevertheless, users want to consider calling a macro as calling a
  typical C function. For this reason, many programmers put a semicolon
  at the end of macro calls.

  \nl

  Another reason is the identation which will not be correct if the
  programmer do not put the semicolon in some text editors.

  \nl

  In many cases, putting a semicolon at the end of a macro call
  just result with an empty instruction and the compiler just ignore it,
  but think about this example:

  \begin{verbatim}
    if (opts == 1)
      call(listen, 42);
    else
      call(send, 42);
  \end{verbatim}
\end{frame}

% 3)

\begin{frame}[containsverbatim]
  \frametitle{Explanations}

  This example is \alert{not} correct because it will be expanded to:

  \begin{verbatim}
    if (opts == 1)
      {
        if (listen != NULL)
          listen(42);
      };
    else
      {
        if (send != NULL)
          send(42);
      };
  \end{verbatim}

  The problem here is the semicolon before the \textbf{else}.
\end{frame}

% 4)

\begin{frame}[containsverbatim]
  \frametitle{Solution}

  The only way to solve this problem is to used the famous
  \textbf{do while()}.

  \begin{verbatim}
    #define call(function, argument)                                    \
      do                                                                \
      {                                                                 \
        if (function != NULL)                                           \
          function(argument);                                           \
      } while (0)
  \end{verbatim}
\end{frame}

% 5)

\begin{frame}[containsverbatim]
  \frametitle{Example}

  Our example now will be expanded to:

  \begin{verbatim}
    if (opts == 1)
      do
      {
        if (listen != NULL)
          listen(42);
      } while (0);
    else
      do
      {
        if (send != NULL)
          send(42);
      } while (0);
  \end{verbatim}
\end{frame}

%
% side effects
%

\subsection{Side Effects}

% 1)

\begin{frame}[containsverbatim]
  \frametitle{Problem}

  A side effect is an execution which produces a non-direct modification
  or which produces an non-wanted effect.

  \nl

  Consider this macro:

  \begin{verbatim}
    #define MIN(x, y)           (x) < (y) ? (x) : (y)

    MIN(fibonacci(12345678987654321), 4)
  \end{verbatim}

  will be expanded to:

  \begin{verbatim}
    (fibonacci(12345678987654321)) < (4) ? (fibonacci(12345678987654321)) : (4)
  \end{verbatim}

  You can see the double call to the function which will certainly take
  much time to compute.
\end{frame}

% 2)

\begin{frame}[containsverbatim]
  \frametitle{Solution}

  Think about this and prefer a more complex form:

  \begin{verbatim}
    #define MIN(x, y)                                                   \
      (                                                                 \
        {                                                               \
          typeof(x)     _x_ = (x);                                      \
          typeof(y)     _y_ = (y);                                      \
                                                                        \
          (_x_) < (_y_) ? (_x_) : (_y_);                                \
        }                                                               \
      )
  \end{verbatim}
\end{frame}

% 3)

\begin{frame}[containsverbatim]
  \frametitle{Problem}

  Let's see the most popular example of side effects to be sure
  you understood this kind of problems and the fact the preprocessor just copy
  and paste portions of code.

  \nl

  With the first non-corrected macro MIN():

  \begin{verbatim}
    int         a = 4;
    int	        b = 2;

    MIN(a++, b);
  \end{verbatim}

  This example seems correct, but let's see its expansion:

  \begin{verbatim}
    (a++) < (b) ? (a++) : (b);
  \end{verbatim}

  Once again, you can notice the problem, the \textbf{a} varible is finally
  incremented two times while the programmer just wanted to post-increment
  the \textbf{a} variable one time.

  \nl

  The second corrected macro also correct this problem ensuring the arguments
  to be evaluated only one time.
\end{frame}

%
% self-referenced macros
%

\subsection{Self-referenced Macros}

% 1)

\begin{frame}[containsverbatim]
  \frametitle{Overview}

  The macros expansion was made to avoid infinite expansion.

  \nl

  This fact seems to be a limitation (and it is) but also provides
  some interesting features.

  \begin{verbatim}
    #define foo(x, y)   foo(y, x)

    foo(4, 3)
  \end{verbatim}

  will be expanded to:

  \begin{verbatim}
    foo(3, 4)
  \end{verbatim}

  and nothing more.
\end{frame}

% 2)

\begin{frame}[containsverbatim]
  \frametitle{Example}

  Thanks to this, because it permits to overload code fragments, the most
  common overload handling function calls:

  \begin{verbatim}
    #define free(buf)                                                   \
      if ((buf))                                                        \
        free((buf));

    #define malloc(size)                                                \
      (                                                                 \
        {                                                               \
          void*         buf;                                            \
                                                                        \
          if ((buf = malloc((size))) == NULL)                           \
            perror(``malloc'');                                         \
                                                                        \
          buf;                                                          \
        }                                                               \
      )
  \end{verbatim}
\end{frame}

%
% case studies
%

\section{Case Studies}

% 1)

\begin{frame}
  \frametitle{kaneton set system}

  The kaneton set system is used to simulate functions overloading and
  functions forwarding.

  \nl

  The main problems are inherent to the language used: the C and the C
  preprocessor.

  \nl

  We wanted a generic interface used by the programmer and we wanted our
  system to distribute, forward the function calls to more specific
  functions.

  \nl

  The set manager is used to manage complex data structures. Then the
  programmer just call the set manager to ask it to build a precise data
  structure and to manage it: elements and memory.

  \nl

  The goal of the kaneton set system was to provide this very special
  interface using macros.
\end{frame}

% 2)

\begin{frame}[containsverbatim]
  \frametitle{Interface}

  The kaneton set system is composed of an interface.

  \nl

  The \textbf{set\_rsv()} function reserves a set specifying the desired
  data structure to use. This function returns a set identifier.

  \nl

  Then the user also use the identifier to manipulate its set. From this
  identifier the kaneton set system will be able to retrieve the correct
  data structure type so the correct functions to forward to.

  \begin{verbatim}
    #define set_rsv(_type_, _args_...)                                      \
      set_rsv_##_type_(_args_)

    #define set_add(_setid_, _args_...)                                     \
      set_trap(set_add, _setid_, ##_args_)

    #define set_head(_setid_, _args_...)                                    \
      set_trap(set_head, _setid_, ##_args_)
  \end{verbatim}
\end{frame}

% 3)

\begin{frame}[containsverbatim]
  \frametitle{Implementation}

  The concatenation was widely used for solving problems related to
  the kaneton set system. Let's take a look to the set manager and its
  interface using a trap system to forward function calls to specific
  set manager: linked-list, array etc..

  \nl

  \begin{verbatim}
    #define set_trap(_func_, _setid_, _args_...)                            \
      {                                                                     \
        o_set*            _set_;                                            \
                                                                            \
        if (set_descriptor((_setid_), &_set_) == ERROR_NONE)                \
          {                                                                 \
            switch (_set_->type)                                            \
              {                                                             \
                case SET_TYPE_ARRAY:                                        \
                  _r_ = _func_##_array((_setid_), ##_args_);                \
                  break;                                                    \
                ...                                                         \
              }                                                             \
          }                                                                 \
      }
  \end{verbatim}
\end{frame}

% 4)

\begin{frame}
  \frametitle{Templates}

  The C preprocessor can also be used to generate C functions.

  \nl

  The most common use of this technique is to generate a function given
  its types. Then we will be able to generate kind of overloaded
  functions.

  \nl

  Consider a function template which is common but the types used inside
  it change. The programmer has the choice: either program the N functions
  with the same code but with different types or generate the functions
  with macros.
\end{frame}

% 5)

\begin{frame}
  \frametitle{Functions Generation}

  Let's take an example, with a function which finds the highest value in
  the array and then swaps it with the first, so places the highest value
  as first. The programmer has to write a function for arrays of:
  characters, shorts and integers.

  \nl

  We will generate a function called \textbf{dump\_array} suffixed with
  its type to easily retrieve the function name and to avoid conflicts.
\end{frame}

% 6)

\begin{frame}[containsverbatim]
  \frametitle{Example}

  \begin{verbatim}
    #define make_dump_array(T)                                          \
      void              dump_array_##T(T*               array,          \
                                       unsigned int     size)           \
        {                                                               \
          unsigned int  max;                                            \
          T             tmp;                                            \
          unsigned int  i;                                              \
                                                                        \
          for (max = 0, i = 0; i < size; i++)                           \
            if (array[i] > array[max])                                  \
              max = i;                                                  \
                                                                        \
          tmp = array[0];                                               \
          array[0] = array[max];                                        \
          array[max] = tmp;                                             \
        }

    #define dump_array(T, array, size)                                  \
      dump_array_##T(array, size)
  \end{verbatim}
\end{frame}

% 7)

\begin{frame}[containsverbatim]
  \frametitle{Example}

  \begin{verbatim}
    make_dump_array(int)
    make_dump_array(char)

    int                 main(void)
    {
      int               iarray[5] = { 1, 2, 3, 4, 5 };
      char              carray[2] = { 'o', 'k' };

      dump_array(int, iarray, 5);
      dump_array(char, carray, 2);
    }
  \end{verbatim}
\end{frame}

%
% bibliography
%

\section{Bibliography}

\begin{thebibliography}{3}

  \bibitem{Howto}
    GNU C Preprocessor Howto

  \bibitem{Queue}
    Queue.h
    \newblock /usr/include/sys/queue.h
    \newblock A linked-list manager using macros

  \bibitem{Bpt}
    Bpt.h
    \newblock http://www.lse.epita.fr/
    \newblock A balanced+ tree manager using macros
\end{thebibliography}

\end{document}
