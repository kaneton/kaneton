%%
%% copyright quintard julien
%% 
%% kaneton
%% 
%% inline-assembly.tex
%% 
%% path          /home/mycure/kaneton/papers/lectures/inline-assembly
%% 
%% made by mycure
%%         quintard julien   [quinta_j@epita.fr]
%% 
%% started on    Tue Jul  5 12:23:08 2005   mycure
%% last update   Tue Oct  4 09:23:26 2005   mycure
%%

%
% class
%

\documentclass[9pt]{beamer}

%
% packages
%

\usepackage{pgf,pgfarrows,pgfnodes,pgfautomata,pgfheaps,pgfshade}
\usepackage{colortbl}
\usepackage{times}

\usepackage{amsmath,amssymb}

\usepackage{graphics}
\usepackage{graphicx}
\usepackage{color}
\usepackage{xcolor}

\usepackage[english]{babel}
\usepackage{enumerate}
\usepackage[latin1]{inputenc}

%
% style
%

\usepackage{beamerthemesplit}
\setbeamercovered{dynamic}

%
% verbatim font
%

\definecolor{verbatimcolor}{rgb}{0,0.4,0}

\makeatletter
\renewcommand{\verbatim@font}
  {\ttfamily\footnotesize\color{verbatimcolor}\selectfont}
\makeatother

%
% new line
%

\newcommand{\nl}[0]{\vspace{0.4cm}}

%
% title
%

\title{The Inline Assembly}

%
% authors
%

\author
{
  Julien~Quintard\inst{1} \\
  {\tiny julien.quintard@gmail.com}
}

\institute
{
  \inst{1} kaneton distributed operating system project
}

%
% logos
%

\pgfdeclareimage[interpolate=true,width=49pt,height=18pt]
		{upmc}{../../logos/upmc}
\pgfdeclareimage[interpolate=true,width=18pt,height=18pt]
		{lip6}{pictures/lip6}
\pgfdeclareimage[interpolate=true,width=34pt,height=18pt]
		{epita}{pictures/epita}
\pgfdeclareimage[interpolate=true,width=25pt,height=18pt]
		{lse}{pictures/lse}

\logo{\pgfuseimage{upmc} \hspace{0.1cm} \pgfuseimage{lip6} \hspace{0.1cm}
      \pgfuseimage{epita} \hspace{0.1cm} \pgfuseimage{lse} \hspace{0.1cm}}

%
% date
%

\date{\today}

%
% table of contents at the beginning of each section
%

\AtBeginSection[]
{
  \begin{frame}<beamer>
   \frametitle{Outline}
    \tableofcontents[current]
  \end{frame}
}

%
% table of contents at the beginning of each subsection
%

\AtBeginSubsection[]
{
  \begin{frame}<beamer>
   \frametitle{Outline}
    \tableofcontents[current,currentsubsection]
  \end{frame}
}

%
% document
%

\begin{document}

%
% title frame
%

\begin{frame}
  \titlepage
\end{frame}

%
% outline frame
%

\begin{frame}
  \frametitle{Outline}
  \tableofcontents
\end{frame}

%
% overview
%

\section{Overview}

% 1)

\begin{frame}
  \frametitle{Description}

  The inline assembly is used to insert assembly code into C source code.

  \nl

  The inline assembly reduces the function-call overhead while reducing
  the number of little assembly source files.

  \nl

  Moreover, the inline assembly source code can evolve into the C
  source one, using C variables, C constants etc..

  \nl

  The inline assembly is commonly used to improve performances. Neverthless
  system programmers appreciate the inline assembly because some
  processor instructions are only available in assembly.

  \nl

  gcc uses the keyword \textbf{asm} to introduce inline assembly.
\end{frame}

%
% syntax
%

\section{Syntax}

% 1)

\begin{frame}
  \frametitle{Overview}

  The gcc inline assembly uses AT\&T assembly syntax which is
  different from the Intel one.

  \nl

  We will study the major differences between the AT\&T syntax and the
  Intel one.
\end{frame}

% 2)

\begin{frame}
  \frametitle{Source-Destination Ordering}

  \begin{itemize}
    \item
      \textbf{Intel}: \textit{opcode dst src}
    \item
      \textbf{AT\&T}: \textit{opcode src dst}
  \end{itemize}
\end{frame}

% 3)

\begin{frame}
  \frametitle{Register Naming}

  Register names are prefixed by \textbf{\%} for example \textit{\%eax}.

  \nl

  With C input arguments, the registers are prefixed by \textbf{\%\%}
  like \textit{\%\%eax}.
\end{frame}

% 4)

\begin{frame}
  \frametitle{Immediate Operand}

  Immediate operands are prefixed by \textbf{\$}.

  \nl

  In Intel syntax, hexadecimal numbers are suffixed by \textbf{h}.
  Instead the AT\&T syntax uses \textbf{0x} as prefix.
\end{frame}

% 5)

\begin{frame}
  \frametitle{Operand Size}

  Intel syntax uses keywords \textbf{byte}, \textbf{word} and
  \textbf{dword} to specify the operand size while AT\&T syntax uses
  a suffix to the opcode: \textbf{b}, \textbf{w}, \textbf{l}.

  \begin{itemize}
    \item
      \textbf{Intel}: \textit{mov al, byte ptr foo}
    \item
      \textbf{AT\&T}: \textit{movb foo, \%al}
  \end{itemize}
\end{frame}

% 6)

\begin{frame}
  \frametitle{Memory Operands}

  The Intel syntax uses \textbf{[} and \textbf{]} to specify the
  base register while AT\&T uses \textbf{(} and \textbf{)}.

  \nl

  Moreover, the syntaxes for the shifts are also different:

  \begin{itemize}
    \item
      \textbf{Intel}:
      \textit{[basepointer + indexpointer * indexscale + immed32]}
    \item
      \textbf{AT\&T}: \textit{immed32(basepointer,indexpointer,indexscale)}
  \end{itemize}

  You could think of the formula to calculate the address as:

  \nl

  \textbf{immed32 + basepointer + indexpointer * indexscale}
\end{frame}

% 7)

\begin{frame}[containsverbatim]
  \frametitle{Example}

  The equivalent C source code:

  \begin{verbatim}
    *(p + 1)
  \end{verbatim}

  where \textbf{p} is a \textit{char*}.

  \nl

  The AT\&T assembly code:

  \begin{verbatim}
    1(%eax)
  \end{verbatim}

  where \textbf{\%eax} has the value of \textbf{p}.

  \nl

  The Intel assembly code:

  \begin{verbatim}
    [eax + 1]
  \end{verbatim}
\end{frame}

% 8)

\begin{frame}
  \frametitle{Major Differencies}

  \begin{center}

  \begin{tabular}{|p{4.9cm}|p{4.9cm}|}
    \hline
    \textbf{Intel Code} & \textbf{AT\&T Code} \\
    \hline
    mov eax, 1 & movl \$1, \%eax \\
    \hline
    mov ebx, 0ffh & movl \$0xff, \%ebx \\
    \hline
    int 80h & int \$0x80 \\
    \hline
    mov ebx, eax & movl \%eax, \%ebx \\
    \hline
    mov eax, [ecx] & movl (\%ecx), \%eax \\
    \hline
    mov eax, [ebx + 3] & movl 3(\%ebx), \%eax \\
    \hline
    mov eax, [ebx + 20h] & movl 0x20(\%ebx), \%eax \\
    \hline
    add eax, [ebx + ecx * 2h] & addl (\%ebx, \%ecx, 0x2), \%eax \\
    \hline
    lea eax, [ebx + ecx] & leal (\%ebx, \%ecx), \%eax \\
    \hline
    sub eax, [ebx + ecx * 4h - 20h] & subl -0x20(\%ebx, \%ecx, 0x4), \%eax \\
    \hline
  \end{tabular}

  \end{center}
\end{frame}

%
% basic use
%

\section{Basic Use}

% 1)

\begin{frame}[containsverbatim]
  \frametitle{Overview}

  The format of the gcc inline assembly is the following:

  \begin{verbatim}
    asm(``assembly code'');
  \end{verbatim}

  Example:

  \begin{verbatim}
    asm(``movl %eax, %ecx'');
  \end{verbatim}

  You can also use the keyword \textbf{\_\_asm\_\_} in the case of a
  name conflict with your previous source code.
\end{frame}

% 2)

\begin{frame}[containsverbatim]
  \frametitle{Registers}

  You can use multi-lines source code thinking to insert the sequence
  \verb|\n\t| at the end of each line.

  \nl

  Example:

  \begin{verbatim}
    asm(``movl %cr0, %eax\n\t''
        ``orl %eax, $1\n\t''
        ``movl %eax, %cr0'');
  \end{verbatim}

  You can notice in this example that we overwrite the register %eax.

  \nl

  \alert{*} gcc does not know anything about the inline assembly.
  Overwritting the \textbf{eax} register could result with an error because
  gcc maybe used this register to hold a variable or anything else.
\end{frame}

%
% extended use
%

\section{Extended Use}

% 1)

\begin{frame}[containsverbatim]
  \frametitle{Overview}

  In the basic use, we had only instructions. In the extended one
  we can specify more options like the input C operands, the
  output C operands and the clobbered registers i.e the
  modified registers.

  \nl

  The syntax of the inline assembly becomes:

  \begin{verbatim}
    asm(assembler template
        : output operands                  /* optional */
        : input operands                   /* optional */
        : list of clobbered registers      /* optional */
       );
  \end{verbatim}
\end{frame}

% 2)

\begin{frame}
  \frametitle{Rules}

  Be careful with this syntax:

  \begin{itemize}[<+->]
    \item
      the outputs are located before the inputs
    \item
      think to use colons to separate sections: inputs, outputs etc..
    \item
      do not specify an empty clobbered registers. In fact the
      last list provided must always contain at least one element
  \end{itemize}
\end{frame}

% 3)

\begin{frame}[containsverbatim]
  \frametitle{Example}

  Let's study some examples:

  \nl

  The following one is \alert{incorrect}:

  \begin{verbatim}
    asm(``...''
        :
        : ``...''
        :
       );
  \end{verbatim}

  Neverthless, this one is correct:

  \begin{verbatim}
    asm(``...''
        :
        : ``...''
       );
  \end{verbatim}
\end{frame}

% 4)

\begin{frame}[containsverbatim]
  \frametitle{Problem}

  Come back to our example which is a famous one in the low-level system
  programming field:

  \begin{verbatim}
    asm(``movl %cr0, %eax\n\t''
        ``orl %eax, $1\n\t''
        ``movl %eax, %cr0'');
  \end{verbatim}

  You can notice that the \textbf{eax} register is overwritten, so
  we have to tell gcc to restore the contents of the \textbf{eax} register,
  here come the clobbered register list.
\end{frame}

% 5)

\begin{frame}[containsverbatim]
  \frametitle{Solution}

  The clobber list contains the name of the registers modified by the
  inline assembly source code.

  \nl

  Let's see how to use it to resolve our problem:

  \begin{verbatim}
    asm(``movl %cr0, %eax\n\t''
        ``orl %eax, $1\n\t''
        ``movl %eax, %cr0''
        :
        :
        : ``%eax''
       );
  \end{verbatim}
\end{frame}

% 6)

\begin{frame}[containsverbatim]
  \frametitle{Example}

  Now, another example.

  \nl

  We have two C variables \textbf{A} and \textbf{B} and we want to
  move the contents of \textbf{A} into \textbf{B} but using inline
  assembly for an unknown reason:

  \begin{verbatim}
    asm(``movl %1, %%eax\n\t''
        ``movl %%eax, %0''
        : ``=r'' (b)
        : ``r'' (a)
        : ``%eax''
       );
    \end{verbatim}
\end{frame}

% 7)

\begin{frame}
  \frametitle{Explanations}

  \begin{enumerate}[<+->]
    \item
      the operands \%0-9 specify the input and outputs. Note that the first
      value is used for the first output. If no output is present, then
      \%0 will specify the first input.
    \item
      the constraint \textbf{``r''} is used to precise the nature of
      the operand. Be careful with it. In this case, it tells gcc to use
      the operand as a register. The constraint \textbf{``=''} is used for
      the output operands specifying a write-only operand.
    \item
      the double prefix \textbf{\%\%} is used when using input and/or
      output operands to avoid conflicts with registers.
    \item
      finally, the clobbered register \textbf{eax} is indicated to tell
      gcc to restore this register's contents.
  \end{enumerate}

  Now we will look at each field in details.

  \nl

  Note that we do not have to tell gcc the memory was modified because
  we used the constraints \textbf{``r''} so the registers were used.
\end{frame}

%
% template
%

\subsection{Template}

% 1)

\begin{frame}
  \frametitle{Overview}

  There are only two rules for the template:

  \begin{enumerate}[<+->]
    \item
      finish each line with the sequence \verb|\n\t|
    \item
      use \textbf{\%0, ...} to use C operands.
  \end{enumerate}
\end{frame}

%
% operands
%

\subsection{Operands}

% 1)

\begin{frame}[containsverbatim]
  \frametitle{Overview}

  The syntax used for the operands is:

  \begin{verbatim}
    ``constraint'' (operand)
  \end{verbatim}

  The constraints are primarily used to decide the addressing mode for
  operands: register, memory, etc..

  \nl

  The operands are separated by commas inside a list.

  \nl

  Operands are numbered from 0 to n-1.

  \nl

  Output operands must be lvalues but this is obvious. Input operands
  are not restricted.

  \nl

  gcc will assume that the values in the output operands are dead
  and can be overwritten.
\end{frame}

% 2)

\begin{frame}[containsverbatim]
  \frametitle{Read-Write Outputs}

  gcc inline assembly also allows read-write outputs operands.

  \begin{verbatim}
    asm(``orl %0, %0''
        : ``=r'' (x)
        : ``0'' (x)
       );
  \end{verbatim}

  In this example, we specify gcc to use the C variable \textbf{x}
  as the output operand.

  \nl

  For the input operand, we use the constraint \textbf{``0''} to
  tell gcc to use the same register for the input operand as the
  operand numbered \textbf{0} do.

  \nl

  So in this case, the input and output operand will be stored in the same
  register which is not precised, so gcc will decide.

  \nl

  So, in this example we have a read-write operand \textbf{x}.
\end{frame}

% 3)

\begin{frame}[containsverbatim]
  \frametitle{Specify the Registers}

  Note that we can also explicitly tell gcc which register to use for
  storing an operand:

  \begin{verbatim}
    asm(``orl %0, %0''
        : ``=c'' (x)
        : ``c'' (x)
       );
  \end{verbatim}

  In this case, we tell gcc to use the \textbf{ecx} register.
\end{frame}

% 4)

\begin{frame}
  \frametitle{Notices}

  You can notice that in the two previous examples, we put nothing into
  the clobbered list.

  \nl

  In the first example, we told gcc to decide the register to use so gcc
  knows which register to restore.

  \nl

  In the second one, we specify gcc to use the \textbf{ecx} register.
  Once again, gcc perfectly knows which register to restore.
\end{frame}

%
% clobber list
%

\subsection{Clobber List}

% 1)

\begin{frame}
  \frametitle{Overview}

  Some instructions clobber the hardware registers.

  \nl

  For this reason, the inline assembly user has to indicate gcc the names
  of the registers to restore.

  \nl

  Note that gcc already knows the clobbered registers used via the
  constraints of the input and output operands.

  \nl

  Be careful with instructions which implicitly use some registers.
\end{frame}

% 2)

\begin{frame}[containsverbatim]
  \frametitle{Example}

  Let's see an example:

  \begin{verbatim}
    asm(``movl %0, %%ebx\n\t''
        ``movl %1, %%ecx\n\t''
        ``call foo''
        :
        : ``g'' (from), ``g'' (to)
        : ``%ebx'', ``%ecx''
       );
  \end{verbatim}
\end{frame}

% 3)

\begin{frame}
  \frametitle{Condition Code and Memory}

  The user can also specify two other elements in the clobber list:

  \begin{enumerate}
    \item
      \textbf{``cc''} meaning that the condition code was modified i.e
      for example the bits of the \textbf{eflags} register was modified:
      \textit{jnz}, \textit{je} etc..
    \item
      \textbf{``memory''} when a variable is set as output operand. This is
      very useful because gcc maybe holds this variable in a register.
      This constraint tells gcc to update its registers if needed.
      This constraint is equivalent to set all the registers in the
      clobber list.
  \end{enumerate}
\end{frame}

%
% volatile
%

\section{Volatile}

% 1)

\begin{frame}
  \frametitle{Overview}

  First, the keyword \textbf{volatile} is used between the \textbf{asm}
  one and the first parenthesis.

  \nl

  You can also use the keyword \textbf{\_\_volatile\_\_} to avoid conflicts.

  \nl

  The keyword \textbf{volatile} is useful in one precise case:

  \nl

  When the user wants to tell gcc to avoid optimizations.
  For example if your assembly code is located inside a loop, gcc
  will certainly move your assembly code out of the loop.
  Using volatile avoid this optimization.

  \nl

  Be careful, if your assembly code does not have any side effect,
  it is better not to put the \textbf{volatile} keyword to allow gcc
  optimizations.
\end{frame}

%
% constraints
%

\section{Constraints}

% 1)

\begin{frame}
  \frametitle{Overview}

  You probably noticed within this course that the constraints have got
  a lot to do with inline assembly.

  \nl

  Indeed, with them it is possible to specify the location of
  operands: memory, registers etc..

  \nl

  We will now list the different types of operands.
\end{frame}

% 2)

\begin{frame}
  \frametitle{Register Operand Constraint: \textbf{``r''}}

  When an operand is specified with a register operand constraint,
  gcc stores the operand in a GPR (General Purpose Register).

  \nl

  The user can also explicitly tell gcc the register to use. For this,
  he has to use specific constraints listed below:

  \begin{center}

  \begin{tabular}{|p{2cm}|p{4cm}|}
    \hline
    \textbf{Constraint} & \textbf{Register(s)} \\
    \hline
    r & any General Purpose Register \\
    \hline
    a & \%eax, \%ax, \%al \\
    \hline
    b & \%ebx, \%bx, \%bl \\
    \hline
    c & \%ecx, \%cx, \%cl \\
    \hline
    d & \%edx, \%dx, \%dl \\
    \hline
    S & \%esi, \%si \\
    \hline
    D & \%edi, \%di \\
    \hline
  \end{tabular}

  \end{center}
\end{frame}

% 3)

\begin{frame}[containsverbatim]
  \frametitle{Memory Operand Constraint: \textbf{``m''}}

  When the operands are in the memory, any operation performed on them
  will occur directly in the memory location, as opposed to register
  constraints.

  \nl

  Register constraints are usually used with instructions which only accept
  registers as operands. Another case is to speed up the process, for example
  if a variable is needed for three consecutive instructions.

  \nl

  The memory constraint can be used most effeciently in cases where a C
  variable needs to be updated inside inline assembly.

  \nl

  For example:

  \begin{verbatim}
    asm(``sidt %0\n''
        : ''=m''(idtr)
       );
  \end{verbatim}
\end{frame}

% 4)

\begin{frame}[containsverbatim]
  \frametitle{Matching Digit Constraint}

  As seen before, in some cases the programmer wants to use an operand
  both in input and in output.

  \begin{verbatim}
    asm(``incl %0''
        : ''=a'' (var)
        : ''0'' (var)
       );
  \end{verbatim}

  In this example the register \textbf{eax} is used as output operand.
  Then the input operand \textbf{var} uses the constraint \textbf{``0''}
  specifying gcc to use the same constraint than the \textbf{0th} operand.

  \nl

  After all, this inline assembly source code will use the \textbf{eax}
  register both as input and as output.
\end{frame}

% 5)

\begin{frame}
  \frametitle{Other Constraints}

  \begin{itemize}[<+->]
    \item
      \textbf{``i''}: an immediate integer operand (one with constant
      value) is allowed.

      \nl

      this includes symbolic constants whose values will be known only at
      assembly time.
    \item
      \textbf{``n''}: an immediate integer operand with a known numeric
      value is allowed.
    \item
      \textbf{``g''}: any register, memory or immediate integer operand
      is allowed, except for registers that are not general purpose registers.
  \end{itemize}
\end{frame}

% 6)

\begin{frame}
  \frametitle{x86 Specific Constraints}

  \begin{itemize}[<+->]
    \item
      \textbf{``q''}: registers a, b, c or d
    \item
      \textbf{``I''}: constant in range 0 to 31 (for 32-bit shifts)
    \item
      \textbf{``J''}: constant in range 0 to 63 (for 64-bit shifts)
    \item
      \textbf{``K''}: 0xff
    \item
      \textbf{``L''}: 0xffff
    \item
      \textbf{``M''}: 0, 1, 2, 4 (shifts for lea instruction)
    \item
      \textbf{``N''}: constant in range 0 to 255 (for out instruction)
  \end{itemize}
\end{frame}

% 7)

\begin{frame}
  \frametitle{Modifiers}

  There are also constraint modifiers for more precision over the effects
  of constraints.

  \nl

  The common constraints modifiers are:

  \begin{itemize}[<+->]
    \item
      \textbf{``=''}: means that this operand is write-only, the previous
      value is discarded and replaced by output data.
    \item
      \textbf{``\&''}: means that this operand is an earlyclobber operand,
      which is modified before the instruction is finished using the input
      operands. Therefore, this operand may not lie in a register
      that is used as an input operand or as part of any memory address.
      An input operand can be tied to an earlyclobber operand if its
      only use as an input occurs before the early result is written.
  \end{itemize}
\end{frame}

%
% bibliography
%

\section{Bibliography}

\begin{thebibliography}{10}
  \bibitem{Howto}
    GCC Inline Assembly Howto

  \bibitem{Intel-2A}
    IA-32 Intel Architecture
    \newblock Software Developer's Manual
    \newblock Volume 2A: Instruction Set Reference, A-M

  \bibitem{Intel-2B}
    IA-32 Intel Architecture
    \newblock Software Developer's Manual
    \newblock Volume 2B: Instruction Set Reference, N-Z
\end{thebibliography}

\end{document}
