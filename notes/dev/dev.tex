\documentclass[10pt,a4wide]{article}
\usepackage[english]{babel}
\usepackage{a4wide}
\usepackage{graphicx}
\usepackage{graphics}
\usepackage{fancyheadings}
\pagestyle{fancy}

\bibliographystyle{plain}

\lhead{{\scriptsize kaneton project}}
\rhead{dev notes}
\rfoot{\scriptsize EPITA System Lab}

\title{dev}

\author{Julien Quintard - \small{quinta\_j@epita.fr} \\
        Jean-Pascal Billaud - \small{billau\_j@epita.fr} \\ \\
	\small{last updated by} \\
	Julien Quintard - \small{quinta\_j@epita.fr}}

\date{\today}

\begin{document}
\maketitle

\section{Notes}

\begin{enumerate}

\item explication g\'en\'erale sur le d\'eveloppement d'un kernel: outils
      m\'ethode etc..

\item nasm

\item gcc

\item bochs (mount -o loop)

\item grub

\item hierarchisation du code

\end{enumerate}

\end{document}

